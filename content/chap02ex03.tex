\section{译者补充:牛顿迭代法}\label{sec:译者补充:牛顿迭代法}
\begin{remark}
    本节内容不是原书内容,而是译者参照教材补充的,请酌情参考和斧正。
\end{remark}

科学计算中常常需要求解非线性方程
\begin{align}\label{eq:02ex0301}
    f(x)=0\, .
\end{align}
即求函数的零点。因为没有通用解法,所以常用数值求解算法。
其基本思想是从给定的若干个近似初始值出发,
按某种规律产生收敛的迭代数列$\{x_k\}_{k=0}^{\infty}$,
使其逼近\refeq{02ex0301}的某个解。

\subsection{基本定义与定理}\label{sub:基本定义与定理02}
\begin{definition}
    设非线性方程$f(x)=0$中$f(x)$是连续函数,
    如果有
    \begin{align}\label{eq:02ex0302}
        f(x^*)=0\, ,
    \end{align}
    则称$x^*$为该方程的\keyindex{根}{root}{},
    或称为函数$f(x)$的\keyindex{零点}{zero}{};
    如果有
    \begin{align}\label{eq:02ex0303}
        f(x)=(x-x^*)^mg(x)\, ,
    \end{align}
    其中$g(x)$在$x^*$邻域内连续,$g(x^*)\neq0$,且$m$为正整数,
    则称$x^*$为该方程的$m$\keyindex{重根}{multiple root}{root根}。
    当$m=1$时,称$x^*$为该方程的\keyindex{单根}{simple root}{root根}。
\end{definition}

\begin{example}
    方程$f(x)=(x-4)(x+2)^3=0$中,$4$是单根,$-2$是3重根。
\end{example}

本节将使用如下重要定理:
\begin{theorem}[\protect\keyindex{介值定理}{intermediate value theorem}{}]
    若实值函数$f$在闭区间$[a,b]$上连续,则对于介于$f(a)$和$f(b)$之间的任意数$u$,即
    \begin{align}\label{eq:02ex0304}
        \min\{f(a),f(b)\}<u<\max\{f(a),f(b)\}\, ,
    \end{align}
    则必存在$\xi\in(a,b)$使得$f(\xi)=u$。
\end{theorem}
\begin{corollary}
    若实值函数$f$在闭区间$[a,b]$上连续,且$f(a)f(b)<0$,则必存在$\xi\in(a,b)$使得$f(\xi)=0$。
\end{corollary}
\begin{theorem}[\protect\keyindex{拉格朗日中值定理}{Lagrange's mean value theorem}{}]
    若实值函数$f$在闭区间$[a,b]$上连续,在开区间$(a,b)$上可导,其中$a<b$,
    则存在$\xi\in(a,b)$使得
    \begin{align}\label{eq:02ex0305}
        f'(\xi)=\frac{f(b)-f(a)}{b-a}\, .
    \end{align}
\end{theorem}

\subsection{简单迭代法}\label{sub:简单迭代法}
\begin{definition}
    在有根区间$[a,b]$上,将方程$f(x)=0$等价变形为
    \begin{align}\label{eq:02ex0306}
        x=\varphi(x)\, .
    \end{align}
    在$[a,b]$上选取$x_0$作为初始近似值,使用如下迭代公式
    \begin{align}\label{eq:02ex0307}
        x_{k+1}=\varphi(x_k),\quad k=0,1,\cdots\, ,
    \end{align}
    构造序列$\{x_k\}_{k=0}^\infty$。
    若有$\displaystyle\lim\limits_{k\rightarrow\infty}{x_k}=x^*$,
    且函数$\varphi(x)$在$x^*$邻域内连续,则对上式取极限有$x^*=\varphi(x^*)$。
    因此$x^*$是\refeq{02ex0306}的根,也即$f(x)=0$的根。
    称$\varphi(x)$为\keyindex{迭代函数}{iterative function}{},
    所得序列$\{x_k\}_{k=0}^\infty$为\keyindex{迭代序列}{iterative sequence}{},
    这种求方程根近似值的方法称为简单迭代法,简称\keyindex{迭代法}{iterative method}{}。
\end{definition}

\begin{theorem}[全局收敛性定理]
    若函数$\varphi(x)$在区间$[a,b]$上具有一阶连续导数$\varphi'(x)$,且满足条件:
    \begin{enumerate}
        \item 对任意$x\in[a,b]$,有$\varphi(x)\in[a,b]$;
        \item 存在常数$L\in(0,1)$,使得对任意$x\in[a,b]$有$|\varphi'(x)|\le L$成立。
    \end{enumerate}
    则
    \begin{enumerate}
        \item 方程$x=\varphi(x)$在区间$[a,b]$上有唯一实根$x^*$;
        \item 对于任意$x_0\in[a,b]$,迭代公\refeq{02ex0307}收敛,且\begin{align}\label{eq:02ex0308}
                  \lim\limits_{k\rightarrow\infty}{x_k}=x^*\, ;
              \end{align}
        \item 迭代公\refeq{02ex0307}的后验和先验误差估计式分别为
              \begin{align}
                  |x_k-x^*| & \le\frac{L}{1-L}|x_k-x_{k-1}|\, , \label{eq:02ex0309} \\
                  |x_k-x^*| & \le\frac{L^k}{1-L}|x_1-x_0|\, ;\label{eq:02ex0310}
              \end{align}
        \item 存在极限\begin{align}\label{eq:02ex0311}
                  \lim\limits_{k\rightarrow\infty}{\frac{x_{k+1}-x^*}{x_k-x^*}}=\varphi'(x^*)\, .
              \end{align}
    \end{enumerate}
\end{theorem}
\begin{prove}
    (1) 构造函数$g(x)=x-\varphi(x)$,则由条件1有
    \begin{align}\label{eq:02ex0312}
        g(a) & =a-\varphi(a)\le0\, , \\
        g(b) & =b-\varphi(b)\ge0\, .
    \end{align}
    因此$g(x)$在$[a,b]$上至少存在一个零点。
    又因为当$x\in[a,b]$时,由条件2有
    \begin{align}\label{eq:02ex0313}
        g'(x)=1-\varphi'(x)\ge1-L>0\, ,
    \end{align}
    所以$g(x)$在$[a,b]$上是严格单调增函数,存在唯一零点,
    即$x=\varphi(x)$在区间$[a,b]$上有唯一实根,记为$x^*$。

    (2) 由$x_0\in[a,b]$及条件1知$x_k\in[a,b]$,$k=1,2,\cdots$。
    考虑到$x_{k+1}=\varphi(x_k)$以及$x^*=\varphi(x^*)$,
    两者作差并利用拉格朗日中值定理得
    \begin{align}\label{eq:02ex0314}
        x_{k+1}-x^*=\varphi(x_k)-\varphi(x^*)=\varphi'(\xi_k)(x_k-x^*),\quad k=0,1,\cdots\, ,
    \end{align}
    其中$\xi_k$介于$x_k$与$x^*$之间。所以由条件2得
    \begin{align}\label{eq:02ex0314.5}
        |x_{k+1}-x^*|\le L|x_k-x^*|,\quad k=0,1,\cdots\, .
    \end{align}
    反复递推得
    \begin{align}\label{eq:02ex0315}
        0\le|x_{k+1}-x^*| & \le L|x_k-x^*|\nonumber                     \\
                          & \le L^2|x_{k-1}-x^*|\nonumber               \\
                          & \le\cdots\nonumber                          \\
                          & \le L^{k+1}|x_0-x^*|,\quad k=0,1,\cdots\, .
    \end{align}
    因为$L\in(0,1)$,所以对上式取极限易得$\displaystyle\lim\limits_{k\rightarrow\infty}{|x_{k+1}-x^*|}=0$,
    即$\displaystyle\lim\limits_{k\rightarrow\infty}{x_k}=x^*$。

    (3) 由\refeq{02ex0314.5}得
    \begin{align}\label{eq:02ex0316}
        |x_k-x^*| & =|x_k-x_{k+1}+x_{k+1}-x^*|\nonumber                \\
                  & \le|x_k-x_{k+1}|+|x_{k+1}-x^*|\nonumber            \\
                  & \le|x_{k+1}-x_k|+L|x_k-x^*|,\quad k=0,1,\cdots\, ,
    \end{align}
    从而
    \begin{align}\label{eq:02ex0317}
        |x_k-x^*|\le\frac{1}{1-L}|x_{k+1}-x_k|\, .
    \end{align}
    又因为根据拉格朗日中值定理有
    \begin{align}\label{eq:02ex0318}
        |x_{k+1}-x_k| & =|\varphi(x_k)-\varphi(x_{k-1})|\nonumber \\
                      & =|\varphi'(\eta_k)(x_k-x_{k-1})|\nonumber \\
                      & \le L|x_k-x_{k-1}|,\quad k=1,2,\cdots\, ,
    \end{align}
    其中$\eta_k$介于$x_k$与$x_{k-1}$之间。反复递推得
    \begin{align}\label{eq:02ex0319}
        |x_k-x_{k-1}|\le L |x_{k-1}-x_{k-2}|\le\cdots\le L^{k-1}|x_1-x_0|,\quad k=1,2,\cdots\, .
    \end{align}
    联合\refeq{02ex0317}、\refeq{02ex0318}和\refeq{02ex0319}得误差估计
    \begin{align}\label{eq:02ex0320}
        |x_k-x^*|\le\frac{L}{1-L}|x_k-x_{k-1}|\le\frac{L^k}{1-L}|x_1-x_0|,\quad k=1,2,\cdots\, .
    \end{align}

    (4) 由\refeq{02ex0314}得
    \begin{align}\label{eq:02ex0321}
        \frac{x_{k+1}-x^*}{x_k-x^*}=\varphi'(\xi_k),\quad k=0,1,\cdots\, .
    \end{align}
    注意$\xi_k$介于$x_k$与$x^*$之间,故$\displaystyle\lim\limits_{k\rightarrow\infty}{\xi_k}=x^*$。
    由$\varphi'(x)$的连续性,上式取极限得
    \begin{align}\label{eq:02ex0322}
        \lim\limits_{k\rightarrow\infty}{\frac{x_{k+1}-x^*}{x_k-x^*}}=\varphi'(x^*)\, .
    \end{align}
\end{prove}

\begin{theorem}[局部收敛性定理]
    若对于方程$x=\varphi(x)$的根$x^*$,存在闭邻域$U(x^*,\delta)=[x^*-\delta,x^*+\delta]$,$(\delta>0)$
    和常数$L\in(0,1)$,使得$\varphi'(x)$连续且$|\varphi'(x)|\le L$,
    则对任意$x_0\in U(x^*,\delta)$,迭代$x_{k+1}=\varphi(x_k)$收敛。
\end{theorem}
\begin{prove}
    由所给条件,对任意$x\in U(x^*,\delta)$,有
    \begin{align}\label{eq:02ex0323}
        |\varphi(x)-x^*|=|\varphi(x)-\varphi(x^*)|=|\varphi'(\eta)(x-x^*)|\le L\delta<\delta\, ,
    \end{align}
    其中$\eta$介于$x$与$x^*$之间。于是$\varphi(x)\in U(x^*,\delta)$。
    根据全局收敛性定理,迭代$x_{k+1}=\varphi(x_k)$对任意$x_0\in U(x^*,\delta)$收敛。
\end{prove}

接下来引入收敛阶的概念刻画迭代法收敛速度。
\begin{definition}
    设迭代序列$\{x_k\}_{k=0}^\infty$收敛到根$x^*$,记$e_k=x_k-x^*$。
    若存在常数$c>0$和实数$p\ge1$使得
    \begin{align}\label{eq:02ex0324}
        \lim\limits_{k\rightarrow\infty}{\frac{|e_{k+1}|}{|e_k|^p}}=c\, ,
    \end{align}
    则称序列$\{x_k\}_{k=0}^\infty$是\keyindex{$p$阶收敛}{converge with order $p$}{}的,
    $p$为\keyindex{收敛阶}{order of convergence}{},
    $c$为\keyindex{渐进误差常数}{asymptotic error constant}{},简称渐进常数,
    也称为\keyindex{收敛速率}{rate of convergence}{}。
    当$p=1$且$0<c<1$时,称$\{x_k\}_{k=0}^\infty$是\keyindex{线性收敛}{linear convergence}{}的;
    当$p>1$时,称为\keyindex{超线性收敛}{superlinear convergence}{}。
    显然$p$越大,$c$越小,收敛越快。
\end{definition}

\subsection{牛顿迭代法}\label{sub:牛顿迭代法}
\begin{theorem}[\keyindex{泰勒中值定理}{Taylor's mean value theorem}{}]
    若函数$f(x)$在含$x_0$的开区间$(a,b)$上有直到$(n+1)$阶导数,则对任意$x\in(a,b)$有
    \begin{align}\label{eq:02ex0325.1}
        f(x)=f(x_0)+f'(x_0)(x-x_0)+\frac{f''(x_0)}{2!}(x-x_0)^2+\cdots+\frac{f^{(n)}(x_0)}{n!}(x-x_0)^n+R_n(x)\, ,
    \end{align}
    其中
    \begin{align}\label{eq:02ex0325.2}
        R_n(x)=\frac{f^{(n+1)}(\xi)}{(n+1)!}(x-x_0)^{n+1}\, .
    \end{align}
    这里$\xi$介于$x$和$x_0$之间。
    $R_n(x)$称为\refeq{02ex0325.1}的\keyindex{拉格朗日型余项}{Lagrange form of the remainder}{}。
    \refeq{02ex0325.1}称为$f(x)$按$(x-x_0)$的幂展开的
    带有拉格朗日型余项的$n$阶\keyindex{泰勒公式}{Taylor's formula}{}。
\end{theorem}

求解非线性方程$f(x)=0$时,将简单迭代法中的迭代函数取为
\begin{align}\label{eq:02ex0325}
    \varphi(x)=x-\frac{f(x)}{f'(x)}\, ,
\end{align}
便得到牛顿迭代法,简称\keyindex{牛顿法}{Newton's method}{}。

\begin{theorem}
    对于方程$f(x)=0$的单根$x^*$,其二阶导数$f''(x)$在$x^*$邻域上连续且$f'(x)\neq0$,
    则存在$\delta>0$,使得对任意$x_0\in U(x^*,\delta)=[x^*-\delta,x^*+\delta]$,
    牛顿法产生的迭代序列$\{x_k\}_{k=0}^\infty$至少二阶收敛。
\end{theorem}
\begin{prove}
    牛顿法中,迭代函数导数为
    \begin{align}\label{eq:02ex0326}
        \varphi'(x)=\frac{f(x)f''(x)}{(f'(x))^2}\, .
    \end{align}
    显然$\varphi'(x)$在$x^*$邻域上连续。又因为代入$f(x^*)=0$得$\varphi'(x^*)=0$,
    所以$\varphi'(x)$必在$x^*$的某个闭邻域$U(x^*,\delta)$上有
    \begin{align}\label{eq:02ex0327}
        |\varphi'(x)|\le L<1\, .
    \end{align}
    根据局部收敛性定理,牛顿法产生的迭代序列$\{x_k\}_{k=0}^\infty$在该邻域上收敛。

    将$f(x^*)$在$x_k$处作一阶泰勒展开有
    \begin{align}\label{eq:02ex0328}
        f(x^*)=f(x_k)+f'(x_k)(x^*-x_k)+\frac{1}{2}f''(\xi_k)(x^*-x_k)^2=0\, ,
    \end{align}
    其中$\xi_k$介于$x^*$和$x_k$之间。又由牛顿迭代公式得
    \begin{align}\label{eq:02ex0329}
        x_{k+1}=x_k-\frac{f(x_k)}{f'(x_k)}\, ,
    \end{align}
    整理得
    \begin{align}\label{eq:02ex0330}
        f(x_k)+f'(x_k)(x_{k+1}-x_k)=0\, .
    \end{align}
    将\refeq{02ex0328}和\refeq{02ex0330}相减得
    \begin{align}\label{eq:02ex0331}
        x^*-x_{k+1}=\frac{f''(\xi_k)}{2f'(x_k)}(x^*-x_k)^2\, ,
    \end{align}
    因此
    \begin{align}\label{eq:02ex0332}
        \lim\limits_{k\rightarrow\infty}{\left|\frac{x^*-x_{k+1}}{(x^*-x_k)^2}\right|}=\lim\limits_{k\rightarrow\infty}{\left|\frac{f''(\xi_k)}{2f'(x_k)}\right|}=\left|\frac{f''(x^*)}{2f'(x^*)}\right|\neq0\, ,
    \end{align}
    即牛顿法至少具有二阶收敛速度。
\end{prove}