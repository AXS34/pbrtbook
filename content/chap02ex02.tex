\section{译者补充:分解旋转矩阵}\label{sec:译者补充:分解旋转矩阵}
\begin{remark}
    本节内容不是原书内容,而是译者自学补充的,请酌情参考和斧正。
\end{remark}

本节内容主要依据文献\citep{doi:10.1137/0907079,10.5555/155294.155324}整理而成,
给出分解旋转矩阵算法收敛性和最优性的证明。

从只含有旋转$\bm R$和缩放$\bm S$的$3\times3$可逆矩阵$\bm M=\bm R\bm S$中
分解出旋转矩阵,可以利用序列$\displaystyle\{\bm M_i\}_{i=0}^{\infty}$:
\begin{align}
    \bm M_0     & =\bm M\, ,                                                         \\
    \bm M_{i+1} & =\frac{1}{2}(\bm M_i+\bm M_i^{-\mathrm{T}}),\quad i=0,1,\cdots\, ,
\end{align}
此时$\displaystyle \lim\limits_{i \to \infty}{\bm M_i}$必存在
且在Frobenius范数度量下最接近$\bm M$,可作为$\bm R$或$-\bm R$,
然后反解$\bm S=\bm R^{-1}\bm M$。

证明该算法之前,先介绍实方阵的正交对角分解定理:
\begin{theorem}
    对于任意可逆方阵$\bm A\in\mathbb{R}^{n\times n}$,
    存在正交矩阵$\bm P$和$\bm Q$以及对角矩阵$\bm \varSigma$,使得
    \begin{align}
        \bm A=\bm P\bm \varSigma\bm Q^{-1}\, ,
    \end{align}
    其中$\bm \varSigma=\mathrm{diag}(\sigma_1,\sigma_2,\cdots,\sigma_n)$,
    $\sigma_k>0$,$k=1,2,\cdots,n$。
\end{theorem}

我们从两方面证明这个算法:
\begin{itemize}
    \item 收敛性:该算法中必存在极限$\displaystyle\lim\limits_{i\rightarrow\infty}\bm M_i$,
          且为正交矩阵。
    \item 最优性:在Frobenius范数度量下$\displaystyle\lim\limits_{i\rightarrow\infty}\bm M_i$
          最接近$\bm M$。
\end{itemize}

\subsection{收敛性}\label{sub:收敛性02ex02}
\begin{prove}
    依据实方阵的正交对角分解定理,存在正交矩阵$\bm P$和$\bm Q$以及对角矩阵$\bm \varSigma$,使得
    \begin{align}\label{eq:02ex.04}
        \bm M=\bm P\bm \varSigma\bm Q^{-1}\, ,
    \end{align}
    其中${\bm \varSigma}=\mathrm{diag}(\sigma_1,\sigma_2,\sigma_3)$,
    $\sigma_k>0$,$k=1,2,3$。

    构造序列$\displaystyle\{\bm M_i\}_{i=0}^{\infty}$:
    \begin{align}
        \bm M_0     & =\bm M\, ,                                                           \\
        \bm M_{i+1} & =\frac{1}{2}(\bm M_i+\bm M_i^{-\mathrm{T}}),\quad i=0,1,2,\cdots\, .
    \end{align}
    再构造另一个序列$\displaystyle\{\bm D_i\}_{i=0}^{\infty}$:
    \begin{align}
        \bm D_i=\bm P^{-1}\bm M_i\bm Q,\quad i=0,1,2,\cdots\, .
    \end{align}
    取$i=0$可得
    \begin{align}
        \bm D_0=\bm \varSigma\, .
    \end{align}
    即$\bm D_0$是对角矩阵且主对角线均为正。

    另一方面,还可得到递推关系
    \begin{align}
        \bm D_{i+1}=\frac{1}{2}(\bm D_i+\bm D_i^{-\mathrm{T}})\, .
    \end{align}
    可以看出如果$\bm D_i$是主对角线为正数的对角矩阵,则$\bm D_{i+1}$也是。
    由归纳法,对于$i=0,1,2,\cdots$,$\bm D_i$均是这样的矩阵,记为
    \begin{align}
        \bm D_i=\mathrm{diag}\left(d_k^{(i)}\right)\in\mathbb{R}^{3\times 3}, \quad d_k^{(i)}>0, \quad k=1,2,3\, .
    \end{align}

    因此可以独立考虑$\bm D_i$主对角线上的各个分量,即
    \begin{align}
        d_k^{(0)}   & =\sigma_k>0,\qquad k=1,2,3\, ,                                                                     \\
        d_k^{(i+1)} & = \frac{1}{2}\left(d_k^{(i)}+\frac{1}{d_k^{(i)}}\right),\quad i=0,1,2,\cdots\label{eq:02ex.03}\, .
    \end{align}

    显然,由基本不等式有
    \begin{align}
        d_k^{(i+1)}\ge\sqrt{d_k^{(i)}\cdot\frac{1}{d_k^{(i)}}}=1,\quad i=0,1,2,\cdots\, .
    \end{align}
    并且对于$i=1,2,\cdots$有
    \begin{align}
        d_k^{(i)}-d_k^{(i+1)}=\frac{1}{2}\left(d_k^{(i)}-\frac{1}{d_k^{(i)}}\right)\ge0\, .
    \end{align}
    所以数列$\{d_k^{(i)}\}_{i=0}^{\infty}$是一个存在下界的单调减数列,必存在极限$d_k=\lim\limits_{i\rightarrow\infty}{d_k^{(i)}}$。
    对\refeq{02ex.03}两边取极限可求得$d_k=1$,因此存在
    \begin{align}
        \lim\limits_{i\rightarrow\infty}\bm D_i & =\bm I\, ,                                                                  \\
        \lim\limits_{i\rightarrow\infty}\bm M_i & =\lim\limits_{i\rightarrow\infty}\bm P\bm D_i\bm Q^{-1}=\bm P\bm Q^{-1}\, .
    \end{align}
    $\bm P\bm Q^{-1}$显然是正交矩阵,注意
    此时$\bm M$剩余的分量$\bm Q\bm \varSigma\bm Q^{-1}$是对称正定矩阵。
\end{prove}

\subsection{最优性}\label{sub:最优性02ex02}
\begin{prove}
    我们求解最优化问题:寻找在Frobenius范数度量下最接近$3\times3$可逆矩阵$\bm M$的正交矩阵$\bm X$,即
    \begin{align}
        \min\limits_{\bm X}{\|\bm X-\bm M\|_{\mathrm{F}}^2}\, , \\
        \mathrm{s.t.}\quad \bm X^{\mathrm{T}}\bm X=\bm I\, .
    \end{align}
    其中$\displaystyle\|\bm X-\bm M\|_{\mathrm{F}}^2=\sum\limits_{i,j}{(x_{i,j}-m_{i,j})^2}$。
    注意到矩阵恒等式$\|\bm A\|_{\mathrm{F}}^2=\mathrm{tr}(\bm A^{\mathrm{T}}\bm A)$,
    利用拉格朗日乘数法,记拉格朗日乘子为对称矩阵$\bm U$,可转换为优化问题:
    \begin{align}
        \min\limits_{\bm X}{\mathrm{tr}((\bm X-\bm M)^{\mathrm{T}}(\bm X-\bm M)+(\bm X^{\mathrm{T}}\bm X-\bm I)\bm U)}\, .
    \end{align}

    尽管这是矩阵形式的优化问题,但本质上是求二次函数最小值,可以通过计算导数的零点求得,令:
    \begin{align}
        2(\bm X-\bm M)+2\bm X\bm U=\bm O\, ,
    \end{align}
    整理可得
    \begin{align}
        \bm X(\bm I+\bm U)=\bm M\, .
    \end{align}

    记$\bm D=\bm I+\bm U$,显然它是对称矩阵。考虑到$\bm X^{\mathrm{T}}\bm X=\bm I$,于是有
    \begin{align}
        (\bm M\bm D^{-1})^{\mathrm{T}}(\bm M\bm D^{-1})=\bm I\, .
    \end{align}
    整理得
    \begin{align}
        \bm D^2=\bm M^{\mathrm{T}}\bm M\, .
    \end{align}
    代入$\bm M$的分解\refeq{02ex.04},得到$\bm D^2$为
    \begin{align}
        \bm D^2=\bm Q\varSigma^2\bm Q^{-1}\, ,
    \end{align}
    于是$\bm D$的相似对角化为
    \begin{align}
        \bm D=\bm Q\mathrm{diag}(\pm\sigma_1,\pm\sigma_2,\pm\sigma_3)\bm Q^{-1}\, .
    \end{align}
    注意到$\bm X$是最小化问题的解,所以二阶导数$2(\bm I+\bm U)$即$2\bm D$各项须为正,
    所以$\varSigma^2$对角线开方只能取正值,因此$\bm D$的是唯一确定的:
    \begin{align}
        \bm D=\bm Q\varSigma\bm Q^{-1}\, .
    \end{align}
    因此最优解为
    \begin{align}
        \bm X=\bm M\bm D^{-1}=\bm P\bm Q^{-1}\, .
    \end{align}
    它恰是前面证明了的极限$\lim\limits_{i\rightarrow\infty}\bm M_i$,
    即迭代算法收敛到该优化问题最优解。
\end{prove}

要注意的是,我们没有说一定是$\bm R$取$\bm P\bm Q^{-1}$以及$\bm S$取$\bm Q\varSigma\bm Q^{-1}$。
这是因为$\bm P\bm Q^{-1}$的行列式可能为负,
而我们一般应保持旋转矩阵行列式为正的性质,
所以此时$\bm R$应取$-\bm P\bm Q^{-1}$,$\bm S$应取$-\bm Q\varSigma\bm Q^{-1}$。
事实上,这种情况是缩放矩阵$\bm S$含有负数因子进行了镜像操作造成的。