\section{微分几何基础}\label{sec:微分几何基础}
\begin{remark}
    本节内容不是原书内容,而是译者参照教材补充的,请酌情参考和斧正。
\end{remark}

\subsection{曲线的概念}\label{sub:曲线的概念}
\begin{definition}
    给出两个集合$E$和$E'$,如果集合$E$中的每一个点(或称元素)$x$,
    有$E'$中的点$x'$和它对应,
    则我们说给定了$E$到$E'$的一个\keyindex{映射}{mapping}{}$f$。
    $x'$称为点$x$的\keyindex{像}{image}{},
    $x$称为$x'$的\keyindex{原像}{preimage}{}。
\end{definition}
\begin{definition}
    对于任取集合$E$中的点$x_1$和$x_2$,如果$x_1\neq x_2$时有$f(x_1)\neq f(x_2)$,
    则称映射$f$是\keyindex{一一的}{one-to-one}{}(或称\keyindex{单的}{injective}{})。
\end{definition}
\begin{definition}
    在欧氏空间中给出两个集合$E,E'$,
    对于$E$中任一个点$x_0$和任一个数$\varepsilon>0$,
    存在数$\delta>0$,使得对于$E$中与$x_0$的距离小于$\delta$的任意一点$x$来说,
    点$f(x)$与$f(x_0)$间的距离小于$\varepsilon$,
    则称映射$f$为\keyindex{连续}{continuous}{}的。
\end{definition}
\begin{definition}
    如果$f(E)=E'$,就说$f$是从$E$到$E'$的\keyindex{在上映射}{onto mapping}{mapping映射}
    (或称\keyindex{满映射}{surjective mapping}{mapping映射})。
\end{definition}
\begin{definition}
    如果一个开的直线段到三维欧氏空间内建立的对应$f$是一一的、双方连续的在上映射
    (这种映射称为\keyindex{拓扑映射}{topological mapping}{mapping映射}或\keyindex{同胚}{homeomorphism}{}),
    则我们把三维欧氏空间中的映射的像称为\keyindex{简单曲线段}{%simple curve segment
    }{curve曲线}。
\end{definition}

我们以后所讨论的曲线都是简单曲线段,不另做声明。

我们可以确立曲线的方程。在直线段上引入坐标$t(a<t<b)$,
在空间中引入笛卡尔直角坐标$(x,y,z)$,
则上述映射的解析表达式是
\begin{align}\label{eq:03ex01.1}
    \left\{\begin{array}{c}
        x=f(t), \\
        y=g(t), \\
        z=h(t),
    \end{array}
    \right.\quad a<t<b\, .
\end{align}
习惯上常把\refeq{03ex01.1}中的函数关系符号$f,g,h$分别
写作$x,y,z$,于是\refeq{03ex01.1}可写为
\begin{align}\label{eq:03ex01.2}
    \left\{\begin{array}{c}
        x=x(t), \\
        y=y(t), \\
        z=z(t),
    \end{array}
    \right.\quad a<t<b\, .
\end{align}
\refeq{03ex01.2}称为曲线的\keyindex{参数表示}{parametric representation}{}或\keyindex{参数方程}{parametric equation}{},
$t$称为曲线的\keyindex{参数}{parameter}{}。

由于向量函数$\bm r(t)$可表示为$\bm r(t)=x(t)\mathbf{i}+y(t)\mathbf{j}+z(t)\mathbf{k}$,
因而曲线的参数方程\refeq{03ex01.2}也可以写成向量函数的形式:
\begin{align}\label{eq:03ex01.3}
    \bm r=\bm r(t),\quad a<t<b\, .
\end{align}
即在空间中给定一点$O$,以该点作为始点放上对于$t$的所有值的向量$\bm r(t)$。
于是对于$t$的每个值,我们得到确定的向量$\overrightarrow{OM}=\bm r(t)$,
它的始点是点$O$,而终点$M$则与$t$值有关,当$t$在$(a,b)$内变化时,
点$M$在空间中画出一条轨迹,这就是由参数$t$所给定的曲线。
点$M$的向量表达式称为曲线的\keyindex{向量参数表示}{}{parametric representation参数表示}。

\begin{definition}
    如果曲线的参数表示式中的函数
    是\keyindex{$k$阶连续可微}{$k$-times continuously differentiable}{}的函数,
    则把该曲线称为\keyindex{$C^k$类曲线}{%$C^k$-curve
    }{curve曲线}。
    当$k=1$时,也就是$C^1$类曲线
    又称为\keyindex{光滑曲线}{smooth curve}{curve曲线}。
\end{definition}

\begin{definition}
    给出$C^1$类的曲线$\bm r=\bm r(t)$,假设对于该曲线上一点($t=t_0$)有
    \begin{align}\label{eq:03ex01.4}
        \bm r'(t_0)\neq \bm 0\, ,
    \end{align}
    则这一点称为曲线的\keyindex{正常点}{%regular point
    }{point点}。
    注意\refeq{03ex01.4}表示$x'(t_0),y'(t_0),z'(t_0)$中至少有一个不等于零。
\end{definition}

以后我们只考虑曲线的正常点。实际上$\bm r'(t_0)=\bm 0$是很特殊的。
如果在一段曲线上$\bm r'(t_0)\equiv\bm 0$,则$\bm r(t)$变成常向量,
这时这段曲线缩成一点,所以一段曲线上$\bm r'(t_0)=\bm 0$的点一般是孤立点。
\begin{definition}
    曲线上所有点都是正常点时,称该曲线为\keyindex{正则曲线}{%regular curve
    }{curve曲线}。
\end{definition}

\begin{definition}
    给出曲线上一点$P$,点$Q$是曲线上$P$的邻近一点,
    使点$Q$沿曲线趋近于点$P$,若\keyindex{割线}{secant line}{}$PQ$趋近于一定位置,
    则把$PQ$的这个极限位置称为曲线在$P$点的\keyindex{切线}{tangent line}{},
    而定点$P$称为\keyindex{切点}{tangent point}{}。
\end{definition}

\begin{definition}
    设曲线参数方程是$\bm r=\bm r(t)$,点$P$对应参数$t_0$,
    若$\bm r(t)$在$t_0$处可微,则
    \begin{align}\label{eq:03ex01.5}
        \bm r'(t_0)=\lim\limits_{\Delta t\rightarrow0}{\frac{\bm r(t_0+\Delta t)-\bm r(t_0)}{\Delta t}}
    \end{align}
    称为曲线在点$P$的\keyindex{切向量}{tangent vector}{vector向量}。
\end{definition}

由于我们已经规定只研究曲线的正常点,所以曲线上一点的切向量是存在的,
它就是切线上的一个非零向量,其正向和曲线参数$t$的增量方向一致。

\begin{corollary}
    曲线在参数$t_0$处的切线方程是
    \begin{align}\label{eq:03ex01.6}
        \frac{X-x(t_0)}{x'(t_0)}=\frac{Y-y(t_0)}{y'(t_0)}=\frac{Z-z(t_0)}{z'(t_0)}\, .
    \end{align}
\end{corollary}

\begin{definition}
    经过切点而垂直于切线的平面称为
    曲线的\keyindex{法平面}{normal plane}{}或\keyindex{法面}{}{normal plane法平面}。
\end{definition}

\begin{corollary}
    曲线在参数$t_0$处的法面方程是
    \begin{align}\label{eq:03ex01.7}
        x'(t_0)(X-x(t_0))+y'(t_0)(Y-y(t_0))+z'(t_0)(Z-z(t_0))=0\, .
    \end{align}
\end{corollary}

\begin{corollary}
    曲线$\bm r=\bm r(t)$中从$\bm r(a)$到$\bm r(t)$的弧长是
    \begin{align}\label{eq:03ex01.8}
        \sigma(t)=\int_a^t{|\bm r'(t)|\mathrm{d}t},\quad t>a\, .
    \end{align}
\end{corollary}

\subsection{曲面的概念}\label{sub:曲面的概念}
\begin{definition}
    平面上不自交的简单\keyindex{闭曲线}{closed curve}{curve曲线}称为\keyindex{若尔当曲线}{Jordan curve}{curve曲线}。
    它将平面分为两个都以该曲线为边界的部分,其中一个是有限的,另一个是无限的;
    有限的区域称为\keyindex{初等区域}{}{},即若尔当曲线的内部。
\end{definition}
\begin{example}
    正方形或矩形内部,圆或椭圆内部都是初等区域。
\end{example}

\begin{definition}
    如果平面上初等区域到三维欧氏空间内建立的映射是一一的、双方连续的在上映射,
    则把三维欧氏空间中的像称为\keyindex{简单曲面}{%simple surface
    }{surface曲面}。
\end{definition}
\begin{example}
    矩形纸片(初等区域)卷成的带有裂缝的圆柱面是简单曲面。
\end{example}

我们假定以后所讨论的曲面都是简单曲面,不另作说明。

给出平面上一初等区域$G$,$G$中的点的笛卡尔坐标是$(u,v)$,
$G$经过上述映射$f$后的像是曲面$S$。
对于空间的笛卡尔坐标系来说,$S$上的点的坐标是$(x,y,z)$,
则可以写出$f$的解析表达式:
\begin{align}\label{eq:03ex01.9}
    \begin{array}{l}
        x=f_1(u,v)\, , \\
        y=f_2(u,v)\, , \\
        z=f_3(u,v)\, ,
    \end{array}\quad (u,v)\in G\, .
\end{align}
称\refeq{03ex01.9}为曲面$S$的\keyindex{参数表示}{parametric representation}{}或\keyindex{参数方程}{parametric equation}{},
$u$和$v$称为曲面的\keyindex{参数}{parameter}{}或\keyindex{曲纹坐标}{%curvilinear coordinate
}{}。

习惯上常把\refeq{03ex01.9}中的函数关系符号$f_1,f_2$和$f_3$分别写成$x,y$和$z$,即
\begin{align}\label{eq:03ex01.10}
    \begin{array}{l}
        x=x(u,v)\, , \\
        y=y(u,v)\, , \\
        z=z(u,v)\, ,
    \end{array}\quad (u,v)\in G\, .
\end{align}
有时也将其简写称向量函数的形式:
\begin{align}\label{eq:03ex01.11}
    \bm r=\bm r(u,v),\quad (u,v)\in G\, .
\end{align}

\begin{definition}
    初等区域$G$所在平面上的坐标直线$v=$常数或$u=$常数
    在曲面上的像称为曲面的\keyindex{坐标曲线}{coordinate curve}{}。
    使$v$等于常数$v_0$而$u$变动时的曲线$\bm r=\bm r(u,v_0)$叫$u$-曲线;
    使$u$等于常数$u_0$而$v$变动时的曲线$\bm r=\bm r(u_0,v)$叫$v$-曲线。
    这两族坐标曲线在曲面上构成的坐标网称为曲面上的\keyindex{曲纹坐标网}{}{}。
\end{definition}

\begin{definition}
    若曲面方程中的函数有直到$k$阶的连续\keyindex{偏微商}{partial derivative}{},
    则该曲面称为\keyindex{$k$阶正则曲面}{}{surface曲面}或\keyindex{$C^k$类曲面}{}{surface曲面}。
    特别地,$C^1$类曲面又称为\keyindex{光滑曲面}{}{surface曲面}。
\end{definition}

以后我们假定所讨论的曲面都是光滑的。

\begin{definition}
    在曲面$\bm r=\bm r(u,v)$上$(u_0,v_0)$点处两条坐标曲线的切向量分别为
    \begin{align}\label{eq:03ex01.12}
        \bm r_u(u_0,v_0) & =\frac{\partial \bm r}{\partial u}(u_0,v_0)\, , \\
        \bm r_v(u_0,v_0) & =\frac{\partial \bm r}{\partial v}(u_0,v_0)\, .
    \end{align}
    若它们不平行,即$\bm r_u\times\bm r_v$在$(u_0,v_0)$点不等于$\bm 0$,
    则称该点为曲面的\keyindex{正常点}{}{point点}。
\end{definition}

以后我们只讨论曲面的正常点。

若曲面上点的曲纹坐标由下列方程确定:
\begin{align}\label{eq:03ex01.13}
    u & =u(t)\, , \\
    v & =v(t)\, ,
\end{align}
其中$t$是自变量,代入曲面的参数方程可得
该点的\keyindex{向径}{radius vector}{vector向量}为
\begin{align}\label{eq:03ex01.14}
    \bm r=\bm r\left(u(t),v(t)\right)=\bm r(t)\, .
\end{align}
当$t$在某区间上变动时,
关于$t$的函数$\bm r$相应的终点在曲面上确定了某一曲线,
该曲线在曲面上$(u_0,v_0)$点处的切方向称为
曲面在该点的\keyindex{切方向}{tangent direction}{direction方向}或\keyindex{方向}{direction}{}。
它平行于
\begin{align}\label{eq:03ex01.15}
    \bm r'(t)=\bm r_u\frac{\mathrm{d}u}{\mathrm{d}t}+\bm r_v\frac{\mathrm{d}v}{\mathrm{d}t}\, ,
\end{align}
其中$\bm r_u$和$\bm r_v$分别是在该点的两条坐标曲线的切向量。
$\bm r'(t),\bm r_u$和$\bm r_v$共面。
\begin{definition}
    曲面上正常点处的所有切方向都在
    过该点的坐标曲线的切向量$\bm r_u$和$\bm r_v$所决定的平面上,
    该平面称为曲面在这一点的\keyindex{切平面}{tangent plane}{}。
\end{definition}
\begin{corollary}
    曲面$\bm r=\bm r(u,v)$在点$P_0(u_0,v_0)$处的切平面方程为
    \begin{align}\label{eq:03ex01.16}
        \left|
        \begin{array}{ccc}
            X-x(u_0,v_0) & Y-y(u_0,v_0) & Z-z(u_0,v_0) \\
            x_u(u_0,v_0) & y_u(u_0,v_0) & z_u(u_0,v_0) \\
            x_v(u_0,v_0) & y_v(u_0,v_0) & z_v(u_0,v_0)
        \end{array}\right|=0\, .
    \end{align}
\end{corollary}

\begin{definition}
    曲线在正常点处垂直于切平面的方向称为曲面的\keyindex{法方向}{normal direction}{direction方向}。
    过该点平行于法方向的直线称为曲面在该点的\keyindex{法线}{normal}{}。
\end{definition}
\begin{corollary}
    曲面$\bm r=\bm r(u,v)$的法向量为$\displaystyle\bm N=\bm r_u\times\bm r_v$,
    单位法向量为$\displaystyle\frac{\bm r_u\times\bm r_v}{|\bm r_u\times\bm r_v|}$。
\end{corollary}
\begin{corollary}
    曲面$\bm r=\bm r(u,v)$在点$P_0(u_0,v_0)$处的法线方程为
    \begin{align}\label{eq:03ex01.17}
        \frac{X-x(u_0,v_0)}{\left|
            \begin{array}{cc}
                y_u(u_0,v_0) & z_u(u_0,v_0) \\
                y_v(u_0,v_0) & z_v(u_0,v_0)
            \end{array}
            \right|}=\frac{Y-y(u_0,v_0)}{\left|
            \begin{array}{cc}
                z_u(u_0,v_0) & x_u(u_0,v_0) \\
                z_v(u_0,v_0) & x_v(u_0,v_0)
            \end{array}
            \right|}=\frac{Z-z(u_0,v_0)}{\left|
            \begin{array}{cc}
                x_u(u_0,v_0) & y_u(u_0,v_0) \\
                x_v(u_0,v_0) & y_v(u_0,v_0)
            \end{array}
            \right|}\, .
    \end{align}
\end{corollary}

曲面$S$为$\bm r=\bm r(u,v)$,其上的曲线$C$对应为$u=u(t),v=v(t)$,
若以$s$表示曲面上曲线的弧长,则有
\begin{align}\label{eq:03ex01.18}
    \mathrm{d}s^2=\mathrm{d}\bm r^2=(\bm r_u\mathrm{d}u+\bm r_v\mathrm{d}v)^2=\bm r_u^2\mathrm{d}u^2+2\bm r_u\bm r_v\mathrm{d}u\mathrm{d}v+\bm r_v^2\mathrm{d}v^2\, .
\end{align}
令
\begin{align}\label{eq:03ex01.19}
    E=\bm r_u\cdot\bm r_u,\quad F=\bm r_u\cdot\bm r_v,\quad G=\bm r_v\cdot\bm r_v\, ,
\end{align}
则有
\begin{align}\label{eq:03ex01.20}
    \mathrm{d}s^2=E\mathrm{d}u^2+2F\mathrm{d}u\mathrm{d}v+G\mathrm{d}v^2\, .
\end{align}
该二次形式决定曲面上曲线的弧长。
设曲线$C$上两点$A(t_0),B(t_1)$,则弧长为
\begin{align}\label{eq:03ex01.21}
    s=\int_{t_0}^{t_1}{\frac{\mathrm{d}s}{\mathrm{d}t}\mathrm{d}t}=\int_{t_0}^{t_1}
    {\sqrt{E\left(\frac{\mathrm{d}u}{\mathrm{d}t}\right)^2+
        2F\frac{\mathrm{d}u}{\mathrm{d}t}\frac{\mathrm{d}v}{\mathrm{d}t}+
        G\left(\frac{\mathrm{d}v}{\mathrm{d}t}\right)^2}
    \mathrm{d}t},\quad t_0<t_1\, .
\end{align}

\begin{definition}
    称$I$为曲面$S$的\keyindex{第一基本形式}{the first fundamental form}{}:
    \begin{align}\label{eq:03ex01.22}
        I=E\mathrm{d}u^2+2F\mathrm{d}u\mathrm{d}v+G\mathrm{d}v^2\, .
    \end{align}
    其中系数
    \begin{align}\label{eq:03ex01.23}
        E=\bm r_u\cdot\bm r_u,\quad F=\bm r_u\cdot\bm r_v,\quad G=\bm r_v\cdot\bm r_v
    \end{align}
    称为曲面$S$的\keyindex{第一类基本量}{coefficient of the first fundamental form}{}。
\end{definition}

\begin{corollary}
    曲面的第一基本形式是正定的,即
    \begin{align}\label{eq:03ex01.24}
        E>0,\quad G>0,\quad EG-F^2>0\, .
    \end{align}
\end{corollary}