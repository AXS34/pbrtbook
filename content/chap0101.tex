\section{文学编程}\label{sec:文学编程}

在编写\TeX 排版系统时,Donald Knuth新提出一种简单但具有革命性的
编程方法论:\emph{程序应该写得更便于人类使用而不是更便于计算机理解}。
他将其称作\keyindex{文学编程}{literate programming}{}。
本书(包括本章)就是一个长长的文学程序。
这意味着在阅读本书的过程中,
你会读到pbrt渲染系统的\emph{完整}实现,
而不仅仅是高层叙述。

文学程序是由\keyindex{元语言}{metalanguage}{}
写成的,该语言把文档格式语言(例如\TeX 或HTML)
和编程语言(例如C++)结合起来。
两套分离的系统会这样处理程序:\keyindex{编排器}{weaver}{literate programming文学编程}
把文学程序转换成适合排版的文档,\keyindex{整合器}{tangler}{literate programming文学编程}
则生成可供编译的源码\sidenote{译者注:我不太确定编排器和整合器的翻译是否合适。}。
虽然我们的文学编程系统是自研的,
但很大程度上受到了Norman Ramsey的\emph{noweb}系统的影响。

文学编程元语言提供了两个重要功能。
第一个是把行文与源码结合起来。
这个功能让程序的说明和实际源码一样重要,
促使设计和文档做得更细致。
第二个是提供了与输入编译器的顺序完全不同的向读者展示程序代码的机制。
因此可以按逻辑顺序阐述程序。
每一段具有名称的代码块叫作\keyindex{代码片}{fragment}{},
每个代码片可以通过名称引用其他代码片。

例如,考虑一个负责初始化程序全部全局变量的函数
\footnote{本节的代码仅用作示例,不属于pbrt的一部分。}
{\ttfamily InitGlobals()}:
\begin{lstlisting}
void InitGlobals() {
    nMarbles = 25.7;
    shoeSize = 13;
    dielectric = true;
}
\end{lstlisting}

这个函数虽然很简短,但很难在没有任何上下文的情况下搞懂它。
比如为什么变量{\ttfamily nMarbles}采用浮点值?
刚看这段代码时,
就得在整个程序里寻找每个变量是在哪里声明的、怎么用的,
好弄清它的目的和合法值的含义。
尽管这样的系统结构对编译器来说没问题,
但人类阅读者更愿意看到
每个变量的初始化代码是分开呈现的,
而且最好紧挨着实际声明和使用这些变量的代码。

在文学程序中,可以把\refvar{InitGlobals}{()}写成这样:
\begin{lstlisting}
`\initcode{Function Definitions}{=}`
void `\initvar{InitGlobals}{()}` {
    `\refcode{Initialize Global Variables}{\dag}`
}
\end{lstlisting}

这就定义了称作\refcode{Function Definitions}{}的代码片,
包含了函数\refvar{InitGlobals}{()}的定义。
函数\refvar{InitGlobals}{()}自己又引用了另一
代码片\refcode{Initialize Global Variables}{}。
因为初始化的代码片还没有定义,
所以我们只知道这个函数可能会对全局变量赋值
(然而我们可以通过单击右边的加号\sidenote{译者注:本中译版改为直接点击代码片名称。}向前跳转;
这样可以展开代码片最终全部的代码)。

现在有了代码片名称仅仅是有了正确的抽象层级,
因为还没有声明过任何变量。
之后在程序某处引入全局变量{\ttfamily shoeSize}时,
我们可以这样写:
\begin{lstlisting}
    `\initcode{Initialize Global Variables}{=}\initnext{InitializeGlobalVariables}`
    shoeSize = 13;
\end{lstlisting}

这里我们开始定义\refcode{Initialize Global Variables}{}的内容了。
当文学程序整合成待编译的源码时,
文学编程系统会把代码{\ttfamily shoeSize = 13;}
替换到函数\refvar{InitGlobals}{()}的定义内。
等号后的符号{\codecolor $\downarrow$}表示后续还有代码添加到该代码片。
点击它即可跳转到下一处。

后文我们也许又定义了另一个全局变量{\ttfamily dielectric},
可以这样把它的初始化添到代码片之后:
\begin{lstlisting}
    `\refcode{Initialize Global Variables}{+=}\lastcode{InitializeGlobalVariables}`
    dielectric = true;
\end{lstlisting}

代码片名后的符号{\codecolor +=}表示我们之前已经定义过该代码片了。
此外符号{\codecolor $\uparrow$}回链到
之前\refcode{Initialize Global Variables}{}添加代码的地方。

当整合时,这三个代码片转换为代码:
\begin{lstlisting}
void InitGlobals() {
    `\hypertarget{code:Initialize Global Variables}{\color[RGB]{115,48,11}\scriptsize\rmfamily// Initialize Global Variables}`
    shoeSize = 13;
    dielectric = true;
}
\end{lstlisting}

这样,我们可以把复杂函数分解为逻辑不同的部分,使之更容易理解。
例如我们可以这样把一个复杂函数写作一系列代码片:
\begin{lstlisting}
`\refcode{Function Definitions}{+=}`
void `\initvar{complexFunc}{(int x, int y, double *values)}` {
    `\refcode{Check validity of arguments}{}`
    if (x < y) {
        `\refcode{Swap parameter values}{}`
    }
    `\refcode{Do precomputation before loop}{}`
    `\refcode{Loop through and update \textbackslash mono\{values\} array}{}`
\end{lstlisting}

同样,编译时\refvar{complexFunc}{()}内每段代码片的内容都内联展开。
在文档中,我们可以依次介绍每个代码片的实现。
这种分解让我们每次只展示一小段代码,使之更易于理解。
这种编程风格的另一优点是,通过把函数分解为逻辑片,
每片有了单一且明确的目的,可以独立编写、验证、阅读。
一般我们尽量让每段代码片少于10行。

在某种意义上,文学编程系统只是个增强了的宏替换包,
完成重排程序源码的任务。
这变化看似微不足道,但事实上文学编程和其他软件构建系统方法迥然不同。

