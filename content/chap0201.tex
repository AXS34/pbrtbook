\section{坐标系统}\label{sec:坐标系统}

按计算机图形学中的经典做法,
pbrt用三个坐标值$x,y$和$z$表示\keyindex{点}{point}{}、
\keyindex{向量}{vector}{}和\keyindex{法向量}{normal vector}{vector向量}。
这些值须在\keyindex{坐标系统}{coordinate system}{}下才有意义,
即定义了空间的原点并给出三个线性独立的向量定义$x,y$和$z$坐标轴。
总之,这样的原点和三个向量称为\keyindex{坐标系}{frame}{}。
给定3D中的任意一点或方向,其$(x,y,z)$坐标值取决于它和坐标系的关系。
\reffig{2.1}以2D形式给出例子说明了这点。
\begin{figure}[htbp]
    \centering%LaTeX with PSTricks extensions
%%Creator: Inkscape 1.0.1 (3bc2e813f5, 2020-09-07)
%%Please note this file requires PSTricks extensions
\psset{xunit=.5pt,yunit=.5pt,runit=.5pt}
\begin{pspicture}(327.3999939,474.47000122)
{
\newrgbcolor{curcolor}{0 0 0}
\pscustom[linewidth=1,linecolor=curcolor]
{
\newpath
\moveto(5.5,318.58000122)
\lineto(5.5,38.68000122)
\lineto(286.4,38.68000122)
}
}
{
\newrgbcolor{curcolor}{0 0 0}
\pscustom[linestyle=none,fillstyle=solid,fillcolor=curcolor]
{
\newpath
\moveto(0,313.67000122)
\lineto(5.5,317.93000122)
\lineto(11.01,313.67000122)
\lineto(5.5,326.68000122)
\closepath
}
}
{
\newrgbcolor{curcolor}{0.65098041 0.65098041 0.65098041}
\pscustom[linestyle=none,fillstyle=solid,fillcolor=curcolor]
{
\newpath
\moveto(1.2,315.22000122)
\lineto(5.5,325.37000122)
\lineto(5.5,318.56000122)
\closepath
}
}
{
\newrgbcolor{curcolor}{0.40000001 0.40000001 0.40000001}
\pscustom[linestyle=none,fillstyle=solid,fillcolor=curcolor]
{
\newpath
\moveto(9.8,315.22000122)
\lineto(5.5,325.37000122)
\lineto(5.5,318.56000122)
\closepath
}
}
{
\newrgbcolor{curcolor}{0 0 0}
\pscustom[linestyle=none,fillstyle=solid,fillcolor=curcolor]
{
\newpath
\moveto(281.49,33.18000122)
\lineto(285.75,38.68000122)
\lineto(281.49,44.18000122)
\lineto(294.5,38.68000122)
\closepath
}
}
{
\newrgbcolor{curcolor}{0.65098041 0.65098041 0.65098041}
\pscustom[linestyle=none,fillstyle=solid,fillcolor=curcolor]
{
\newpath
\moveto(283.05,34.38000122)
\lineto(293.19,38.68000122)
\lineto(286.38,38.68000122)
\closepath
}
}
{
\newrgbcolor{curcolor}{0.40000001 0.40000001 0.40000001}
\pscustom[linestyle=none,fillstyle=solid,fillcolor=curcolor]
{
\newpath
\moveto(283.05,42.98000122)
\lineto(293.19,38.68000122)
\lineto(286.38,38.68000122)
\closepath
}
}
{
\newrgbcolor{curcolor}{0 0 0}
\pscustom[linestyle=none,fillstyle=solid,fillcolor=curcolor]
{
\newpath
\moveto(219,243.68000793)
\curveto(219,249.91677562)(211.46003012,253.03903011)(207.05050397,248.62950396)
\curveto(202.64097782,244.21997781)(205.76323232,236.68000793)(212,236.68000793)
\curveto(218.23676768,236.68000793)(221.35902218,244.21997781)(216.94949603,248.62950396)
\curveto(212.53996988,253.03903011)(205,249.91677562)(205,243.68000793)
\curveto(205,237.44324025)(212.53996988,234.32098576)(216.94949603,238.73051191)
\curveto(221.35902218,243.14003806)(218.23676768,250.68000793)(212,250.68000793)
\curveto(205.76323232,250.68000793)(202.64097782,243.14003806)(207.05050397,238.73051191)
\curveto(211.46003012,234.32098576)(219,237.44324025)(219,243.68000793)
\closepath
}
}
{
\newrgbcolor{curcolor}{0 0 0}
\pscustom[linewidth=1,linecolor=curcolor]
{
\newpath
\moveto(219,243.68000793)
\curveto(219,249.91677562)(211.46003012,253.03903011)(207.05050397,248.62950396)
\curveto(202.64097782,244.21997781)(205.76323232,236.68000793)(212,236.68000793)
\curveto(218.23676768,236.68000793)(221.35902218,244.21997781)(216.94949603,248.62950396)
\curveto(212.53996988,253.03903011)(205,249.91677562)(205,243.68000793)
\curveto(205,237.44324025)(212.53996988,234.32098576)(216.94949603,238.73051191)
\curveto(221.35902218,243.14003806)(218.23676768,250.68000793)(212,250.68000793)
\curveto(205.76323232,250.68000793)(202.64097782,243.14003806)(207.05050397,238.73051191)
\curveto(211.46003012,234.32098576)(219,237.44324025)(219,243.68000793)
\closepath
}
}
{
\newrgbcolor{curcolor}{0 0 0}
\pscustom[linewidth=0.5,linecolor=curcolor]
{
\newpath
\moveto(212.5,38.29998779)
\lineto(212.5,243.68000793)
}
}
{
\newrgbcolor{curcolor}{0 0 0}
\pscustom[linewidth=0.5,linecolor=curcolor]
{
\newpath
\moveto(212.5,36.30000122)
\lineto(212.5,26.33000122)
\lineto(5.56,26.33000122)
\lineto(5.56,36.30000122)
}
}
{
\newrgbcolor{curcolor}{0 0 0}
\pscustom[linewidth=1,linecolor=curcolor,linestyle=dashed,dash=4 4]
{
\newpath
\moveto(257.15,467.24000122)
\lineto(197.77,349.88000122)
\lineto(193.09,340.62000122)
\lineto(320.17,276.32000122)
}
}
{
\newrgbcolor{curcolor}{0 0 0}
\pscustom[linestyle=none,fillstyle=solid,fillcolor=curcolor]
{
\newpath
\moveto(250.02,465.34000122)
\lineto(256.85,466.66000122)
\lineto(259.84,460.37000122)
\lineto(260.81,474.47000122)
\closepath
}
}
{
\newrgbcolor{curcolor}{0.65098041 0.65098041 0.65098041}
\pscustom[linestyle=none,fillstyle=solid,fillcolor=curcolor]
{
\newpath
\moveto(251.8,466.19000122)
\lineto(260.21,473.30000122)
\lineto(257.14,467.22000122)
\closepath
}
}
{
\newrgbcolor{curcolor}{0.40000001 0.40000001 0.40000001}
\pscustom[linestyle=none,fillstyle=solid,fillcolor=curcolor]
{
\newpath
\moveto(259.47,462.31000122)
\lineto(260.21,473.30000122)
\lineto(257.14,467.22000122)
\closepath
}
}
{
\newrgbcolor{curcolor}{0 0 0}
\pscustom[linestyle=none,fillstyle=solid,fillcolor=curcolor]
{
\newpath
\moveto(313.31,273.63000122)
\lineto(319.59,276.61000122)
\lineto(318.27,283.45000122)
\lineto(327.4,272.66000122)
\closepath
}
}
{
\newrgbcolor{curcolor}{0.65098041 0.65098041 0.65098041}
\pscustom[linestyle=none,fillstyle=solid,fillcolor=curcolor]
{
\newpath
\moveto(315.24,274.00000122)
\lineto(326.23,273.25000122)
\lineto(320.15,276.33000122)
\closepath
}
}
{
\newrgbcolor{curcolor}{0.40000001 0.40000001 0.40000001}
\pscustom[linestyle=none,fillstyle=solid,fillcolor=curcolor]
{
\newpath
\moveto(319.12,281.67000122)
\lineto(326.23,273.25000122)
\lineto(320.15,276.33000122)
\closepath
}
}
{
\newrgbcolor{curcolor}{0 0 0}
\pscustom[linewidth=1,linecolor=curcolor]
{
\newpath
\moveto(215.15,249.80000122)
\lineto(251.83,322.30000122)
\lineto(197.77,349.88000122)
}
}
{
\newrgbcolor{curcolor}{0 0 0}
\pscustom[linestyle=none,fillstyle=solid,fillcolor=curcolor]
{
\newpath
\moveto(183.85300767,237.52316943)
\curveto(183.78504651,237.21734418)(183.75106592,237.1833636)(183.71708534,237.14938301)
\curveto(183.61514359,237.11540243)(183.37727951,237.11540243)(183.17339602,237.11540243)
\curveto(182.7996096,237.11540243)(182.39184261,237.11540243)(182.39184261,236.50375194)
\curveto(182.39184261,236.26588786)(182.59572611,236.09598495)(182.83359019,236.09598495)
\curveto(183.44524068,236.09598495)(184.15883292,236.16394611)(184.80446399,236.16394611)
\curveto(185.5860174,236.16394611)(186.40155138,236.09598495)(187.1491242,236.09598495)
\curveto(187.28504654,236.09598495)(187.69281353,236.09598495)(187.69281353,236.74161602)
\curveto(187.69281353,237.11540243)(187.3530077,237.11540243)(187.1491242,237.11540243)
\curveto(186.84329896,237.11540243)(186.46951255,237.11540243)(186.19766789,237.14938301)
\lineto(187.1491242,240.92122771)
\curveto(187.45494945,240.61540246)(188.16854169,240.1396743)(189.3238815,240.1396743)
\curveto(193.0957262,240.1396743)(195.44038641,243.57171317)(195.44038641,246.52802387)
\curveto(195.44038641,249.21248991)(193.43553203,250.09598507)(191.63456114,250.09598507)
\curveto(190.10543491,250.09598507)(188.98407568,249.2464705)(188.64426985,248.94064525)
\curveto(187.79475528,250.09598507)(186.3675708,250.09598507)(186.12970672,250.09598507)
\curveto(185.34815332,250.09598507)(184.70252224,249.65423749)(184.26077467,248.87268408)
\curveto(183.71708534,247.98918893)(183.4112601,246.83384912)(183.4112601,246.73190737)
\curveto(183.4112601,246.42608212)(183.75106592,246.42608212)(183.95494942,246.42608212)
\curveto(184.1928135,246.42608212)(184.26077467,246.42608212)(184.36271641,246.52802387)
\curveto(184.43067758,246.56200445)(184.43067758,246.62996562)(184.56659991,247.17365494)
\curveto(184.9743669,248.90666467)(185.48407565,249.31443166)(186.02776497,249.31443166)
\curveto(186.26562905,249.31443166)(186.53747371,249.2464705)(186.53747371,248.53287826)
\curveto(186.53747371,248.19307243)(186.46951255,247.88724718)(186.40155138,247.58142194)
\closepath
\moveto(188.84815335,247.92122777)
\curveto(189.45980384,248.66880059)(190.47922132,249.31443166)(191.53261939,249.31443166)
\curveto(192.8918427,249.31443166)(192.99378445,248.15909185)(192.99378445,247.68336369)
\curveto(192.99378445,246.56200445)(192.24621163,243.87753841)(191.9064058,243.02802384)
\curveto(191.22679414,241.46491703)(190.17339608,240.92122771)(189.28990092,240.92122771)
\curveto(187.99863878,240.92122771)(187.48893003,241.94064519)(187.48893003,242.17850927)
\lineto(187.52291062,242.48433452)
\closepath
\moveto(188.84815335,247.92122777)
}
}
{
\newrgbcolor{curcolor}{0 0 0}
\pscustom[linestyle=none,fillstyle=solid,fillcolor=curcolor]
{
\newpath
\moveto(230.24069439,340.48560115)
\lineto(228.11913079,338.5184158)
\lineto(220.70478482,342.32258442)
\lineto(220.88072762,342.66549792)
\curveto(224.08256631,343.53595448)(226.45170074,344.37315206)(227.98813092,345.17709064)
\curveto(229.5245611,345.98102922)(230.56065307,346.90509204)(231.09640682,347.9492791)
\curveto(231.50535494,348.74632129)(231.59733428,349.52647573)(231.37234484,350.28974243)
\curveto(231.14735539,351.05300913)(230.69503647,351.6090002)(230.01538809,351.95771566)
\curveto(229.39752593,352.27472971)(228.74793084,352.37777302)(228.06660281,352.26684559)
\curveto(227.39462356,352.15892663)(226.76212717,351.8161077)(226.16911367,351.23838879)
\lineto(225.82620017,351.41433159)
\curveto(226.56714171,352.47812308)(227.41273345,353.15259836)(228.36297539,353.43775743)
\curveto(229.31939595,353.71974636)(230.27644941,353.61505494)(231.23413576,353.12368316)
\curveto(232.25360833,352.60060997)(232.93515136,351.8372487)(233.27876485,350.83359935)
\curveto(233.62855696,349.82677986)(233.57520289,348.87850936)(233.11870266,347.98878784)
\curveto(232.79217819,347.35238981)(232.31736679,346.79207515)(231.69426848,346.30784387)
\curveto(230.72348772,345.54149756)(229.43837402,344.86228081)(227.83892739,344.27019362)
\curveto(225.43817238,343.37897352)(223.95749367,342.85863642)(223.39689126,342.70918232)
\lineto(226.67773936,341.02583771)
\curveto(227.34503049,340.68346253)(227.82419699,340.46883141)(228.11523884,340.38194434)
\curveto(228.41245932,340.29188713)(228.70723712,340.26552453)(228.99957225,340.30285653)
\curveto(229.29507752,340.34636715)(229.5944804,340.46592962)(229.89778089,340.66154395)
\closepath
}
}
{
\newrgbcolor{curcolor}{0 0 0}
\pscustom[linestyle=none,fillstyle=solid,fillcolor=curcolor]
{
\newpath
\moveto(224.76749728,138.20995139)
\curveto(223.64944349,139.12661661)(222.92722241,139.86272656)(222.60083404,140.41828124)
\curveto(222.2813901,140.97383592)(222.12166813,141.5502239)(222.12166813,142.14744518)
\curveto(222.12166813,143.06411039)(222.47583424,143.85230359)(223.18416645,144.51202477)
\curveto(223.89249867,145.17869039)(224.8334694,145.51202319)(226.00707866,145.51202319)
\curveto(227.14596575,145.51202319)(228.06263097,145.2029959)(228.75707432,144.58494133)
\curveto(229.45151766,143.96688675)(229.79873934,143.26202675)(229.79873934,142.47036133)
\curveto(229.79873934,141.94258439)(229.61123963,141.40439079)(229.23624023,140.85578055)
\curveto(228.86124082,140.3071703)(228.07999205,139.66133799)(226.89249393,138.91828361)
\curveto(228.11471422,137.97384066)(228.92374072,137.23078627)(229.31957343,136.68912046)
\curveto(229.84735037,135.98078825)(230.11123884,135.23426165)(230.11123884,134.44954067)
\curveto(230.11123884,133.45648668)(229.73276722,132.60579358)(228.97582397,131.89746137)
\curveto(228.21888072,131.19607358)(227.22582673,130.84537969)(225.99666201,130.84537969)
\curveto(224.65638635,130.84537969)(223.61124911,131.26551792)(222.86125029,132.10579437)
\curveto(222.26402902,132.77940442)(221.96541838,133.51551437)(221.96541838,134.31412421)
\curveto(221.96541838,134.93912323)(222.17375138,135.55717781)(222.59041739,136.16828795)
\curveto(223.01402783,136.78634253)(223.73972113,137.46689701)(224.76749728,138.20995139)
\closepath
\moveto(226.40291137,139.32453297)
\curveto(227.23624338,140.07453178)(227.76402033,140.66480863)(227.9862422,141.0953635)
\curveto(228.20846407,141.53286281)(228.31957501,142.02591759)(228.31957501,142.57452783)
\curveto(228.31957501,143.30369335)(228.11471422,143.87313689)(227.70499264,144.28285847)
\curveto(227.29527107,144.69952448)(226.73624417,144.90785748)(226.02791196,144.90785748)
\curveto(225.31957975,144.90785748)(224.74319177,144.70299669)(224.29874802,144.29327512)
\curveto(223.85430428,143.88355354)(223.63208241,143.40438763)(223.63208241,142.85577739)
\curveto(223.63208241,142.49466685)(223.72236005,142.13355631)(223.90291532,141.77244577)
\curveto(224.09041502,141.41133523)(224.35430349,141.06758577)(224.69458073,140.7411974)
\closepath
\moveto(225.25707984,137.81411869)
\curveto(224.68069187,137.32800834)(224.25360921,136.79675918)(223.97583187,136.2203712)
\curveto(223.69805453,135.65092766)(223.55916586,135.03287308)(223.55916586,134.36620747)
\curveto(223.55916586,133.47037555)(223.80222103,132.75162668)(224.28833137,132.20996087)
\curveto(224.78138615,131.67523949)(225.40638516,131.40787881)(226.16332841,131.40787881)
\curveto(226.91332723,131.40787881)(227.51402072,131.61968403)(227.9654089,132.04329447)
\curveto(228.41679708,132.46690491)(228.64249116,132.98079299)(228.64249116,133.5849587)
\curveto(228.64249116,134.08495791)(228.51054693,134.53287387)(228.24665846,134.92870658)
\curveto(227.75360368,135.66481653)(226.75707747,136.62662056)(225.25707984,137.81411869)
\closepath
}
}
{
\newrgbcolor{curcolor}{0 0 0}
\pscustom[linestyle=none,fillstyle=solid,fillcolor=curcolor]
{
\newpath
\moveto(106.21592934,14.52114139)
\curveto(105.09787555,15.43780661)(104.37565447,16.17391656)(104.0492661,16.72947124)
\curveto(103.72982216,17.28502592)(103.57010019,17.8614139)(103.57010019,18.45863518)
\curveto(103.57010019,19.37530039)(103.9242663,20.16349359)(104.63259851,20.82321477)
\curveto(105.34093073,21.48988039)(106.28190146,21.82321319)(107.45551072,21.82321319)
\curveto(108.59439781,21.82321319)(109.51106303,21.5141859)(110.20550638,20.89613133)
\curveto(110.89994972,20.27807675)(111.2471714,19.57321675)(111.2471714,18.78155133)
\curveto(111.2471714,18.25377439)(111.05967169,17.71558079)(110.68467229,17.16697055)
\curveto(110.30967288,16.6183603)(109.52842411,15.97252799)(108.34092599,15.22947361)
\curveto(109.56314628,14.28503066)(110.37217278,13.54197627)(110.76800549,13.00031046)
\curveto(111.29578243,12.29197825)(111.5596709,11.54545165)(111.5596709,10.76073067)
\curveto(111.5596709,9.76767668)(111.18119928,8.91698358)(110.42425603,8.20865137)
\curveto(109.66731278,7.50726358)(108.67425879,7.15656969)(107.44509407,7.15656969)
\curveto(106.10481841,7.15656969)(105.05968117,7.57670792)(104.30968235,8.41698437)
\curveto(103.71246108,9.09059442)(103.41385044,9.82670437)(103.41385044,10.62531421)
\curveto(103.41385044,11.25031323)(103.62218344,11.86836781)(104.03884945,12.47947795)
\curveto(104.46245989,13.09753253)(105.18815319,13.77808701)(106.21592934,14.52114139)
\closepath
\moveto(107.85134343,15.63572297)
\curveto(108.68467544,16.38572178)(109.21245239,16.97599863)(109.43467426,17.4065535)
\curveto(109.65689613,17.84405281)(109.76800707,18.33710759)(109.76800707,18.88571783)
\curveto(109.76800707,19.61488335)(109.56314628,20.18432689)(109.1534247,20.59404847)
\curveto(108.74370313,21.01071448)(108.18467623,21.21904748)(107.47634402,21.21904748)
\curveto(106.76801181,21.21904748)(106.19162383,21.01418669)(105.74718008,20.60446512)
\curveto(105.30273634,20.19474354)(105.08051447,19.71557763)(105.08051447,19.16696739)
\curveto(105.08051447,18.80585685)(105.17079211,18.44474631)(105.35134738,18.08363577)
\curveto(105.53884708,17.72252523)(105.80273555,17.37877577)(106.14301279,17.0523874)
\closepath
\moveto(106.7055119,14.12530869)
\curveto(106.12912393,13.63919834)(105.70204127,13.10794918)(105.42426393,12.5315612)
\curveto(105.14648659,11.96211766)(105.00759792,11.34406308)(105.00759792,10.67739747)
\curveto(105.00759792,9.78156555)(105.25065309,9.06281668)(105.73676343,8.52115087)
\curveto(106.22981821,7.98642949)(106.85481722,7.71906881)(107.61176047,7.71906881)
\curveto(108.36175929,7.71906881)(108.96245278,7.93087403)(109.41384096,8.35448447)
\curveto(109.86522914,8.77809491)(110.09092322,9.29198299)(110.09092322,9.8961487)
\curveto(110.09092322,10.39614791)(109.95897899,10.84406387)(109.69509052,11.23989658)
\curveto(109.20203574,11.97600653)(108.20550953,12.93781056)(106.7055119,14.12530869)
\closepath
}
}
{
\newrgbcolor{curcolor}{0 0 0}
\pscustom[linestyle=none,fillstyle=solid,fillcolor=curcolor]
{
\newpath
\moveto(235.10634431,277.72855707)
\lineto(239.29893945,275.57741132)
\lineto(238.6706551,274.3528816)
\lineto(234.47805996,276.50402735)
\closepath
}
}
{
\newrgbcolor{curcolor}{0 0 0}
\pscustom[linestyle=none,fillstyle=solid,fillcolor=curcolor]
{
\newpath
\moveto(247.54661029,270.98712763)
\lineto(246.95577335,269.83558313)
\lineto(245.47985011,270.59285301)
\lineto(243.99027529,267.68966334)
\lineto(242.65221302,268.37619922)
\lineto(244.14178783,271.27938889)
\lineto(239.486953,273.66770158)
\lineto(240.01953841,274.70571353)
\lineto(248.85681351,279.37085812)
\lineto(249.74885503,278.91316754)
\lineto(246.07068705,271.74439751)
\closepath
\moveto(244.73262477,272.43093339)
\lineto(247.53285899,277.88860559)
\lineto(240.87251783,274.41148538)
\closepath
}
}
\end{pspicture}

    \caption{2D中,点$\bm p$的坐标$(x,y)$由该点与特定2D坐标系统的关系定义。
        这里展示了两个坐标系统;
        该点关于实线坐标系统的坐标为$(8,8)$,但关于虚线坐标系统的坐标为$(2,-4)$。
        两种情况下,2D点$\bm p$都处于空间中的同一绝对位置。}
    \label{fig:2.1}
\end{figure}

在一般的$n$维情形下,坐标系原点$\bm p_\mathrm{o}$及其
$n$个线性独立的\keyindex{基向量}{basis vector}{vector向量}定义了一个$n$维\keyindex{仿射空间}{affine space}{}。
该空间内所有向量$\bm v$都能表示为基向量的线性组合。
给定向量$\bm v$和基向量$\bm v_i$,存在一组唯一的标量值$s_i$使得
\begin{align*}
    \bm v=s_1\bm v_1+\cdots+s_n\bm v_n\, .
\end{align*}
标量$s_i$是$\bm v$相对于基$\{\bm v_1,\bm v_2,\ldots \bm v_n\}$的\keyindex{表示}{representation}{},
即我们用向量保存的坐标值。
同样,对于所有点$\bm p$,存在唯一的标量$s_i$使得该点
可以用原点$\bm p_\mathrm{o}$和基向量表示:
\begin{align*}
    \bm p=\bm p_\mathrm{o}+s_1\bm v_1+\cdots+s_n\bm v_n\, .
\end{align*}

因此,尽管3D中的点和向量都用$x,y$和$z$表示,
它们却是不同的数学对象,不能随便互换。

以坐标系统形式定义点和向量造成了一个悖论:
我们需要一个点和一组向量定义坐标系,
但我们只能在特定坐标系下才能有意义地谈论点和向量。
因此,在三维中我们需要一个原点为$(0,0,0)$、
基向量为$(1,0,0)$、$(0,1,0)$以及$(0,0,1)$
的\keyindex{标准坐标系}{standard frame}{frame坐标系}。
其他所有坐标系都参照这个规范坐标系统定义,
该坐标系称为\keyindex{世界空间}{world space}{}。

\subsection{坐标系统惯用手}\label{sub:坐标系统惯用手}
如\reffig{2.2}所示,三个坐标轴有两种排布方式。
给定垂直的$x$和$y$坐标轴,$z$轴可以指向两个方向中的一个。
这两种选择分别称为\keyindex{左手}{left-handed}{}和\keyindex{右手}{right-handed}{}。
任选一种都行但暗含了整个系统中的一些几何操作要如何实现。
pbrt使用左手坐标系统。
\begin{figure}
    \centering%LaTeX with PSTricks extensions
%%Creator: Inkscape 1.0.1 (3bc2e813f5, 2020-09-07)
%%Please note this file requires PSTricks extensions
\psset{xunit=.5pt,yunit=.5pt,runit=.5pt}
\begin{pspicture}(492.61999512,311.95001221)
{
\newrgbcolor{curcolor}{0 0 0}
\pscustom[linewidth=1,linecolor=curcolor]
{
\newpath
\moveto(317.31,282.98001221)
\lineto(317.31,128.95001221)
\lineto(471.91,128.95001221)
}
}
{
\newrgbcolor{curcolor}{0 0 0}
\pscustom[linestyle=none,fillstyle=solid,fillcolor=curcolor]
{
\newpath
\moveto(311.81,278.07001221)
\lineto(317.31,282.33001221)
\lineto(322.81,278.07001221)
\lineto(317.31,291.09001221)
\closepath
}
}
{
\newrgbcolor{curcolor}{0.65098041 0.65098041 0.65098041}
\pscustom[linestyle=none,fillstyle=solid,fillcolor=curcolor]
{
\newpath
\moveto(313.01,279.63001221)
\lineto(317.31,289.77001221)
\lineto(317.31,282.96001221)
\closepath
}
}
{
\newrgbcolor{curcolor}{0.40000001 0.40000001 0.40000001}
\pscustom[linestyle=none,fillstyle=solid,fillcolor=curcolor]
{
\newpath
\moveto(321.61,279.63001221)
\lineto(317.31,289.77001221)
\lineto(317.31,282.96001221)
\closepath
}
}
{
\newrgbcolor{curcolor}{0 0 0}
\pscustom[linestyle=none,fillstyle=solid,fillcolor=curcolor]
{
\newpath
\moveto(467,123.44001221)
\lineto(471.26,128.95001221)
\lineto(467,134.45001221)
\lineto(480.01,128.95001221)
\closepath
}
}
{
\newrgbcolor{curcolor}{0.65098041 0.65098041 0.65098041}
\pscustom[linestyle=none,fillstyle=solid,fillcolor=curcolor]
{
\newpath
\moveto(468.56,124.65001221)
\lineto(478.7,128.95001221)
\lineto(471.89,128.95001221)
\closepath
}
}
{
\newrgbcolor{curcolor}{0.40000001 0.40000001 0.40000001}
\pscustom[linestyle=none,fillstyle=solid,fillcolor=curcolor]
{
\newpath
\moveto(468.56,133.24001221)
\lineto(478.7,128.95001221)
\lineto(471.89,128.95001221)
\closepath
}
}
{
\newrgbcolor{curcolor}{0 0 0}
\pscustom[linewidth=1,linecolor=curcolor]
{
\newpath
\moveto(5.5,277.76001221)
\lineto(5.5,123.73001221)
\lineto(160.1,123.73001221)
}
}
{
\newrgbcolor{curcolor}{0 0 0}
\pscustom[linestyle=none,fillstyle=solid,fillcolor=curcolor]
{
\newpath
\moveto(0,272.86001221)
\lineto(5.5,277.11001221)
\lineto(11.01,272.86001221)
\lineto(5.5,285.87001221)
\closepath
}
}
{
\newrgbcolor{curcolor}{0.65098041 0.65098041 0.65098041}
\pscustom[linestyle=none,fillstyle=solid,fillcolor=curcolor]
{
\newpath
\moveto(1.2,274.41001221)
\lineto(5.5,284.56001221)
\lineto(5.5,277.75001221)
\closepath
}
}
{
\newrgbcolor{curcolor}{0.40000001 0.40000001 0.40000001}
\pscustom[linestyle=none,fillstyle=solid,fillcolor=curcolor]
{
\newpath
\moveto(9.8,274.41001221)
\lineto(5.5,284.56001221)
\lineto(5.5,277.75001221)
\closepath
}
}
{
\newrgbcolor{curcolor}{0 0 0}
\pscustom[linestyle=none,fillstyle=solid,fillcolor=curcolor]
{
\newpath
\moveto(155.19,118.23001221)
\lineto(159.45,123.73001221)
\lineto(155.19,129.23001221)
\lineto(168.21,123.73001221)
\closepath
}
}
{
\newrgbcolor{curcolor}{0.65098041 0.65098041 0.65098041}
\pscustom[linestyle=none,fillstyle=solid,fillcolor=curcolor]
{
\newpath
\moveto(156.75,119.43001221)
\lineto(166.89,123.73001221)
\lineto(160.08,123.73001221)
\closepath
}
}
{
\newrgbcolor{curcolor}{0.40000001 0.40000001 0.40000001}
\pscustom[linestyle=none,fillstyle=solid,fillcolor=curcolor]
{
\newpath
\moveto(156.75,128.03001221)
\lineto(166.89,123.73001221)
\lineto(160.08,123.73001221)
\closepath
}
}
{
\newrgbcolor{curcolor}{0 0 0}
\pscustom[linewidth=1,linecolor=curcolor]
{
\newpath
\moveto(114.43000031,232.70001221)
\lineto(5.51999998,123.78001404)
}
}
{
\newrgbcolor{curcolor}{0 0 0}
\pscustom[linestyle=none,fillstyle=solid,fillcolor=curcolor]
{
\newpath
\moveto(107.07,233.12001221)
\lineto(113.97,232.23001221)
\lineto(114.86,225.33001221)
\lineto(120.17,238.42001221)
\closepath
}
}
{
\newrgbcolor{curcolor}{0.65098041 0.65098041 0.65098041}
\pscustom[linestyle=none,fillstyle=solid,fillcolor=curcolor]
{
\newpath
\moveto(109.02,233.37001221)
\lineto(119.24,237.50001221)
\lineto(114.42,232.68001221)
\closepath
}
}
{
\newrgbcolor{curcolor}{0.40000001 0.40000001 0.40000001}
\pscustom[linestyle=none,fillstyle=solid,fillcolor=curcolor]
{
\newpath
\moveto(115.1,227.29001221)
\lineto(119.24,237.50001221)
\lineto(114.42,232.68001221)
\closepath
}
}
{
\newrgbcolor{curcolor}{0 0 0}
\pscustom[linewidth=1,linecolor=curcolor]
{
\newpath
\moveto(317.42999268,129.31001282)
\lineto(208.50999451,20.39001465)
}
}
{
\newrgbcolor{curcolor}{0 0 0}
\pscustom[linestyle=none,fillstyle=solid,fillcolor=curcolor]
{
\newpath
\moveto(208.09,27.75001221)
\lineto(208.97,20.85001221)
\lineto(215.88,19.97001221)
\lineto(202.78,14.66001221)
\closepath
}
}
{
\newrgbcolor{curcolor}{0.65098041 0.65098041 0.65098041}
\pscustom[linestyle=none,fillstyle=solid,fillcolor=curcolor]
{
\newpath
\moveto(207.84,25.80001221)
\lineto(203.71,15.59001221)
\lineto(208.53,20.40001221)
\closepath
}
}
{
\newrgbcolor{curcolor}{0.40000001 0.40000001 0.40000001}
\pscustom[linestyle=none,fillstyle=solid,fillcolor=curcolor]
{
\newpath
\moveto(213.92,19.72001221)
\lineto(203.71,15.59001221)
\lineto(208.53,20.40001221)
\closepath
}
}
{
\newrgbcolor{curcolor}{0 0 0}
\pscustom[linestyle=none,fillstyle=solid,fillcolor=curcolor]
{
\newpath
\moveto(487.1955217,130.8487217)
\curveto(487.3321743,131.3953321)(487.84462155,133.41095795)(489.34780015,133.41095795)
\curveto(489.4502896,133.41095795)(489.9969,133.41095795)(490.44102095,133.13765275)
\curveto(489.82608425,133.00100015)(489.41612645,132.4885529)(489.41612645,131.9419425)
\curveto(489.41612645,131.600311)(489.6552685,131.1903532)(490.23604205,131.1903532)
\curveto(490.71432615,131.1903532)(491.39758915,131.56614785)(491.39758915,132.45438975)
\curveto(491.39758915,133.5817737)(490.1335526,133.88924205)(489.3819633,133.88924205)
\curveto(488.11792675,133.88924205)(487.36633745,132.72769495)(487.09303225,132.24941085)
\curveto(486.54642185,133.68426315)(485.38487475,133.88924205)(484.7357749,133.88924205)
\curveto(482.481007,133.88924205)(481.21697045,131.08786375)(481.21697045,130.54125335)
\curveto(481.21697045,130.3021113)(481.4561125,130.3021113)(481.49027565,130.3021113)
\curveto(481.6610914,130.3021113)(481.7294177,130.3704376)(481.76358085,130.54125335)
\curveto(482.51517015,132.86434755)(483.95002245,133.41095795)(484.70161175,133.41095795)
\curveto(485.11156955,133.41095795)(485.86315885,133.20597905)(485.86315885,131.9419425)
\curveto(485.86315885,131.2586795)(485.4873642,129.8238272)(484.70161175,126.7491437)
\curveto(484.35998025,125.41678085)(483.5742278,124.4943758)(482.6176596,124.4943758)
\curveto(482.481007,124.4943758)(482.0027229,124.4943758)(481.5244388,124.767681)
\curveto(482.0710492,124.9043336)(482.5493333,125.34845455)(482.5493333,125.96339125)
\curveto(482.5493333,126.5441648)(482.0710492,126.71498055)(481.76358085,126.71498055)
\curveto(481.08031785,126.71498055)(480.5678706,126.16837015)(480.5678706,125.450944)
\curveto(480.5678706,124.46021265)(481.62692825,124.0160917)(482.58349645,124.0160917)
\curveto(484.0525119,124.0160917)(484.83826435,125.55343345)(484.8724275,125.6559229)
\curveto(485.1457327,124.87017045)(485.93148515,124.0160917)(487.22968485,124.0160917)
\curveto(489.48445275,124.0160917)(490.71432615,126.81747)(490.71432615,127.3640804)
\curveto(490.71432615,127.60322245)(490.5435104,127.60322245)(490.4751841,127.60322245)
\curveto(490.2702052,127.60322245)(490.23604205,127.500733)(490.16771575,127.3640804)
\curveto(489.4502896,125.00682305)(487.98127415,124.4943758)(487.29801115,124.4943758)
\curveto(486.4439324,124.4943758)(486.1023009,125.1776388)(486.1023009,125.9292281)
\curveto(486.1023009,126.4075122)(486.20479035,126.8857963)(486.4439324,127.8423645)
\closepath
\moveto(487.1955217,130.8487217)
}
}
{
\newrgbcolor{curcolor}{0 0 0}
\pscustom[linestyle=none,fillstyle=solid,fillcolor=curcolor]
{
\newpath
\moveto(321.81166725,305.2585204)
\curveto(321.9141567,305.56598875)(321.9141567,305.6001519)(321.9141567,305.77096765)
\curveto(321.9141567,306.1467623)(321.60668835,306.3517412)(321.26505685,306.3517412)
\curveto(321.06007795,306.3517412)(320.71844645,306.2150886)(320.51346755,305.90762025)
\curveto(320.4793044,305.77096765)(320.2743255,305.1218678)(320.2059992,304.71191)
\curveto(320.03518345,304.1652996)(319.89853085,303.5503629)(319.76187825,302.96958935)
\lineto(318.7711469,299.0408271)
\curveto(318.7028206,298.73335875)(317.7462524,297.196017)(316.3114001,297.196017)
\curveto(315.2181793,297.196017)(314.97903725,298.1525852)(314.97903725,298.9725008)
\curveto(314.97903725,299.96323215)(315.3548319,301.32975815)(316.07225805,303.24289455)
\curveto(316.41388955,304.13113645)(316.516379,304.3702785)(316.516379,304.81439945)
\curveto(316.516379,305.77096765)(315.833116,306.59088325)(314.7398952,306.59088325)
\curveto(312.65594305,306.59088325)(311.8701906,303.4137103)(311.8701906,303.24289455)
\curveto(311.8701906,303.0037525)(312.0751695,303.0037525)(312.10933265,303.0037525)
\curveto(312.3484747,303.0037525)(312.3484747,303.0720788)(312.45096415,303.4137103)
\curveto(313.06590085,305.4634993)(313.9199796,306.11259915)(314.6715689,306.11259915)
\curveto(314.84238465,306.11259915)(315.2181793,306.11259915)(315.2181793,305.42933615)
\curveto(315.2181793,304.88272575)(314.97903725,304.3019522)(314.84238465,303.8919944)
\curveto(313.95414275,301.5689002)(313.5783481,300.3390268)(313.5783481,299.3141323)
\curveto(313.5783481,297.36683275)(314.9448741,296.7177329)(316.2430738,296.7177329)
\curveto(317.09715255,296.7177329)(317.8145787,297.09352755)(318.4295154,297.70846425)
\curveto(318.1562102,296.5810803)(317.882905,295.4878595)(317.02882625,294.3263124)
\curveto(316.4480527,293.60888625)(315.6281371,292.9597864)(314.63740575,292.9597864)
\curveto(314.3299374,292.9597864)(313.33920605,293.0281127)(312.9634114,293.88219145)
\curveto(313.3050429,293.88219145)(313.61251125,293.88219145)(313.88581645,294.15549665)
\curveto(314.1249585,294.3263124)(314.3299374,294.63378075)(314.3299374,295.04373855)
\curveto(314.3299374,295.72700155)(313.74916385,295.79532785)(313.54418495,295.79532785)
\curveto(313.0317377,295.79532785)(312.31431155,295.45369635)(312.31431155,294.3946387)
\curveto(312.31431155,293.3014179)(313.27087975,292.4815023)(314.63740575,292.4815023)
\curveto(316.8580105,292.4815023)(319.1127784,294.462965)(319.7277151,296.9227118)
\closepath
\moveto(321.81166725,305.2585204)
}
}
{
\newrgbcolor{curcolor}{0 0 0}
\pscustom[linestyle=none,fillstyle=solid,fillcolor=curcolor]
{
\newpath
\moveto(127.0204015,239.3646719)
\curveto(128.1819486,240.62870845)(128.83104845,241.17531885)(129.6168009,241.85858185)
\curveto(129.6168009,241.85858185)(130.94916375,243.02012895)(131.7349162,243.8058814)
\curveto(133.81886835,245.82150725)(134.29715245,246.8805649)(134.29715245,246.98305435)
\curveto(134.29715245,247.18803325)(134.09217355,247.18803325)(134.0580104,247.18803325)
\curveto(133.88719465,247.18803325)(133.8530315,247.1538701)(133.7163789,246.9488912)
\curveto(133.06727905,245.88983355)(132.6231581,245.54820205)(132.11071085,245.54820205)
\curveto(131.56410045,245.54820205)(131.3249584,245.88983355)(130.9833269,246.2656282)
\curveto(130.5733691,246.7439123)(130.19757445,247.18803325)(129.4801483,247.18803325)
\curveto(127.8403171,247.18803325)(126.84958575,245.1724074)(126.84958575,244.6941233)
\curveto(126.84958575,244.59163385)(126.91791205,244.45498125)(127.0887278,244.45498125)
\curveto(127.2937067,244.45498125)(127.32786985,244.5574707)(127.39619615,244.6941233)
\curveto(127.80615395,245.7190178)(129.0701905,245.7190178)(129.24100625,245.7190178)
\curveto(129.6851272,245.7190178)(130.095085,245.5823652)(130.60753225,245.41154945)
\curveto(131.49577415,245.06991795)(131.7349162,245.06991795)(132.2815266,245.06991795)
\curveto(131.49577415,244.1475129)(129.6851272,242.576008)(129.2751694,242.2343765)
\lineto(127.2937067,240.3895664)
\curveto(125.82469125,238.92055095)(125.0389388,237.69067755)(125.0389388,237.5198618)
\curveto(125.0389388,237.3148829)(125.27808085,237.3148829)(125.312244,237.3148829)
\curveto(125.48305975,237.3148829)(125.5172229,237.34904605)(125.6538755,237.5881881)
\curveto(126.16632275,238.37394055)(126.8154226,238.9547141)(127.53284875,238.9547141)
\curveto(128.01113285,238.9547141)(128.2502749,238.7497352)(128.7968853,238.1347985)
\curveto(129.1385168,237.6565144)(129.5484746,237.3148829)(130.1634113,237.3148829)
\curveto(132.3498529,237.3148829)(133.61388945,240.08209805)(133.61388945,240.6628716)
\curveto(133.61388945,240.76536105)(133.5114,240.90201365)(133.34058425,240.90201365)
\curveto(133.13560535,240.90201365)(133.1014422,240.76536105)(133.0331159,240.5945453)
\curveto(132.52066865,239.19385615)(131.1199795,238.78389835)(130.40255335,238.78389835)
\curveto(129.99259555,238.78389835)(129.58263775,238.92055095)(129.1385168,239.05720355)
\curveto(128.3869275,239.33050875)(128.045296,239.4329982)(127.60117505,239.4329982)
\curveto(127.5670119,239.4329982)(127.2253804,239.4329982)(127.0204015,239.3646719)
\closepath
\moveto(127.0204015,239.3646719)
}
}
{
\newrgbcolor{curcolor}{0 0 0}
\pscustom[linestyle=none,fillstyle=solid,fillcolor=curcolor]
{
\newpath
\moveto(180.7951717,125.3069599)
\curveto(180.9318243,125.8535703)(181.44427155,127.86919615)(182.94745015,127.86919615)
\curveto(183.0499396,127.86919615)(183.59655,127.86919615)(184.04067095,127.59589095)
\curveto(183.42573425,127.45923835)(183.01577645,126.9467911)(183.01577645,126.4001807)
\curveto(183.01577645,126.0585492)(183.2549185,125.6485914)(183.83569205,125.6485914)
\curveto(184.31397615,125.6485914)(184.99723915,126.02438605)(184.99723915,126.91262795)
\curveto(184.99723915,128.0400119)(183.7332026,128.34748025)(182.9816133,128.34748025)
\curveto(181.71757675,128.34748025)(180.96598745,127.18593315)(180.69268225,126.70764905)
\curveto(180.14607185,128.14250135)(178.98452475,128.34748025)(178.3354249,128.34748025)
\curveto(176.080657,128.34748025)(174.81662045,125.54610195)(174.81662045,124.99949155)
\curveto(174.81662045,124.7603495)(175.0557625,124.7603495)(175.08992565,124.7603495)
\curveto(175.2607414,124.7603495)(175.3290677,124.8286758)(175.36323085,124.99949155)
\curveto(176.11482015,127.32258575)(177.54967245,127.86919615)(178.30126175,127.86919615)
\curveto(178.71121955,127.86919615)(179.46280885,127.66421725)(179.46280885,126.4001807)
\curveto(179.46280885,125.7169177)(179.0870142,124.2820654)(178.30126175,121.2073819)
\curveto(177.95963025,119.87501905)(177.1738778,118.952614)(176.2173096,118.952614)
\curveto(176.080657,118.952614)(175.6023729,118.952614)(175.1240888,119.2259192)
\curveto(175.6706992,119.3625718)(176.1489833,119.80669275)(176.1489833,120.42162945)
\curveto(176.1489833,121.002403)(175.6706992,121.17321875)(175.36323085,121.17321875)
\curveto(174.67996785,121.17321875)(174.1675206,120.62660835)(174.1675206,119.9091822)
\curveto(174.1675206,118.91845085)(175.22657825,118.4743299)(176.18314645,118.4743299)
\curveto(177.6521619,118.4743299)(178.43791435,120.01167165)(178.4720775,120.1141611)
\curveto(178.7453827,119.32840865)(179.53113515,118.4743299)(180.82933485,118.4743299)
\curveto(183.08410275,118.4743299)(184.31397615,121.2757082)(184.31397615,121.8223186)
\curveto(184.31397615,122.06146065)(184.1431604,122.06146065)(184.0748341,122.06146065)
\curveto(183.8698552,122.06146065)(183.83569205,121.9589712)(183.76736575,121.8223186)
\curveto(183.0499396,119.46506125)(181.58092415,118.952614)(180.89766115,118.952614)
\curveto(180.0435824,118.952614)(179.7019509,119.635877)(179.7019509,120.3874663)
\curveto(179.7019509,120.8657504)(179.80444035,121.3440345)(180.0435824,122.3006027)
\closepath
\moveto(180.7951717,125.3069599)
}
}
{
\newrgbcolor{curcolor}{0 0 0}
\pscustom[linestyle=none,fillstyle=solid,fillcolor=curcolor]
{
\newpath
\moveto(193.5046815,6.3258519)
\curveto(194.6662286,7.58988845)(195.31532845,8.13649885)(196.1010809,8.81976185)
\curveto(196.1010809,8.81976185)(197.43344375,9.98130895)(198.2191962,10.7670614)
\curveto(200.30314835,12.78268725)(200.78143245,13.8417449)(200.78143245,13.94423435)
\curveto(200.78143245,14.14921325)(200.57645355,14.14921325)(200.5422904,14.14921325)
\curveto(200.37147465,14.14921325)(200.3373115,14.1150501)(200.2006589,13.9100712)
\curveto(199.55155905,12.85101355)(199.1074381,12.50938205)(198.59499085,12.50938205)
\curveto(198.04838045,12.50938205)(197.8092384,12.85101355)(197.4676069,13.2268082)
\curveto(197.0576491,13.7050923)(196.68185445,14.14921325)(195.9644283,14.14921325)
\curveto(194.3245971,14.14921325)(193.33386575,12.1335874)(193.33386575,11.6553033)
\curveto(193.33386575,11.55281385)(193.40219205,11.41616125)(193.5730078,11.41616125)
\curveto(193.7779867,11.41616125)(193.81214985,11.5186507)(193.88047615,11.6553033)
\curveto(194.29043395,12.6801978)(195.5544705,12.6801978)(195.72528625,12.6801978)
\curveto(196.1694072,12.6801978)(196.579365,12.5435452)(197.09181225,12.37272945)
\curveto(197.98005415,12.03109795)(198.2191962,12.03109795)(198.7658066,12.03109795)
\curveto(197.98005415,11.1086929)(196.1694072,9.537188)(195.7594494,9.1955565)
\lineto(193.7779867,7.3507464)
\curveto(192.30897125,5.88173095)(191.5232188,4.65185755)(191.5232188,4.4810418)
\curveto(191.5232188,4.2760629)(191.76236085,4.2760629)(191.796524,4.2760629)
\curveto(191.96733975,4.2760629)(192.0015029,4.31022605)(192.1381555,4.5493681)
\curveto(192.65060275,5.33512055)(193.2997026,5.9158941)(194.01712875,5.9158941)
\curveto(194.49541285,5.9158941)(194.7345549,5.7109152)(195.2811653,5.0959785)
\curveto(195.6227968,4.6176944)(196.0327546,4.2760629)(196.6476913,4.2760629)
\curveto(198.8341329,4.2760629)(200.09816945,7.04327805)(200.09816945,7.6240516)
\curveto(200.09816945,7.72654105)(199.99568,7.86319365)(199.82486425,7.86319365)
\curveto(199.61988535,7.86319365)(199.5857222,7.72654105)(199.5173959,7.5557253)
\curveto(199.00494865,6.15503615)(197.6042595,5.74507835)(196.88683335,5.74507835)
\curveto(196.47687555,5.74507835)(196.06691775,5.88173095)(195.6227968,6.01838355)
\curveto(194.8712075,6.29168875)(194.529576,6.3941782)(194.08545505,6.3941782)
\curveto(194.0512919,6.3941782)(193.7096604,6.3941782)(193.5046815,6.3258519)
\closepath
\moveto(193.5046815,6.3258519)
}
}
{
\newrgbcolor{curcolor}{0 0 0}
\pscustom[linestyle=none,fillstyle=solid,fillcolor=curcolor]
{
\newpath
\moveto(12.98054725,301.5502304)
\curveto(13.0830367,301.85769875)(13.0830367,301.8918619)(13.0830367,302.06267765)
\curveto(13.0830367,302.4384723)(12.77556835,302.6434512)(12.43393685,302.6434512)
\curveto(12.22895795,302.6434512)(11.88732645,302.5067986)(11.68234755,302.19933025)
\curveto(11.6481844,302.06267765)(11.4432055,301.4135778)(11.3748792,301.00362)
\curveto(11.20406345,300.4570096)(11.06741085,299.8420729)(10.93075825,299.26129935)
\lineto(9.9400269,295.3325371)
\curveto(9.8717006,295.02506875)(8.9151324,293.487727)(7.4802801,293.487727)
\curveto(6.3870593,293.487727)(6.14791725,294.4442952)(6.14791725,295.2642108)
\curveto(6.14791725,296.25494215)(6.5237119,297.62146815)(7.24113805,299.53460455)
\curveto(7.58276955,300.42284645)(7.685259,300.6619885)(7.685259,301.10610945)
\curveto(7.685259,302.06267765)(7.001996,302.88259325)(5.9087752,302.88259325)
\curveto(3.82482305,302.88259325)(3.0390706,299.7054203)(3.0390706,299.53460455)
\curveto(3.0390706,299.2954625)(3.2440495,299.2954625)(3.27821265,299.2954625)
\curveto(3.5173547,299.2954625)(3.5173547,299.3637888)(3.61984415,299.7054203)
\curveto(4.23478085,301.7552093)(5.0888596,302.40430915)(5.8404489,302.40430915)
\curveto(6.01126465,302.40430915)(6.3870593,302.40430915)(6.3870593,301.72104615)
\curveto(6.3870593,301.17443575)(6.14791725,300.5936622)(6.01126465,300.1837044)
\curveto(5.12302275,297.8606102)(4.7472281,296.6307368)(4.7472281,295.6058423)
\curveto(4.7472281,293.65854275)(6.1137541,293.0094429)(7.4119538,293.0094429)
\curveto(8.26603255,293.0094429)(8.9834587,293.38523755)(9.5983954,294.00017425)
\curveto(9.3250902,292.8727903)(9.051785,291.7795695)(8.19770625,290.6180224)
\curveto(7.6169327,289.90059625)(6.7970171,289.2514964)(5.80628575,289.2514964)
\curveto(5.4988174,289.2514964)(4.50808605,289.3198227)(4.1322914,290.17390145)
\curveto(4.4739229,290.17390145)(4.78139125,290.17390145)(5.05469645,290.44720665)
\curveto(5.2938385,290.6180224)(5.4988174,290.92549075)(5.4988174,291.33544855)
\curveto(5.4988174,292.01871155)(4.91804385,292.08703785)(4.71306495,292.08703785)
\curveto(4.2006177,292.08703785)(3.48319155,291.74540635)(3.48319155,290.6863487)
\curveto(3.48319155,289.5931279)(4.43975975,288.7732123)(5.80628575,288.7732123)
\curveto(8.0268905,288.7732123)(10.2816584,290.754675)(10.8965951,293.2144218)
\closepath
\moveto(12.98054725,301.5502304)
}
}
\end{pspicture}

    \caption{(左图)在左手坐标系统中,当$x$轴朝右且$y$轴朝上时,$z$轴指入页面。
        (右图)在右手坐标系统中,$z$轴指出页面。}
    \label{fig:2.2}
\end{figure}