\section{坐标系统}\label{sec:坐标系统}

按计算机图形学中的经典做法,
pbrt用三个坐标值$x,y$和$z$表示\keyindex{点}{point}{}、
\keyindex{向量}{vector}{}和\keyindex{法向量}{normal vector}{vector向量}。
这些值须在\keyindex{坐标系统}{coordinate system}{}下才有意义,
即定义了空间的原点并给出三个线性独立的向量定义$x,y$和$z$坐标轴。
总之,定义了坐标系的原点和三个向量称为\keyindex{坐标系}{frame}{}。
给定3D中的任意一点或方向,其$(x,y,z)$坐标值取决于它和坐标系的关系。
\reffig{2.1}以2D形式给出例子说明了这点。
\begin{figure}
    %LaTeX with PSTricks extensions
%%Creator: Inkscape 1.0.1 (3bc2e813f5, 2020-09-07)
%%Please note this file requires PSTricks extensions
\psset{xunit=.5pt,yunit=.5pt,runit=.5pt}
\begin{pspicture}(327.3999939,474.47000122)
{
\newrgbcolor{curcolor}{0 0 0}
\pscustom[linewidth=1,linecolor=curcolor]
{
\newpath
\moveto(5.5,318.58000122)
\lineto(5.5,38.68000122)
\lineto(286.4,38.68000122)
}
}
{
\newrgbcolor{curcolor}{0 0 0}
\pscustom[linestyle=none,fillstyle=solid,fillcolor=curcolor]
{
\newpath
\moveto(0,313.67000122)
\lineto(5.5,317.93000122)
\lineto(11.01,313.67000122)
\lineto(5.5,326.68000122)
\closepath
}
}
{
\newrgbcolor{curcolor}{0.65098041 0.65098041 0.65098041}
\pscustom[linestyle=none,fillstyle=solid,fillcolor=curcolor]
{
\newpath
\moveto(1.2,315.22000122)
\lineto(5.5,325.37000122)
\lineto(5.5,318.56000122)
\closepath
}
}
{
\newrgbcolor{curcolor}{0.40000001 0.40000001 0.40000001}
\pscustom[linestyle=none,fillstyle=solid,fillcolor=curcolor]
{
\newpath
\moveto(9.8,315.22000122)
\lineto(5.5,325.37000122)
\lineto(5.5,318.56000122)
\closepath
}
}
{
\newrgbcolor{curcolor}{0 0 0}
\pscustom[linestyle=none,fillstyle=solid,fillcolor=curcolor]
{
\newpath
\moveto(281.49,33.18000122)
\lineto(285.75,38.68000122)
\lineto(281.49,44.18000122)
\lineto(294.5,38.68000122)
\closepath
}
}
{
\newrgbcolor{curcolor}{0.65098041 0.65098041 0.65098041}
\pscustom[linestyle=none,fillstyle=solid,fillcolor=curcolor]
{
\newpath
\moveto(283.05,34.38000122)
\lineto(293.19,38.68000122)
\lineto(286.38,38.68000122)
\closepath
}
}
{
\newrgbcolor{curcolor}{0.40000001 0.40000001 0.40000001}
\pscustom[linestyle=none,fillstyle=solid,fillcolor=curcolor]
{
\newpath
\moveto(283.05,42.98000122)
\lineto(293.19,38.68000122)
\lineto(286.38,38.68000122)
\closepath
}
}
{
\newrgbcolor{curcolor}{0 0 0}
\pscustom[linestyle=none,fillstyle=solid,fillcolor=curcolor]
{
\newpath
\moveto(219,243.68000793)
\curveto(219,249.91677562)(211.46003012,253.03903011)(207.05050397,248.62950396)
\curveto(202.64097782,244.21997781)(205.76323232,236.68000793)(212,236.68000793)
\curveto(218.23676768,236.68000793)(221.35902218,244.21997781)(216.94949603,248.62950396)
\curveto(212.53996988,253.03903011)(205,249.91677562)(205,243.68000793)
\curveto(205,237.44324025)(212.53996988,234.32098576)(216.94949603,238.73051191)
\curveto(221.35902218,243.14003806)(218.23676768,250.68000793)(212,250.68000793)
\curveto(205.76323232,250.68000793)(202.64097782,243.14003806)(207.05050397,238.73051191)
\curveto(211.46003012,234.32098576)(219,237.44324025)(219,243.68000793)
\closepath
}
}
{
\newrgbcolor{curcolor}{0 0 0}
\pscustom[linewidth=1,linecolor=curcolor]
{
\newpath
\moveto(219,243.68000793)
\curveto(219,249.91677562)(211.46003012,253.03903011)(207.05050397,248.62950396)
\curveto(202.64097782,244.21997781)(205.76323232,236.68000793)(212,236.68000793)
\curveto(218.23676768,236.68000793)(221.35902218,244.21997781)(216.94949603,248.62950396)
\curveto(212.53996988,253.03903011)(205,249.91677562)(205,243.68000793)
\curveto(205,237.44324025)(212.53996988,234.32098576)(216.94949603,238.73051191)
\curveto(221.35902218,243.14003806)(218.23676768,250.68000793)(212,250.68000793)
\curveto(205.76323232,250.68000793)(202.64097782,243.14003806)(207.05050397,238.73051191)
\curveto(211.46003012,234.32098576)(219,237.44324025)(219,243.68000793)
\closepath
}
}
{
\newrgbcolor{curcolor}{0 0 0}
\pscustom[linewidth=0.5,linecolor=curcolor]
{
\newpath
\moveto(212.5,38.29998779)
\lineto(212.5,243.68000793)
}
}
{
\newrgbcolor{curcolor}{0 0 0}
\pscustom[linewidth=0.5,linecolor=curcolor]
{
\newpath
\moveto(212.5,36.30000122)
\lineto(212.5,26.33000122)
\lineto(5.56,26.33000122)
\lineto(5.56,36.30000122)
}
}
{
\newrgbcolor{curcolor}{0 0 0}
\pscustom[linewidth=1,linecolor=curcolor,linestyle=dashed,dash=4 4]
{
\newpath
\moveto(257.15,467.24000122)
\lineto(197.77,349.88000122)
\lineto(193.09,340.62000122)
\lineto(320.17,276.32000122)
}
}
{
\newrgbcolor{curcolor}{0 0 0}
\pscustom[linestyle=none,fillstyle=solid,fillcolor=curcolor]
{
\newpath
\moveto(250.02,465.34000122)
\lineto(256.85,466.66000122)
\lineto(259.84,460.37000122)
\lineto(260.81,474.47000122)
\closepath
}
}
{
\newrgbcolor{curcolor}{0.65098041 0.65098041 0.65098041}
\pscustom[linestyle=none,fillstyle=solid,fillcolor=curcolor]
{
\newpath
\moveto(251.8,466.19000122)
\lineto(260.21,473.30000122)
\lineto(257.14,467.22000122)
\closepath
}
}
{
\newrgbcolor{curcolor}{0.40000001 0.40000001 0.40000001}
\pscustom[linestyle=none,fillstyle=solid,fillcolor=curcolor]
{
\newpath
\moveto(259.47,462.31000122)
\lineto(260.21,473.30000122)
\lineto(257.14,467.22000122)
\closepath
}
}
{
\newrgbcolor{curcolor}{0 0 0}
\pscustom[linestyle=none,fillstyle=solid,fillcolor=curcolor]
{
\newpath
\moveto(313.31,273.63000122)
\lineto(319.59,276.61000122)
\lineto(318.27,283.45000122)
\lineto(327.4,272.66000122)
\closepath
}
}
{
\newrgbcolor{curcolor}{0.65098041 0.65098041 0.65098041}
\pscustom[linestyle=none,fillstyle=solid,fillcolor=curcolor]
{
\newpath
\moveto(315.24,274.00000122)
\lineto(326.23,273.25000122)
\lineto(320.15,276.33000122)
\closepath
}
}
{
\newrgbcolor{curcolor}{0.40000001 0.40000001 0.40000001}
\pscustom[linestyle=none,fillstyle=solid,fillcolor=curcolor]
{
\newpath
\moveto(319.12,281.67000122)
\lineto(326.23,273.25000122)
\lineto(320.15,276.33000122)
\closepath
}
}
{
\newrgbcolor{curcolor}{0 0 0}
\pscustom[linewidth=1,linecolor=curcolor]
{
\newpath
\moveto(215.15,249.80000122)
\lineto(251.83,322.30000122)
\lineto(197.77,349.88000122)
}
}
{
\newrgbcolor{curcolor}{0 0 0}
\pscustom[linestyle=none,fillstyle=solid,fillcolor=curcolor]
{
\newpath
\moveto(183.85300767,237.52316943)
\curveto(183.78504651,237.21734418)(183.75106592,237.1833636)(183.71708534,237.14938301)
\curveto(183.61514359,237.11540243)(183.37727951,237.11540243)(183.17339602,237.11540243)
\curveto(182.7996096,237.11540243)(182.39184261,237.11540243)(182.39184261,236.50375194)
\curveto(182.39184261,236.26588786)(182.59572611,236.09598495)(182.83359019,236.09598495)
\curveto(183.44524068,236.09598495)(184.15883292,236.16394611)(184.80446399,236.16394611)
\curveto(185.5860174,236.16394611)(186.40155138,236.09598495)(187.1491242,236.09598495)
\curveto(187.28504654,236.09598495)(187.69281353,236.09598495)(187.69281353,236.74161602)
\curveto(187.69281353,237.11540243)(187.3530077,237.11540243)(187.1491242,237.11540243)
\curveto(186.84329896,237.11540243)(186.46951255,237.11540243)(186.19766789,237.14938301)
\lineto(187.1491242,240.92122771)
\curveto(187.45494945,240.61540246)(188.16854169,240.1396743)(189.3238815,240.1396743)
\curveto(193.0957262,240.1396743)(195.44038641,243.57171317)(195.44038641,246.52802387)
\curveto(195.44038641,249.21248991)(193.43553203,250.09598507)(191.63456114,250.09598507)
\curveto(190.10543491,250.09598507)(188.98407568,249.2464705)(188.64426985,248.94064525)
\curveto(187.79475528,250.09598507)(186.3675708,250.09598507)(186.12970672,250.09598507)
\curveto(185.34815332,250.09598507)(184.70252224,249.65423749)(184.26077467,248.87268408)
\curveto(183.71708534,247.98918893)(183.4112601,246.83384912)(183.4112601,246.73190737)
\curveto(183.4112601,246.42608212)(183.75106592,246.42608212)(183.95494942,246.42608212)
\curveto(184.1928135,246.42608212)(184.26077467,246.42608212)(184.36271641,246.52802387)
\curveto(184.43067758,246.56200445)(184.43067758,246.62996562)(184.56659991,247.17365494)
\curveto(184.9743669,248.90666467)(185.48407565,249.31443166)(186.02776497,249.31443166)
\curveto(186.26562905,249.31443166)(186.53747371,249.2464705)(186.53747371,248.53287826)
\curveto(186.53747371,248.19307243)(186.46951255,247.88724718)(186.40155138,247.58142194)
\closepath
\moveto(188.84815335,247.92122777)
\curveto(189.45980384,248.66880059)(190.47922132,249.31443166)(191.53261939,249.31443166)
\curveto(192.8918427,249.31443166)(192.99378445,248.15909185)(192.99378445,247.68336369)
\curveto(192.99378445,246.56200445)(192.24621163,243.87753841)(191.9064058,243.02802384)
\curveto(191.22679414,241.46491703)(190.17339608,240.92122771)(189.28990092,240.92122771)
\curveto(187.99863878,240.92122771)(187.48893003,241.94064519)(187.48893003,242.17850927)
\lineto(187.52291062,242.48433452)
\closepath
\moveto(188.84815335,247.92122777)
}
}
{
\newrgbcolor{curcolor}{0 0 0}
\pscustom[linestyle=none,fillstyle=solid,fillcolor=curcolor]
{
\newpath
\moveto(230.24069439,340.48560115)
\lineto(228.11913079,338.5184158)
\lineto(220.70478482,342.32258442)
\lineto(220.88072762,342.66549792)
\curveto(224.08256631,343.53595448)(226.45170074,344.37315206)(227.98813092,345.17709064)
\curveto(229.5245611,345.98102922)(230.56065307,346.90509204)(231.09640682,347.9492791)
\curveto(231.50535494,348.74632129)(231.59733428,349.52647573)(231.37234484,350.28974243)
\curveto(231.14735539,351.05300913)(230.69503647,351.6090002)(230.01538809,351.95771566)
\curveto(229.39752593,352.27472971)(228.74793084,352.37777302)(228.06660281,352.26684559)
\curveto(227.39462356,352.15892663)(226.76212717,351.8161077)(226.16911367,351.23838879)
\lineto(225.82620017,351.41433159)
\curveto(226.56714171,352.47812308)(227.41273345,353.15259836)(228.36297539,353.43775743)
\curveto(229.31939595,353.71974636)(230.27644941,353.61505494)(231.23413576,353.12368316)
\curveto(232.25360833,352.60060997)(232.93515136,351.8372487)(233.27876485,350.83359935)
\curveto(233.62855696,349.82677986)(233.57520289,348.87850936)(233.11870266,347.98878784)
\curveto(232.79217819,347.35238981)(232.31736679,346.79207515)(231.69426848,346.30784387)
\curveto(230.72348772,345.54149756)(229.43837402,344.86228081)(227.83892739,344.27019362)
\curveto(225.43817238,343.37897352)(223.95749367,342.85863642)(223.39689126,342.70918232)
\lineto(226.67773936,341.02583771)
\curveto(227.34503049,340.68346253)(227.82419699,340.46883141)(228.11523884,340.38194434)
\curveto(228.41245932,340.29188713)(228.70723712,340.26552453)(228.99957225,340.30285653)
\curveto(229.29507752,340.34636715)(229.5944804,340.46592962)(229.89778089,340.66154395)
\closepath
}
}
{
\newrgbcolor{curcolor}{0 0 0}
\pscustom[linestyle=none,fillstyle=solid,fillcolor=curcolor]
{
\newpath
\moveto(224.76749728,138.20995139)
\curveto(223.64944349,139.12661661)(222.92722241,139.86272656)(222.60083404,140.41828124)
\curveto(222.2813901,140.97383592)(222.12166813,141.5502239)(222.12166813,142.14744518)
\curveto(222.12166813,143.06411039)(222.47583424,143.85230359)(223.18416645,144.51202477)
\curveto(223.89249867,145.17869039)(224.8334694,145.51202319)(226.00707866,145.51202319)
\curveto(227.14596575,145.51202319)(228.06263097,145.2029959)(228.75707432,144.58494133)
\curveto(229.45151766,143.96688675)(229.79873934,143.26202675)(229.79873934,142.47036133)
\curveto(229.79873934,141.94258439)(229.61123963,141.40439079)(229.23624023,140.85578055)
\curveto(228.86124082,140.3071703)(228.07999205,139.66133799)(226.89249393,138.91828361)
\curveto(228.11471422,137.97384066)(228.92374072,137.23078627)(229.31957343,136.68912046)
\curveto(229.84735037,135.98078825)(230.11123884,135.23426165)(230.11123884,134.44954067)
\curveto(230.11123884,133.45648668)(229.73276722,132.60579358)(228.97582397,131.89746137)
\curveto(228.21888072,131.19607358)(227.22582673,130.84537969)(225.99666201,130.84537969)
\curveto(224.65638635,130.84537969)(223.61124911,131.26551792)(222.86125029,132.10579437)
\curveto(222.26402902,132.77940442)(221.96541838,133.51551437)(221.96541838,134.31412421)
\curveto(221.96541838,134.93912323)(222.17375138,135.55717781)(222.59041739,136.16828795)
\curveto(223.01402783,136.78634253)(223.73972113,137.46689701)(224.76749728,138.20995139)
\closepath
\moveto(226.40291137,139.32453297)
\curveto(227.23624338,140.07453178)(227.76402033,140.66480863)(227.9862422,141.0953635)
\curveto(228.20846407,141.53286281)(228.31957501,142.02591759)(228.31957501,142.57452783)
\curveto(228.31957501,143.30369335)(228.11471422,143.87313689)(227.70499264,144.28285847)
\curveto(227.29527107,144.69952448)(226.73624417,144.90785748)(226.02791196,144.90785748)
\curveto(225.31957975,144.90785748)(224.74319177,144.70299669)(224.29874802,144.29327512)
\curveto(223.85430428,143.88355354)(223.63208241,143.40438763)(223.63208241,142.85577739)
\curveto(223.63208241,142.49466685)(223.72236005,142.13355631)(223.90291532,141.77244577)
\curveto(224.09041502,141.41133523)(224.35430349,141.06758577)(224.69458073,140.7411974)
\closepath
\moveto(225.25707984,137.81411869)
\curveto(224.68069187,137.32800834)(224.25360921,136.79675918)(223.97583187,136.2203712)
\curveto(223.69805453,135.65092766)(223.55916586,135.03287308)(223.55916586,134.36620747)
\curveto(223.55916586,133.47037555)(223.80222103,132.75162668)(224.28833137,132.20996087)
\curveto(224.78138615,131.67523949)(225.40638516,131.40787881)(226.16332841,131.40787881)
\curveto(226.91332723,131.40787881)(227.51402072,131.61968403)(227.9654089,132.04329447)
\curveto(228.41679708,132.46690491)(228.64249116,132.98079299)(228.64249116,133.5849587)
\curveto(228.64249116,134.08495791)(228.51054693,134.53287387)(228.24665846,134.92870658)
\curveto(227.75360368,135.66481653)(226.75707747,136.62662056)(225.25707984,137.81411869)
\closepath
}
}
{
\newrgbcolor{curcolor}{0 0 0}
\pscustom[linestyle=none,fillstyle=solid,fillcolor=curcolor]
{
\newpath
\moveto(106.21592934,14.52114139)
\curveto(105.09787555,15.43780661)(104.37565447,16.17391656)(104.0492661,16.72947124)
\curveto(103.72982216,17.28502592)(103.57010019,17.8614139)(103.57010019,18.45863518)
\curveto(103.57010019,19.37530039)(103.9242663,20.16349359)(104.63259851,20.82321477)
\curveto(105.34093073,21.48988039)(106.28190146,21.82321319)(107.45551072,21.82321319)
\curveto(108.59439781,21.82321319)(109.51106303,21.5141859)(110.20550638,20.89613133)
\curveto(110.89994972,20.27807675)(111.2471714,19.57321675)(111.2471714,18.78155133)
\curveto(111.2471714,18.25377439)(111.05967169,17.71558079)(110.68467229,17.16697055)
\curveto(110.30967288,16.6183603)(109.52842411,15.97252799)(108.34092599,15.22947361)
\curveto(109.56314628,14.28503066)(110.37217278,13.54197627)(110.76800549,13.00031046)
\curveto(111.29578243,12.29197825)(111.5596709,11.54545165)(111.5596709,10.76073067)
\curveto(111.5596709,9.76767668)(111.18119928,8.91698358)(110.42425603,8.20865137)
\curveto(109.66731278,7.50726358)(108.67425879,7.15656969)(107.44509407,7.15656969)
\curveto(106.10481841,7.15656969)(105.05968117,7.57670792)(104.30968235,8.41698437)
\curveto(103.71246108,9.09059442)(103.41385044,9.82670437)(103.41385044,10.62531421)
\curveto(103.41385044,11.25031323)(103.62218344,11.86836781)(104.03884945,12.47947795)
\curveto(104.46245989,13.09753253)(105.18815319,13.77808701)(106.21592934,14.52114139)
\closepath
\moveto(107.85134343,15.63572297)
\curveto(108.68467544,16.38572178)(109.21245239,16.97599863)(109.43467426,17.4065535)
\curveto(109.65689613,17.84405281)(109.76800707,18.33710759)(109.76800707,18.88571783)
\curveto(109.76800707,19.61488335)(109.56314628,20.18432689)(109.1534247,20.59404847)
\curveto(108.74370313,21.01071448)(108.18467623,21.21904748)(107.47634402,21.21904748)
\curveto(106.76801181,21.21904748)(106.19162383,21.01418669)(105.74718008,20.60446512)
\curveto(105.30273634,20.19474354)(105.08051447,19.71557763)(105.08051447,19.16696739)
\curveto(105.08051447,18.80585685)(105.17079211,18.44474631)(105.35134738,18.08363577)
\curveto(105.53884708,17.72252523)(105.80273555,17.37877577)(106.14301279,17.0523874)
\closepath
\moveto(106.7055119,14.12530869)
\curveto(106.12912393,13.63919834)(105.70204127,13.10794918)(105.42426393,12.5315612)
\curveto(105.14648659,11.96211766)(105.00759792,11.34406308)(105.00759792,10.67739747)
\curveto(105.00759792,9.78156555)(105.25065309,9.06281668)(105.73676343,8.52115087)
\curveto(106.22981821,7.98642949)(106.85481722,7.71906881)(107.61176047,7.71906881)
\curveto(108.36175929,7.71906881)(108.96245278,7.93087403)(109.41384096,8.35448447)
\curveto(109.86522914,8.77809491)(110.09092322,9.29198299)(110.09092322,9.8961487)
\curveto(110.09092322,10.39614791)(109.95897899,10.84406387)(109.69509052,11.23989658)
\curveto(109.20203574,11.97600653)(108.20550953,12.93781056)(106.7055119,14.12530869)
\closepath
}
}
{
\newrgbcolor{curcolor}{0 0 0}
\pscustom[linestyle=none,fillstyle=solid,fillcolor=curcolor]
{
\newpath
\moveto(235.10634431,277.72855707)
\lineto(239.29893945,275.57741132)
\lineto(238.6706551,274.3528816)
\lineto(234.47805996,276.50402735)
\closepath
}
}
{
\newrgbcolor{curcolor}{0 0 0}
\pscustom[linestyle=none,fillstyle=solid,fillcolor=curcolor]
{
\newpath
\moveto(247.54661029,270.98712763)
\lineto(246.95577335,269.83558313)
\lineto(245.47985011,270.59285301)
\lineto(243.99027529,267.68966334)
\lineto(242.65221302,268.37619922)
\lineto(244.14178783,271.27938889)
\lineto(239.486953,273.66770158)
\lineto(240.01953841,274.70571353)
\lineto(248.85681351,279.37085812)
\lineto(249.74885503,278.91316754)
\lineto(246.07068705,271.74439751)
\closepath
\moveto(244.73262477,272.43093339)
\lineto(247.53285899,277.88860559)
\lineto(240.87251783,274.41148538)
\closepath
}
}
\end{pspicture}

    \caption{2D中,点$\bm p$的坐标$(x,y)$由该点和特定2D坐标系统的关系定义。
        这里展示了两个坐标系统;
        该点关于实线坐标系统的坐标为$(8,8)$但
        关于虚线坐标系的坐标为$(2,-4)$。
        两种情况下,2D点$\bm p$都处于空间中的同一绝对位置。}
    \label{fig:2.1}
\end{figure}