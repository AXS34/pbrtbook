\section{基于物理的渲染简史}\label{sec:基于物理的渲染简史}

20世纪70年代早期计算机图形学中,
最重要的问题是解决可见性算法和几何表示那样的基础问题。
那时兆字节的RAM还是稀有昂贵的奢侈品,
每秒能执行百万次浮点运算的计算机要花费数十万美元,
因此计算机图形学中可能达到的复杂度相应地受到限制,
任何为了渲染而尝试精确模拟物理的做法都是不切实际的。

随着计算机越来越强大和廉价,
考虑计算需求更大的渲染方法成为可能,
这也让基于物理的方法变得可行。
\keyindex{布林定律}{Blinn's law}{}简洁地解释了这一过程:
“随着技术进步,渲染时间保持不变。”

Jim Blinn简单的表述抓住了一条重要的约束:
给定某数量必须要渲染的图像
(对研究论文可能是几张,对故事片则超过十万张),
每张只可能花费这么多处理时间。
有人只有特定数量的计算资源,
有人必须在特定时间内完成渲染,
因此有必要限制每张图像的最大计算量。

布林定律也表达了他的观察:
人们想要渲染的图像和他们能渲染的图像还存在差距:
随着计算机越来越快,
内容创作者会持续利用增加的计算能力
和更精巧的渲染算法渲染更加复杂的场景,
而不只是更快地渲染和以前一样的场景。
渲染会持续消耗其可用的全部计算力。

\subsection{研究}\label{sub:研究}
20世纪80年代图形学研究者开始认真考虑基于物理的渲染方法。
\citeauthor{10.1145/358876.358882}的
论文\parencite*{10.1145/358876.358882}介绍了
为全局光照效果使用光线追踪的思想,
打开了精确模拟场景中光分布的大门。
他的方法产生的渲染图像与以前见过的明显不同,
让人兴奋不已。

基于物理的渲染中另一个值得注意的早期进步是
Cook和Torrance的反射模型\parencite*{10.1145/800224.806819,10.1145/357290.357293},
将微面反射\sidenote{译者注:原文microfacet reflection。}模型引入图形学。
除其他贡献外,他们表明了
对微面反射精确建模可以让精确渲染金属表面成为可能;
更早方法无法很好地渲染金属。

不久后,Goral等\parencite*{10.1145/800031.808601}将热传递文献与渲染联系起来,
展示了如何利用基于物理的光传输近似加入全局漫射\sidenote{译者注:原文diffuse。}光效果。
该方法基于有限元\sidenote{译者注:原文finite-element。}方法,
场景中表面区域互换能量。
在有相关物理单位后,该方法称为“光能传递”
\sidenote{译者注:原文radiosity,有光能传递、辐射着色、辐射度等含义。
    此处按上下文暂译为光能传递。}。
Cohen和Greenberg\parencite*{10.1145/325334.325171}以及
Nishita和Nakamae\parencite*{10.1145/325334.325169}的后续工作引入了重要改进。
再一次,基于物理的方法得到了在以前的渲染图像中从未见过的带有光照特效的图像,
带动许多研究者在这一领域竞相提升。

尽管光能传递方法主要基于物理单位和能量守恒,
但它不能得到可行的渲染算法的事实随着时间推移越来越明显:
其渐进复杂度是难以控制的$O(n^2)$,
且需要能够沿阴影边界再细分\sidenote{译者注:原文re-tessellate。}几何模型以获得好的结果;
研究者为这个目的开发出稳定高效的细分算法是很困难的,
光能传递在实际应用中受到限制。

在光能传递的年代,一小部分研究者追求
基于光线追踪和蒙特卡洛积分的基于物理的渲染方法。
那时许多人对他们的工作持怀疑态度;
蒙特卡洛积分的变化在图像中引起令人讨厌的噪声看起来是不可避免的,
而基于光能传递的方法至少在相对简单的场景上能快速给出视觉上讨人喜欢的结果。

1984年,Cook、Porter和Carpenter\parencite*{10.1145/800031.808590}引入了分布式光线追踪,
将Whitted的算法推广到从相机计算运动模糊和散焦模糊、
来自光泽表面的模糊反射以及来自面光源的照明,
表明光线追踪能生成众多重要的光照效果。

之后不久,\citet{10.1145/15922.15902}引入了路径追踪;
他提出了渲染问题的严格公式(光传输方程)
并说明了如何运用蒙特卡洛积分解决它。
这项工作需要大量计算:
用路径追踪渲染一幅有两个球体的$256\times256$像素图像
需要在一台IBM 4341计算机上运行7小时,
其首次发布花费了约280,000美元\citep{farmer1981comparing}。
von Herzen和Kajiya\parencite*{10.1145/800031.808594}也
把体积渲染方程引入图形学;
该方程严格描述了介质中的光散射。

Cook等和Kajiya的工作都再次得到了前所未见的图像,
证明了基于物理的方法的价值。
随后几年, Arvo和Kirk\parencite*{10.1145/97879.97886}以及
Kirk和Arvo\parencite*{10.1145/122718.122735}的论文
描述了关于蒙特卡洛合成逼真图像的重要工作。
\citeauthor{10.5555/124947}的博士论文\parencite*{10.5555/124947}和
后续Shirley等\parencite*{10.1145/226150.226151}的工作
是对基于蒙特卡洛方法的重要贡献。
\citeauthor{10.5555/63450}的书籍《\citetitle{10.5555/63450}》\parencite*{10.5555/63450}是
最早以基于物理的框架论述渲染的书之一,
\citeauthor{10.5555/527570}的《\citetitle{10.5555/527570}》\parencite*{10.5555/527570}严格
奠定了该领域的基础。
\citeauthor{10.1145/192161.192286}的\emph{Radiance}渲染系统
是早期基于物理的开源渲染系统,专注光照设计\citep{10.1145/192161.192286},
\citeauthor{slusallek1996vision}的\emph{Vision}渲染器专门缩小基于物理的方法
和当时广泛使用的不基于物理的RenderMan接口之间的差距\citep{slusallek1996vision}。

继Torrance和Cook的工作后,
康奈尔大学\sidenote{译者注:Cornell University。}计算机图形学计划的
许多研究都调研了基于物理的方法。
Greenberg等\parencite*{10.1145/258734.258914}总结了这些工作的动机,
基于真实世界材料属性的度量和对人类视觉系统的深刻理解,
他们强烈支持物理精确渲染。

\citeauthor{veach1997robust}的工作使基于物理的渲染迈出关键一步,
他的博士论文\citep{veach1997robust}对此进行了详细描述。
\citeauthor{veach1997robust}改进了蒙特卡洛渲染的关键理论基础
还开发了多重要性采样\sidenote{译者注:原文multiple importance sampling。}、
双向路径追踪和Metropolis光传输那样的新算法,极大提高了效率。
以布林定律为指导,我们认为这些效率上的显著提升对这些方法的实际应用至关重要。

大约这时,随着计算机变得更快更并行化,
许多研究者开始追求实时光线追踪;
Wald、Slusallek和Benthin写了一篇很有影响力的论文,描述了一个
比以往的光线追踪器高效得多的高度优化的光线追踪器\citep{10.1007/978-3-7091-6242-2_26}。
许多后续论文介绍了越来越高效的光线追踪算法。
尽管这些工作大多数都不是基于物理的,
但其结果让光线追踪加速结构和光线追踪几何部分的性能有了巨大进步。
因为基于物理的渲染一般大量使用光线追踪,
所以这项工作反过来也带来了和更快的计算机相同的好处,
使得用物理方法渲染更复杂的场景成为可能。

至此,我们将结束对基于物理的渲染研究进展中关键步骤的总结;
还有许多工作没有提到。
本书所有后续章节的“扩展阅读”部分会详细介绍这些工作。

\subsection{制作}\label{sub:制作}
随着20世纪80年代出现更强大的计算机,
计算机图形学开始用于动画和电影制作。