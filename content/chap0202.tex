\section{向量}\label{sec:向量}

pbrt提供了2D和3D向量类。
两者都由基本向量元素参数化,
因此易于实例化整数和浮点类型的向量。

\begin{lstlisting}
`\initcode{Vector Declarations}{=}\initnext{VectorDeclarations}`
template <typename T> class `\initvar{Vector2}{}` {
public:
    `\refcode{Vector2 Public Methods}{}`
    `\refcode{Vector2 Public Data}{}`
};

`\refcode{Vector Declarations}{+=}\lastnext{VectorDeclarations}`
template <typename T> class `\initvar{Vector3}{}` {
public:
    `\refcode{Vector3 Public Methods}{}`
    `\refcode{Vector3 Public Data}{}`
};
\end{lstlisting}

接下来,我们一般只介绍\refvar{Vector3}{}方法的实现;
它们都有\refvar{Vector2}{}的相应版本,实现的区别也很简单。

一个向量表示为分量的元组,以所在定义空间$x,y,z$(3D)轴的形式给出其表达。
一个3D向量$\bm v$的各个分量写作$v_x,v_y$和$v_z$。

\begin{lstlisting}
`\initcode{Vector2 Public Data}{=}`
T `\initvar[Vector2::x]{x}{}`, `\initvar[Vector2::y]{y}{}`;

`\initcode{Vector3 Public Data}{=}`
T `\initvar[Vector3::x]{x}{}`, `\initvar[Vector3::y]{y}{}`, `\initvar[Vector3::z]{z}{}`;
\end{lstlisting}

另一实现方式是单个模板类再用一个维度整数参数化,
并以若干{\ttfamily T}值构成的数组表示坐标。
尽管这种方法能减少代码总量,
但向量的单个分量不能以{\ttfamily v.x}等形式访问。
我们认为这种情况下,为了更透明地访问元素,
向量实现中多一点代码是值得的。

然而,一些例程发现能简单地遍历向量的分量是很有用的;
向量类也提供了C++操作符以索引分量,
这样给定向量{\ttfamily v},有{\ttfamily v[0]==v.x}等。
\begin{lstlisting}
`\initcode{Vector3 Public Methods}{=}\initnext{Vector3PublicMethods}`
T operator[](int i) const { 
    Assert(i >= 0 && i <= 2);
    if (i == 0) return x;
    if (i == 1) return y;
    return z;
}
T &operator[](int i) { 
    Assert(i >= 0 && i <= 2);
    if (i == 0) return x;
    if (i == 1) return y;
    return z;
}
\end{lstlisting}

为了方便起见,{\ttfamily typedef}指定了许多向量常用的类型,
这样在别处代码中它们会有更简洁的名字。
\begin{lstlisting}
`\refcode{Vector Declarations}{+=}\lastcode{VectorDeclarations}`
typedef `\refvar{Vector2}{}`<`\refvar{Float}{}`> `\initvar{Vector2f}{}`;
typedef `\refvar{Vector2}{}`<int>   `\initvar{Vector2i}{}`;
typedef `\refvar{Vector3}{}`<`\refvar{Float}{}`> `\initvar{Vector3f}{}`;
typedef `\refvar{Vector3}{}`<int>   `\initvar{Vector3i}{}`;
\end{lstlisting}

接触过面向对象设计的读者可能质疑我们让向量元素数据可公开访问的决定。
通常,数据成员只在其类内是可访问的,
外部代码想要访问或修改类的内容必须通过良好定义的选择器\sidenote{译者注:原文selector。}
和修改器\sidenote{译者注:原文mutator。}函数API。
尽管我们通常统一这条设计原则
(但也有第\refchap{绪论}中“扩展阅读”一节关于面向数据的设计的讨论),
可是这里它并不合适。
选择器和修改器函数的目的是隐藏类内实现细节。
对于向量,隐藏其设计的基本部分没有好处且会增加使用代码的体积。

默认情况下,值$(x,y,z)$设为零,
但类的用户也可以为每个分类提供值。
如果用户确实提供了值,
我们就用宏\refvar{Assert}{()}检查确认其中没有
“not a number”(NaN)值。
当以优化模式编译时,该宏从编译代码中消失,
节约验证这种情况的开销。
NaN几乎一定表明系统存在bug;
如果NaN是某些计算生成的,
我们就要尽快排查以便隔离其来源
(更多关于NaN值的讨论详见\refsub{浮点算术})。
\begin{lstlisting}
`\refcode{Vector3 Public Methods}{+=}\lastnext{Vector3PublicMethods}`
`\refvar{Vector3}{}`() { x = y = z = 0; }
`\refvar{Vector3}{}`(T x, T y, T z)
    : x(x), y(y), z(z) {
    `\refvar{Assert}{}`(!`\refvar{HasNaNs}{}`());
}
\end{lstlisting}

检查NaN的代码只是为$x,y$和$z$分量中的每一个调用函数{\ttfamily std::isnan()}。
\begin{lstlisting}
`\refcode{Vector3 Public Methods}{+=}\lastnext{Vector3PublicMethods}`
bool `\initvar{HasNaNs}{}`() const {
    return std::isnan(x) || std::isnan(y) || std::isnan(z);
}
\end{lstlisting}

向量的加法和减法逐分量进行。
向量加减法的一般几何解释如\reffig{2.3}和\reffig{2.4}所示。
\begin{figure}[htbp]
    \centering%LaTeX with PSTricks extensions
%%Creator: Inkscape 1.0.1 (3bc2e813f5, 2020-09-07)
%%Please note this file requires PSTricks extensions
\psset{xunit=.5pt,yunit=.5pt,runit=.5pt}
\begin{pspicture}(357.38000488,201.63999939)
{
\newrgbcolor{curcolor}{0 0 0}
\pscustom[linewidth=1,linecolor=curcolor]
{
\newpath
\moveto(0.38999999,0.47999573)
\lineto(92.40000153,28.56999207)
}
}
{
\newrgbcolor{curcolor}{0 0 0}
\pscustom[linestyle=none,fillstyle=solid,fillcolor=curcolor]
{
\newpath
\moveto(89.32,21.86999939)
\lineto(91.78,28.37999939)
\lineto(86.1,32.39999939)
\lineto(100.16,30.92999939)
\closepath
}
}
{
\newrgbcolor{curcolor}{0.65098041 0.65098041 0.65098041}
\pscustom[linestyle=none,fillstyle=solid,fillcolor=curcolor]
{
\newpath
\moveto(90.45,23.47999939)
\lineto(98.9,30.54999939)
\lineto(92.39,28.55999939)
\closepath
}
}
{
\newrgbcolor{curcolor}{0.40000001 0.40000001 0.40000001}
\pscustom[linestyle=none,fillstyle=solid,fillcolor=curcolor]
{
\newpath
\moveto(87.94,31.69999939)
\lineto(98.9,30.54999939)
\lineto(92.39,28.55999939)
\closepath
}
}
{
\newrgbcolor{curcolor}{0 0 0}
\pscustom[linewidth=1,linecolor=curcolor,linestyle=dashed,dash=4]
{
\newpath
\moveto(141.63999939,182.25)
\lineto(0.38999999,0.47999573)
}
}
{
\newrgbcolor{curcolor}{0 0 0}
\pscustom[linewidth=1,linecolor=curcolor]
{
\newpath
\moveto(100.16000366,30.92999268)
\lineto(139.5,174.43999863)
}
}
{
\newrgbcolor{curcolor}{0 0 0}
\pscustom[linestyle=none,fillstyle=solid,fillcolor=curcolor]
{
\newpath
\moveto(143.51,168.24999939)
\lineto(139.33,173.80999939)
\lineto(132.89,171.15999939)
\lineto(141.64,182.24999939)
\closepath
}
}
{
\newrgbcolor{curcolor}{0.65098041 0.65098041 0.65098041}
\pscustom[linestyle=none,fillstyle=solid,fillcolor=curcolor]
{
\newpath
\moveto(142.76,170.06999939)
\lineto(141.29,180.98999939)
\lineto(139.49,174.41999939)
\closepath
}
}
{
\newrgbcolor{curcolor}{0.40000001 0.40000001 0.40000001}
\pscustom[linestyle=none,fillstyle=solid,fillcolor=curcolor]
{
\newpath
\moveto(134.47,172.33999939)
\lineto(141.29,180.98999939)
\lineto(139.49,174.41999939)
\closepath
}
}
{
\newrgbcolor{curcolor}{0 0 0}
\pscustom[linewidth=1,linecolor=curcolor]
{
\newpath
\moveto(196.33999634,0.47999573)
\lineto(288.3500061,28.56999207)
}
}
{
\newrgbcolor{curcolor}{0 0 0}
\pscustom[linestyle=none,fillstyle=solid,fillcolor=curcolor]
{
\newpath
\moveto(285.26,21.86999939)
\lineto(287.73,28.37999939)
\lineto(282.05,32.39999939)
\lineto(296.1,30.92999939)
\closepath
}
}
{
\newrgbcolor{curcolor}{0.65098041 0.65098041 0.65098041}
\pscustom[linestyle=none,fillstyle=solid,fillcolor=curcolor]
{
\newpath
\moveto(286.4,23.47999939)
\lineto(294.85,30.54999939)
\lineto(288.33,28.55999939)
\closepath
}
}
{
\newrgbcolor{curcolor}{0.40000001 0.40000001 0.40000001}
\pscustom[linestyle=none,fillstyle=solid,fillcolor=curcolor]
{
\newpath
\moveto(283.89,31.69999939)
\lineto(294.85,30.54999939)
\lineto(288.33,28.55999939)
\closepath
}
}
{
\newrgbcolor{curcolor}{0 0 0}
\pscustom[linewidth=1,linecolor=curcolor,linestyle=dashed,dash=4]
{
\newpath
\moveto(337.58999634,182.25)
\lineto(196.33999634,0.47999573)
}
}
{
\newrgbcolor{curcolor}{0 0 0}
\pscustom[linewidth=1,linecolor=curcolor]
{
\newpath
\moveto(296.1000061,30.92999268)
\lineto(335.44000244,174.43999863)
}
}
{
\newrgbcolor{curcolor}{0 0 0}
\pscustom[linestyle=none,fillstyle=solid,fillcolor=curcolor]
{
\newpath
\moveto(339.45,168.24999939)
\lineto(335.27,173.80999939)
\lineto(328.84,171.15999939)
\lineto(337.59,182.24999939)
\closepath
}
}
{
\newrgbcolor{curcolor}{0.65098041 0.65098041 0.65098041}
\pscustom[linestyle=none,fillstyle=solid,fillcolor=curcolor]
{
\newpath
\moveto(338.71,170.06999939)
\lineto(337.24,180.98999939)
\lineto(335.44,174.41999939)
\closepath
}
}
{
\newrgbcolor{curcolor}{0.40000001 0.40000001 0.40000001}
\pscustom[linestyle=none,fillstyle=solid,fillcolor=curcolor]
{
\newpath
\moveto(330.42,172.33999939)
\lineto(337.24,180.98999939)
\lineto(335.44,174.41999939)
\closepath
}
}
{
\newrgbcolor{curcolor}{0 0 0}
\pscustom[linewidth=1,linecolor=curcolor]
{
\newpath
\moveto(237.69999695,151.65999985)
\lineto(329.70999146,179.75)
}
}
{
\newrgbcolor{curcolor}{0 0 0}
\pscustom[linestyle=none,fillstyle=solid,fillcolor=curcolor]
{
\newpath
\moveto(326.62,173.04999939)
\lineto(329.08,179.54999939)
\lineto(323.4,183.57999939)
\lineto(337.46,182.10999939)
\closepath
}
}
{
\newrgbcolor{curcolor}{0.65098041 0.65098041 0.65098041}
\pscustom[linestyle=none,fillstyle=solid,fillcolor=curcolor]
{
\newpath
\moveto(327.76,174.64999939)
\lineto(336.2,181.72999939)
\lineto(329.69,179.73999939)
\closepath
}
}
{
\newrgbcolor{curcolor}{0.40000001 0.40000001 0.40000001}
\pscustom[linestyle=none,fillstyle=solid,fillcolor=curcolor]
{
\newpath
\moveto(325.25,182.87999939)
\lineto(336.2,181.72999939)
\lineto(329.69,179.73999939)
\closepath
}
}
{
\newrgbcolor{curcolor}{0 0 0}
\pscustom[linewidth=1,linecolor=curcolor]
{
\newpath
\moveto(196.13000488,0.16000366)
\lineto(235.47000122,143.65999985)
}
}
{
\newrgbcolor{curcolor}{0 0 0}
\pscustom[linestyle=none,fillstyle=solid,fillcolor=curcolor]
{
\newpath
\moveto(239.48,137.46999939)
\lineto(235.3,143.03999939)
\lineto(228.87,140.37999939)
\lineto(237.62,151.47999939)
\closepath
}
}
{
\newrgbcolor{curcolor}{0.65098041 0.65098041 0.65098041}
\pscustom[linestyle=none,fillstyle=solid,fillcolor=curcolor]
{
\newpath
\moveto(238.74,139.28999939)
\lineto(237.27,150.20999939)
\lineto(235.47,143.63999939)
\closepath
}
}
{
\newrgbcolor{curcolor}{0.40000001 0.40000001 0.40000001}
\pscustom[linestyle=none,fillstyle=solid,fillcolor=curcolor]
{
\newpath
\moveto(230.44,141.56999939)
\lineto(237.27,150.20999939)
\lineto(235.47,143.63999939)
\closepath
}
}
{
\newrgbcolor{curcolor}{0 0 0}
\pscustom[linestyle=none,fillstyle=solid,fillcolor=curcolor]
{
\newpath
\moveto(78.88808297,13.53043718)
\curveto(78.88808297,15.51807615)(77.41448856,15.51807615)(77.41448856,15.51807615)
\curveto(76.52347798,15.51807615)(75.73527632,14.59279594)(75.73527632,13.87313355)
\curveto(75.73527632,13.25628008)(76.21505125,12.98212298)(76.38639943,12.87931406)
\curveto(77.31167965,12.33099986)(77.48302783,11.91976421)(77.48302783,11.50852856)
\curveto(77.48302783,11.02875364)(76.21505125,6.23100439)(73.67909807,6.23100439)
\curveto(72.10269475,6.23100439)(72.10269475,7.53325061)(72.10269475,7.94448626)
\curveto(72.10269475,9.21246285)(72.71954822,10.78886618)(73.40494097,12.53661769)
\curveto(73.57628916,12.98212298)(73.64482843,13.1877408)(73.64482843,13.53043718)
\curveto(73.64482843,14.79841376)(72.37685185,15.48380651)(71.17741453,15.48380651)
\curveto(68.84707918,15.48380651)(67.75045078,12.53661769)(67.75045078,12.0911124)
\curveto(67.75045078,11.78268566)(68.09314716,11.78268566)(68.29876498,11.78268566)
\curveto(68.53865245,11.78268566)(68.71000063,11.78268566)(68.77853991,12.05684276)
\curveto(69.4982023,14.42144775)(70.66336997,14.69560485)(71.04033598,14.69560485)
\curveto(71.21168417,14.69560485)(71.417302,14.69560485)(71.417302,14.25009956)
\curveto(71.417302,13.736055)(71.1431449,13.11920153)(71.07460562,12.94785334)
\curveto(70.08078613,10.41190016)(69.73808976,9.41808068)(69.73808976,8.35572191)
\curveto(69.73808976,6.0596562)(71.62291982,5.44280273)(73.54201952,5.44280273)
\curveto(77.31167965,5.44280273)(78.88808297,11.64560711)(78.88808297,13.53043718)
\closepath
\moveto(78.88808297,13.53043718)
}
}
{
\newrgbcolor{curcolor}{0 0 0}
\pscustom[linestyle=none,fillstyle=solid,fillcolor=curcolor]
{
\newpath
\moveto(149.03592335,151.55031599)
\curveto(149.10446262,151.89301236)(149.24154117,152.44132656)(149.24154117,152.54413548)
\curveto(149.24154117,153.0239104)(148.8988448,153.50368533)(148.21345205,153.50368533)
\curveto(147.87075567,153.50368533)(147.04828437,153.33233714)(146.77412727,152.33851765)
\curveto(146.39716126,150.96773215)(146.02019525,149.39132883)(145.64322923,147.88346478)
\curveto(145.47188105,147.12953275)(145.47188105,146.85537565)(145.47188105,146.54694891)
\curveto(145.47188105,145.96436508)(145.54042032,145.99863471)(145.54042032,145.86155616)
\curveto(145.54042032,145.75874725)(144.95783648,144.42223139)(143.6898599,144.42223139)
\curveto(141.90783875,144.42223139)(141.90783875,145.75874725)(141.90783875,146.20425254)
\curveto(141.90783875,147.12953275)(142.18199585,148.15762188)(143.07300642,150.38514831)
\curveto(143.24435461,150.8306536)(143.44997243,151.34469816)(143.44997243,151.72166418)
\curveto(143.44997243,152.98964076)(142.18199585,153.67503351)(140.98255853,153.67503351)
\curveto(138.65222318,153.67503351)(137.55559478,150.72784469)(137.55559478,150.2823394)
\curveto(137.55559478,149.97391266)(137.89829116,149.97391266)(138.10390898,149.97391266)
\curveto(138.34379645,149.97391266)(138.51514463,149.97391266)(138.58368391,150.24806976)
\curveto(139.3033463,152.68121403)(140.50278361,152.88683185)(140.84547998,152.88683185)
\curveto(140.98255853,152.88683185)(141.222446,152.88683185)(141.222446,152.44132656)
\curveto(141.222446,151.927282)(141.01682817,151.3789678)(140.74267107,150.72784469)
\curveto(139.88593013,148.60312716)(139.54323376,147.50649876)(139.54323376,146.61548819)
\curveto(139.54323376,144.18234393)(141.66795128,143.63402973)(143.55278135,143.63402973)
\curveto(143.964017,143.63402973)(144.99210612,143.63402973)(145.98592561,144.86773668)
\curveto(146.56850945,144.11380465)(147.56232893,143.63402973)(149.27581081,143.63402973)
\curveto(150.57805703,143.63402973)(151.74322471,144.2508832)(152.7370442,146.20425254)
\curveto(153.62805477,147.8149255)(154.27917788,150.59076614)(154.27917788,151.72166418)
\curveto(154.27917788,153.70930315)(152.83985311,153.70930315)(152.80558347,153.70930315)
\curveto(151.98311217,153.70930315)(151.16064087,152.81829258)(151.16064087,152.06436055)
\curveto(151.16064087,151.44750708)(151.60614616,151.17334998)(151.81176398,151.07054106)
\curveto(152.66850492,150.5564965)(152.90839238,150.17953049)(152.90839238,149.69975556)
\curveto(152.90839238,149.35705919)(152.36007818,147.2666113)(151.67468543,146.06717399)
\curveto(151.05783196,144.97054559)(150.37243921,144.42223139)(149.41288936,144.42223139)
\curveto(147.83648603,144.42223139)(147.83648603,145.72447761)(147.83648603,146.13571326)
\curveto(147.83648603,146.7182971)(147.90502531,147.02672384)(148.17918241,148.12335224)
\curveto(148.31626096,148.77447535)(148.59041806,149.83683411)(148.72749661,150.35087868)
\closepath
\moveto(149.03592335,151.55031599)
}
}
{
\newrgbcolor{curcolor}{0 0 0}
\pscustom[linestyle=none,fillstyle=solid,fillcolor=curcolor]
{
\newpath
\moveto(28.61004297,120.73253718)
\curveto(28.61004297,122.72017615)(27.13644856,122.72017615)(27.13644856,122.72017615)
\curveto(26.24543798,122.72017615)(25.45723632,121.79489594)(25.45723632,121.07523355)
\curveto(25.45723632,120.45838008)(25.93701125,120.18422298)(26.10835943,120.08141406)
\curveto(27.03363965,119.53309986)(27.20498783,119.12186421)(27.20498783,118.71062856)
\curveto(27.20498783,118.23085364)(25.93701125,113.43310439)(23.40105807,113.43310439)
\curveto(21.82465475,113.43310439)(21.82465475,114.73535061)(21.82465475,115.14658626)
\curveto(21.82465475,116.41456285)(22.44150822,117.99096618)(23.12690097,119.73871769)
\curveto(23.29824916,120.18422298)(23.36678843,120.3898408)(23.36678843,120.73253718)
\curveto(23.36678843,122.00051376)(22.09881185,122.68590651)(20.89937453,122.68590651)
\curveto(18.56903918,122.68590651)(17.47241078,119.73871769)(17.47241078,119.2932124)
\curveto(17.47241078,118.98478566)(17.81510716,118.98478566)(18.02072498,118.98478566)
\curveto(18.26061245,118.98478566)(18.43196063,118.98478566)(18.50049991,119.25894276)
\curveto(19.2201623,121.62354775)(20.38532997,121.89770485)(20.76229598,121.89770485)
\curveto(20.93364417,121.89770485)(21.139262,121.89770485)(21.139262,121.45219956)
\curveto(21.139262,120.938155)(20.8651049,120.32130153)(20.79656562,120.14995334)
\curveto(19.80274613,117.61400016)(19.46004976,116.62018068)(19.46004976,115.55782191)
\curveto(19.46004976,113.2617562)(21.34487982,112.64490273)(23.26397952,112.64490273)
\curveto(27.03363965,112.64490273)(28.61004297,118.84770711)(28.61004297,120.73253718)
\closepath
\moveto(28.61004297,120.73253718)
}
}
{
\newrgbcolor{curcolor}{0 0 0}
\pscustom[linestyle=none,fillstyle=solid,fillcolor=curcolor]
{
\newpath
\moveto(45.25328987,117.61400016)
\lineto(52.00440846,117.61400016)
\curveto(52.3128352,117.61400016)(52.96395831,117.61400016)(52.96395831,118.29939291)
\curveto(52.96395831,118.95051603)(52.3128352,118.95051603)(52.00440846,118.95051603)
\lineto(45.25328987,118.95051603)
\lineto(45.25328987,125.73590425)
\curveto(45.25328987,126.01006135)(45.25328987,126.66118446)(44.60216676,126.66118446)
\curveto(43.91677401,126.66118446)(43.91677401,126.01006135)(43.91677401,125.73590425)
\lineto(43.91677401,118.95051603)
\lineto(37.16565542,118.95051603)
\curveto(36.85722869,118.95051603)(36.24037521,118.95051603)(36.24037521,118.29939291)
\curveto(36.24037521,117.61400016)(36.89149832,117.61400016)(37.16565542,117.61400016)
\lineto(43.91677401,117.61400016)
\lineto(43.91677401,110.86288158)
\curveto(43.91677401,110.58872448)(43.91677401,109.93760136)(44.60216676,109.93760136)
\curveto(45.25328987,109.93760136)(45.25328987,110.55445484)(45.25328987,110.86288158)
\closepath
\moveto(45.25328987,117.61400016)
}
}
{
\newrgbcolor{curcolor}{0 0 0}
\pscustom[linestyle=none,fillstyle=solid,fillcolor=curcolor]
{
\newpath
\moveto(71.39883124,120.56118899)
\curveto(71.46737051,120.90388536)(71.60444906,121.45219956)(71.60444906,121.55500848)
\curveto(71.60444906,122.0347834)(71.26175269,122.51455833)(70.57635994,122.51455833)
\curveto(70.23366356,122.51455833)(69.41119226,122.34321014)(69.13703516,121.34939065)
\curveto(68.76006915,119.97860515)(68.38310314,118.40220183)(68.00613712,116.89433778)
\curveto(67.83478894,116.14040575)(67.83478894,115.86624865)(67.83478894,115.55782191)
\curveto(67.83478894,114.97523808)(67.90332821,115.00950771)(67.90332821,114.87242916)
\curveto(67.90332821,114.76962025)(67.32074437,113.43310439)(66.05276779,113.43310439)
\curveto(64.27074664,113.43310439)(64.27074664,114.76962025)(64.27074664,115.21512554)
\curveto(64.27074664,116.14040575)(64.54490374,117.16849488)(65.43591431,119.39602131)
\curveto(65.6072625,119.8415266)(65.81288032,120.35557116)(65.81288032,120.73253718)
\curveto(65.81288032,122.00051376)(64.54490374,122.68590651)(63.34546642,122.68590651)
\curveto(61.01513107,122.68590651)(59.91850267,119.73871769)(59.91850267,119.2932124)
\curveto(59.91850267,118.98478566)(60.26119905,118.98478566)(60.46681687,118.98478566)
\curveto(60.70670434,118.98478566)(60.87805252,118.98478566)(60.9465918,119.25894276)
\curveto(61.66625419,121.69208703)(62.8656915,121.89770485)(63.20838787,121.89770485)
\curveto(63.34546642,121.89770485)(63.58535389,121.89770485)(63.58535389,121.45219956)
\curveto(63.58535389,120.938155)(63.37973606,120.3898408)(63.10557896,119.73871769)
\curveto(62.24883802,117.61400016)(61.90614165,116.51737176)(61.90614165,115.62636119)
\curveto(61.90614165,113.19321693)(64.03085917,112.64490273)(65.91568924,112.64490273)
\curveto(66.32692489,112.64490273)(67.35501401,112.64490273)(68.3488335,113.87860968)
\curveto(68.93141734,113.12467765)(69.92523682,112.64490273)(71.6387187,112.64490273)
\curveto(72.94096492,112.64490273)(74.1061326,113.2617562)(75.09995209,115.21512554)
\curveto(75.99096266,116.8257985)(76.64208577,119.60163914)(76.64208577,120.73253718)
\curveto(76.64208577,122.72017615)(75.202761,122.72017615)(75.16849136,122.72017615)
\curveto(74.34602006,122.72017615)(73.52354876,121.82916558)(73.52354876,121.07523355)
\curveto(73.52354876,120.45838008)(73.96905405,120.18422298)(74.17467187,120.08141406)
\curveto(75.03141281,119.5673695)(75.27130027,119.19040349)(75.27130027,118.71062856)
\curveto(75.27130027,118.36793219)(74.72298607,116.2774843)(74.03759332,115.07804699)
\curveto(73.42073985,113.98141859)(72.7353471,113.43310439)(71.77579725,113.43310439)
\curveto(70.19939392,113.43310439)(70.19939392,114.73535061)(70.19939392,115.14658626)
\curveto(70.19939392,115.7291701)(70.2679332,116.03759684)(70.5420903,117.13422524)
\curveto(70.67916885,117.78534835)(70.95332595,118.84770711)(71.0904045,119.36175168)
\closepath
\moveto(71.39883124,120.56118899)
}
}
{
\newrgbcolor{curcolor}{0 0 0}
\pscustom[linestyle=none,fillstyle=solid,fillcolor=curcolor]
{
\newpath
\moveto(284.96123197,14.64258118)
\curveto(284.96123197,16.63022015)(283.48763756,16.63022015)(283.48763756,16.63022015)
\curveto(282.59662698,16.63022015)(281.80842532,15.70493994)(281.80842532,14.98527755)
\curveto(281.80842532,14.36842408)(282.28820025,14.09426698)(282.45954843,13.99145806)
\curveto(283.38482865,13.44314386)(283.55617683,13.03190821)(283.55617683,12.62067256)
\curveto(283.55617683,12.14089764)(282.28820025,7.34314839)(279.75224707,7.34314839)
\curveto(278.17584375,7.34314839)(278.17584375,8.64539461)(278.17584375,9.05663026)
\curveto(278.17584375,10.32460685)(278.79269722,11.90101018)(279.47808997,13.64876169)
\curveto(279.64943816,14.09426698)(279.71797743,14.2998848)(279.71797743,14.64258118)
\curveto(279.71797743,15.91055776)(278.45000085,16.59595051)(277.25056353,16.59595051)
\curveto(274.92022818,16.59595051)(273.82359978,13.64876169)(273.82359978,13.2032564)
\curveto(273.82359978,12.89482966)(274.16629616,12.89482966)(274.37191398,12.89482966)
\curveto(274.61180145,12.89482966)(274.78314963,12.89482966)(274.85168891,13.16898676)
\curveto(275.5713513,15.53359175)(276.73651897,15.80774885)(277.11348498,15.80774885)
\curveto(277.28483317,15.80774885)(277.490451,15.80774885)(277.490451,15.36224356)
\curveto(277.490451,14.848199)(277.2162939,14.23134553)(277.14775462,14.05999734)
\curveto(276.15393513,11.52404416)(275.81123876,10.53022468)(275.81123876,9.46786591)
\curveto(275.81123876,7.1718002)(277.69606882,6.55494673)(279.61516852,6.55494673)
\curveto(283.38482865,6.55494673)(284.96123197,12.75775111)(284.96123197,14.64258118)
\closepath
\moveto(284.96123197,14.64258118)
}
}
{
\newrgbcolor{curcolor}{0 0 0}
\pscustom[linestyle=none,fillstyle=solid,fillcolor=curcolor]
{
\newpath
\moveto(318.65489297,188.78961418)
\curveto(318.65489297,190.77725315)(317.18129856,190.77725315)(317.18129856,190.77725315)
\curveto(316.29028798,190.77725315)(315.50208632,189.85197294)(315.50208632,189.13231055)
\curveto(315.50208632,188.51545708)(315.98186125,188.24129998)(316.15320943,188.13849106)
\curveto(317.07848965,187.59017686)(317.24983783,187.17894121)(317.24983783,186.76770556)
\curveto(317.24983783,186.28793064)(315.98186125,181.49018139)(313.44590807,181.49018139)
\curveto(311.86950475,181.49018139)(311.86950475,182.79242761)(311.86950475,183.20366326)
\curveto(311.86950475,184.47163985)(312.48635822,186.04804318)(313.17175097,187.79579469)
\curveto(313.34309916,188.24129998)(313.41163843,188.4469178)(313.41163843,188.78961418)
\curveto(313.41163843,190.05759076)(312.14366185,190.74298351)(310.94422453,190.74298351)
\curveto(308.61388918,190.74298351)(307.51726078,187.79579469)(307.51726078,187.3502894)
\curveto(307.51726078,187.04186266)(307.85995716,187.04186266)(308.06557498,187.04186266)
\curveto(308.30546245,187.04186266)(308.47681063,187.04186266)(308.54534991,187.31601976)
\curveto(309.2650123,189.68062475)(310.43017997,189.95478185)(310.80714598,189.95478185)
\curveto(310.97849417,189.95478185)(311.184112,189.95478185)(311.184112,189.50927656)
\curveto(311.184112,188.995232)(310.9099549,188.37837853)(310.84141562,188.20703034)
\curveto(309.84759613,185.67107716)(309.50489976,184.67725768)(309.50489976,183.61489891)
\curveto(309.50489976,181.3188332)(311.38972982,180.70197973)(313.30882952,180.70197973)
\curveto(317.07848965,180.70197973)(318.65489297,186.90478411)(318.65489297,188.78961418)
\closepath
\moveto(318.65489297,188.78961418)
}
}
{
\newrgbcolor{curcolor}{0 0 0}
\pscustom[linestyle=none,fillstyle=solid,fillcolor=curcolor]
{
\newpath
\moveto(349.89827935,159.46756599)
\curveto(349.96681862,159.81026236)(350.10389717,160.35857656)(350.10389717,160.46138548)
\curveto(350.10389717,160.9411604)(349.7612008,161.42093533)(349.07580805,161.42093533)
\curveto(348.73311167,161.42093533)(347.91064037,161.24958714)(347.63648327,160.25576765)
\curveto(347.25951726,158.88498215)(346.88255125,157.30857883)(346.50558523,155.80071478)
\curveto(346.33423705,155.04678275)(346.33423705,154.77262565)(346.33423705,154.46419891)
\curveto(346.33423705,153.88161508)(346.40277632,153.91588471)(346.40277632,153.77880616)
\curveto(346.40277632,153.67599725)(345.82019248,152.33948139)(344.5522159,152.33948139)
\curveto(342.77019475,152.33948139)(342.77019475,153.67599725)(342.77019475,154.12150254)
\curveto(342.77019475,155.04678275)(343.04435185,156.07487188)(343.93536242,158.30239831)
\curveto(344.10671061,158.7479036)(344.31232843,159.26194816)(344.31232843,159.63891418)
\curveto(344.31232843,160.90689076)(343.04435185,161.59228351)(341.84491453,161.59228351)
\curveto(339.51457918,161.59228351)(338.41795078,158.64509469)(338.41795078,158.1995894)
\curveto(338.41795078,157.89116266)(338.76064716,157.89116266)(338.96626498,157.89116266)
\curveto(339.20615245,157.89116266)(339.37750063,157.89116266)(339.44603991,158.16531976)
\curveto(340.1657023,160.59846403)(341.36513961,160.80408185)(341.70783598,160.80408185)
\curveto(341.84491453,160.80408185)(342.084802,160.80408185)(342.084802,160.35857656)
\curveto(342.084802,159.844532)(341.87918417,159.2962178)(341.60502707,158.64509469)
\curveto(340.74828613,156.52037716)(340.40558976,155.42374876)(340.40558976,154.53273819)
\curveto(340.40558976,152.09959393)(342.53030728,151.55127973)(344.41513735,151.55127973)
\curveto(344.826373,151.55127973)(345.85446212,151.55127973)(346.84828161,152.78498668)
\curveto(347.43086545,152.03105465)(348.42468493,151.55127973)(350.13816681,151.55127973)
\curveto(351.44041303,151.55127973)(352.60558071,152.1681332)(353.5994002,154.12150254)
\curveto(354.49041077,155.7321755)(355.14153388,158.50801614)(355.14153388,159.63891418)
\curveto(355.14153388,161.62655315)(353.70220911,161.62655315)(353.66793947,161.62655315)
\curveto(352.84546817,161.62655315)(352.02299687,160.73554258)(352.02299687,159.98161055)
\curveto(352.02299687,159.36475708)(352.46850216,159.09059998)(352.67411998,158.98779106)
\curveto(353.53086092,158.4737465)(353.77074838,158.09678049)(353.77074838,157.61700556)
\curveto(353.77074838,157.27430919)(353.22243418,155.1838613)(352.53704143,153.98442399)
\curveto(351.92018796,152.88779559)(351.23479521,152.33948139)(350.27524536,152.33948139)
\curveto(348.69884203,152.33948139)(348.69884203,153.64172761)(348.69884203,154.05296326)
\curveto(348.69884203,154.6355471)(348.76738131,154.94397384)(349.04153841,156.04060224)
\curveto(349.17861696,156.69172535)(349.45277406,157.75408411)(349.58985261,158.26812868)
\closepath
\moveto(349.89827935,159.46756599)
}
}
{
\newrgbcolor{curcolor}{0 0 0}
\pscustom[linestyle=none,fillstyle=solid,fillcolor=curcolor]
{
\newpath
\moveto(218.15226645,131.07402599)
\curveto(218.22080572,131.41672236)(218.35788427,131.96503656)(218.35788427,132.06784548)
\curveto(218.35788427,132.5476204)(218.0151879,133.02739533)(217.32979515,133.02739533)
\curveto(216.98709877,133.02739533)(216.16462747,132.85604714)(215.89047037,131.86222765)
\curveto(215.51350436,130.49144215)(215.13653835,128.91503883)(214.75957233,127.40717478)
\curveto(214.58822415,126.65324275)(214.58822415,126.37908565)(214.58822415,126.07065891)
\curveto(214.58822415,125.48807508)(214.65676342,125.52234471)(214.65676342,125.38526616)
\curveto(214.65676342,125.28245725)(214.07417958,123.94594139)(212.806203,123.94594139)
\curveto(211.02418185,123.94594139)(211.02418185,125.28245725)(211.02418185,125.72796254)
\curveto(211.02418185,126.65324275)(211.29833895,127.68133188)(212.18934952,129.90885831)
\curveto(212.36069771,130.3543636)(212.56631553,130.86840816)(212.56631553,131.24537418)
\curveto(212.56631553,132.51335076)(211.29833895,133.19874351)(210.09890163,133.19874351)
\curveto(207.76856628,133.19874351)(206.67193788,130.25155469)(206.67193788,129.8060494)
\curveto(206.67193788,129.49762266)(207.01463426,129.49762266)(207.22025208,129.49762266)
\curveto(207.46013955,129.49762266)(207.63148773,129.49762266)(207.70002701,129.77177976)
\curveto(208.4196894,132.20492403)(209.61912671,132.41054185)(209.96182308,132.41054185)
\curveto(210.09890163,132.41054185)(210.3387891,132.41054185)(210.3387891,131.96503656)
\curveto(210.3387891,131.450992)(210.13317127,130.9026778)(209.85901417,130.25155469)
\curveto(209.00227323,128.12683716)(208.65957686,127.03020876)(208.65957686,126.13919819)
\curveto(208.65957686,123.70605393)(210.78429438,123.15773973)(212.66912445,123.15773973)
\curveto(213.0803601,123.15773973)(214.10844922,123.15773973)(215.10226871,124.39144668)
\curveto(215.68485255,123.63751465)(216.67867203,123.15773973)(218.39215391,123.15773973)
\curveto(219.69440013,123.15773973)(220.85956781,123.7745932)(221.8533873,125.72796254)
\curveto(222.74439787,127.3386355)(223.39552098,130.11447614)(223.39552098,131.24537418)
\curveto(223.39552098,133.23301315)(221.95619621,133.23301315)(221.92192657,133.23301315)
\curveto(221.09945527,133.23301315)(220.27698397,132.34200258)(220.27698397,131.58807055)
\curveto(220.27698397,130.97121708)(220.72248926,130.69705998)(220.92810708,130.59425106)
\curveto(221.78484802,130.0802065)(222.02473548,129.70324049)(222.02473548,129.22346556)
\curveto(222.02473548,128.88076919)(221.47642128,126.7903213)(220.79102853,125.59088399)
\curveto(220.17417506,124.49425559)(219.48878231,123.94594139)(218.52923246,123.94594139)
\curveto(216.95282913,123.94594139)(216.95282913,125.24818761)(216.95282913,125.65942326)
\curveto(216.95282913,126.2420071)(217.02136841,126.55043384)(217.29552551,127.64706224)
\curveto(217.43260406,128.29818535)(217.70676116,129.36054411)(217.84383971,129.87458868)
\closepath
\moveto(218.15226645,131.07402599)
}
}
{
\newrgbcolor{curcolor}{0 0 0}
\pscustom[linestyle=none,fillstyle=solid,fillcolor=curcolor]
{
\newpath
\moveto(257.53798597,148.04322418)
\curveto(257.53798597,150.03086315)(256.06439156,150.03086315)(256.06439156,150.03086315)
\curveto(255.17338098,150.03086315)(254.38517932,149.10558294)(254.38517932,148.38592055)
\curveto(254.38517932,147.76906708)(254.86495425,147.49490998)(255.03630243,147.39210106)
\curveto(255.96158265,146.84378686)(256.13293083,146.43255121)(256.13293083,146.02131556)
\curveto(256.13293083,145.54154064)(254.86495425,140.74379139)(252.32900107,140.74379139)
\curveto(250.75259775,140.74379139)(250.75259775,142.04603761)(250.75259775,142.45727326)
\curveto(250.75259775,143.72524985)(251.36945122,145.30165318)(252.05484397,147.04940469)
\curveto(252.22619216,147.49490998)(252.29473143,147.7005278)(252.29473143,148.04322418)
\curveto(252.29473143,149.31120076)(251.02675485,149.99659351)(249.82731753,149.99659351)
\curveto(247.49698218,149.99659351)(246.40035378,147.04940469)(246.40035378,146.6038994)
\curveto(246.40035378,146.29547266)(246.74305016,146.29547266)(246.94866798,146.29547266)
\curveto(247.18855545,146.29547266)(247.35990363,146.29547266)(247.42844291,146.56962976)
\curveto(248.1481053,148.93423475)(249.31327297,149.20839185)(249.69023898,149.20839185)
\curveto(249.86158717,149.20839185)(250.067205,149.20839185)(250.067205,148.76288656)
\curveto(250.067205,148.248842)(249.7930479,147.63198853)(249.72450862,147.46064034)
\curveto(248.73068913,144.92468716)(248.38799276,143.93086768)(248.38799276,142.86850891)
\curveto(248.38799276,140.5724432)(250.27282282,139.95558973)(252.19192252,139.95558973)
\curveto(255.96158265,139.95558973)(257.53798597,146.15839411)(257.53798597,148.04322418)
\closepath
\moveto(257.53798597,148.04322418)
}
}
{
\newrgbcolor{curcolor}{0 0 0}
\pscustom[linestyle=none,fillstyle=solid,fillcolor=curcolor]
{
\newpath
\moveto(274.18123287,144.92468716)
\lineto(280.93235146,144.92468716)
\curveto(281.2407782,144.92468716)(281.89190131,144.92468716)(281.89190131,145.61007991)
\curveto(281.89190131,146.26120303)(281.2407782,146.26120303)(280.93235146,146.26120303)
\lineto(274.18123287,146.26120303)
\lineto(274.18123287,153.04659125)
\curveto(274.18123287,153.32074835)(274.18123287,153.97187146)(273.53010976,153.97187146)
\curveto(272.84471701,153.97187146)(272.84471701,153.32074835)(272.84471701,153.04659125)
\lineto(272.84471701,146.26120303)
\lineto(266.09359842,146.26120303)
\curveto(265.78517169,146.26120303)(265.16831821,146.26120303)(265.16831821,145.61007991)
\curveto(265.16831821,144.92468716)(265.81944132,144.92468716)(266.09359842,144.92468716)
\lineto(272.84471701,144.92468716)
\lineto(272.84471701,138.17356858)
\curveto(272.84471701,137.89941148)(272.84471701,137.24828836)(273.53010976,137.24828836)
\curveto(274.18123287,137.24828836)(274.18123287,137.86514184)(274.18123287,138.17356858)
\closepath
\moveto(274.18123287,144.92468716)
}
}
{
\newrgbcolor{curcolor}{0 0 0}
\pscustom[linestyle=none,fillstyle=solid,fillcolor=curcolor]
{
\newpath
\moveto(300.32677424,147.87187599)
\curveto(300.39531351,148.21457236)(300.53239206,148.76288656)(300.53239206,148.86569548)
\curveto(300.53239206,149.3454704)(300.18969569,149.82524533)(299.50430294,149.82524533)
\curveto(299.16160656,149.82524533)(298.33913526,149.65389714)(298.06497816,148.66007765)
\curveto(297.68801215,147.28929215)(297.31104614,145.71288883)(296.93408012,144.20502478)
\curveto(296.76273194,143.45109275)(296.76273194,143.17693565)(296.76273194,142.86850891)
\curveto(296.76273194,142.28592508)(296.83127121,142.32019471)(296.83127121,142.18311616)
\curveto(296.83127121,142.08030725)(296.24868737,140.74379139)(294.98071079,140.74379139)
\curveto(293.19868964,140.74379139)(293.19868964,142.08030725)(293.19868964,142.52581254)
\curveto(293.19868964,143.45109275)(293.47284674,144.47918188)(294.36385731,146.70670831)
\curveto(294.5352055,147.1522136)(294.74082332,147.66625816)(294.74082332,148.04322418)
\curveto(294.74082332,149.31120076)(293.47284674,149.99659351)(292.27340942,149.99659351)
\curveto(289.94307407,149.99659351)(288.84644567,147.04940469)(288.84644567,146.6038994)
\curveto(288.84644567,146.29547266)(289.18914205,146.29547266)(289.39475987,146.29547266)
\curveto(289.63464734,146.29547266)(289.80599552,146.29547266)(289.8745348,146.56962976)
\curveto(290.59419719,149.00277403)(291.7936345,149.20839185)(292.13633087,149.20839185)
\curveto(292.27340942,149.20839185)(292.51329689,149.20839185)(292.51329689,148.76288656)
\curveto(292.51329689,148.248842)(292.30767906,147.7005278)(292.03352196,147.04940469)
\curveto(291.17678102,144.92468716)(290.83408465,143.82805876)(290.83408465,142.93704819)
\curveto(290.83408465,140.50390393)(292.95880217,139.95558973)(294.84363224,139.95558973)
\curveto(295.25486789,139.95558973)(296.28295701,139.95558973)(297.2767765,141.18929668)
\curveto(297.85936034,140.43536465)(298.85317982,139.95558973)(300.5666617,139.95558973)
\curveto(301.86890792,139.95558973)(303.0340756,140.5724432)(304.02789509,142.52581254)
\curveto(304.91890566,144.1364855)(305.57002877,146.91232614)(305.57002877,148.04322418)
\curveto(305.57002877,150.03086315)(304.130704,150.03086315)(304.09643436,150.03086315)
\curveto(303.27396306,150.03086315)(302.45149176,149.13985258)(302.45149176,148.38592055)
\curveto(302.45149176,147.76906708)(302.89699705,147.49490998)(303.10261487,147.39210106)
\curveto(303.95935581,146.8780565)(304.19924327,146.50109049)(304.19924327,146.02131556)
\curveto(304.19924327,145.67861919)(303.65092907,143.5881713)(302.96553632,142.38873399)
\curveto(302.34868285,141.29210559)(301.6632901,140.74379139)(300.70374025,140.74379139)
\curveto(299.12733692,140.74379139)(299.12733692,142.04603761)(299.12733692,142.45727326)
\curveto(299.12733692,143.0398571)(299.1958762,143.34828384)(299.4700333,144.44491224)
\curveto(299.60711185,145.09603535)(299.88126895,146.15839411)(300.0183475,146.67243868)
\closepath
\moveto(300.32677424,147.87187599)
}
}
\end{pspicture}

    \caption{(左图)向量加法:$\bm v+\bm w$。
        (右图)注意和$\bm v+\bm w$表示由$\bm v$与$\bm w$构成的平行四边形的对角线,
        表明向量加法满足交换律:$\bm v+\bm w=\bm w+\bm v$。}
    \label{fig:2.3}
\end{figure}
\begin{figure}[htbp]
    \centering%LaTeX with PSTricks extensions
%%Creator: Inkscape 1.0.1 (3bc2e813f5, 2020-09-07)
%%Please note this file requires PSTricks extensions
\psset{xunit=.5pt,yunit=.5pt,runit=.5pt}
\begin{pspicture}(323.75,200.27999878)
{
\newrgbcolor{curcolor}{0 0 0}
\pscustom[linewidth=1,linecolor=curcolor]
{
\newpath
\moveto(0.95999998,0.61999512)
\lineto(92.97000122,28.69999695)
}
}
{
\newrgbcolor{curcolor}{0 0 0}
\pscustom[linestyle=none,fillstyle=solid,fillcolor=curcolor]
{
\newpath
\moveto(89.88,22.00999878)
\lineto(92.35,28.51999878)
\lineto(86.67,32.52999878)
\lineto(100.72,31.06999878)
\closepath
}
}
{
\newrgbcolor{curcolor}{0.65098041 0.65098041 0.65098041}
\pscustom[linestyle=none,fillstyle=solid,fillcolor=curcolor]
{
\newpath
\moveto(91.02,23.60999878)
\lineto(99.47,30.68999878)
\lineto(92.95,28.69999878)
\closepath
}
}
{
\newrgbcolor{curcolor}{0.40000001 0.40000001 0.40000001}
\pscustom[linestyle=none,fillstyle=solid,fillcolor=curcolor]
{
\newpath
\moveto(88.51,31.83999878)
\lineto(99.47,30.68999878)
\lineto(92.95,28.69999878)
\closepath
}
}
{
\newrgbcolor{curcolor}{0 0 0}
\pscustom[linewidth=1,linecolor=curcolor,linestyle=dashed,dash=4]
{
\newpath
\moveto(99.80999756,31.53999329)
\lineto(41.06000137,151.68000031)
}
}
{
\newrgbcolor{curcolor}{0 0 0}
\pscustom[linewidth=1,linecolor=curcolor]
{
\newpath
\moveto(0.47999999,0.13999939)
\lineto(39.83000183,143.63999939)
}
}
{
\newrgbcolor{curcolor}{0 0 0}
\pscustom[linestyle=none,fillstyle=solid,fillcolor=curcolor]
{
\newpath
\moveto(43.84,137.44999878)
\lineto(39.65,143.00999878)
\lineto(33.22,140.35999878)
\lineto(41.97,151.44999878)
\closepath
}
}
{
\newrgbcolor{curcolor}{0.65098041 0.65098041 0.65098041}
\pscustom[linestyle=none,fillstyle=solid,fillcolor=curcolor]
{
\newpath
\moveto(43.09,139.26999878)
\lineto(41.62,150.18999878)
\lineto(39.82,143.61999878)
\closepath
}
}
{
\newrgbcolor{curcolor}{0.40000001 0.40000001 0.40000001}
\pscustom[linestyle=none,fillstyle=solid,fillcolor=curcolor]
{
\newpath
\moveto(34.8,141.53999878)
\lineto(41.62,150.18999878)
\lineto(39.82,143.61999878)
\closepath
}
}
{
\newrgbcolor{curcolor}{0 0 0}
\pscustom[linewidth=1,linecolor=curcolor]
{
\newpath
\moveto(165.36999512,0.61999512)
\lineto(257.38000488,28.69999695)
}
}
{
\newrgbcolor{curcolor}{0 0 0}
\pscustom[linestyle=none,fillstyle=solid,fillcolor=curcolor]
{
\newpath
\moveto(254.29,22.00999878)
\lineto(256.75,28.51999878)
\lineto(251.07,32.52999878)
\lineto(265.13,31.06999878)
\closepath
}
}
{
\newrgbcolor{curcolor}{0.65098041 0.65098041 0.65098041}
\pscustom[linestyle=none,fillstyle=solid,fillcolor=curcolor]
{
\newpath
\moveto(255.43,23.60999878)
\lineto(263.87,30.68999878)
\lineto(257.36,28.69999878)
\closepath
}
}
{
\newrgbcolor{curcolor}{0.40000001 0.40000001 0.40000001}
\pscustom[linestyle=none,fillstyle=solid,fillcolor=curcolor]
{
\newpath
\moveto(252.92,31.83999878)
\lineto(263.87,30.68999878)
\lineto(257.36,28.69999878)
\closepath
}
}
{
\newrgbcolor{curcolor}{0 0 0}
\pscustom[linewidth=1,linecolor=curcolor,linestyle=dashed,dash=4]
{
\newpath
\moveto(264.22000122,31.53999329)
\lineto(205.46000671,151.68000031)
}
}
{
\newrgbcolor{curcolor}{0 0 0}
\pscustom[linewidth=1,linecolor=curcolor]
{
\newpath
\moveto(164.88999939,0.13999939)
\lineto(204.22999573,143.63999939)
}
}
{
\newrgbcolor{curcolor}{0 0 0}
\pscustom[linestyle=none,fillstyle=solid,fillcolor=curcolor]
{
\newpath
\moveto(208.24,137.44999878)
\lineto(204.06,143.00999878)
\lineto(197.63,140.35999878)
\lineto(206.37,151.44999878)
\closepath
}
}
{
\newrgbcolor{curcolor}{0.65098041 0.65098041 0.65098041}
\pscustom[linestyle=none,fillstyle=solid,fillcolor=curcolor]
{
\newpath
\moveto(207.49,139.26999878)
\lineto(206.03,150.18999878)
\lineto(204.23,143.61999878)
\closepath
}
}
{
\newrgbcolor{curcolor}{0.40000001 0.40000001 0.40000001}
\pscustom[linestyle=none,fillstyle=solid,fillcolor=curcolor]
{
\newpath
\moveto(199.2,141.53999878)
\lineto(206.03,150.18999878)
\lineto(204.23,143.61999878)
\closepath
}
}
{
\newrgbcolor{curcolor}{0 0 0}
\pscustom[linewidth=1,linecolor=curcolor]
{
\newpath
\moveto(206.49000549,150.94999695)
\lineto(298.5,179.03999901)
}
}
{
\newrgbcolor{curcolor}{0 0 0}
\pscustom[linestyle=none,fillstyle=solid,fillcolor=curcolor]
{
\newpath
\moveto(295.41,172.33999878)
\lineto(297.88,178.84999878)
\lineto(292.19,182.86999878)
\lineto(306.25,181.39999878)
\closepath
}
}
{
\newrgbcolor{curcolor}{0.65098041 0.65098041 0.65098041}
\pscustom[linestyle=none,fillstyle=solid,fillcolor=curcolor]
{
\newpath
\moveto(296.55,173.93999878)
\lineto(304.99,181.01999878)
\lineto(298.48,179.02999878)
\closepath
}
}
{
\newrgbcolor{curcolor}{0.40000001 0.40000001 0.40000001}
\pscustom[linestyle=none,fillstyle=solid,fillcolor=curcolor]
{
\newpath
\moveto(294.04,182.16999878)
\lineto(304.99,181.01999878)
\lineto(298.48,179.02999878)
\closepath
}
}
{
\newrgbcolor{curcolor}{0 0 0}
\pscustom[linewidth=1,linecolor=curcolor]
{
\newpath
\moveto(264.3999939,30.5)
\lineto(303.73999023,174.00999832)
}
}
{
\newrgbcolor{curcolor}{0 0 0}
\pscustom[linestyle=none,fillstyle=solid,fillcolor=curcolor]
{
\newpath
\moveto(307.75,167.81999878)
\lineto(303.57,173.37999878)
\lineto(297.13,170.72999878)
\lineto(305.88,181.81999878)
\closepath
}
}
{
\newrgbcolor{curcolor}{0.65098041 0.65098041 0.65098041}
\pscustom[linestyle=none,fillstyle=solid,fillcolor=curcolor]
{
\newpath
\moveto(307,169.63999878)
\lineto(305.54,180.55999878)
\lineto(303.74,173.98999878)
\closepath
}
}
{
\newrgbcolor{curcolor}{0.40000001 0.40000001 0.40000001}
\pscustom[linestyle=none,fillstyle=solid,fillcolor=curcolor]
{
\newpath
\moveto(298.71,171.90999878)
\lineto(305.54,180.55999878)
\lineto(303.74,173.98999878)
\closepath
}
}
{
\newrgbcolor{curcolor}{0 0 0}
\pscustom[linestyle=none,fillstyle=solid,fillcolor=curcolor]
{
\newpath
\moveto(85.74187628,97.84176128)
\curveto(85.81137962,98.189278)(85.95038631,98.74530475)(85.95038631,98.84955977)
\curveto(85.95038631,99.33608317)(85.60286959,99.82260658)(84.90783615,99.82260658)
\curveto(84.56031943,99.82260658)(83.72627931,99.64884822)(83.44826593,98.64104974)
\curveto(83.06599754,97.25098286)(82.68372915,95.65240596)(82.30146076,94.12333239)
\curveto(82.1277024,93.35879561)(82.1277024,93.08078224)(82.1277024,92.76801719)
\curveto(82.1277024,92.17723877)(82.19720575,92.21199044)(82.19720575,92.07298375)
\curveto(82.19720575,91.96872874)(81.60642732,90.61341353)(80.32061547,90.61341353)
\curveto(78.51352853,90.61341353)(78.51352853,91.96872874)(78.51352853,92.42050047)
\curveto(78.51352853,93.35879561)(78.7915419,94.40134577)(79.69508537,96.66020444)
\curveto(79.86884373,97.11197617)(80.07735376,97.63325125)(80.07735376,98.01551964)
\curveto(80.07735376,99.3013315)(78.7915419,99.99636494)(77.57523339,99.99636494)
\curveto(75.2121197,99.99636494)(74.1000662,97.00772116)(74.1000662,96.55594942)
\curveto(74.1000662,96.24318438)(74.44758292,96.24318438)(74.65609295,96.24318438)
\curveto(74.89935465,96.24318438)(75.07311301,96.24318438)(75.14261636,96.52119775)
\curveto(75.87240147,98.98856646)(77.08870998,99.19707649)(77.4362267,99.19707649)
\curveto(77.57523339,99.19707649)(77.81849509,99.19707649)(77.81849509,98.74530475)
\curveto(77.81849509,98.22402967)(77.60998506,97.66800292)(77.33197168,97.00772116)
\curveto(76.46317989,94.8531175)(76.11566317,93.741064)(76.11566317,92.83752053)
\curveto(76.11566317,90.37015183)(78.27026682,89.81412508)(80.18160878,89.81412508)
\curveto(80.59862884,89.81412508)(81.641179,89.81412508)(82.64897748,91.06518527)
\curveto(83.2397559,90.30064849)(84.24755439,89.81412508)(85.98513798,89.81412508)
\curveto(87.30570151,89.81412508)(88.48725836,90.43965517)(89.49505684,92.42050047)
\curveto(90.39860031,94.05382905)(91.05888207,96.86871447)(91.05888207,98.01551964)
\curveto(91.05888207,100.03111661)(89.59931186,100.03111661)(89.56456018,100.03111661)
\curveto(88.73052006,100.03111661)(87.89647993,99.12757314)(87.89647993,98.36303636)
\curveto(87.89647993,97.73750627)(88.34825167,97.45949289)(88.5567617,97.35523788)
\curveto(89.4255535,96.8339628)(89.6688152,96.45169441)(89.6688152,95.965171)
\curveto(89.6688152,95.61765428)(89.11278845,93.4978023)(88.41775501,92.28149378)
\curveto(87.79222492,91.16944028)(87.09719148,90.61341353)(86.12414467,90.61341353)
\curveto(84.52556776,90.61341353)(84.52556776,91.93397706)(84.52556776,92.35099713)
\curveto(84.52556776,92.94177555)(84.59507111,93.2545406)(84.87308448,94.3665941)
\curveto(85.01209117,95.02687586)(85.29010454,96.10417769)(85.42911123,96.62545277)
\closepath
\moveto(85.74187628,97.84176128)
}
}
{
\newrgbcolor{curcolor}{0 0 0}
\pscustom[linestyle=none,fillstyle=solid,fillcolor=curcolor]
{
\newpath
\moveto(114.05057461,94.8531175)
\curveto(114.36333966,94.8531175)(115.02362143,94.8531175)(115.02362143,95.54815094)
\curveto(115.02362143,96.20843271)(114.39809133,96.20843271)(114.05057461,96.20843271)
\lineto(100.49742258,96.20843271)
\curveto(100.14990586,96.20843271)(99.4896241,96.20843271)(99.4896241,95.54815094)
\curveto(99.4896241,94.8531175)(100.14990586,94.8531175)(100.49742258,94.8531175)
\closepath
\moveto(114.05057461,94.8531175)
}
}
{
\newrgbcolor{curcolor}{0 0 0}
\pscustom[linestyle=none,fillstyle=solid,fillcolor=curcolor]
{
\newpath
\moveto(134.1000664,98.01551964)
\curveto(134.1000664,100.03111661)(132.60574451,100.03111661)(132.60574451,100.03111661)
\curveto(131.70220104,100.03111661)(130.90291259,99.09282147)(130.90291259,98.36303636)
\curveto(130.90291259,97.73750627)(131.38943599,97.45949289)(131.56319435,97.35523788)
\curveto(132.50148949,96.79921113)(132.67524785,96.38219106)(132.67524785,95.965171)
\curveto(132.67524785,95.4786476)(131.38943599,90.61341353)(128.81781227,90.61341353)
\curveto(127.21923537,90.61341353)(127.21923537,91.93397706)(127.21923537,92.35099713)
\curveto(127.21923537,93.63680899)(127.84476546,95.23538589)(128.5397989,97.00772116)
\curveto(128.71355726,97.45949289)(128.7830606,97.66800292)(128.7830606,98.01551964)
\curveto(128.7830606,99.3013315)(127.49724874,99.99636494)(126.28094023,99.99636494)
\curveto(123.91782654,99.99636494)(122.80577304,97.00772116)(122.80577304,96.55594942)
\curveto(122.80577304,96.24318438)(123.15328976,96.24318438)(123.36179979,96.24318438)
\curveto(123.60506149,96.24318438)(123.77881985,96.24318438)(123.8483232,96.52119775)
\curveto(124.57810831,98.91906311)(125.75966515,99.19707649)(126.14193354,99.19707649)
\curveto(126.3156919,99.19707649)(126.52420193,99.19707649)(126.52420193,98.74530475)
\curveto(126.52420193,98.22402967)(126.24618856,97.59849958)(126.17668521,97.42474122)
\curveto(125.16888673,94.8531175)(124.82137001,93.84531902)(124.82137001,92.76801719)
\curveto(124.82137001,90.43965517)(126.73271196,89.81412508)(128.67880559,89.81412508)
\curveto(132.50148949,89.81412508)(134.1000664,96.10417769)(134.1000664,98.01551964)
\closepath
\moveto(134.1000664,98.01551964)
}
}
{
\newrgbcolor{curcolor}{0 0 0}
\pscustom[linestyle=none,fillstyle=solid,fillcolor=curcolor]
{
\newpath
\moveto(25.63875628,134.93762028)
\curveto(25.70825962,135.285137)(25.84726631,135.84116375)(25.84726631,135.94541877)
\curveto(25.84726631,136.43194217)(25.49974959,136.91846558)(24.80471615,136.91846558)
\curveto(24.45719943,136.91846558)(23.62315931,136.74470722)(23.34514593,135.73690874)
\curveto(22.96287754,134.34684186)(22.58060915,132.74826496)(22.19834076,131.21919139)
\curveto(22.0245824,130.45465461)(22.0245824,130.17664124)(22.0245824,129.86387619)
\curveto(22.0245824,129.27309777)(22.09408575,129.30784944)(22.09408575,129.16884275)
\curveto(22.09408575,129.06458774)(21.50330732,127.70927253)(20.21749547,127.70927253)
\curveto(18.41040853,127.70927253)(18.41040853,129.06458774)(18.41040853,129.51635947)
\curveto(18.41040853,130.45465461)(18.6884219,131.49720477)(19.59196537,133.75606344)
\curveto(19.76572373,134.20783517)(19.97423376,134.72911025)(19.97423376,135.11137864)
\curveto(19.97423376,136.3971905)(18.6884219,137.09222394)(17.47211339,137.09222394)
\curveto(15.1089997,137.09222394)(13.9969462,134.10358016)(13.9969462,133.65180842)
\curveto(13.9969462,133.33904338)(14.34446292,133.33904338)(14.55297295,133.33904338)
\curveto(14.79623465,133.33904338)(14.96999301,133.33904338)(15.03949636,133.61705675)
\curveto(15.76928147,136.08442546)(16.98558998,136.29293549)(17.3331067,136.29293549)
\curveto(17.47211339,136.29293549)(17.71537509,136.29293549)(17.71537509,135.84116375)
\curveto(17.71537509,135.31988867)(17.50686506,134.76386192)(17.22885168,134.10358016)
\curveto(16.36005989,131.9489765)(16.01254317,130.836923)(16.01254317,129.93337953)
\curveto(16.01254317,127.46601083)(18.16714682,126.90998408)(20.07848878,126.90998408)
\curveto(20.49550884,126.90998408)(21.538059,126.90998408)(22.54585748,128.16104427)
\curveto(23.1366359,127.39650749)(24.14443439,126.90998408)(25.88201798,126.90998408)
\curveto(27.20258151,126.90998408)(28.38413836,127.53551417)(29.39193684,129.51635947)
\curveto(30.29548031,131.14968805)(30.95576207,133.96457347)(30.95576207,135.11137864)
\curveto(30.95576207,137.12697561)(29.49619186,137.12697561)(29.46144018,137.12697561)
\curveto(28.62740006,137.12697561)(27.79335993,136.22343214)(27.79335993,135.45889536)
\curveto(27.79335993,134.83336527)(28.24513167,134.55535189)(28.4536417,134.45109688)
\curveto(29.3224335,133.9298218)(29.5656952,133.54755341)(29.5656952,133.06103)
\curveto(29.5656952,132.71351328)(29.00966845,130.5936613)(28.31463501,129.37735278)
\curveto(27.68910492,128.26529928)(26.99407148,127.70927253)(26.02102467,127.70927253)
\curveto(24.42244776,127.70927253)(24.42244776,129.02983606)(24.42244776,129.44685613)
\curveto(24.42244776,130.03763455)(24.49195111,130.3503996)(24.76996448,131.4624531)
\curveto(24.90897117,132.12273486)(25.18698454,133.20003669)(25.32599123,133.72131177)
\closepath
\moveto(25.63875628,134.93762028)
}
}
{
\newrgbcolor{curcolor}{0 0 0}
\pscustom[linestyle=none,fillstyle=solid,fillcolor=curcolor]
{
\newpath
\moveto(186.71928728,136.18631028)
\curveto(186.78879062,136.533827)(186.92779731,137.08985375)(186.92779731,137.19410877)
\curveto(186.92779731,137.68063217)(186.58028059,138.16715558)(185.88524715,138.16715558)
\curveto(185.53773043,138.16715558)(184.70369031,137.99339722)(184.42567693,136.98559874)
\curveto(184.04340854,135.59553186)(183.66114015,133.99695496)(183.27887176,132.46788139)
\curveto(183.1051134,131.70334461)(183.1051134,131.42533124)(183.1051134,131.11256619)
\curveto(183.1051134,130.52178777)(183.17461675,130.55653944)(183.17461675,130.41753275)
\curveto(183.17461675,130.31327774)(182.58383832,128.95796253)(181.29802647,128.95796253)
\curveto(179.49093953,128.95796253)(179.49093953,130.31327774)(179.49093953,130.76504947)
\curveto(179.49093953,131.70334461)(179.7689529,132.74589477)(180.67249637,135.00475344)
\curveto(180.84625473,135.45652517)(181.05476476,135.97780025)(181.05476476,136.36006864)
\curveto(181.05476476,137.6458805)(179.7689529,138.34091394)(178.55264439,138.34091394)
\curveto(176.1895307,138.34091394)(175.0774772,135.35227016)(175.0774772,134.90049842)
\curveto(175.0774772,134.58773338)(175.42499392,134.58773338)(175.63350395,134.58773338)
\curveto(175.87676565,134.58773338)(176.05052401,134.58773338)(176.12002736,134.86574675)
\curveto(176.84981247,137.33311546)(178.06612098,137.54162549)(178.4136377,137.54162549)
\curveto(178.55264439,137.54162549)(178.79590609,137.54162549)(178.79590609,137.08985375)
\curveto(178.79590609,136.56857867)(178.58739606,136.01255192)(178.30938268,135.35227016)
\curveto(177.44059089,133.1976665)(177.09307417,132.085613)(177.09307417,131.18206953)
\curveto(177.09307417,128.71470083)(179.24767782,128.15867408)(181.15901978,128.15867408)
\curveto(181.57603984,128.15867408)(182.61859,128.15867408)(183.62638848,129.40973427)
\curveto(184.2171669,128.64519749)(185.22496539,128.15867408)(186.96254898,128.15867408)
\curveto(188.28311251,128.15867408)(189.46466936,128.78420417)(190.47246784,130.76504947)
\curveto(191.37601131,132.39837805)(192.03629307,135.21326347)(192.03629307,136.36006864)
\curveto(192.03629307,138.37566561)(190.57672286,138.37566561)(190.54197118,138.37566561)
\curveto(189.70793106,138.37566561)(188.87389093,137.47212214)(188.87389093,136.70758536)
\curveto(188.87389093,136.08205527)(189.32566267,135.80404189)(189.5341727,135.69978688)
\curveto(190.4029645,135.1785118)(190.6462262,134.79624341)(190.6462262,134.30972)
\curveto(190.6462262,133.96220328)(190.09019945,131.8423513)(189.39516601,130.62604278)
\curveto(188.76963592,129.51398928)(188.07460248,128.95796253)(187.10155567,128.95796253)
\curveto(185.50297876,128.95796253)(185.50297876,130.27852606)(185.50297876,130.69554613)
\curveto(185.50297876,131.28632455)(185.57248211,131.5990896)(185.85049548,132.7111431)
\curveto(185.98950217,133.37142486)(186.26751554,134.44872669)(186.40652223,134.97000177)
\closepath
\moveto(186.71928728,136.18631028)
}
}
{
\newrgbcolor{curcolor}{0 0 0}
\pscustom[linestyle=none,fillstyle=solid,fillcolor=curcolor]
{
\newpath
\moveto(314.91774728,161.99249028)
\curveto(314.98725062,162.340007)(315.12625731,162.89603375)(315.12625731,163.00028877)
\curveto(315.12625731,163.48681217)(314.77874059,163.97333558)(314.08370715,163.97333558)
\curveto(313.73619043,163.97333558)(312.90215031,163.79957722)(312.62413693,162.79177874)
\curveto(312.24186854,161.40171186)(311.85960015,159.80313496)(311.47733176,158.27406139)
\curveto(311.3035734,157.50952461)(311.3035734,157.23151124)(311.3035734,156.91874619)
\curveto(311.3035734,156.32796777)(311.37307675,156.36271944)(311.37307675,156.22371275)
\curveto(311.37307675,156.11945774)(310.78229832,154.76414253)(309.49648647,154.76414253)
\curveto(307.68939953,154.76414253)(307.68939953,156.11945774)(307.68939953,156.57122947)
\curveto(307.68939953,157.50952461)(307.9674129,158.55207477)(308.87095637,160.81093344)
\curveto(309.04471473,161.26270517)(309.25322476,161.78398025)(309.25322476,162.16624864)
\curveto(309.25322476,163.4520605)(307.9674129,164.14709394)(306.75110439,164.14709394)
\curveto(304.3879907,164.14709394)(303.2759372,161.15845016)(303.2759372,160.70667842)
\curveto(303.2759372,160.39391338)(303.62345392,160.39391338)(303.83196395,160.39391338)
\curveto(304.07522565,160.39391338)(304.24898401,160.39391338)(304.31848736,160.67192675)
\curveto(305.04827247,163.13929546)(306.26458098,163.34780549)(306.6120977,163.34780549)
\curveto(306.75110439,163.34780549)(306.99436609,163.34780549)(306.99436609,162.89603375)
\curveto(306.99436609,162.37475867)(306.78585606,161.81873192)(306.50784268,161.15845016)
\curveto(305.63905089,159.0038465)(305.29153417,157.891793)(305.29153417,156.98824953)
\curveto(305.29153417,154.52088083)(307.44613782,153.96485408)(309.35747978,153.96485408)
\curveto(309.77449984,153.96485408)(310.81705,153.96485408)(311.82484848,155.21591427)
\curveto(312.4156269,154.45137749)(313.42342539,153.96485408)(315.16100898,153.96485408)
\curveto(316.48157251,153.96485408)(317.66312936,154.59038417)(318.67092784,156.57122947)
\curveto(319.57447131,158.20455805)(320.23475307,161.01944347)(320.23475307,162.16624864)
\curveto(320.23475307,164.18184561)(318.77518286,164.18184561)(318.74043118,164.18184561)
\curveto(317.90639106,164.18184561)(317.07235093,163.27830214)(317.07235093,162.51376536)
\curveto(317.07235093,161.88823527)(317.52412267,161.61022189)(317.7326327,161.50596688)
\curveto(318.6014245,160.9846918)(318.8446862,160.60242341)(318.8446862,160.1159)
\curveto(318.8446862,159.76838328)(318.28865945,157.6485313)(317.59362601,156.43222278)
\curveto(316.96809592,155.32016928)(316.27306248,154.76414253)(315.30001567,154.76414253)
\curveto(313.70143876,154.76414253)(313.70143876,156.08470606)(313.70143876,156.50172613)
\curveto(313.70143876,157.09250455)(313.77094211,157.4052696)(314.04895548,158.5173231)
\curveto(314.18796217,159.17760486)(314.46597554,160.25490669)(314.60498223,160.77618177)
\closepath
\moveto(314.91774728,161.99249028)
}
}
{
\newrgbcolor{curcolor}{0 0 0}
\pscustom[linestyle=none,fillstyle=solid,fillcolor=curcolor]
{
\newpath
\moveto(88.89289956,17.31863994)
\curveto(88.89289956,19.33423691)(87.39857767,19.33423691)(87.39857767,19.33423691)
\curveto(86.4950342,19.33423691)(85.69574575,18.39594177)(85.69574575,17.66615666)
\curveto(85.69574575,17.04062657)(86.18226915,16.76261319)(86.35602751,16.65835818)
\curveto(87.29432265,16.10233143)(87.46808101,15.68531136)(87.46808101,15.2682913)
\curveto(87.46808101,14.7817679)(86.18226915,9.91653383)(83.61064543,9.91653383)
\curveto(82.01206853,9.91653383)(82.01206853,11.23709736)(82.01206853,11.65411743)
\curveto(82.01206853,12.93992929)(82.63759862,14.53850619)(83.33263206,16.31084146)
\curveto(83.50639042,16.76261319)(83.57589376,16.97112322)(83.57589376,17.31863994)
\curveto(83.57589376,18.6044518)(82.2900819,19.29948524)(81.07377339,19.29948524)
\curveto(78.7106597,19.29948524)(77.5986062,16.31084146)(77.5986062,15.85906972)
\curveto(77.5986062,15.54630468)(77.94612292,15.54630468)(78.15463295,15.54630468)
\curveto(78.39789465,15.54630468)(78.57165301,15.54630468)(78.64115636,15.82431805)
\curveto(79.37094147,18.22218341)(80.55249831,18.50019679)(80.9347667,18.50019679)
\curveto(81.10852506,18.50019679)(81.31703509,18.50019679)(81.31703509,18.04842505)
\curveto(81.31703509,17.52714997)(81.03902172,16.90161988)(80.96951837,16.72786152)
\curveto(79.96171989,14.1562378)(79.61420317,13.14843932)(79.61420317,12.07113749)
\curveto(79.61420317,9.74277547)(81.52554512,9.11724538)(83.47163875,9.11724538)
\curveto(87.29432265,9.11724538)(88.89289956,15.40729799)(88.89289956,17.31863994)
\closepath
\moveto(88.89289956,17.31863994)
}
}
{
\newrgbcolor{curcolor}{0 0 0}
\pscustom[linestyle=none,fillstyle=solid,fillcolor=curcolor]
{
\newpath
\moveto(251.63834656,18.98355504)
\curveto(251.63834656,20.99915201)(250.14402467,20.99915201)(250.14402467,20.99915201)
\curveto(249.2404812,20.99915201)(248.44119275,20.06085687)(248.44119275,19.33107176)
\curveto(248.44119275,18.70554167)(248.92771615,18.42752829)(249.10147451,18.32327328)
\curveto(250.03976965,17.76724653)(250.21352801,17.35022646)(250.21352801,16.9332064)
\curveto(250.21352801,16.446683)(248.92771615,11.58144893)(246.35609243,11.58144893)
\curveto(244.75751553,11.58144893)(244.75751553,12.90201246)(244.75751553,13.31903253)
\curveto(244.75751553,14.60484439)(245.38304562,16.20342129)(246.07807906,17.97575656)
\curveto(246.25183742,18.42752829)(246.32134076,18.63603832)(246.32134076,18.98355504)
\curveto(246.32134076,20.2693669)(245.0355289,20.96440034)(243.81922039,20.96440034)
\curveto(241.4561067,20.96440034)(240.3440532,17.97575656)(240.3440532,17.52398482)
\curveto(240.3440532,17.21121978)(240.69156992,17.21121978)(240.90007995,17.21121978)
\curveto(241.14334165,17.21121978)(241.31710001,17.21121978)(241.38660336,17.48923315)
\curveto(242.11638847,19.88709851)(243.29794531,20.16511189)(243.6802137,20.16511189)
\curveto(243.85397206,20.16511189)(244.06248209,20.16511189)(244.06248209,19.71334015)
\curveto(244.06248209,19.19206507)(243.78446872,18.56653498)(243.71496537,18.39277662)
\curveto(242.70716689,15.8211529)(242.35965017,14.81335442)(242.35965017,13.73605259)
\curveto(242.35965017,11.40769057)(244.27099212,10.78216048)(246.21708575,10.78216048)
\curveto(250.03976965,10.78216048)(251.63834656,17.07221309)(251.63834656,18.98355504)
\closepath
\moveto(251.63834656,18.98355504)
}
}
{
\newrgbcolor{curcolor}{0 0 0}
\pscustom[linestyle=none,fillstyle=solid,fillcolor=curcolor]
{
\newpath
\moveto(291.18007856,190.88602864)
\curveto(291.18007856,192.90162561)(289.68575667,192.90162561)(289.68575667,192.90162561)
\curveto(288.7822132,192.90162561)(287.98292475,191.96333047)(287.98292475,191.23354536)
\curveto(287.98292475,190.60801527)(288.46944815,190.33000189)(288.64320651,190.22574688)
\curveto(289.58150165,189.66972013)(289.75526001,189.25270006)(289.75526001,188.83568)
\curveto(289.75526001,188.3491566)(288.46944815,183.48392253)(285.89782443,183.48392253)
\curveto(284.29924753,183.48392253)(284.29924753,184.80448606)(284.29924753,185.22150613)
\curveto(284.29924753,186.50731799)(284.92477762,188.10589489)(285.61981106,189.87823016)
\curveto(285.79356942,190.33000189)(285.86307276,190.53851192)(285.86307276,190.88602864)
\curveto(285.86307276,192.1718405)(284.5772609,192.86687394)(283.36095239,192.86687394)
\curveto(280.9978387,192.86687394)(279.8857852,189.87823016)(279.8857852,189.42645842)
\curveto(279.8857852,189.11369338)(280.23330192,189.11369338)(280.44181195,189.11369338)
\curveto(280.68507365,189.11369338)(280.85883201,189.11369338)(280.92833536,189.39170675)
\curveto(281.65812047,191.78957211)(282.83967731,192.06758549)(283.2219457,192.06758549)
\curveto(283.39570406,192.06758549)(283.60421409,192.06758549)(283.60421409,191.61581375)
\curveto(283.60421409,191.09453867)(283.32620072,190.46900858)(283.25669737,190.29525022)
\curveto(282.24889889,187.7236265)(281.90138217,186.71582802)(281.90138217,185.63852619)
\curveto(281.90138217,183.31016417)(283.81272412,182.68463408)(285.75881775,182.68463408)
\curveto(289.58150165,182.68463408)(291.18007856,188.97468669)(291.18007856,190.88602864)
\closepath
\moveto(291.18007856,190.88602864)
}
}
{
\newrgbcolor{curcolor}{0 0 0}
\pscustom[linestyle=none,fillstyle=solid,fillcolor=curcolor]
{
\newpath
\moveto(234.29267028,124.53190328)
\curveto(234.36217362,124.87942)(234.50118031,125.43544675)(234.50118031,125.53970177)
\curveto(234.50118031,126.02622517)(234.15366359,126.51274858)(233.45863015,126.51274858)
\curveto(233.11111343,126.51274858)(232.27707331,126.33899022)(231.99905993,125.33119174)
\curveto(231.61679154,123.94112486)(231.23452315,122.34254796)(230.85225476,120.81347439)
\curveto(230.6784964,120.04893761)(230.6784964,119.77092424)(230.6784964,119.45815919)
\curveto(230.6784964,118.86738077)(230.74799975,118.90213244)(230.74799975,118.76312575)
\curveto(230.74799975,118.65887074)(230.15722132,117.30355553)(228.87140947,117.30355553)
\curveto(227.06432253,117.30355553)(227.06432253,118.65887074)(227.06432253,119.11064247)
\curveto(227.06432253,120.04893761)(227.3423359,121.09148777)(228.24587937,123.35034644)
\curveto(228.41963773,123.80211817)(228.62814776,124.32339325)(228.62814776,124.70566164)
\curveto(228.62814776,125.9914735)(227.3423359,126.68650694)(226.12602739,126.68650694)
\curveto(223.7629137,126.68650694)(222.6508602,123.69786316)(222.6508602,123.24609142)
\curveto(222.6508602,122.93332638)(222.99837692,122.93332638)(223.20688695,122.93332638)
\curveto(223.45014865,122.93332638)(223.62390701,122.93332638)(223.69341036,123.21133975)
\curveto(224.42319547,125.67870846)(225.63950398,125.88721849)(225.9870207,125.88721849)
\curveto(226.12602739,125.88721849)(226.36928909,125.88721849)(226.36928909,125.43544675)
\curveto(226.36928909,124.91417167)(226.16077906,124.35814492)(225.88276568,123.69786316)
\curveto(225.01397389,121.5432595)(224.66645717,120.431206)(224.66645717,119.52766253)
\curveto(224.66645717,117.06029383)(226.82106082,116.50426708)(228.73240278,116.50426708)
\curveto(229.14942284,116.50426708)(230.191973,116.50426708)(231.19977148,117.75532727)
\curveto(231.7905499,116.99079049)(232.79834839,116.50426708)(234.53593198,116.50426708)
\curveto(235.85649551,116.50426708)(237.03805236,117.12979717)(238.04585084,119.11064247)
\curveto(238.94939431,120.74397105)(239.60967607,123.55885647)(239.60967607,124.70566164)
\curveto(239.60967607,126.72125861)(238.15010586,126.72125861)(238.11535418,126.72125861)
\curveto(237.28131406,126.72125861)(236.44727393,125.81771514)(236.44727393,125.05317836)
\curveto(236.44727393,124.42764827)(236.89904567,124.14963489)(237.1075557,124.04537988)
\curveto(237.9763475,123.5241048)(238.2196092,123.14183641)(238.2196092,122.655313)
\curveto(238.2196092,122.30779628)(237.66358245,120.1879443)(236.96854901,118.97163578)
\curveto(236.34301892,117.85958228)(235.64798548,117.30355553)(234.67493867,117.30355553)
\curveto(233.07636176,117.30355553)(233.07636176,118.62411906)(233.07636176,119.04113913)
\curveto(233.07636176,119.63191755)(233.14586511,119.9446826)(233.42387848,121.0567361)
\curveto(233.56288517,121.71701786)(233.84089854,122.79431969)(233.97990523,123.31559477)
\closepath
\moveto(234.29267028,124.53190328)
}
}
{
\newrgbcolor{curcolor}{0 0 0}
\pscustom[linestyle=none,fillstyle=solid,fillcolor=curcolor]
{
\newpath
\moveto(262.60136861,121.5432595)
\curveto(262.91413366,121.5432595)(263.57441543,121.5432595)(263.57441543,122.23829294)
\curveto(263.57441543,122.89857471)(262.94888533,122.89857471)(262.60136861,122.89857471)
\lineto(249.04821658,122.89857471)
\curveto(248.70069986,122.89857471)(248.0404181,122.89857471)(248.0404181,122.23829294)
\curveto(248.0404181,121.5432595)(248.70069986,121.5432595)(249.04821658,121.5432595)
\closepath
\moveto(262.60136861,121.5432595)
}
}
{
\newrgbcolor{curcolor}{0 0 0}
\pscustom[linestyle=none,fillstyle=solid,fillcolor=curcolor]
{
\newpath
\moveto(282.6508604,124.70566164)
\curveto(282.6508604,126.72125861)(281.15653851,126.72125861)(281.15653851,126.72125861)
\curveto(280.25299504,126.72125861)(279.45370659,125.78296347)(279.45370659,125.05317836)
\curveto(279.45370659,124.42764827)(279.94022999,124.14963489)(280.11398835,124.04537988)
\curveto(281.05228349,123.48935313)(281.22604185,123.07233306)(281.22604185,122.655313)
\curveto(281.22604185,122.1687896)(279.94022999,117.30355553)(277.36860627,117.30355553)
\curveto(275.77002937,117.30355553)(275.77002937,118.62411906)(275.77002937,119.04113913)
\curveto(275.77002937,120.32695099)(276.39555946,121.92552789)(277.0905929,123.69786316)
\curveto(277.26435126,124.14963489)(277.3338546,124.35814492)(277.3338546,124.70566164)
\curveto(277.3338546,125.9914735)(276.04804274,126.68650694)(274.83173423,126.68650694)
\curveto(272.46862054,126.68650694)(271.35656704,123.69786316)(271.35656704,123.24609142)
\curveto(271.35656704,122.93332638)(271.70408376,122.93332638)(271.91259379,122.93332638)
\curveto(272.15585549,122.93332638)(272.32961385,122.93332638)(272.3991172,123.21133975)
\curveto(273.12890231,125.60920511)(274.31045915,125.88721849)(274.69272754,125.88721849)
\curveto(274.8664859,125.88721849)(275.07499593,125.88721849)(275.07499593,125.43544675)
\curveto(275.07499593,124.91417167)(274.79698256,124.28864158)(274.72747921,124.11488322)
\curveto(273.71968073,121.5432595)(273.37216401,120.53546102)(273.37216401,119.45815919)
\curveto(273.37216401,117.12979717)(275.28350596,116.50426708)(277.22959959,116.50426708)
\curveto(281.05228349,116.50426708)(282.6508604,122.79431969)(282.6508604,124.70566164)
\closepath
\moveto(282.6508604,124.70566164)
}
}
\end{pspicture}

    \caption{(左图)向量减法。
        (右图)如果我们考虑由两个向量构成的平行四边形,
        则$\bm w-\bm v$(虚线)与$-\bm v-\bm w$(未画出)表示对角线。}
    \label{fig:2.4}
\end{figure}

\begin{lstlisting}
`\refcode{Vector3 Public Methods}{+=}\lastnext{Vector3PublicMethods}`
`\refvar{Vector3}{}`<T> operator+(const `\refvar{Vector3}{}`<T> &v) const {
    return `\refvar{Vector3}{}`(x + v.x, y + v.y, z + v.z);
}
`\refvar{Vector3}{}`<T>& operator+=(const `\refvar{Vector3}{}`<T> &v) {
    x += v.x; y += v.y; z += v.z;
    return *this;
}
\end{lstlisting}

两个向量的减法类似,此处不再说明。

向量可以逐分量与标量相乘,从而改变其长度。
为了涵盖源码中可能用到的这一运算的所有不同形式
(即{\ttfamily v*s}、{\ttfamily s*v}以及{\ttfamily v*=s}),需要三个函数:
\begin{lstlisting}
`\refcode{Vector3 Public Methods}{+=}\lastnext{Vector3PublicMethods}`
`\refvar{Vector3}{}`<T> operator*(T s) const { return `\refvar{Vector3}{}`<T>(s*x, s*y, s*z); }
`\refvar{Vector3}{}`<T> &operator*=(T s) {
    x *= s; y *= s; z *= s;
    return *this;
}
\end{lstlisting}
\begin{lstlisting}
`\initcode{Geometry Inline Functions}{=}\initnext{GeometryInlineFunctions}`
template <typename T> inline `\refvar{Vector3}{}`<T>
operator*(T s, const `\refvar{Vector3}{}`<T> &v) { return v * s; }
\end{lstlisting}

类似地,向量也可以逐分量除以一个标量。
标量除法的代码和标量乘法相似,
但一个标量除以向量是没有定义的所以是不允许的。

这些方法的实现中,我们用单次除法计算标量的倒数再执行三次逐分量乘法。
这是一个避免除法运算的技巧,
因为在现代CPU上除法比乘法慢得多
\footnote{一个常见误解认为因为编译器会执行必要分析所以这类优化是不必要的。
    编译器一般会被限制执行许多这种转换。
    对于除法,IEEE浮点标准要求对所有\protect{\ttfamily x}都有\protect{\ttfamily x/x=1},
    但如果我们计算\protect{\ttfamily 1/x}将其存储到变量中再和\protect{\ttfamily x}相乘,
    并不能保证结果是1。
    在这种情况下,我们愿意失去这种保证以换取更高的性能。
    关于这个问题的更多讨论详见\protect\refsec{控制舍入误差}。}。

我们用宏\refvar{Assert}{()}保证提供的除数不为零;
这永远不应发生,否则说明系统中有bug。
\begin{lstlisting}
`\refcode{Vector3 Public Methods}{+=}\lastnext{Vector3PublicMethods}`
`\refvar{Vector3}{}`<T> operator/(T f) const {
    `\refvar{Assert}{}`(f != 0);
    Float inv = (Float)1 / f;
    return `\refvar{Vector3}{}`<T>(x * inv, y * inv, z * inv);
}

`\refvar{Vector3}{}`<T> &operator/=(T f) {
    `\refvar{Assert}{}`(f != 0);
    Float inv = (Float)1 / f;
    x *= inv; y *= inv; z *= inv;
    return *this;
}
\end{lstlisting}

\refvar{Vector3}{}类也提供一元负运算,返回与原始向量指向相反方向的新向量。
\begin{lstlisting}
`\refcode{Vector3 Public Methods}{+=}\lastnext{Vector3PublicMethods}`
`\refvar{Vector3}{}`<T> operator-() const { return `\refvar{Vector3}{}`<T>(-x, -y, -z); }
\end{lstlisting}

最后,\refvar{Abs}{()}返回各分量取绝对值的向量。
\begin{lstlisting}
`\refcode{Geometry Inline Functions}{+=}\lastnext{GeometryInlineFunctions}`
template <typename T> `\refvar{Vector3}{}`<T> `\initvar{Abs}{}`(const `\refvar{Vector3}{}`<T> &v) {
    return `\refvar{Vector3}{}`<T>(std::abs(v.x), std::abs(v.y), std::abs(v.z));
}
\end{lstlisting}

\subsection{点积与叉积}\label{sub:点积与叉积}
向量两个很有用的运算是\keyindex{点积}{dot product}{}(也
称\keyindex{数量积}{scalar product}{}或\keyindex{内积}{inner product}{})
与\keyindex{叉积}{cross product}{}。
对于两个向量$\bm v$和$\bm w$,它们的点积$(\bm v \cdot \bm w)$定义为
\begin{align*}
    v_x w_x+ v_y w_y+ v_z w_z\, .
\end{align*}
\begin{lstlisting}
`\refcode{Geometry Inline Functions}{+=}\lastnext{GeometryInlineFunctions}`
template <typename T> inline T
`\initvar{Dot}{}`(const `\refvar{Vector3}{}`<T> &v1, const `\refvar{Vector3}{}`<T> &v2) {
    return v1.x * v2.x + v1.y * v2.y + v1.z * v2.z;
}
\end{lstlisting}

两向量的点积与其夹角有简单关系:
\begin{align}\label{eq:2.1}
    (\bm v \cdot \bm w)=\|\bm v\|\|\bm w\|\cos\theta\, ,
\end{align}
其中$\theta$是$\bm v$和$\bm w$的夹角,
$\|\bm v\|$是向量$\bm v$的长度。
由此可得,
若$\bm v$和$\bm w$都不是\keyindex{退化}{degenerate}{}的——等于$(0,0,0)$,
则$(\bm v \cdot \bm w)$等于零当且仅当$\bm v$和$\bm w$垂直。
称两两垂直的一组向量是\keyindex{正交}{orthogonal}{}的。
称一组正交的单位向量是\keyindex{标准正交}{orthonormal}{}的。

由\refeq{2.1}可以立即得到,如果$\bm v$和$\bm w$是单位向量,
则它们的点积是其夹角的余弦。
由于渲染中经常需要计算两向量夹角的余弦,
所以我们常常利用这个性质。
一些基本属性可直接从定义推导得到。
例如,如果$\bm u,\bm v$和$\bm w$是向量且$s$是标量值,则:
\begin{align*}
    (\bm u\cdot\bm v)         & =(\bm v\cdot\bm u)\, ,                   \\
    (s\bm u\cdot\bm v)        & =s(\bm u\cdot\bm v)\, ,                  \\
    (\bm u\cdot(\bm v+\bm w)) & =(\bm u\cdot\bm v)+(\bm u\cdot\bm w)\, .
\end{align*}

我们还经常需要计算点积的绝对值。
函数\refvar{AbsDot}{()}完成这项工作,
所以不用单独调用{\ttfamily std::abs()}了。
\begin{lstlisting}
`\refcode{Geometry Inline Functions}{+=}\lastnext{GeometryInlineFunctions}`
template <typename T>
inline T `\initvar{AbsDot}{}`(const `\refvar{Vector3}{}`<T> &v1, const `\refvar{Vector3}{}`<T> &v2) {
    return std::abs(`\refvar{Dot}{}`(v1, v2));
}
\end{lstlisting}

叉积是3D向量另一个很有用的运算。
给定两个3D向量,叉积$\bm v\times\bm w$是同时垂直于它们的向量。
给定正交向量$\bm v$和$\bm w$,
则$\bm v\times\bm w$定义为
令$(\bm v,\bm w,\bm v\times\bm w)$构成正交坐标系统的向量。

叉积定义为:
\begin{align*}
    (\bm v\times\bm w)_x & = v_y w_z- v_z w_y\, , \\
    (\bm v\times\bm w)_y & = v_z w_x- v_x w_z\, , \\
    (\bm v\times\bm w)_z & = v_x w_y- v_y w_x\, .
\end{align*}

记住上式的一个方法是计算矩阵的行列式:
\begin{align*}
    \bm v\times\bm w=\left|
    \begin{array}{ccc}
        \mathbf{i} & \mathbf{j} & \mathbf{k} \\
        v_x        & v_y        & v_z        \\
        w_x        & w_y        & w_z
    \end{array}\right|\, ,
\end{align*}
其中$\mathbf{i},\mathbf{j}$和$\mathbf{k}$分别表示
轴$(1,0,0)$、$(0,1,0)$和$(0,0,1)$。
注意该式只是辅助记忆的,并没有严格的数学定义,
因为该矩阵的元素混有标量和向量。

此处的实现中,函数\refvar{Cross}{()}在减法前将
向量元素转化为双精度(不论\refvar{Float}{}的类型)。
这里使用32位浮点数值的额外精度能防止巨量消失
\sidenote{译者注:原文catastrophic cancellation。}
造成的误差,即两个值非常接近的浮点数相减造成的误差。
这不是理论问题:这种改变对于修复以前由此造成的bug是必要的。
浮点舍入误差的更多信息详见\refsec{控制舍入误差}。
\begin{lstlisting}
`\refcode{Geometry Inline Functions}{+=}\lastnext{GeometryInlineFunctions}`
template <typename T> inline `\refvar{Vector3}{}`<T>
`\initvar{Cross}{}`(const `\refvar{Vector3}{}`<T> &v1, const `\refvar{Vector3}{}`<T> &v2) {
    double v1x = v1.x, v1y = v1.y, v1z = v1.z;
    double v2x = v2.x, v2y = v2.y, v2z = v2.z;
    return `\refvar{Vector3}{}`<T>((v1y * v2z) - (v1z * v2y),
                      (v1z * v2x) - (v1x * v2z),
                      (v1x * v2y) - (v1y * v2x));
} 
\end{lstlisting}

由点积的定义可得
\begin{align}\label{eq:2.2}
    \|\bm v\times\bm w\|=\|\bm v\|\|\bm w\||\sin\theta|\, ,
\end{align}
其中$\theta$是$\bm v$和$\bm w$之间的夹角。
一个重要推论是,两垂直的单位向量的叉积本身也是单位向量。
注意如果$\bm v$和$\bm w$平行则叉积结果是退化的。

该定义还给出了计算平行四边形面积的简便方法(\reffig{2.5})。
如果平行四边形两邻边由向量$\bm v_1$和$\bm v_2$给定,
其高为$h$,面积则为$\|\bm v_1\|h$。
因为$h=\sin\theta\|\bm v_2\|$,
我们可以用\refeq{2.2}得到面积为$\|\bm v_1\times\bm v_2\|$。
\begin{figure}[htbp]
    \centering%LaTeX with PSTricks extensions
%%Creator: Inkscape 1.0.1 (3bc2e813f5, 2020-09-07)
%%Please note this file requires PSTricks extensions
\psset{xunit=.5pt,yunit=.5pt,runit=.5pt}
\begin{pspicture}(377.16000366,249.58000183)
{
\newrgbcolor{curcolor}{0 0 0}
\pscustom[linewidth=1,linecolor=curcolor]
{
\newpath
\moveto(251.99000549,24.02000427)
\lineto(356.48001099,214.28000259)
}
}
{
\newrgbcolor{curcolor}{0 0 0}
\pscustom[linestyle=none,fillstyle=solid,fillcolor=curcolor]
{
\newpath
\moveto(358.94,207.33000183)
\lineto(356.16,213.71000183)
\lineto(349.29,212.63000183)
\lineto(360.38,221.38000183)
\closepath
}
}
{
\newrgbcolor{curcolor}{0.65098041 0.65098041 0.65098041}
\pscustom[linestyle=none,fillstyle=solid,fillcolor=curcolor]
{
\newpath
\moveto(358.63,209.27000183)
\lineto(359.75,220.23000183)
\lineto(356.47,214.27000183)
\closepath
}
}
{
\newrgbcolor{curcolor}{0.40000001 0.40000001 0.40000001}
\pscustom[linestyle=none,fillstyle=solid,fillcolor=curcolor]
{
\newpath
\moveto(351.1,213.41000183)
\lineto(359.75,220.23000183)
\lineto(356.47,214.27000183)
\closepath
}
}
{
\newrgbcolor{curcolor}{0 0 0}
\pscustom[linewidth=1,linecolor=curcolor]
{
\newpath
\moveto(0.43000001,24.5)
\lineto(243.88000488,24.04000854)
}
}
{
\newrgbcolor{curcolor}{0 0 0}
\pscustom[linestyle=none,fillstyle=solid,fillcolor=curcolor]
{
\newpath
\moveto(238.96,18.54000183)
\lineto(243.23,24.04000183)
\lineto(238.98,29.55000183)
\lineto(251.99,24.02000183)
\closepath
}
}
{
\newrgbcolor{curcolor}{0.65098041 0.65098041 0.65098041}
\pscustom[linestyle=none,fillstyle=solid,fillcolor=curcolor]
{
\newpath
\moveto(240.52,19.74000183)
\lineto(250.67,24.02000183)
\lineto(243.86,24.04000183)
\closepath
}
}
{
\newrgbcolor{curcolor}{0.40000001 0.40000001 0.40000001}
\pscustom[linestyle=none,fillstyle=solid,fillcolor=curcolor]
{
\newpath
\moveto(240.54,28.34000183)
\lineto(250.67,24.02000183)
\lineto(243.86,24.04000183)
\closepath
}
}
{
\newrgbcolor{curcolor}{0 0 0}
\pscustom[linewidth=1,linecolor=curcolor]
{
\newpath
\moveto(0.44,24.08999634)
\lineto(104.93000031,214.35000229)
}
}
{
\newrgbcolor{curcolor}{0 0 0}
\pscustom[linestyle=none,fillstyle=solid,fillcolor=curcolor]
{
\newpath
\moveto(107.39,207.40000183)
\lineto(104.62,213.78000183)
\lineto(97.74,212.70000183)
\lineto(108.83,221.45000183)
\closepath
}
}
{
\newrgbcolor{curcolor}{0.65098041 0.65098041 0.65098041}
\pscustom[linestyle=none,fillstyle=solid,fillcolor=curcolor]
{
\newpath
\moveto(107.09,209.34000183)
\lineto(108.2,220.30000183)
\lineto(104.92,214.33000183)
\closepath
}
}
{
\newrgbcolor{curcolor}{0.40000001 0.40000001 0.40000001}
\pscustom[linestyle=none,fillstyle=solid,fillcolor=curcolor]
{
\newpath
\moveto(99.55,213.48000183)
\lineto(108.2,220.30000183)
\lineto(104.92,214.33000183)
\closepath
}
}
{
\newrgbcolor{curcolor}{0 0 0}
\pscustom[linewidth=1,linecolor=curcolor]
{
\newpath
\moveto(108.98000336,221.65000153)
\lineto(352.42999268,221.19000244)
}
}
{
\newrgbcolor{curcolor}{0 0 0}
\pscustom[linestyle=none,fillstyle=solid,fillcolor=curcolor]
{
\newpath
\moveto(347.51,215.69000183)
\lineto(351.78,221.19000183)
\lineto(347.53,226.70000183)
\lineto(360.53,221.17000183)
\closepath
}
}
{
\newrgbcolor{curcolor}{0.65098041 0.65098041 0.65098041}
\pscustom[linestyle=none,fillstyle=solid,fillcolor=curcolor]
{
\newpath
\moveto(349.07,216.89000183)
\lineto(359.22,221.17000183)
\lineto(352.41,221.19000183)
\closepath
}
}
{
\newrgbcolor{curcolor}{0.40000001 0.40000001 0.40000001}
\pscustom[linestyle=none,fillstyle=solid,fillcolor=curcolor]
{
\newpath
\moveto(349.09,225.49000183)
\lineto(359.22,221.17000183)
\lineto(352.41,221.19000183)
\closepath
}
}
{
\newrgbcolor{curcolor}{0 0 0}
\pscustom[linewidth=1,linecolor=curcolor,linestyle=dashed,dash=4]
{
\newpath
\moveto(108.81999969,222.10000229)
\lineto(108.81999969,24.74000549)
}
}
{
\newrgbcolor{curcolor}{0 0 0}
\pscustom[linestyle=none,fillstyle=solid,fillcolor=curcolor]
{
\newpath
\moveto(125.97587083,140.78313345)
\curveto(125.97587083,140.78313345)(125.97587083,141.00561317)(125.72160829,141.00561317)
\curveto(125.24486604,141.00561317)(123.78285644,140.84669908)(123.24254855,140.78313345)
\curveto(123.08363447,140.78313345)(122.86115475,140.75135063)(122.86115475,140.40173964)
\curveto(122.86115475,140.14747711)(123.05185165,140.14747711)(123.337897,140.14747711)
\curveto(124.32316434,140.14747711)(124.35494715,140.02034584)(124.35494715,139.79786612)
\lineto(124.29138152,139.38468949)
\lineto(121.36736234,127.72039559)
\curveto(121.27201389,127.43435023)(121.27201389,127.40256741)(121.27201389,127.27543615)
\curveto(121.27201389,126.79869389)(121.68519051,126.70334544)(121.87588741,126.70334544)
\curveto(122.19371559,126.70334544)(122.51154376,126.95760797)(122.60689221,127.24365333)
\lineto(122.98828601,128.76922855)
\lineto(123.43324546,130.58084913)
\curveto(123.56037672,131.05759139)(123.68750799,131.50255083)(123.78285644,131.94751027)
\curveto(123.81463926,132.07464154)(124.00533616,132.7420807)(124.00533616,132.86921197)
\curveto(124.0689018,133.05990887)(124.70455814,134.17230748)(125.40378012,134.74439818)
\curveto(125.84873956,135.06222636)(126.45261309,135.44362016)(127.34253197,135.44362016)
\curveto(128.20066803,135.44362016)(128.42314775,134.74439818)(128.42314775,134.01339339)
\curveto(128.42314775,132.9327776)(127.66036014,130.7079804)(127.18361788,129.46845053)
\curveto(127.0247038,129.02349109)(126.89757253,128.76922855)(126.89757253,128.35605193)
\curveto(126.89757253,127.40256741)(127.62857732,126.70334544)(128.58206184,126.70334544)
\curveto(130.48903087,126.70334544)(131.22003567,129.65914743)(131.22003567,129.81806152)
\curveto(131.22003567,130.04054124)(131.06112158,130.04054124)(130.99755595,130.04054124)
\curveto(130.77507623,130.04054124)(130.77507623,129.97697561)(130.67972777,129.65914743)
\curveto(130.39368242,128.57853165)(129.72624326,127.14830488)(128.61384466,127.14830488)
\curveto(128.26423367,127.14830488)(128.1371024,127.33900178)(128.1371024,127.81574404)
\curveto(128.1371024,128.32426911)(128.29601649,128.80101137)(128.48671339,129.24597081)
\curveto(128.80454156,130.13588969)(129.72624326,132.5513838)(129.72624326,133.72734803)
\curveto(129.72624326,135.03044354)(128.93167283,135.8885796)(127.4060976,135.8885796)
\curveto(126.13478492,135.8885796)(125.14951758,135.25292326)(124.38672997,134.33122156)
\closepath
\moveto(125.97587083,140.78313345)
}
}
{
\newrgbcolor{curcolor}{0 0 0}
\pscustom[linestyle=none,fillstyle=solid,fillcolor=curcolor]
{
\newpath
\moveto(232.47691621,16.61136893)
\curveto(232.47691621,18.45477232)(231.11025508,18.45477232)(231.11025508,18.45477232)
\curveto(230.28390183,18.45477232)(229.55289703,17.59663626)(229.55289703,16.9291971)
\curveto(229.55289703,16.35710639)(229.99785647,16.10284385)(230.15677056,16.0074954)
\curveto(231.01490662,15.49897033)(231.17382071,15.11757652)(231.17382071,14.73618271)
\curveto(231.17382071,14.29122327)(229.99785647,9.84162887)(227.645928,9.84162887)
\curveto(226.18391841,9.84162887)(226.18391841,11.04937592)(226.18391841,11.43076973)
\curveto(226.18391841,12.60673396)(226.75600912,14.06874355)(227.39166546,15.68966723)
\curveto(227.55057955,16.10284385)(227.61414519,16.29354075)(227.61414519,16.61136893)
\curveto(227.61414519,17.78733316)(226.43818095,18.42298951)(225.32578235,18.42298951)
\curveto(223.16455078,18.42298951)(222.14750063,15.68966723)(222.14750063,15.2764906)
\curveto(222.14750063,14.99044525)(222.4653288,14.99044525)(222.6560257,14.99044525)
\curveto(222.87850542,14.99044525)(223.03741951,14.99044525)(223.10098514,15.24470779)
\curveto(223.76842431,17.43772217)(224.84904009,17.69198471)(225.19865108,17.69198471)
\curveto(225.35756516,17.69198471)(225.54826207,17.69198471)(225.54826207,17.27880809)
\curveto(225.54826207,16.80206583)(225.29399953,16.22997512)(225.2304339,16.07106103)
\curveto(224.3087322,13.71913256)(223.99090403,12.79743086)(223.99090403,11.81216353)
\curveto(223.99090403,9.68271478)(225.73895897,9.11062407)(227.51879673,9.11062407)
\curveto(231.01490662,9.11062407)(232.47691621,14.86331398)(232.47691621,16.61136893)
\closepath
\moveto(232.47691621,16.61136893)
}
}
{
\newrgbcolor{curcolor}{0 0 0}
\pscustom[linestyle=none,fillstyle=solid,fillcolor=curcolor]
{
\newpath
\moveto(237.73200639,15.25690569)
\curveto(237.73200639,15.63829949)(237.73200639,15.63829949)(237.31882977,15.63829949)
\curveto(236.39712807,14.74838061)(235.12581538,14.74838061)(234.55372468,14.74838061)
\lineto(234.55372468,14.23985554)
\curveto(234.87155285,14.23985554)(235.82503736,14.23985554)(236.58782498,14.62124934)
\lineto(236.58782498,7.40654984)
\curveto(236.58782498,6.92980758)(236.58782498,6.73911068)(235.18938102,6.73911068)
\lineto(234.64907313,6.73911068)
\lineto(234.64907313,6.2305856)
\curveto(234.90333566,6.2305856)(236.65139061,6.29415124)(237.15991568,6.29415124)
\curveto(237.60487513,6.29415124)(239.38471289,6.2305856)(239.70254106,6.2305856)
\lineto(239.70254106,6.73911068)
\lineto(239.16223317,6.73911068)
\curveto(237.73200639,6.73911068)(237.73200639,6.92980758)(237.73200639,7.40654984)
\closepath
\moveto(237.73200639,15.25690569)
}
}
{
\newrgbcolor{curcolor}{0 0 0}
\pscustom[linestyle=none,fillstyle=solid,fillcolor=curcolor]
{
\newpath
\moveto(332.42167821,239.64600493)
\curveto(332.42167821,241.48940832)(331.05501708,241.48940832)(331.05501708,241.48940832)
\curveto(330.22866383,241.48940832)(329.49765903,240.63127226)(329.49765903,239.9638331)
\curveto(329.49765903,239.39174239)(329.94261847,239.13747985)(330.10153256,239.0421314)
\curveto(330.95966862,238.53360633)(331.11858271,238.15221252)(331.11858271,237.77081871)
\curveto(331.11858271,237.32585927)(329.94261847,232.87626487)(327.59069,232.87626487)
\curveto(326.12868041,232.87626487)(326.12868041,234.08401192)(326.12868041,234.46540573)
\curveto(326.12868041,235.64136996)(326.70077112,237.10337955)(327.33642746,238.72430323)
\curveto(327.49534155,239.13747985)(327.55890719,239.32817675)(327.55890719,239.64600493)
\curveto(327.55890719,240.82196916)(326.38294295,241.45762551)(325.27054435,241.45762551)
\curveto(323.10931278,241.45762551)(322.09226263,238.72430323)(322.09226263,238.3111266)
\curveto(322.09226263,238.02508125)(322.4100908,238.02508125)(322.6007877,238.02508125)
\curveto(322.82326742,238.02508125)(322.98218151,238.02508125)(323.04574714,238.27934379)
\curveto(323.71318631,240.47235817)(324.79380209,240.72662071)(325.14341308,240.72662071)
\curveto(325.30232716,240.72662071)(325.49302407,240.72662071)(325.49302407,240.31344409)
\curveto(325.49302407,239.83670183)(325.23876153,239.26461112)(325.1751959,239.10569703)
\curveto(324.2534942,236.75376856)(323.93566603,235.83206686)(323.93566603,234.84679953)
\curveto(323.93566603,232.71735078)(325.68372097,232.14526007)(327.46355873,232.14526007)
\curveto(330.95966862,232.14526007)(332.42167821,237.89794998)(332.42167821,239.64600493)
\closepath
\moveto(332.42167821,239.64600493)
}
}
{
\newrgbcolor{curcolor}{0 0 0}
\pscustom[linestyle=none,fillstyle=solid,fillcolor=curcolor]
{
\newpath
\moveto(337.67676839,238.29154169)
\curveto(337.67676839,238.67293549)(337.67676839,238.67293549)(337.26359177,238.67293549)
\curveto(336.34189007,237.78301661)(335.07057738,237.78301661)(334.49848668,237.78301661)
\lineto(334.49848668,237.27449154)
\curveto(334.81631485,237.27449154)(335.76979936,237.27449154)(336.53258698,237.65588534)
\lineto(336.53258698,230.44118584)
\curveto(336.53258698,229.96444358)(336.53258698,229.77374668)(335.13414302,229.77374668)
\lineto(334.59383513,229.77374668)
\lineto(334.59383513,229.2652216)
\curveto(334.84809766,229.2652216)(336.59615261,229.32878724)(337.10467768,229.32878724)
\curveto(337.54963713,229.32878724)(339.32947489,229.2652216)(339.64730306,229.2652216)
\lineto(339.64730306,229.77374668)
\lineto(339.10699517,229.77374668)
\curveto(337.67676839,229.77374668)(337.67676839,229.96444358)(337.67676839,230.44118584)
\closepath
\moveto(337.67676839,238.29154169)
}
}
{
\newrgbcolor{curcolor}{0 0 0}
\pscustom[linestyle=none,fillstyle=solid,fillcolor=curcolor]
{
\newpath
\moveto(361.13058821,197.56400493)
\curveto(361.13058821,199.40740832)(359.76392708,199.40740832)(359.76392708,199.40740832)
\curveto(358.93757383,199.40740832)(358.20656903,198.54927226)(358.20656903,197.8818331)
\curveto(358.20656903,197.30974239)(358.65152847,197.05547985)(358.81044256,196.9601314)
\curveto(359.66857862,196.45160633)(359.82749271,196.07021252)(359.82749271,195.68881871)
\curveto(359.82749271,195.24385927)(358.65152847,190.79426487)(356.2996,190.79426487)
\curveto(354.83759041,190.79426487)(354.83759041,192.00201192)(354.83759041,192.38340573)
\curveto(354.83759041,193.55936996)(355.40968112,195.02137955)(356.04533746,196.64230323)
\curveto(356.20425155,197.05547985)(356.26781719,197.24617675)(356.26781719,197.56400493)
\curveto(356.26781719,198.73996916)(355.09185295,199.37562551)(353.97945435,199.37562551)
\curveto(351.81822278,199.37562551)(350.80117263,196.64230323)(350.80117263,196.2291266)
\curveto(350.80117263,195.94308125)(351.1190008,195.94308125)(351.3096977,195.94308125)
\curveto(351.53217742,195.94308125)(351.69109151,195.94308125)(351.75465714,196.19734379)
\curveto(352.42209631,198.39035817)(353.50271209,198.64462071)(353.85232308,198.64462071)
\curveto(354.01123716,198.64462071)(354.20193407,198.64462071)(354.20193407,198.23144409)
\curveto(354.20193407,197.75470183)(353.94767153,197.18261112)(353.8841059,197.02369703)
\curveto(352.9624042,194.67176856)(352.64457603,193.75006686)(352.64457603,192.76479953)
\curveto(352.64457603,190.63535078)(354.39263097,190.06326007)(356.17246873,190.06326007)
\curveto(359.66857862,190.06326007)(361.13058821,195.81594998)(361.13058821,197.56400493)
\closepath
\moveto(361.13058821,197.56400493)
}
}
{
\newrgbcolor{curcolor}{0 0 0}
\pscustom[linestyle=none,fillstyle=solid,fillcolor=curcolor]
{
\newpath
\moveto(368.8011725,189.7576298)
\lineto(368.32443024,189.7576298)
\curveto(368.29264743,189.43980162)(368.13373334,188.61344838)(367.94303644,188.48631711)
\curveto(367.84768798,188.39096866)(366.7670722,188.39096866)(366.54459248,188.39096866)
\lineto(363.93840147,188.39096866)
\curveto(365.43219388,189.69406416)(365.94071895,190.10724079)(366.7670722,190.77467995)
\curveto(367.81590517,191.60103319)(368.8011725,192.49095207)(368.8011725,193.8258304)
\curveto(368.8011725,195.54210252)(367.30738009,196.59093549)(365.49575951,196.59093549)
\curveto(363.74770457,196.59093549)(362.53995751,195.35140562)(362.53995751,194.04831012)
\curveto(362.53995751,193.34908814)(363.14383104,193.25373969)(363.30274513,193.25373969)
\curveto(363.6205733,193.25373969)(364.03374992,193.50800222)(364.03374992,194.0165273)
\curveto(364.03374992,194.27078984)(363.93840147,194.77931491)(363.20739668,194.77931491)
\curveto(363.65235612,195.76458224)(364.60584063,196.08241042)(365.27327979,196.08241042)
\curveto(366.70350657,196.08241042)(367.43451136,194.97001182)(367.43451136,193.8258304)
\curveto(367.43451136,192.58630053)(366.54459248,191.63281601)(366.09963304,191.12429094)
\lineto(362.6988716,187.7235295)
\curveto(362.53995751,187.59639823)(362.53995751,187.56461541)(362.53995751,187.1832216)
\lineto(368.38799588,187.1832216)
\closepath
\moveto(368.8011725,189.7576298)
}
}
{
\newrgbcolor{curcolor}{0 0 0}
\pscustom[linestyle=none,fillstyle=solid,fillcolor=curcolor]
{
\newpath
\moveto(78.12907821,208.08450493)
\curveto(78.12907821,209.92790832)(76.76241708,209.92790832)(76.76241708,209.92790832)
\curveto(75.93606383,209.92790832)(75.20505903,209.06977226)(75.20505903,208.4023331)
\curveto(75.20505903,207.83024239)(75.65001847,207.57597985)(75.80893256,207.4806314)
\curveto(76.66706862,206.97210633)(76.82598271,206.59071252)(76.82598271,206.20931871)
\curveto(76.82598271,205.76435927)(75.65001847,201.31476487)(73.29809,201.31476487)
\curveto(71.83608041,201.31476487)(71.83608041,202.52251192)(71.83608041,202.90390573)
\curveto(71.83608041,204.07986996)(72.40817112,205.54187955)(73.04382746,207.16280323)
\curveto(73.20274155,207.57597985)(73.26630719,207.76667675)(73.26630719,208.08450493)
\curveto(73.26630719,209.26046916)(72.09034295,209.89612551)(70.97794435,209.89612551)
\curveto(68.81671278,209.89612551)(67.79966263,207.16280323)(67.79966263,206.7496266)
\curveto(67.79966263,206.46358125)(68.1174908,206.46358125)(68.3081877,206.46358125)
\curveto(68.53066742,206.46358125)(68.68958151,206.46358125)(68.75314714,206.71784379)
\curveto(69.42058631,208.91085817)(70.50120209,209.16512071)(70.85081308,209.16512071)
\curveto(71.00972716,209.16512071)(71.20042407,209.16512071)(71.20042407,208.75194409)
\curveto(71.20042407,208.27520183)(70.94616153,207.70311112)(70.8825959,207.54419703)
\curveto(69.9608942,205.19226856)(69.64306603,204.27056686)(69.64306603,203.28529953)
\curveto(69.64306603,201.15585078)(71.39112097,200.58376007)(73.17095873,200.58376007)
\curveto(76.66706862,200.58376007)(78.12907821,206.33644998)(78.12907821,208.08450493)
\closepath
\moveto(78.12907821,208.08450493)
}
}
{
\newrgbcolor{curcolor}{0 0 0}
\pscustom[linestyle=none,fillstyle=solid,fillcolor=curcolor]
{
\newpath
\moveto(85.7996625,200.2781298)
\lineto(85.32292024,200.2781298)
\curveto(85.29113743,199.96030162)(85.13222334,199.13394838)(84.94152644,199.00681711)
\curveto(84.84617798,198.91146866)(83.7655622,198.91146866)(83.54308248,198.91146866)
\lineto(80.93689147,198.91146866)
\curveto(82.43068388,200.21456416)(82.93920895,200.62774079)(83.7655622,201.29517995)
\curveto(84.81439517,202.12153319)(85.7996625,203.01145207)(85.7996625,204.3463304)
\curveto(85.7996625,206.06260252)(84.30587009,207.11143549)(82.49424951,207.11143549)
\curveto(80.74619457,207.11143549)(79.53844751,205.87190562)(79.53844751,204.56881012)
\curveto(79.53844751,203.86958814)(80.14232104,203.77423969)(80.30123513,203.77423969)
\curveto(80.6190633,203.77423969)(81.03223992,204.02850222)(81.03223992,204.5370273)
\curveto(81.03223992,204.79128984)(80.93689147,205.29981491)(80.20588668,205.29981491)
\curveto(80.65084612,206.28508224)(81.60433063,206.60291042)(82.27176979,206.60291042)
\curveto(83.70199657,206.60291042)(84.43300136,205.49051182)(84.43300136,204.3463304)
\curveto(84.43300136,203.10680053)(83.54308248,202.15331601)(83.09812304,201.64479094)
\lineto(79.6973616,198.2440295)
\curveto(79.53844751,198.11689823)(79.53844751,198.08511541)(79.53844751,197.7037216)
\lineto(85.38648588,197.7037216)
\closepath
\moveto(85.7996625,200.2781298)
}
}
\end{pspicture}

    \caption{由向量$\bm v_1$和$\bm v_2$给定邻边的平行四边形面积等于$\|\bm v_1\|h$。
    由\protect\refeq{2.2},$\bm v_1$和$\bm v_2$点积的长度等于
    两向量长度之积乘以它们夹角的正弦——平行四边形面积。}
    \label{fig:2.5}
\end{figure}

\subsection{规范化}\label{sub:规范化}
通常有必要对向量\keyindex{规范化}{normalize}{}
\sidenote{译者注:也称为归一化、标准化。},
即计算一个单位长度的新向量指向相同方向。
规范化的向量常称为\keyindex{单位向量}{unit vector}{vector向量}。
本书中用记号$\hat{\bm v}$表示向量$\bm v$的规范化版本。
为了规范化向量,首先要能计算其长度。
\begin{lstlisting}
`\refcode{Vector3 Public Methods}{+=}\lastcode{Vector3PublicMethods}`
Float `\initvar{LengthSquared}{}`() const { return x * x + y * y + z * z; }
Float `\initvar{Length}{}`() const { return std::sqrt(`\refvar{LengthSquared}{}`()); }
\end{lstlisting}

\refvar{Normalize}{()}规范化一个向量。
它用向量的长度$\|\bm v\|$去除每个分量,返回一个新向量;
它\emph{不会}原位规范化向量:
\begin{lstlisting}
`\refcode{Geometry Inline Functions}{+=}\lastnext{GeometryInlineFunctions}`
template <typename T> inline `\refvar{Vector3}{}`<T>
`\initvar{Normalize}{}`(const `\refvar{Vector3}{}`<T> &v) { return v / v.`\refvar{Length}{}`(); }
\end{lstlisting}