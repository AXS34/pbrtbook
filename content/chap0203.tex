\section{点}\label{sec:点}

点是2D或3D空间中的零维位置。
pbrt中的类\refvar{Point2}{}和\refvar{Point3}{}以
明确的方式表示点:使用关于坐标系统的$x,y,z$(3D)坐标。
尽管向量也使用同样的表示,
但事实上点表示一个位置而向量表示一个方向,
由此它们的处理方式有很多重要的差异。
文中点记为$\bm p$。

本节中,我们这里仍只涵盖类\refvar{Point3}{}中3D点方法的实现。
\begin{lstlisting}
`\initcode{Point Declarations}{=}\initnext{PointDeclarations}`
template <typename T> class `\initvar{Point2}{}` {
public:
    `\refcode{Point2 Public Methods}{}`
    `\refcode{Point2 Public Data}{}`
};
\end{lstlisting}

\begin{lstlisting}
`\refcode{Point Declarations}{+=}\lastnext{PointDeclarations}`
template <typename T> class `\initvar{Point3}{}` {
public:
    `\refcode{Point3 Public Methods}{}`
    `\refcode{Point3 Public Data}{}`
};
\end{lstlisting}

像向量那样,常用点类型有更短的类型名称会很有用。
\begin{lstlisting}
`\refcode{Point Declarations}{+=}\lastcode{PointDeclarations}`
typedef `\refvar{Point2}{}`<`\refvar{Float}{}`> `\initvar{Point2f}{}`;
typedef `\refvar{Point2}{}`<int>   `\initvar{Point2i}{}`;
typedef `\refvar{Point3}{}`<`\refvar{Float}{}`> `\initvar{Point3f}{}`;
typedef `\refvar{Point3}{}`<int>   `\initvar{Point3i}{}`;

`\initcode{Point2 Public Data}{=}`
T x, y;

`\initcode{Point3 Public Data}{=}`
T x, y, z;
\end{lstlisting}

还像向量那样,\refvar{Point3}{}的构造函数接收参数
设置{\ttfamily x},{\ttfamily y}和{\ttfamily z}坐标值。
\begin{lstlisting}
`\initcode{Point3 Public Methods}{=}\initnext{Point3PublicMethods}`
`\refvar{Point3}{}`() { x = y = z = 0; }
`\refvar{Point3}{}`(T x, T y, T z) : x(x), y(y), z(z) {
    `\refvar{Assert}{}`(!HasNaNs());
}
\end{lstlisting}

通过丢弃$z$坐标将\refvar{Point3}{}转换为\refvar{Point2}{}很有用。
完成这个转换的构造函数有修饰符{\ttfamily explicit}
所以该转换不会在缺乏显示类型转换时发生,以免意外。
\begin{lstlisting}
`\initcode{Point2 Public Methods}{=}`
explicit `\refvar{Point2}{}`(const `\refvar{Point3}{}`<T> &p) : x(p.x), y(p.y) {
    `\refvar{Assert}{}`(!HasNaNs());
}
\end{lstlisting}

