\section{点}\label{sec:点}

点是2D或3D空间中的零维位置。
pbrt中的类\refvar{Point2}{}和\refvar{Point3}{}以
明确的方式表示点:使用关于坐标系统的$x,y,z$(3D)坐标。
尽管向量也使用同样的表示,
但事实上点表示一个位置而向量表示一个方向,
由此它们的处理方式有很多重要的差异。
文中点记为$\bm p$。

本节中,我们这里仍只涵盖类\refvar{Point3}{}中3D点方法的实现。
\begin{lstlisting}
`\initcode{Point Declarations}{=}\initnext{PointDeclarations}`
template <typename T> class `\initvar{Point2}{}` {
public:
    `\refcode{Point2 Public Methods}{}`
    `\refcode{Point2 Public Data}{}`
};
\end{lstlisting}

\begin{lstlisting}
`\refcode{Point Declarations}{+=}\lastnext{PointDeclarations}`
template <typename T> class `\initvar{Point3}{}` {
public:
    `\refcode{Point3 Public Methods}{}`
    `\refcode{Point3 Public Data}{}`
};
\end{lstlisting}

像向量那样,常用点类型有更短的类型名称会很有用。
\begin{lstlisting}
`\refcode{Point Declarations}{+=}\lastcode{PointDeclarations}`
typedef `\refvar{Point2}{}`<`\refvar{Float}{}`> `\initvar{Point2f}{}`;
typedef `\refvar{Point2}{}`<int>   `\initvar{Point2i}{}`;
typedef `\refvar{Point3}{}`<`\refvar{Float}{}`> `\initvar{Point3f}{}`;
typedef `\refvar{Point3}{}`<int>   `\initvar{Point3i}{}`;

`\initcode{Point2 Public Data}{=}`
T x, y;

`\initcode{Point3 Public Data}{=}`
T x, y, z;
\end{lstlisting}

还像向量那样,\refvar{Point3}{}的构造函数接收参数
设置{\ttfamily x},{\ttfamily y}和{\ttfamily z}坐标值。
\begin{lstlisting}
`\initcode{Point3 Public Methods}{=}\initnext{Point3PublicMethods}`
`\refvar{Point3}{}`() { x = y = z = 0; }
`\refvar{Point3}{}`(T x, T y, T z) : x(x), y(y), z(z) {
    `\refvar{Assert}{}`(!HasNaNs());
}
\end{lstlisting}

通过丢弃$z$坐标将\refvar{Point3}{}转换为\refvar{Point2}{}很有用。
完成这个转换的构造函数有修饰符{\ttfamily explicit}
所以该转换不会在缺乏显式类型转换时发生,以免意外。
\begin{lstlisting}
`\initcode{Point2 Public Methods}{=}`
explicit `\refvar{Point2}{}`(const `\refvar{Point3}{}`<T> &p) : x(p.x), y(p.y) {
    `\refvar{Assert}{}`(!HasNaNs());
}
\end{lstlisting}

将一种元素类型的点(例如\refvar{Point3f}{})转换为另一种点(例如\refvar{Point3i}{})
和将点转换为不同元素类型的向量都很有用。
下列构造函数和转换操作提供了这些转换形式。
它们也都要求有显式类型转换,使得在使用它们时源码更加清楚。
\begin{lstlisting}
`\refcode{Point3 Public Methods}{+=}\lastnext{Point3PublicMethods}`
template <typename U> explicit `\refvar{Point3}{}`(const `\refvar{Point3}{}`<U> &p)
    : x((T)p.x), y((T)p.y), z((T)p.z) { 
        `\refvar{Assert}{}`(!HasNaNs());
}
template <typename U> explicit operator `\refvar{Vector3}{}`<U>() const {
    return `\refvar{Vector3}{}`<U>(x, y, z);
}
\end{lstlisting}

某些\refvar{Point3}{}方法返回或接收\refvar{Vector3}{}。
例如,可以把向量加到点上,使之朝给定方向偏移得到一个新点。
\begin{lstlisting}
`\refcode{Point3 Public Methods}{+=}\lastnext{Point3PublicMethods}`
`\refvar{Point3}{}`<T> operator+(const `\refvar{Vector3}{}`<T> &v) const {
    return `\refvar{Point3}{}`<T>(x + v.x, y + v.y, z + v.z);
}
`\refvar{Point3}{}`<T> &operator+=(const `\refvar{Vector3}{}`<T> &v) {
    x += v.x; y += v.y; z += v.z;
    return *this;
}
\end{lstlisting}

或者,从一个点减去另一个点,得到两者间的向量,如\reffig{2.6}所示。
\begin{figure}
    \centering%LaTeX with PSTricks extensions
%%Creator: Inkscape 1.0.1 (3bc2e813f5, 2020-09-07)
%%Please note this file requires PSTricks extensions
\psset{xunit=.5pt,yunit=.5pt,runit=.5pt}
\begin{pspicture}(340.70999146,77.08999634)
{
\newrgbcolor{curcolor}{0 0 0}
\pscustom[linestyle=none,fillstyle=solid,fillcolor=curcolor]
{
\newpath
\moveto(36.07000017,17.15999603)
\curveto(36.07000017,22.70180941)(29.37019859,25.47615546)(25.45201977,21.55797664)
\curveto(21.53384095,17.63979782)(24.308187,10.93999624)(29.85000038,10.93999624)
\curveto(35.39181376,10.93999624)(38.16615981,17.63979782)(34.24798099,21.55797664)
\curveto(30.32980217,25.47615546)(23.63000059,22.70180941)(23.63000059,17.15999603)
\curveto(23.63000059,11.61818265)(30.32980217,8.84383661)(34.24798099,12.76201542)
\curveto(38.16615981,16.68019424)(35.39181376,23.37999582)(29.85000038,23.37999582)
\curveto(24.308187,23.37999582)(21.53384095,16.68019424)(25.45201977,12.76201542)
\curveto(29.37019859,8.84383661)(36.07000017,11.61818265)(36.07000017,17.15999603)
\closepath
}
}
{
\newrgbcolor{curcolor}{0 0 0}
\pscustom[linestyle=none,fillstyle=solid,fillcolor=curcolor]
{
\newpath
\moveto(313.01000834,52.3999958)
\curveto(313.01000834,57.94180919)(306.31020676,60.71615523)(302.39202794,56.79797641)
\curveto(298.47384912,52.87979759)(301.24819516,46.17999601)(306.79000854,46.17999601)
\curveto(312.33182193,46.17999601)(315.10616797,52.87979759)(311.18798915,56.79797641)
\curveto(307.26981033,60.71615523)(300.57000875,57.94180919)(300.57000875,52.3999958)
\curveto(300.57000875,46.85818242)(307.26981033,44.08383638)(311.18798915,48.0020152)
\curveto(315.10616797,51.92019401)(312.33182193,58.61999559)(306.79000854,58.61999559)
\curveto(301.24819516,58.61999559)(298.47384912,51.92019401)(302.39202794,48.0020152)
\curveto(306.31020676,44.08383638)(313.01000834,46.85818242)(313.01000834,52.3999958)
\closepath
}
}
{
\newrgbcolor{curcolor}{0 0 0}
\pscustom[linewidth=1,linecolor=curcolor]
{
\newpath
\moveto(29.22999954,17.56999588)
\lineto(292.79000854,50.56999588)
}
}
{
\newrgbcolor{curcolor}{0 0 0}
\pscustom[linestyle=none,fillstyle=solid,fillcolor=curcolor]
{
\newpath
\moveto(288.6,44.49999634)
\lineto(292.14,50.48999634)
\lineto(287.24,55.41999634)
\lineto(300.83,51.56999634)
\closepath
}
}
{
\newrgbcolor{curcolor}{0.65098041 0.65098041 0.65098041}
\pscustom[linestyle=none,fillstyle=solid,fillcolor=curcolor]
{
\newpath
\moveto(290,45.87999634)
\lineto(299.53,51.40999634)
\lineto(292.77,50.56999634)
\closepath
}
}
{
\newrgbcolor{curcolor}{0.40000001 0.40000001 0.40000001}
\pscustom[linestyle=none,fillstyle=solid,fillcolor=curcolor]
{
\newpath
\moveto(288.93,54.40999634)
\lineto(299.53,51.40999634)
\lineto(292.77,50.56999634)
\closepath
}
}
{
\newrgbcolor{curcolor}{0 0 0}
\pscustom[linestyle=none,fillstyle=solid,fillcolor=curcolor]
{
\newpath
\moveto(8.70998398,8.48346722)
\curveto(8.6491669,8.20979035)(8.61875836,8.17938181)(8.58834982,8.14897327)
\curveto(8.4971242,8.11856473)(8.28426442,8.11856473)(8.10181317,8.11856473)
\curveto(7.76731923,8.11856473)(7.40241674,8.11856473)(7.40241674,7.571211)
\curveto(7.40241674,7.35835122)(7.58486798,7.20630851)(7.79772777,7.20630851)
\curveto(8.3450815,7.20630851)(8.98366085,7.26712559)(9.56142312,7.26712559)
\curveto(10.26081956,7.26712559)(10.99062453,7.20630851)(11.65961243,7.20630851)
\curveto(11.78124659,7.20630851)(12.14614908,7.20630851)(12.14614908,7.78407079)
\curveto(12.14614908,8.11856473)(11.84206367,8.11856473)(11.65961243,8.11856473)
\curveto(11.38593556,8.11856473)(11.05144161,8.11856473)(10.80817329,8.14897327)
\lineto(11.65961243,11.52432128)
\curveto(11.93328929,11.25064442)(12.57186864,10.82492485)(13.60575903,10.82492485)
\curveto(16.98110703,10.82492485)(19.07929634,13.89618745)(19.07929634,16.54173049)
\curveto(19.07929634,18.94400519)(17.28519244,19.73462725)(15.67353979,19.73462725)
\curveto(14.30515546,19.73462725)(13.30167362,18.97441374)(12.99758821,18.70073687)
\curveto(12.2373747,19.73462725)(10.96021599,19.73462725)(10.74735621,19.73462725)
\curveto(10.04795977,19.73462725)(9.4701975,19.33931622)(9.07488647,18.63991979)
\curveto(8.58834982,17.84929773)(8.31467296,16.81540735)(8.31467296,16.72418173)
\curveto(8.31467296,16.45050486)(8.61875836,16.45050486)(8.80120961,16.45050486)
\curveto(9.01406939,16.45050486)(9.07488647,16.45050486)(9.16611209,16.54173049)
\curveto(9.22692918,16.57213903)(9.22692918,16.63295611)(9.34856334,17.11949276)
\curveto(9.71346583,18.67032833)(10.16959393,19.03523082)(10.65613058,19.03523082)
\curveto(10.86899037,19.03523082)(11.11225869,18.97441374)(11.11225869,18.33583438)
\curveto(11.11225869,18.03174898)(11.05144161,17.75807211)(10.99062453,17.48439524)
\closepath
\moveto(13.18003946,17.78848065)
\curveto(13.72739319,18.45746854)(14.63964941,19.03523082)(15.58231417,19.03523082)
\curveto(16.79865579,19.03523082)(16.88988141,18.00134044)(16.88988141,17.57562087)
\curveto(16.88988141,16.57213903)(16.22089352,14.16986432)(15.91680811,13.4096508)
\curveto(15.3086373,12.01085793)(14.36597254,11.52432128)(13.57535048,11.52432128)
\curveto(12.41982594,11.52432128)(11.96369783,12.4365775)(11.96369783,12.64943729)
\lineto(11.99410637,12.92311415)
\closepath
\moveto(13.18003946,17.78848065)
}
}
{
\newrgbcolor{curcolor}{0 0 0}
\pscustom[linestyle=none,fillstyle=solid,fillcolor=curcolor]
{
\newpath
\moveto(319.18217398,49.14054722)
\curveto(319.1213569,48.86687035)(319.09094836,48.83646181)(319.06053982,48.80605327)
\curveto(318.9693142,48.77564473)(318.75645442,48.77564473)(318.57400317,48.77564473)
\curveto(318.23950923,48.77564473)(317.87460674,48.77564473)(317.87460674,48.228291)
\curveto(317.87460674,48.01543122)(318.05705798,47.86338851)(318.26991777,47.86338851)
\curveto(318.8172715,47.86338851)(319.45585085,47.92420559)(320.03361312,47.92420559)
\curveto(320.73300956,47.92420559)(321.46281453,47.86338851)(322.13180243,47.86338851)
\curveto(322.25343659,47.86338851)(322.61833908,47.86338851)(322.61833908,48.44115079)
\curveto(322.61833908,48.77564473)(322.31425367,48.77564473)(322.13180243,48.77564473)
\curveto(321.85812556,48.77564473)(321.52363161,48.77564473)(321.28036329,48.80605327)
\lineto(322.13180243,52.18140128)
\curveto(322.40547929,51.90772442)(323.04405864,51.48200485)(324.07794903,51.48200485)
\curveto(327.45329703,51.48200485)(329.55148634,54.55326745)(329.55148634,57.19881049)
\curveto(329.55148634,59.60108519)(327.75738244,60.39170725)(326.14572979,60.39170725)
\curveto(324.77734546,60.39170725)(323.77386362,59.63149374)(323.46977821,59.35781687)
\curveto(322.7095647,60.39170725)(321.43240599,60.39170725)(321.21954621,60.39170725)
\curveto(320.52014977,60.39170725)(319.9423875,59.99639622)(319.54707647,59.29699979)
\curveto(319.06053982,58.50637773)(318.78686296,57.47248735)(318.78686296,57.38126173)
\curveto(318.78686296,57.10758486)(319.09094836,57.10758486)(319.27339961,57.10758486)
\curveto(319.48625939,57.10758486)(319.54707647,57.10758486)(319.63830209,57.19881049)
\curveto(319.69911918,57.22921903)(319.69911918,57.29003611)(319.82075334,57.77657276)
\curveto(320.18565583,59.32740833)(320.64178393,59.69231082)(321.12832058,59.69231082)
\curveto(321.34118037,59.69231082)(321.58444869,59.63149374)(321.58444869,58.99291438)
\curveto(321.58444869,58.68882898)(321.52363161,58.41515211)(321.46281453,58.14147524)
\closepath
\moveto(323.65222946,58.44556065)
\curveto(324.19958319,59.11454854)(325.11183941,59.69231082)(326.05450417,59.69231082)
\curveto(327.27084579,59.69231082)(327.36207141,58.65842044)(327.36207141,58.23270087)
\curveto(327.36207141,57.22921903)(326.69308352,54.82694432)(326.38899811,54.0667308)
\curveto(325.7808273,52.66793793)(324.83816254,52.18140128)(324.04754048,52.18140128)
\curveto(322.89201594,52.18140128)(322.43588783,53.0936575)(322.43588783,53.30651729)
\lineto(322.46629637,53.58019415)
\closepath
\moveto(323.65222946,58.44556065)
}
}
{
\newrgbcolor{curcolor}{0 0 0}
\pscustom[linestyle=none,fillstyle=solid,fillcolor=curcolor]
{
\newpath
\moveto(333.75297263,66.05669365)
\curveto(333.87460679,66.26955343)(333.87460679,66.39118759)(333.87460679,66.48241322)
\curveto(333.87460679,66.90813278)(333.5097043,67.21221819)(333.08398473,67.21221819)
\curveto(332.56703954,67.21221819)(332.41499684,66.78649862)(332.35417976,66.57363884)
\lineto(330.56007586,60.7047905)
\curveto(330.52966732,60.67438196)(330.46885024,60.52233925)(330.46885024,60.49193071)
\curveto(330.46885024,60.33988801)(330.89456981,60.18784531)(331.01620397,60.18784531)
\curveto(331.10742959,60.18784531)(331.10742959,60.21825385)(331.19865522,60.43111363)
\closepath
\moveto(333.75297263,66.05669365)
}
}
{
\newrgbcolor{curcolor}{0 0 0}
\pscustom[linestyle=none,fillstyle=solid,fillcolor=curcolor]
{
\newpath
\moveto(174.10556466,49.58617044)
\curveto(174.10556466,51.34986579)(172.79799741,51.34986579)(172.79799741,51.34986579)
\curveto(172.00737536,51.34986579)(171.30797892,50.52883519)(171.30797892,49.89025584)
\curveto(171.30797892,49.34290211)(171.73369849,49.09963379)(171.88574119,49.00840816)
\curveto(172.70677179,48.52187151)(172.85881449,48.15696903)(172.85881449,47.79206654)
\curveto(172.85881449,47.36634697)(171.73369849,43.10915128)(169.48346648,43.10915128)
\curveto(168.08467362,43.10915128)(168.08467362,44.26467583)(168.08467362,44.62957831)
\curveto(168.08467362,45.75469432)(168.63202735,47.15348719)(169.24019816,48.70432276)
\curveto(169.39224086,49.09963379)(169.45305794,49.28208503)(169.45305794,49.58617044)
\curveto(169.45305794,50.71128644)(168.32794194,51.31945725)(167.26364302,51.31945725)
\curveto(165.19586226,51.31945725)(164.22278896,48.70432276)(164.22278896,48.30901173)
\curveto(164.22278896,48.03533486)(164.52687436,48.03533486)(164.70932561,48.03533486)
\curveto(164.92218539,48.03533486)(165.07422809,48.03533486)(165.13504518,48.27860319)
\curveto(165.77362453,50.37679249)(166.80751491,50.62006082)(167.14200886,50.62006082)
\curveto(167.29405156,50.62006082)(167.4765028,50.62006082)(167.4765028,50.22474979)
\curveto(167.4765028,49.76862168)(167.23323448,49.22126795)(167.1724174,49.06922524)
\curveto(166.29056972,46.81899324)(165.98648431,45.93714556)(165.98648431,44.9944808)
\curveto(165.98648431,42.95710858)(167.65895405,42.40975485)(169.36183232,42.40975485)
\curveto(172.70677179,42.40975485)(174.10556466,47.9137007)(174.10556466,49.58617044)
\closepath
\moveto(174.10556466,49.58617044)
}
}
\end{pspicture}

    \caption{求两点间的向量。向量$\bm v=\bm p'-\bm p$由点$\bm p'$和$\bm p$的逐元素减法得到。}
    \label{fig:2.6}
\end{figure}

\begin{lstlisting}
`\refcode{Point3 Public Methods}{+=}\lastnext{Point3PublicMethods}`
`\refvar{Vector3}{}`<T> operator-(const `\refvar{Point3}{}`<T> &p) const {
    return `\refvar{Vector3}{}`<T>(x - p.x, y - p.y, z - p.z);
}
\end{lstlisting}

从一个点减去向量得到另一个点。
\begin{lstlisting}
`\refcode{Point3 Public Methods}{+=}\lastcode{Point3PublicMethods}`
`\refvar{Point3}{}`<T> operator-(const `\refvar{Vector3}{}`<T> &v) const {
    return `\refvar{Point3}{}`<T>(x - v.x, y - v.y, z - v.z);
}
`\refvar{Point3}{}`<T> &operator-=(const `\refvar{Vector3}{}`<T> &v) {
    x -= v.x; y -= v.y; z -= v.z;
    return *this;
}
\end{lstlisting}

两点间的距离可以由它们相减所得向量的长度算得:
\begin{lstlisting}
`\refcode{Geometry Inline Functions}{+=}\lastnext{GeometryInlineFunctions}`
template <typename T> inline Float
`\initvar{Distance}{}`(const `\refvar{Point3}{}`<T> &p1, const `\refvar{Point3}{}`<T> &p2) {
    return (p1 - p2).`\refvar{Length}{}`();
}
template <typename T> inline Float
`\initvar{DistanceSquared}{}`(const `\refvar{Point3}{}`<T> &p1, const `\refvar{Point3}{}`<T> &p2) {
    return (p1 - p2).`\refvar{LengthSquared}{}`();
}
\end{lstlisting}

尽管通常用标量对点赋权或把两点相加没有数学意义,
但为了能计算点的加权和,点类仍允许这类操作,
只要所用权重和为一就有数学意义。
数乘代码以及与点求和的代码和向量的相应代码一样,
所以这里不再赘述。

值得一提的是,两点间的线性\keyindex{插值}{interpolate}{}很有用。
\refvar{Lerp}{()}在{\ttfamily t==0}时返回{\ttfamily p0},
{\ttfamily t==1}时返回{\ttfamily p1},
{\ttfamily t}取其余值时则在两点间插值。
对于{\ttfamily t<0}或{\ttfamily t>1},
\refvar{Lerp}{()}进行\keyindex{外推}{extrapolate}{}。
\begin{lstlisting}
`\refcode{Geometry Inline Functions}{+=}\lastnext{GeometryInlineFunctions}`
template <typename T> `\refvar{Point3}{}`<T>
`\initvar{Lerp}{}`(Float t, const `\refvar{Point3}{}`<T> &p0, const `\refvar{Point3}{}`<T> &p1) {
    return (1 - t) * p0 + t * p1;
}
\end{lstlisting}

函数\refvar[Point3::Min]{Min}{()}和\refvar[Point3::Max]{Max}{()}返回
给定两点逐元素取最小和最大值后表示的点。
\begin{lstlisting}
`\refcode{Geometry Inline Functions}{+=}\lastnext{GeometryInlineFunctions}`
template <typename T> `\refvar{Point3}{}`<T>
`\initvar[Point3::Min]{Min}{}`(const `\refvar{Point3}{}`<T> &p1, const `\refvar{Point3}{}`<T> &p2) {
    return `\refvar{Point3}{}`<T>(std::min(p1.x, p2.x), std::min(p1.y, p2.y), 
                     std::min(p1.z, p2.z));
}
template <typename T> `\refvar{Point3}{}`<T>
`\initvar[Point3::Max]{Max}{}`(const `\refvar{Point3}{}`<T> &p1, const `\refvar{Point3}{}`<T> &p2) {
    return `\refvar{Point3}{}`<T>(std::max(p1.x, p2.x), std::max(p1.y, p2.y), 
                     std::max(p1.z, p2.z));
}
\end{lstlisting}

\refvar{Floor}{()}、\refvar{Ceil}{()}和\refvar[Point3::Abs]{Abs}{()}对
给定点逐元素应用相应操作。
\begin{lstlisting}
`\refcode{Geometry Inline Functions}{+=}\lastnext{GeometryInlineFunctions}`
template <typename T> `\refvar{Point3}{}`<T> `\initvar{Floor}{}`(const `\refvar{Point3}{}`<T> &p) {
    return `\refvar{Point3}{}`<T>(std::floor(p.x), std::floor(p.y), std::floor(p.z));
}
template <typename T> `\refvar{Point3}{}`<T> `\initvar{Ceil}{}`(const `\refvar{Point3}{}`<T> &p) {
    return `\refvar{Point3}{}`<T>(std::ceil(p.x), std::ceil(p.y), std::ceil(p.z));
}
template <typename T> `\refvar{Point3}{}`<T> `\initvar[Point3::Abs]{Abs}{}`(const `\refvar{Point3}{}`<T> &p) {
    return `\refvar{Point3}{}`<T>(std::abs(p.x), std::abs(p.y), std::abs(p.z));
}
\end{lstlisting}

最后,\refvar[Point3::Permute]{Permute}{()}根据提供的顺序对坐标值排序。
\begin{lstlisting}
`\refcode{Geometry Inline Functions}{+=}\lastnext{GeometryInlineFunctions}`
template <typename T> `\refvar{Point3}{}`<T>
`\initvar[Point3::Permute]{Permute}{}`(const `\refvar{Point3}{}`<T> &p, int x, int y, int z) {
    return `\refvar{Point3}{}`<T>(p[x], p[y], p[z]);
}
\end{lstlisting}