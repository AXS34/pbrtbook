\section{习题}\label{sec:习题03}

\begin{enumerate}
    \item \circletwo 像三角网格和细分曲面那样基于网格的形状的
          一个良好性质是形状的顶点可以变换到世界空间,
          这样在执行光线相交测试之前就不需要将射线变换到物体空间中。
          有趣的是,光线-二次曲面相交也可以做同样的事。
          本章二次曲面的隐式形式都形如
          \begin{align*}
              Ax^2+Bxy+Cxz+Dy^2+Eyz+Fz^2+G=0\, ,
          \end{align*}
          其中一些常数$A\ldots G$为零。更一般地,
          我们可以用方程
          \begin{align*}
              Ax^2+By^2+Cz^2+2Dxy+2Eyz+2Fxz+2Gx+2Hy+2Iz+J=0
          \end{align*}
          定义二次曲面,其中参数$A\ldots J$的大多数并不与之前的$A\ldots G$直接相关。
          该形式中,二次曲面可以表示为$4\times4$的对称矩阵$\bm Q$:
          \begin{align*}
              [x\ y\ z\ 1]\left[
                  \begin{array}{cccc}
                      A & D & F & G \\
                      D & B & E & H \\
                      F & E & C & I \\
                      G & H & I & J
                  \end{array}
                  \right]\left[\begin{array}{c}
                      x \\y\\z\\1
                  \end{array}\right]=\bm p^{\mathrm{T}}\bm Q\bm p=0\, .
          \end{align*}
          有了该表示,首先证明表示被矩阵$\bm M$变换的二次曲面的矩阵$\bm Q'$为
          \begin{align*}
              \bm Q'=(\bm M^{\mathrm{T}})^{-1}\bm Q\bm M^{-1}\, .
          \end{align*}
          这样可以证明对于任意满足$\bm p^{\mathrm{T}}\bm Q\bm p=0$的点$\bm p$,
          如果我们对$\bm p$施加变换$\bm M$并算得$\bm p'=\bm M\bm p$,
          则我们会发现$\bm Q'$满足$(\bm p')^{\mathrm{T}}\bm Q'\bm p'=0$。
          接着,将射线方程代入之前更一般的二次曲面方程计算二次方程$at^2+bt+c=0$
          用传入函数\refvar{Quadratic}{()}的矩阵$\bm Q$的元素形式表示的系数。
          现在在pbrt中实现该方法并用它替代原来的二次曲面相交例程。
          注意如果$\theta_{\max}$等等不是$2\pi$,
          则你仍需要将得到的世界空间命中点变换到物体空间以测试它们。
          性能与原始方案相比如何?
    \item \circleone 为二次曲面改进物体空间边界框例程
          以适当考虑$\displaystyle\phi_{\max}<\frac{3\pi}{2}$,
          并在可能时计算更紧的边界框。
          当渲染部分二次曲面形状场景时性能提升了多少?
    \item \circletwo 还有以许多方式优化pbrt中各种二次曲面图元实现的空间。
          例如,对于完整球体而言相交例程中一些与部分球体相关的测试是不必要的。
          此外,一些二次曲面调用的三角函数其实可以
          用观察特定图元几何结构的方式变为更简单的表达式。
          研究加速这些方法的方式。
          当渲染二次曲面场景时这样做对pbrt总运行时间有多少改进?
    \item \circleone 目前pbrt在三角形需要时每次都重新计算
          偏导数$\displaystyle\frac{\partial \bm p}{\partial u}$
          和$\displaystyle\frac{\partial \bm p}{\partial v}$,
          即使它们对每个三角形都是常数。
          预先计算这些向量并分析速度/存储取舍,尤其是对于大型三角网格。
          场景的深度复杂度和图像中的三角形大小如何影响这种取舍?
    \item \circletwo 在pbrt中将支持任意顶点数量和凸或\keyindex{凹}{concave}{}多边形
          的通用\keyindex{多边形}{polygon}{}图元实现为新的\refvar{Shape}{}。
          你可以假设已经提供了有效多边形且多边形的所有顶点都在同一平面上,
          但你可能想在不是这种情况时发出警告。
          计算光线-多边形相交的一个高效技术是由多边形的法线和该多边形所在平面上的一点求得该平面方程。
          然后计算射线与该平面的相交并将交点和多边形顶点投影到2D。
          再运用2D点在多边形内的测试确定该点是否在多边形内。
          这样做的一个简单方式是高效进行2D光线追踪计算,
          将射线与每条边线段相交,并计数它穿过了多少次。
          如果它穿过了奇数次,则该点在多边形内且存在相交。
          该思路的图示见\reffig{3.47}。
          你可能发现阅读\citet{HAINES199424}调研了许多高效的点在多边形内测试方法的论文会很有帮助。
          里面描述的一些技术可能有益于优化该测试。
          此外,Schneider和Eberly \parencite*{10.5555/2821579}的13.3.3节
          讨论了适用所有极端情况的策略:例如当2D射线精确对齐一边或穿过多边形顶点时。
          \begin{figure}[htbp]
              \centering%LaTeX with PSTricks extensions
%%Creator: Inkscape 1.0.1 (3bc2e813f5, 2020-09-07)
%%Please note this file requires PSTricks extensions
\psset{xunit=.35pt,yunit=.35pt,runit=.35pt}
\begin{pspicture}(505.30999756,398.66000366)
{
\newrgbcolor{curcolor}{1 1 1}
\pscustom[linestyle=none,fillstyle=solid,fillcolor=curcolor]
{
\newpath
\moveto(215.15,269.37000366)
\lineto(245.27,249.90000366)
\lineto(300.28,288.28000366)
\lineto(275.4,352.23000366)
\lineto(196.38,362.80000366)
\lineto(152.72,317.20000366)
\lineto(228.25,312.19000366)
\closepath
}
}
{
\newrgbcolor{curcolor}{0 0 0}
\pscustom[linewidth=1,linecolor=curcolor]
{
\newpath
\moveto(215.15,269.37000366)
\lineto(245.27,249.90000366)
\lineto(300.28,288.28000366)
\lineto(275.4,352.23000366)
\lineto(196.38,362.80000366)
\lineto(152.72,317.20000366)
\lineto(228.25,312.19000366)
\closepath
}
}
{
\newrgbcolor{curcolor}{0 0 0}
\pscustom[linewidth=1,linecolor=curcolor]
{
\newpath
\moveto(337.72,0.50000366)
\lineto(33.41,1.79000366)
\lineto(121.23,173.21000366)
\lineto(425.54,171.92000366)
\closepath
}
}
{
\newrgbcolor{curcolor}{0.60000002 0.60000002 0.60000002}
\pscustom[linestyle=none,fillstyle=solid,fillcolor=curcolor]
{
\newpath
\moveto(215.6,58.28000366)
\lineto(245.72,43.00000366)
\lineto(300.73,73.12000366)
\lineto(275.85,123.33000366)
\lineto(196.83,131.62000366)
\lineto(153.17,95.82000366)
\lineto(228.7,91.90000366)
\closepath
}
}
{
\newrgbcolor{curcolor}{0 0 0}
\pscustom[linewidth=1,linecolor=curcolor]
{
\newpath
\moveto(215.6,58.28000366)
\lineto(245.72,43.00000366)
\lineto(300.73,73.12000366)
\lineto(275.85,123.33000366)
\lineto(196.83,131.62000366)
\lineto(153.17,95.82000366)
\lineto(228.7,91.90000366)
\closepath
}
}
{
\newrgbcolor{curcolor}{0 0 0}
\pscustom[linewidth=1,linecolor=curcolor]
{
\newpath
\moveto(320.79,206.68000366)
\lineto(1.25,271.96000366)
\lineto(184.52,398.13000366)
\lineto(504.06,332.84000366)
\closepath
}
}
{
\newrgbcolor{curcolor}{0 0 0}
\pscustom[linewidth=1,linecolor=curcolor,linestyle=dashed,dash=2]
{
\newpath
\moveto(245.6000061,222.32000732)
\lineto(245.6000061,42.42001343)
}
}
{
\newrgbcolor{curcolor}{0.60000002 0.60000002 0.60000002}
\pscustom[linewidth=1,linecolor=curcolor,linestyle=dashed,dash=2]
{
\newpath
\moveto(245.6000061,249.77000427)
\lineto(245.6000061,222.32000732)
}
}
{
\newrgbcolor{curcolor}{0 0 0}
\pscustom[linewidth=1,linecolor=curcolor,linestyle=dashed,dash=2]
{
\newpath
\moveto(300.58999634,210.83000183)
\lineto(300.58999634,74)
}
}
{
\newrgbcolor{curcolor}{0.60000002 0.60000002 0.60000002}
\pscustom[linewidth=1,linecolor=curcolor,linestyle=dashed,dash=2]
{
\newpath
\moveto(300.58999634,288.3500061)
\lineto(300.58999634,210.83000183)
}
}
{
\newrgbcolor{curcolor}{0 0 0}
\pscustom[linewidth=1,linecolor=curcolor,linestyle=dashed,dash=2]
{
\newpath
\moveto(215.28999329,228.22000122)
\lineto(215.28999329,57.55001831)
}
}
{
\newrgbcolor{curcolor}{0.60000002 0.60000002 0.60000002}
\pscustom[linewidth=1,linecolor=curcolor,linestyle=dashed,dash=2]
{
\newpath
\moveto(215.28999329,270.16000366)
\lineto(215.28999329,228.22000122)
}
}
{
\newrgbcolor{curcolor}{0 0 0}
\pscustom[linewidth=1,linecolor=curcolor,linestyle=dashed,dash=2]
{
\newpath
\moveto(229.24000549,225.21000671)
\lineto(229.24000549,91.91000366)
}
}
{
\newrgbcolor{curcolor}{0.60000002 0.60000002 0.60000002}
\pscustom[linewidth=1,linecolor=curcolor,linestyle=dashed,dash=2]
{
\newpath
\moveto(229.24000549,311.94000244)
\lineto(229.24000549,225.21000671)
}
}
{
\newrgbcolor{curcolor}{0 0 0}
\pscustom[linewidth=1,linecolor=curcolor,linestyle=dashed,dash=2]
{
\newpath
\moveto(152.27000427,241.1000061)
\lineto(152.27000427,96.3999939)
}
}
{
\newrgbcolor{curcolor}{0.60000002 0.60000002 0.60000002}
\pscustom[linewidth=1,linecolor=curcolor,linestyle=dashed,dash=2]
{
\newpath
\moveto(152.27000427,318.18000031)
\lineto(152.27000427,241.1000061)
}
}
{
\newrgbcolor{curcolor}{0 0 0}
\pscustom[linewidth=1,linecolor=curcolor,linestyle=dashed,dash=2]
{
\newpath
\moveto(196.66000366,232.08000183)
\lineto(196.66000366,132.05999756)
}
}
{
\newrgbcolor{curcolor}{0.60000002 0.60000002 0.60000002}
\pscustom[linewidth=1,linecolor=curcolor,linestyle=dashed,dash=2]
{
\newpath
\moveto(196.66000366,363.00000381)
\lineto(196.66000366,232.08000183)
}
}
{
\newrgbcolor{curcolor}{0 0 0}
\pscustom[linewidth=1,linecolor=curcolor,linestyle=dashed,dash=2]
{
\newpath
\moveto(275.38000488,216.08999634)
\lineto(275.38000488,123.33001709)
}
}
{
\newrgbcolor{curcolor}{0.60000002 0.60000002 0.60000002}
\pscustom[linewidth=1,linecolor=curcolor,linestyle=dashed,dash=2]
{
\newpath
\moveto(275.38000488,352.9600029)
\lineto(275.38000488,216.08999634)
}
}
{
\newrgbcolor{curcolor}{1 1 1}
\pscustom[linestyle=none,fillstyle=solid,fillcolor=curcolor]
{
\newpath
\moveto(230.68999624,312.16000366)
\curveto(230.68999624,314.25377558)(228.15872074,315.30196098)(226.67837988,313.82162012)
\curveto(225.19803902,312.34127926)(226.24622442,309.81000376)(228.33999634,309.81000376)
\curveto(230.43376826,309.81000376)(231.48195366,312.34127926)(230.00161279,313.82162012)
\curveto(228.52127193,315.30196098)(225.98999643,314.25377558)(225.98999643,312.16000366)
\curveto(225.98999643,310.06623174)(228.52127193,309.01804634)(230.00161279,310.49838721)
\curveto(231.48195366,311.97872807)(230.43376826,314.51000357)(228.33999634,314.51000357)
\curveto(226.24622442,314.51000357)(225.19803902,311.97872807)(226.67837988,310.49838721)
\curveto(228.15872074,309.01804634)(230.68999624,310.06623174)(230.68999624,312.16000366)
\closepath
}
}
{
\newrgbcolor{curcolor}{0 0 0}
\pscustom[linewidth=1,linecolor=curcolor]
{
\newpath
\moveto(230.68999624,312.16000366)
\curveto(230.68999624,314.25377558)(228.15872074,315.30196098)(226.67837988,313.82162012)
\curveto(225.19803902,312.34127926)(226.24622442,309.81000376)(228.33999634,309.81000376)
\curveto(230.43376826,309.81000376)(231.48195366,312.34127926)(230.00161279,313.82162012)
\curveto(228.52127193,315.30196098)(225.98999643,314.25377558)(225.98999643,312.16000366)
\curveto(225.98999643,310.06623174)(228.52127193,309.01804634)(230.00161279,310.49838721)
\curveto(231.48195366,311.97872807)(230.43376826,314.51000357)(228.33999634,314.51000357)
\curveto(226.24622442,314.51000357)(225.19803902,311.97872807)(226.67837988,310.49838721)
\curveto(228.15872074,309.01804634)(230.68999624,310.06623174)(230.68999624,312.16000366)
\closepath
}
}
{
\newrgbcolor{curcolor}{0 0 0}
\pscustom[linestyle=none,fillstyle=solid,fillcolor=curcolor]
{
\newpath
\moveto(270.08001089,85.3999939)
\curveto(270.08001089,87.49376582)(267.54873539,88.54195121)(266.06839453,87.06161035)
\curveto(264.58805367,85.58126949)(265.63623906,83.04999399)(267.73001099,83.04999399)
\curveto(269.82378291,83.04999399)(270.8719683,85.58126949)(269.39162744,87.06161035)
\curveto(267.91128658,88.54195121)(265.38001108,87.49376582)(265.38001108,85.3999939)
\curveto(265.38001108,83.30622197)(267.91128658,82.25803658)(269.39162744,83.73837744)
\curveto(270.8719683,85.2187183)(269.82378291,87.7499938)(267.73001099,87.7499938)
\curveto(265.63623906,87.7499938)(264.58805367,85.2187183)(266.06839453,83.73837744)
\curveto(267.54873539,82.25803658)(270.08001089,83.30622197)(270.08001089,85.3999939)
\closepath
}
}
{
\newrgbcolor{curcolor}{0 0 0}
\pscustom[linewidth=1,linecolor=curcolor]
{
\newpath
\moveto(270.08001089,85.3999939)
\curveto(270.08001089,87.49376582)(267.54873539,88.54195121)(266.06839453,87.06161035)
\curveto(264.58805367,85.58126949)(265.63623906,83.04999399)(267.73001099,83.04999399)
\curveto(269.82378291,83.04999399)(270.8719683,85.58126949)(269.39162744,87.06161035)
\curveto(267.91128658,88.54195121)(265.38001108,87.49376582)(265.38001108,85.3999939)
\curveto(265.38001108,83.30622197)(267.91128658,82.25803658)(269.39162744,83.73837744)
\curveto(270.8719683,85.2187183)(269.82378291,87.7499938)(267.73001099,87.7499938)
\curveto(265.63623906,87.7499938)(264.58805367,85.2187183)(266.06839453,83.73837744)
\curveto(267.54873539,82.25803658)(270.08001089,83.30622197)(270.08001089,85.3999939)
\closepath
}
}
{
\newrgbcolor{curcolor}{1 1 1}
\pscustom[linestyle=none,fillstyle=solid,fillcolor=curcolor]
{
\newpath
\moveto(217.04999685,269.16000366)
\curveto(217.04999685,271.25377558)(214.51872135,272.30196098)(213.03838049,270.82162012)
\curveto(211.55803963,269.34127926)(212.60622503,266.81000376)(214.69999695,266.81000376)
\curveto(216.79376887,266.81000376)(217.84195427,269.34127926)(216.3616134,270.82162012)
\curveto(214.88127254,272.30196098)(212.34999704,271.25377558)(212.34999704,269.16000366)
\curveto(212.34999704,267.06623174)(214.88127254,266.01804634)(216.3616134,267.49838721)
\curveto(217.84195427,268.97872807)(216.79376887,271.51000357)(214.69999695,271.51000357)
\curveto(212.60622503,271.51000357)(211.55803963,268.97872807)(213.03838049,267.49838721)
\curveto(214.51872135,266.01804634)(217.04999685,267.06623174)(217.04999685,269.16000366)
\closepath
}
}
{
\newrgbcolor{curcolor}{0 0 0}
\pscustom[linewidth=1,linecolor=curcolor]
{
\newpath
\moveto(217.04999685,269.16000366)
\curveto(217.04999685,271.25377558)(214.51872135,272.30196098)(213.03838049,270.82162012)
\curveto(211.55803963,269.34127926)(212.60622503,266.81000376)(214.69999695,266.81000376)
\curveto(216.79376887,266.81000376)(217.84195427,269.34127926)(216.3616134,270.82162012)
\curveto(214.88127254,272.30196098)(212.34999704,271.25377558)(212.34999704,269.16000366)
\curveto(212.34999704,267.06623174)(214.88127254,266.01804634)(216.3616134,267.49838721)
\curveto(217.84195427,268.97872807)(216.79376887,271.51000357)(214.69999695,271.51000357)
\curveto(212.60622503,271.51000357)(211.55803963,268.97872807)(213.03838049,267.49838721)
\curveto(214.51872135,266.01804634)(217.04999685,267.06623174)(217.04999685,269.16000366)
\closepath
}
}
{
\newrgbcolor{curcolor}{1 1 1}
\pscustom[linestyle=none,fillstyle=solid,fillcolor=curcolor]
{
\newpath
\moveto(247.98000479,250.29000854)
\curveto(247.98000479,252.38378047)(245.44872929,253.43196586)(243.96838843,251.951625)
\curveto(242.48804756,250.47128414)(243.53623296,247.94000864)(245.63000488,247.94000864)
\curveto(247.72377681,247.94000864)(248.7719622,250.47128414)(247.29162134,251.951625)
\curveto(245.81128048,253.43196586)(243.28000498,252.38378047)(243.28000498,250.29000854)
\curveto(243.28000498,248.19623662)(245.81128048,247.14805123)(247.29162134,248.62839209)
\curveto(248.7719622,250.10873295)(247.72377681,252.64000845)(245.63000488,252.64000845)
\curveto(243.53623296,252.64000845)(242.48804756,250.10873295)(243.96838843,248.62839209)
\curveto(245.44872929,247.14805123)(247.98000479,248.19623662)(247.98000479,250.29000854)
\closepath
}
}
{
\newrgbcolor{curcolor}{0 0 0}
\pscustom[linewidth=1,linecolor=curcolor]
{
\newpath
\moveto(247.98000479,250.29000854)
\curveto(247.98000479,252.38378047)(245.44872929,253.43196586)(243.96838843,251.951625)
\curveto(242.48804756,250.47128414)(243.53623296,247.94000864)(245.63000488,247.94000864)
\curveto(247.72377681,247.94000864)(248.7719622,250.47128414)(247.29162134,251.951625)
\curveto(245.81128048,253.43196586)(243.28000498,252.38378047)(243.28000498,250.29000854)
\curveto(243.28000498,248.19623662)(245.81128048,247.14805123)(247.29162134,248.62839209)
\curveto(248.7719622,250.10873295)(247.72377681,252.64000845)(245.63000488,252.64000845)
\curveto(243.53623296,252.64000845)(242.48804756,250.10873295)(243.96838843,248.62839209)
\curveto(245.44872929,247.14805123)(247.98000479,248.19623662)(247.98000479,250.29000854)
\closepath
}
}
{
\newrgbcolor{curcolor}{1 1 1}
\pscustom[linestyle=none,fillstyle=solid,fillcolor=curcolor]
{
\newpath
\moveto(277.42000723,352.29000473)
\curveto(277.42000723,354.38377665)(274.88873173,355.43196205)(273.40839087,353.95162119)
\curveto(271.92805001,352.47128032)(272.9762354,349.94000483)(275.07000732,349.94000483)
\curveto(277.16377925,349.94000483)(278.21196464,352.47128032)(276.73162378,353.95162119)
\curveto(275.25128292,355.43196205)(272.72000742,354.38377665)(272.72000742,352.29000473)
\curveto(272.72000742,350.19623281)(275.25128292,349.14804741)(276.73162378,350.62838827)
\curveto(278.21196464,352.10872914)(277.16377925,354.64000463)(275.07000732,354.64000463)
\curveto(272.9762354,354.64000463)(271.92805001,352.10872914)(273.40839087,350.62838827)
\curveto(274.88873173,349.14804741)(277.42000723,350.19623281)(277.42000723,352.29000473)
\closepath
}
}
{
\newrgbcolor{curcolor}{0 0 0}
\pscustom[linewidth=1,linecolor=curcolor]
{
\newpath
\moveto(277.42000723,352.29000473)
\curveto(277.42000723,354.38377665)(274.88873173,355.43196205)(273.40839087,353.95162119)
\curveto(271.92805001,352.47128032)(272.9762354,349.94000483)(275.07000732,349.94000483)
\curveto(277.16377925,349.94000483)(278.21196464,352.47128032)(276.73162378,353.95162119)
\curveto(275.25128292,355.43196205)(272.72000742,354.38377665)(272.72000742,352.29000473)
\curveto(272.72000742,350.19623281)(275.25128292,349.14804741)(276.73162378,350.62838827)
\curveto(278.21196464,352.10872914)(277.16377925,354.64000463)(275.07000732,354.64000463)
\curveto(272.9762354,354.64000463)(271.92805001,352.10872914)(273.40839087,350.62838827)
\curveto(274.88873173,349.14804741)(277.42000723,350.19623281)(277.42000723,352.29000473)
\closepath
}
}
{
\newrgbcolor{curcolor}{1 1 1}
\pscustom[linestyle=none,fillstyle=solid,fillcolor=curcolor]
{
\newpath
\moveto(198.42000723,362.61000443)
\curveto(198.42000723,364.70377635)(195.88873173,365.75196174)(194.40839087,364.27162088)
\curveto(192.92805001,362.79128002)(193.9762354,360.26000452)(196.07000732,360.26000452)
\curveto(198.16377925,360.26000452)(199.21196464,362.79128002)(197.73162378,364.27162088)
\curveto(196.25128292,365.75196174)(193.72000742,364.70377635)(193.72000742,362.61000443)
\curveto(193.72000742,360.5162325)(196.25128292,359.46804711)(197.73162378,360.94838797)
\curveto(199.21196464,362.42872883)(198.16377925,364.96000433)(196.07000732,364.96000433)
\curveto(193.9762354,364.96000433)(192.92805001,362.42872883)(194.40839087,360.94838797)
\curveto(195.88873173,359.46804711)(198.42000723,360.5162325)(198.42000723,362.61000443)
\closepath
}
}
{
\newrgbcolor{curcolor}{0 0 0}
\pscustom[linewidth=1,linecolor=curcolor]
{
\newpath
\moveto(198.42000723,362.61000443)
\curveto(198.42000723,364.70377635)(195.88873173,365.75196174)(194.40839087,364.27162088)
\curveto(192.92805001,362.79128002)(193.9762354,360.26000452)(196.07000732,360.26000452)
\curveto(198.16377925,360.26000452)(199.21196464,362.79128002)(197.73162378,364.27162088)
\curveto(196.25128292,365.75196174)(193.72000742,364.70377635)(193.72000742,362.61000443)
\curveto(193.72000742,360.5162325)(196.25128292,359.46804711)(197.73162378,360.94838797)
\curveto(199.21196464,362.42872883)(198.16377925,364.96000433)(196.07000732,364.96000433)
\curveto(193.9762354,364.96000433)(192.92805001,362.42872883)(194.40839087,360.94838797)
\curveto(195.88873173,359.46804711)(198.42000723,360.5162325)(198.42000723,362.61000443)
\closepath
}
}
{
\newrgbcolor{curcolor}{1 1 1}
\pscustom[linestyle=none,fillstyle=solid,fillcolor=curcolor]
{
\newpath
\moveto(154.42000723,317.37000275)
\curveto(154.42000723,319.46377467)(151.88873173,320.51196006)(150.40839087,319.0316192)
\curveto(148.92805001,317.55127834)(149.9762354,315.02000284)(152.07000732,315.02000284)
\curveto(154.16377925,315.02000284)(155.21196464,317.55127834)(153.73162378,319.0316192)
\curveto(152.25128292,320.51196006)(149.72000742,319.46377467)(149.72000742,317.37000275)
\curveto(149.72000742,315.27623082)(152.25128292,314.22804543)(153.73162378,315.70838629)
\curveto(155.21196464,317.18872715)(154.16377925,319.72000265)(152.07000732,319.72000265)
\curveto(149.9762354,319.72000265)(148.92805001,317.18872715)(150.40839087,315.70838629)
\curveto(151.88873173,314.22804543)(154.42000723,315.27623082)(154.42000723,317.37000275)
\closepath
}
}
{
\newrgbcolor{curcolor}{0 0 0}
\pscustom[linewidth=1,linecolor=curcolor]
{
\newpath
\moveto(154.42000723,317.37000275)
\curveto(154.42000723,319.46377467)(151.88873173,320.51196006)(150.40839087,319.0316192)
\curveto(148.92805001,317.55127834)(149.9762354,315.02000284)(152.07000732,315.02000284)
\curveto(154.16377925,315.02000284)(155.21196464,317.55127834)(153.73162378,319.0316192)
\curveto(152.25128292,320.51196006)(149.72000742,319.46377467)(149.72000742,317.37000275)
\curveto(149.72000742,315.27623082)(152.25128292,314.22804543)(153.73162378,315.70838629)
\curveto(155.21196464,317.18872715)(154.16377925,319.72000265)(152.07000732,319.72000265)
\curveto(149.9762354,319.72000265)(148.92805001,317.18872715)(150.40839087,315.70838629)
\curveto(151.88873173,314.22804543)(154.42000723,315.27623082)(154.42000723,317.37000275)
\closepath
}
}
{
\newrgbcolor{curcolor}{1 1 1}
\pscustom[linestyle=none,fillstyle=solid,fillcolor=curcolor]
{
\newpath
\moveto(302.55999136,287.51000214)
\curveto(302.55999136,289.60377406)(300.02871586,290.65195945)(298.548375,289.17161859)
\curveto(297.06803414,287.69127773)(298.11621953,285.16000223)(300.20999146,285.16000223)
\curveto(302.30376338,285.16000223)(303.35194877,287.69127773)(301.87160791,289.17161859)
\curveto(300.39126705,290.65195945)(297.85999155,289.60377406)(297.85999155,287.51000214)
\curveto(297.85999155,285.41623021)(300.39126705,284.36804482)(301.87160791,285.84838568)
\curveto(303.35194877,287.32872654)(302.30376338,289.86000204)(300.20999146,289.86000204)
\curveto(298.11621953,289.86000204)(297.06803414,287.32872654)(298.548375,285.84838568)
\curveto(300.02871586,284.36804482)(302.55999136,285.41623021)(302.55999136,287.51000214)
\closepath
}
}
{
\newrgbcolor{curcolor}{0 0 0}
\pscustom[linewidth=1,linecolor=curcolor]
{
\newpath
\moveto(302.55999136,287.51000214)
\curveto(302.55999136,289.60377406)(300.02871586,290.65195945)(298.548375,289.17161859)
\curveto(297.06803414,287.69127773)(298.11621953,285.16000223)(300.20999146,285.16000223)
\curveto(302.30376338,285.16000223)(303.35194877,287.69127773)(301.87160791,289.17161859)
\curveto(300.39126705,290.65195945)(297.85999155,289.60377406)(297.85999155,287.51000214)
\curveto(297.85999155,285.41623021)(300.39126705,284.36804482)(301.87160791,285.84838568)
\curveto(303.35194877,287.32872654)(302.30376338,289.86000204)(300.20999146,289.86000204)
\curveto(298.11621953,289.86000204)(297.06803414,287.32872654)(298.548375,285.84838568)
\curveto(300.02871586,284.36804482)(302.55999136,285.41623021)(302.55999136,287.51000214)
\closepath
}
}
{
\newrgbcolor{curcolor}{1 1 1}
\pscustom[linestyle=none,fillstyle=solid,fillcolor=curcolor]
{
\newpath
\moveto(231.46000051,91.97000122)
\curveto(231.46000051,94.06377314)(228.92872502,95.11195854)(227.44838415,93.63161768)
\curveto(225.96804329,92.15127682)(227.01622869,89.62000132)(229.11000061,89.62000132)
\curveto(231.20377253,89.62000132)(232.25195793,92.15127682)(230.77161707,93.63161768)
\curveto(229.2912762,95.11195854)(226.76000071,94.06377314)(226.76000071,91.97000122)
\curveto(226.76000071,89.8762293)(229.2912762,88.8280439)(230.77161707,90.30838476)
\curveto(232.25195793,91.78872563)(231.20377253,94.32000113)(229.11000061,94.32000113)
\curveto(227.01622869,94.32000113)(225.96804329,91.78872563)(227.44838415,90.30838476)
\curveto(228.92872502,88.8280439)(231.46000051,89.8762293)(231.46000051,91.97000122)
\closepath
}
}
{
\newrgbcolor{curcolor}{0 0 0}
\pscustom[linewidth=1,linecolor=curcolor]
{
\newpath
\moveto(231.46000051,91.97000122)
\curveto(231.46000051,94.06377314)(228.92872502,95.11195854)(227.44838415,93.63161768)
\curveto(225.96804329,92.15127682)(227.01622869,89.62000132)(229.11000061,89.62000132)
\curveto(231.20377253,89.62000132)(232.25195793,92.15127682)(230.77161707,93.63161768)
\curveto(229.2912762,95.11195854)(226.76000071,94.06377314)(226.76000071,91.97000122)
\curveto(226.76000071,89.8762293)(229.2912762,88.8280439)(230.77161707,90.30838476)
\curveto(232.25195793,91.78872563)(231.20377253,94.32000113)(229.11000061,94.32000113)
\curveto(227.01622869,94.32000113)(225.96804329,91.78872563)(227.44838415,90.30838476)
\curveto(228.92872502,88.8280439)(231.46000051,89.8762293)(231.46000051,91.97000122)
\closepath
}
}
{
\newrgbcolor{curcolor}{1 1 1}
\pscustom[linestyle=none,fillstyle=solid,fillcolor=curcolor]
{
\newpath
\moveto(277.98000479,123)
\curveto(277.98000479,125.09377192)(275.44872929,126.14195732)(273.96838843,124.66161646)
\curveto(272.48804756,123.18127559)(273.53623296,120.6500001)(275.63000488,120.6500001)
\curveto(277.72377681,120.6500001)(278.7719622,123.18127559)(277.29162134,124.66161646)
\curveto(275.81128048,126.14195732)(273.28000498,125.09377192)(273.28000498,123)
\curveto(273.28000498,120.90622808)(275.81128048,119.85804268)(277.29162134,121.33838354)
\curveto(278.7719622,122.81872441)(277.72377681,125.3499999)(275.63000488,125.3499999)
\curveto(273.53623296,125.3499999)(272.48804756,122.81872441)(273.96838843,121.33838354)
\curveto(275.44872929,119.85804268)(277.98000479,120.90622808)(277.98000479,123)
\closepath
}
}
{
\newrgbcolor{curcolor}{0 0 0}
\pscustom[linewidth=1,linecolor=curcolor]
{
\newpath
\moveto(277.98000479,123)
\curveto(277.98000479,125.09377192)(275.44872929,126.14195732)(273.96838843,124.66161646)
\curveto(272.48804756,123.18127559)(273.53623296,120.6500001)(275.63000488,120.6500001)
\curveto(277.72377681,120.6500001)(278.7719622,123.18127559)(277.29162134,124.66161646)
\curveto(275.81128048,126.14195732)(273.28000498,125.09377192)(273.28000498,123)
\curveto(273.28000498,120.90622808)(275.81128048,119.85804268)(277.29162134,121.33838354)
\curveto(278.7719622,122.81872441)(277.72377681,125.3499999)(275.63000488,125.3499999)
\curveto(273.53623296,125.3499999)(272.48804756,122.81872441)(273.96838843,121.33838354)
\curveto(275.44872929,119.85804268)(277.98000479,120.90622808)(277.98000479,123)
\closepath
}
}
{
\newrgbcolor{curcolor}{1 1 1}
\pscustom[linestyle=none,fillstyle=solid,fillcolor=curcolor]
{
\newpath
\moveto(302.82000113,73.3999939)
\curveto(302.82000113,75.49376582)(300.28872563,76.54195121)(298.80838476,75.06161035)
\curveto(297.3280439,73.58126949)(298.3762293,71.04999399)(300.47000122,71.04999399)
\curveto(302.56377314,71.04999399)(303.61195854,73.58126949)(302.13161768,75.06161035)
\curveto(300.65127682,76.54195121)(298.12000132,75.49376582)(298.12000132,73.3999939)
\curveto(298.12000132,71.30622197)(300.65127682,70.25803658)(302.13161768,71.73837744)
\curveto(303.61195854,73.2187183)(302.56377314,75.7499938)(300.47000122,75.7499938)
\curveto(298.3762293,75.7499938)(297.3280439,73.2187183)(298.80838476,71.73837744)
\curveto(300.28872563,70.25803658)(302.82000113,71.30622197)(302.82000113,73.3999939)
\closepath
}
}
{
\newrgbcolor{curcolor}{0 0 0}
\pscustom[linewidth=1,linecolor=curcolor]
{
\newpath
\moveto(302.82000113,73.3999939)
\curveto(302.82000113,75.49376582)(300.28872563,76.54195121)(298.80838476,75.06161035)
\curveto(297.3280439,73.58126949)(298.3762293,71.04999399)(300.47000122,71.04999399)
\curveto(302.56377314,71.04999399)(303.61195854,73.58126949)(302.13161768,75.06161035)
\curveto(300.65127682,76.54195121)(298.12000132,75.49376582)(298.12000132,73.3999939)
\curveto(298.12000132,71.30622197)(300.65127682,70.25803658)(302.13161768,71.73837744)
\curveto(303.61195854,73.2187183)(302.56377314,75.7499938)(300.47000122,75.7499938)
\curveto(298.3762293,75.7499938)(297.3280439,73.2187183)(298.80838476,71.73837744)
\curveto(300.28872563,70.25803658)(302.82000113,71.30622197)(302.82000113,73.3999939)
\closepath
}
}
{
\newrgbcolor{curcolor}{1 1 1}
\pscustom[linestyle=none,fillstyle=solid,fillcolor=curcolor]
{
\newpath
\moveto(247.96000051,43.08001709)
\curveto(247.96000051,45.17378901)(245.42872502,46.22197441)(243.94838415,44.74163355)
\curveto(242.46804329,43.26129268)(243.51622869,40.73001719)(245.61000061,40.73001719)
\curveto(247.70377253,40.73001719)(248.75195793,43.26129268)(247.27161707,44.74163355)
\curveto(245.7912762,46.22197441)(243.26000071,45.17378901)(243.26000071,43.08001709)
\curveto(243.26000071,40.98624517)(245.7912762,39.93805977)(247.27161707,41.41840063)
\curveto(248.75195793,42.8987415)(247.70377253,45.43001699)(245.61000061,45.43001699)
\curveto(243.51622869,45.43001699)(242.46804329,42.8987415)(243.94838415,41.41840063)
\curveto(245.42872502,39.93805977)(247.96000051,40.98624517)(247.96000051,43.08001709)
\closepath
}
}
{
\newrgbcolor{curcolor}{0 0 0}
\pscustom[linewidth=1,linecolor=curcolor]
{
\newpath
\moveto(247.96000051,43.08001709)
\curveto(247.96000051,45.17378901)(245.42872502,46.22197441)(243.94838415,44.74163355)
\curveto(242.46804329,43.26129268)(243.51622869,40.73001719)(245.61000061,40.73001719)
\curveto(247.70377253,40.73001719)(248.75195793,43.26129268)(247.27161707,44.74163355)
\curveto(245.7912762,46.22197441)(243.26000071,45.17378901)(243.26000071,43.08001709)
\curveto(243.26000071,40.98624517)(245.7912762,39.93805977)(247.27161707,41.41840063)
\curveto(248.75195793,42.8987415)(247.70377253,45.43001699)(245.61000061,45.43001699)
\curveto(243.51622869,45.43001699)(242.46804329,42.8987415)(243.94838415,41.41840063)
\curveto(245.42872502,39.93805977)(247.96000051,40.98624517)(247.96000051,43.08001709)
\closepath
}
}
{
\newrgbcolor{curcolor}{1 1 1}
\pscustom[linestyle=none,fillstyle=solid,fillcolor=curcolor]
{
\newpath
\moveto(218.04999685,58.02999878)
\curveto(218.04999685,60.1237707)(215.51872135,61.1719561)(214.03838049,59.69161524)
\curveto(212.55803963,58.21127437)(213.60622503,55.67999887)(215.69999695,55.67999887)
\curveto(217.79376887,55.67999887)(218.84195427,58.21127437)(217.3616134,59.69161524)
\curveto(215.88127254,61.1719561)(213.34999704,60.1237707)(213.34999704,58.02999878)
\curveto(213.34999704,55.93622686)(215.88127254,54.88804146)(217.3616134,56.36838232)
\curveto(218.84195427,57.84872318)(217.79376887,60.37999868)(215.69999695,60.37999868)
\curveto(213.60622503,60.37999868)(212.55803963,57.84872318)(214.03838049,56.36838232)
\curveto(215.51872135,54.88804146)(218.04999685,55.93622686)(218.04999685,58.02999878)
\closepath
}
}
{
\newrgbcolor{curcolor}{0 0 0}
\pscustom[linewidth=1,linecolor=curcolor]
{
\newpath
\moveto(218.04999685,58.02999878)
\curveto(218.04999685,60.1237707)(215.51872135,61.1719561)(214.03838049,59.69161524)
\curveto(212.55803963,58.21127437)(213.60622503,55.67999887)(215.69999695,55.67999887)
\curveto(217.79376887,55.67999887)(218.84195427,58.21127437)(217.3616134,59.69161524)
\curveto(215.88127254,61.1719561)(213.34999704,60.1237707)(213.34999704,58.02999878)
\curveto(213.34999704,55.93622686)(215.88127254,54.88804146)(217.3616134,56.36838232)
\curveto(218.84195427,57.84872318)(217.79376887,60.37999868)(215.69999695,60.37999868)
\curveto(213.60622503,60.37999868)(212.55803963,57.84872318)(214.03838049,56.36838232)
\curveto(215.51872135,54.88804146)(218.04999685,55.93622686)(218.04999685,58.02999878)
\closepath
}
}
{
\newrgbcolor{curcolor}{1 1 1}
\pscustom[linestyle=none,fillstyle=solid,fillcolor=curcolor]
{
\newpath
\moveto(199.29000235,131.94000244)
\curveto(199.29000235,134.03377436)(196.75872685,135.08195976)(195.27838599,133.6016189)
\curveto(193.79804512,132.12127804)(194.84623052,129.59000254)(196.94000244,129.59000254)
\curveto(199.03377436,129.59000254)(200.08195976,132.12127804)(198.6016189,133.6016189)
\curveto(197.12127804,135.08195976)(194.59000254,134.03377436)(194.59000254,131.94000244)
\curveto(194.59000254,129.84623052)(197.12127804,128.79804512)(198.6016189,130.27838599)
\curveto(200.08195976,131.75872685)(199.03377436,134.29000235)(196.94000244,134.29000235)
\curveto(194.84623052,134.29000235)(193.79804512,131.75872685)(195.27838599,130.27838599)
\curveto(196.75872685,128.79804512)(199.29000235,129.84623052)(199.29000235,131.94000244)
\closepath
}
}
{
\newrgbcolor{curcolor}{0 0 0}
\pscustom[linewidth=1,linecolor=curcolor]
{
\newpath
\moveto(199.29000235,131.94000244)
\curveto(199.29000235,134.03377436)(196.75872685,135.08195976)(195.27838599,133.6016189)
\curveto(193.79804512,132.12127804)(194.84623052,129.59000254)(196.94000244,129.59000254)
\curveto(199.03377436,129.59000254)(200.08195976,132.12127804)(198.6016189,133.6016189)
\curveto(197.12127804,135.08195976)(194.59000254,134.03377436)(194.59000254,131.94000244)
\curveto(194.59000254,129.84623052)(197.12127804,128.79804512)(198.6016189,130.27838599)
\curveto(200.08195976,131.75872685)(199.03377436,134.29000235)(196.94000244,134.29000235)
\curveto(194.84623052,134.29000235)(193.79804512,131.75872685)(195.27838599,130.27838599)
\curveto(196.75872685,128.79804512)(199.29000235,129.84623052)(199.29000235,131.94000244)
\closepath
}
}
{
\newrgbcolor{curcolor}{1 1 1}
\pscustom[linestyle=none,fillstyle=solid,fillcolor=curcolor]
{
\newpath
\moveto(154.35999441,95.48001099)
\curveto(154.35999441,97.57378291)(151.82871891,98.6219683)(150.34837805,97.14162744)
\curveto(148.86803719,95.66128658)(149.91622258,93.13001108)(152.00999451,93.13001108)
\curveto(154.10376643,93.13001108)(155.15195183,95.66128658)(153.67161096,97.14162744)
\curveto(152.1912701,98.6219683)(149.6599946,97.57378291)(149.6599946,95.48001099)
\curveto(149.6599946,93.38623906)(152.1912701,92.33805367)(153.67161096,93.81839453)
\curveto(155.15195183,95.29873539)(154.10376643,97.83001089)(152.00999451,97.83001089)
\curveto(149.91622258,97.83001089)(148.86803719,95.29873539)(150.34837805,93.81839453)
\curveto(151.82871891,92.33805367)(154.35999441,93.38623906)(154.35999441,95.48001099)
\closepath
}
}
{
\newrgbcolor{curcolor}{0 0 0}
\pscustom[linewidth=1,linecolor=curcolor]
{
\newpath
\moveto(154.35999441,95.48001099)
\curveto(154.35999441,97.57378291)(151.82871891,98.6219683)(150.34837805,97.14162744)
\curveto(148.86803719,95.66128658)(149.91622258,93.13001108)(152.00999451,93.13001108)
\curveto(154.10376643,93.13001108)(155.15195183,95.66128658)(153.67161096,97.14162744)
\curveto(152.1912701,98.6219683)(149.6599946,97.57378291)(149.6599946,95.48001099)
\curveto(149.6599946,93.38623906)(152.1912701,92.33805367)(153.67161096,93.81839453)
\curveto(155.15195183,95.29873539)(154.10376643,97.83001089)(152.00999451,97.83001089)
\curveto(149.91622258,97.83001089)(148.86803719,95.29873539)(150.34837805,93.81839453)
\curveto(151.82871891,92.33805367)(154.35999441,93.38623906)(154.35999441,95.48001099)
\closepath
}
}
{
\newrgbcolor{curcolor}{0 0 0}
\pscustom[linewidth=1,linecolor=curcolor]
{
\newpath
\moveto(267.73001099,217.5)
\lineto(267.73001099,86.08999634)
}
}
{
\newrgbcolor{curcolor}{0.60000002 0.60000002 0.60000002}
\pscustom[linewidth=1,linecolor=curcolor]
{
\newpath
\moveto(267.73001099,308.20000458)
\lineto(267.73001099,217.5)
}
}
{
\newrgbcolor{curcolor}{0 0 0}
\pscustom[linestyle=none,fillstyle=solid,fillcolor=curcolor]
{
\newpath
\moveto(270.08001089,308.20000458)
\curveto(270.08001089,310.2937765)(267.54873539,311.3419619)(266.06839453,309.86162103)
\curveto(264.58805367,308.38128017)(265.63623906,305.85000467)(267.73001099,305.85000467)
\curveto(269.82378291,305.85000467)(270.8719683,308.38128017)(269.39162744,309.86162103)
\curveto(267.91128658,311.3419619)(265.38001108,310.2937765)(265.38001108,308.20000458)
\curveto(265.38001108,306.10623266)(267.91128658,305.05804726)(269.39162744,306.53838812)
\curveto(270.8719683,308.01872898)(269.82378291,310.55000448)(267.73001099,310.55000448)
\curveto(265.63623906,310.55000448)(264.58805367,308.01872898)(266.06839453,306.53838812)
\curveto(267.54873539,305.05804726)(270.08001089,306.10623266)(270.08001089,308.20000458)
\closepath
}
}
{
\newrgbcolor{curcolor}{0 0 0}
\pscustom[linewidth=1,linecolor=curcolor]
{
\newpath
\moveto(270.08001089,308.20000458)
\curveto(270.08001089,310.2937765)(267.54873539,311.3419619)(266.06839453,309.86162103)
\curveto(264.58805367,308.38128017)(265.63623906,305.85000467)(267.73001099,305.85000467)
\curveto(269.82378291,305.85000467)(270.8719683,308.38128017)(269.39162744,309.86162103)
\curveto(267.91128658,311.3419619)(265.38001108,310.2937765)(265.38001108,308.20000458)
\curveto(265.38001108,306.10623266)(267.91128658,305.05804726)(269.39162744,306.53838812)
\curveto(270.8719683,308.01872898)(269.82378291,310.55000448)(267.73001099,310.55000448)
\curveto(265.63623906,310.55000448)(264.58805367,308.01872898)(266.06839453,306.53838812)
\curveto(267.54873539,305.05804726)(270.08001089,306.10623266)(270.08001089,308.20000458)
\closepath
}
}
\end{pspicture}

              \caption{光线-多边形相交测试可通过求射线与多边形平面交点、
                  将命中点和多边形顶点投影到轴对齐平面以及做2D点在多边形内测试来完成。}
              \label{fig:3.47}
          \end{figure}
    \item \circletwo \keyindex{体素构造表示}{constructive solid geometry}{}(CSG)是
          经典的实体建模技术,通过考虑诸多图元形状的并、交、差构建复杂形状。
          例如,如果一个形状建模为圆柱体和一组与之部分重合的球体的差,
          则球体可用于创建圆柱体上的凹坑。
          详见\citet{10.5555/74803}了解更多关于CSG的信息。
          向pbrt添加CSG支持并渲染展示可以用CSG渲染的有趣图像。
          你可能想阅读首次描述了光线追踪可以怎样用于渲染CSG描述的模型的\citet{ROTH1982109},
          以及讨论CSG光线追踪精度相关问题的\citet{10.5555/93267.93276}。
    \item \circletwo 程序化描述参数曲面:编写一个接收形如$f(u,v)\rightarrow(x,y,z)$的通用数学表达式
          将参数曲面描述为$(u,v)$函数的\refvar{Shape}{}。
          在网格位置$(u,v)$处求给定函数的值,并创建近似该给定曲面的三角网格。
    \item
\end{enumerate}