\section{变换}\label{sec:变换}

通常,\keyindex{变换}{transformation}{}$\bm T$是
从点到点和从向量到向量的映射:
\begin{align*}
    \bm p'=\bm T(\bm p), \  \bm v'=\bm T(\bm v)\, .
\end{align*}

变换$\bm T$可以是任意过程。
然而,本章我们考虑所有可能变换的一个子集。
特别地,它们是:
\begin{itemize}
    \item \keyindex{线性的}{linear}{}:对于任意线性变换$\bm T$和
          标量$s$,都有$\bm T(s\bm v)=s\bm T(\bm v)$以及
          $\bm T(\bm v_1+\bm v_2)=\bm T(\bm v_1)+\bm T(\bm v_2)$。
          这两点性质可极大简化关于变换的推导。
    \item \keyindex{连续的}{continuous}{}:简单说,
          $\bm T$把$\bm p$和$\bm v$的邻域映射为$\bm p'$和$\bm v'$的邻域。
    \item \keyindex{一对一}{one-to-one}{}且\keyindex{可逆}{invertible}{}:
          对于每个$\bm p$,$\bm T$把$\bm p$映射为唯一一个$\bm p'$。
          此外,存在一个逆变换$\bm T^{-1}$将$\bm p'$映射回$\bm p$。
\end{itemize}

我们经常想取某一坐标系下的点、向量或法线并求它在另一坐标系下的坐标值。
运用线性代数的基本性质,可以证明一个$4\times4$的矩阵
能表示点或向量从一个坐标系到另一个坐标系的线性变换。
此外,这样的$4\times4$矩阵足以表达固定坐标系内点和向量的所有线性变换,
例如空间中平移或绕一点旋转。
因此,有两种不同(且不兼容!)的方式解释矩阵:
\begin{itemize}
    \item \keyindex{坐标系的变换}{transformation of the frame}{}:给定一点,
          矩阵可表示如何计算同一坐标系下的\emph{新}点来代表对原始点的变换
          (例如朝某个方向平移)。
    \item \keyindex{从一个坐标系到另一个的变换}{transformation from one frame to another}{}:
          矩阵可依据一点或向量在原坐标系的坐标来表示其在新坐标系的坐标。
\end{itemize}

pbrt中所用的变换大多数是点从一个坐标系到另一个坐标系的变换。

通常,变换使得在最方便的坐标空间内工作成为可能。
例如,我们编写定义虚拟相机的例程,
假设相机位于原点,看向$z$轴,且$y$向上,$x$轴向右。
这些假设极大简化了相机实现。
然后,为了把相机置于场景中任意一点并看向任意方向,
我们只需要构造一个变换把场景坐标系统中的点映射到相机坐标系统中
(关于pbrt中相机坐标空间的更多信息详见\refsub{相机坐标空间})。

\subsection{齐次坐标}\label{sub:齐次坐标}
给定由$(\bm p_\mathrm{o},\bm v_1,\bm v_2,\bm v_3)$定义的坐标系,
具有相同坐标$(x,y,z)$的点$(p_x,p_y,p_z)$和向量$(v_x,v_y,v_z)$在表示上有歧义。
利用本章开头介绍的点和向量的表示,
我们可以把点写作内积$[s_1\ s_2\ s_3\ 1][\bm v_1\ \bm v_2\ \bm v_3\ \bm p_\mathrm{o}]^\mathrm{T}$,
把向量写作内积$[s'_1\ s'_2\ s'_3\ 0][\bm v_1\ \bm v_2\ \bm v_3\ \bm p_\mathrm{o}]^\mathrm{T}$。
这样有三个$s_i$以及一个零或一的四维向量称作点或向量的\keyindex{齐次}{homogeneous}{}表示。
齐次表示的第四个坐标有时称作\keyindex{权重}{weight}{}。
对于一个点,它的值可以是任意非零标量:
齐次点$[1,3,-2,1]$和$[-2,-6,4,-2]$描述了同一个笛卡尔点$(1,3,-2)$。
把齐次点转换为普通点需要用前三个分量除以权重
\sidenote{译者注:这里调整了原文对各种括号的混用。
    齐次坐标和矩阵等用方括号,一般坐标等用圆括号。}:
\begin{align*}
    [x,y,z,w]\rightarrow\left(\frac{x}{w},\frac{y}{w},\frac{z}{w}\right)\, .
\end{align*}

我们将利用这些事实来看看变换矩阵为什么可以描述
如何将一个坐标系下的点和向量映射到另一个坐标系。
考虑矩阵$\bm M$描述从一个坐标系到另一个坐标系的变换:
\begin{align*}
    M=\left[
        \begin{array}{cccc}
            m_{0,0} & m_{0,1} & m_{0,2} & m_{0,3} \\
            m_{1,0} & m_{1,1} & m_{1,2} & m_{1,3} \\
            m_{2,0} & m_{2,1} & m_{2,2} & m_{2,3} \\
            m_{3,0} & m_{3,1} & m_{3,2} & m_{3,3}
        \end{array}
        \right]\, .
\end{align*}

(本书中,我们定义矩阵元素时索引从零开始,这样公式和源码更能直接对应。)
若$\bm M$表示的变换被应用到$x$轴向量$(1,0,0)$上,我们有
\begin{align*}
    \bm M\bm x=\bm M[1\ 0\ 0\ 0]^\mathrm{T}=[m_{0,0}\ m_{1,0}\ m_{2,0}\ m_{3,0}]^\mathrm{T}\, .
\end{align*}

因此,直接阅读矩阵的列就能知道当前坐标系统的基向量和原点是怎样被矩阵变换的:
\begin{align*}
    \bm M\bm y & =[m_{0,1}\ m_{1,1}\ m_{2,1}\ m_{3,1}]^\mathrm{T}\, , \\
    \bm M\bm z & =[m_{0,2}\ m_{1,2}\ m_{2,2}\ m_{3,2}]^\mathrm{T}\, , \\
    \bm M\bm p & =[m_{0,3}\ m_{1,3}\ m_{2,3}\ m_{3,3}]^\mathrm{T}\, .
\end{align*}

一般通过表征基是如何变换的,
我们就能知道该基表示的任意指定点或向量是如何被变换的。
因为当前坐标系统的点和向量由当前坐标系表示,
直接对它们施加变换等价于对当前坐标系统的基施加变换并
用变换后的基求出坐标。

我们并不在代码中显式地使用齐次坐标;
pbrt中没有类{\ttfamily Homogeneous}。
然而,下节各种变换例程都隐式地将点、向量和法线转换为齐次形式,
变换齐次点,再转换回来返回结果。
这样就在一个地方(即变换的实现)隔离了齐次坐标的细节。
\begin{lstlisting}
`\initcode{Transform Declarations}{=}`
class `\initvar{Transform}{}` {
public:
    `\refcode{Transform Public Methods}{}`
private:
    `\refcode{Transform Private Data}{}`
};
\end{lstlisting}

变换由矩阵\refvar[Transform::m]{m}{},即一个\refvar{Matrix4x4}{}对象的元素表示。
底层类\refvar{Matrix4x4}{}定义在\refsub{4x4矩阵}。
矩阵\refvar[Transform::m]{m}{}按\keyindex{行优先}{row-major}{}形式存储,
所以元素{\ttfamily m[i][j]}对应$m_{i,j}$,
其中$i$是行数,$j$是列数。
为了方便,\refvar{Transform}{}还存储了
矩阵\refvar[Transform::m]{m}{}的逆
\refvar{Transform::mInv}{}成员;
因为pbrt的需要,易获取的逆比按需重复计算更好。

变换的这种表示相对更耗内存:
假设存储一个\refvar{Float}{}值需要4字节,一个
\refvar{Transform}{}就需要128字节存储。
幼稚地使用该方法会造成浪费;
如果一个场景有几百万个形状但仅有几千个不同的变换,
就更没理由在内存中冗余地多次保存同一个变换。
因此pbrt中的\refvar{Shape}{}保存的是\refvar{Transform}{}的指针,
\refsub{形状}定义的场景指定代码使用\refvar{TransformCache}{}来保证
所有共享同一变换的形状指向内存中该变换的单一实例。

这个共享变换的决定意味着丧失了灵活性,然而:
如果\refvar{Transform}{}被场景中的多个物体(以及不希望它改变的对象)共享,
则\refvar{Transform}{}的元素在创建后不能被修改。
现实中这点限制不是问题,
因为场景中的变换通常是在pbrt解析场景描述文件时创建的,
之后在渲染时不需要改变。
\begin{lstlisting}
`\initcode{Transform Private Data}{=}`
`\refvar{Matrix4x4}{}` `\initvar[Transform::m]{m}{}`, `\initvar[Transform::mInv]{mInv}{}`;
\end{lstlisting}

\subsection{基本运算}\label{sub:基本运算}
当创建新的\refvar{Transform}{}时,
它默认为\keyindex{恒等变换}{identity transformation}{transformation变换}——
将每个点和向量映射为它自己的变换。
该变换由\keyindex{单位矩阵}{identity matrix}{}表示:
\begin{align*}
    \bm I=\left[
        \begin{array}{cccc}
            1 & 0 & 0 & 0 \\
            0 & 1 & 0 & 0 \\
            0 & 0 & 1 & 0 \\
            0 & 0 & 0 & 1
        \end{array}
        \right]\, .
\end{align*}

这里的实现依赖\refvar{Matrix4x4}{}默认构造函数
把\refvar[Transform::m]{m}{}和\refvar[Transform::mInv]{mInv}{}填充为单位矩阵。
\begin{lstlisting}
`\initcode{Transform Public Methods}{=}\initnext{TransformPublicMethods}`
`\refvar{Transform}{}`() { }
\end{lstlisting}

\refvar{Transform}{}也可以从给定矩阵创建。
这时,给定矩阵必须是显式可逆的。
\begin{lstlisting}
`\refcode{Transform Public Methods}{+=}\lastnext{TransformPublicMethods}`
`\refvar{Transform}{}`(const `\refvar{Float}{}` mat[4][4]) {
    `\refvar[Transform::m]{m}{}` = `\refvar{Matrix4x4}{}`(mat[0][0], mat[0][1], mat[0][2], mat[0][3],
                  mat[1][0], mat[1][1], mat[1][2], mat[1][3],
                  mat[2][0], mat[2][1], mat[2][2], mat[2][3],
                  mat[3][0], mat[3][1], mat[3][2], mat[3][3]);
    `\refvar[Transform::mInv]{mInv}{}` = `\refvar[Matrix4x4::Inverse]{Inverse}{}`(`\refvar[Transform::m]{m}{}`);
}
\end{lstlisting}

\begin{lstlisting}
`\refcode{Transform Public Methods}{+=}\lastnext{TransformPublicMethods}`
`\refvar{Transform}{}`(const `\refvar{Matrix4x4}{}` &m) : `\refvar[Transform::m]{m}{}`(m), `\refvar[Transform::mInv]{mInv}{}`(`\refvar[Matrix4x4::Inverse]{Inverse}{}`(m)) { }
\end{lstlisting}

最常用的构造函数取变换矩阵的引用以及显式提供的逆。
在构造函数中计算逆是一种很好的方法,
因为许多几何变换有很简单的逆,
我们可以避免由于计算一般$4\times4$矩阵的逆
而造成的开销和潜在的数值精度损失。
\begin{lstlisting}
`\refcode{Transform Public Methods}{+=}\lastnext{TransformPublicMethods}`
`\refvar{Transform}{}`(const `\refvar{Matrix4x4}{}` &m, const `\refvar{Matrix4x4}{}` &mInv) 
   : `\refvar[Transform::m]{m}{}`(m), `\refvar[Transform::mInv]{mInv}{}`(mInv) {
}
\end{lstlisting}

\refvar{Transform}{}表示\refvar{Transform}{}的逆只需
通过交换\refvar[Transform::mInv]{mInv}{}和\refvar[Transform::m]{m}{}并返回。
\begin{lstlisting}
`\refcode{Transform Public Methods}{+=}\lastnext{TransformPublicMethods}`
friend `\refvar{Transform}{}` `\initvar[Transform::Inverse]{Inverse}{}`(const `\refvar{Transform}{}` &t) {
    return `\refvar{Transform}{}`(t.`\refvar[Transform::mInv]{mInv}{}`, t.`\refvar[Transform::m]{m}{}`);
}
\end{lstlisting}

转置变换中的两个矩阵以计算新变换也很有用。
\begin{lstlisting}
`\refcode{Transform Public Methods}{+=}\lastnext{TransformPublicMethods}`
friend `\refvar{Transform}{}` `\initvar[Transform::Transpose]{Transpose}{}`(const `\refvar{Transform}{}` &t) {
    return `\refvar{Transform}{}`(`\refvar[Matrix4x4::Transpose]{Transpose}{}`(t.`\refvar[Transform::m]{m}{}`), `\refvar[Matrix4x4::Transpose]{Transpose}{}`(t.`\refvar[Transform::mInv]{mInv}{}`));
}
\end{lstlisting}

我们提供了\refvar{Transform}{}相等(和不相等)测试方法;
它们的实现很简单就不介绍了\sidenote{译者注:即逐元素比较。}。
\refvar{Transform}{}也提供了方法{\ttfamily IsIdentity()}检查变换是否是恒等的。

\subsection{平移}\label{sub:平移}
最简单的变换之一就是\keyindex{平移变换}{translation transformation}{transformation变换},
$\bm T(\Delta x,\Delta y, \Delta z)$。
当施加到一点$\bm p$上时,它将$\bm p$的坐标平移$\Delta x,\Delta y$和$\Delta z$,
如\reffig{2.10}所示。例如,
\begin{figure}[htbp]
    \centering%LaTeX with PSTricks extensions
%%Creator: Inkscape 1.0.1 (3bc2e813f5, 2020-09-07)
%%Please note this file requires PSTricks extensions
\psset{xunit=.5pt,yunit=.5pt,runit=.5pt}
\begin{pspicture}(200.08000183,227.47000122)
{
\newrgbcolor{curcolor}{0 0 0}
\pscustom[linewidth=1,linecolor=curcolor]
{
\newpath
\moveto(5.5,160.20000122)
\lineto(5.5,6.16000122)
\lineto(160.1,6.16000122)
}
}
{
\newrgbcolor{curcolor}{0 0 0}
\pscustom[linestyle=none,fillstyle=solid,fillcolor=curcolor]
{
\newpath
\moveto(0,155.29000122)
\lineto(5.5,159.55000122)
\lineto(11.01,155.29000122)
\lineto(5.5,168.30000122)
\closepath
}
}
{
\newrgbcolor{curcolor}{0.65098041 0.65098041 0.65098041}
\pscustom[linestyle=none,fillstyle=solid,fillcolor=curcolor]
{
\newpath
\moveto(1.2,156.84000122)
\lineto(5.5,166.99000122)
\lineto(5.5,160.18000122)
\closepath
}
}
{
\newrgbcolor{curcolor}{0.40000001 0.40000001 0.40000001}
\pscustom[linestyle=none,fillstyle=solid,fillcolor=curcolor]
{
\newpath
\moveto(9.8,156.84000122)
\lineto(5.5,166.99000122)
\lineto(5.5,160.18000122)
\closepath
}
}
{
\newrgbcolor{curcolor}{0 0 0}
\pscustom[linestyle=none,fillstyle=solid,fillcolor=curcolor]
{
\newpath
\moveto(155.19,0.66000122)
\lineto(159.45,6.16000122)
\lineto(155.19,11.67000122)
\lineto(168.21,6.16000122)
\closepath
}
}
{
\newrgbcolor{curcolor}{0.65098041 0.65098041 0.65098041}
\pscustom[linestyle=none,fillstyle=solid,fillcolor=curcolor]
{
\newpath
\moveto(156.75,1.86000122)
\lineto(166.89,6.16000122)
\lineto(160.08,6.16000122)
\closepath
}
}
{
\newrgbcolor{curcolor}{0.40000001 0.40000001 0.40000001}
\pscustom[linestyle=none,fillstyle=solid,fillcolor=curcolor]
{
\newpath
\moveto(156.75,10.46000122)
\lineto(166.89,6.16000122)
\lineto(160.08,6.16000122)
\closepath
}
}
{
\newrgbcolor{curcolor}{0 0 0}
\pscustom[linestyle=none,fillstyle=solid,fillcolor=curcolor]
{
\newpath
\moveto(84.42999887,79.00999451)
\curveto(84.42999887,84.55180789)(77.73019729,87.32615393)(73.81201848,83.40797512)
\curveto(69.89383966,79.4897963)(72.6681857,72.78999472)(78.20999908,72.78999472)
\curveto(83.75181247,72.78999472)(86.52615851,79.4897963)(82.60797969,83.40797512)
\curveto(78.68980087,87.32615393)(71.98999929,84.55180789)(71.98999929,79.00999451)
\curveto(71.98999929,73.46818112)(78.68980087,70.69383508)(82.60797969,74.6120139)
\curveto(86.52615851,78.53019272)(83.75181247,85.2299943)(78.20999908,85.2299943)
\curveto(72.6681857,85.2299943)(69.89383966,78.53019272)(73.81201848,74.6120139)
\curveto(77.73019729,70.69383508)(84.42999887,73.46818112)(84.42999887,79.00999451)
\closepath
}
}
{
\newrgbcolor{curcolor}{0 0 0}
\pscustom[linewidth=1,linecolor=curcolor,linestyle=dashed,dash=4]
{
\newpath
\moveto(78.21,79.01000122)
\lineto(166.54,79.01000122)
\lineto(166.54,214.43000122)
}
}
{
\newrgbcolor{curcolor}{0 0 0}
\pscustom[linestyle=none,fillstyle=solid,fillcolor=curcolor]
{
\newpath
\moveto(173.05999613,212.30000114)
\curveto(173.05999613,217.84181453)(166.36019455,220.61616057)(162.44201573,216.69798175)
\curveto(158.52383691,212.77980293)(161.29818296,206.08000135)(166.83999634,206.08000135)
\curveto(172.38180972,206.08000135)(175.15615576,212.77980293)(171.23797695,216.69798175)
\curveto(167.31979813,220.61616057)(160.61999655,217.84181453)(160.61999655,212.30000114)
\curveto(160.61999655,206.75818776)(167.31979813,203.98384172)(171.23797695,207.90202054)
\curveto(175.15615576,211.82019935)(172.38180972,218.52000093)(166.83999634,218.52000093)
\curveto(161.29818296,218.52000093)(158.52383691,211.82019935)(162.44201573,207.90202054)
\curveto(166.36019455,203.98384172)(173.05999613,206.75818776)(173.05999613,212.30000114)
\closepath
}
}
{
\newrgbcolor{curcolor}{0 0 0}
\pscustom[linestyle=none,fillstyle=solid,fillcolor=curcolor]
{
\newpath
\moveto(72.01400587,53.52381757)
\curveto(71.94561637,53.21606481)(71.91142162,53.18187006)(71.87722687,53.14767531)
\curveto(71.77464261,53.11348056)(71.53527935,53.11348056)(71.33011084,53.11348056)
\curveto(70.95396857,53.11348056)(70.54363156,53.11348056)(70.54363156,52.49797503)
\curveto(70.54363156,52.25861177)(70.74880006,52.08763801)(70.98816333,52.08763801)
\curveto(71.60366885,52.08763801)(72.32175864,52.15602751)(72.97145892,52.15602751)
\curveto(73.7579382,52.15602751)(74.57861224,52.08763801)(75.33089677,52.08763801)
\curveto(75.46767578,52.08763801)(75.8780128,52.08763801)(75.8780128,52.73733829)
\curveto(75.8780128,53.11348056)(75.53606528,53.11348056)(75.33089677,53.11348056)
\curveto(75.02314401,53.11348056)(74.64700174,53.11348056)(74.37344373,53.14767531)
\lineto(75.33089677,56.94329273)
\curveto(75.63864954,56.63553997)(76.35673932,56.15681344)(77.51936087,56.15681344)
\curveto(81.3149783,56.15681344)(83.67441616,59.61048335)(83.67441616,62.58542674)
\curveto(83.67441616,65.28681211)(81.65692581,66.17587565)(79.84460398,66.17587565)
\curveto(78.30584016,66.17587565)(77.17741336,65.32100686)(76.83546584,65.0132541)
\curveto(75.98059705,66.17587565)(74.54441749,66.17587565)(74.30505423,66.17587565)
\curveto(73.51857494,66.17587565)(72.86887466,65.73134388)(72.42434289,64.9448646)
\curveto(71.87722687,64.05580106)(71.5694741,62.8931795)(71.5694741,62.79059525)
\curveto(71.5694741,62.48284248)(71.91142162,62.48284248)(72.11659013,62.48284248)
\curveto(72.35595339,62.48284248)(72.42434289,62.48284248)(72.52692715,62.58542674)
\curveto(72.59531665,62.61962149)(72.59531665,62.68801099)(72.73209566,63.23512702)
\curveto(73.14243267,64.97905935)(73.65535395,65.38939637)(74.20246997,65.38939637)
\curveto(74.44183323,65.38939637)(74.71539125,65.32100686)(74.71539125,64.60291708)
\curveto(74.71539125,64.26096956)(74.64700174,63.9532168)(74.57861224,63.64546404)
\closepath
\moveto(77.04063435,63.98741155)
\curveto(77.65613988,64.73969609)(78.68198243,65.38939637)(79.74201973,65.38939637)
\curveto(81.10980979,65.38939637)(81.21239404,64.22677481)(81.21239404,63.74804829)
\curveto(81.21239404,62.61962149)(80.46010951,59.91823612)(80.11816199,59.06336733)
\curveto(79.43426696,57.49040876)(78.37422966,56.94329273)(77.48516612,56.94329273)
\curveto(76.18576556,56.94329273)(75.67284429,57.96913528)(75.67284429,58.20849854)
\lineto(75.70703904,58.5162513)
\closepath
\moveto(77.04063435,63.98741155)
}
}
{
\newrgbcolor{curcolor}{0 0 0}
\pscustom[linestyle=none,fillstyle=solid,fillcolor=curcolor]
{
\newpath
\moveto(176.98786387,203.48647417)
\curveto(176.91947437,203.17872141)(176.88527962,203.14452666)(176.85108487,203.11033191)
\curveto(176.74850061,203.07613716)(176.50913735,203.07613716)(176.30396884,203.07613716)
\curveto(175.92782657,203.07613716)(175.51748956,203.07613716)(175.51748956,202.46063163)
\curveto(175.51748956,202.22126837)(175.72265806,202.05029461)(175.96202133,202.05029461)
\curveto(176.57752685,202.05029461)(177.29561664,202.11868411)(177.94531692,202.11868411)
\curveto(178.7317962,202.11868411)(179.55247024,202.05029461)(180.30475477,202.05029461)
\curveto(180.44153378,202.05029461)(180.8518708,202.05029461)(180.8518708,202.69999489)
\curveto(180.8518708,203.07613716)(180.50992328,203.07613716)(180.30475477,203.07613716)
\curveto(179.99700201,203.07613716)(179.62085974,203.07613716)(179.34730173,203.11033191)
\lineto(180.30475477,206.90594933)
\curveto(180.61250754,206.59819657)(181.33059732,206.11947004)(182.49321887,206.11947004)
\curveto(186.2888363,206.11947004)(188.64827416,209.57313995)(188.64827416,212.54808334)
\curveto(188.64827416,215.24946871)(186.63078381,216.13853225)(184.81846198,216.13853225)
\curveto(183.27969816,216.13853225)(182.15127136,215.28366346)(181.80932384,214.9759107)
\curveto(180.95445505,216.13853225)(179.51827549,216.13853225)(179.27891223,216.13853225)
\curveto(178.49243294,216.13853225)(177.84273266,215.69400048)(177.39820089,214.9075212)
\curveto(176.85108487,214.01845766)(176.5433321,212.8558361)(176.5433321,212.75325185)
\curveto(176.5433321,212.44549908)(176.88527962,212.44549908)(177.09044813,212.44549908)
\curveto(177.32981139,212.44549908)(177.39820089,212.44549908)(177.50078515,212.54808334)
\curveto(177.56917465,212.58227809)(177.56917465,212.65066759)(177.70595366,213.19778362)
\curveto(178.11629067,214.94171595)(178.62921195,215.35205297)(179.17632797,215.35205297)
\curveto(179.41569123,215.35205297)(179.68924925,215.28366346)(179.68924925,214.56557368)
\curveto(179.68924925,214.22362616)(179.62085974,213.9158734)(179.55247024,213.60812064)
\closepath
\moveto(182.01449235,213.95006815)
\curveto(182.62999788,214.70235269)(183.65584043,215.35205297)(184.71587773,215.35205297)
\curveto(186.08366779,215.35205297)(186.18625204,214.18943141)(186.18625204,213.71070489)
\curveto(186.18625204,212.58227809)(185.43396751,209.88089272)(185.09201999,209.02602393)
\curveto(184.40812496,207.45306536)(183.34808766,206.90594933)(182.45902412,206.90594933)
\curveto(181.15962356,206.90594933)(180.64670229,207.93179188)(180.64670229,208.17115514)
\lineto(180.68089704,208.4789079)
\closepath
\moveto(182.01449235,213.95006815)
}
}
{
\newrgbcolor{curcolor}{0 0 0}
\pscustom[linestyle=none,fillstyle=solid,fillcolor=curcolor]
{
\newpath
\moveto(193.37289352,222.50887422)
\curveto(193.50967253,222.74823748)(193.50967253,222.88501648)(193.50967253,222.98760074)
\curveto(193.50967253,223.46632726)(193.09933551,223.80827478)(192.62060899,223.80827478)
\curveto(192.03929821,223.80827478)(191.86832445,223.32954825)(191.79993495,223.09018499)
\lineto(189.78244461,216.49059794)
\curveto(189.74824986,216.45640319)(189.67986035,216.28542943)(189.67986035,216.25123468)
\curveto(189.67986035,216.08026092)(190.15858687,215.90928717)(190.29536588,215.90928717)
\curveto(190.39795013,215.90928717)(190.39795013,215.94348192)(190.50053439,216.18284518)
\closepath
\moveto(193.37289352,222.50887422)
}
}
{
\newrgbcolor{curcolor}{0 0 0}
\pscustom[linestyle=none,fillstyle=solid,fillcolor=curcolor]
{
\newpath
\moveto(178.80189615,9.25652985)
\curveto(178.93867516,9.80364588)(179.45159643,11.82113622)(180.9561655,11.82113622)
\curveto(181.05874976,11.82113622)(181.60586578,11.82113622)(182.05039755,11.54757821)
\curveto(181.43489202,11.4107992)(181.02455501,10.89787793)(181.02455501,10.3507619)
\curveto(181.02455501,10.00881439)(181.26391827,9.59847737)(181.84522904,9.59847737)
\curveto(182.32395556,9.59847737)(183.0078506,9.97461964)(183.0078506,10.86368318)
\curveto(183.0078506,11.99210998)(181.74264479,12.29986274)(180.99036025,12.29986274)
\curveto(179.72515445,12.29986274)(178.97286991,11.13724119)(178.6993119,10.65851467)
\curveto(178.15219587,12.09469423)(176.98957432,12.29986274)(176.33987404,12.29986274)
\curveto(174.08302044,12.29986274)(172.81781463,9.49589311)(172.81781463,8.94877709)
\curveto(172.81781463,8.70941383)(173.05717789,8.70941383)(173.09137264,8.70941383)
\curveto(173.2623464,8.70941383)(173.3307359,8.77780333)(173.36493066,8.94877709)
\curveto(174.11721519,11.2740202)(175.55339476,11.82113622)(176.30567929,11.82113622)
\curveto(176.71601631,11.82113622)(177.46830084,11.61596771)(177.46830084,10.3507619)
\curveto(177.46830084,9.66686687)(177.09215858,8.23068731)(176.30567929,5.15315967)
\curveto(175.96373177,3.81956436)(175.17725249,2.89630606)(174.21979944,2.89630606)
\curveto(174.08302044,2.89630606)(173.60429392,2.89630606)(173.12556739,3.16986408)
\curveto(173.67268342,3.30664308)(174.15140994,3.75117485)(174.15140994,4.36668038)
\curveto(174.15140994,4.94799116)(173.67268342,5.11896491)(173.36493066,5.11896491)
\curveto(172.68103562,5.11896491)(172.16811435,4.57184889)(172.16811435,3.85375911)
\curveto(172.16811435,2.86211131)(173.22815165,2.41757954)(174.18560469,2.41757954)
\curveto(175.65597901,2.41757954)(176.4424583,3.95634336)(176.47665305,4.05892762)
\curveto(176.75021106,3.27244833)(177.53669035,2.41757954)(178.83609091,2.41757954)
\curveto(181.09294451,2.41757954)(182.32395556,5.22154917)(182.32395556,5.76866519)
\curveto(182.32395556,6.00802846)(182.15298181,6.00802846)(182.0845923,6.00802846)
\curveto(181.87942379,6.00802846)(181.84522904,5.9054442)(181.77683954,5.76866519)
\curveto(181.05874976,3.40922734)(179.58837544,2.89630606)(178.90448041,2.89630606)
\curveto(178.04961162,2.89630606)(177.7076641,3.58020109)(177.7076641,4.33248563)
\curveto(177.7076641,4.81121215)(177.81024836,5.28993867)(178.04961162,6.24739172)
\closepath
\moveto(178.80189615,9.25652985)
}
}
{
\newrgbcolor{curcolor}{0 0 0}
\pscustom[linestyle=none,fillstyle=solid,fillcolor=curcolor]
{
\newpath
\moveto(11.63492206,186.97506843)
\curveto(11.73750631,187.2828212)(11.73750631,187.31701595)(11.73750631,187.48798971)
\curveto(11.73750631,187.86413197)(11.42975355,188.06930048)(11.08780603,188.06930048)
\curveto(10.88263752,188.06930048)(10.54069001,187.93252148)(10.3355215,187.62476871)
\curveto(10.30132674,187.48798971)(10.09615824,186.83828943)(10.02776873,186.42795241)
\curveto(9.85679497,185.88083638)(9.72001597,185.26533085)(9.58323696,184.68402008)
\lineto(8.59158917,180.75162365)
\curveto(8.52319966,180.44387088)(7.56574662,178.90510706)(6.12956705,178.90510706)
\curveto(5.035335,178.90510706)(4.79597174,179.86256011)(4.79597174,180.68323414)
\curveto(4.79597174,181.67488194)(5.17211401,183.042672)(5.89020379,184.95757809)
\curveto(6.23215131,185.84664163)(6.33473556,186.08600489)(6.33473556,186.53053666)
\curveto(6.33473556,187.48798971)(5.65084053,188.30866374)(4.55660848,188.30866374)
\curveto(2.47072864,188.30866374)(1.68424935,185.12855185)(1.68424935,184.95757809)
\curveto(1.68424935,184.71821483)(1.88941786,184.71821483)(1.92361261,184.71821483)
\curveto(2.16297587,184.71821483)(2.16297587,184.78660433)(2.26556013,185.12855185)
\curveto(2.88106566,187.18023694)(3.73593444,187.82993722)(4.48821898,187.82993722)
\curveto(4.65919274,187.82993722)(5.035335,187.82993722)(5.035335,187.14604219)
\curveto(5.035335,186.59892616)(4.79597174,186.01761539)(4.65919274,185.60727837)
\curveto(3.7701292,183.28203526)(3.39398693,182.05102421)(3.39398693,181.02518166)
\curveto(3.39398693,179.07608082)(4.76177699,178.42638054)(6.06117755,178.42638054)
\curveto(6.91604634,178.42638054)(7.63413612,178.80252281)(8.24964165,179.41802834)
\curveto(7.97608364,178.28960154)(7.70252563,177.19536949)(6.84765684,176.03274793)
\curveto(6.26634606,175.31465815)(5.44567202,174.66495787)(4.45402423,174.66495787)
\curveto(4.14627146,174.66495787)(3.15462367,174.73334737)(2.7784814,175.58821616)
\curveto(3.12042892,175.58821616)(3.42818168,175.58821616)(3.70173969,175.86177417)
\curveto(3.94110295,176.03274793)(4.14627146,176.3405007)(4.14627146,176.75083771)
\curveto(4.14627146,177.43473275)(3.56496069,177.50312225)(3.35979218,177.50312225)
\curveto(2.8468709,177.50312225)(2.12878112,177.16117473)(2.12878112,176.10113744)
\curveto(2.12878112,175.00690539)(3.08623416,174.18623135)(4.45402423,174.18623135)
\curveto(6.67668308,174.18623135)(8.93353668,176.16952694)(9.54904221,178.63154905)
\closepath
\moveto(11.63492206,186.97506843)
}
}
{
\newrgbcolor{curcolor}{0 0 0}
\pscustom[linestyle=none,fillstyle=solid,fillcolor=curcolor]
{
\newpath
\moveto(178.00422101,145.34186325)
\curveto(177.86744201,145.64961601)(177.7990525,145.75220027)(177.42291024,145.75220027)
\curveto(177.08096272,145.75220027)(177.01257322,145.64961601)(176.87579421,145.34186325)
\lineto(169.48972787,130.56973057)
\curveto(169.38714362,130.36456206)(169.38714362,130.33036731)(169.38714362,130.29617256)
\curveto(169.38714362,130.1251988)(169.52392262,130.1251988)(169.86587014,130.1251988)
\lineto(185.01414508,130.1251988)
\curveto(185.3560926,130.1251988)(185.4928716,130.1251988)(185.4928716,130.29617256)
\curveto(185.4928716,130.33036731)(185.4928716,130.36456206)(185.39028735,130.56973057)
\closepath
\moveto(176.7390152,143.56373617)
\lineto(182.62051247,131.76654688)
\lineto(170.85751794,131.76654688)
\closepath
\moveto(176.7390152,143.56373617)
}
}
{
\newrgbcolor{curcolor}{0 0 0}
\pscustom[linestyle=none,fillstyle=solid,fillcolor=curcolor]
{
\newpath
\moveto(197.13029059,138.43452343)
\curveto(197.23287485,138.7422762)(197.23287485,138.77647095)(197.23287485,138.94744471)
\curveto(197.23287485,139.32358697)(196.92512208,139.52875548)(196.58317457,139.52875548)
\curveto(196.37800606,139.52875548)(196.03605854,139.39197648)(195.83089003,139.08422371)
\curveto(195.79669528,138.94744471)(195.59152677,138.29774443)(195.52313727,137.88740741)
\curveto(195.35216351,137.34029138)(195.2153845,136.72478585)(195.0786055,136.14347508)
\lineto(194.0869577,132.21107865)
\curveto(194.0185682,131.90332588)(193.06111516,130.36456206)(191.62493559,130.36456206)
\curveto(190.53070354,130.36456206)(190.29134028,131.32201511)(190.29134028,132.14268914)
\curveto(190.29134028,133.13433694)(190.66748255,134.502127)(191.38557233,136.41703309)
\curveto(191.72751984,137.30609663)(191.8301041,137.54545989)(191.8301041,137.98999166)
\curveto(191.8301041,138.94744471)(191.14620907,139.76811874)(190.05197702,139.76811874)
\curveto(187.96609717,139.76811874)(187.17961789,136.58800685)(187.17961789,136.41703309)
\curveto(187.17961789,136.17766983)(187.3847864,136.17766983)(187.41898115,136.17766983)
\curveto(187.65834441,136.17766983)(187.65834441,136.24605933)(187.76092866,136.58800685)
\curveto(188.37643419,138.63969194)(189.23130298,139.28939222)(189.98358751,139.28939222)
\curveto(190.15456127,139.28939222)(190.53070354,139.28939222)(190.53070354,138.60549719)
\curveto(190.53070354,138.05838116)(190.29134028,137.47707039)(190.15456127,137.06673337)
\curveto(189.26549773,134.74149026)(188.88935546,133.51047921)(188.88935546,132.48463666)
\curveto(188.88935546,130.53553582)(190.25714553,129.88583554)(191.55654609,129.88583554)
\curveto(192.41141488,129.88583554)(193.12950466,130.26197781)(193.74501019,130.87748334)
\curveto(193.47145217,129.74905654)(193.19789416,128.65482449)(192.34302537,127.49220293)
\curveto(191.7617146,126.77411315)(190.94104056,126.12441287)(189.94939276,126.12441287)
\curveto(189.64164,126.12441287)(188.6499922,126.19280237)(188.27384994,127.04767116)
\curveto(188.61579745,127.04767116)(188.92355022,127.04767116)(189.19710823,127.32122917)
\curveto(189.43647149,127.49220293)(189.64164,127.7999557)(189.64164,128.21029271)
\curveto(189.64164,128.89418775)(189.06032922,128.96257725)(188.85516071,128.96257725)
\curveto(188.34223944,128.96257725)(187.62414966,128.62062973)(187.62414966,127.56059244)
\curveto(187.62414966,126.46636039)(188.5816027,125.64568635)(189.94939276,125.64568635)
\curveto(192.17205161,125.64568635)(194.42890522,127.62898194)(195.04441075,130.09100405)
\closepath
\moveto(197.13029059,138.43452343)
}
}
{
\newrgbcolor{curcolor}{0 0 0}
\pscustom[linestyle=none,fillstyle=solid,fillcolor=curcolor]
{
\newpath
\moveto(123.93874001,70.36053775)
\curveto(123.80196101,70.66829051)(123.7335715,70.77087477)(123.35742924,70.77087477)
\curveto(123.01548172,70.77087477)(122.94709222,70.66829051)(122.81031321,70.36053775)
\lineto(115.42424687,55.58840507)
\curveto(115.32166262,55.38323656)(115.32166262,55.34904181)(115.32166262,55.31484706)
\curveto(115.32166262,55.1438733)(115.45844162,55.1438733)(115.80038914,55.1438733)
\lineto(130.94866408,55.1438733)
\curveto(131.2906116,55.1438733)(131.4273906,55.1438733)(131.4273906,55.31484706)
\curveto(131.4273906,55.34904181)(131.4273906,55.38323656)(131.32480635,55.58840507)
\closepath
\moveto(122.6735342,68.58241067)
\lineto(128.55503147,56.78522138)
\lineto(116.79203694,56.78522138)
\closepath
\moveto(122.6735342,68.58241067)
}
}
{
\newrgbcolor{curcolor}{0 0 0}
\pscustom[linestyle=none,fillstyle=solid,fillcolor=curcolor]
{
\newpath
\moveto(139.74791869,61.74346035)
\curveto(139.8846977,62.29057638)(140.39761897,64.30806672)(141.90218804,64.30806672)
\curveto(142.00477229,64.30806672)(142.55188832,64.30806672)(142.99642009,64.03450871)
\curveto(142.38091456,63.8977297)(141.97057754,63.38480843)(141.97057754,62.8376924)
\curveto(141.97057754,62.49574489)(142.2099408,62.08540787)(142.79125158,62.08540787)
\curveto(143.2699781,62.08540787)(143.95387313,62.46155014)(143.95387313,63.35061368)
\curveto(143.95387313,64.47904048)(142.68866732,64.78679324)(141.93638279,64.78679324)
\curveto(140.67117698,64.78679324)(139.91889245,63.62417169)(139.64533444,63.14544517)
\curveto(139.09821841,64.58162473)(137.93559686,64.78679324)(137.28589658,64.78679324)
\curveto(135.02904297,64.78679324)(133.76383717,61.98282361)(133.76383717,61.43570759)
\curveto(133.76383717,61.19634433)(134.00320043,61.19634433)(134.03739518,61.19634433)
\curveto(134.20836894,61.19634433)(134.27675844,61.26473383)(134.31095319,61.43570759)
\curveto(135.06323773,63.7609507)(136.49941729,64.30806672)(137.25170183,64.30806672)
\curveto(137.66203884,64.30806672)(138.41432338,64.10289821)(138.41432338,62.8376924)
\curveto(138.41432338,62.15379737)(138.03818111,60.71761781)(137.25170183,57.64009017)
\curveto(136.90975431,56.30649486)(136.12327502,55.38323656)(135.16582198,55.38323656)
\curveto(135.02904297,55.38323656)(134.55031645,55.38323656)(134.07158993,55.65679458)
\curveto(134.61870596,55.79357358)(135.09743248,56.23810535)(135.09743248,56.85361088)
\curveto(135.09743248,57.43492166)(134.61870596,57.60589541)(134.31095319,57.60589541)
\curveto(133.62705816,57.60589541)(133.11413689,57.05877939)(133.11413689,56.34068961)
\curveto(133.11413689,55.34904181)(134.17417419,54.90451004)(135.13162723,54.90451004)
\curveto(136.60200155,54.90451004)(137.38848083,56.44327386)(137.42267558,56.54585812)
\curveto(137.6962336,55.75937883)(138.48271288,54.90451004)(139.78211344,54.90451004)
\curveto(142.03896704,54.90451004)(143.2699781,57.70847967)(143.2699781,58.25559569)
\curveto(143.2699781,58.49495896)(143.09900434,58.49495896)(143.03061484,58.49495896)
\curveto(142.82544633,58.49495896)(142.79125158,58.3923747)(142.72286208,58.25559569)
\curveto(142.00477229,55.89615784)(140.53439798,55.38323656)(139.85050294,55.38323656)
\curveto(138.99563416,55.38323656)(138.65368664,56.06713159)(138.65368664,56.81941613)
\curveto(138.65368664,57.29814265)(138.75627089,57.77686917)(138.99563416,58.73432222)
\closepath
\moveto(139.74791869,61.74346035)
}
}
\end{pspicture}

    \caption{2D中的平移。分别将偏移量$\Delta x$和$\Delta y$添加到点的坐标改变其在空间中的位置。}
    \label{fig:2.10}
\end{figure}

平移有一些简单的性质:
\begin{align*}
    \bm T(0,0,0)                         & =\bm I\, ,                                \\
    \bm T(x_1,y_1,z_1)\bm T(x_2,y_2,z_2) & =\bm T(x_1+x_2,y_1+y_2,z_1+z_2)\, ,       \\
    \bm T(x_1,y_1,z_1)\bm T(x_2,y_2,z_2) & =\bm T(x_2,y_2,z_2)\bm T(x_1,y_1,z_1)\, , \\
    \bm T^{-1}(x,y,z)                    & =\bm T(-x,-y,-z)\, .
\end{align*}

平移只影响点,不改变向量。平移变换的矩阵形式是
\begin{align*}
    \bm T(\Delta x,\Delta y, \Delta z)=\left[
        \begin{array}{cccc}
            1 & 0 & 0 & \Delta x \\
            0 & 1 & 0 & \Delta y \\
            0 & 0 & 1 & \Delta z \\
            0 & 0 & 0 & 1
        \end{array}
        \right]\, .
\end{align*}

当考虑对一点的平移矩阵操作时,我们看看齐次坐标的值。
考虑$\bm T(\Delta x,\Delta y, \Delta z)$的矩阵与
齐次坐标为$[x\ y\ z\ 1]^\mathrm{T}$的点$\bm p$的乘积:
\begin{align*}
    \left[
        \begin{array}{cccc}
            1 & 0 & 0 & \Delta x \\
            0 & 1 & 0 & \Delta y \\
            0 & 0 & 1 & \Delta z \\
            0 & 0 & 0 & 1
        \end{array}
        \right]\left[
        \begin{array}{c}
            x \\y\\z\\1
        \end{array}
        \right]=\left[
        \begin{array}{c}
            x+\Delta x \\y+\Delta y\\z+\Delta z\\1
        \end{array}
        \right]\, .
\end{align*}

不出所料,我们计算得到了一个坐标值偏移了$(\Delta x,\Delta y, \Delta z)$的新点。
然而,如果我们将$\bm T$施加到向量$\bm v$,则有
\begin{align*}
    \left[
        \begin{array}{cccc}
            1 & 0 & 0 & \Delta x \\
            0 & 1 & 0 & \Delta y \\
            0 & 0 & 1 & \Delta z \\
            0 & 0 & 0 & 1
        \end{array}
        \right]\left[
        \begin{array}{c}
            x \\y\\z\\0
        \end{array}
        \right]=\left[
        \begin{array}{c}
            x \\y\\z\\0
        \end{array}
        \right]\, .
\end{align*}

结果为同一向量$\bm v$是有意义的,因为向量表示方向所以平移不改变它们。

我们将定义创建新\refvar{Transform}{}矩阵来表示给定平移的例程——
它是平移矩阵等式的简单运用。
该例程完全初始化返回的\refvar{Transform}{},
还初始化表示该平移的逆的矩阵。
\begin{lstlisting}
`\initcode{Transform Method Definitions}{=}\initnext{TransformMethodDefinitions}`
`\refvar{Transform}{}` `\initvar{Translate}{}`(const `\refvar{Vector3f}{}` &delta) {
    `\refvar{Matrix4x4}{}` m(1, 0, 0, delta.x,
                0, 1, 0, delta.y,
                0, 0, 1, delta.z, 
                0, 0, 0,       1);
    `\refvar{Matrix4x4}{}` minv(1, 0, 0, -delta.x,
                   0, 1, 0, -delta.y,
                   0, 0, 1, -delta.z, 
                   0, 0, 0,        1);
    return `\refvar{Transform}{}`(m, minv);
}
\end{lstlisting}

\subsection{缩放}\label{sub:缩放}
另一个基本变换是\keyindex{缩放变换}{scale transformation}{transformation变换},$\bm S(s_x,s_y,s_z)$。
它对点和向量起作用并用缩放因子乘以其$x,y$和$z$分量:$\bm S(2,2,1)(x,y,z)=(2x,2y,z)$。
它有如下基本性质:
\begin{align*}
    \bm S(1,1,1)                         & =\bm I\, ,                                                 \\
    \bm S(x_1,y_1,z_1)\bm S(x_2,y_2,z_2) & =\bm S(x_1x_2,y_1y_2,z_1z_2)\, ,                           \\
    \bm S^{-1}(x,y,z)                    & =\bm S\left(\frac{1}{x},\frac{1}{y},\frac{1}{z}\right)\, .
\end{align*}

我们可以区分\keyindex{均匀缩放}{uniform scaling}{},
即全部三个缩放因子有相同值,
以及\keyindex{非均匀缩放}{nonuniform scaling}{},
即它们有不同的值。
一般的缩放矩阵为
\begin{align*}
    \bm S(x,y,z)=\left[
        \begin{array}{cccc}
            x & 0 & 0 & 0 \\
            0 & y & 0 & 0 \\
            0 & 0 & z & 0 \\
            0 & 0 & 0 & 1
        \end{array}
        \right]\, .
\end{align*}

\begin{lstlisting}
`\refcode{Transform Method Definitions}{+=}\lastnext{TransformMethodDefinitions}`
`\refvar{Transform}{}` `\initvar{Scale}{}`(`\refvar{Float}{}` x, `\refvar{Float}{}` y, `\refvar{Float}{}` z) {
    `\refvar{Matrix4x4}{}` m(x, 0, 0, 0,
                0, y, 0, 0,
                0, 0, z, 0,
                0, 0, 0, 1);
    `\refvar{Matrix4x4}{}` minv(1/x,   0,   0, 0,
                   0,   1/y,   0, 0,
                   0,     0, 1/z, 0,
                   0,     0,   0, 1);
    return `\refvar{Transform}{}`(m, minv);
}
\end{lstlisting}

能够测试变换是否有缩放项是很有用的;
一个简单的方式是对三个坐标轴做该变换并看
它们的长度是否明显不等于一。
\begin{lstlisting}
`\refcode{Transform Public Methods}{+=}\lastnext{TransformPublicMethods}`
bool `\initvar{HasScale}{}`() const {
    `\refvar{Float}{}` la2 = (*this)(`\refvar{Vector3f}{}`(1, 0, 0)).`\refvar{LengthSquared}{}`();
    `\refvar{Float}{}` lb2 = (*this)(`\refvar{Vector3f}{}`(0, 1, 0)).`\refvar{LengthSquared}{}`();
    `\refvar{Float}{}` lc2 = (*this)(`\refvar{Vector3f}{}`(0, 0, 1)).`\refvar{LengthSquared}{}`();
#define NOT_ONE(x) ((x) < .999f || (x) > 1.001f)
    return (NOT_ONE(la2) || NOT_ONE(lb2) || NOT_ONE(lc2));
#undef NOT_ONE
}
\end{lstlisting}

\subsection{x,y和z轴旋转}\label{sub:x,y和z轴旋转}
另一个有用的变换类型是\keyindex{旋转变换}{rotation transformation}{transformation变换},$\bm R$。
通常,我们可以定义从原点到任意方向的任意轴再绕轴旋转给定角度。
这类最常见的是绕$x,y$和$z$坐标轴旋转。
我们把这类旋转写作$\bm R_x(\theta)$、$\bm R_y(\theta)$等。
绕任意轴旋转记作$\bm R_{(x,y,z)}(\theta)$。

旋转也有一些基本性质:
\begin{align*}
    \bm R_a(0)=\bm I\, , \\
\end{align*}