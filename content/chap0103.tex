\section{pbrt:系统概述}\label{sec:pbrt:系统概述}

pbrt是使用标准的\keyindex{面向对象}{object-oriented}{}技术构建的:
重要实体都定义了抽象\keyindex{基类}{base class}{class类}(例如
抽象基类\refvar{Shape}{}定义了所有几何形状必须实现的接口,
光源的抽象基类\refvar{Light}{}也有相似设计)。
系统大部分都是纯粹由这些抽象基类提供的接口来实现的;
例如检查光源与着色点之间遮挡物体的代码
调用\refvar{Shape}{}的相交方法
而不要考虑场景中出现的特定类型的形状。
这种方式使得扩展系统变得很容易,
新增一种形状只需要实现一个完成\refvar{Shape}{}接口的类并链接到系统。

pbrt用10个关键抽象基类写成,列于\reftab{1.1}。
向系统添加这些类的新实现很简单;
实现必须从适当的基类继承,
编译和链接到可执行文件,
且必须修改附录第\refchap{场景描述接口}中的对象创建例程
以创建解析场景描述文件所需要的对象。
\refsec{添加新物体的实现}将讨论这种扩展系统的方法的更多细节。


\begin{table}[h]
    \centering
    \begin{tabular}{l l l}
        \toprule
        \textbf{基类}         & \textbf{目录}           & \textbf{章节}                   \\
        \midrule
        \refvar{Shape}{}      & \ttfamily shapes/       & \refsec{基本形状接口}           \\
        \refvar{Aggregate}{}  & \ttfamily accelerators/ & \refsec{聚合}                   \\
        \refvar{Camera}{}     & \ttfamily cameras/      & \refsec{相机模型}               \\
        \refvar{Sampler}{}    & \ttfamily samplers/     & \refsec{采样接口}               \\
        \refvar{Filter}{}     & \ttfamily filters/      & \refsec{图像重构}               \\
        \refvar{Material}{}   & \ttfamily materials/    & \refsec{材质接口与实现}         \\
        \refvar{Texture}{}    & \ttfamily textures/     & \refsec{纹理接口与基本纹理}     \\
        \refvar{Medium}{}     & \ttfamily media/        & \refsec{介质}                   \\
        \refvar{Light}{}      & \ttfamily lights/       & \refsec{光源接口}               \\
        \refvar{Integrator}{} & \ttfamily integrators/  & \refsub{积分器接口与采样积分器} \\
        \bottomrule
    \end{tabular}
    \caption{主要接口类型。pbrt大部分由此处列出的10个关键抽象基类实现。
        每个的实现都很容易添加到系统中扩展其功能。}
    \label{tab:1.1}
\end{table}

pbrt源码发布于\href{https://pbrt.org/}{\ttfamily pbrt.org}。
(大量场景示例\footnote{\url{http://pbrt.org/scenes-v3.html}}也可分开下载。)
所有的pbrt核心代码均在目录{\ttfamily src/core}内,
函数\refvar{main}{()}在
短文件\href{https://github.com/mmp/pbrt-v3/tree/master/src/main/pbrt.cpp}{\ttfamily main/pbrt.cpp}内。

\subsection{场景表示}\label{sub:场景表示}
% TODO: pbrtInit pbrtCleanup
\begin{lstlisting}
`\initcode{Main program}{=}`
int `\initvar{main}{}`(int argc, char *argv[]) {
    Options options;
    std::vector<std::string> filenames;
    `\refcode{Process command-line arguments}{}` 
    `\refvar{pbrtInit}{}`(options);
    `\refcode{Process scene description}{}` 
    `\refvar{pbrtCleanup}{}`();
    return 0;
}
\end{lstlisting}

\subsection{积分器接口与采样积分器}\label{sub:积分器接口与采样积分器}

\begin{lstlisting}
`\initcode{Integrator Declarations}{=}`
class `\initvar{Integrator}{}` {
public:
    `\refcode{Integrator Interface}{}`
};
\end{lstlisting}