\section{pbrt:系统概述}\label{sec:pbrt:系统概述}

pbrt是使用标准的\keyindex{面向对象}{object-oriented}{}技术构建的:
重要实体都定义了抽象\keyindex{基类}{base class}{class类}(例如
抽象基类{\ttfamily Shape}定义了所有几何形状必须实现的接口,
光源的抽象基类{\ttfamily Light}也有相似设计)。
系统大部分都是纯粹由这些抽象基类提供的接口来实现的;
例如检查光源与着色点之间遮挡物体的代码
调用{\ttfamily Shape}的相交方法
而不要考虑场景中出现的特定类型的形状。
这种方式使得扩展系统变得很容易,
新增一种形状只需要实现一个完成{\ttfamily Shape}接口的类并链接到系统。

pbrt用10个关键抽象基类写成,列于\reftab{1.1}。

\begin{table}[h]
    \centering
    \begin{tabular}{l l l}
        \toprule
        \textbf{基类}                                             & \textbf{目录}           & \textbf{章节}                   \\
        \midrule
        \hyperref[code:overview_Shape]{\ttfamily Shape}           & \ttfamily shapes/       & \refsec{基本形状接口}           \\
        \hyperref[code:overview_Aggregate]{\ttfamily Aggregate}   & \ttfamily accelerators/ & \refsec{聚合}                   \\
        \hyperref[code:overview_Camera]{\ttfamily Camera}         & \ttfamily cameras/      & \refsec{相机模型}               \\
        \hyperref[code:overview_Sampler]{\ttfamily Sampler}       & \ttfamily samplers/     & \refsec{采样接口}               \\
        \hyperref[code:overview_Filter]{\ttfamily Filter}         & \ttfamily filters/      & \refsec{图像重构}               \\
        \hyperref[code:overview_Material]{\ttfamily Material}     & \ttfamily materials/    & \refsec{材质接口与实现}         \\
        \hyperref[code:overview_Texture]{\ttfamily Texture}       & \ttfamily textures/     & \refsec{纹理接口与基本纹理}     \\
        \hyperref[code:overview_Medium]{\ttfamily Medium}         & \ttfamily media/        & \refsec{介质}                   \\
        \hyperref[code:overview_Light]{\ttfamily Light}           & \ttfamily lights/       & \refsec{光源接口}               \\
        \hyperref[code:overview_Integrator]{\ttfamily Integrator} & \ttfamily integrators/  & \refsub{积分器接口与采样积分器} \\
        \bottomrule
    \end{tabular}
    \caption{主要接口类型。pbrt大部分由此处列出的10个关键抽象基类实现。
        每个的实现都很容易添加到系统中扩展其功能。}
    \label{tab:1.1}
\end{table}

\subsection{积分器接口与采样积分器}\label{sub:积分器接口与采样积分器}

\label{code:overview_Integrator}
\begin{lstlisting}
`\refcode{Integrator Declarations}{=}`
class Integrator {
public:
    `\refcode{Integrator Interface}{}`
};
\end{lstlisting}