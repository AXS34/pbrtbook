\section{辐射度学}\label{sec:辐射度学}
 
\keyindex{辐射度学}{radiometry}{}提供了一系列描述光传播与反射的思想和数学工具。
它为贯穿本书剩余部分的渲染算法构建了推导的基础。
有趣的是,辐射度学不是用物理光学基本原理推导产生的,
而是基于穿过空间的粒子流来对光进行抽象并建立的。
因此,尽管辐射度学和\keyindex{麦克斯韦方程组}{Maxwell's equations}{}之间
已经建立了联系,为辐射度学提供了坚实的物理基础,
但是像光的\keyindex{偏振}{polarization}{}那样的效应并不符合该框架。

\keyindex{辐射转移}{radiative transfer}{}是关于辐射能量转移现象的研究。
它基于辐射度量原则并在\keyindex{几何光学}{geometrical optics}{}
\sidenote{译者注:原文写作geometric optics。}层面上操作,
其中光的宏观性质足以描述光如何与比其波长大得多的物体交互。
与光的\keyindex{波动光学}{wave optics}{}模型中的现象结合并不罕见,
但这些结果需要用辐射转移的基本抽象语言来表达。
(\citet{PREISENDORFER19653}已经将辐射转移理论与
麦克斯韦描述电磁场的经典方程组联系起来。
他的框架既论证了它们的等价性又使得把一个世界观的结果应用到另一个世界观更容易了。
\citet{Fante:81}在该领域做了更多最近的工作。)


\subsection{亮度和光度学}\label{sub:亮度和光度学}