\chapterimage{Pictures/chap01/nightsnow-cut1368.png}

\chapter{绪论}\label{chap:绪论}

\keyindex{渲染}{rendering}{render渲染}是由
3D\keyindex{场景}{scene}{}
描述生成图像的过程。
显然,这是一项十分庞大的任务,有许多解决方案。
\keyindex{基于物理的}{physically based}{physics物理}
的技术便是要模拟现实,
即运用物理学规律对光与物质的
\keyindex{相互作用}{interaction}{}建模。
尽管基于物理的方法看上去是实现渲染最显然的办法,
但最近十年它才在实践中得到广泛运用。
本章末的\ref{sec:基于物理的渲染简史}节
将给出基于物理的渲染的简史
以及它近来在电影
\keyindex{离线渲染}{offline rendering}{render渲染}和
游戏\keyindex{交互式渲染}{interactive rendering}{render渲染}方面的应用。

本书将介绍\emph{pbrt}这一基于
\keyindex{光线追踪}{ray-tracing}{ray光线}算法的基于物理的渲染系统。
大多数计算机\keyindex{图形学}{graphics}{}书籍都主讲算法和理论,
偶尔附上一小段代码。
相反,本书将理论和一个功能齐全的渲染系统的完整实现结合起来。
系统的完整代码\footnote{\url{https://github.com/mmp/pbrt-v3}}
可在BSD许可证下获取。
在pbrt网站\url{https://pbrt.org}还可获取示例场景、渲染数据等更多信息。


\section{文学编程}\label{sec:文学编程}

在编写\TeX 排版系统时,Donald Knuth提出了一种新的
简单但具有革命性的编程方法论:
\emph{程序应该写得更便于人类使用而不是更便于计算机理解}。
他把这套方法论称作\keyindex{文学编程}{literate programming}{}。
本书(包括你正在阅读的本章)就是一个长长的文学程序。
这意味着在阅读本书的过程中,
你会读到pbrt渲染系统的\emph{完整}实现,
而不仅仅是高层叙述。

文学程序是\keyindex{元语言}{metalanguage}{}
写成的,该语言把文档格式语言(例如\TeX 或HTML)
和编程语言(例如C++)结合起来。
两套分离的系统会这样处理程序:
\keyindex{编排器}{weaver}{literate programming文学编程}
把文学程序转换成适合排版的文档,
\keyindex{整合器}{tangler}{literate programming文学编程}
则生成可供编译的源码
\sidenote{译者注:我不太确定编排器和整合器的翻译是否合适。}。
虽然我们的文学编程系统是自研的,
但很大程度上受到了Norman Ramsey的\emph{noweb}系统的影响。

文学编程元语言提供了两个重要功能。
第一个是把行文与源码结合的能力。
这个功能让对程序的阐述和它实际的源码一样重要,
促使设计和文档做得更细致。
第二个是该语言提供了向读者展示程序代码的机制,
其与输入编译器的顺序完全不同。
因此可以按逻辑顺序阐述程序。
每一段具有名称的代码块叫作\keyindex{代码片}{fragment}{},
且每个代码片可以通过名称引用其他代码片。

例如,考虑一个负责初始化程序全部全局变量的函数
\footnote{本节的代码仅用作示例,不属于pbrt的一部分。}
{\ttfamily InitGlobals()}:
\begin{lstlisting}
void InitGlobals() {
    nMarbles = 25.7;
    shoeSize = 13;
    dielectric = true;
}
\end{lstlisting}

虽然很简短,但没有任何上下文的情况下很难搞懂这个函数。
比如为什么变量{\ttfamily nMarbles}采用浮点值?
刚看这段代码时,
就得在整个程序里寻找每个变量是在哪里声明的、怎么用的,
好搞清楚它的目的和合法值的含义。
尽管这样的系统结构对编译器来说没问题,
但人类阅读者更愿意看到
每个变量的初始化代码是分开呈现的,
且紧挨着实际声明和使用这些变量的代码。

在文学程序中,可以把{\ttfamily InitGlobals()}写成这样:
\begin{lstlisting}
`\initcode{Function Definitions}{=}`
void InitGlobals() {
    `\refcode{Initialize Global Variables}{\dag}`
}
\end{lstlisting}

这就定义了称作\refcode{Function Definitions}{}的代码片,
包含了函数{\ttfamily InitGlobals()}的定义。
函数{\ttfamily InitGlobals()}自己又引用了另一代码片
\refcode{Initialize Global Variables}{}。
因为初始化的代码片还没有定义,
所以我们只知道这个函数可能会对全局变量赋值
(然而我们可以通过单击右边的加号
\sidenote{译者注:本中译版改为直接点击代码片名称。}
向前跳转;
这样可以展开代码片最终全部的代码)。

现在有了代码片名称仅仅是有了正确的抽象层级,
因为还没有声明过任何变量。
之后在程序某处引入全局变量{\ttfamily shoeSize}时,
我们可以这样写:
\begin{lstlisting}
    `\refcode{Initialize Global Variables}{=}\initnext{InitializeGlobalVariables}`
    shoeSize = 13;
\end{lstlisting}

这里我们开始定义\refcode{Initialize Global Variables}{}的内容了。
当文学程序整合成待编译的源码时,
文学编程系统会把代码{\ttfamily shoeSize = 13;}
替换到函数{\ttfamily InitGlobals()}的定义内。
等号后的符号{\codecolor $\downarrow$}表示后续还有代码添加到该代码片。
点击它即可跳转到下一处。

后文我们也许定义了另一个全局变量{\ttfamily dielectric},
可以这样把它的初始化添到代码片之后:
\begin{lstlisting}
    `\refcode{Initialize Global Variables}{+=}\lastcode{InitializeGlobalVariables}`
    dielectric = true;
\end{lstlisting}

代码片名后的符号{\codecolor +=}表示我们之前已经定义过该代码片了。
此外符号{\codecolor $\uparrow$}回链到
之前\refcode{Initialize Global Variables}{}添加代码的地方。

当整合时,这三个代码片转换为代码:
\begin{lstlisting}
void InitGlobals() {
    `\hypertarget{code:Initialize Global Variables}{\color[RGB]{115,48,11}\scriptsize\rmfamily// Initialize Global Variables}`
    shoeSize = 13;
    dielectric = true;
}
\end{lstlisting}

这样,我们可以把复杂函数分解为逻辑不同的部分,使之易懂。
例如我们可以这样把一个复杂函数写作一系列代码片:
\begin{lstlisting}
`\refcode{Function Definitions}{+=}`
void complexFunc(int x, int y, double *values) {
    `\refcode{Check validity of arguments}{}`
    if (x < y) {
        `\refcode{Swap parameter values}{}`
    }
    `\refcode{Do precomputation before loop}{}`
    `\refcode{Loop through and update \textbackslash mono\{values\} array}{}`
\end{lstlisting}

同样,编译时{\ttfamily complexFunc()}内每段代码片的内容都内联展开。
在文档中,我们可以依次介绍每个代码片及其实现。
这种分解让我们每次只展示一小段代码,使之更易于理解。
这种编程风格的另一优点是,通过把函数分解为逻辑片,
每片有了单一且明确的目的,可以独立编写、验证、阅读。
总之,我们尽量让每段代码片少于10行。

在某种意义上,文学编程系统只是个增强了的宏替换包,
完成重排程序源码的任务。
这看起来是微不足道的变化,但事实上文学编程和其他软件构建系统方法迥然不同。

\section{基于物理的渲染简史}\label{sec:基于物理的渲染简史}

