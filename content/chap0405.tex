\section{扩展阅读}\label{sec:扩展阅读04}
引入光线追踪算法之后,涌现了大量尝试寻找高效方法对其加速的研究,
主要是通过开发改进的光线追踪加速结构。
《\citetitle{10.5555/94788}》\citep{10.5555/94788}中Arvo和Kirk的章节
总结了1989年最新进展并为区分不同光线相交加速方法提供了优秀的分类方案。

Kirk和Arvo \parencite*{Kirk88theray}引入了\keyindex{元层次}{meta-hierarchies}{}的统一原则。
它们证明了通过让实现的加速数据结构与场景图元遵照相同的接口,
很容易混合与匹配不同的相交加速框架。
pbrt遵循这一模型,因为\refvar{Aggregate}{}继承自基类\refvar{Primitive}{}。

\subsection{网格}\label{sub:网格}
Fujimoto、Tanaka和Iwata\parencite*{4056861}引入了均匀网格,
即把场景边界分解为等长网格的空间细分方法。
Amanatides和Woo \parencite*{10.2312:egtp.19871000}
以及Cleary和Wyvill \parencite*{Cleary1988}描述了更高效的网格遍历方法。
Snyder和Barr \parencite*{10.1145/37401.37417}描述了
对该方法的大量改进并证明了网格对于渲染极其复杂场景的用处。
Jevans和Wyvill \parencite*{Jevans1989:23}引入了层次化网格,
即含有许多图元的网格自我细化为小格。
Cazals、Drettakis和Puech \parencite*{cazals1995filtering}以及
Klimaszewski和Sederberg \parencite*{576857}为层次化网格开发了更复杂的技术。

Ize等\parencite*{4061545}为网格的并行创建开发了高效算法。
他们的有趣发现之一是随着所用处理核数量的增长,
网格创建性能很快被有效内存带宽所限制。

选择最优网格分辨率对于从网格中获得优异性能很重要。
Ize等\parencite*{4342587}有该话题的优秀论文,
为完全自动化选择分辨率以及在使用层次化网格时决定何时细化为子网格提供了坚实基础。
他们用大量简化假设推导出理论结果,然后证明了这些结果渲染真实世界场景的适用性。
他们的论文也包括对该领域前人工作很好的筛选引用。

Lagae和Dutré \parencite*{lagae2008compact}基于
哈希法\sidenote{译者注:即hashing,也称散列法。}为均匀网格
描述了一种新颖的表示,它具有的优良性质是不仅每个图元
拥有对网格的单个索引,而且每个网格也只有单个图元索引。
他们证明了该表示有很低的内存使用量且仍然非常高效。

Hunt和Mark \parencite*{4634613}证明了在透视空间中构建网格,
即投影中心是相机或光源时,能让追踪相机或光源发出的光线高效得多。
尽管该方法需要多种加速结构,但从为不同种类光线专门设计的多种结构中获得的性能提升可以很高。
他们的方法也因在某种意义上是栅格化和光线追踪的中间地带而令人瞩目。

\subsection{包围盒层次}\label{sub:包围盒层次}
\citet{10.1145/360349.360354}首先建议为标准可见曲面确定算法使用包围盒来剔除物体集。
在此基础上,Rubin和Whitted \parencite*{10.1145/800250.807479}首先为快速光线追踪
的场景表示开发了层次化数据结构,尽管他们的方法依赖于用户去定义层次。
Kay和Kajiya \parencite*{10.1145/15922.15916}基于用厚板集定界物体实现了最早之一的实用物体细分方法。
Goldsmith和Salmon \parencite*{4057175}描述了自动计算包围盒层次的首个算法。
尽管他们的算法是基于依据盒的表面积来估计光线与包围盒相交的概率,
但它比现代SAH BVH算法低效得多。

本章的\refvar{BVHAccel}{}实现基于\citet{4342588}以及Günther等\parencite*{4342598}描述的构建算法。
边界框测试则是Williams等\parencite*{10.1145/1198555.1198748}引入的。
Eisemann等\parencite*{10.1080/2151237X.2007.10129248}开发了甚至更高效的边界框测试,
当同一光线对许多边界框做相交测试时它进行额外的预计算以换取更高的性能;
我们把实现他们的方法留作习题。

pbrt中用的BVH遍历算法由多位研究者同时开发出来;
见Boulos和Haines \parencite*{bouloshaines2006}的批注了解更多细节和背景。
树遍历的另一选项是Kay和Kajiya \parencite*{10.1145/15922.15916};
他们维护一个按光线距离排序的节点堆。
在单片存储\sidenote{译者注:原文on-chip memory。}数量相对有限的GPU上,
为每条光线维护一个将要访问的节点的栈可能会有极其高的内存开销。
Foley和Sugerman \parencite*{10.1145/1071866.1071869}引入了
“无栈”\sidenote{译者注:原文stackless。}kd树遍历算法,
它周期性地从树根开始回溯和搜索以找到下一个要访问的节点而不是显式保存所有要访问的节点。
\citet{10.5555/1921479.1921496}对该方法做了大量改进,
减少了从树根重新遍历的频率并将该方法应用于BVH。

许多研究者已经为构建BVH后提升其质量开发了许多技术。
Yoon等\parencite*{10.2312:EGWR:EGSR07:073-084}和\citet{4634624}提出了
对BVH做局部调整的算法,Kopta等\parencite*{10.1145/2159616.2159649}在
一个动画的多个坐标系上复用BVH,通过更新包围运动物体的部分来保持其质量。
也见Bittner等\parencite*{BittnerFast2013}、Karras和Aila \parencite*{10.1145/2492045.2492055}
以及Bittner等\parencite*{BITTNER2015135}了解该领域的最新工作。

当前大多数构建BVH方法都基于自顶向下的树构建,
首先创建树节点然后将图元划分到孩子中并继续递归。
Walter等\parencite*{4634626}证明了另一个方法,
他说明自底向上的构建即首先创建叶子然后聚为父亲节点是可行选项。
Gu等\parencite*{10.1145/2492045.2492054}开发了该方法高效得多的实现并
证明了其对并行实现的适应性。

BVH的一个缺点是即便少量与边界框重合的相对较大的图元也会极大降低BVH的效率:
仅仅因为下沉到叶子的几何体的重合边界框\sidenote{译者注:此句翻译不确定。},
许多树节点就会重合。Ernst和Greiner \parencite*{4342593}提出
“分割剪裁”\sidenote{译者注:原文split clipping。}的办法;
提出了树中每个图元只出现一次的约束,
并且巨大输入图元的边界框被细分为更紧致的子框集再用于树的构建。
Dammertz和Keller \parencite*{4634636}观察到有问题的图元是
那些相对于其表面积在其边界框内有大量空白空间的,
所以它们细分了最异常的三角形并报告有巨大性能提升。
Stich等\parencite*{10.1145/1572769.1572771}开发了
在BVH构建期间划分图元的方法,使得当发现SAH开销下降时可以只划分图元。
也见Popov等\parencite*{10.1145/1572769.1572772}理论上
优化BVH划分算法及其与之前方法的关系,
以及Karras和Aila \parencite*{10.1145/2492045.2492055}改进
决定何时分割三角形的准则。
Woop等\parencite*{10.5555/2980009.2980014}开发了一种方法为长细几何体如毛发等构建BVH;
因为这类几何体相对于其边界框体积来说非常细,
在大多数加速结构上它一般都有很差的性能。

BVH的内存要求可以非常大。在我们的实现中,每个节点为32字节。
场景中每个图元至多需要2个BVH树节点,每个图元的总开销可以高达64字节。
Cline等\parencite*{10.5555/2383894.2383909}建议为BVH节点使用更紧实的表示,牺牲一些效率。
首先,他们量化了每个节点中保存的边界框,用8或16字节来编码其相对于节点父亲边界框的位置。
然后,他们使用了{\itshape 隐式索引}\sidenote{译者注:原文implicit indexing。},
其中节点$i$的孩子在节点数组中的位置为$2i$和$2i+1$(假设分支系数为$2\times$)。
他们证明节约了大量内存,而性能影响适中。
Bauszat等\parencite*{10.2312:PE:VMV:VMV10:227-234}开发了另一种在空间上高效的BVH表示。
也见Segovia和Ernst \parencite*{10.5555/1839214.1839242}开发的BVH节点与三角网格的紧实表示。

Yoon和Manocha \parencite*{10.1111/j.1467-8659.2006.00970.x}为
缓存高效的BVH和kd树布局提出了算法并展示了来自它们的性能提升。
也见\citet{10.5555/1121584}的书籍了解该话题的广泛讨论。

Lauterbach等\parencite*{10.1111/j.1467-8659.2009.01377.x}引入了线性BVH。
Pantaleoni和Luebke \parencite*{10.5555/1921479.1921493}在树的上层用SAH开发了HLBVH推广型。
他们还注意到莫顿编码值的高位可用于高效寻找图元群集——两种思想都用于我们HLBVH的开发。
Garanzha等\parencite*{10.1145/2018323.2018333}对HLBVH引入更多改进,
它们大多数都针对GPU实现。

不像HLBVH路线,这里\refvar{BVHAccel}{}中的BVH构建实现没有并行化。
详见\citet{5669303}了解始终利用SAH进行高性能并行BVH构建的方法。

\subsection{kd树}\label{sub:kd树}
\citet{6429331}为光线相交计算引入使用了八叉树。