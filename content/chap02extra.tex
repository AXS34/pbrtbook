\section{译者补充:四元数}\label{sec:译者补充:四元数}
\begin{remark}
    本节内容不是原书内容,而是译者自学补充的,请酌情参考和斧正。
\end{remark}

本节内容主要依据文献\citep{10.5555/90767.90913,enwiki:1014110231,enwiki:1013104981}整理而成,
给出四元数相关数学推导,具体介绍
四元数的定义、性质及其在几何变换中的运用。

\subsection{四元数的定义}\label{sub:四元数的定义}
\begin{definition}
    \keyindex{四元数}{quaternion}{}记作
    \begin{align}
        {\bm q}=c+x\mathbf{i}+y\mathbf{j}+z\mathbf{k}\, ,
    \end{align}
    其中$c,x,y,z$是实数,
    $\mathbf{i},\mathbf{j},\mathbf{k}$是\keyindex{基四元数}{basic quaternion}{quaternion四元数},
    也称\keyindex{基元}{basis element}{quaternion四元数}。

    基四元数可视为分别指向三个空间轴向的单位向量。
    此时${\bm q}$可以看作由一个标量和一个向量构成:
    其中$c$称为\keyindex{实部}{real part}{}或\keyindex{标量部}{scalar part}{quaternion四元数},
    $x\mathbf{i}+y\mathbf{j}+z\mathbf{k}$称为\keyindex{虚部}{imaginary part}{quaternion四元数}、\keyindex{纯部}{pure part}{quaternion四元数}或\keyindex{向量部}{vector part}{quaternion四元数}。
\end{definition}
\begin{definition}
    当$c=0$且$xyz\neq 0$时,${\bm q}$称为\keyindex{向量四元数}{vector quaternion}{quaternion四元数}。
\end{definition}
\begin{definition}
    当$x=y=z=0$时,${\bm q}$称为\keyindex{标量四元数}{scalar quaternion}{quaternion四元数}。
    其中,当$c=x=y=z=0$时,${\bm q}$称为\keyindex{零四元数}{zero quaternion}{quaternion四元数},记作$0$。
\end{definition}
\begin{notation}
    在实际书写时,我们做如下约定:
    \begin{itemize}
        \item $c,x,y,z$之一等于$0$时,略写相应项;
        \item $x,y,z$之一等于$1$时,相应项简写为$\mathbf{i,j}$或$\mathbf{k}$;
        \item 也可写作${\bm q}=c+{\bm u}$,其中$c$为标量,${\bm u}=x\mathbf{i}+y\mathbf{j}+z\mathbf{k}$为向量;
        \item 所有四元数构成的集合记作$\mathbb{H}$。
    \end{itemize}
\end{notation}
\begin{remark}
    因为实数域$\mathbb{R}$、向量空间$\mathbb{R}^3$分别
    与$\mathbb{H}$的子集\keyindex{同构}{isomorphic}{},
    所以即便在这种记法下
    实数与标量四元数、三维向量与向量四元数记号一样,
    其逻辑也是自洽的。
\end{remark}
\subsection{四元数的运算}\label{sub:四元数的运算}
分别记四元数为
\begin{align}
    {\bm q}   & =c+{\bm u}=c+x\mathbf{i}+y\mathbf{j}+z\mathbf{k}\, ,             \\
    {\bm q}_1 & =c_1+{\bm u}_1=c_1+x_1\mathbf{i}+y_1\mathbf{j}+z_1\mathbf{k}\, , \\
    {\bm q}_2 & =c_2+{\bm u}_2=c_2+x_2\mathbf{i}+y_2\mathbf{j}+z_2\mathbf{k}\, ,
\end{align}
其中$c,x,y,z,c_1,x_1,y_1,z_1,c_2,x_2,y_2,z_2\in\mathbb{R}$,${\bm u},{\bm u}_1,{\bm u}_2\in\mathbb{R}_3$。
\begin{definition}
    四元数\keyindex{加法}{addition}{}为
    \begin{align}
        {\bm q}_1+{\bm q}_2 & =(c_1+c_2)+({\bm u}_1+{\bm u}_2)\nonumber                                  \\
                            & =(c_1+c_2)+(x_1+x_2)\mathbf{i}+(y_1+y_2)\mathbf{j}+(z_1+z_2)\mathbf{k}\, .
    \end{align}
\end{definition}
\begin{definition}
    基元$\mathbf{i},\mathbf{j},\mathbf{k}$的乘法为
    \begin{align}
        1\mathbf{i}  & =\mathbf{i}1=\mathbf{i}, & 1\mathbf{j}  & =\mathbf{j}1=\mathbf{j}, & 1\mathbf{k}  & =\mathbf{k}1=\mathbf{k}\, , \\
        \mathbf{i}^2 & =-1,                     & \mathbf{j}^2 & =-1,                     & \mathbf{k}^2 & =-1\, ,                     \\
        \mathbf{ij}  & =\mathbf{k},             & \mathbf{jk}  & =\mathbf{i},             & \mathbf{ki}  & =\mathbf{j}\, ,             \\
        \mathbf{ji}  & =-\mathbf{k},            & \mathbf{kj}  & =-\mathbf{i},            & \mathbf{ik}  & =-\mathbf{j}\, .
    \end{align}
\end{definition}
\begin{definition}
    对于$\lambda\in\mathbb{R}$,四元数的\keyindex{数乘}{scalar multiplication}{}为
    \begin{align}
        \lambda{\bm q} & =\lambda c+\lambda {\bm u}\nonumber                                              \\
                       & =\lambda c+(\lambda x)\mathbf{i}+(\lambda y)\mathbf{j}+(\lambda z)\mathbf{k}\, .
    \end{align}
\end{definition}
