\section{译者补充:四元数}\label{sec:译者补充:四元数}
\begin{remark}
    本节内容不是原书内容,而是译者自学补充的,请酌情参考和斧正。
\end{remark}

本节内容主要依据文献\citep{10.5555/90767.90913,enwiki:1013104981}整理而成,
给出四元数相关数学推导,具体介绍
四元数的定义、性质及其在几何变换中的运用。

\subsection{四元数的定义}\label{sub:四元数的定义}
\begin{definition}
    \keyindex{四元数}{quaternion}{}记作
    \begin{align}
        {\bm q}=c+x\mathbf{i}+y\mathbf{j}+z\mathbf{k}\, ,
    \end{align}
    其中$c, x, y, z$是实数,
    $\mathbf{i}, \mathbf{j}, \mathbf{k}$是\keyindex{基四元数}{basic quaternion}{quaternion四元数},
    也称\keyindex{基元}{basis element}{quaternion四元数}。

    基四元数可视为分别指向三个空间轴向的单位向量。
    此时${\bm q}$可以看作由一个标量和一个向量构成:
    其中$c$称为\keyindex{实部}{real part}{}或\keyindex{标量部}{scalar part}{quaternion四元数},
    $x\mathbf{i}+y\mathbf{j}+z\mathbf{k}$称为\keyindex{虚部}{imaginary part}{quaternion四元数}、\keyindex{纯部}{pure part}{quaternion四元数}或\keyindex{向量部}{vector part}{quaternion四元数}。
\end{definition}
\begin{definition}
    当$c=0$且$xyz\neq 0$时,${\bm q}$称为\keyindex{向量四元数}{vector quaternion}{quaternion四元数}。
\end{definition}
\begin{definition}
    当$x=y=z=0$时,${\bm q}$称为\keyindex{标量四元数}{scalar quaternion}{quaternion四元数}。
    其中,当$c=x=y=z=0$时,${\bm q}$称为\keyindex{零四元数}{zero quaternion}{quaternion四元数},记作$0$。
\end{definition}
\begin{notation}
    在实际书写时,我们做如下约定:
    \begin{itemize}
        \item $c, x, y, z$之一等于$0$时,略写相应项;
        \item $x, y, z$之一等于$1$时,相应项简写为$\mathbf{i, j}$或$\mathbf{k}$;
        \item 也可写作${\bm q}=c+{\bm u}$,其中$c$为标量,${\bm u}=x\mathbf{i}+y\mathbf{j}+z\mathbf{k}$为向量;
        \item 所有四元数构成的集合记作$\mathbb{H}$。
    \end{itemize}
\end{notation}
\begin{remark}
    因为实数域$\mathbb{R}$、向量空间$\mathbb{R}^3$分别
    与$\mathbb{H}$的子集\keyindex{同构}{isomorphic}{},
    所以即便在这种记法下
    实数与标量四元数、三维向量与向量四元数记号一样,
    其逻辑也是自洽的。
\end{remark}
\subsection{四元数的运算}\label{sub:四元数的运算}
分别记四元数为
\begin{align}
    {\bm q}   & =c+{\bm u}=c+x\mathbf{i}+y\mathbf{j}+z\mathbf{k}\, ,             \\
    {\bm q}_n & =c_n+{\bm u}_n=c_n+x_n\mathbf{i}+y_n\mathbf{j}+z_n\mathbf{k}\, ,
\end{align}
其中$c, x, y, z, c_n, x_n, y_n, z_n\in\mathbb{R}$,${\bm u}, {\bm u}_n\in\mathbb{R}^3$,$n$为下标。
\begin{definition}
    四元数\keyindex{加法}{addition}{}为
    \begin{align}
        {\bm q}_1+{\bm q}_2 & =(c_1+c_2)+({\bm u}_1+{\bm u}_2)\nonumber                                  \\
                            & =(c_1+c_2)+(x_1+x_2)\mathbf{i}+(y_1+y_2)\mathbf{j}+(z_1+z_2)\mathbf{k}\, .
    \end{align}
\end{definition}
\begin{proposition}
    四元数加法的\keyindex{单位元}{identity element}{}是零四元数。
\end{proposition}
\begin{proposition}
    四元数加法的\keyindex{逆元}{inverse element}{}是
    \begin{align}
        -{\bm q}=-c-x\mathbf{i}-y\mathbf{j}-z\mathbf{k}\, .
    \end{align}
\end{proposition}
\begin{proposition}
    四元数的加法满足\keyindex{交换律}{law of commutation}{}
    \begin{align}
        {\bm q}_1+{\bm q}_2={\bm q}_2+{\bm q}_1, \quad \forall {\bm q}_1, {\bm q}_2\in \mathbb{H}\, .
    \end{align}
\end{proposition}
\begin{proposition}
    四元数的加法满足\keyindex{结合律}{law of association}{}
    \begin{align}
        ({\bm q}_1+{\bm q}_2)+{\bm q}_3={\bm q}_1+({\bm q}_2+{\bm q}_3), \quad \forall {\bm q}_1, {\bm q}_2, {\bm q}_3\in \mathbb{H}\, .
    \end{align}
\end{proposition}
\begin{notation}
    约定向量的\keyindex{数乘}{scalar multiplication}{}、\keyindex{内积}{inner product}{}、\keyindex{叉积}{cross product}{}分别记作
    \begin{align}
        \lambda{\bm u}           & =(\lambda x)\mathbf{i}+(\lambda y)\mathbf{j}+(\lambda z)\mathbf{k},\quad \forall \lambda\in\mathbb{R}\, , \\
        {\bm u}_1\cdot{\bm u}_2  & =x_1x_2+y_1y_2+z_1z_2\, ,                                                                                 \\
        {\bm u}_1\times{\bm u}_2 & =(y_1z_2-y_2z_1)\mathbf{i}+(z_1x_2-z_2x_1)\mathbf{j}+(x_1y_2-x_2y_1)\mathbf{k}\, .
    \end{align}
\end{notation}
\begin{definition}
    四元数${\bm q}$与实数$\lambda$的\keyindex{数乘}{scalar multiplication}{}为
    \begin{align}
        \lambda{\bm q}={\bm q}\lambda & =\lambda c+\lambda {\bm u}\nonumber                                              \\
                                      & =\lambda c+(\lambda x)\mathbf{i}+(\lambda y)\mathbf{j}+(\lambda z)\mathbf{k}\, .
    \end{align}
\end{definition}
\begin{proposition}
    四元数的数乘对加法满足\keyindex{分配律}{law of distribution}{}
    \begin{align}
        (\lambda_1+\lambda_2){\bm q} & =\lambda_1{\bm q}+\lambda_2{\bm q}, \quad \forall \lambda_1, \lambda_2\in\mathbb{R}, \forall {\bm q}\in\mathbb{H}\, , \\
        \lambda({\bm q}_1+{\bm q}_2) & =\lambda{\bm q}_1+\lambda{\bm q}_2, \quad \forall \lambda\in\mathbb{R}, \forall {\bm q}_1, {\bm q}_2\in\mathbb{H}\, .
    \end{align}
\end{proposition}
\begin{definition}
    基元$\mathbf{i}, \mathbf{j}, \mathbf{k}$的乘法为
    \begin{align}
        1\mathbf{i}  & =\mathbf{i}1=\mathbf{i}, & 1\mathbf{j}  & =\mathbf{j}1=\mathbf{j}, & 1\mathbf{k}  & =\mathbf{k}1=\mathbf{k}\, ,\nonumber \\
        \mathbf{i}^2 & =-1,                     & \mathbf{j}^2 & =-1,                     & \mathbf{k}^2 & =-1\, ,                    \nonumber \\
        \mathbf{ij}  & =\mathbf{k},             & \mathbf{jk}  & =\mathbf{i},             & \mathbf{ki}  & =\mathbf{j}\, ,            \nonumber \\
        \mathbf{ji}  & =-\mathbf{k},            & \mathbf{kj}  & =-\mathbf{i},            & \mathbf{ik}  & =-\mathbf{j}\, .
    \end{align}
\end{definition}
\begin{definition}
    四元数\keyindex{乘法}{multiplication}{}为
    \begin{align}
        {\bm q}_1{\bm q}_2 & =(c_1+{\bm u}_1)(c_2+{\bm u}_2)\nonumber                                                                      \\
                           & =(c_1+x_1\mathbf{i}+y_1\mathbf{j}+z_1\mathbf{k})(c_2+x_2\mathbf{i}+y_2\mathbf{j}+z_2\mathbf{k})\nonumber      \\
                           & =\quad c_1c_2+c_1x_2\mathbf{i}+c_1y_2\mathbf{j}+c_1z_2\mathbf{k}\nonumber                                     \\
                           & \quad +x_1c_2\mathbf{i}+x_1x_2\mathbf{i}^2+x_1y_2\mathbf{i}\mathbf{j}+x_1z_2\mathbf{i}\mathbf{k}\nonumber     \\
                           & \quad +y_1c_2\mathbf{j}+y_1x_2\mathbf{j}\mathbf{i}+y_1y_2\mathbf{j}^2+y_1z_2\mathbf{j}\mathbf{k}\nonumber     \\
                           & \quad +z_1c_2\mathbf{k}+z_1x_2\mathbf{k}\mathbf{i}+z_1y_2\mathbf{k}\mathbf{j}+z_1z_2\mathbf{k}^2\nonumber     \\
                           & =\quad c_1c_2-(x_1x_2+y_1y_2+z_1z_2)\nonumber                                                                 \\
                           & \quad +c_1(x_2\mathbf{i}+y_2\mathbf{j}+z_2\mathbf{k})+c_2(x_1\mathbf{i}+y_1\mathbf{j}+z_1\mathbf{k})\nonumber \\
                           & \quad +(y_1z_2-y_2z_1)\mathbf{i}+(z_1x_2-z_2x_1)\mathbf{j}+(x_1y_2-x_2y_1)\mathbf{k}\nonumber                 \\
                           & =(c_1c_2-{\bm u}_1\cdot{\bm u}_2)+(c_1{\bm u}_2+c_2{\bm u}_1+{\bm u}_1\times{\bm u}_2)\, ,
    \end{align}
    也称为\keyindex{哈密顿积}{Hamilton product}{}。
\end{definition}
\begin{proposition}
    四元数乘法的单位元是$1$。
\end{proposition}
\begin{proposition}
    四元数的乘法不满足交换律。
\end{proposition}
\begin{example}
    取
    \begin{align}
        {\bm q}_1 & =1+2\mathbf{i}+3\mathbf{j}+4\mathbf{k}\, , \\
        {\bm q}_2 & =5+6\mathbf{i}+7\mathbf{j}+8\mathbf{k}\, .
    \end{align}
    于是易得
    \begin{align}
        {\bm q}_1{\bm q}_2 & =-60+12\mathbf{i}+30\mathbf{j}+24\mathbf{k}\, , \\
        {\bm q}_2{\bm q}_1 & =-60+20\mathbf{i}+14\mathbf{j}+32\mathbf{k}\, .
    \end{align}
    显然${\bm q}_1{\bm q}_2\neq{\bm q}_2{\bm q}_1$。
\end{example}
\begin{proposition}
    四元数的乘法满足结合律
    \begin{align}
        ({\bm q}_1{\bm q}_2){\bm q}_3={\bm q}_1({\bm q}_2{\bm q}_3), \quad \forall {\bm q}_1, {\bm q}_2, {\bm q}_3\in \mathbb{H}\, .
    \end{align}
\end{proposition}
\begin{prove}
    根据定义有
    \begin{align}
        ({\bm q}_1{\bm q}_2){\bm q}_3 & =((c_1+{\bm u}_1)(c_2+{\bm u}_2))(c_3+{\bm u}_3)\nonumber                                                                                           \\
                                      & =((c_1c_2-{\bm u}_1\cdot{\bm u}_2)+(c_1{\bm u}_2+c_2{\bm u}_1+{\bm u}_1\times{\bm u}_2))(c_3+{\bm u}_3)\nonumber                                    \\
                                      & =\quad ((c_1c_2-{\bm u}_1\cdot{\bm u}_2)c_3-(c_1{\bm u}_2+c_2{\bm u}_1+{\bm u}_1\times{\bm u}_2)\cdot{\bm u}_3)\nonumber                            \\
                                      & \quad +((c_1c_2-{\bm u}_1\cdot{\bm u}_2){\bm u}_3+c_3(c_1{\bm u}_2+c_2{\bm u}_1+{\bm u}_1\times{\bm u}_2)\nonumber                                  \\
                                      & \quad +(c_1{\bm u}_2+c_2{\bm u}_1+{\bm u}_1\times{\bm u}_2)\times{\bm u}_3)\nonumber                                                                \\
                                      & =\quad c_1c_2c_3-c_3{\bm u}_1\cdot{\bm u}_2-c_1{\bm u}_2\cdot{\bm u}_3-c_2{\bm u}_1\cdot{\bm u}_3-({\bm u}_1\times{\bm u}_2)\cdot{\bm u}_3\nonumber \\
                                      & \quad +c_1c_2{\bm u}_3-({\bm u}_1\cdot{\bm u}_2){\bm u}_3+c_3c_1{\bm u}_2+c_2c_3{\bm u}_1+c_3{\bm u}_1\times{\bm u}_2\nonumber                      \\
                                      & \quad +c_1{\bm u}_2\times{\bm u}_3+c_2{\bm u}_1\times{\bm u}_3+({\bm u}_1\times{\bm u}_2)\times{\bm u}_3\nonumber                                   \\
                                      & =\quad  c_1c_2c_3-({\bm u}_1\times{\bm u}_2)\cdot{\bm u}_3+({\bm u}_1\times{\bm u}_2)\times{\bm u}_3-({\bm u}_1\cdot{\bm u}_2){\bm u}_3\nonumber    \\
                                      & \quad +c_2c_3{\bm u}_1+c_1{\bm u}_2\times{\bm u}_3-c_1{\bm u}_2\cdot{\bm u}_3\nonumber                                                              \\
                                      & \quad +c_3c_1{\bm u}_2+c_2{\bm u}_1\times{\bm u}_3-c_2{\bm u}_1\cdot{\bm u}_3\nonumber                                                              \\
                                      & \quad +c_1c_2{\bm u}_3+c_3{\bm u}_1\times{\bm u}_2-c_3{\bm u}_1\cdot{\bm u}_2\, .
    \end{align}
    同理有
    \begin{align}
        {\bm q}_1({\bm q}_2{\bm q}_3) & =(c_1+{\bm u}_1)((c_2+{\bm u}_2)(c_3+{\bm u}_3))\nonumber                                                                                           \\
                                      & =(c_1+{\bm u}_1)((c_2c_3-{\bm u}_2\cdot{\bm u}_3)+(c_2{\bm u}_3+c_3{\bm u}_2+{\bm u}_2\times{\bm u}_3))\nonumber                                    \\
                                      & =\quad(c_1(c_2c_3-{\bm u}_2\cdot{\bm u}_3)-{\bm u}_1\cdot(c_2{\bm u}_3+c_3{\bm u}_2+{\bm u}_2\times{\bm u}_3))\nonumber                             \\
                                      & \quad +(c_1(c_2{\bm u}_3+c_3{\bm u}_2+{\bm u}_2\times{\bm u}_3)+(c_2c_3-{\bm u}_2\cdot{\bm u}_3){\bm u}_1\nonumber                                  \\
                                      & \quad +{\bm u}_1\times(c_2{\bm u}_3+c_3{\bm u}_2+{\bm u}_2\times{\bm u}_3))\nonumber                                                                \\
                                      & =\quad c_1c_2c_3-c_1{\bm u}_2\cdot{\bm u}_3-c_2{\bm u}_1\cdot{\bm u}_3-c_3{\bm u}_1\cdot{\bm u}_2-{\bm u}_1\cdot({\bm u}_2\times{\bm u}_3)\nonumber \\
                                      & \quad +c_1c_2{\bm u}_3+c_1c_3{\bm u}_2+c_1{\bm u}_2\times{\bm u}_3+c_2c_3{\bm u}_1-({\bm u}_2\cdot{\bm u}_3){\bm u}_1\nonumber                      \\
                                      & \quad +c_2{\bm u}_1\times{\bm u}_3+c_3{\bm u}_1\times{\bm u}_2+{\bm u}_1\times({\bm u}_2\times{\bm u}_3)\nonumber                                   \\
                                      & =\quad c_1c_2c_3-{\bm u}_1\cdot({\bm u}_2\times{\bm u}_3)+{\bm u}_1\times({\bm u}_2\times{\bm u}_3)-({\bm u}_2\cdot{\bm u}_3){\bm u}_1\nonumber     \\
                                      & \quad +c_2c_3{\bm u}_1+c_1{\bm u}_2\times{\bm u}_3-c_1{\bm u}_2\cdot{\bm u}_3\nonumber                                                              \\
                                      & \quad +c_3c_1{\bm u}_2+c_2{\bm u}_1\times{\bm u}_3-c_2{\bm u}_1\cdot{\bm u}_3\nonumber                                                              \\
                                      & \quad +c_1c_2{\bm u}_3+c_3{\bm u}_1\times{\bm u}_2-c_3{\bm u}_1\cdot{\bm u}_2\, .
    \end{align}
    注意到向量运算有\sidenote{它们其实是向量的\protect\keyindex{标量三重积}{scalar triple product}{}的两种等价定义,也称\protect\keyindex{混合积}{mixed product}{}。}
    \begin{align}
        ({\bm u}_1\times{\bm u}_2)\cdot{\bm u}_3\equiv{\bm u}_1\cdot({\bm u}_2\times{\bm u}_3)\, ,
    \end{align}
    以及\sidenote{第一个和第三个等号的变换利用了\protect\keyindex{向量三重积}{vector triple product}{}的性质,
        也称为\protect\keyindex{三重积展开}{triple product expansion}{}或\protect\keyindex{拉格朗日公式}{Lagrange's formula}{}。}
    \begin{align}
          & ({\bm u}_1\times{\bm u}_2)\times{\bm u}_3-({\bm u}_1\cdot{\bm u}_2){\bm u}_3\nonumber                             \\
        = & ({\bm u}_1\cdot{\bm u}_3){\bm u}_2-({\bm u}_2\cdot{\bm u}_3){\bm u}_1-({\bm u}_1\cdot{\bm u}_2){\bm u}_3\nonumber \\
        = & ({\bm u}_1\cdot{\bm u}_3){\bm u}_2-({\bm u}_1\cdot{\bm u}_2){\bm u}_3-({\bm u}_2\cdot{\bm u}_3){\bm u}_1\nonumber \\
        = & ({\bm u}_3\times{\bm u}_2)\times{\bm u}_1-({\bm u}_2\cdot{\bm u}_3){\bm u}_1\nonumber                             \\
        = & {\bm u}_1\times(-{\bm u}_3\times{\bm u}_2)-({\bm u}_2\cdot{\bm u}_3){\bm u}_1\nonumber                            \\
        = & {\bm u}_1\times({\bm u}_2\times{\bm u}_3)-({\bm u}_2\cdot{\bm u}_3){\bm u}_1\, .
    \end{align}
    所以$({\bm q}_1{\bm q}_2){\bm q}_3={\bm q}_1({\bm q}_2{\bm q}_3)$成立。
\end{prove}
\begin{proposition}
    四元数的乘法对加法满足分配律
    \begin{align}
        {\bm q}_1({\bm q}_2+{\bm q}_3)={\bm q}_1{\bm q}_2+{\bm q}_1{\bm q}_3, \quad \forall {\bm q}_1, {\bm q}_2, {\bm q}_3\in \mathbb{H}\, , \\
        ({\bm q}_1+{\bm q}_2){\bm q}_3={\bm q}_1{\bm q}_3+{\bm q}_2{\bm q}_3, \quad \forall {\bm q}_1, {\bm q}_2, {\bm q}_3\in \mathbb{H}\, .
    \end{align}
\end{proposition}
\begin{prove}
    根据定义有
    \begin{align}
        {\bm q}_1({\bm q}_2+{\bm q}_3) & =(c_1+{\bm u}_1)((c_2+{\bm u}_2)+(c_3+{\bm u}_3))\nonumber                                            \\
                                       & =(c_1+{\bm u}_1)((c_2+c_3)+({\bm u}_2+{\bm u}_3))\nonumber                                            \\
                                       & =\quad (c_1(c_2+c_3)-{\bm u}_1\cdot({\bm u}_2+{\bm u}_3))\nonumber                                    \\
                                       & \quad +(c_1({\bm u}_2+{\bm u}_3)+(c_2+c_3){\bm u}_1+{\bm u}_1\times({\bm u}_2+{\bm u}_3))\nonumber    \\
                                       & =\quad (c_1c_2-{\bm u}_1\cdot{\bm u}_2)+(c_1{\bm u}_2+c_2{\bm u}_1+{\bm u}_1\times{\bm u}_2)\nonumber \\
                                       & \quad +(c_1c_3-{\bm u}_1\cdot{\bm u}_3)+(c_1{\bm u}_3+c_3{\bm u}_1{\bm u}_1\times{\bm u}_3)\nonumber  \\
                                       & ={\bm q}_1{\bm q}_2+{\bm q}_1{\bm q}_3\, .
    \end{align}
    \begin{align}
        ({\bm q}_1+{\bm q}_2){\bm q}_3 & =((c_1+{\bm u}_1)+(c_2+{\bm u}_2))(c_3+{\bm u}_3)\nonumber                                            \\
                                       & =((c_1+c_2)+({\bm u}_1+{\bm u}_2))(c_3+{\bm u}_3)\nonumber                                            \\
                                       & =\quad ((c_1+c_2)c_3-({\bm u}_1+{\bm u}_2)\cdot{\bm u}_3)\nonumber                                    \\
                                       & \quad +((c_1+c_2){\bm u}_3+c_3({\bm u}_1+{\bm u}_2)+({\bm u}_1+{\bm u}_2)\times{\bm u}_3)\nonumber    \\
                                       & =\quad (c_1c_3-{\bm u}_1\cdot{\bm u}_3)+(c_1{\bm u}_3+c_3{\bm u}_1+{\bm u}_1\times{\bm u}_3)\nonumber \\
                                       & \quad +(c_2c_3-{\bm u}_2\cdot{\bm u}_3)+(c_2{\bm u}_3+c_3{\bm u}_2+{\bm u}_2\times{\bm u}_3)\nonumber \\
                                       & ={\bm q}_1{\bm q}_3+{\bm q}_2{\bm q}_3\, .
    \end{align}
\end{prove}
\begin{definition}
    称四元数${\bm q}=c+{\bm u}$与$\bar{\bm q}=c-{\bm u}$互为\keyindex{共轭}{conjugate}{}。
\end{definition}
\begin{proposition}
    四元数积的共轭等于逆序共轭的积,即
    \begin{align}
        \overline{{\bm q}_1{\bm q}_2}=\bar{\bm q}_2\bar{\bm q}_1, \quad\forall {\bm q}_1, {\bm q}_2\in\mathbb{H}\, .
    \end{align}
\end{proposition}
\begin{prove}
    根据定义有
    \begin{align}
        \overline{{\bm q}_1{\bm q}_2} & =(c_1c_2-{\bm u}_1\cdot{\bm u}_2)-(c_1{\bm u}_2+c_2{\bm u}_1+{\bm u}_1\times{\bm u}_2)\nonumber                   \\
                                      & =(c_2c_1-(-{\bm u}_2)\cdot(-{\bm u}_1))+(c_2(-{\bm u}_1)+c_1(-{\bm u}_2)+(-{\bm u}_2)\times(-{\bm u}_1))\nonumber \\
                                      & =(c_2-{\bm u}_2)(c_1-{\bm u}_1)\nonumber                                                                          \\
                                      & =\bar{\bm q}_2\bar{\bm q}_1\, .
    \end{align}
\end{prove}
\begin{definition}
    四元数的\keyindex{模}{magnitude}{}为
    \begin{align}
        \|{\bm q}\|=\sqrt{{\bm q}\bar{\bm q}}=\sqrt{\bar{\bm q}{\bm q}}=\sqrt{c^2+x^2+y^2+z^2}\, .
    \end{align}
\end{definition}
\begin{definition}
    模为$1$的四元数${\bm q}$称为\keyindex{单位四元数}{unit quaternion}{quaternion四元数}。
\end{definition}
\begin{definition}
    称$、\displaystyle\frac{1}{\|{\bm q}\|}{\bm q}$为非零四元数${\bm q}$的\keyindex{规范化四元数}{versor}{quaternion四元数}。
\end{definition}
\begin{proposition}
    当向量四元数${\bm q}$是单位四元数即$c=0$且$x^2+y^2+z^2=1$时,有
    \begin{align}
        {\bm q}^2={\bm q}{\bm q}=-1\, .
    \end{align}
\end{proposition}
\begin{prove}
    依据哈密顿积定义有
    \begin{align}
        {\bm q}^2={\bm q}{\bm q} & =(c^2-{\bm u}\cdot{\bm u})+(2c{\bm u}+{\bm u}\times{\bm u})\nonumber \\
                                 & =-{\bm u}\cdot{\bm u}\nonumber                                       \\
                                 & =-(x^2+y^2+z^2)\nonumber                                             \\
                                 & =-1\, .
    \end{align}
\end{prove}
\begin{proposition}
    四元数取共轭后模不变,即
    \begin{align}
        \|\bar{\bm q}\|=\|{\bm q}\|\, .
    \end{align}
\end{proposition}
\begin{proposition}
    在四元数的数乘中有
    \begin{align}
        \|\lambda{\bm q}\|=|\lambda|\|{\bm q}\|, \quad \forall \lambda\in\mathbb{R}, \forall {\bm q}\in\mathbb{H}\, .
    \end{align}
\end{proposition}
\begin{proposition}
    在四元数的乘法中有
    \begin{align}
        \|{\bm q}_1{\bm q}_2\|=\|{\bm q}_1\|\|{\bm q}_2\|,\quad \forall {\bm q}_1{\bm q}_2\in\mathbb{H}\, .
    \end{align}
\end{proposition}
\begin{prove}
    根据定义可得
    \begin{align}
        \|{\bm q}_1{\bm q}_2\|^2 = & \|(c_1c_2-x_1x_2-y_1y_2-z_1z_2)\nonumber                    \\
                                   & +(c_1x_2+c_2x_1+y_1z_2-y_2z_1)\mathbf{i}\nonumber           \\
                                   & +(c_1y_2+c_2y_1+z_1x_2-z_2x_1)\mathbf{j}\nonumber           \\
                                   & +(c_1z_2+c_2z_1+x_1y_2-x_2y_1)\mathbf{k}\|^2\nonumber       \\
        =                          & (c_1c_2-x_1x_2-y_1y_2-z_1z_2)^2\nonumber                    \\
                                   & +(c_1x_2+c_2x_1+y_1z_2-y_2z_1)^2\nonumber                   \\
                                   & +(c_1y_2+c_2y_1+z_1x_2-z_2x_1)^2\nonumber                   \\
                                   & +(c_1z_2+c_2z_1+x_1y_2-x_2y_1)^2\nonumber                   \\
        =                          & (c_1^2+x_1^2+y_1^2+z_1^2)(c_2^2+x_2^2+y_2^2+z_2^2)\nonumber \\
        =                          & \|{\bm q}_1\|^2\|{\bm q}_2\|^2\, ,
    \end{align}
    所以$\|{\bm q}_1{\bm q}_2\|=\|{\bm q}_1\|\|{\bm q}_2\|$。
\end{prove}

\begin{proposition}
    四元数${\bm q}$与其共轭的积恒有
    \begin{align}
        {\bm q}\bar{\bm q}=\bar{\bm q}{\bm q}=\|{\bm q}\|^2\, .
    \end{align}
\end{proposition}
\begin{prove}
    根据定义有
    \begin{align}
        {\bm q}\bar{\bm q} & =(c+{\bm u})(c-{\bm u})\nonumber                                                      \\
                           & =(c^2-{\bm u}\cdot(-{\bm u}))+(c(-{\bm u})+c{\bm u}+{\bm u}\times(-{\bm u}))\nonumber \\
                           & =c^2+{\bm u}\cdot{\bm u}\nonumber                                                     \\
                           & =\|{\bm q}\|^2\, .
    \end{align}
    同理有$\bar{\bm q}{\bm q}=\|{\bm q}\|^2$。
\end{prove}
\begin{definition}
    非零四元数${\bm q}$的\keyindex{逆}{inverse}{}是
    \begin{align}
        {\bm q}^{-1}=\frac{1}{\|{\bm q}\|^2}\bar{\bm q}\, .
    \end{align}
    它是哈密顿积的逆元。
\end{definition}
\begin{corollary}
    对于非零四元数${\bm q}$有
    \begin{align}
        {\bm q}{\bm q}^{-1}={\bm q}^{-1}{\bm q}=1\, .
    \end{align}
\end{corollary}
\begin{corollary}
    单位四元数的逆与共轭相等,即
    \begin{align}
        \|{\bm q}\|=1 \Rightarrow {\bm q}^{-1}=\bar{\bm q}\, .
    \end{align}
\end{corollary}

\subsection{四元数与旋转变换}\label{sub:四元数与旋转变换}
如\reffig{2.ex1}所示,点$P$绕单位向量$\bm n$给定的轴顺时针旋转角度$\theta$得到点$P'$。
我们现在说明该一般旋转变换和四元数运算的关系。
\begin{figure}[htbp]
    \centering%LaTeX with PSTricks extensions
%%Creator: Inkscape 1.0.1 (3bc2e813f5, 2020-09-07)
%%Please note this file requires PSTricks extensions
\psset{xunit=.25pt,yunit=.25pt,runit=.25pt}
\begin{pspicture}(438.27934686,472.27909118)
{
\newrgbcolor{curcolor}{0 0 0}
\pscustom[linewidth=4.04194013,linecolor=curcolor,linestyle=dashed,dash=6.41655016 2.13884997]
{
\newpath
\moveto(319.93053732,53.44303071)
\lineto(186.29461795,409.0229337)
}
}
{
\newrgbcolor{curcolor}{0 0 0}
\pscustom[linestyle=none,fillstyle=solid,fillcolor=curcolor]
{
\newpath
\moveto(191.98244641,393.88869807)
\lineto(202.39347846,389.16549449)
\lineto(184.87266084,412.80649261)
\lineto(187.25924283,383.47766603)
\closepath
}
}
{
\newrgbcolor{curcolor}{0 0 0}
\pscustom[linewidth=2.15570147,linecolor=curcolor]
{
\newpath
\moveto(191.98244641,393.88869807)
\lineto(202.39347846,389.16549449)
\lineto(184.87266084,412.80649261)
\lineto(187.25924283,383.47766603)
\closepath
}
}
{
\newrgbcolor{curcolor}{0 0 0}
\pscustom[linewidth=4.14531002,linecolor=curcolor]
{
\newpath
\moveto(184.55716913,413.7263252)
\lineto(411.21262866,413.7263252)
}
}
{
\newrgbcolor{curcolor}{0 0 0}
\pscustom[linestyle=none,fillstyle=solid,fillcolor=curcolor]
{
\newpath
\moveto(394.63138859,413.7263252)
\lineto(386.34076855,405.43570516)
\lineto(415.35793868,413.7263252)
\lineto(386.34076855,422.01694524)
\closepath
}
}
{
\newrgbcolor{curcolor}{0 0 0}
\pscustom[linewidth=2.21083208,linecolor=curcolor]
{
\newpath
\moveto(394.63138859,413.7263252)
\lineto(386.34076855,405.43570516)
\lineto(415.35793868,413.7263252)
\lineto(386.34076855,422.01694524)
\closepath
}
}
{
\newrgbcolor{curcolor}{0 0 0}
\pscustom[linewidth=4.11835988,linecolor=curcolor]
{
\newpath
\moveto(417.52037092,412.81663857)
\curveto(390.12953195,356.63810721)(335.92909659,307.69617224)(267.1004223,276.99036162)
\curveto(198.27174801,246.284551)(120.58876958,236.39072937)(51.51046353,249.53249838)
}
}
{
\newrgbcolor{curcolor}{0 0 0}
\pscustom[linewidth=4.33349279,linecolor=curcolor]
{
\newpath
\moveto(184.47597354,413.93886693)
\lineto(56.00338016,254.5407315)
}
}
{
\newrgbcolor{curcolor}{0 0 0}
\pscustom[linestyle=none,fillstyle=solid,fillcolor=curcolor]
{
\newpath
\moveto(66.88101129,268.03679377)
\lineto(65.57179572,280.22364048)
\lineto(53.28397237,251.16671593)
\lineto(79.067858,269.34600934)
\closepath
}
}
{
\newrgbcolor{curcolor}{0 0 0}
\pscustom[linewidth=2.31119622,linecolor=curcolor]
{
\newpath
\moveto(66.88101129,268.03679377)
\lineto(65.57179572,280.22364048)
\lineto(53.28397237,251.16671593)
\lineto(79.067858,269.34600934)
\closepath
}
}
{
\newrgbcolor{curcolor}{0 0 0}
\pscustom[linewidth=4.03604417,linecolor=curcolor,linestyle=dashed,dash=6.40720987 2.13574004]
{
\newpath
\moveto(319.95211843,53.41566693)
\lineto(57.08415496,243.20679764)
}
}
{
\newrgbcolor{curcolor}{0 0 0}
\pscustom[linestyle=none,fillstyle=solid,fillcolor=curcolor]
{
\newpath
\moveto(70.17327323,233.75643129)
\lineto(81.44301553,235.57580725)
\lineto(53.81187539,245.56938923)
\lineto(71.99264919,222.48668898)
\closepath
}
}
{
\newrgbcolor{curcolor}{0 0 0}
\pscustom[linewidth=2.15255696,linecolor=curcolor]
{
\newpath
\moveto(70.17327323,233.75643129)
\lineto(81.44301553,235.57580725)
\lineto(53.81187539,245.56938923)
\lineto(71.99264919,222.48668898)
\closepath
}
}
{
\newrgbcolor{curcolor}{0 0 0}
\pscustom[linewidth=4.09065817,linecolor=curcolor,linestyle=dashed,dash=6.49390984 2.16463995]
{
\newpath
\moveto(319.87335307,53.45697717)
\lineto(325.14689008,302.93138347)
}
}
{
\newrgbcolor{curcolor}{0 0 0}
\pscustom[linestyle=none,fillstyle=solid,fillcolor=curcolor]
{
\newpath
\moveto(324.80108436,286.57240529)
\lineto(332.80767059,278.22001334)
\lineto(325.23334151,307.02112801)
\lineto(316.44869241,278.56581906)
\closepath
}
}
{
\newrgbcolor{curcolor}{0 0 0}
\pscustom[linewidth=2.18168442,linecolor=curcolor]
{
\newpath
\moveto(324.80108436,286.57240529)
\lineto(332.80767059,278.22001334)
\lineto(325.23334151,307.02112801)
\lineto(316.44869241,278.56581906)
\closepath
}
}
{
\newrgbcolor{curcolor}{0 0 0}
\pscustom[linewidth=3.99711489,linecolor=curcolor]
{
\newpath
\moveto(184.4355326,414.00131984)
\lineto(321.50663811,314.29944646)
}
}
{
\newrgbcolor{curcolor}{0 0 0}
\pscustom[linestyle=none,fillstyle=solid,fillcolor=curcolor]
{
\newpath
\moveto(308.57681303,323.70425704)
\lineto(297.40949519,321.94174979)
\lineto(324.73909438,311.94824381)
\lineto(306.81430577,334.87157487)
\closepath
}
}
{
\newrgbcolor{curcolor}{0 0 0}
\pscustom[linewidth=2.13179467,linecolor=curcolor]
{
\newpath
\moveto(308.57681303,323.70425704)
\lineto(297.40949519,321.94174979)
\lineto(324.73909438,311.94824381)
\lineto(306.81430577,334.87157487)
\closepath
}
}
{
\newrgbcolor{curcolor}{0 0 0}
\pscustom[linewidth=5.66929134,linecolor=curcolor]
{
\newpath
\moveto(319.88601449,53.5031252)
\lineto(271.92800504,181.15954095)
}
}
{
\newrgbcolor{curcolor}{0 0 0}
\pscustom[linestyle=none,fillstyle=solid,fillcolor=curcolor]
{
\newpath
\moveto(279.90315055,159.9310016)
\lineto(294.50499297,153.30430468)
\lineto(269.93421866,186.46667578)
\lineto(273.27645363,145.32915918)
\closepath
}
}
{
\newrgbcolor{curcolor}{0 0 0}
\pscustom[linewidth=3.02362214,linecolor=curcolor]
{
\newpath
\moveto(279.90315055,159.9310016)
\lineto(294.50499297,153.30430468)
\lineto(269.93421866,186.46667578)
\lineto(273.27645363,145.32915918)
\closepath
}
}
{
\newrgbcolor{curcolor}{0 0 0}
\pscustom[linestyle=none,fillstyle=solid,fillcolor=curcolor]
{
\newpath
\moveto(191.90295935,354.97039515)
\curveto(191.90295935,357.94677311)(191.12343179,364.25385972)(186.51713258,364.25385972)
\curveto(180.21004597,364.25385972)(173.26516408,351.49795421)(173.26516408,341.15149752)
\curveto(173.26516408,336.89952901)(174.54075463,331.86803295)(178.65099085,331.86803295)
\curveto(185.0289436,331.86803295)(191.90295935,344.83653689)(191.90295935,354.97039515)
\closepath
\moveto(178.01319557,348.80504082)
\curveto(178.79272313,351.63968649)(179.71398297,355.25385972)(181.55650266,358.51370224)
\curveto(182.76122707,360.71055263)(184.39114833,363.26173374)(186.44626644,363.26173374)
\curveto(188.64311683,363.26173374)(188.9265814,360.35622193)(188.9265814,357.80504082)
\curveto(188.9265814,355.53732429)(188.57225069,353.26960775)(187.50925857,348.80504082)
\closepath
\moveto(187.08406172,347.31685185)
\curveto(186.58799872,345.26173374)(185.66673888,341.43496208)(183.89508534,338.17511956)
\curveto(182.33603022,335.05700933)(180.63524282,332.86015893)(178.65099085,332.86015893)
\curveto(177.16280187,332.86015893)(176.24154203,334.20661563)(176.24154203,338.38771799)
\curveto(176.24154203,340.30110382)(176.5250066,342.92315106)(177.65886486,347.31685185)
\closepath
\moveto(187.08406172,347.31685185)
}
}
{
\newrgbcolor{curcolor}{0 0 0}
\pscustom[linestyle=none,fillstyle=solid,fillcolor=curcolor]
{
\newpath
\moveto(30.73228219,225.2181932)
\lineto(38.45669164,225.2181932)
\curveto(44.83464439,225.2181932)(51.14173101,229.89535856)(51.14173101,234.92685462)
\curveto(51.14173101,238.47016171)(48.16535305,241.80087037)(42.21259715,241.80087037)
\lineto(27.61417195,241.80087037)
\curveto(26.76377825,241.80087037)(26.26771526,241.80087037)(26.26771526,240.95047667)
\curveto(26.26771526,240.38354753)(26.62204597,240.38354753)(27.54330581,240.38354753)
\curveto(28.11023494,240.38354753)(28.96062864,240.31268139)(29.45669164,240.31268139)
\curveto(30.2362192,240.17094911)(30.44881762,240.10008297)(30.44881762,239.53315383)
\curveto(30.44881762,239.39142155)(30.44881762,239.24968926)(30.30708534,238.68276013)
\lineto(24.21259715,214.44653966)
\curveto(23.7874003,212.67488612)(23.71653416,212.32055541)(20.10236093,212.32055541)
\curveto(19.39369951,212.32055541)(18.89763652,212.32055541)(18.89763652,211.47016171)
\curveto(18.89763652,210.90323257)(19.39369951,210.90323257)(19.53543179,210.90323257)
\curveto(20.81102234,210.90323257)(23.99999872,211.04496486)(25.27558927,211.04496486)
\curveto(26.26771526,211.04496486)(27.25984124,210.97409871)(28.18110109,210.97409871)
\curveto(29.17322707,210.97409871)(30.16535305,210.90323257)(31.0866129,210.90323257)
\curveto(31.4409436,210.90323257)(32.00787274,210.90323257)(32.00787274,211.82449241)
\curveto(32.00787274,212.32055541)(31.58267589,212.32055541)(30.73228219,212.32055541)
\curveto(29.10236093,212.32055541)(27.82677038,212.32055541)(27.82677038,213.10008297)
\curveto(27.82677038,213.38354753)(27.89763652,213.59614596)(27.96850266,213.87961052)
\closepath
\moveto(33.99212471,238.68276013)
\curveto(34.41732156,240.24181525)(34.4881877,240.38354753)(36.47243967,240.38354753)
\lineto(40.79527431,240.38354753)
\curveto(44.55117983,240.38354753)(46.96062864,239.17882312)(46.96062864,236.06071289)
\curveto(46.96062864,234.28905934)(46.0393688,230.39142155)(44.26771526,228.76150029)
\curveto(41.99999872,226.70638218)(39.30708534,226.35205147)(37.32283337,226.35205147)
\lineto(30.94488061,226.35205147)
\closepath
\moveto(33.99212471,238.68276013)
}
}
{
\newrgbcolor{curcolor}{0 0 0}
\pscustom[linestyle=none,fillstyle=solid,fillcolor=curcolor]
{
\newpath
\moveto(348.62542398,294.50255352)
\lineto(356.34983343,294.50255352)
\curveto(362.72778619,294.50255352)(369.0348728,299.17971887)(369.0348728,304.21121493)
\curveto(369.0348728,307.75452202)(366.05849485,311.08523068)(360.10573894,311.08523068)
\lineto(345.50731375,311.08523068)
\curveto(344.65692005,311.08523068)(344.16085705,311.08523068)(344.16085705,310.23483698)
\curveto(344.16085705,309.66790785)(344.51518776,309.66790785)(345.4364476,309.66790785)
\curveto(346.00337674,309.66790785)(346.85377044,309.59704171)(347.34983343,309.59704171)
\curveto(348.12936099,309.45530942)(348.34195942,309.38444328)(348.34195942,308.81751415)
\curveto(348.34195942,308.67578186)(348.34195942,308.53404958)(348.20022713,307.96712045)
\lineto(342.10573894,283.73089997)
\curveto(341.68054209,281.95924643)(341.60967595,281.60491572)(337.99550272,281.60491572)
\curveto(337.28684131,281.60491572)(336.79077831,281.60491572)(336.79077831,280.75452202)
\curveto(336.79077831,280.18759289)(337.28684131,280.18759289)(337.42857359,280.18759289)
\curveto(338.70416414,280.18759289)(341.89314052,280.32932517)(343.16873107,280.32932517)
\curveto(344.16085705,280.32932517)(345.15298304,280.25845903)(346.07424288,280.25845903)
\curveto(347.06636886,280.25845903)(348.05849485,280.18759289)(348.97975469,280.18759289)
\curveto(349.3340854,280.18759289)(349.90101453,280.18759289)(349.90101453,281.10885273)
\curveto(349.90101453,281.60491572)(349.47581768,281.60491572)(348.62542398,281.60491572)
\curveto(346.99550272,281.60491572)(345.71991217,281.60491572)(345.71991217,282.38444328)
\curveto(345.71991217,282.66790785)(345.79077831,282.88050627)(345.86164446,283.16397084)
\closepath
\moveto(351.8852665,307.96712045)
\curveto(352.31046335,309.52617556)(352.38132949,309.66790785)(354.36558146,309.66790785)
\lineto(358.68841611,309.66790785)
\curveto(362.44432162,309.66790785)(364.85377044,308.46318344)(364.85377044,305.3450732)
\curveto(364.85377044,303.57341966)(363.9325106,299.67578186)(362.16085705,298.0458606)
\curveto(359.89314052,295.99074249)(357.20022713,295.63641178)(355.21597516,295.63641178)
\lineto(348.83802241,295.63641178)
\closepath
\moveto(351.8852665,307.96712045)
}
}
{
\newrgbcolor{curcolor}{0 0 0}
\pscustom[linestyle=none,fillstyle=solid,fillcolor=curcolor]
{
\newpath
\moveto(379.37565467,313.79911325)
\curveto(379.65911923,314.29517624)(379.65911923,314.57864081)(379.65911923,314.79123923)
\curveto(379.65911923,315.78336522)(378.80872553,316.49202663)(377.81659955,316.49202663)
\curveto(376.61187514,316.49202663)(376.25754443,315.49990065)(376.11581215,315.00383766)
\lineto(371.93470978,301.3266723)
\curveto(371.86384364,301.25580616)(371.72211136,300.90147545)(371.72211136,300.83060931)
\curveto(371.72211136,300.4762786)(372.71423734,300.12194789)(372.99770191,300.12194789)
\curveto(373.21030034,300.12194789)(373.21030034,300.19281404)(373.42289876,300.68887703)
\closepath
\moveto(379.37565467,313.79911325)
}
}
{
\newrgbcolor{curcolor}{0 0 0}
\pscustom[linestyle=none,fillstyle=solid,fillcolor=curcolor]
{
\newpath
\moveto(352.38538266,39.59055226)
\curveto(352.38538266,46.88976486)(347.56648502,51.77952863)(340.83420156,51.77952863)
\curveto(331.12554014,51.77952863)(321.13341416,41.50393808)(321.13341416,30.94488297)
\curveto(321.13341416,23.43307194)(326.23577636,18.89763887)(332.7554614,18.89763887)
\curveto(342.32239053,18.89763887)(352.38538266,28.81889871)(352.38538266,39.59055226)
\closepath
\moveto(332.96805983,20.10236328)
\curveto(328.5034929,20.10236328)(325.38538266,23.71653651)(325.38538266,29.66929241)
\curveto(325.38538266,31.72441052)(326.02317794,38.31496171)(329.49561888,43.55905619)
\curveto(332.61372912,48.30708769)(337.0074299,50.64567037)(340.62160313,50.64567037)
\curveto(344.3066425,50.64567037)(348.34601258,48.09448926)(348.34601258,41.3622058)
\curveto(348.34601258,38.10236328)(347.14128817,31.08661525)(342.67672124,25.48819005)
\curveto(340.47987085,22.72441052)(336.79483148,20.10236328)(332.96805983,20.10236328)
\closepath
\moveto(332.96805983,20.10236328)
}
}
{
\newrgbcolor{curcolor}{0 0 0}
\pscustom[linestyle=none,fillstyle=solid,fillcolor=curcolor]
{
\newpath
\moveto(184.68589479,449.90901273)
\curveto(185.11109164,451.53893399)(185.18195778,451.96413084)(188.4418003,451.96413084)
\curveto(189.64652471,451.96413084)(190.00085542,451.96413084)(190.00085542,452.88539068)
\curveto(190.00085542,453.38145367)(189.50479242,453.38145367)(189.36306014,453.38145367)
\curveto(188.08746959,453.38145367)(184.82762707,453.23972139)(183.55203652,453.23972139)
\curveto(182.27644597,453.23972139)(179.08746959,453.38145367)(177.7410129,453.38145367)
\curveto(177.38668219,453.38145367)(176.8906192,453.38145367)(176.8906192,452.46019383)
\curveto(176.8906192,451.96413084)(177.31581605,451.96413084)(178.16620975,451.96413084)
\curveto(178.23707589,451.96413084)(179.08746959,451.96413084)(179.86699715,451.8932647)
\curveto(180.71739085,451.75153241)(181.07172156,451.75153241)(181.07172156,451.11373714)
\curveto(181.07172156,450.97200486)(181.07172156,450.90113871)(180.92998927,450.26334344)
\lineto(178.23707589,439.27909147)
\lineto(164.41817825,439.27909147)
\lineto(167.11109164,449.90901273)
\curveto(167.46542234,451.53893399)(167.60715463,451.96413084)(170.86699715,451.96413084)
\curveto(172.07172156,451.96413084)(172.42605227,451.96413084)(172.42605227,452.88539068)
\curveto(172.42605227,453.38145367)(171.92998927,453.38145367)(171.78825699,453.38145367)
\curveto(170.51266644,453.38145367)(167.25282392,453.23972139)(165.97723337,453.23972139)
\curveto(164.70164282,453.23972139)(161.51266644,453.38145367)(160.16620975,453.38145367)
\curveto(159.81187904,453.38145367)(159.31581605,453.38145367)(159.31581605,452.46019383)
\curveto(159.31581605,451.96413084)(159.7410129,451.96413084)(160.5914066,451.96413084)
\curveto(160.66227274,451.96413084)(161.51266644,451.96413084)(162.292194,451.8932647)
\curveto(163.07172156,451.75153241)(163.49691841,451.75153241)(163.49691841,451.11373714)
\curveto(163.49691841,450.97200486)(163.49691841,450.83027257)(163.35518612,450.26334344)
\lineto(157.26069794,426.02712297)
\curveto(156.83550109,424.25546942)(156.76463494,423.90113871)(153.15046172,423.90113871)
\curveto(152.37093416,423.90113871)(151.94573731,423.90113871)(151.94573731,422.97987887)
\curveto(151.94573731,422.48381588)(152.51266644,422.48381588)(152.58353258,422.48381588)
\curveto(153.85912313,422.48381588)(157.04809951,422.62554816)(158.32369006,422.62554816)
\curveto(159.2449499,422.62554816)(160.23707589,422.55468202)(161.22920187,422.55468202)
\curveto(162.22132786,422.55468202)(163.21345384,422.48381588)(164.13471368,422.48381588)
\curveto(164.48904439,422.48381588)(165.05597353,422.48381588)(165.05597353,423.40507572)
\curveto(165.05597353,423.90113871)(164.63077668,423.90113871)(163.78038297,423.90113871)
\curveto(162.07959557,423.90113871)(160.87487116,423.90113871)(160.87487116,424.68066627)
\curveto(160.87487116,424.96413084)(160.94573731,425.17672926)(160.94573731,425.46019383)
\lineto(164.06384754,437.86176863)
\lineto(177.81187904,437.86176863)
\curveto(175.96935935,430.34995761)(174.90636723,426.02712297)(174.6937688,425.38932769)
\curveto(174.26857195,423.90113871)(173.41817825,423.90113871)(170.58353258,423.90113871)
\curveto(169.94573731,423.90113871)(169.52054046,423.90113871)(169.52054046,422.97987887)
\curveto(169.52054046,422.48381588)(170.08746959,422.48381588)(170.15833573,422.48381588)
\curveto(171.43392628,422.48381588)(174.62290266,422.62554816)(175.89849321,422.62554816)
\curveto(176.81975305,422.62554816)(177.81187904,422.55468202)(178.80400502,422.55468202)
\curveto(179.79613101,422.55468202)(180.78825699,422.48381588)(181.70951683,422.48381588)
\curveto(182.06384754,422.48381588)(182.63077668,422.48381588)(182.63077668,423.40507572)
\curveto(182.63077668,423.90113871)(182.20557983,423.90113871)(181.35518612,423.90113871)
\curveto(179.72526486,423.90113871)(178.44967431,423.90113871)(178.44967431,424.68066627)
\curveto(178.44967431,424.96413084)(178.52054046,425.17672926)(178.5914066,425.46019383)
\closepath
\moveto(184.68589479,449.90901273)
}
}
{
\newrgbcolor{curcolor}{0 0 0}
\pscustom[linestyle=none,fillstyle=solid,fillcolor=curcolor]
{
\newpath
\moveto(304.67515866,441.70371887)
\curveto(304.67515866,445.81395509)(301.62791457,445.81395509)(301.62791457,445.81395509)
\curveto(299.78539488,445.81395509)(298.15547362,443.90056926)(298.15547362,442.41238029)
\curveto(298.15547362,441.13678974)(299.1475996,440.5698606)(299.50193031,440.35726218)
\curveto(301.41531614,439.22340391)(301.76964685,438.37301021)(301.76964685,437.52261651)
\curveto(301.76964685,436.53049052)(299.1475996,426.60923068)(293.90350512,426.60923068)
\curveto(290.6436626,426.60923068)(290.6436626,429.30214407)(290.6436626,430.15253777)
\curveto(290.6436626,432.77458501)(291.91925315,436.03442753)(293.33657598,439.64860076)
\curveto(293.69090669,440.5698606)(293.83263897,440.99505745)(293.83263897,441.70371887)
\curveto(293.83263897,444.32576612)(291.21059173,445.74308895)(288.73027677,445.74308895)
\curveto(283.91137913,445.74308895)(281.6436626,439.64860076)(281.6436626,438.72734092)
\curveto(281.6436626,438.08954564)(282.35232401,438.08954564)(282.77752086,438.08954564)
\curveto(283.27358386,438.08954564)(283.62791457,438.08954564)(283.76964685,438.65647478)
\curveto(285.25783583,443.54623856)(287.66728464,444.11316769)(288.4468122,444.11316769)
\curveto(288.80114291,444.11316769)(289.22633976,444.11316769)(289.22633976,443.19190785)
\curveto(289.22633976,442.12891572)(288.65941063,440.85332517)(288.51767834,440.49899446)
\curveto(286.46256023,435.25489997)(285.75389882,433.19978186)(285.75389882,431.00293147)
\curveto(285.75389882,426.25489997)(289.65153661,424.97930942)(293.62004055,424.97930942)
\curveto(301.41531614,424.97930942)(304.67515866,437.80608108)(304.67515866,441.70371887)
\closepath
\moveto(304.67515866,441.70371887)
}
}
{
\newrgbcolor{curcolor}{0 0 0}
\pscustom[linestyle=none,fillstyle=solid,fillcolor=curcolor]
{
\newpath
\moveto(112.6411603,350.28383049)
\curveto(112.99549101,351.55942104)(113.491554,353.75627144)(113.491554,354.03973601)
\curveto(113.491554,355.03186199)(112.78289258,356.02398797)(111.36556975,356.02398797)
\curveto(110.65690833,356.02398797)(109.02698707,355.66965726)(108.38919179,353.61453915)
\curveto(108.24745951,353.04761002)(106.1923414,344.82713758)(105.83801069,343.3389486)
\curveto(105.55454612,342.27595648)(105.27108156,341.00036593)(105.12934927,340.14997223)
\curveto(104.34982172,339.0869801)(102.57816817,337.24446041)(100.23958549,337.24446041)
\curveto(97.61753825,337.24446041)(97.54667211,339.51217695)(97.54667211,340.50430293)
\curveto(97.54667211,343.26808246)(98.96399494,346.81138955)(100.23958549,350.07123207)
\curveto(100.66478234,351.27595648)(100.80651463,351.55942104)(100.80651463,352.3389486)
\curveto(100.80651463,354.96099585)(98.18446738,356.37831868)(95.70415242,356.37831868)
\curveto(90.88525479,356.37831868)(88.61753825,350.28383049)(88.61753825,349.36257065)
\curveto(88.61753825,348.72477538)(89.32619967,348.72477538)(89.75139652,348.72477538)
\curveto(90.24745951,348.72477538)(90.60179022,348.72477538)(90.7435225,349.29170451)
\curveto(92.23171148,354.32320057)(94.71202644,354.74839742)(95.42068786,354.74839742)
\curveto(95.77501857,354.74839742)(96.20021542,354.74839742)(96.20021542,353.82713758)
\curveto(96.20021542,352.76414545)(95.63328628,351.4885549)(95.491554,350.99249191)
\curveto(93.57816817,346.24446041)(92.72777447,343.69327931)(92.72777447,341.49642892)
\curveto(92.72777447,336.32320057)(97.26320754,335.61453915)(99.88525479,335.61453915)
\curveto(101.23171148,335.61453915)(103.21596345,335.75627144)(105.62541227,338.09485412)
\curveto(107.11360124,335.89800372)(109.73564849,335.61453915)(110.79864061,335.61453915)
\curveto(112.49942801,335.61453915)(113.77501857,336.535799)(114.76714455,338.16572026)
\curveto(115.83013668,340.00823994)(116.46793195,342.41768876)(116.46793195,342.63028719)
\curveto(116.46793195,343.26808246)(115.75927053,343.26808246)(115.33407368,343.26808246)
\curveto(114.83801069,343.26808246)(114.69627841,343.26808246)(114.48367998,343.05548404)
\curveto(114.3419477,342.98461789)(114.3419477,342.91375175)(114.12934927,341.77989349)
\curveto(113.20808943,338.2365864)(112.21596345,337.24446041)(111.01123904,337.24446041)
\curveto(110.37344376,337.24446041)(110.01911305,337.66965726)(110.01911305,338.87438167)
\curveto(110.01911305,339.65390923)(110.16084534,340.43343679)(110.58604219,342.20509034)
\curveto(110.9403729,343.48068089)(111.36556975,345.25233443)(111.64903431,346.24446041)
\closepath
\moveto(112.6411603,350.28383049)
}
}
{
\newrgbcolor{curcolor}{0 0 0}
\pscustom[linestyle=none,fillstyle=solid,fillcolor=curcolor]
{
\newpath
\moveto(292.26190701,371.60798999)
\curveto(292.61623772,372.88358054)(293.11230071,375.08043093)(293.11230071,375.3638955)
\curveto(293.11230071,376.35602149)(292.40363929,377.34814747)(290.98631646,377.34814747)
\curveto(290.27765504,377.34814747)(288.64773378,376.99381676)(288.0099385,374.93869865)
\curveto(287.86820622,374.37176952)(285.81308811,366.15129708)(285.4587574,364.6631081)
\curveto(285.17529283,363.60011597)(284.89182827,362.32452542)(284.75009598,361.47413172)
\curveto(283.97056842,360.4111396)(282.19891488,358.56861991)(279.8603322,358.56861991)
\curveto(277.23828496,358.56861991)(277.16741882,360.83633645)(277.16741882,361.82846243)
\curveto(277.16741882,364.59224196)(278.58474165,368.13554904)(279.8603322,371.39539156)
\curveto(280.28552905,372.60011597)(280.42726134,372.88358054)(280.42726134,373.6631081)
\curveto(280.42726134,376.28515534)(277.80521409,377.70247818)(275.32489913,377.70247818)
\curveto(270.50600149,377.70247818)(268.23828496,371.60798999)(268.23828496,370.68673015)
\curveto(268.23828496,370.04893487)(268.94694638,370.04893487)(269.37214323,370.04893487)
\curveto(269.86820622,370.04893487)(270.22253693,370.04893487)(270.36426921,370.61586401)
\curveto(271.85245819,375.64736007)(274.33277315,376.07255692)(275.04143457,376.07255692)
\curveto(275.39576527,376.07255692)(275.82096212,376.07255692)(275.82096212,375.15129708)
\curveto(275.82096212,374.08830495)(275.25403299,372.8127144)(275.11230071,372.31665141)
\curveto(273.19891488,367.56861991)(272.34852118,365.01743881)(272.34852118,362.82058841)
\curveto(272.34852118,357.64736007)(276.88395425,356.93869865)(279.50600149,356.93869865)
\curveto(280.85245819,356.93869865)(282.83671016,357.08043093)(285.24615897,359.41901361)
\curveto(286.73434795,357.22216322)(289.3563952,356.93869865)(290.41938732,356.93869865)
\curveto(292.12017472,356.93869865)(293.39576527,357.85995849)(294.38789126,359.48987975)
\curveto(295.45088338,361.33239944)(296.08867866,363.74184826)(296.08867866,363.95444668)
\curveto(296.08867866,364.59224196)(295.38001724,364.59224196)(294.95482039,364.59224196)
\curveto(294.4587574,364.59224196)(294.31702512,364.59224196)(294.10442669,364.37964353)
\curveto(293.96269441,364.30877739)(293.96269441,364.23791125)(293.75009598,363.10405298)
\curveto(292.82883614,359.56074589)(291.83671016,358.56861991)(290.63198575,358.56861991)
\curveto(289.99419047,358.56861991)(289.63985976,358.99381676)(289.63985976,360.19854117)
\curveto(289.63985976,360.97806873)(289.78159205,361.75759629)(290.2067889,363.52924983)
\curveto(290.5611196,364.80484038)(290.98631646,366.57649393)(291.26978102,367.56861991)
\closepath
\moveto(292.26190701,371.60798999)
}
}
{
\newrgbcolor{curcolor}{0 0 0}
\pscustom[linestyle=none,fillstyle=solid,fillcolor=curcolor]
{
\newpath
\moveto(306.7355898,390.90454972)
\curveto(307.01905437,391.40061271)(307.01905437,391.68407728)(307.01905437,391.89667571)
\curveto(307.01905437,392.88880169)(306.16866067,393.59746311)(305.17653468,393.59746311)
\curveto(303.97181027,393.59746311)(303.61747956,392.60533712)(303.47574728,392.10927413)
\lineto(299.29464492,378.43210878)
\curveto(299.22377878,378.36124263)(299.08204649,378.00691193)(299.08204649,377.93604578)
\curveto(299.08204649,377.58171508)(300.07417248,377.22738437)(300.35763704,377.22738437)
\curveto(300.57023547,377.22738437)(300.57023547,377.29825051)(300.78283389,377.7943135)
\closepath
\moveto(306.7355898,390.90454972)
}
}
{
\newrgbcolor{curcolor}{0 0 0}
\pscustom[linestyle=none,fillstyle=solid,fillcolor=curcolor]
{
\newpath
\moveto(243.28119935,174.03727478)
\curveto(242.64340408,176.30499131)(240.16308912,177.43884958)(237.68277416,177.43884958)
\curveto(236.0528529,177.43884958)(234.77726234,176.51758974)(233.78513636,174.88766848)
\curveto(232.65127809,173.04514879)(232.01348282,170.63569997)(232.01348282,170.42310155)
\curveto(232.01348282,169.78530627)(232.72214423,169.78530627)(233.14734109,169.78530627)
\curveto(233.64340408,169.78530627)(233.78513636,169.78530627)(233.99773479,169.9979047)
\curveto(234.13946707,170.06877084)(234.13946707,170.21050312)(234.42293164,171.34436139)
\curveto(235.27332534,174.81680234)(236.26545132,175.80892832)(237.47017573,175.80892832)
\curveto(238.17883715,175.80892832)(238.53316786,175.38373147)(238.53316786,174.17900706)
\curveto(238.53316786,173.3994795)(238.32056943,172.69081808)(237.89537258,170.8482984)
\curveto(237.54104187,169.57270785)(237.11584502,167.8010543)(236.9032466,166.80892832)
\lineto(235.27332534,160.43097556)
\curveto(235.13159305,159.86404643)(234.91899463,158.94278659)(234.91899463,158.65932202)
\curveto(234.91899463,157.73806218)(235.62765605,156.67507005)(237.04497888,156.67507005)
\curveto(239.38356156,156.67507005)(239.87962455,158.73018816)(240.23395526,160.07664486)
\lineto(241.50954581,165.17900706)
\curveto(241.65127809,165.74593619)(242.78513636,170.42310155)(242.92686864,170.56483383)
\curveto(242.92686864,170.8482984)(244.2024592,172.83255037)(245.69064817,174.10814092)
\curveto(246.96623872,175.1002669)(248.4544277,175.80892832)(250.36781353,175.80892832)
\curveto(251.50167179,175.80892832)(252.63553006,175.38373147)(252.63553006,173.18688108)
\curveto(252.63553006,170.56483383)(250.65127809,165.32073934)(249.80088439,163.05302281)
\curveto(249.23395526,161.8482984)(249.16308912,161.49396769)(249.16308912,160.71444013)
\curveto(249.16308912,158.09239289)(251.78513636,156.67507005)(254.26545132,156.67507005)
\curveto(259.01348282,156.67507005)(261.28119935,162.84042438)(261.28119935,163.69081808)
\curveto(261.28119935,164.32861336)(260.64340408,164.32861336)(260.21820723,164.32861336)
\curveto(259.65127809,164.32861336)(259.36781353,164.32861336)(259.1552151,163.76168423)
\curveto(257.66702612,158.73018816)(255.25757731,158.30499131)(254.54891589,158.30499131)
\curveto(254.19458518,158.30499131)(253.76938833,158.30499131)(253.76938833,159.22625115)
\curveto(253.76938833,160.28924328)(254.19458518,161.42310155)(254.61978203,162.55695982)
\curveto(255.39930959,164.3994795)(257.4544277,169.71444013)(257.4544277,172.19475509)
\curveto(257.4544277,176.37585745)(253.98198675,177.43884958)(250.72214423,177.43884958)
\curveto(249.80088439,177.43884958)(246.47017573,177.43884958)(243.28119935,174.03727478)
\closepath
\moveto(243.28119935,174.03727478)
}
}
{
\newrgbcolor{curcolor}{0 0 0}
\pscustom[linewidth=4.11835987,linecolor=curcolor]
{
\newpath
\moveto(219.85386688,388.23896906)
\curveto(201.63722016,372.12253463)(174.1960068,365.70765373)(152.97820323,372.60557078)
}
}
{
\newrgbcolor{curcolor}{0 0 0}
\pscustom[linestyle=none,fillstyle=solid,fillcolor=curcolor]
{
\newpath
\moveto(208.16111797,305.4522538)
\curveto(208.30285025,306.09004908)(208.30285025,306.23178136)(208.30285025,306.23178136)
\curveto(208.30285025,306.86957663)(207.80678726,307.08217506)(207.31072427,307.08217506)
\curveto(207.16899198,307.08217506)(207.09812584,307.08217506)(207.0272597,307.01130892)
\lineto(201.14536994,306.72784435)
\curveto(200.50757466,306.72784435)(199.7280471,306.65697821)(199.7280471,305.38138766)
\curveto(199.7280471,304.6018601)(200.50757466,304.6018601)(200.93277151,304.6018601)
\curveto(201.4288345,304.6018601)(202.20836206,304.6018601)(202.7752912,304.46012782)
\lineto(196.25560616,278.38138766)
\curveto(196.25560616,278.23965538)(196.11387387,277.53099396)(196.11387387,277.38926167)
\curveto(196.11387387,276.25540341)(196.96426757,275.33414356)(198.31072427,275.33414356)
\curveto(198.52332269,275.33414356)(200.5784408,275.33414356)(201.21623608,277.88532467)
\lineto(203.90914946,288.58611207)
\curveto(204.12174789,289.43650577)(204.12174789,289.50737191)(204.83040931,290.42863175)
\curveto(205.53907072,291.42075774)(207.80678726,294.46800183)(211.56269277,294.46800183)
\curveto(212.76741718,294.46800183)(213.83040931,294.04280498)(213.83040931,291.84595459)
\curveto(213.83040931,289.22390734)(211.84615734,283.90894671)(210.99576364,281.71209632)
\curveto(210.49970064,280.50737191)(210.35796836,280.1530412)(210.35796836,279.37351364)
\curveto(210.35796836,276.7514664)(212.9800156,275.33414356)(215.46033057,275.33414356)
\curveto(220.2792282,275.33414356)(222.4760786,281.49949789)(222.4760786,282.3498916)
\curveto(222.4760786,282.98768687)(221.83828332,282.98768687)(221.41308647,282.98768687)
\curveto(220.91702348,282.98768687)(220.56269277,282.98768687)(220.42096049,282.42075774)
\curveto(218.93277151,277.38926167)(216.45245655,276.96406482)(215.74379513,276.96406482)
\curveto(215.38946442,276.96406482)(214.96426757,276.96406482)(214.96426757,277.88532467)
\curveto(214.96426757,278.87745065)(215.46033057,280.1530412)(215.74379513,280.86170262)
\curveto(216.59418883,283.05855301)(218.64930694,288.37351364)(218.64930694,290.8538286)
\curveto(218.64930694,295.03493097)(215.176866,296.09792309)(211.91702348,296.09792309)
\curveto(210.92489749,296.09792309)(208.16111797,296.09792309)(205.11387387,293.33414356)
\closepath
\moveto(208.16111797,305.4522538)
}
}
{
\newrgbcolor{curcolor}{0 0 0}
\pscustom[linestyle=none,fillstyle=solid,fillcolor=curcolor]
{
\newpath
\moveto(150.92990234,122.39517462)
\curveto(150.78817006,121.75737934)(150.71730392,121.6865132)(150.64643778,121.61564706)
\curveto(150.43383935,121.54478092)(149.93777636,121.54478092)(149.51257951,121.54478092)
\curveto(148.73305195,121.54478092)(147.88265825,121.54478092)(147.88265825,120.26919037)
\curveto(147.88265825,119.77312738)(148.3078551,119.41879667)(148.80391809,119.41879667)
\curveto(150.07950864,119.41879667)(151.56769762,119.56052895)(152.91415431,119.56052895)
\curveto(154.54407557,119.56052895)(156.24486297,119.41879667)(157.80391809,119.41879667)
\curveto(158.08738266,119.41879667)(158.93777636,119.41879667)(158.93777636,120.76525336)
\curveto(158.93777636,121.54478092)(158.22911494,121.54478092)(157.80391809,121.54478092)
\curveto(157.16612282,121.54478092)(156.38659526,121.54478092)(155.81966612,121.61564706)
\lineto(157.80391809,129.48178879)
\curveto(158.44171337,128.84399352)(159.92990234,127.85186753)(162.33935116,127.85186753)
\curveto(170.2054929,127.85186753)(175.09525668,135.00934785)(175.09525668,141.17470218)
\curveto(175.09525668,146.77312738)(170.91415431,148.61564706)(167.1582488,148.61564706)
\curveto(163.96927242,148.61564706)(161.63068975,146.84399352)(160.92202833,146.20619824)
\curveto(159.15037479,148.61564706)(156.17399683,148.61564706)(155.67793384,148.61564706)
\curveto(154.04801258,148.61564706)(152.70155589,147.69438722)(151.78029605,146.06446596)
\curveto(150.64643778,144.22194627)(150.0086425,141.81249745)(150.0086425,141.59989903)
\curveto(150.0086425,140.96210375)(150.71730392,140.96210375)(151.14250077,140.96210375)
\curveto(151.63856376,140.96210375)(151.78029605,140.96210375)(151.99289447,141.17470218)
\curveto(152.13462675,141.24556832)(152.13462675,141.3873006)(152.41809132,142.52115887)
\curveto(153.26848502,146.1353321)(154.33147715,146.9857258)(155.46533542,146.9857258)
\curveto(155.96139841,146.9857258)(156.52832754,146.84399352)(156.52832754,145.35580454)
\curveto(156.52832754,144.64714312)(156.38659526,144.00934785)(156.24486297,143.37155257)
\closepath
\moveto(161.34722518,144.08021399)
\curveto(162.62281573,145.63926911)(164.74879998,146.9857258)(166.94565038,146.9857258)
\curveto(169.78029605,146.9857258)(169.99289447,144.57627698)(169.99289447,143.584151)
\curveto(169.99289447,141.24556832)(168.43383935,135.64714312)(167.72517794,133.87548958)
\curveto(166.3078551,130.61564706)(164.11100471,129.48178879)(162.26848502,129.48178879)
\curveto(159.57557164,129.48178879)(158.51257951,131.60777304)(158.51257951,132.10383604)
\lineto(158.58344565,132.74163131)
\closepath
\moveto(161.34722518,144.08021399)
}
}
{
\newrgbcolor{curcolor}{0 0 0}
\pscustom[linestyle=none,fillstyle=solid,fillcolor=curcolor]
{
\newpath
\moveto(337.30304792,179.97386942)
\curveto(337.16131564,179.33607415)(337.09044949,179.26520801)(337.01958335,179.19434186)
\curveto(336.80698493,179.12347572)(336.31092194,179.12347572)(335.88572509,179.12347572)
\curveto(335.10619753,179.12347572)(334.25580383,179.12347572)(334.25580383,177.84788517)
\curveto(334.25580383,177.35182218)(334.68100068,176.99749147)(335.17706367,176.99749147)
\curveto(336.45265422,176.99749147)(337.9408432,177.13922375)(339.28729989,177.13922375)
\curveto(340.91722115,177.13922375)(342.61800855,176.99749147)(344.17706367,176.99749147)
\curveto(344.46052823,176.99749147)(345.31092194,176.99749147)(345.31092194,178.34394816)
\curveto(345.31092194,179.12347572)(344.60226052,179.12347572)(344.17706367,179.12347572)
\curveto(343.53926839,179.12347572)(342.75974083,179.12347572)(342.1928117,179.19434186)
\lineto(344.17706367,187.0604836)
\curveto(344.81485894,186.42268832)(346.30304792,185.43056234)(348.71249674,185.43056234)
\curveto(356.57863847,185.43056234)(361.46840225,192.58804265)(361.46840225,198.75339698)
\curveto(361.46840225,204.35182218)(357.28729989,206.19434186)(353.53139438,206.19434186)
\curveto(350.342418,206.19434186)(348.00383532,204.42268832)(347.2951739,203.78489304)
\curveto(345.52352036,206.19434186)(342.54714241,206.19434186)(342.05107942,206.19434186)
\curveto(340.42115816,206.19434186)(339.07470146,205.27308202)(338.15344162,203.64316076)
\curveto(337.01958335,201.80064108)(336.38178808,199.39119226)(336.38178808,199.17859383)
\curveto(336.38178808,198.54079856)(337.09044949,198.54079856)(337.51564634,198.54079856)
\curveto(338.01170934,198.54079856)(338.15344162,198.54079856)(338.36604005,198.75339698)
\curveto(338.50777233,198.82426312)(338.50777233,198.96599541)(338.7912369,200.09985367)
\curveto(339.6416306,203.7140269)(340.70462272,204.5644206)(341.83848099,204.5644206)
\curveto(342.33454398,204.5644206)(342.90147312,204.42268832)(342.90147312,202.93449934)
\curveto(342.90147312,202.22583793)(342.75974083,201.58804265)(342.61800855,200.95024738)
\closepath
\moveto(347.72037075,201.65890879)
\curveto(348.99596131,203.21796391)(351.12194556,204.5644206)(353.31879595,204.5644206)
\curveto(356.15344162,204.5644206)(356.36604005,202.15497178)(356.36604005,201.1628458)
\curveto(356.36604005,198.82426312)(354.80698493,193.22583793)(354.09832351,191.45418438)
\curveto(352.68100068,188.19434186)(350.48415028,187.0604836)(348.6416306,187.0604836)
\curveto(345.94871721,187.0604836)(344.88572509,189.18646785)(344.88572509,189.68253084)
\lineto(344.95659123,190.32032612)
\closepath
\moveto(347.72037075,201.65890879)
}
}
{
\newrgbcolor{curcolor}{0 0 0}
\pscustom[linestyle=none,fillstyle=solid,fillcolor=curcolor]
{
\newpath
\moveto(371.25983311,219.39641341)
\curveto(371.54329767,219.8924764)(371.54329767,220.17594097)(371.54329767,220.38853939)
\curveto(371.54329767,221.38066537)(370.69290397,222.08932679)(369.70077799,222.08932679)
\curveto(368.49605358,222.08932679)(368.14172287,221.09720081)(367.99999059,220.60113782)
\lineto(363.81888822,206.92397246)
\curveto(363.74802208,206.85310632)(363.6062898,206.49877561)(363.6062898,206.42790947)
\curveto(363.6062898,206.07357876)(364.59841578,205.71924805)(364.88188035,205.71924805)
\curveto(365.09447878,205.71924805)(365.09447878,205.79011419)(365.3070772,206.28617719)
\closepath
\moveto(371.25983311,219.39641341)
}
}
{
\newrgbcolor{curcolor}{0 0 0}
\pscustom[linestyle=none,fillstyle=solid,fillcolor=curcolor]
{
\newpath
\moveto(425.01949857,450.12232926)
\curveto(427.21634896,453.66563635)(429.12973479,453.80736863)(430.75965605,453.87823478)
\curveto(431.32658518,453.94910092)(431.39745132,454.65776234)(431.39745132,454.72862848)
\curveto(431.39745132,455.08295919)(431.11398675,455.29555761)(430.75965605,455.29555761)
\curveto(429.62579778,455.29555761)(428.27934109,455.15382533)(427.07461668,455.15382533)
\curveto(425.5864277,455.15382533)(424.02737258,455.29555761)(422.61004975,455.29555761)
\curveto(422.32658518,455.29555761)(421.75965605,455.29555761)(421.75965605,454.44516391)
\curveto(421.75965605,453.94910092)(422.11398675,453.87823478)(422.46831746,453.87823478)
\curveto(423.67304187,453.80736863)(424.52343557,453.31130564)(424.52343557,452.3900458)
\curveto(424.52343557,451.68138438)(423.8856403,450.6892584)(423.8856403,450.6892584)
\lineto(409.99587652,428.57902218)
\lineto(406.87776628,452.53177808)
\curveto(406.87776628,453.31130564)(407.94075841,453.87823478)(409.99587652,453.87823478)
\curveto(410.63367179,453.87823478)(411.12973479,453.87823478)(411.12973479,454.79949462)
\curveto(411.12973479,455.15382533)(410.77540408,455.29555761)(410.56280565,455.29555761)
\curveto(408.72028597,455.29555761)(406.80690014,455.15382533)(404.89351431,455.15382533)
\lineto(402.41319935,455.15382533)
\curveto(401.63367179,455.15382533)(400.78327809,455.29555761)(400.00375053,455.29555761)
\curveto(399.64941983,455.29555761)(399.15335683,455.29555761)(399.15335683,454.44516391)
\curveto(399.15335683,453.87823478)(399.50768754,453.87823478)(400.2872151,453.87823478)
\curveto(402.76753006,453.87823478)(402.8383962,453.45303793)(402.98012849,452.31917966)
\lineto(406.52343557,424.46878596)
\curveto(406.66516786,423.54752612)(406.87776628,423.40579383)(407.44469542,423.40579383)
\curveto(408.15335683,423.40579383)(408.36595526,423.61839226)(408.72028597,424.18532139)
\closepath
\moveto(425.01949857,450.12232926)
}
}
\end{pspicture}

    \caption{旋转的几何表达。}
    \label{fig:2.ex1}
\end{figure}

作点$P$在$\bm n$所在直线(过点$O$)上的投影$H$,即$PH\perp OH$。
点$P$绕$\bm n$顺时针旋转角度$\theta$得到点$P'$,即$\angle PHP'=\theta$且$HP'\perp OH$。
最后,作点$P$绕$\bm n$顺时针旋转角度$\displaystyle\frac{\pi}{2}$得到的点$V$,即$HV\perp HP$且$HV\perp OH$。
记向量
\begin{align}
    \bm h=\overrightarrow{OH},\quad
    \bm p=\overrightarrow{OP},\quad
    \bm p'=\overrightarrow{OP'},\quad
    \bm u=\overrightarrow{HP},\quad
    \bm u'=\overrightarrow{HP'},\quad
    \bm v=\overrightarrow{HV}\, .
\end{align}
注意到$\bm n$是单位向量,根据上述几何关系,易得
\begin{align}
    \bm h  & =(\bm n\cdot\bm p)\bm n\, ,          \\
    \bm u  & =\bm p-\bm h\, ,                     \\
    \bm v  & =\bm n\times\bm u\, ,                \\
    \bm u' & =\bm u\cos\theta+\bm v\sin\theta\, , \\
    \bm p' & =\bm h+\bm u'\, .
\end{align}
将这些式子从前往后依次代入可得\sidenote{注意这是普通的向量运算。}
\begin{align}
    \bm p' & =(\bm n\cdot\bm p)\bm n+(\bm p-(\bm n\cdot\bm p)\bm n)\cos\theta+\bm n\times(\bm p-(\bm n\cdot\bm p)\bm n)\sin\theta\nonumber \\
           & =\bm p\cos\theta+(1-\cos\theta)(\bm n\cdot\bm p)\bm n+\bm n\times\bm p\sin\theta\, .
\end{align}
另一方面,我们构造三个四元数$\bm a,\bm a'$和$\bm q$:
\begin{align}
    \bm a  & =\bm p\, ,                                                  \\
    \bm a' & =\bm p'\, ,                                                 \\
    \bm q  & =c+\bm s=\cos\frac{\theta}{2}+\bm n\sin\frac{\theta}{2}\, .
\end{align}
其中$\displaystyle c=\cos\frac{\theta}{2}$,$\displaystyle \bm s=\bm n\sin\frac{\theta}{2}$。于是有\sidenote{注意这是四元数运算。}
\begin{align}
    \bm q\bm a\bar{\bm q} & =(c+\bm s)\bm p(c-\bm s)\nonumber                                                                                                                                                                \\
                          & =(-\bm s\cdot\bm p+(c\bm p+\bm s\times\bm p))(c-\bm s)\nonumber                                                                                                                                  \\
                          & =(-c\bm s\cdot\bm p+(c\bm p+\bm s\times\bm p)\cdot\bm s)+((\bm s\cdot\bm p)\bm s+c(c\bm p+\bm s\times\bm p)-(c\bm p+\bm s\times\bm p)\times\bm s)\nonumber                                       \\
                          & =(\bm s\cdot\bm p)\bm s+c(c\bm p+\bm s\times\bm p)-(c\bm p+\bm s\times\bm p)\times\bm s\nonumber                                                                                                 \\
                          & =(\bm s\cdot\bm p)\bm s+c^2\bm p+c\bm s\times\bm p-c\bm p\times\bm s-\bm s\times\bm p\times\bm s\nonumber                                                                                        \\
                          & =(\bm s\cdot\bm p)\bm s+c^2\bm p+2c\bm s\times\bm p-((\bm s\cdot\bm s)\bm p-(\bm p\cdot\bm s)\bm s)\nonumber                                                                                     \\
                          & =(c^2-\bm s\cdot\bm s)\bm p+2(\bm s\cdot\bm p)\bm s+2c\bm s\times\bm p\nonumber                                                                                                                  \\
                          & =\left(\cos^2\frac{\theta}{2}-\bm n\cdot\bm n\sin^2\frac{\theta}{2}\right)\bm p+2(\bm n\cdot\bm p)\bm n\sin^2\frac{\theta}{2}+2\bm n\times\bm p\cos\frac{\theta}{2}\sin\frac{\theta}{2}\nonumber \\
                          & =\bm p\cos\theta+(1-\cos\theta)(\bm n\cdot\bm p)\bm n+\bm n\times\bm p\sin\theta\nonumber                                                                                                        \\
                          & =\bm a'\, .
\end{align}
其中前三个等号运用了四元数乘法计算规则,
第四个等号算得实部为零,
第六个等号再次运用了向量三重积的性质,
第九个等号运用了三角函数倍角公式。

上式给出了旋转变换与四元数的关系是:
\begin{proposition}
    向量$\bm v$绕任意单位向量$\bm n$顺时针旋转角度$\theta$得到新向量$\bm v'$,
    则$\bm v'$是四元数积$\bm q\bm v\bar{\bm q}$的虚部,其中
    \begin{align}
        \bm q=\cos\frac{\theta}{2}+\bm n\sin\frac{\theta}{2}\, .
    \end{align}

    对应的$3\times3$旋转矩阵$\bm R_{\bm n}(\theta)$为\sidenote{你可以动手验算一下。}
    \begin{align}
        \bm R_{\bm n}(\theta)=\bm I\cos\theta+(1-\cos\theta)\bm n\bm n^\mathrm{T}+\bm A_{\bm n}\sin\theta\, .
    \end{align}
    其中$\bm I$为单位矩阵,$\bm A_{\bm n}$为
    \begin{align}
        \bm A_{\bm n}=\left[
            \begin{array}{ccc}
                0    & -n_z & n_y  \\
                n_z  & 0    & -n_x \\
                -n_y & n_x  & 0
            \end{array}
            \right]\, .
    \end{align}

    对应的$4\times4$旋转矩阵只需在$\bm R_{\bm n}(\theta)$右边和底部添加全零列(行)即可。
\end{proposition}