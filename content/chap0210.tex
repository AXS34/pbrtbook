\section{交互作用}\label{sec:交互作用}

本章最后的抽象\refvar{SurfaceInteraction}{},表示在2D曲面上某点的局部信息。
例如,第\refchap{形状}的光线-形状交点例程在\refvar{SurfaceInteraction}{}里
返回在交点处的局部微分几何相关信息。
之后,第\refchap{纹理}的纹理代码计算\refvar{SurfaceInteraction}{}表示的曲面上给定点的材料属性。
紧密相关的类\refvar{MediumInteraction}{}用于表示在烟或云等介质中发生光散射的点;
在介绍了额外的预备知识后,它将在\refsec{介质}中定义。
这些类的实现在文件\href{https://github.com/mmp/pbrt-v3/tree/master/src/core/interaction.h}{\ttfamily core/interaction.h}和
\href{https://github.com/mmp/pbrt-v3/tree/master/src/core/interaction.cpp}{\ttfamily core/interaction.cpp}中。

\refvar{SurfaceInteraction}{}和\refvar{MediumInteraction}{}都继承自
一般类\refvar{Interaction}{},它提供了一些常见的成员变量和方法。
系统的一些部分(特别是光源实现)对\refvar{Interaction}{}进行操作,
而曲面和介质相互作用的区别对其并不重要。

\begin{lstlisting}
`\initcode{Interaction Declarations}{=}\initnext{InteractionDeclarations}`
struct `\initvar{Interaction}{}` {
    `\refcode{Interaction Public Methods}{}`
    `\refcode{Interaction Public Data}{}`
};
\end{lstlisting}

有许多\refvar{Interaction}{}构造函数可用;
根据要构造的交互作用类型以及哪种类型的信息相关,接收相应参数集。
以下是其中最一般的一个。
\begin{lstlisting}
`\initcode{Interaction Public Methods}{=}\initnext{InteractionPublicMethods}`
`\refvar{Interaction}{}`(const `\refvar{Point3f}{}` &p, const `\refvar{Normal3f}{}` &n, const `\refvar{Vector3f}{}` &pError,
        const `\refvar{Vector3f}{}` &wo, `\refvar{Float}{}` time,
        const `\refvar{MediumInterface}{}` &mediumInterface)
    : `\refvar[Interaction::p]{p}{}`(p), `\refvar[Interaction::time]{time}{}`(time), `\refvar{pError}{}`(pError), `\refvar[Interaction::wo]{wo}{}`(wo), `\refvar[Interaction::n]{n}{}`(n),
      `\refvar{mediumInterface}{}`(mediumInterface) { }
\end{lstlisting}

所有交互作用必须有关联的点$\bm p$和时间。
\begin{lstlisting}
`\initcode{Interaction Public Data}{=}\initnext{InteractionPublicData}`
`\refvar{Point3f}{}` `\initvar[Interaction::p]{p}{}`;
`\refvar{Float}{}` `\initvar[Interaction::time]{time}{}`;
\end{lstlisting}

对于由光线相交计算的点$\bm p$处的交互作用,
\refvar[Interaction::p]{p}{}值中通常会出现浮点误差。
\refvar{pError}{}给出了该误差的保守边界;
对于介质中的点它取$(0,0,0)$。
关于pbrt控制舍入误差的方法见\refsec{控制舍入误差}{},
关于怎样为不同形状计算该边界见\refsub{定界交点误差}。
\begin{lstlisting}
`\refcode{Interaction Public Data}{+=}\lastnext{InteractionPublicData}`
`\refvar{Vector3f}{}` `\initvar{pError}{}`;
\end{lstlisting}

对于沿光线(要么来自光线-形状交点要么来自在介质中传播的光线)的交互作用,
负的光线方向保存于\refvar[Interaction::wo]{wo}{},
对应于$\bm \omega_{\mathrm{o}}$,
即我们计算一点光量时用来表示出射方向的记号。
对于没有用到出射方向记号的其他类型的交互作用点
(例如形状曲面上由随机采样点求得的),
\refvar[Interaction::wo]{wo}{}的值为$(0,0,0)$。
\begin{lstlisting}
`\refcode{Interaction Public Data}{+=}\lastnext{InteractionPublicData}`
`\refvar{Vector3f}{}` `\initvar[Interaction::wo]{wo}{}`;
\end{lstlisting}

对于曲面上的交互作用,\refvar[Interaction::n]{n}{}保存了该点处的曲面法线。
\begin{lstlisting}
`\refcode{Interaction Public Data}{+=}\lastnext{InteractionPublicData}`
`\refvar{Normal3f}{}` `\initvar[Interaction::n]{n}{}`;
\end{lstlisting}
\begin{lstlisting}
`\refcode{Interaction Public Methods}{+=}\lastnext{InteractionPublicMethods}`
bool `\initvar{IsSurfaceInteraction}{}`() const {
    return `\refvar[Interaction::n]{n}{}` != `\refvar{Normal3f}{}`();
}
\end{lstlisting}

交互作用也需要记录这些点处的散射介质(如果有的话);
这由\refsub{介质交互}定义的类
\refvar{MediumInterface}{}的实例负责处理。
\begin{lstlisting}
`\refcode{Interaction Public Data}{+=}\lastcode{InteractionPublicData}`
`\refvar{MediumInterface}{}` `\initvar{mediumInterface}{}`;
\end{lstlisting}

\subsection{表面交互}\label{sub:表面交互}
\refvar{SurfaceInteraction}{}表示曲面上特定点
(通常是光线与曲面相交求得的位置)的几何信息。
有了这个抽象后处理曲面上点的大部分系统就不用考虑该点所在几何形状的特定类型;
抽象\refvar{SurfaceInteraction}{}提供了关于曲面点的充足信息
以允许一般化实现pbrt剩余部分的着色和几何操作。
\begin{lstlisting}
`\initcode{SurfaceInteraction Declarations}{=}`
class `\initvar{SurfaceInteraction}{}` : public `\refvar{Interaction}{}` {
public:
    `\refcode{SurfaceInteraction Public Methods}{}`
    `\refcode{SurfaceInteraction Public Data}{}`
};
\end{lstlisting}
\begin{lstlisting}
`\initcode{SurfaceInteraction Public Methods}{=}`
`\refvar{SurfaceInteraction}{}`() { }
`\refvar{SurfaceInteraction}{}`(const `\refvar{Point3f}{}` &p, const `\refvar{Vector3f}{}` &pError,const `\refvar{Point2f}{}` &uv,
    const `\refvar{Vector3f}{}` &wo, const `\refvar{Vector3f}{}` &dpdu, const `\refvar{Vector3f}{}` &dpdv,
    const `\refvar{Normal3f}{}` &dndu, const `\refvar{Normal3f}{}` &dndv,
    `\refvar{Float}{}` time, const `\refvar{Shape}{}` *sh);
void `\initvar{SetShadingGeometry}{}`(const `\refvar{Vector3f}{}` &dpdu, const `\refvar{Vector3f}{}` &dpdv,
     const `\refvar{Normal3f}{}` &dndu, const `\refvar{Normal3f}{}` &dndv, bool orientationIsAuthoritative);
void `\initvar{ComputeScatteringFunctions}{}`(const `\refvar{RayDifferential}{}` &ray,
    `\refvar{MemoryArena}{}` &arena, bool allowMultipleLobes = false,
    `\refvar{TransportMode}{}` mode = `\refvar{TransportMode}{}`::`\refvar{Radiance}{}`);
void `\initvar{ComputeDifferentials}{}`(const `\refvar{RayDifferential}{}` &r) const;
`\refvar{Spectrum}{}` `\initvar[SurfaceInteraction::Le]{Le}{}`(const `\refvar{Vector3f}{}` &w) const;
\end{lstlisting}

除了来自基类\refvar{Interaction}{}的点\refvar[Interaction::p]{p}{}和
曲面法线\refvar[Interaction::n]{n}{},
\refvar{SurfaceInteraction}{}还存储了来自曲面参数化的坐标$(u,v)$以及
该点的参数化偏导数$\displaystyle\frac{\partial \bm p}{\partial u}$
和$\displaystyle\frac{\partial \bm p}{\partial v}$。
这些值的描述见\reffig{2.19}。
保有指向该点所在\refvar{Shape}{}(下章将介绍类\refvar{Shape}{})的指针
以及曲面法线的偏导数也很有用。
\begin{lstlisting}
`\initcode{SurfaceInteraction Public Data}{=}\initnext{SurfaceInteractionPublicData}`
`\refvar{Point2f}{}` `\initvar[SurfaceInteraction::uv]{uv}{}`;
`\refvar{Vector3f}{}` `\initvar[SurfaceInteraction::dpdu]{dpdu}{}`, `\initvar[SurfaceInteraction::dpdv]{dpdv}{}`;
`\refvar{Normal3f}{}` `\initvar[SurfaceInteraction::dndu]{dndu}{}`, `\initvar[SurfaceInteraction::dndv]{dndv}{}`;
const `\refvar{Shape}{}` *`\initvar{shape}{}` = nullptr;
\end{lstlisting}

\begin{figure}[htbp]
    \centering%LaTeX with PSTricks extensions
%%Creator: Inkscape 1.0.1 (3bc2e813f5, 2020-09-07)
%%Please note this file requires PSTricks extensions
\psset{xunit=.5pt,yunit=.5pt,runit=.5pt}
\begin{pspicture}(179.24000549,190.82000732)
{
\newrgbcolor{curcolor}{0 0 0}
\pscustom[linewidth=1,linecolor=curcolor]
{
\newpath
\moveto(151.91,155.49000732)
\curveto(151.91,155.49000732)(140.58,133.65000732)(107.56,136.66000732)
\curveto(74.54,139.67000732)(58.56,162.51000732)(25.73,135.85000732)
\curveto(6.22,120.00000732)(4.94,107.63000732)(0.47,95.05000732)
}
}
{
\newrgbcolor{curcolor}{0 0 0}
\pscustom[linewidth=1,linecolor=curcolor]
{
\newpath
\moveto(178.75999451,59.71000671)
\lineto(151.86999512,155.93000793)
}
}
{
\newrgbcolor{curcolor}{0 0 0}
\pscustom[linewidth=1,linecolor=curcolor]
{
\newpath
\moveto(27.26000023,0.63000488)
\lineto(0.54000002,95.49000549)
}
}
{
\newrgbcolor{curcolor}{0 0 0}
\pscustom[linewidth=1,linecolor=curcolor]
{
\newpath
\moveto(178.74,60.23000732)
\curveto(178.74,60.23000732)(167.14,38.01000732)(134.12,41.02000732)
\curveto(101.1,44.03000732)(85.12,66.86000732)(52.29,40.21000732)
\curveto(33.35,24.82000732)(31.47,12.37000732)(27.29,0.21000732)
}
}
{
\newrgbcolor{curcolor}{0 0 0}
\pscustom[linewidth=0.5,linecolor=curcolor]
{
\newpath
\moveto(118.28,89.89000732)
\lineto(55.03,89.21000732)
\lineto(40.66,118.93000732)
\lineto(103.92,119.61000732)
\closepath
}
}
{
\newrgbcolor{curcolor}{0 0 0}
\pscustom[linewidth=1,linecolor=curcolor]
{
\newpath
\moveto(79.68000031,103.34000397)
\lineto(79.68000031,164.19000816)
}
}
{
\newrgbcolor{curcolor}{0 0 0}
\pscustom[linestyle=none,fillstyle=solid,fillcolor=curcolor]
{
\newpath
\moveto(85.19,159.28000732)
\lineto(79.68,163.54000732)
\lineto(74.18,159.28000732)
\lineto(79.68,172.29000732)
\closepath
}
}
{
\newrgbcolor{curcolor}{0.65098041 0.65098041 0.65098041}
\pscustom[linestyle=none,fillstyle=solid,fillcolor=curcolor]
{
\newpath
\moveto(83.99,160.84000732)
\lineto(79.68,170.98000732)
\lineto(79.68,164.17000732)
\closepath
}
}
{
\newrgbcolor{curcolor}{0.40000001 0.40000001 0.40000001}
\pscustom[linestyle=none,fillstyle=solid,fillcolor=curcolor]
{
\newpath
\moveto(75.39,160.84000732)
\lineto(79.68,170.98000732)
\lineto(79.68,164.17000732)
\closepath
}
}
{
\newrgbcolor{curcolor}{0 0 0}
\pscustom[linestyle=none,fillstyle=solid,fillcolor=curcolor]
{
\newpath
\moveto(83.07000375,103.88000488)
\curveto(83.07000375,106.91820179)(79.39696119,108.4391858)(77.24889196,106.29111658)
\curveto(75.10082274,104.14304736)(76.62180676,100.4700048)(79.66000366,100.4700048)
\curveto(82.69820057,100.4700048)(84.21918458,104.14304736)(82.07111536,106.29111658)
\curveto(79.92304614,108.4391858)(76.25000358,106.91820179)(76.25000358,103.88000488)
\curveto(76.25000358,100.84180798)(79.92304614,99.32082396)(82.07111536,101.46889319)
\curveto(84.21918458,103.61696241)(82.69820057,107.29000497)(79.66000366,107.29000497)
\curveto(76.62180676,107.29000497)(75.10082274,103.61696241)(77.24889196,101.46889319)
\curveto(79.39696119,99.32082396)(83.07000375,100.84180798)(83.07000375,103.88000488)
\closepath
}
}
{
\newrgbcolor{curcolor}{0 0 0}
\pscustom[linewidth=1,linecolor=curcolor]
{
\newpath
\moveto(79.51000214,103.76000977)
\lineto(108.83999634,71.65000916)
}
}
{
\newrgbcolor{curcolor}{0 0 0}
\pscustom[linestyle=none,fillstyle=solid,fillcolor=curcolor]
{
\newpath
\moveto(101.46,71.57000732)
\lineto(108.4,72.13000732)
\lineto(109.59,78.99000732)
\lineto(114.3,65.67000732)
\closepath
}
}
{
\newrgbcolor{curcolor}{0.65098041 0.65098041 0.65098041}
\pscustom[linestyle=none,fillstyle=solid,fillcolor=curcolor]
{
\newpath
\moveto(103.4,71.23000732)
\lineto(113.42,66.64000732)
\lineto(108.82,71.67000732)
\closepath
}
}
{
\newrgbcolor{curcolor}{0.40000001 0.40000001 0.40000001}
\pscustom[linestyle=none,fillstyle=solid,fillcolor=curcolor]
{
\newpath
\moveto(109.75,77.03000732)
\lineto(113.42,66.64000732)
\lineto(108.82,71.67000732)
\closepath
}
}
{
\newrgbcolor{curcolor}{0 0 0}
\pscustom[linewidth=1,linecolor=curcolor]
{
\newpath
\moveto(122.47000122,99.83000946)
\lineto(79.51000214,103.96000671)
}
}
{
\newrgbcolor{curcolor}{0 0 0}
\pscustom[linestyle=none,fillstyle=solid,fillcolor=curcolor]
{
\newpath
\moveto(118.11,105.77000732)
\lineto(121.82,99.89000732)
\lineto(117.05,94.82000732)
\lineto(130.53,99.05000732)
\closepath
}
}
{
\newrgbcolor{curcolor}{0.65098041 0.65098041 0.65098041}
\pscustom[linestyle=none,fillstyle=solid,fillcolor=curcolor]
{
\newpath
\moveto(119.54,104.43000732)
\lineto(129.23,99.18000732)
\lineto(122.45,99.83000732)
\closepath
}
}
{
\newrgbcolor{curcolor}{0.40000001 0.40000001 0.40000001}
\pscustom[linestyle=none,fillstyle=solid,fillcolor=curcolor]
{
\newpath
\moveto(118.72,95.87000732)
\lineto(129.23,99.18000732)
\lineto(122.45,99.83000732)
\closepath
}
}
{
\newrgbcolor{curcolor}{0 0 0}
\pscustom[linestyle=none,fillstyle=solid,fillcolor=curcolor]
{
\newpath
\moveto(59.57826732,97.90235348)
\curveto(59.50326732,97.56485348)(59.46576732,97.52735348)(59.42826732,97.48985348)
\curveto(59.31576732,97.45235348)(59.05326732,97.45235348)(58.82826732,97.45235348)
\curveto(58.41576732,97.45235348)(57.96576732,97.45235348)(57.96576732,96.77735348)
\curveto(57.96576732,96.51485348)(58.19076732,96.32735348)(58.45326732,96.32735348)
\curveto(59.12826732,96.32735348)(59.91576732,96.40235348)(60.62826732,96.40235348)
\curveto(61.49076732,96.40235348)(62.39076732,96.32735348)(63.21576732,96.32735348)
\curveto(63.36576732,96.32735348)(63.81576732,96.32735348)(63.81576732,97.03985348)
\curveto(63.81576732,97.45235348)(63.44076732,97.45235348)(63.21576732,97.45235348)
\curveto(62.87826732,97.45235348)(62.46576732,97.45235348)(62.16576732,97.48985348)
\lineto(63.21576732,101.65235348)
\curveto(63.55326732,101.31485348)(64.34076732,100.78985348)(65.61576732,100.78985348)
\curveto(69.77826732,100.78985348)(72.36576732,104.57735348)(72.36576732,107.83985348)
\curveto(72.36576732,110.80235348)(70.15326732,111.77735348)(68.16576732,111.77735348)
\curveto(66.47826732,111.77735348)(65.24076732,110.83985348)(64.86576732,110.50235348)
\curveto(63.92826732,111.77735348)(62.35326732,111.77735348)(62.09076732,111.77735348)
\curveto(61.22826732,111.77735348)(60.51576732,111.28985348)(60.02826732,110.42735348)
\curveto(59.42826732,109.45235348)(59.09076732,108.17735348)(59.09076732,108.06485348)
\curveto(59.09076732,107.72735348)(59.46576732,107.72735348)(59.69076732,107.72735348)
\curveto(59.95326732,107.72735348)(60.02826732,107.72735348)(60.14076732,107.83985348)
\curveto(60.21576732,107.87735348)(60.21576732,107.95235348)(60.36576732,108.55235348)
\curveto(60.81576732,110.46485348)(61.37826732,110.91485348)(61.97826732,110.91485348)
\curveto(62.24076732,110.91485348)(62.54076732,110.83985348)(62.54076732,110.05235348)
\curveto(62.54076732,109.67735348)(62.46576732,109.33985348)(62.39076732,109.00235348)
\closepath
\moveto(65.09076732,109.37735348)
\curveto(65.76576732,110.20235348)(66.89076732,110.91485348)(68.05326732,110.91485348)
\curveto(69.55326732,110.91485348)(69.66576732,109.63985348)(69.66576732,109.11485348)
\curveto(69.66576732,107.87735348)(68.84076732,104.91485348)(68.46576732,103.97735348)
\curveto(67.71576732,102.25235348)(66.55326732,101.65235348)(65.57826732,101.65235348)
\curveto(64.15326732,101.65235348)(63.59076732,102.77735348)(63.59076732,103.03985348)
\lineto(63.62826732,103.37735348)
\closepath
\moveto(65.09076732,109.37735348)
}
}
{
\newrgbcolor{curcolor}{0 0 0}
\pscustom[linestyle=none,fillstyle=solid,fillcolor=curcolor]
{
\newpath
\moveto(78.04515732,183.11689348)
\curveto(77.70765732,184.31689348)(76.39515732,184.91689348)(75.08265732,184.91689348)
\curveto(74.22015732,184.91689348)(73.54515732,184.42939348)(73.02015732,183.56689348)
\curveto(72.42015732,182.59189348)(72.08265732,181.31689348)(72.08265732,181.20439348)
\curveto(72.08265732,180.86689348)(72.45765732,180.86689348)(72.68265732,180.86689348)
\curveto(72.94515732,180.86689348)(73.02015732,180.86689348)(73.13265732,180.97939348)
\curveto(73.20765732,181.01689348)(73.20765732,181.09189348)(73.35765732,181.69189348)
\curveto(73.80765732,183.52939348)(74.33265732,184.05439348)(74.97015732,184.05439348)
\curveto(75.34515732,184.05439348)(75.53265732,183.82939348)(75.53265732,183.19189348)
\curveto(75.53265732,182.77939348)(75.42015732,182.40439348)(75.19515732,181.42939348)
\curveto(75.00765732,180.75439348)(74.78265732,179.81689348)(74.67015732,179.29189348)
\lineto(73.80765732,175.91689348)
\curveto(73.73265732,175.61689348)(73.62015732,175.12939348)(73.62015732,174.97939348)
\curveto(73.62015732,174.49189348)(73.99515732,173.92939348)(74.74515732,173.92939348)
\curveto(75.98265732,173.92939348)(76.24515732,175.01689348)(76.43265732,175.72939348)
\lineto(77.10765732,178.42939348)
\curveto(77.18265732,178.72939348)(77.78265732,181.20439348)(77.85765732,181.27939348)
\curveto(77.85765732,181.42939348)(78.53265732,182.47939348)(79.32015732,183.15439348)
\curveto(79.99515732,183.67939348)(80.78265732,184.05439348)(81.79515732,184.05439348)
\curveto(82.39515732,184.05439348)(82.99515732,183.82939348)(82.99515732,182.66689348)
\curveto(82.99515732,181.27939348)(81.94515732,178.50439348)(81.49515732,177.30439348)
\curveto(81.19515732,176.66689348)(81.15765732,176.47939348)(81.15765732,176.06689348)
\curveto(81.15765732,174.67939348)(82.54515732,173.92939348)(83.85765732,173.92939348)
\curveto(86.37015732,173.92939348)(87.57015732,177.19189348)(87.57015732,177.64189348)
\curveto(87.57015732,177.97939348)(87.23265732,177.97939348)(87.00765732,177.97939348)
\curveto(86.70765732,177.97939348)(86.55765732,177.97939348)(86.44515732,177.67939348)
\curveto(85.65765732,175.01689348)(84.38265732,174.79189348)(84.00765732,174.79189348)
\curveto(83.82015732,174.79189348)(83.59515732,174.79189348)(83.59515732,175.27939348)
\curveto(83.59515732,175.84189348)(83.82015732,176.44189348)(84.04515732,177.04189348)
\curveto(84.45765732,178.01689348)(85.54515732,180.82939348)(85.54515732,182.14189348)
\curveto(85.54515732,184.35439348)(83.70765732,184.91689348)(81.98265732,184.91689348)
\curveto(81.49515732,184.91689348)(79.73265732,184.91689348)(78.04515732,183.11689348)
\closepath
\moveto(78.04515732,183.11689348)
}
}
\end{pspicture}

    \caption{点$\bm p$附近的局部微分结构。曲面的参数化偏导数$\displaystyle\frac{\partial \bm p}{\partial u}$
        和$\displaystyle\frac{\partial \bm p}{\partial v}$位于切平面内但不必正交。
        曲面法线$\bm n$由$\displaystyle\frac{\partial \bm p}{\partial u}$
        和$\displaystyle\frac{\partial \bm p}{\partial v}$的叉积给出。
        向量$\displaystyle\frac{\partial \bm n}{\partial u}$和
        $\displaystyle\frac{\partial \bm n}{\partial v}$(此处没有画出)
        记录当我们沿曲面移动$u$和$v$时曲面法线的微分变化。}
    \label{fig:2.19}
\end{figure}

该表示隐式地假设了形状有参数化描述——
对于一定范围的$(u,v)$值,曲面上的点由某函数$f$给出即$\bm p=f(u,v)$。
尽管这并不适用于所有形状,
但pbrt支持的所有形状至少都有一个局部参数化描述,
所以既然这在别处(例如第\refchap{纹理}的纹理抗锯齿)很有帮助,
我们就继续使用参数化表示。

\refvar{SurfaceInteraction}{}的构造函数接收所有设置这些值的参数。
它计算偏导数的叉积作为法线。
\begin{lstlisting}
`\initcode{SurfaceInteraction Method Definitions}{=}\initnext{SurfaceInteractionMethodDefinitions}`
`\refvar{SurfaceInteraction}{}`::`\refvar{SurfaceInteraction}{}`(const `\refvar{Point3f}{}` &p,
    const `\refvar{Vector3f}{}` &pError, const `\refvar{Point2f}{}` &uv, const `\refvar{Vector3f}{}` &wo,
    const `\refvar{Vector3f}{}` &dpdu, const `\refvar{Vector3f}{}` &dpdv,
    const `\refvar{Normal3f}{}` &dndu, const `\refvar{Normal3f}{}` &dndv,
    `\refvar{Float}{}` time, const `\refvar{Shape}{}` *shape)
    : `\refvar{Interaction}{}`(p, `\refvar{Normal3f}{}`(`\refvar{Normalize}{}`(`\refvar{Cross}{}`(dpdu, dpdv))), pError, wo,
    time, nullptr),
    `\refvar[SurfaceInteraction::uv]{uv}{}`(uv), `\refvar[SurfaceInteraction::dpdu]{dpdu}{}`(dpdu), `\refvar[SurfaceInteraction::dpdv]{dpdv}{}`(dpdv), `\refvar[SurfaceInteraction::dndu]{dndu}{}`(dndu), `\refvar[SurfaceInteraction::dndv]{dndv}{}`(dndv),
    `\refvar{shape}{}`(shape) {
    `\refcode{Initialize shading geometry from true geometry}{}`
    `\refcode{Adjust normal based on orientation and handedness}{}`
}
\end{lstlisting}

\refvar{SurfaceInteraction}{}存储了曲面法线以及各种偏导数的第二个实例
来表示这些量因为凹凸贴图或用三角形为每个顶点插值法线而可能生成的扰动值。
系统一些部分使用该着色几何体\sidenote{译者注:原文shading geometry。},而其他的则需要处理原始量。
\begin{lstlisting}
`\refcode{SurfaceInteraction Public Data}{+=}\lastnext{SurfaceInteractionPublicData}`
struct {
    `\refvar{Normal3f}{}` `\initvar[shading::n]{n}{}`;
    `\refvar{Vector3f}{}` `\initvar[shading::dpdu]{dpdu}{}`, `\initvar[shading::dpdv]{dpdv}{}`;
    `\refvar{Normal3f}{}` `\initvar[shading::dndu]{dndu}{}`, `\initvar[shading::dndv]{dndv}{}`;
} `\initvar{shading}{}`;
\end{lstlisting}

着色几何值在构造函数中初始化以匹配原始曲面几何体。
如果存在着色几何体,则通常直到\refvar{SurfaceInteraction}{}
构造函数运行一段时间后才计算它。
稍后定义的方法\refvar{SetShadingGeometry}{()}
将更新着色几何体。
\begin{lstlisting}
`\initcode{Initialize shading geometry from true geometry}{=}`
`\refvar{shading}{}`.`\refvar[shading::n]{n}{}` = n;
`\refvar{shading}{}`.`\refvar[shading::dpdu]{dpdu}{}` = dpdu;
`\refvar{shading}{}`.`\refvar[shading::dpdv]{dpdv}{}` = dpdv;
`\refvar{shading}{}`.`\refvar[shading::dndu]{dndu}{}` = dndu;
`\refvar{shading}{}`.`\refvar[shading::dndv]{dndv}{}` = dndv;
\end{lstlisting}

曲面法线对pbrt有特殊意义,
它假设,对于闭合形状,法线指向形状外部。
对于用作面光源的几何体,光只从法线指向的那一侧曲面发出。