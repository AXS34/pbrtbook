\section{交互作用}\label{sec:交互作用}

本章最后的抽象\refvar{SurfaceInteraction}{},表示在2D曲面上一点的局部信息。
例如,第\refchap{形状}的光线-形状交点例程在\refvar{SurfaceInteraction}{}里
返回在交点处关于局部微分几何的信息。
之后,第\refchap{纹理}的纹理代码计算\refvar{SurfaceInteraction}{}表示的曲面上给定点的材料属性。
紧密相关的类\refvar{MediumInteraction}{}用于表示在烟或云等介质中发生光散射的点;
在介绍了额外的预备知识后,它将在\refsec{介质}中定义。
这些类的实现在文件\href{https://github.com/mmp/pbrt-v3/tree/master/src/core/interaction.h}{\ttfamily core/interaction.h}和
\href{https://github.com/mmp/pbrt-v3/tree/master/src/core/interaction.cpp}{\ttfamily core/interaction.cpp}中。

\refvar{SurfaceInteraction}{}和\refvar{MediumInteraction}{}都继承自
一般类\refvar{Interaction}{},它提供了一些常见的成员变量和方法。
系统的一些部分(特别是光源实现)对\refvar{Interaction}{}进行操作,
而曲面和介质相互作用的区别对其并不重要。

\begin{lstlisting}
`\initcode{Interaction Declarations}{=}\initnext{InteractionDeclarations}`
struct `\initvar{Interaction}{}` {
    `\refcode{Interaction Public Methods}{}`
    `\refcode{Interaction Public Data}{}`
};
\end{lstlisting}

有许多\refvar{Interaction}{}构造函数可用;
根据要构造的交互作用类型以及哪种类型的信息相关,接收相应参数集。
以下是其中最一般的一个。
\begin{lstlisting}
`\initcode{Interaction Public Methods}{=}\initnext{InteractionPublicMethods}`
`\refvar{Interaction}{}`(const `\refvar{Point3f}{}` &p, const `\refvar{Normal3f}{}` &n, const `\refvar{Vector3f}{}` &pError,
        const `\refvar{Vector3f}{}` &wo, `\refvar{Float}{}` time,
        const `\refvar{MediumInterface}{}` &mediumInterface)
    : `\refvar[Interaction::p]{p}{}`(p), `\refvar[Interaction::time]{time}{}`(time), `\refvar{pError}{}`(pError), `\refvar[Interaction::wo]{wo}{}`(wo), `\refvar[Interaction::n]{n}{}`(n),
      `\refvar{mediumInterface}{}`(mediumInterface) { }
\end{lstlisting}

所有交互作用必须有关联的点$\bm p$和时间。
\begin{lstlisting}
`\initcode{Interaction Public Data}{=}\initnext{InteractionPublicData}`
`\refvar{Point3f}{}` `\initvar[Interaction::p]{p}{}`;
`\refvar{Float}{}` `\initvar[Interaction::time]{time}{}`;
\end{lstlisting}

对于由光线相交计算的点$\bm p$处的交互作用,
\refvar[Interaction::p]{p}{}值中通常会出现浮点误差。
\refvar{pError}{}给出了该误差的保守边界;
对于介质中的点它取$(0,0,0)$。
关于pbrt控制舍入误差的方法见\refsec{控制舍入误差}{},
关于怎样为不同形状计算该边界见\refsub{定界交点误差}。
\begin{lstlisting}
`\refcode{Interaction Public Data}{+=}\lastnext{InteractionPublicData}`
`\refvar{Vector3f}{}` `\initvar{pError}{}`;
\end{lstlisting}

对于沿光线(要么来自光线-形状交点要么来自在介质中传播的光线)的交互作用,
负的光线方向保存于\refvar[Interaction::wo]{wo}{},
对应于$\bm \omega_{\mathrm{o}}$,
即我们计算一点光量时用来表示出射方向的记号。
对于没有用到出射方向记号的其他类型的交互作用点
(例如形状曲面上由随机采样点求得的),
\refvar[Interaction::wo]{wo}{}的值为$(0,0,0)$。
\begin{lstlisting}
`\refcode{Interaction Public Data}{+=}\lastnext{InteractionPublicData}`
`\refvar{Vector3f}{}` `\initvar[Interaction::wo]{wo}{}`;
\end{lstlisting}

对于曲面上的交互作用,\refvar[Interaction::n]{n}{}保存了该点处的曲面法线。
\begin{lstlisting}
`\refcode{Interaction Public Data}{+=}\lastnext{InteractionPublicData}`
`\refvar{Normal3f}{}` `\initvar[Interaction::n]{n}{}`;
\end{lstlisting}
\begin{lstlisting}
`\refcode{Interaction Public Methods}{+=}\lastnext{InteractionPublicMethods}`
bool `\initvar{IsSurfaceInteraction}{}`() const {
    return `\refvar[Interaction::n]{n}{}` != `\refvar{Normal3f}{}`();
}
\end{lstlisting}

交互作用也需要记录这些点处的散射介质(如果有的话);
这由\refsub{介质交互}定义的类
\refvar{MediumInterface}{}的实例负责处理。
\begin{lstlisting}
`\refcode{Interaction Public Data}{+=}\lastcode{InteractionPublicData}`
`\refvar{MediumInterface}{}` `\initvar{mediumInterface}{}`;
\end{lstlisting}

\subsection{表面交互}\label{sub:表面交互}
\begin{lstlisting}
`\initcode{SurfaceInteraction Declarations}{=}`
class `\initvar{SurfaceInteraction}{}` : public `\refvar{Interaction}{}` {
public:
    `\refcode{SurfaceInteraction Public Methods}{}`
    `\refcode{SurfaceInteraction Public Data}{}`
};
\end{lstlisting}

\begin{lstlisting}
`\initcode{SurfaceInteraction Public Data}{=}\initnext{SurfaceInteractionPublicData}`
`\refvar{Point2f}{}` `\initvar[SurfaceInteraction::uv]{uv}{}`;
`\refvar{Vector3f}{}` `\initvar[SurfaceInteraction::dpdu]{dpdu}{}`, `\initvar[SurfaceInteraction::dpdv]{dpdv}{}`;
`\refvar{Normal3f}{}` `\initvar[SurfaceInteraction::dndu]{dndu}{}`, `\initvar[SurfaceInteraction::dndv]{dndv}{}`;
const `\refvar{Shape}{}` *`\initvar{shape}{}` = nullptr;
\end{lstlisting}

\begin{lstlisting}
`\initcode{SurfaceInteraction Method Definitions}{=}\initnext{SurfaceInteractionMethodDefinitions}`
`\refvar{SurfaceInteraction}{}`::`\refvar{SurfaceInteraction}{}`(const `\refvar{Point3f}{}` &p,
    const `\refvar{Vector3f}{}` &pError, const `\refvar{Point2f}{}` &uv, const `\refvar{Vector3f}{}` &wo,
    const `\refvar{Vector3f}{}` &dpdu, const `\refvar{Vector3f}{}` &dpdv,
    const `\refvar{Normal3f}{}` &dndu, const `\refvar{Normal3f}{}` &dndv,
    `\refvar{Float}{}` time, const `\refvar{Shape}{}` *shape)
    : `\refvar{Interaction}{}`(p, `\refvar{Normal3f}{}`(`\refvar{Normalize}{}`(`\refvar{Cross}{}`(dpdu, dpdv))), pError, wo,
    time, nullptr),
    `\refvar[SurfaceInteraction::uv]{uv}{}`(uv), `\refvar[SurfaceInteraction::dpdu]{dpdu}{}`(dpdu), `\refvar[SurfaceInteraction::dpdv]{dpdv}{}`(dpdv), `\refvar[SurfaceInteraction::dndu]{dndu}{}`(dndu), `\refvar[SurfaceInteraction::dndv]{dndv}{}`(dndv),
    `\refvar{shape}{}`(shape) {
    `\refcode{Initialize shading geometry from true geometry}{}`
    `\refcode{Adjust normal based on orientation and handedness}{}`
}
\end{lstlisting}

\begin{lstlisting}
`\refcode{SurfaceInteraction Public Data}{+=}\lastnext{SurfaceInteractionPublicData}`
struct {
    `\refvar{Normal3f}{}` `\initvar[shading::n]{n}{}`;
    `\refvar{Vector3f}{}` `\initvar[shading::dpdu]{dpdu}{}`, `\initvar[shading::dpdv]{dpdv}{}`;
    `\refvar{Normal3f}{}` `\initvar[shading::dndu]{dndu}{}`, `\initvar[shading::dndv]{dndv}{}`;
} `\initvar{shading}{}`;
\end{lstlisting}