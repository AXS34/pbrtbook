\section{扩展阅读}\label{sec:扩展阅读02}
\citeauthor{10.1007/978-3-642-61542-9_19}、\citeauthor{10.1145/282957.282969}
以及他们的合作者主张优雅的“无坐标”方法来为图形学描述向量几何,
不再强调位置和方向恰好由关于特定坐标系的坐标$(x,y,z)$表示的事实,
点和向量自己记录它们是用哪个坐标系表示的
\citep{10.1145/282957.282969,10.1007/978-3-642-61542-9_19,Mann97acoordinate}。
这让软件层面能保证一个坐标系的向量与另一个坐标系的点相加等常见错误
能通过首先将其转换到公共坐标系来被明确地处理。
尽管在与计算机图形学中的坐标系打交道时
该方法背后的原则很值得理解和记住,但我们在pbrt中没有遵循该方法。

\citet{10.5555/2821579}的《\citetitle{10.5555/2821579}》受
无坐标方法的影响深入介绍了本章的许多内容
\sidenote{译者注:原文引用年份似乎有误,已修正。}。
它也有许多有用的图形学几何算法。
\citet{10.5555/63448}的
《\citetitle{10.5555/63448}》对本章内容有经典和更传统的介绍。
注意他们的书使用行向量表示点和向量,
这意味着在其框架下我们的矩阵可能要转置,
以及他们的点和向量与矩阵相乘要转换为($\bm p\bm M$),
而不是像我们那样用矩阵乘点($\bm M\bm p$)。
本章只简要提到了齐次坐标,但它是投影几何的基础,
更是许多优雅算法的基础。
\citet{10.5555/113163}的书对这些内容有着精彩介绍。

有许多线性代数和向量几何的好书。
我们发现\citet{lang2012introduction}和\citet{buck1956advanced}
分别是这些内容很好的参考。
也可详见\citet{10.5555/2829183}等的书《\citetitle{10.5555/2829183}》了解
基于图形学对线性代数的详实介绍。

在\citet{inproceedings}和\citet{TURKOWSKI1990539}的论文发表后,
图形学学界首次广泛理解了怎样变换法线向量的细节。

\citet{10.1145/325334.325242}将四元数引入图形学并展示了它们在动画旋转中的用处。
\citet{10.5555/155294.155324}描述了为动画变换使用矩阵极分解;
\citet{doi:10.1137/0907079}通过将矩阵与其逆转置相加开发了
从旋转和缩放合成矩阵中成功提取旋转的算法。
《\emph{\citefield{SHOEMAKE1994207}{booktitle}}》中
\citet{SHOEMAKE1991351,SHOEMAKE1994207,10.5555/180895.180914}的章节
分别给出更多关于从矩阵到四元数转换和矩阵极分解实现的推导细节。

我们在本章阐述中遵循\citet{Blow_2004}的球面线性插值推导。
\citet{Bloom2003ErrorsAO}讨论了在计算机图形学中
动画旋转插值需要的性质以及哪些方法提供哪些性质。
\refvar{Slerp}{()}函数基于多项式逼近三角函数的更高效实现
见\citet{doi:10.1080/2151237X.2011.610255}。
更多旋转插值的精巧方法见\citet{10.1145/258734.258870}
与\citet{10.1145/502122.502124}。
\citet{doi:10.1080/10867651.1999.10487509}阐述了
高效计算矩阵来旋转向量的技术。

区间运算是渲染中常用的工具;\citet{moore1966interval}的书中有写得很好的介绍。