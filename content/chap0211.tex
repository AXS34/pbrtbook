\section{扩展阅读}\label{sec:扩展阅读2}
\citeauthor{10.1007/978-3-642-61542-9_19}、\citeauthor{10.1145/282957.282969}
以及他们的合作者主张优雅的“无坐标”方法来为图形学描述向量几何,
不再强调位置和方向恰好由关于特定坐标系统的坐标$(x,y,z)$表示的事实,
点和向量自己记录它们是用哪个坐标系统表示的
\citep{10.1145/282957.282969,10.1007/978-3-642-61542-9_19,Mann97acoordinate}。
这让软件层面能保证一个坐标系统的向量与另一个坐标系统的点相加等常见错误
能通过首先将其转换到公共坐标系统来被明确地处理。
尽管在与计算机图形学中的坐标系统打交道时
该方法背后的原则很值得理解和记住,但我们在pbrt中没有遵循该方法。

Schneider和Eberly的《\citetitle{10.5555/2821579}》受到
无坐标方法的影响并深入介绍了本章的许多内容\citep{10.5555/2821579}
\sidenote{译者注:原文引用年份似乎有误,已修正。}。
它也有许多有用的图形学几何算法。
Rogers和Adams\parencite*{10.5555/63448}的
《\citetitle{10.5555/63448}》对本章内容有着经典和更传统的介绍。
注意他们的书使用行向量表示点和向量,
然而这意味着在他们的框架下我们的矩阵可能要转置,
以及他们的点和向量与矩阵相乘时要转换为($\bm p\bm M$),
而不是像我们那样用矩阵乘点($\bm M\bm p$)。
本章只简要提到了齐次坐标,但它是投影几何的基础,
投影几何又是许多优雅算法的基础。
\citeauthor{10.5555/113163}的书对这些内容有着精彩介绍\citep{10.5555/113163}。

有许多关于线性代数和向量几何的好书。
我们发现\citet{lang2012introduction}和\citet{buck1956advanced}
分别是这些内容很好的参考。
也可详见Akenine-Möller等的书《\citetitle{10.5555/2829183}》\parencite*{10.5555/2829183}了解
基于图形学对线性代数的详实介绍。

