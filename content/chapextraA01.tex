\section{主要包含文件}\label{sec:主要包含文件}
\begin{lstlisting}
`\initcode{Global Include Files}{=}\initnext{GlobalIncludeFiles}`
#include <algorithm>
#include <cinttypes>
#include <cmath>
#include <iostream>
#include <limits>
#include <memory>
#include <string>
#include <vector>
\end{lstlisting}

\begin{lstlisting}
`\refcode{Global Forward Declarations}{+=}\lastnext{GlobalForwardDeclarations}`
#ifdef PBRT_FLOAT_AS_DOUBLE
typedef double `\initvar{Float}{}`;
#else
typedef float Float;
#endif // PBRT\_FLOAT\_AS\_DOUBLE
\end{lstlisting}

\subsection{实用函数}\label{sub:实用函数}
\subsubsection*{钳位}
\begin{lstlisting}
`\refcode{Global Inline Functions}{+=}\lastnext{GlobalInlineFunctions}`
template <typename T, typename U, typename V>
inline T `\initvar{Clamp}{}`(T val, U low, V high) {
    if (val < low) return low;
    else if (val > high) return high;
    else return val;
}
\end{lstlisting}

\subsubsection*{有用的常数}
\begin{lstlisting}
`\refcode{Global Constants}{+=}\lastnext{GlobalConstants}`
static const `\refvar{Float}{}` `\initvar{Pi}{}`      = 3.14159265358979323846;
static const `\refvar{Float}{}` `\initvar{InvPi}{}`   = 0.31830988618379067154;
static const `\refvar{Float}{}` `\initvar{Inv2Pi}{}`  = 0.15915494309189533577;
static const `\refvar{Float}{}` `\initvar{Inv4Pi}{}`  = 0.07957747154594766788;
static const `\refvar{Float}{}` `\initvar{PiOver2}{}` = 1.57079632679489661923;
static const `\refvar{Float}{}` `\initvar{PiOver4}{}` = 0.78539816339744830961;
static const `\refvar{Float}{}` `\initvar{Sqrt2}{}`   = 1.41421356237309504880;
\end{lstlisting}

\subsubsection*{角的度量之间的转化}
\begin{lstlisting}
`\refcode{Global Inline Functions}{+=}\lastnext{GlobalInlineFunctions}`
inline `\refvar{Float}{}` `\initvar{Radians}{}`(`\refvar{Float}{}` deg) { 
    return (`\refvar{Pi}{}` / 180) * deg; 
}
inline `\refvar{Float}{}` `\initvar{Degrees}{}`(`\refvar{Float}{}` rad) { 
    return (180 / `\refvar{Pi}{}`) * rad; 
}
\end{lstlisting}

\subsubsection*{基于2的运算}
\begin{lstlisting}
`\refcode{Global Inline Functions}{+=}\lastnext{GlobalInlineFunctions}`
inline int `\initvar{Log2Int}{}`(uint32_t v) {
    return 31 - __builtin_clz(v);
}
\end{lstlisting}

\subsubsection*{区间搜索}
\begin{lstlisting}
`\refcode{Global Inline Functions}{+=}\lastnext{GlobalInlineFunctions}`
template <typename Predicate> int `\initvar{FindInterval}{}`(int size,
        const Predicate &pred) {
    int first = 0, len = size;
    while (len > 0) {
        int half = len >> 1, middle = first + half;
        `\refcode{Bisect range based on value of pred at middle}{}`
    }
    return `\refvar{Clamp}{}`(first - 1, 0, size - 2);
}
\end{lstlisting}

\subsection{伪随机数}\label{sub:伪随机数}
{\initvar{RNG}{}}
\begin{lstlisting}
`\initcode{RNG Public Methods}{=}\initnext{RNGPublicMethods}`
`\refvar{RNG}{}`();
`\refvar{RNG}{}`(uint64_t sequenceIndex) { `\refvar{SetSequence}{}`(sequenceIndex); }
\end{lstlisting}

\begin{lstlisting}
`\refcode{RNG Public Methods}{+=}\lastnext{RNGPublicMethods}`
void `\initvar{SetSequence}{}`(uint64_t sequenceIndex);
\end{lstlisting}

\begin{lstlisting}
`\refcode{RNG Public Methods}{+=}\lastnext{RNGPublicMethods}`
uint32_t `\initvar{UniformUInt32}{}`(uint32_t b) {
    uint32_t threshold = (~b + 1u) % b;
    while (true) {
        uint32_t r = `\refvar{UniformUInt32}{}`();
        if (r >= threshold)
            return r % b;
    }
}
\end{lstlisting}

\begin{lstlisting}
`\refcode{RNG Public Methods}{+=}\lastcode{RNGPublicMethods}`
`\refvar{Float}{}` `\initvar{UniformFloat}{}`() {
    return std::min(`\refvar{OneMinusEpsilon}{}`, `\refvar{UniformUInt32}{}`() * 0x1p-32f);
}
\end{lstlisting}