\section{基本形状接口}\label{sec:基本形状接口}

\refvar{Shape}{}的接口定义于
源文件\href{https://github.com/mmp/pbrt-v3/tree/master/src/core/shape.h}{\ttfamily core/shape.h}中,
\href{https://github.com/mmp/pbrt-v3/tree/master/src/core/shape.cpp}{\ttfamily core/shape.cpp}中
可以找到\refvar{Shape}{}公共方法的定义。
基类\refvar{Shape}{}定义了通用形状接口。
它也暴露了一些对所有\refvar{Shape}{}实现有用的公有数据成员。
\begin{lstlisting}
`\initcode{Shape Declarations}{=}`
class `\initvar{Shape}{}` {
public:
`\refcode{Shape Interface}{}`
`\refcode{Shape Public Data}{}`
};
\end{lstlisting}

所有形状都定义在物体的坐标空间中;
例如,所有球体都定义在球心位于原点的坐标系中。
为了在场景别处放置球体,必须提供描述从物体空间到世界空间映射的变换。
类\refvar{Shape}{}保存了该变换及其逆。

\refvar{Shape}{}也接收一个布尔参数\refvar{reverseOrientation}{}来
表示其曲面法线方向是否与默认的相反。
该能力很有用,因为曲面法线的朝向用于决定形状哪一面是“外面”。
例如,发光形状只在曲面法线所在那侧是发光的。
该参数值通过pbrt输入文件中的{\ttfamily ReverseOrientation}语句控制。

\refvar{Shape}{}还保存了调用\refvar[SwapsHandedness]{Transform::SwapsHandedness}{()}
来进行物体到世界变换的返回值。
每次找到光线相交处就调用的\refvar{SurfaceInteraction}{}构造函数需要该值,
所以\refvar{Shape}{}构造函数一次性计算并保存它。
\begin{lstlisting}
`\initcode{Shape Method Definitions}{=}\initnext{ShapeMethodDefinitions}`
`\refvar{Shape}{}`::`\refvar{Shape}{}`(const `\refvar{Transform}{}` *ObjectToWorld,
        const `\refvar{Transform}{}` *WorldToObject, bool reverseOrientation)
    : `\refvar{ObjectToWorld}{}`(ObjectToWorld), `\refvar{WorldToObject}{}`(WorldToObject),
      `\refvar{reverseOrientation}{}`(reverseOrientation),
      `\refvar{transformSwapsHandedness}{}`(ObjectToWorld->`\refvar{SwapsHandedness}{}`()) {
}
\end{lstlisting}

一个重要细节是形状保存了指向其变换的指针而不是直接保存\refvar{Transform}{}
对象。回想\refsec{变换}\refvar{Transform}{}对象由总共32个浮点数表示,需要128字节内存;
因为场景中许多形状经常被施加同样的变换,
pbrt维持了一个\refvar{Transform}{}池使它们可以复用,
并把指向共享\refvar{Transform}{}的指针传给形状。
这样,\refvar{Shape}{}析构函数不会删除其\refvar{Transform}{}指针,
而是让\refvar{Transform}{}控制代码来管理那些内存。
\begin{lstlisting}
`\initcode{Shape Public Data}{=}`
const `\refvar{Transform}{}` *`\initvar{ObjectToWorld}{}`, *`\initvar{WorldToObject}{}`;
const bool `\initvar{reverseOrientation}{}`;
const bool `\initvar{transformSwapsHandedness}{}`;
\end{lstlisting}

\subsection{边界}\label{sub:边界}
pbrt要渲染的场景中经常包含处理计算量极大的物体。
对于许多操作,有一个3D\keyindex{包围盒}{bounding volume}{}来封住物体常常很有用。
例如,如果一条光线没有穿过某个包围盒,对于该光线pbrt可以避免处理里面所有物体。

轴对齐边界框是方便的包围盒,因为它们只需要六个浮点值保存且适合很多形状。
而且测试光线与轴对齐边界框相交性的成本极低。
因此每个\refvar{Shape}{}实现必须能
用\refvar{Bounds3f}{}表示的轴对齐边界框包围自己。
有两种不同的包围方法。
第一个是\refvar{ObjectBound}{()},返回在形状的物体空间里的边界框。
\begin{lstlisting}
`\initcode{Shape Interface}{=}\initnext{ShapeInterface}`
virtual `\refvar{Bounds3f}{}` `\initvar{ObjectBound}{}`() const = 0;
\end{lstlisting}

第二个方法是\refvar[Shape::WorldBound]{WorldBound}{()},返回世界空间的边界框。
pbrt提供该方法的默认实现将物体空间边界变换到世界空间。
然而可以轻松计算更紧致的世界空间边界的形状应该重载该方法。
这种形状的一个例子是三角形(\reffig{3.1})。
\begin{figure}[htbp]
    \centering%LaTeX with PSTricks extensions
%%Creator: Inkscape 1.0.1 (3bc2e813f5, 2020-09-07)
%%Please note this file requires PSTricks extensions
\psset{xunit=.5pt,yunit=.5pt,runit=.5pt}
\begin{pspicture}(629.46002197,511.55999756)
{
\newrgbcolor{curcolor}{1 1 1}
\pscustom[linestyle=none,fillstyle=solid,fillcolor=curcolor]
{
\newpath
\moveto(339.26,140.32999756)
\lineto(567.76,140.32999756)
\lineto(567.76,25.82999756)
\closepath
}
}
{
\newrgbcolor{curcolor}{0 0 0}
\pscustom[linewidth=1,linecolor=curcolor]
{
\newpath
\moveto(339.26,140.32999756)
\lineto(567.76,140.32999756)
\lineto(567.76,25.82999756)
\closepath
}
}
{
\newrgbcolor{curcolor}{1 1 1}
\pscustom[linestyle=none,fillstyle=solid,fillcolor=curcolor]
{
\newpath
\moveto(3.48,364.86999756)
\lineto(172.22,210.78999756)
\lineto(95.02,126.23999756)
\closepath
}
}
{
\newrgbcolor{curcolor}{0 0 0}
\pscustom[linewidth=1,linecolor=curcolor]
{
\newpath
\moveto(3.48,364.86999756)
\lineto(172.22,210.78999756)
\lineto(95.02,126.23999756)
\closepath
}
}
{
\newrgbcolor{curcolor}{0 0 0}
\pscustom[linewidth=1,linecolor=curcolor,linestyle=dashed,dash=4]
{
\newpath
\moveto(0.5,367.80000305)
\lineto(173.5,367.80000305)
\lineto(173.5,124.80000305)
\lineto(0.5,124.80000305)
\closepath
}
}
{
\newrgbcolor{curcolor}{0 0 0}
\pscustom[linewidth=1,linecolor=curcolor,linestyle=dashed,dash=4]
{
\newpath
\moveto(334.76000977,142.32998657)
\lineto(570.76000977,142.32998657)
\lineto(570.76000977,26.32998657)
\lineto(334.76000977,26.32998657)
\closepath
}
}
{
\newrgbcolor{curcolor}{1 1 1}
\pscustom[linestyle=none,fillstyle=solid,fillcolor=curcolor]
{
\newpath
\moveto(339.88,393.97999756)
\lineto(568.38,393.68999756)
\lineto(568.23,279.18999756)
\closepath
}
}
{
\newrgbcolor{curcolor}{0 0 0}
\pscustom[linewidth=1,linecolor=curcolor]
{
\newpath
\moveto(339.88,393.97999756)
\lineto(568.38,393.68999756)
\lineto(568.23,279.18999756)
\closepath
}
}
{
\newrgbcolor{curcolor}{0.60000002 0.60000002 0.60000002}
\pscustom[linewidth=1,linecolor=curcolor,linestyle=dashed,dash=4]
{
\newpath
\moveto(335.68396305,394.16577219)
\lineto(463.57916173,510.66391876)
\lineto(627.21528669,331.01921773)
\lineto(499.32008801,214.52107116)
\closepath
}
}
{
\newrgbcolor{curcolor}{0 0 0}
\pscustom[linewidth=1,linecolor=curcolor,linestyle=dashed,dash=4]
{
\newpath
\moveto(333.95999146,511.05999756)
\lineto(628.95999146,511.05999756)
\lineto(628.95999146,214.05999756)
\lineto(333.95999146,214.05999756)
\closepath
}
}
{
\newrgbcolor{curcolor}{0 0 0}
\pscustom[linewidth=1,linecolor=curcolor]
{
\newpath
\moveto(208.1499939,162.76998901)
\lineto(291.20001221,124.44000244)
}
}
{
\newrgbcolor{curcolor}{0 0 0}
\pscustom[linestyle=none,fillstyle=solid,fillcolor=curcolor]
{
\newpath
\moveto(284.43,121.49999756)
\lineto(290.61,124.70999756)
\lineto(289.05,131.49999756)
\lineto(298.56,121.04999756)
\closepath
}
}
{
\newrgbcolor{curcolor}{0.65098041 0.65098041 0.65098041}
\pscustom[linestyle=none,fillstyle=solid,fillcolor=curcolor]
{
\newpath
\moveto(286.35,121.93999756)
\lineto(297.36,121.59999756)
\lineto(291.18,124.44999756)
\closepath
}
}
{
\newrgbcolor{curcolor}{0.40000001 0.40000001 0.40000001}
\pscustom[linestyle=none,fillstyle=solid,fillcolor=curcolor]
{
\newpath
\moveto(289.96,129.74999756)
\lineto(297.36,121.59999756)
\lineto(291.18,124.44999756)
\closepath
}
}
{
\newrgbcolor{curcolor}{0 0 0}
\pscustom[linewidth=1,linecolor=curcolor]
{
\newpath
\moveto(207.8500061,308.70999146)
\lineto(290.36999512,340.97999573)
}
}
{
\newrgbcolor{curcolor}{0 0 0}
\pscustom[linestyle=none,fillstyle=solid,fillcolor=curcolor]
{
\newpath
\moveto(287.8,334.05999756)
\lineto(289.76,340.73999756)
\lineto(283.79,344.30999756)
\lineto(297.91,343.92999756)
\closepath
}
}
{
\newrgbcolor{curcolor}{0.65098041 0.65098041 0.65098041}
\pscustom[linestyle=none,fillstyle=solid,fillcolor=curcolor]
{
\newpath
\moveto(288.81,335.74999756)
\lineto(296.69,343.44999756)
\lineto(290.35,340.96999756)
\closepath
}
}
{
\newrgbcolor{curcolor}{0.40000001 0.40000001 0.40000001}
\pscustom[linestyle=none,fillstyle=solid,fillcolor=curcolor]
{
\newpath
\moveto(285.68,343.75999756)
\lineto(296.69,343.44999756)
\lineto(290.35,340.96999756)
\closepath
}
}
{
\newrgbcolor{curcolor}{0 0 0}
\pscustom[linestyle=none,fillstyle=solid,fillcolor=curcolor]
{
\newpath
\moveto(337.81564237,191.32581977)
\lineto(336.02657987,191.32581977)
\curveto(335.10470487,192.38311143)(334.38855904,193.53675727)(333.87814237,194.78675727)
\curveto(333.36772571,196.03675727)(333.11251737,197.44561143)(333.11251737,199.01331977)
\curveto(333.11251737,200.5810281)(333.36772571,201.98988227)(333.87814237,203.23988227)
\curveto(334.38855904,204.48988227)(335.10470487,205.6435281)(336.02657987,206.70081977)
\lineto(337.81564237,206.70081977)
\lineto(337.81564237,206.62269477)
\curveto(337.39376737,206.24248643)(336.99012154,205.80238227)(336.60470487,205.30238227)
\curveto(336.22449654,204.8075906)(335.87032987,204.2294656)(335.54220487,203.56800727)
\curveto(335.22970487,202.92738227)(334.97449654,202.2216531)(334.77657987,201.45081977)
\curveto(334.58387154,200.67998643)(334.48751737,199.86748643)(334.48751737,199.01331977)
\curveto(334.48751737,198.12269477)(334.58126737,197.3075906)(334.76876737,196.56800727)
\curveto(334.96147571,195.82842393)(335.21928821,195.12529893)(335.54220487,194.45863227)
\curveto(335.85470487,193.81800727)(336.21147571,193.23988227)(336.61251737,192.72425727)
\curveto(337.01355904,192.20342393)(337.41460071,191.76331977)(337.81564237,191.40394477)
\closepath
}
}
{
\newrgbcolor{curcolor}{0 0 0}
\pscustom[linestyle=none,fillstyle=solid,fillcolor=curcolor]
{
\newpath
\moveto(347.19064237,194.54456977)
\lineto(345.72970487,194.54456977)
\lineto(345.72970487,195.47425727)
\curveto(345.59949654,195.3857156)(345.42241321,195.2607156)(345.19845487,195.09925727)
\curveto(344.97970487,194.94300727)(344.76616321,194.81800727)(344.55782987,194.72425727)
\curveto(344.31303821,194.6044656)(344.03178821,194.50550727)(343.71407987,194.42738227)
\curveto(343.39637154,194.34404893)(343.02397571,194.30238227)(342.59689237,194.30238227)
\curveto(341.81043404,194.30238227)(341.14376737,194.56279893)(340.59689237,195.08363227)
\curveto(340.05001737,195.6044656)(339.77657987,196.2685281)(339.77657987,197.07581977)
\curveto(339.77657987,197.7372781)(339.91720487,198.27113227)(340.19845487,198.67738227)
\curveto(340.48491321,199.0888406)(340.89116321,199.41175727)(341.41720487,199.64613227)
\curveto(341.94845487,199.88050727)(342.58647571,200.03936143)(343.33126737,200.12269477)
\curveto(344.07605904,200.2060281)(344.87553821,200.2685281)(345.72970487,200.31019477)
\lineto(345.72970487,200.53675727)
\curveto(345.72970487,200.8700906)(345.66980904,201.14613227)(345.55001737,201.36488227)
\curveto(345.43543404,201.58363227)(345.26876737,201.75550727)(345.05001737,201.88050727)
\curveto(344.84168404,202.00029893)(344.59168404,202.0810281)(344.30001737,202.12269477)
\curveto(344.00835071,202.16436143)(343.70366321,202.18519477)(343.38595487,202.18519477)
\curveto(343.00053821,202.18519477)(342.57085071,202.13311143)(342.09689237,202.02894477)
\curveto(341.62293404,201.92998643)(341.13335071,201.7841531)(340.62814237,201.59144477)
\lineto(340.55001737,201.59144477)
\lineto(340.55001737,203.08363227)
\curveto(340.83647571,203.16175727)(341.25053821,203.24769477)(341.79220487,203.34144477)
\curveto(342.33387154,203.43519477)(342.86772571,203.48206977)(343.39376737,203.48206977)
\curveto(344.00835071,203.48206977)(344.54220487,203.42998643)(344.99532987,203.32581977)
\curveto(345.45366321,203.22686143)(345.84949654,203.05498643)(346.18282987,202.81019477)
\curveto(346.51095487,202.57061143)(346.76095487,202.2607156)(346.93282987,201.88050727)
\curveto(347.10470487,201.50029893)(347.19064237,201.02894477)(347.19064237,200.46644477)
\closepath
\moveto(345.72970487,196.69300727)
\lineto(345.72970487,199.12269477)
\curveto(345.28178821,199.0966531)(344.75314237,199.0575906)(344.14376737,199.00550727)
\curveto(343.53960071,198.95342393)(343.06043404,198.8779031)(342.70626737,198.77894477)
\curveto(342.28439237,198.6591531)(341.94324654,198.4716531)(341.68282987,198.21644477)
\curveto(341.42241321,197.96644477)(341.29220487,197.6200906)(341.29220487,197.17738227)
\curveto(341.29220487,196.67738227)(341.44324654,196.2997781)(341.74532987,196.04456977)
\curveto(342.04741321,195.79456977)(342.50835071,195.66956977)(343.12814237,195.66956977)
\curveto(343.64376737,195.66956977)(344.11512154,195.7685281)(344.54220487,195.96644477)
\curveto(344.96928821,196.16956977)(345.36512154,196.41175727)(345.72970487,196.69300727)
\closepath
}
}
{
\newrgbcolor{curcolor}{0 0 0}
\pscustom[linestyle=none,fillstyle=solid,fillcolor=curcolor]
{
\newpath
\moveto(354.42501737,199.01331977)
\curveto(354.42501737,197.44561143)(354.16980904,196.03675727)(353.65939237,194.78675727)
\curveto(353.14897571,193.53675727)(352.43282987,192.38311143)(351.51095487,191.32581977)
\lineto(349.72189237,191.32581977)
\lineto(349.72189237,191.40394477)
\curveto(350.12293404,191.76331977)(350.52397571,192.20342393)(350.92501737,192.72425727)
\curveto(351.33126737,193.23988227)(351.68803821,193.81800727)(351.99532987,194.45863227)
\curveto(352.31824654,195.12529893)(352.57345487,195.82842393)(352.76095487,196.56800727)
\curveto(352.95366321,197.3075906)(353.05001737,198.12269477)(353.05001737,199.01331977)
\curveto(353.05001737,199.86748643)(352.95366321,200.67998643)(352.76095487,201.45081977)
\curveto(352.56824654,202.2216531)(352.31303821,202.92738227)(351.99532987,203.56800727)
\curveto(351.66720487,204.2294656)(351.31043404,204.8075906)(350.92501737,205.30238227)
\curveto(350.54480904,205.80238227)(350.14376737,206.24248643)(349.72189237,206.62269477)
\lineto(349.72189237,206.70081977)
\lineto(351.51095487,206.70081977)
\curveto(352.43282987,205.6435281)(353.14897571,204.48988227)(353.65939237,203.23988227)
\curveto(354.16980904,201.98988227)(354.42501737,200.5810281)(354.42501737,199.01331977)
\closepath
}
}
{
\newrgbcolor{curcolor}{0 0 0}
\pscustom[linestyle=none,fillstyle=solid,fillcolor=curcolor]
{
\newpath
\moveto(338.77179538,5.00714437)
\lineto(336.98273288,5.00714437)
\curveto(336.06085788,6.06443603)(335.34471204,7.21808187)(334.83429538,8.46808187)
\curveto(334.32387871,9.71808187)(334.06867038,11.12693603)(334.06867038,12.69464437)
\curveto(334.06867038,14.2623527)(334.32387871,15.67120687)(334.83429538,16.92120687)
\curveto(335.34471204,18.17120687)(336.06085788,19.3248527)(336.98273288,20.38214437)
\lineto(338.77179538,20.38214437)
\lineto(338.77179538,20.30401937)
\curveto(338.34992038,19.92381103)(337.94627454,19.48370687)(337.56085788,18.98370687)
\curveto(337.18064954,18.4889152)(336.82648288,17.9107902)(336.49835788,17.24933187)
\curveto(336.18585788,16.60870687)(335.93064954,15.9029777)(335.73273288,15.13214437)
\curveto(335.54002454,14.36131103)(335.44367038,13.54881103)(335.44367038,12.69464437)
\curveto(335.44367038,11.80401937)(335.53742038,10.9889152)(335.72492038,10.24933187)
\curveto(335.91762871,9.50974853)(336.17544121,8.80662353)(336.49835788,8.13995687)
\curveto(336.81085788,7.49933187)(337.16762871,6.92120687)(337.56867038,6.40558187)
\curveto(337.96971204,5.88474853)(338.37075371,5.44464437)(338.77179538,5.08526937)
\closepath
}
}
{
\newrgbcolor{curcolor}{0 0 0}
\pscustom[linestyle=none,fillstyle=solid,fillcolor=curcolor]
{
\newpath
\moveto(349.04523288,12.65558187)
\curveto(349.04523288,11.9264152)(348.94106621,11.2701652)(348.73273288,10.68683187)
\curveto(348.52960788,10.10349853)(348.25356621,9.6139152)(347.90460788,9.21808187)
\curveto(347.53481621,8.80662353)(347.12856621,8.4967277)(346.68585788,8.28839437)
\curveto(346.24314954,8.08526937)(345.75617038,7.98370687)(345.22492038,7.98370687)
\curveto(344.73012871,7.98370687)(344.29783704,8.0436027)(343.92804538,8.16339437)
\curveto(343.55825371,8.2779777)(343.19367038,8.4342277)(342.83429538,8.63214437)
\lineto(342.74054538,8.22589437)
\lineto(341.36554538,8.22589437)
\lineto(341.36554538,20.38214437)
\lineto(342.83429538,20.38214437)
\lineto(342.83429538,16.03839437)
\curveto(343.24575371,16.37693603)(343.68325371,16.6529777)(344.14679538,16.86651937)
\curveto(344.61033704,17.08526937)(345.13117038,17.19464437)(345.70929538,17.19464437)
\curveto(346.74054538,17.19464437)(347.55304538,16.79881103)(348.14679538,16.00714437)
\curveto(348.74575371,15.2154777)(349.04523288,14.0982902)(349.04523288,12.65558187)
\closepath
\moveto(347.52960788,12.61651937)
\curveto(347.52960788,13.65818603)(347.35773288,14.44724853)(347.01398288,14.98370687)
\curveto(346.67023288,15.52537353)(346.11554538,15.79620687)(345.34992038,15.79620687)
\curveto(344.92283704,15.79620687)(344.49054538,15.70245687)(344.05304538,15.51495687)
\curveto(343.61554538,15.3326652)(343.20929538,15.09568603)(342.83429538,14.80401937)
\lineto(342.83429538,9.80401937)
\curveto(343.25096204,9.61651937)(343.60773288,9.48631103)(343.90460788,9.41339437)
\curveto(344.20669121,9.3404777)(344.54783704,9.30401937)(344.92804538,9.30401937)
\curveto(345.74054538,9.30401937)(346.37596204,9.56964437)(346.83429538,10.10089437)
\curveto(347.29783704,10.6373527)(347.52960788,11.47589437)(347.52960788,12.61651937)
\closepath
}
}
{
\newrgbcolor{curcolor}{0 0 0}
\pscustom[linestyle=none,fillstyle=solid,fillcolor=curcolor]
{
\newpath
\moveto(355.74054538,12.69464437)
\curveto(355.74054538,11.12693603)(355.48533704,9.71808187)(354.97492038,8.46808187)
\curveto(354.46450371,7.21808187)(353.74835788,6.06443603)(352.82648288,5.00714437)
\lineto(351.03742038,5.00714437)
\lineto(351.03742038,5.08526937)
\curveto(351.43846204,5.44464437)(351.83950371,5.88474853)(352.24054538,6.40558187)
\curveto(352.64679538,6.92120687)(353.00356621,7.49933187)(353.31085788,8.13995687)
\curveto(353.63377454,8.80662353)(353.88898288,9.50974853)(354.07648288,10.24933187)
\curveto(354.26919121,10.9889152)(354.36554538,11.80401937)(354.36554538,12.69464437)
\curveto(354.36554538,13.54881103)(354.26919121,14.36131103)(354.07648288,15.13214437)
\curveto(353.88377454,15.9029777)(353.62856621,16.60870687)(353.31085788,17.24933187)
\curveto(352.98273288,17.9107902)(352.62596204,18.4889152)(352.24054538,18.98370687)
\curveto(351.86033704,19.48370687)(351.45929538,19.92381103)(351.03742038,20.30401937)
\lineto(351.03742038,20.38214437)
\lineto(352.82648288,20.38214437)
\curveto(353.74835788,19.3248527)(354.46450371,18.17120687)(354.97492038,16.92120687)
\curveto(355.48533704,15.67120687)(355.74054538,14.2623527)(355.74054538,12.69464437)
\closepath
}
}
\end{pspicture}

    \caption{(a)三角形的世界空间边界框通过将其物体空间边界框
        变换到世界空间再寻找包围结果边界框的边界框算得;可能得到肥大的边界框。
        (b)然而,如果三角形的顶点首先从物体空间变换到世界空间再包围,边界框可能合适得多。}
    \label{fig:3.1}
\end{figure}

\begin{lstlisting}
`\refcode{Shape Method Definitions}{+=}\lastnext{ShapeMethodDefinitions}`
`\refvar{Bounds3f}{}` `\refvar{Shape}{}`::`\initvar[Shape::WorldBound]{WorldBound}{}`() const {
    return (*`\refvar{ObjectToWorld}{}`)(`\refvar{ObjectBound}{}`());
}
\end{lstlisting}

\subsection{光线-边界相交}\label{sub:光线-边界相交}
有了\refvar{Bounds3f}{}实例包围形状的用法后,我们将添加一个\refvar{Bounds3}{}方法,即
\refvar{Bounds3::IntersectP}{()},以检查光线-框相交性,
并且如果有的话还返回相交处的两个参数$t$值。

认识边界框的一种方式是将其看作三块厚板\sidenote{译者注:原文slab。}的交集,
每一块是两平行平面之间的空间区域。
为了求光线与框的相交性,我们让光线依次与框的三块厚板相交。
因为厚板与三个坐标轴对齐,在光线-厚板相交中可以应用大量优化。

基本的光线-边界框相交算法算法工作如下:
我们从一个参数化区间开始,它沿我们想要求交的光线覆盖了位置$t$的范围;
该范围通常是$(0,+\infty)$。
然后我们先后计算光线与每个厚板相交的两个参数化$t$位置。
我们计算每个厚板相交区间与当前相交区间的交集,
如果我们发现结果区间退化则返回失败。
如果检查完全部三个厚板后区间是非退化的,
我们就有了光线位于框内的参数范围。
\reffig{3.2}{}说明了该过程,
\reffig{3.3}展示了光线与厚板相交的基本几何结构。
\begin{figure}[htb]
    \centering%LaTeX with PSTricks extensions
%%Creator: Inkscape 1.0.1 (3bc2e813f5, 2020-09-07)
%%Please note this file requires PSTricks extensions
\psset{xunit=.35pt,yunit=.35pt,runit=.35pt}
\begin{pspicture}(405.83999634,310.61999512)
{
\newrgbcolor{curcolor}{0 0.44313726 0.73725492}
\pscustom[linewidth=1.5,linecolor=curcolor]
{
\newpath
\moveto(139.65869141,93.92999268)
\lineto(321.72869873,216.02999878)
}
}
{
\newrgbcolor{curcolor}{0.60000002 0.60000002 0.60000002}
\pscustom[linewidth=1,linecolor=curcolor]
{
\newpath
\moveto(105.2986908,310.61999512)
\lineto(105.2986908,21.85998535)
}
}
{
\newrgbcolor{curcolor}{0.60000002 0.60000002 0.60000002}
\pscustom[linewidth=1,linecolor=curcolor]
{
\newpath
\moveto(321.53869629,310.61999512)
\lineto(321.53869629,21.85998535)
}
}
{
\newrgbcolor{curcolor}{0.60000002 0.60000002 0.60000002}
\pscustom[linewidth=1,linecolor=curcolor]
{
\newpath
\moveto(34.41869354,239.12999725)
\lineto(395.06869507,239.12999725)
}
}
{
\newrgbcolor{curcolor}{0.60000002 0.60000002 0.60000002}
\pscustom[linewidth=1,linecolor=curcolor]
{
\newpath
\moveto(34.41869354,93.97000122)
\lineto(395.06869507,93.97000122)
}
}
{
\newrgbcolor{curcolor}{0 0 0}
\pscustom[linestyle=none,fillstyle=solid,fillcolor=curcolor]
{
\newpath
\moveto(109.33870268,70.9099884)
\curveto(109.33870268,74.49167497)(105.00860571,76.28474111)(102.47627785,73.75241325)
\curveto(99.94394999,71.22008539)(101.73701613,66.88998842)(105.3187027,66.88998842)
\curveto(108.90038926,66.88998842)(110.69345541,71.22008539)(108.16112755,73.75241325)
\curveto(105.62879968,76.28474111)(101.29870272,74.49167497)(101.29870272,70.9099884)
\curveto(101.29870272,67.32830184)(105.62879968,65.53523569)(108.16112755,68.06756355)
\curveto(110.69345541,70.59989142)(108.90038926,74.92998838)(105.3187027,74.92998838)
\curveto(101.73701613,74.92998838)(99.94394999,70.59989142)(102.47627785,68.06756355)
\curveto(105.00860571,65.53523569)(109.33870268,67.32830184)(109.33870268,70.9099884)
\closepath
}
}
{
\newrgbcolor{curcolor}{0 0 0}
\pscustom[linestyle=none,fillstyle=solid,fillcolor=curcolor]
{
\newpath
\moveto(143.72869444,94.02999878)
\curveto(143.72869444,97.61168535)(139.39859747,99.40475149)(136.86626961,96.87242363)
\curveto(134.33394175,94.34009576)(136.12700789,90.0099988)(139.70869446,90.0099988)
\curveto(143.29038102,90.0099988)(145.08344717,94.34009576)(142.55111931,96.87242363)
\curveto(140.01879144,99.40475149)(135.68869448,97.61168535)(135.68869448,94.02999878)
\curveto(135.68869448,90.44831221)(140.01879144,88.65524607)(142.55111931,91.18757393)
\curveto(145.08344717,93.71990179)(143.29038102,98.04999876)(139.70869446,98.04999876)
\curveto(136.12700789,98.04999876)(134.33394175,93.71990179)(136.86626961,91.18757393)
\curveto(139.39859747,88.65524607)(143.72869444,90.44831221)(143.72869444,94.02999878)
\closepath
}
}
{
\newrgbcolor{curcolor}{0 0 0}
\pscustom[linestyle=none,fillstyle=solid,fillcolor=curcolor]
{
\newpath
\moveto(360.25870848,239.1499939)
\curveto(360.25870848,242.73168046)(355.92861151,244.52474661)(353.39628365,241.99241875)
\curveto(350.86395578,239.46009088)(352.65702193,235.12999392)(356.2387085,235.12999392)
\curveto(359.82039506,235.12999392)(361.61346121,239.46009088)(359.08113334,241.99241875)
\curveto(356.54880548,244.52474661)(352.21870852,242.73168046)(352.21870852,239.1499939)
\curveto(352.21870852,235.56830733)(356.54880548,233.77524118)(359.08113334,236.30756905)
\curveto(361.61346121,238.83989691)(359.82039506,243.16999388)(356.2387085,243.16999388)
\curveto(352.65702193,243.16999388)(350.86395578,238.83989691)(353.39628365,236.30756905)
\curveto(355.92861151,233.77524118)(360.25870848,235.56830733)(360.25870848,239.1499939)
\closepath
}
}
{
\newrgbcolor{curcolor}{0 0 0}
\pscustom[linestyle=none,fillstyle=solid,fillcolor=curcolor]
{
\newpath
\moveto(325.53870726,215.86999512)
\curveto(325.53870726,219.45168168)(321.20861029,221.24474783)(318.67628243,218.71241997)
\curveto(316.14395456,216.1800921)(317.93702071,211.84999514)(321.51870728,211.84999514)
\curveto(325.10039384,211.84999514)(326.89345999,216.1800921)(324.36113212,218.71241997)
\curveto(321.82880426,221.24474783)(317.49870729,219.45168168)(317.49870729,215.86999512)
\curveto(317.49870729,212.28830855)(321.82880426,210.49524241)(324.36113212,213.02757027)
\curveto(326.89345999,215.55989813)(325.10039384,219.8899951)(321.51870728,219.8899951)
\curveto(317.93702071,219.8899951)(316.14395456,215.55989813)(318.67628243,213.02757027)
\curveto(321.20861029,210.49524241)(325.53870726,212.28830855)(325.53870726,215.86999512)
\closepath
}
}
{
\newrgbcolor{curcolor}{0 0 0}
\pscustom[linewidth=1,linecolor=curcolor]
{
\newpath
\moveto(349.478695,247.86999512)
\lineto(133.998695,103.35999512)
\lineto(136.498695,99.08999512)
}
}
{
\newrgbcolor{curcolor}{0 0 0}
\pscustom[linewidth=1,linecolor=curcolor]
{
\newpath
\moveto(111.66869354,60.75999451)
\lineto(328.11871338,205.92999268)
}
}
{
\newrgbcolor{curcolor}{0 0 0}
\pscustom[linewidth=1,linecolor=curcolor]
{
\newpath
\moveto(112.248703,60.87998962)
\lineto(108.83869171,65.61000061)
}
}
{
\newrgbcolor{curcolor}{0 0 0}
\pscustom[linewidth=1,linecolor=curcolor]
{
\newpath
\moveto(349.0887146,247.51999664)
\lineto(351.91870117,243.41999817)
}
}
{
\newrgbcolor{curcolor}{0 0 0}
\pscustom[linewidth=1,linecolor=curcolor]
{
\newpath
\moveto(327.75869751,205.61999512)
\lineto(324.72869873,209.94999695)
}
}
{
\newrgbcolor{curcolor}{0 0 0}
\pscustom[linewidth=1.5,linecolor=curcolor]
{
\newpath
\moveto(321.72869873,216.02999878)
\lineto(378.07870483,253.8299942)
}
}
{
\newrgbcolor{curcolor}{0 0 0}
\pscustom[linestyle=none,fillstyle=solid,fillcolor=curcolor]
{
\newpath
\moveto(377.268695,253.28999512)
\lineto(367.368695,256.57999512)
\lineto(388.178695,260.59999512)
\lineto(376.568695,242.86999512)
\closepath
}
}
{
\newrgbcolor{curcolor}{0.65098041 0.65098041 0.65098041}
\pscustom[linestyle=none,fillstyle=solid,fillcolor=curcolor]
{
\newpath
\moveto(386.548695,259.49999512)
\lineto(378.058695,253.80999512)
\lineto(377.498695,245.66999512)
\closepath
}
}
{
\newrgbcolor{curcolor}{0.40000001 0.40000001 0.40000001}
\pscustom[linestyle=none,fillstyle=solid,fillcolor=curcolor]
{
\newpath
\moveto(386.548695,259.49999512)
\lineto(378.058695,253.80999512)
\lineto(370.318695,256.37999512)
\closepath
}
}
{
\newrgbcolor{curcolor}{0 0 0}
\pscustom[linewidth=1.5,linecolor=curcolor]
{
\newpath
\moveto(65.13869476,43.94998169)
\lineto(139.65869141,93.92999268)
}
}
{
\newrgbcolor{curcolor}{0 0 0}
\pscustom[linestyle=none,fillstyle=solid,fillcolor=curcolor]
{
\newpath
\moveto(246.18959279,136.3946905)
\curveto(246.36959279,137.1146905)(247.04459279,139.7696905)(249.02459279,139.7696905)
\curveto(249.15959279,139.7696905)(249.87959279,139.7696905)(250.46459279,139.4096905)
\curveto(249.65459279,139.2296905)(249.11459279,138.5546905)(249.11459279,137.8346905)
\curveto(249.11459279,137.3846905)(249.42959279,136.8446905)(250.19459279,136.8446905)
\curveto(250.82459279,136.8446905)(251.72459279,137.3396905)(251.72459279,138.5096905)
\curveto(251.72459279,139.9946905)(250.05959279,140.3996905)(249.06959279,140.3996905)
\curveto(247.40459279,140.3996905)(246.41459279,138.8696905)(246.05459279,138.2396905)
\curveto(245.33459279,140.1296905)(243.80459279,140.3996905)(242.94959279,140.3996905)
\curveto(239.97959279,140.3996905)(238.31459279,136.7096905)(238.31459279,135.9896905)
\curveto(238.31459279,135.6746905)(238.62959279,135.6746905)(238.67459279,135.6746905)
\curveto(238.89959279,135.6746905)(238.98959279,135.7646905)(239.03459279,135.9896905)
\curveto(240.02459279,139.0496905)(241.91459279,139.7696905)(242.90459279,139.7696905)
\curveto(243.44459279,139.7696905)(244.43459279,139.4996905)(244.43459279,137.8346905)
\curveto(244.43459279,136.9346905)(243.93959279,135.0446905)(242.90459279,130.9946905)
\curveto(242.45459279,129.2396905)(241.41959279,128.0246905)(240.15959279,128.0246905)
\curveto(239.97959279,128.0246905)(239.34959279,128.0246905)(238.71959279,128.3846905)
\curveto(239.43959279,128.5646905)(240.06959279,129.1496905)(240.06959279,129.9596905)
\curveto(240.06959279,130.7246905)(239.43959279,130.9496905)(239.03459279,130.9496905)
\curveto(238.13459279,130.9496905)(237.45959279,130.2296905)(237.45959279,129.2846905)
\curveto(237.45959279,127.9796905)(238.85459279,127.3946905)(240.11459279,127.3946905)
\curveto(242.04959279,127.3946905)(243.08459279,129.4196905)(243.12959279,129.5546905)
\curveto(243.48959279,128.5196905)(244.52459279,127.3946905)(246.23459279,127.3946905)
\curveto(249.20459279,127.3946905)(250.82459279,131.0846905)(250.82459279,131.8046905)
\curveto(250.82459279,132.1196905)(250.59959279,132.1196905)(250.50959279,132.1196905)
\curveto(250.23959279,132.1196905)(250.19459279,131.9846905)(250.10459279,131.8046905)
\curveto(249.15959279,128.6996905)(247.22459279,128.0246905)(246.32459279,128.0246905)
\curveto(245.19959279,128.0246905)(244.74959279,128.9246905)(244.74959279,129.9146905)
\curveto(244.74959279,130.5446905)(244.88459279,131.1746905)(245.19959279,132.4346905)
\closepath
\moveto(246.18959279,136.3946905)
}
}
{
\newrgbcolor{curcolor}{0 0 0}
\pscustom[linestyle=none,fillstyle=solid,fillcolor=curcolor]
{
\newpath
\moveto(102.91989879,14.9150665)
\curveto(103.09989879,15.6350665)(103.77489879,18.2900665)(105.75489879,18.2900665)
\curveto(105.88989879,18.2900665)(106.60989879,18.2900665)(107.19489879,17.9300665)
\curveto(106.38489879,17.7500665)(105.84489879,17.0750665)(105.84489879,16.3550665)
\curveto(105.84489879,15.9050665)(106.15989879,15.3650665)(106.92489879,15.3650665)
\curveto(107.55489879,15.3650665)(108.45489879,15.8600665)(108.45489879,17.0300665)
\curveto(108.45489879,18.5150665)(106.78989879,18.9200665)(105.79989879,18.9200665)
\curveto(104.13489879,18.9200665)(103.14489879,17.3900665)(102.78489879,16.7600665)
\curveto(102.06489879,18.6500665)(100.53489879,18.9200665)(99.67989879,18.9200665)
\curveto(96.70989879,18.9200665)(95.04489879,15.2300665)(95.04489879,14.5100665)
\curveto(95.04489879,14.1950665)(95.35989879,14.1950665)(95.40489879,14.1950665)
\curveto(95.62989879,14.1950665)(95.71989879,14.2850665)(95.76489879,14.5100665)
\curveto(96.75489879,17.5700665)(98.64489879,18.2900665)(99.63489879,18.2900665)
\curveto(100.17489879,18.2900665)(101.16489879,18.0200665)(101.16489879,16.3550665)
\curveto(101.16489879,15.4550665)(100.66989879,13.5650665)(99.63489879,9.5150665)
\curveto(99.18489879,7.7600665)(98.14989879,6.5450665)(96.88989879,6.5450665)
\curveto(96.70989879,6.5450665)(96.07989879,6.5450665)(95.44989879,6.9050665)
\curveto(96.16989879,7.0850665)(96.79989879,7.6700665)(96.79989879,8.4800665)
\curveto(96.79989879,9.2450665)(96.16989879,9.4700665)(95.76489879,9.4700665)
\curveto(94.86489879,9.4700665)(94.18989879,8.7500665)(94.18989879,7.8050665)
\curveto(94.18989879,6.5000665)(95.58489879,5.9150665)(96.84489879,5.9150665)
\curveto(98.77989879,5.9150665)(99.81489879,7.9400665)(99.85989879,8.0750665)
\curveto(100.21989879,7.0400665)(101.25489879,5.9150665)(102.96489879,5.9150665)
\curveto(105.93489879,5.9150665)(107.55489879,9.6050665)(107.55489879,10.3250665)
\curveto(107.55489879,10.6400665)(107.32989879,10.6400665)(107.23989879,10.6400665)
\curveto(106.96989879,10.6400665)(106.92489879,10.5050665)(106.83489879,10.3250665)
\curveto(105.88989879,7.2200665)(103.95489879,6.5450665)(103.05489879,6.5450665)
\curveto(101.92989879,6.5450665)(101.47989879,7.4450665)(101.47989879,8.4350665)
\curveto(101.47989879,9.0650665)(101.61489879,9.6950665)(101.92989879,10.9550665)
\closepath
\moveto(102.91989879,14.9150665)
}
}
{
\newrgbcolor{curcolor}{0 0 0}
\pscustom[linestyle=none,fillstyle=solid,fillcolor=curcolor]
{
\newpath
\moveto(120.08361949,8.31733701)
\curveto(120.08361949,10.52233701)(119.81361949,12.14233701)(118.91361949,13.53733701)
\curveto(118.28361949,14.43733701)(117.02361949,15.24733701)(115.44861949,15.24733701)
\curveto(110.76861949,15.24733701)(110.76861949,9.75733701)(110.76861949,8.31733701)
\curveto(110.76861949,6.87733701)(110.76861949,1.52233701)(115.44861949,1.52233701)
\curveto(120.08361949,1.52233701)(120.08361949,6.87733701)(120.08361949,8.31733701)
\closepath
\moveto(115.44861949,2.10733701)
\curveto(114.50361949,2.10733701)(113.28861949,2.64733701)(112.88361949,4.26733701)
\curveto(112.61361949,5.43733701)(112.61361949,7.10233701)(112.61361949,8.58733701)
\curveto(112.61361949,10.07233701)(112.61361949,11.60233701)(112.88361949,12.68233701)
\curveto(113.33361949,14.25733701)(114.59361949,14.70733701)(115.44861949,14.70733701)
\curveto(116.52861949,14.70733701)(117.56361949,14.03233701)(117.92361949,12.86233701)
\curveto(118.23861949,11.78233701)(118.28361949,10.34233701)(118.28361949,8.58733701)
\curveto(118.28361949,7.10233701)(118.28361949,5.61733701)(118.01361949,4.35733701)
\curveto(117.60861949,2.51233701)(116.25861949,2.10733701)(115.44861949,2.10733701)
\closepath
\moveto(115.44861949,2.10733701)
}
}
{
\newrgbcolor{curcolor}{0 0 0}
\pscustom[linestyle=none,fillstyle=solid,fillcolor=curcolor]
{
\newpath
\moveto(317.55206279,15.4598175)
\curveto(317.73206279,16.1798175)(318.40706279,18.8348175)(320.38706279,18.8348175)
\curveto(320.52206279,18.8348175)(321.24206279,18.8348175)(321.82706279,18.4748175)
\curveto(321.01706279,18.2948175)(320.47706279,17.6198175)(320.47706279,16.8998175)
\curveto(320.47706279,16.4498175)(320.79206279,15.9098175)(321.55706279,15.9098175)
\curveto(322.18706279,15.9098175)(323.08706279,16.4048175)(323.08706279,17.5748175)
\curveto(323.08706279,19.0598175)(321.42206279,19.4648175)(320.43206279,19.4648175)
\curveto(318.76706279,19.4648175)(317.77706279,17.9348175)(317.41706279,17.3048175)
\curveto(316.69706279,19.1948175)(315.16706279,19.4648175)(314.31206279,19.4648175)
\curveto(311.34206279,19.4648175)(309.67706279,15.7748175)(309.67706279,15.0548175)
\curveto(309.67706279,14.7398175)(309.99206279,14.7398175)(310.03706279,14.7398175)
\curveto(310.26206279,14.7398175)(310.35206279,14.8298175)(310.39706279,15.0548175)
\curveto(311.38706279,18.1148175)(313.27706279,18.8348175)(314.26706279,18.8348175)
\curveto(314.80706279,18.8348175)(315.79706279,18.5648175)(315.79706279,16.8998175)
\curveto(315.79706279,15.9998175)(315.30206279,14.1098175)(314.26706279,10.0598175)
\curveto(313.81706279,8.3048175)(312.78206279,7.0898175)(311.52206279,7.0898175)
\curveto(311.34206279,7.0898175)(310.71206279,7.0898175)(310.08206279,7.4498175)
\curveto(310.80206279,7.6298175)(311.43206279,8.2148175)(311.43206279,9.0248175)
\curveto(311.43206279,9.7898175)(310.80206279,10.0148175)(310.39706279,10.0148175)
\curveto(309.49706279,10.0148175)(308.82206279,9.2948175)(308.82206279,8.3498175)
\curveto(308.82206279,7.0448175)(310.21706279,6.4598175)(311.47706279,6.4598175)
\curveto(313.41206279,6.4598175)(314.44706279,8.4848175)(314.49206279,8.6198175)
\curveto(314.85206279,7.5848175)(315.88706279,6.4598175)(317.59706279,6.4598175)
\curveto(320.56706279,6.4598175)(322.18706279,10.1498175)(322.18706279,10.8698175)
\curveto(322.18706279,11.1848175)(321.96206279,11.1848175)(321.87206279,11.1848175)
\curveto(321.60206279,11.1848175)(321.55706279,11.0498175)(321.46706279,10.8698175)
\curveto(320.52206279,7.7648175)(318.58706279,7.0898175)(317.68706279,7.0898175)
\curveto(316.56206279,7.0898175)(316.11206279,7.9898175)(316.11206279,8.9798175)
\curveto(316.11206279,9.6098175)(316.24706279,10.2398175)(316.56206279,11.4998175)
\closepath
\moveto(317.55206279,15.4598175)
}
}
{
\newrgbcolor{curcolor}{0 0 0}
\pscustom[linestyle=none,fillstyle=solid,fillcolor=curcolor]
{
\newpath
\moveto(331.07078349,15.25208801)
\curveto(331.07078349,15.79208801)(331.07078349,15.79208801)(330.48578349,15.79208801)
\curveto(329.18078349,14.53208801)(327.38078349,14.53208801)(326.57078349,14.53208801)
\lineto(326.57078349,13.81208801)
\curveto(327.02078349,13.81208801)(328.37078349,13.81208801)(329.45078349,14.35208801)
\lineto(329.45078349,4.13708801)
\curveto(329.45078349,3.46208801)(329.45078349,3.19208801)(327.47078349,3.19208801)
\lineto(326.70578349,3.19208801)
\lineto(326.70578349,2.47208801)
\curveto(327.06578349,2.47208801)(329.54078349,2.56208801)(330.26078349,2.56208801)
\curveto(330.89078349,2.56208801)(333.41078349,2.47208801)(333.86078349,2.47208801)
\lineto(333.86078349,3.19208801)
\lineto(333.09578349,3.19208801)
\curveto(331.07078349,3.19208801)(331.07078349,3.46208801)(331.07078349,4.13708801)
\closepath
\moveto(331.07078349,15.25208801)
}
}
{
\newrgbcolor{curcolor}{0 0 0}
\pscustom[linestyle=none,fillstyle=solid,fillcolor=curcolor]
{
\newpath
\moveto(23.39313879,245.9607725)
\curveto(23.52813879,246.3657725)(23.52813879,246.4107725)(23.52813879,246.6357725)
\curveto(23.52813879,247.1307725)(23.12313879,247.4007725)(22.67313879,247.4007725)
\curveto(22.40313879,247.4007725)(21.95313879,247.2207725)(21.68313879,246.8157725)
\curveto(21.63813879,246.6357725)(21.36813879,245.7807725)(21.27813879,245.2407725)
\curveto(21.05313879,244.5207725)(20.87313879,243.7107725)(20.69313879,242.9457725)
\lineto(19.38813879,237.7707725)
\curveto(19.29813879,237.3657725)(18.03813879,235.3407725)(16.14813879,235.3407725)
\curveto(14.70813879,235.3407725)(14.39313879,236.6007725)(14.39313879,237.6807725)
\curveto(14.39313879,238.9857725)(14.88813879,240.7857725)(15.83313879,243.3057725)
\curveto(16.28313879,244.4757725)(16.41813879,244.7907725)(16.41813879,245.3757725)
\curveto(16.41813879,246.6357725)(15.51813879,247.7157725)(14.07813879,247.7157725)
\curveto(11.33313879,247.7157725)(10.29813879,243.5307725)(10.29813879,243.3057725)
\curveto(10.29813879,242.9907725)(10.56813879,242.9907725)(10.61313879,242.9907725)
\curveto(10.92813879,242.9907725)(10.92813879,243.0807725)(11.06313879,243.5307725)
\curveto(11.87313879,246.2307725)(12.99813879,247.0857725)(13.98813879,247.0857725)
\curveto(14.21313879,247.0857725)(14.70813879,247.0857725)(14.70813879,246.1857725)
\curveto(14.70813879,245.4657725)(14.39313879,244.7007725)(14.21313879,244.1607725)
\curveto(13.04313879,241.1007725)(12.54813879,239.4807725)(12.54813879,238.1307725)
\curveto(12.54813879,235.5657725)(14.34813879,234.7107725)(16.05813879,234.7107725)
\curveto(17.18313879,234.7107725)(18.12813879,235.2057725)(18.93813879,236.0157725)
\curveto(18.57813879,234.5307725)(18.21813879,233.0907725)(17.09313879,231.5607725)
\curveto(16.32813879,230.6157725)(15.24813879,229.7607725)(13.94313879,229.7607725)
\curveto(13.53813879,229.7607725)(12.23313879,229.8507725)(11.73813879,230.9757725)
\curveto(12.18813879,230.9757725)(12.59313879,230.9757725)(12.95313879,231.3357725)
\curveto(13.26813879,231.5607725)(13.53813879,231.9657725)(13.53813879,232.5057725)
\curveto(13.53813879,233.4057725)(12.77313879,233.4957725)(12.50313879,233.4957725)
\curveto(11.82813879,233.4957725)(10.88313879,233.0457725)(10.88313879,231.6507725)
\curveto(10.88313879,230.2107725)(12.14313879,229.1307725)(13.94313879,229.1307725)
\curveto(16.86813879,229.1307725)(19.83813879,231.7407725)(20.64813879,234.9807725)
\closepath
\moveto(23.39313879,245.9607725)
}
}
{
\newrgbcolor{curcolor}{0 0 0}
\pscustom[linestyle=none,fillstyle=solid,fillcolor=curcolor]
{
\newpath
\moveto(30.21692297,243.50304301)
\curveto(30.21692297,244.04304301)(30.21692297,244.04304301)(29.63192297,244.04304301)
\curveto(28.32692297,242.78304301)(26.52692297,242.78304301)(25.71692297,242.78304301)
\lineto(25.71692297,242.06304301)
\curveto(26.16692297,242.06304301)(27.51692297,242.06304301)(28.59692297,242.60304301)
\lineto(28.59692297,232.38804301)
\curveto(28.59692297,231.71304301)(28.59692297,231.44304301)(26.61692297,231.44304301)
\lineto(25.85192297,231.44304301)
\lineto(25.85192297,230.72304301)
\curveto(26.21192297,230.72304301)(28.68692297,230.81304301)(29.40692297,230.81304301)
\curveto(30.03692297,230.81304301)(32.55692297,230.72304301)(33.00692297,230.72304301)
\lineto(33.00692297,231.44304301)
\lineto(32.24192297,231.44304301)
\curveto(30.21692297,231.44304301)(30.21692297,231.71304301)(30.21692297,232.38804301)
\closepath
\moveto(30.21692297,243.50304301)
}
}
{
\newrgbcolor{curcolor}{0 0 0}
\pscustom[linestyle=none,fillstyle=solid,fillcolor=curcolor]
{
\newpath
\moveto(21.75887879,102.6910795)
\curveto(21.89387879,103.0960795)(21.89387879,103.1410795)(21.89387879,103.3660795)
\curveto(21.89387879,103.8610795)(21.48887879,104.1310795)(21.03887879,104.1310795)
\curveto(20.76887879,104.1310795)(20.31887879,103.9510795)(20.04887879,103.5460795)
\curveto(20.00387879,103.3660795)(19.73387879,102.5110795)(19.64387879,101.9710795)
\curveto(19.41887879,101.2510795)(19.23887879,100.4410795)(19.05887879,99.6760795)
\lineto(17.75387879,94.5010795)
\curveto(17.66387879,94.0960795)(16.40387879,92.0710795)(14.51387879,92.0710795)
\curveto(13.07387879,92.0710795)(12.75887879,93.3310795)(12.75887879,94.4110795)
\curveto(12.75887879,95.7160795)(13.25387879,97.5160795)(14.19887879,100.0360795)
\curveto(14.64887879,101.2060795)(14.78387879,101.5210795)(14.78387879,102.1060795)
\curveto(14.78387879,103.3660795)(13.88387879,104.4460795)(12.44387879,104.4460795)
\curveto(9.69887879,104.4460795)(8.66387879,100.2610795)(8.66387879,100.0360795)
\curveto(8.66387879,99.7210795)(8.93387879,99.7210795)(8.97887879,99.7210795)
\curveto(9.29387879,99.7210795)(9.29387879,99.8110795)(9.42887879,100.2610795)
\curveto(10.23887879,102.9610795)(11.36387879,103.8160795)(12.35387879,103.8160795)
\curveto(12.57887879,103.8160795)(13.07387879,103.8160795)(13.07387879,102.9160795)
\curveto(13.07387879,102.1960795)(12.75887879,101.4310795)(12.57887879,100.8910795)
\curveto(11.40887879,97.8310795)(10.91387879,96.2110795)(10.91387879,94.8610795)
\curveto(10.91387879,92.2960795)(12.71387879,91.4410795)(14.42387879,91.4410795)
\curveto(15.54887879,91.4410795)(16.49387879,91.9360795)(17.30387879,92.7460795)
\curveto(16.94387879,91.2610795)(16.58387879,89.8210795)(15.45887879,88.2910795)
\curveto(14.69387879,87.3460795)(13.61387879,86.4910795)(12.30887879,86.4910795)
\curveto(11.90387879,86.4910795)(10.59887879,86.5810795)(10.10387879,87.7060795)
\curveto(10.55387879,87.7060795)(10.95887879,87.7060795)(11.31887879,88.0660795)
\curveto(11.63387879,88.2910795)(11.90387879,88.6960795)(11.90387879,89.2360795)
\curveto(11.90387879,90.1360795)(11.13887879,90.2260795)(10.86887879,90.2260795)
\curveto(10.19387879,90.2260795)(9.24887879,89.7760795)(9.24887879,88.3810795)
\curveto(9.24887879,86.9410795)(10.50887879,85.8610795)(12.30887879,85.8610795)
\curveto(15.23387879,85.8610795)(18.20387879,88.4710795)(19.01387879,91.7110795)
\closepath
\moveto(21.75887879,102.6910795)
}
}
{
\newrgbcolor{curcolor}{0 0 0}
\pscustom[linestyle=none,fillstyle=solid,fillcolor=curcolor]
{
\newpath
\moveto(32.22766297,93.84335001)
\curveto(32.22766297,96.04835001)(31.95766297,97.66835001)(31.05766297,99.06335001)
\curveto(30.42766297,99.96335001)(29.16766297,100.77335001)(27.59266297,100.77335001)
\curveto(22.91266297,100.77335001)(22.91266297,95.28335001)(22.91266297,93.84335001)
\curveto(22.91266297,92.40335001)(22.91266297,87.04835001)(27.59266297,87.04835001)
\curveto(32.22766297,87.04835001)(32.22766297,92.40335001)(32.22766297,93.84335001)
\closepath
\moveto(27.59266297,87.63335001)
\curveto(26.64766297,87.63335001)(25.43266297,88.17335001)(25.02766297,89.79335001)
\curveto(24.75766297,90.96335001)(24.75766297,92.62835001)(24.75766297,94.11335001)
\curveto(24.75766297,95.59835001)(24.75766297,97.12835001)(25.02766297,98.20835001)
\curveto(25.47766297,99.78335001)(26.73766297,100.23335001)(27.59266297,100.23335001)
\curveto(28.67266297,100.23335001)(29.70766297,99.55835001)(30.06766297,98.38835001)
\curveto(30.38266297,97.30835001)(30.42766297,95.86835001)(30.42766297,94.11335001)
\curveto(30.42766297,92.62835001)(30.42766297,91.14335001)(30.15766297,89.88335001)
\curveto(29.75266297,88.03835001)(28.40266297,87.63335001)(27.59266297,87.63335001)
\closepath
\moveto(27.59266297,87.63335001)
}
}
{
\newrgbcolor{curcolor}{0 0 0}
\pscustom[linestyle=none,fillstyle=solid,fillcolor=curcolor]
{
\newpath
\moveto(228.76452479,184.4038315)
\curveto(228.89952479,184.8088315)(228.89952479,184.8538315)(228.89952479,185.0788315)
\curveto(228.89952479,185.5738315)(228.49452479,185.8438315)(228.04452479,185.8438315)
\curveto(227.77452479,185.8438315)(227.32452479,185.6638315)(227.05452479,185.2588315)
\curveto(227.00952479,185.0788315)(226.73952479,184.2238315)(226.64952479,183.6838315)
\curveto(226.42452479,182.9638315)(226.24452479,182.1538315)(226.06452479,181.3888315)
\lineto(224.75952479,176.2138315)
\curveto(224.66952479,175.8088315)(223.40952479,173.7838315)(221.51952479,173.7838315)
\curveto(220.07952479,173.7838315)(219.76452479,175.0438315)(219.76452479,176.1238315)
\curveto(219.76452479,177.4288315)(220.25952479,179.2288315)(221.20452479,181.7488315)
\curveto(221.65452479,182.9188315)(221.78952479,183.2338315)(221.78952479,183.8188315)
\curveto(221.78952479,185.0788315)(220.88952479,186.1588315)(219.44952479,186.1588315)
\curveto(216.70452479,186.1588315)(215.66952479,181.9738315)(215.66952479,181.7488315)
\curveto(215.66952479,181.4338315)(215.93952479,181.4338315)(215.98452479,181.4338315)
\curveto(216.29952479,181.4338315)(216.29952479,181.5238315)(216.43452479,181.9738315)
\curveto(217.24452479,184.6738315)(218.36952479,185.5288315)(219.35952479,185.5288315)
\curveto(219.58452479,185.5288315)(220.07952479,185.5288315)(220.07952479,184.6288315)
\curveto(220.07952479,183.9088315)(219.76452479,183.1438315)(219.58452479,182.6038315)
\curveto(218.41452479,179.5438315)(217.91952479,177.9238315)(217.91952479,176.5738315)
\curveto(217.91952479,174.0088315)(219.71952479,173.1538315)(221.42952479,173.1538315)
\curveto(222.55452479,173.1538315)(223.49952479,173.6488315)(224.30952479,174.4588315)
\curveto(223.94952479,172.9738315)(223.58952479,171.5338315)(222.46452479,170.0038315)
\curveto(221.69952479,169.0588315)(220.61952479,168.2038315)(219.31452479,168.2038315)
\curveto(218.90952479,168.2038315)(217.60452479,168.2938315)(217.10952479,169.4188315)
\curveto(217.55952479,169.4188315)(217.96452479,169.4188315)(218.32452479,169.7788315)
\curveto(218.63952479,170.0038315)(218.90952479,170.4088315)(218.90952479,170.9488315)
\curveto(218.90952479,171.8488315)(218.14452479,171.9388315)(217.87452479,171.9388315)
\curveto(217.19952479,171.9388315)(216.25452479,171.4888315)(216.25452479,170.0938315)
\curveto(216.25452479,168.6538315)(217.51452479,167.5738315)(219.31452479,167.5738315)
\curveto(222.23952479,167.5738315)(225.20952479,170.1838315)(226.01952479,173.4238315)
\closepath
\moveto(228.76452479,184.4038315)
}
}
\end{pspicture}

    \caption{光线与轴对齐边界框相交。我们依次计算与每个厚板的交点,逐渐缩窄参数区间。
        这里的2D形式中,沿光线与$x$和$y$相交范围(蓝色线段)给出了框中光线的范围。}
    \label{fig:3.2}
\end{figure}
\begin{figure}[htb]
    \centering%LaTeX with PSTricks extensions
%%Creator: Inkscape 1.0.1 (3bc2e813f5, 2020-09-07)
%%Please note this file requires PSTricks extensions
\psset{xunit=.5pt,yunit=.5pt,runit=.5pt}
\begin{pspicture}(428.11999512,309.83999634)
{
\newrgbcolor{curcolor}{0.60000002 0.60000002 0.60000002}
\pscustom[linewidth=1,linecolor=curcolor]
{
\newpath
\moveto(40.58000183,309.83999634)
\lineto(40.58000183,21.07998657)
}
}
{
\newrgbcolor{curcolor}{0.60000002 0.60000002 0.60000002}
\pscustom[linewidth=1,linecolor=curcolor]
{
\newpath
\moveto(256.82000732,309.83999634)
\lineto(256.82000732,21.07998657)
}
}
{
\newrgbcolor{curcolor}{0 0 0}
\pscustom[linewidth=1.5,linecolor=curcolor]
{
\newpath
\moveto(0.41999999,43.16998291)
\lineto(313.35998535,253.04999542)
}
}
{
\newrgbcolor{curcolor}{0 0 0}
\pscustom[linestyle=none,fillstyle=solid,fillcolor=curcolor]
{
\newpath
\moveto(311.84,242.08999634)
\lineto(312.55,252.50999634)
\lineto(302.64,255.79999634)
\lineto(323.45,259.81999634)
\closepath
}
}
{
\newrgbcolor{curcolor}{0.65098041 0.65098041 0.65098041}
\pscustom[linestyle=none,fillstyle=solid,fillcolor=curcolor]
{
\newpath
\moveto(312.78,244.88999634)
\lineto(321.82,258.71999634)
\lineto(313.34,253.02999634)
\closepath
}
}
{
\newrgbcolor{curcolor}{0.40000001 0.40000001 0.40000001}
\pscustom[linestyle=none,fillstyle=solid,fillcolor=curcolor]
{
\newpath
\moveto(305.6,255.59999634)
\lineto(321.82,258.71999634)
\lineto(313.34,253.02999634)
\closepath
}
}
{
\newrgbcolor{curcolor}{0 0 0}
\pscustom[linestyle=none,fillstyle=solid,fillcolor=curcolor]
{
\newpath
\moveto(44.61000013,70.12998962)
\curveto(44.61000013,73.71167619)(40.27990317,75.50474234)(37.7475753,72.97241447)
\curveto(35.21524744,70.44008661)(37.00831359,66.10998964)(40.59000015,66.10998964)
\curveto(44.17168672,66.10998964)(45.96475286,70.44008661)(43.432425,72.97241447)
\curveto(40.90009714,75.50474234)(36.57000017,73.71167619)(36.57000017,70.12998962)
\curveto(36.57000017,66.54830306)(40.90009714,64.75523691)(43.432425,67.28756478)
\curveto(45.96475286,69.81989264)(44.17168672,74.1499896)(40.59000015,74.1499896)
\curveto(37.00831359,74.1499896)(35.21524744,69.81989264)(37.7475753,67.28756478)
\curveto(40.27990317,64.75523691)(44.61000013,66.54830306)(44.61000013,70.12998962)
\closepath
}
}
{
\newrgbcolor{curcolor}{0 0 0}
\pscustom[linestyle=none,fillstyle=solid,fillcolor=curcolor]
{
\newpath
\moveto(260.81998777,215.08999634)
\curveto(260.81998777,218.6716829)(256.48989081,220.46474905)(253.95756294,217.93242119)
\curveto(251.42523508,215.40009332)(253.21830123,211.06999636)(256.79998779,211.06999636)
\curveto(260.38167436,211.06999636)(262.1747405,215.40009332)(259.64241264,217.93242119)
\curveto(257.11008478,220.46474905)(252.77998781,218.6716829)(252.77998781,215.08999634)
\curveto(252.77998781,211.50830977)(257.11008478,209.71524363)(259.64241264,212.24757149)
\curveto(262.1747405,214.77989935)(260.38167436,219.10999632)(256.79998779,219.10999632)
\curveto(253.21830123,219.10999632)(251.42523508,214.77989935)(253.95756294,212.24757149)
\curveto(256.48989081,209.71524363)(260.81998777,211.50830977)(260.81998777,215.08999634)
\closepath
}
}
{
\newrgbcolor{curcolor}{0 0 0}
\pscustom[linewidth=1,linecolor=curcolor]
{
\newpath
\moveto(256.8999939,112.25)
\lineto(338.73001099,112.25)
}
}
{
\newrgbcolor{curcolor}{0 0 0}
\pscustom[linestyle=none,fillstyle=solid,fillcolor=curcolor]
{
\newpath
\moveto(333.82,106.73999634)
\lineto(338.08,112.24999634)
\lineto(333.82,117.74999634)
\lineto(346.83,112.24999634)
\closepath
}
}
{
\newrgbcolor{curcolor}{0.65098041 0.65098041 0.65098041}
\pscustom[linestyle=none,fillstyle=solid,fillcolor=curcolor]
{
\newpath
\moveto(335.38,107.93999634)
\lineto(345.52,112.24999634)
\lineto(338.71,112.24999634)
\closepath
}
}
{
\newrgbcolor{curcolor}{0.40000001 0.40000001 0.40000001}
\pscustom[linestyle=none,fillstyle=solid,fillcolor=curcolor]
{
\newpath
\moveto(335.38,116.53999634)
\lineto(345.52,112.24999634)
\lineto(338.71,112.24999634)
\closepath
}
}
{
\newrgbcolor{curcolor}{0 0 0}
\pscustom[linestyle=none,fillstyle=solid,fillcolor=curcolor]
{
\newpath
\moveto(24.8654161,12.75434146)
\curveto(24.9904161,13.25434146)(25.4591661,15.09809146)(26.8341661,15.09809146)
\curveto(26.9279161,15.09809146)(27.4279161,15.09809146)(27.8341661,14.84809146)
\curveto(27.2716661,14.72309146)(26.8966661,14.25434146)(26.8966661,13.75434146)
\curveto(26.8966661,13.44184146)(27.1154161,13.06684146)(27.6466661,13.06684146)
\curveto(28.0841661,13.06684146)(28.7091661,13.41059146)(28.7091661,14.22309146)
\curveto(28.7091661,15.25434146)(27.5529161,15.53559146)(26.8654161,15.53559146)
\curveto(25.7091661,15.53559146)(25.0216661,14.47309146)(24.7716661,14.03559146)
\curveto(24.2716661,15.34809146)(23.2091661,15.53559146)(22.6154161,15.53559146)
\curveto(20.5529161,15.53559146)(19.3966661,12.97309146)(19.3966661,12.47309146)
\curveto(19.3966661,12.25434146)(19.6154161,12.25434146)(19.6466661,12.25434146)
\curveto(19.8029161,12.25434146)(19.8654161,12.31684146)(19.8966661,12.47309146)
\curveto(20.5841661,14.59809146)(21.8966661,15.09809146)(22.5841661,15.09809146)
\curveto(22.9591661,15.09809146)(23.6466661,14.91059146)(23.6466661,13.75434146)
\curveto(23.6466661,13.12934146)(23.3029161,11.81684146)(22.5841661,9.00434146)
\curveto(22.2716661,7.78559146)(21.5529161,6.94184146)(20.6779161,6.94184146)
\curveto(20.5529161,6.94184146)(20.1154161,6.94184146)(19.6779161,7.19184146)
\curveto(20.1779161,7.31684146)(20.6154161,7.72309146)(20.6154161,8.28559146)
\curveto(20.6154161,8.81684146)(20.1779161,8.97309146)(19.8966661,8.97309146)
\curveto(19.2716661,8.97309146)(18.8029161,8.47309146)(18.8029161,7.81684146)
\curveto(18.8029161,6.91059146)(19.7716661,6.50434146)(20.6466661,6.50434146)
\curveto(21.9904161,6.50434146)(22.7091661,7.91059146)(22.7404161,8.00434146)
\curveto(22.9904161,7.28559146)(23.7091661,6.50434146)(24.8966661,6.50434146)
\curveto(26.9591661,6.50434146)(28.0841661,9.06684146)(28.0841661,9.56684146)
\curveto(28.0841661,9.78559146)(27.9279161,9.78559146)(27.8654161,9.78559146)
\curveto(27.6779161,9.78559146)(27.6466661,9.69184146)(27.5841661,9.56684146)
\curveto(26.9279161,7.41059146)(25.5841661,6.94184146)(24.9591661,6.94184146)
\curveto(24.1779161,6.94184146)(23.8654161,7.56684146)(23.8654161,8.25434146)
\curveto(23.8654161,8.69184146)(23.9591661,9.12934146)(24.1779161,10.00434146)
\closepath
\moveto(24.8654161,12.75434146)
}
}
{
\newrgbcolor{curcolor}{0 0 0}
\pscustom[linestyle=none,fillstyle=solid,fillcolor=curcolor]
{
\newpath
\moveto(48.82066208,13.25434146)
\curveto(49.13316208,13.25434146)(49.50816208,13.25434146)(49.50816208,13.62934146)
\curveto(49.50816208,14.03559146)(49.13316208,14.03559146)(48.85191208,14.03559146)
\lineto(36.91441208,14.03559146)
\curveto(36.63316208,14.03559146)(36.25816208,14.03559146)(36.25816208,13.62934146)
\curveto(36.25816208,13.25434146)(36.63316208,13.25434146)(36.91441208,13.25434146)
\closepath
\moveto(48.85191208,9.37934146)
\curveto(49.13316208,9.37934146)(49.50816208,9.37934146)(49.50816208,9.78559146)
\curveto(49.50816208,10.16059146)(49.13316208,10.16059146)(48.82066208,10.16059146)
\lineto(36.91441208,10.16059146)
\curveto(36.63316208,10.16059146)(36.25816208,10.16059146)(36.25816208,9.78559146)
\curveto(36.25816208,9.37934146)(36.63316208,9.37934146)(36.91441208,9.37934146)
\closepath
\moveto(48.85191208,9.37934146)
}
}
{
\newrgbcolor{curcolor}{0 0 0}
\pscustom[linestyle=none,fillstyle=solid,fillcolor=curcolor]
{
\newpath
\moveto(62.82140976,12.75434146)
\curveto(62.94640976,13.25434146)(63.41515976,15.09809146)(64.79015976,15.09809146)
\curveto(64.88390976,15.09809146)(65.38390976,15.09809146)(65.79015976,14.84809146)
\curveto(65.22765976,14.72309146)(64.85265976,14.25434146)(64.85265976,13.75434146)
\curveto(64.85265976,13.44184146)(65.07140976,13.06684146)(65.60265976,13.06684146)
\curveto(66.04015976,13.06684146)(66.66515976,13.41059146)(66.66515976,14.22309146)
\curveto(66.66515976,15.25434146)(65.50890976,15.53559146)(64.82140976,15.53559146)
\curveto(63.66515976,15.53559146)(62.97765976,14.47309146)(62.72765976,14.03559146)
\curveto(62.22765976,15.34809146)(61.16515976,15.53559146)(60.57140976,15.53559146)
\curveto(58.50890976,15.53559146)(57.35265976,12.97309146)(57.35265976,12.47309146)
\curveto(57.35265976,12.25434146)(57.57140976,12.25434146)(57.60265976,12.25434146)
\curveto(57.75890976,12.25434146)(57.82140976,12.31684146)(57.85265976,12.47309146)
\curveto(58.54015976,14.59809146)(59.85265976,15.09809146)(60.54015976,15.09809146)
\curveto(60.91515976,15.09809146)(61.60265976,14.91059146)(61.60265976,13.75434146)
\curveto(61.60265976,13.12934146)(61.25890976,11.81684146)(60.54015976,9.00434146)
\curveto(60.22765976,7.78559146)(59.50890976,6.94184146)(58.63390976,6.94184146)
\curveto(58.50890976,6.94184146)(58.07140976,6.94184146)(57.63390976,7.19184146)
\curveto(58.13390976,7.31684146)(58.57140976,7.72309146)(58.57140976,8.28559146)
\curveto(58.57140976,8.81684146)(58.13390976,8.97309146)(57.85265976,8.97309146)
\curveto(57.22765976,8.97309146)(56.75890976,8.47309146)(56.75890976,7.81684146)
\curveto(56.75890976,6.91059146)(57.72765976,6.50434146)(58.60265976,6.50434146)
\curveto(59.94640976,6.50434146)(60.66515976,7.91059146)(60.69640976,8.00434146)
\curveto(60.94640976,7.28559146)(61.66515976,6.50434146)(62.85265976,6.50434146)
\curveto(64.91515976,6.50434146)(66.04015976,9.06684146)(66.04015976,9.56684146)
\curveto(66.04015976,9.78559146)(65.88390976,9.78559146)(65.82140976,9.78559146)
\curveto(65.63390976,9.78559146)(65.60265976,9.69184146)(65.54015976,9.56684146)
\curveto(64.88390976,7.41059146)(63.54015976,6.94184146)(62.91515976,6.94184146)
\curveto(62.13390976,6.94184146)(61.82140976,7.56684146)(61.82140976,8.25434146)
\curveto(61.82140976,8.69184146)(61.91515976,9.12934146)(62.13390976,10.00434146)
\closepath
\moveto(62.82140976,12.75434146)
}
}
{
\newrgbcolor{curcolor}{0 0 0}
\pscustom[linestyle=none,fillstyle=solid,fillcolor=curcolor]
{
\newpath
\moveto(74.74066024,8.17258487)
\curveto(74.74066024,9.70383487)(74.55316024,10.82883487)(73.92816024,11.79758487)
\curveto(73.49066024,12.42258487)(72.61566024,12.98508487)(71.52191024,12.98508487)
\curveto(68.27191024,12.98508487)(68.27191024,9.17258487)(68.27191024,8.17258487)
\curveto(68.27191024,7.17258487)(68.27191024,3.45383487)(71.52191024,3.45383487)
\curveto(74.74066024,3.45383487)(74.74066024,7.17258487)(74.74066024,8.17258487)
\closepath
\moveto(71.52191024,3.86008487)
\curveto(70.86566024,3.86008487)(70.02191024,4.23508487)(69.74066024,5.36008487)
\curveto(69.55316024,6.17258487)(69.55316024,7.32883487)(69.55316024,8.36008487)
\curveto(69.55316024,9.39133487)(69.55316024,10.45383487)(69.74066024,11.20383487)
\curveto(70.05316024,12.29758487)(70.92816024,12.61008487)(71.52191024,12.61008487)
\curveto(72.27191024,12.61008487)(72.99066024,12.14133487)(73.24066024,11.32883487)
\curveto(73.45941024,10.57883487)(73.49066024,9.57883487)(73.49066024,8.36008487)
\curveto(73.49066024,7.32883487)(73.49066024,6.29758487)(73.30316024,5.42258487)
\curveto(73.02191024,4.14133487)(72.08441024,3.86008487)(71.52191024,3.86008487)
\closepath
\moveto(71.52191024,3.86008487)
}
}
{
\newrgbcolor{curcolor}{0 0 0}
\pscustom[linestyle=none,fillstyle=solid,fillcolor=curcolor]
{
\newpath
\moveto(237.5895741,13.28483146)
\curveto(237.7145741,13.78483146)(238.1833241,15.62858146)(239.5583241,15.62858146)
\curveto(239.6520741,15.62858146)(240.1520741,15.62858146)(240.5583241,15.37858146)
\curveto(239.9958241,15.25358146)(239.6208241,14.78483146)(239.6208241,14.28483146)
\curveto(239.6208241,13.97233146)(239.8395741,13.59733146)(240.3708241,13.59733146)
\curveto(240.8083241,13.59733146)(241.4333241,13.94108146)(241.4333241,14.75358146)
\curveto(241.4333241,15.78483146)(240.2770741,16.06608146)(239.5895741,16.06608146)
\curveto(238.4333241,16.06608146)(237.7458241,15.00358146)(237.4958241,14.56608146)
\curveto(236.9958241,15.87858146)(235.9333241,16.06608146)(235.3395741,16.06608146)
\curveto(233.2770741,16.06608146)(232.1208241,13.50358146)(232.1208241,13.00358146)
\curveto(232.1208241,12.78483146)(232.3395741,12.78483146)(232.3708241,12.78483146)
\curveto(232.5270741,12.78483146)(232.5895741,12.84733146)(232.6208241,13.00358146)
\curveto(233.3083241,15.12858146)(234.6208241,15.62858146)(235.3083241,15.62858146)
\curveto(235.6833241,15.62858146)(236.3708241,15.44108146)(236.3708241,14.28483146)
\curveto(236.3708241,13.65983146)(236.0270741,12.34733146)(235.3083241,9.53483146)
\curveto(234.9958241,8.31608146)(234.2770741,7.47233146)(233.4020741,7.47233146)
\curveto(233.2770741,7.47233146)(232.8395741,7.47233146)(232.4020741,7.72233146)
\curveto(232.9020741,7.84733146)(233.3395741,8.25358146)(233.3395741,8.81608146)
\curveto(233.3395741,9.34733146)(232.9020741,9.50358146)(232.6208241,9.50358146)
\curveto(231.9958241,9.50358146)(231.5270741,9.00358146)(231.5270741,8.34733146)
\curveto(231.5270741,7.44108146)(232.4958241,7.03483146)(233.3708241,7.03483146)
\curveto(234.7145741,7.03483146)(235.4333241,8.44108146)(235.4645741,8.53483146)
\curveto(235.7145741,7.81608146)(236.4333241,7.03483146)(237.6208241,7.03483146)
\curveto(239.6833241,7.03483146)(240.8083241,9.59733146)(240.8083241,10.09733146)
\curveto(240.8083241,10.31608146)(240.6520741,10.31608146)(240.5895741,10.31608146)
\curveto(240.4020741,10.31608146)(240.3708241,10.22233146)(240.3083241,10.09733146)
\curveto(239.6520741,7.94108146)(238.3083241,7.47233146)(237.6833241,7.47233146)
\curveto(236.9020741,7.47233146)(236.5895741,8.09733146)(236.5895741,8.78483146)
\curveto(236.5895741,9.22233146)(236.6833241,9.65983146)(236.9020741,10.53483146)
\closepath
\moveto(237.5895741,13.28483146)
}
}
{
\newrgbcolor{curcolor}{0 0 0}
\pscustom[linestyle=none,fillstyle=solid,fillcolor=curcolor]
{
\newpath
\moveto(261.54482008,13.78483146)
\curveto(261.85732008,13.78483146)(262.23232008,13.78483146)(262.23232008,14.15983146)
\curveto(262.23232008,14.56608146)(261.85732008,14.56608146)(261.57607008,14.56608146)
\lineto(249.63857008,14.56608146)
\curveto(249.35732008,14.56608146)(248.98232008,14.56608146)(248.98232008,14.15983146)
\curveto(248.98232008,13.78483146)(249.35732008,13.78483146)(249.63857008,13.78483146)
\closepath
\moveto(261.57607008,9.90983146)
\curveto(261.85732008,9.90983146)(262.23232008,9.90983146)(262.23232008,10.31608146)
\curveto(262.23232008,10.69108146)(261.85732008,10.69108146)(261.54482008,10.69108146)
\lineto(249.63857008,10.69108146)
\curveto(249.35732008,10.69108146)(248.98232008,10.69108146)(248.98232008,10.31608146)
\curveto(248.98232008,9.90983146)(249.35732008,9.90983146)(249.63857008,9.90983146)
\closepath
\moveto(261.57607008,9.90983146)
}
}
{
\newrgbcolor{curcolor}{0 0 0}
\pscustom[linestyle=none,fillstyle=solid,fillcolor=curcolor]
{
\newpath
\moveto(275.54556776,13.28483146)
\curveto(275.67056776,13.78483146)(276.13931776,15.62858146)(277.51431776,15.62858146)
\curveto(277.60806776,15.62858146)(278.10806776,15.62858146)(278.51431776,15.37858146)
\curveto(277.95181776,15.25358146)(277.57681776,14.78483146)(277.57681776,14.28483146)
\curveto(277.57681776,13.97233146)(277.79556776,13.59733146)(278.32681776,13.59733146)
\curveto(278.76431776,13.59733146)(279.38931776,13.94108146)(279.38931776,14.75358146)
\curveto(279.38931776,15.78483146)(278.23306776,16.06608146)(277.54556776,16.06608146)
\curveto(276.38931776,16.06608146)(275.70181776,15.00358146)(275.45181776,14.56608146)
\curveto(274.95181776,15.87858146)(273.88931776,16.06608146)(273.29556776,16.06608146)
\curveto(271.23306776,16.06608146)(270.07681776,13.50358146)(270.07681776,13.00358146)
\curveto(270.07681776,12.78483146)(270.29556776,12.78483146)(270.32681776,12.78483146)
\curveto(270.48306776,12.78483146)(270.54556776,12.84733146)(270.57681776,13.00358146)
\curveto(271.26431776,15.12858146)(272.57681776,15.62858146)(273.26431776,15.62858146)
\curveto(273.63931776,15.62858146)(274.32681776,15.44108146)(274.32681776,14.28483146)
\curveto(274.32681776,13.65983146)(273.98306776,12.34733146)(273.26431776,9.53483146)
\curveto(272.95181776,8.31608146)(272.23306776,7.47233146)(271.35806776,7.47233146)
\curveto(271.23306776,7.47233146)(270.79556776,7.47233146)(270.35806776,7.72233146)
\curveto(270.85806776,7.84733146)(271.29556776,8.25358146)(271.29556776,8.81608146)
\curveto(271.29556776,9.34733146)(270.85806776,9.50358146)(270.57681776,9.50358146)
\curveto(269.95181776,9.50358146)(269.48306776,9.00358146)(269.48306776,8.34733146)
\curveto(269.48306776,7.44108146)(270.45181776,7.03483146)(271.32681776,7.03483146)
\curveto(272.67056776,7.03483146)(273.38931776,8.44108146)(273.42056776,8.53483146)
\curveto(273.67056776,7.81608146)(274.38931776,7.03483146)(275.57681776,7.03483146)
\curveto(277.63931776,7.03483146)(278.76431776,9.59733146)(278.76431776,10.09733146)
\curveto(278.76431776,10.31608146)(278.60806776,10.31608146)(278.54556776,10.31608146)
\curveto(278.35806776,10.31608146)(278.32681776,10.22233146)(278.26431776,10.09733146)
\curveto(277.60806776,7.94108146)(276.26431776,7.47233146)(275.63931776,7.47233146)
\curveto(274.85806776,7.47233146)(274.54556776,8.09733146)(274.54556776,8.78483146)
\curveto(274.54556776,9.22233146)(274.63931776,9.65983146)(274.85806776,10.53483146)
\closepath
\moveto(275.54556776,13.28483146)
}
}
{
\newrgbcolor{curcolor}{0 0 0}
\pscustom[linestyle=none,fillstyle=solid,fillcolor=curcolor]
{
\newpath
\moveto(284.93356824,13.14057487)
\curveto(284.93356824,13.51557487)(284.93356824,13.51557487)(284.52731824,13.51557487)
\curveto(283.62106824,12.64057487)(282.37106824,12.64057487)(281.80856824,12.64057487)
\lineto(281.80856824,12.14057487)
\curveto(282.12106824,12.14057487)(283.05856824,12.14057487)(283.80856824,12.51557487)
\lineto(283.80856824,5.42182487)
\curveto(283.80856824,4.95307487)(283.80856824,4.76557487)(282.43356824,4.76557487)
\lineto(281.90231824,4.76557487)
\lineto(281.90231824,4.26557487)
\curveto(282.15231824,4.26557487)(283.87106824,4.32807487)(284.37106824,4.32807487)
\curveto(284.80856824,4.32807487)(286.55856824,4.26557487)(286.87106824,4.26557487)
\lineto(286.87106824,4.76557487)
\lineto(286.33981824,4.76557487)
\curveto(284.93356824,4.76557487)(284.93356824,4.95307487)(284.93356824,5.42182487)
\closepath
\moveto(284.93356824,13.14057487)
}
}
{
\newrgbcolor{curcolor}{0 0 0}
\pscustom[linestyle=none,fillstyle=solid,fillcolor=curcolor]
{
\newpath
\moveto(55.7234261,64.02687446)
\lineto(57.5984261,64.02687446)
\curveto(58.0046761,64.02687446)(58.2234261,64.02687446)(58.2234261,64.43312446)
\curveto(58.2234261,64.65187446)(58.0046761,64.65187446)(57.6609261,64.65187446)
\lineto(55.9109261,64.65187446)
\curveto(56.6296761,67.49562446)(56.7234261,67.87062446)(56.7234261,67.99562446)
\curveto(56.7234261,68.33937446)(56.4734261,68.52687446)(56.1296761,68.52687446)
\curveto(56.0671761,68.52687446)(55.5046761,68.52687446)(55.3484261,67.80812446)
\lineto(54.5671761,64.65187446)
\lineto(52.6921761,64.65187446)
\curveto(52.2859261,64.65187446)(52.0984261,64.65187446)(52.0984261,64.27687446)
\curveto(52.0984261,64.02687446)(52.2546761,64.02687446)(52.6609261,64.02687446)
\lineto(54.4109261,64.02687446)
\curveto(52.9734261,58.37062446)(52.8796761,58.02687446)(52.8796761,57.68312446)
\curveto(52.8796761,56.58937446)(53.6296761,55.83937446)(54.7234261,55.83937446)
\curveto(56.7546761,55.83937446)(57.8796761,58.74562446)(57.8796761,58.90187446)
\curveto(57.8796761,59.12062446)(57.7234261,59.12062446)(57.6609261,59.12062446)
\curveto(57.4734261,59.12062446)(57.4421761,59.05812446)(57.3484261,58.83937446)
\curveto(56.5046761,56.74562446)(55.4421761,56.27687446)(54.7546761,56.27687446)
\curveto(54.3484261,56.27687446)(54.1296761,56.52687446)(54.1296761,57.18312446)
\curveto(54.1296761,57.68312446)(54.1921761,57.80812446)(54.2546761,58.15187446)
\closepath
\moveto(55.7234261,64.02687446)
}
}
{
\newrgbcolor{curcolor}{0 0 0}
\pscustom[linestyle=none,fillstyle=solid,fillcolor=curcolor]
{
\newpath
\moveto(66.01317525,57.28886787)
\curveto(66.01317525,58.50761787)(65.41942525,59.22636787)(63.91942525,59.22636787)
\curveto(62.76317525,59.22636787)(62.04442525,58.60136787)(61.63817525,57.85136787)
\lineto(61.63817525,59.22636787)
\lineto(59.57567525,59.07011787)
\lineto(59.57567525,58.57011787)
\curveto(60.51317525,58.57011787)(60.63817525,58.47636787)(60.63817525,57.78886787)
\lineto(60.63817525,54.16386787)
\curveto(60.63817525,53.57011787)(60.48192525,53.57011787)(59.57567525,53.57011787)
\lineto(59.57567525,53.07011787)
\curveto(59.60692525,53.07011787)(60.57567525,53.13261787)(61.16942525,53.13261787)
\curveto(61.66942525,53.13261787)(62.63817525,53.07011787)(62.76317525,53.07011787)
\lineto(62.76317525,53.57011787)
\curveto(61.85692525,53.57011787)(61.73192525,53.57011787)(61.73192525,54.16386787)
\lineto(61.73192525,56.69511787)
\curveto(61.73192525,58.13261787)(62.88817525,58.82011787)(63.79442525,58.82011787)
\curveto(64.76317525,58.82011787)(64.88817525,58.07011787)(64.88817525,57.35136787)
\lineto(64.88817525,54.16386787)
\curveto(64.88817525,53.57011787)(64.76317525,53.57011787)(63.85692525,53.57011787)
\lineto(63.85692525,53.07011787)
\curveto(63.88817525,53.07011787)(64.85692525,53.13261787)(65.45067525,53.13261787)
\curveto(65.95067525,53.13261787)(66.91942525,53.07011787)(67.04442525,53.07011787)
\lineto(67.04442525,53.57011787)
\curveto(66.13817525,53.57011787)(66.01317525,53.57011787)(66.01317525,54.16386787)
\closepath
\moveto(66.01317525,57.28886787)
}
}
{
\newrgbcolor{curcolor}{0 0 0}
\pscustom[linestyle=none,fillstyle=solid,fillcolor=curcolor]
{
\newpath
\moveto(73.75160909,56.25761787)
\curveto(74.06410909,56.25761787)(74.15785909,56.25761787)(74.15785909,56.53886787)
\curveto(74.15785909,57.78886787)(73.47035909,59.28886787)(71.37660909,59.28886787)
\curveto(69.56410909,59.28886787)(68.15785909,57.85136787)(68.15785909,56.13261787)
\curveto(68.15785909,54.35136787)(69.72035909,52.94511787)(71.56410909,52.94511787)
\curveto(73.43910909,52.94511787)(74.15785909,54.44511787)(74.15785909,54.75761787)
\curveto(74.15785909,54.78886787)(74.12660909,54.94511787)(73.90785909,54.94511787)
\curveto(73.72035909,54.94511787)(73.68910909,54.85136787)(73.65785909,54.72636787)
\curveto(73.22035909,53.60136787)(72.15785909,53.38261787)(71.65785909,53.38261787)
\curveto(71.00160909,53.38261787)(70.37660909,53.66386787)(69.97035909,54.19511787)
\curveto(69.43910909,54.85136787)(69.43910909,55.69511787)(69.43910909,56.25761787)
\closepath
\moveto(69.43910909,56.60136787)
\curveto(69.59535909,58.57011787)(70.84535909,58.88261787)(71.37660909,58.88261787)
\curveto(73.09535909,58.88261787)(73.15785909,56.94511787)(73.18910909,56.60136787)
\closepath
\moveto(69.43910909,56.60136787)
}
}
{
\newrgbcolor{curcolor}{0 0 0}
\pscustom[linestyle=none,fillstyle=solid,fillcolor=curcolor]
{
\newpath
\moveto(80.92934346,56.82011787)
\curveto(80.92934346,57.53886787)(80.92934346,58.07011787)(80.27309346,58.60136787)
\curveto(79.71059346,59.07011787)(79.05434346,59.28886787)(78.21059346,59.28886787)
\curveto(76.89809346,59.28886787)(75.96059346,58.78886787)(75.96059346,57.94511787)
\curveto(75.96059346,57.47636787)(76.27309346,57.25761787)(76.64809346,57.25761787)
\curveto(77.02309346,57.25761787)(77.30434346,57.53886787)(77.30434346,57.91386787)
\curveto(77.30434346,58.13261787)(77.17934346,58.44511787)(76.80434346,58.53886787)
\curveto(77.30434346,58.88261787)(78.11684346,58.88261787)(78.17934346,58.88261787)
\curveto(78.96059346,58.88261787)(79.83559346,58.38261787)(79.83559346,57.19511787)
\lineto(79.83559346,56.78886787)
\curveto(79.05434346,56.75761787)(78.14809346,56.69511787)(77.11684346,56.32011787)
\curveto(75.86684346,55.88261787)(75.49184346,55.10136787)(75.49184346,54.47636787)
\curveto(75.49184346,53.28886787)(76.92934346,52.94511787)(77.92934346,52.94511787)
\curveto(79.02309346,52.94511787)(79.67934346,53.57011787)(79.99184346,54.10136787)
\curveto(80.02309346,53.53886787)(80.39809346,53.00761787)(81.05434346,53.00761787)
\curveto(81.08559346,53.00761787)(82.42934346,53.00761787)(82.42934346,54.32011787)
\lineto(82.42934346,55.10136787)
\lineto(81.96059346,55.10136787)
\lineto(81.96059346,54.35136787)
\curveto(81.96059346,54.19511787)(81.96059346,53.53886787)(81.42934346,53.53886787)
\curveto(80.92934346,53.53886787)(80.92934346,54.19511787)(80.92934346,54.35136787)
\closepath
\moveto(79.83559346,55.03886787)
\curveto(79.83559346,53.69511787)(78.64809346,53.32011787)(78.02309346,53.32011787)
\curveto(77.30434346,53.32011787)(76.64809346,53.78886787)(76.64809346,54.47636787)
\curveto(76.64809346,55.25761787)(77.30434346,56.32011787)(79.83559346,56.41386787)
\closepath
\moveto(79.83559346,55.03886787)
}
}
{
\newrgbcolor{curcolor}{0 0 0}
\pscustom[linestyle=none,fillstyle=solid,fillcolor=curcolor]
{
\newpath
\moveto(85.43500447,56.25761787)
\curveto(85.43500447,57.41386787)(85.96625447,58.82011787)(87.31000447,58.82011787)
\curveto(87.18500447,58.72636787)(87.06000447,58.53886787)(87.06000447,58.32011787)
\curveto(87.06000447,57.85136787)(87.43500447,57.69511787)(87.68500447,57.69511787)
\curveto(88.02875447,57.69511787)(88.34125447,57.91386787)(88.34125447,58.32011787)
\curveto(88.34125447,58.78886787)(87.87250447,59.22636787)(87.21625447,59.22636787)
\curveto(86.52875447,59.22636787)(85.74750447,58.78886787)(85.34125447,57.72636787)
\lineto(85.34125447,59.22636787)
\lineto(83.34125447,59.07011787)
\lineto(83.34125447,58.57011787)
\curveto(84.27875447,58.57011787)(84.37250447,58.47636787)(84.37250447,57.78886787)
\lineto(84.37250447,54.16386787)
\curveto(84.37250447,53.57011787)(84.24750447,53.57011787)(83.34125447,53.57011787)
\lineto(83.34125447,53.07011787)
\curveto(83.40375447,53.07011787)(84.34125447,53.13261787)(84.93500447,53.13261787)
\curveto(85.52875447,53.13261787)(86.12250447,53.10136787)(86.74750447,53.07011787)
\lineto(86.74750447,53.57011787)
\lineto(86.46625447,53.57011787)
\curveto(85.43500447,53.57011787)(85.43500447,53.72636787)(85.43500447,54.19511787)
\closepath
\moveto(85.43500447,56.25761787)
}
}
{
\newrgbcolor{curcolor}{0 0 0}
\pscustom[linestyle=none,fillstyle=solid,fillcolor=curcolor]
{
\newpath
\moveto(267.3866091,208.31857046)
\lineto(269.2616091,208.31857046)
\curveto(269.6678591,208.31857046)(269.8866091,208.31857046)(269.8866091,208.72482046)
\curveto(269.8866091,208.94357046)(269.6678591,208.94357046)(269.3241091,208.94357046)
\lineto(267.5741091,208.94357046)
\curveto(268.2928591,211.78732046)(268.3866091,212.16232046)(268.3866091,212.28732046)
\curveto(268.3866091,212.63107046)(268.1366091,212.81857046)(267.7928591,212.81857046)
\curveto(267.7303591,212.81857046)(267.1678591,212.81857046)(267.0116091,212.09982046)
\lineto(266.2303591,208.94357046)
\lineto(264.3553591,208.94357046)
\curveto(263.9491091,208.94357046)(263.7616091,208.94357046)(263.7616091,208.56857046)
\curveto(263.7616091,208.31857046)(263.9178591,208.31857046)(264.3241091,208.31857046)
\lineto(266.0741091,208.31857046)
\curveto(264.6366091,202.66232046)(264.5428591,202.31857046)(264.5428591,201.97482046)
\curveto(264.5428591,200.88107046)(265.2928591,200.13107046)(266.3866091,200.13107046)
\curveto(268.4178591,200.13107046)(269.5428591,203.03732046)(269.5428591,203.19357046)
\curveto(269.5428591,203.41232046)(269.3866091,203.41232046)(269.3241091,203.41232046)
\curveto(269.1366091,203.41232046)(269.1053591,203.34982046)(269.0116091,203.13107046)
\curveto(268.1678591,201.03732046)(267.1053591,200.56857046)(266.4178591,200.56857046)
\curveto(266.0116091,200.56857046)(265.7928591,200.81857046)(265.7928591,201.47482046)
\curveto(265.7928591,201.97482046)(265.8553591,202.09982046)(265.9178591,202.44357046)
\closepath
\moveto(267.3866091,208.31857046)
}
}
{
\newrgbcolor{curcolor}{0 0 0}
\pscustom[linestyle=none,fillstyle=solid,fillcolor=curcolor]
{
\newpath
\moveto(273.36385825,202.86181387)
\lineto(275.11385825,202.86181387)
\lineto(275.11385825,203.36181387)
\lineto(273.30135825,203.36181387)
\lineto(273.30135825,204.92431387)
\curveto(273.30135825,206.20556387)(274.05135825,206.76806387)(274.64510825,206.76806387)
\curveto(274.77010825,206.76806387)(274.92635825,206.76806387)(275.05135825,206.70556387)
\curveto(274.86385825,206.61181387)(274.73885825,206.39306387)(274.73885825,206.14306387)
\curveto(274.73885825,205.76806387)(275.02010825,205.51806387)(275.39510825,205.51806387)
\curveto(275.77010825,205.51806387)(276.05135825,205.76806387)(276.05135825,206.14306387)
\curveto(276.05135825,206.76806387)(275.42635825,207.17431387)(274.67635825,207.17431387)
\curveto(273.58260825,207.17431387)(272.30135825,206.39306387)(272.30135825,204.92431387)
\lineto(272.30135825,203.36181387)
\lineto(271.11385825,203.36181387)
\lineto(271.11385825,202.86181387)
\lineto(272.30135825,202.86181387)
\lineto(272.30135825,198.45556387)
\curveto(272.30135825,197.86181387)(272.17635825,197.86181387)(271.27010825,197.86181387)
\lineto(271.27010825,197.36181387)
\curveto(271.33260825,197.36181387)(272.30135825,197.42431387)(272.86385825,197.42431387)
\curveto(273.45760825,197.42431387)(274.08260825,197.39306387)(274.67635825,197.36181387)
\lineto(274.67635825,197.86181387)
\lineto(274.39510825,197.86181387)
\curveto(273.36385825,197.86181387)(273.36385825,198.01806387)(273.36385825,198.48681387)
\closepath
\moveto(273.36385825,202.86181387)
}
}
{
\newrgbcolor{curcolor}{0 0 0}
\pscustom[linestyle=none,fillstyle=solid,fillcolor=curcolor]
{
\newpath
\moveto(281.64783957,201.11181387)
\curveto(281.64783957,201.83056387)(281.64783957,202.36181387)(280.99158957,202.89306387)
\curveto(280.42908957,203.36181387)(279.77283957,203.58056387)(278.92908957,203.58056387)
\curveto(277.61658957,203.58056387)(276.67908957,203.08056387)(276.67908957,202.23681387)
\curveto(276.67908957,201.76806387)(276.99158957,201.54931387)(277.36658957,201.54931387)
\curveto(277.74158957,201.54931387)(278.02283957,201.83056387)(278.02283957,202.20556387)
\curveto(278.02283957,202.42431387)(277.89783957,202.73681387)(277.52283957,202.83056387)
\curveto(278.02283957,203.17431387)(278.83533957,203.17431387)(278.89783957,203.17431387)
\curveto(279.67908957,203.17431387)(280.55408957,202.67431387)(280.55408957,201.48681387)
\lineto(280.55408957,201.08056387)
\curveto(279.77283957,201.04931387)(278.86658957,200.98681387)(277.83533957,200.61181387)
\curveto(276.58533957,200.17431387)(276.21033957,199.39306387)(276.21033957,198.76806387)
\curveto(276.21033957,197.58056387)(277.64783957,197.23681387)(278.64783957,197.23681387)
\curveto(279.74158957,197.23681387)(280.39783957,197.86181387)(280.71033957,198.39306387)
\curveto(280.74158957,197.83056387)(281.11658957,197.29931387)(281.77283957,197.29931387)
\curveto(281.80408957,197.29931387)(283.14783957,197.29931387)(283.14783957,198.61181387)
\lineto(283.14783957,199.39306387)
\lineto(282.67908957,199.39306387)
\lineto(282.67908957,198.64306387)
\curveto(282.67908957,198.48681387)(282.67908957,197.83056387)(282.14783957,197.83056387)
\curveto(281.64783957,197.83056387)(281.64783957,198.48681387)(281.64783957,198.64306387)
\closepath
\moveto(280.55408957,199.33056387)
\curveto(280.55408957,197.98681387)(279.36658957,197.61181387)(278.74158957,197.61181387)
\curveto(278.02283957,197.61181387)(277.36658957,198.08056387)(277.36658957,198.76806387)
\curveto(277.36658957,199.54931387)(278.02283957,200.61181387)(280.55408957,200.70556387)
\closepath
\moveto(280.55408957,199.33056387)
}
}
{
\newrgbcolor{curcolor}{0 0 0}
\pscustom[linestyle=none,fillstyle=solid,fillcolor=curcolor]
{
\newpath
\moveto(286.15350058,200.54931387)
\curveto(286.15350058,201.70556387)(286.68475058,203.11181387)(288.02850058,203.11181387)
\curveto(287.90350058,203.01806387)(287.77850058,202.83056387)(287.77850058,202.61181387)
\curveto(287.77850058,202.14306387)(288.15350058,201.98681387)(288.40350058,201.98681387)
\curveto(288.74725058,201.98681387)(289.05975058,202.20556387)(289.05975058,202.61181387)
\curveto(289.05975058,203.08056387)(288.59100058,203.51806387)(287.93475058,203.51806387)
\curveto(287.24725058,203.51806387)(286.46600058,203.08056387)(286.05975058,202.01806387)
\lineto(286.05975058,203.51806387)
\lineto(284.05975058,203.36181387)
\lineto(284.05975058,202.86181387)
\curveto(284.99725058,202.86181387)(285.09100058,202.76806387)(285.09100058,202.08056387)
\lineto(285.09100058,198.45556387)
\curveto(285.09100058,197.86181387)(284.96600058,197.86181387)(284.05975058,197.86181387)
\lineto(284.05975058,197.36181387)
\curveto(284.12225058,197.36181387)(285.05975058,197.42431387)(285.65350058,197.42431387)
\curveto(286.24725058,197.42431387)(286.84100058,197.39306387)(287.46600058,197.36181387)
\lineto(287.46600058,197.86181387)
\lineto(287.18475058,197.86181387)
\curveto(286.15350058,197.86181387)(286.15350058,198.01806387)(286.15350058,198.48681387)
\closepath
\moveto(286.15350058,200.54931387)
}
}
{
\newrgbcolor{curcolor}{0 0 0}
\pscustom[linestyle=none,fillstyle=solid,fillcolor=curcolor]
{
\newpath
\moveto(326.1788961,95.91765946)
\curveto(325.8976461,96.91765946)(324.8038961,97.41765946)(323.7101461,97.41765946)
\curveto(322.9913961,97.41765946)(322.4288961,97.01140946)(321.9913961,96.29265946)
\curveto(321.4913961,95.48015946)(321.2101461,94.41765946)(321.2101461,94.32390946)
\curveto(321.2101461,94.04265946)(321.5226461,94.04265946)(321.7101461,94.04265946)
\curveto(321.9288961,94.04265946)(321.9913961,94.04265946)(322.0851461,94.13640946)
\curveto(322.1476461,94.16765946)(322.1476461,94.23015946)(322.2726461,94.73015946)
\curveto(322.6476461,96.26140946)(323.0851461,96.69890946)(323.6163961,96.69890946)
\curveto(323.9288961,96.69890946)(324.0851461,96.51140946)(324.0851461,95.98015946)
\curveto(324.0851461,95.63640946)(323.9913961,95.32390946)(323.8038961,94.51140946)
\curveto(323.6476461,93.94890946)(323.4601461,93.16765946)(323.3663961,92.73015946)
\lineto(322.6476461,89.91765946)
\curveto(322.5851461,89.66765946)(322.4913961,89.26140946)(322.4913961,89.13640946)
\curveto(322.4913961,88.73015946)(322.8038961,88.26140946)(323.4288961,88.26140946)
\curveto(324.4601461,88.26140946)(324.6788961,89.16765946)(324.8351461,89.76140946)
\lineto(325.3976461,92.01140946)
\curveto(325.4601461,92.26140946)(325.9601461,94.32390946)(326.0226461,94.38640946)
\curveto(326.0226461,94.51140946)(326.5851461,95.38640946)(327.2413961,95.94890946)
\curveto(327.8038961,96.38640946)(328.4601461,96.69890946)(329.3038961,96.69890946)
\curveto(329.8038961,96.69890946)(330.3038961,96.51140946)(330.3038961,95.54265946)
\curveto(330.3038961,94.38640946)(329.4288961,92.07390946)(329.0538961,91.07390946)
\curveto(328.8038961,90.54265946)(328.7726461,90.38640946)(328.7726461,90.04265946)
\curveto(328.7726461,88.88640946)(329.9288961,88.26140946)(331.0226461,88.26140946)
\curveto(333.1163961,88.26140946)(334.1163961,90.98015946)(334.1163961,91.35515946)
\curveto(334.1163961,91.63640946)(333.8351461,91.63640946)(333.6476461,91.63640946)
\curveto(333.3976461,91.63640946)(333.2726461,91.63640946)(333.1788961,91.38640946)
\curveto(332.5226461,89.16765946)(331.4601461,88.98015946)(331.1476461,88.98015946)
\curveto(330.9913961,88.98015946)(330.8038961,88.98015946)(330.8038961,89.38640946)
\curveto(330.8038961,89.85515946)(330.9913961,90.35515946)(331.1788961,90.85515946)
\curveto(331.5226461,91.66765946)(332.4288961,94.01140946)(332.4288961,95.10515946)
\curveto(332.4288961,96.94890946)(330.8976461,97.41765946)(329.4601461,97.41765946)
\curveto(329.0538961,97.41765946)(327.5851461,97.41765946)(326.1788961,95.91765946)
\closepath
\moveto(326.1788961,95.91765946)
}
}
{
\newrgbcolor{curcolor}{0 0 0}
\pscustom[linestyle=none,fillstyle=solid,fillcolor=curcolor]
{
\newpath
\moveto(354.01463524,94.94890946)
\curveto(354.32713524,94.94890946)(354.70213524,94.94890946)(354.70213524,95.32390946)
\curveto(354.70213524,95.73015946)(354.32713524,95.73015946)(354.04588524,95.73015946)
\lineto(342.10838524,95.73015946)
\curveto(341.82713524,95.73015946)(341.45213524,95.73015946)(341.45213524,95.32390946)
\curveto(341.45213524,94.94890946)(341.82713524,94.94890946)(342.10838524,94.94890946)
\closepath
\moveto(354.04588524,91.07390946)
\curveto(354.32713524,91.07390946)(354.70213524,91.07390946)(354.70213524,91.48015946)
\curveto(354.70213524,91.85515946)(354.32713524,91.85515946)(354.01463524,91.85515946)
\lineto(342.10838524,91.85515946)
\curveto(341.82713524,91.85515946)(341.45213524,91.85515946)(341.45213524,91.48015946)
\curveto(341.45213524,91.07390946)(341.82713524,91.07390946)(342.10838524,91.07390946)
\closepath
\moveto(354.04588524,91.07390946)
}
}
{
\newrgbcolor{curcolor}{0 0 0}
\pscustom[linestyle=none,fillstyle=solid,fillcolor=curcolor]
{
\newpath
\moveto(367.95790306,83.63640946)
\curveto(367.95790306,83.69890946)(367.95790306,83.73015946)(367.61415306,84.07390946)
\curveto(365.14540306,86.57390946)(364.48915306,90.35515946)(364.48915306,93.41765946)
\curveto(364.48915306,96.88640946)(365.23915306,100.35515946)(367.70790306,102.82390946)
\curveto(367.95790306,103.07390946)(367.95790306,103.10515946)(367.95790306,103.16765946)
\curveto(367.95790306,103.32390946)(367.89540306,103.38640946)(367.77040306,103.38640946)
\curveto(367.55165306,103.38640946)(365.77040306,102.01140946)(364.58290306,99.48015946)
\curveto(363.58290306,97.29265946)(363.33290306,95.07390946)(363.33290306,93.41765946)
\curveto(363.33290306,91.85515946)(363.55165306,89.44890946)(364.64540306,87.16765946)
\curveto(365.86415306,84.73015946)(367.55165306,83.41765946)(367.77040306,83.41765946)
\curveto(367.89540306,83.41765946)(367.95790306,83.48015946)(367.95790306,83.63640946)
\closepath
\moveto(367.95790306,83.63640946)
}
}
{
\newrgbcolor{curcolor}{0 0 0}
\pscustom[linestyle=none,fillstyle=solid,fillcolor=curcolor]
{
\newpath
\moveto(374.98805443,101.16765946)
\curveto(374.98805443,101.66765946)(374.98805443,101.69890946)(374.51930443,101.69890946)
\curveto(373.26930443,100.41765946)(371.51930443,100.41765946)(370.89430443,100.41765946)
\lineto(370.89430443,99.79265946)
\curveto(371.30055443,99.79265946)(372.45680443,99.79265946)(373.48805443,100.32390946)
\lineto(373.48805443,89.98015946)
\curveto(373.48805443,89.26140946)(373.42555443,89.04265946)(371.64430443,89.04265946)
\lineto(371.01930443,89.04265946)
\lineto(371.01930443,88.41765946)
\curveto(371.70680443,88.48015946)(373.42555443,88.48015946)(374.23805443,88.48015946)
\curveto(375.01930443,88.48015946)(376.76930443,88.48015946)(377.45680443,88.41765946)
\lineto(377.45680443,89.04265946)
\lineto(376.83180443,89.04265946)
\curveto(375.01930443,89.04265946)(374.98805443,89.26140946)(374.98805443,89.98015946)
\closepath
\moveto(374.98805443,101.16765946)
}
}
{
\newrgbcolor{curcolor}{0 0 0}
\pscustom[linestyle=none,fillstyle=solid,fillcolor=curcolor]
{
\newpath
\moveto(383.13163963,88.44890946)
\curveto(383.13163963,89.76140946)(382.63163963,90.54265946)(381.85038963,90.54265946)
\curveto(381.19413963,90.54265946)(380.78788963,90.04265946)(380.78788963,89.48015946)
\curveto(380.78788963,88.94890946)(381.19413963,88.41765946)(381.85038963,88.41765946)
\curveto(382.06913963,88.41765946)(382.35038963,88.51140946)(382.53788963,88.66765946)
\curveto(382.60038963,88.73015946)(382.63163963,88.73015946)(382.63163963,88.73015946)
\curveto(382.66288963,88.73015946)(382.66288963,88.73015946)(382.66288963,88.44890946)
\curveto(382.66288963,86.94890946)(381.97538963,85.76140946)(381.31913963,85.10515946)
\curveto(381.10038963,84.88640946)(381.10038963,84.85515946)(381.10038963,84.79265946)
\curveto(381.10038963,84.63640946)(381.19413963,84.57390946)(381.28788963,84.57390946)
\curveto(381.50663963,84.57390946)(383.13163963,86.10515946)(383.13163963,88.44890946)
\closepath
\moveto(383.13163963,88.44890946)
}
}
{
\newrgbcolor{curcolor}{0 0 0}
\pscustom[linestyle=none,fillstyle=solid,fillcolor=curcolor]
{
\newpath
\moveto(397.08139244,94.79265946)
\curveto(397.08139244,96.38640946)(396.98764244,97.98015946)(396.30014244,99.44890946)
\curveto(395.39389244,101.38640946)(393.73764244,101.69890946)(392.92514244,101.69890946)
\curveto(391.70639244,101.69890946)(390.26889244,101.16765946)(389.42514244,99.32390946)
\curveto(388.80014244,97.94890946)(388.70639244,96.38640946)(388.70639244,94.79265946)
\curveto(388.70639244,93.29265946)(388.76889244,91.51140946)(389.61264244,89.98015946)
\curveto(390.45639244,88.38640946)(391.92514244,87.98015946)(392.89389244,87.98015946)
\curveto(393.95639244,87.98015946)(395.48764244,88.38640946)(396.36264244,90.29265946)
\curveto(396.98764244,91.66765946)(397.08139244,93.23015946)(397.08139244,94.79265946)
\closepath
\moveto(392.89389244,88.41765946)
\curveto(392.11264244,88.41765946)(390.92514244,88.91765946)(390.58139244,90.82390946)
\curveto(390.36264244,92.01140946)(390.36264244,93.85515946)(390.36264244,95.04265946)
\curveto(390.36264244,96.32390946)(390.36264244,97.63640946)(390.51889244,98.69890946)
\curveto(390.89389244,101.07390946)(392.39389244,101.26140946)(392.89389244,101.26140946)
\curveto(393.55014244,101.26140946)(394.86264244,100.88640946)(395.23764244,98.91765946)
\curveto(395.45639244,97.79265946)(395.45639244,96.29265946)(395.45639244,95.04265946)
\curveto(395.45639244,93.54265946)(395.45639244,92.19890946)(395.23764244,90.91765946)
\curveto(394.92514244,89.01140946)(393.80014244,88.41765946)(392.89389244,88.41765946)
\closepath
\moveto(392.89389244,88.41765946)
}
}
{
\newrgbcolor{curcolor}{0 0 0}
\pscustom[linestyle=none,fillstyle=solid,fillcolor=curcolor]
{
\newpath
\moveto(401.94964806,88.44890946)
\curveto(401.94964806,89.76140946)(401.44964806,90.54265946)(400.66839806,90.54265946)
\curveto(400.01214806,90.54265946)(399.60589806,90.04265946)(399.60589806,89.48015946)
\curveto(399.60589806,88.94890946)(400.01214806,88.41765946)(400.66839806,88.41765946)
\curveto(400.88714806,88.41765946)(401.16839806,88.51140946)(401.35589806,88.66765946)
\curveto(401.41839806,88.73015946)(401.44964806,88.73015946)(401.44964806,88.73015946)
\curveto(401.48089806,88.73015946)(401.48089806,88.73015946)(401.48089806,88.44890946)
\curveto(401.48089806,86.94890946)(400.79339806,85.76140946)(400.13714806,85.10515946)
\curveto(399.91839806,84.88640946)(399.91839806,84.85515946)(399.91839806,84.79265946)
\curveto(399.91839806,84.63640946)(400.01214806,84.57390946)(400.10589806,84.57390946)
\curveto(400.32464806,84.57390946)(401.94964806,86.10515946)(401.94964806,88.44890946)
\closepath
\moveto(401.94964806,88.44890946)
}
}
{
\newrgbcolor{curcolor}{0 0 0}
\pscustom[linestyle=none,fillstyle=solid,fillcolor=curcolor]
{
\newpath
\moveto(415.89938561,94.79265946)
\curveto(415.89938561,96.38640946)(415.80563561,97.98015946)(415.11813561,99.44890946)
\curveto(414.21188561,101.38640946)(412.55563561,101.69890946)(411.74313561,101.69890946)
\curveto(410.52438561,101.69890946)(409.08688561,101.16765946)(408.24313561,99.32390946)
\curveto(407.61813561,97.94890946)(407.52438561,96.38640946)(407.52438561,94.79265946)
\curveto(407.52438561,93.29265946)(407.58688561,91.51140946)(408.43063561,89.98015946)
\curveto(409.27438561,88.38640946)(410.74313561,87.98015946)(411.71188561,87.98015946)
\curveto(412.77438561,87.98015946)(414.30563561,88.38640946)(415.18063561,90.29265946)
\curveto(415.80563561,91.66765946)(415.89938561,93.23015946)(415.89938561,94.79265946)
\closepath
\moveto(411.71188561,88.41765946)
\curveto(410.93063561,88.41765946)(409.74313561,88.91765946)(409.39938561,90.82390946)
\curveto(409.18063561,92.01140946)(409.18063561,93.85515946)(409.18063561,95.04265946)
\curveto(409.18063561,96.32390946)(409.18063561,97.63640946)(409.33688561,98.69890946)
\curveto(409.71188561,101.07390946)(411.21188561,101.26140946)(411.71188561,101.26140946)
\curveto(412.36813561,101.26140946)(413.68063561,100.88640946)(414.05563561,98.91765946)
\curveto(414.27438561,97.79265946)(414.27438561,96.29265946)(414.27438561,95.04265946)
\curveto(414.27438561,93.54265946)(414.27438561,92.19890946)(414.05563561,90.91765946)
\curveto(413.74313561,89.01140946)(412.61813561,88.41765946)(411.71188561,88.41765946)
\closepath
\moveto(411.71188561,88.41765946)
}
}
{
\newrgbcolor{curcolor}{0 0 0}
\pscustom[linestyle=none,fillstyle=solid,fillcolor=curcolor]
{
\newpath
\moveto(422.45575157,93.41765946)
\curveto(422.45575157,94.94890946)(422.23700157,97.35515946)(421.14325157,99.63640946)
\curveto(419.95575157,102.07390946)(418.23700157,103.38640946)(418.04950157,103.38640946)
\curveto(417.92450157,103.38640946)(417.83075157,103.29265946)(417.83075157,103.16765946)
\curveto(417.83075157,103.10515946)(417.83075157,103.07390946)(418.20575157,102.69890946)
\curveto(420.17450157,100.73015946)(421.29950157,97.57390946)(421.29950157,93.41765946)
\curveto(421.29950157,89.98015946)(420.58075157,86.48015946)(418.11200157,83.98015946)
\curveto(417.83075157,83.73015946)(417.83075157,83.69890946)(417.83075157,83.63640946)
\curveto(417.83075157,83.51140946)(417.92450157,83.41765946)(418.04950157,83.41765946)
\curveto(418.23700157,83.41765946)(420.04950157,84.79265946)(421.20575157,87.32390946)
\curveto(422.23700157,89.51140946)(422.45575157,91.73015946)(422.45575157,93.41765946)
\closepath
\moveto(422.45575157,93.41765946)
}
}
\end{pspicture}

    \caption{光线与轴对齐厚板相交。这里展示的两平面由关于常数$c$的$x=c$描述。
        每个平面的法线为$(1,0,0)$。除非光线平行于平面,否则它会在
        参数位置$t_{\mathrm{near}}$和$t_{\mathrm{far}}$处与厚板相交两次。}
    \label{fig:3.3}
\end{figure}

如果方法\refvar{Bounds3::IntersectP}{()}返回{\ttfamily true},
相交的参数范围在可选参数{\ttfamily hitt0}和{\ttfamily hitt1}内返回。
光线超出范围{\ttfamily (0, \refvar[tMax]{Ray::tMax}{})}的相交部分被忽略。
如果射线端点在框内,{\ttfamily hitt0}返回$0$。
\begin{lstlisting}
`\refcode{Geometry Inline Functions}{+=}\lastnext{GeometryInlineFunctions}`
template <typename T>
inline bool `\refvar{Bounds3}{}`<T>::`\initvar[Bounds3::IntersectP]{IntersectP}{}`(const `\refvar{Ray}{}` &ray, `\refvar{Float}{}` *hitt0,
        `\refvar{Float}{}` *hitt1) const {
    `\refvar{Float}{}` t0 = 0, t1 = ray.`\refvar{tMax}{}`;
    for (int i = 0; i < 3; ++i) {
        `\refcode{Update interval for ith bounding box slab}{}`
    }
    if (hitt0) *hitt0 = t0;
    if (hitt1) *hitt1 = t1;
    return true;
}
\end{lstlisting}

对于每对平面,该例程需要两处光线-平面相交。
例如,两个垂直于$x$轴的平面描述的厚板可以由
通过点$(x_0,0,0)$和$(x_1,0,0)$的平面描述,每个的法线都是$(1,0,0)$。
考虑平面相交的第一个$t$值即$t_0$。
端点为$\bm o$方向为$\bm d$的射线与平面$ax+by+cz+d=0$相交处的参数$t$值
可通过将射线方程代入平面方程求得:
\begin{align*}
    0 & =a(o_x+td_x)+b(o_y+td_y)+c(o_z+td_z)+d      \\
      & =(a,b,c)\cdot\bm o+t(a,b,c)\cdot\bm d+d\, .
\end{align*}
解得$t$为
\begin{align*}
    t=\frac{-d-(a,b,c)\cdot\bm o}{(a,b,c)\cdot\bm d}\, .
\end{align*}

因为平面法线的$y$和$z$分量是零,所以$b$和$c$是零,$a$是1。
平面的系数$d$是$-x_0$。我们可以用这些信息和点积的定义大大简化计算:
\begin{align*}
    t_0=\frac{x_0-o_x}{d_x}\, .
\end{align*}

计算厚板相交处$t$值的代码从计算光线方向相应分量的倒数开始,
这样它可以乘以该因子而不是执行多次除法。
注意尽管它除以该分量,但却不需要检查除数非零。
如果是零,则{\ttfamily invRayDir}会有无限值$-\infty$或$\infty$,
且算法剩余部分仍能正确工作
\footnote{这里假设所用架构支持现代系统通用的IEEE浮点算术。
IEEE浮点算术相关属性是对于所有{\ttfamily v>0},{\ttfamily v/0=}$\infty$,
对所有{\ttfamily w<0},{\ttfamily w/0=-}$\infty$,
其中$\infty$是特殊值即任意正数乘以$\infty$为$\infty$,
任意负数乘以$\infty$为$-\infty$。
关于浮点算术的更多信息见\protect\refsub{浮点算术}。}。
\begin{lstlisting}
`\initcode{Update interval for ith bounding box slab}{=}`
`\refvar{Float}{}` invRayDir = 1 / ray.`\refvar[Ray::d]{d}{}`[i];
`\refvar{Float}{}` tNear = (`\refvar{pMin}{}`[i] - ray.`\refvar[Ray::o]{o}{}`[i]) * invRayDir;
`\refvar{Float}{}` tFar  = (`\refvar{pMax}{}`[i] - ray.`\refvar[Ray::o]{o}{}`[i]) * invRayDir;
`\refcode{Update parametric interval from slab intersection t values}{}`
\end{lstlisting}

两个距离被记下来,这样{\ttfamily tNear}存有最近的相交处而{\ttfamily tFar}的最远。
这给出了参数范围$[${\ttfamily tNear,tFar}$]$,
用于计算和当前范围$[${\ttfamily t0,t1}$]$的交集以计算新范围。
如果新范围为空(即{\ttfamily t0>t1}),则代码会立即返回错误。

这里还有另一个浮点相关的细节:当射线端点在边界框厚板平面上
且射线在厚板平面内时,{\ttfamily tNear}或{\ttfamily tFar}可能会
由$\frac{0}{0}$的表达式算出,得到IEEE浮点“not a number”(NaN)值。
像无限值那样,NaN有明确指定的含义:例如任何含有NaN的逻辑比较结果都是假。
因此应细心编写更新{\ttfamily t0}和{\ttfamily t1}值的代码,
这样如果{\ttfamily tNear}或{\ttfamily tFar}是NaN,则
{\ttfamily t0}或{\ttfamily t1}不会取NaN值而总是保持不变。
\begin{lstlisting}
`\initcode{Update parametric interval from slab intersection t values}{=}`
if (tNear > tFar) std::swap(tNear, tFar);
`\refcode{Update tFar to ensure robust ray–bounds intersection}{}`
t0 = tNear > t0 ? tNear : t0;
t1 = tFar  < t1 ? tFar  : t1;
if (t0 > t1) return false;
\end{lstlisting}

\refvar{Bounds3}{}也提供了特殊化的方法\refvar[Bounds3::IntersectP2]{IntersectP}{()},
它接收射线方向的倒数作为额外参数,
这样每次调用\refvar[Bounds3::IntersectP2]{IntersectP}{()}就不用计算三个倒数了。

这个版本还接收表示每个方向分量是否为负的预计算值,
使得可以去掉原始例程中对算出的{\ttfamily tNear}和{\ttfamily tFar}值的比较
而是直接分别计算近和远的值。
因为原始代码中将这些值从低到高排序的比较取决于算出的值,
它们对于执行的处理器会很低效,
因为在比较之前必须完成计算它们的值。

如果射线段全在边界框内则该例程返回{\ttfamily true},
即使相交处不在射线的范围$(0,${\ttfamily tMax}$)$内也是这样。
\begin{lstlisting}
`\refcode{Geometry Inline Functions}{+=}\lastnext{GeometryInlineFunctions}`
template <typename T>
inline bool `\refvar{Bounds3}{}`<T>::`\initvar[Bounds3::IntersectP2]{IntersectP}{}`(const `\refvar{Ray}{}` &ray, const `\refvar{Vector3f}{}` &invDir,
        const int dirIsNeg[3]) const {
    const `\refvar{Bounds3f}{}` &bounds = *this;
    `\refcode{Check for ray intersection against x and y slabs}{}`
    `\refcode{Check for ray intersection against z slab}{}`
    return (tMin < ray.`\refvar{tMax}{}`) && (tMax > 0);
}
\end{lstlisting}

如果射线方向向量为负\sidenote{译者注:指分量为负。},
则用厚板和较大的两个边界值计算“近”参数化的相交处,较小的计算“远”相交处。
该实现可用该观察直接在每个方向计算近和远参数值。
\begin{lstlisting}
`\initcode{Check for ray intersection against x and y slabs}{=}`
`\refvar{Float}{}` tMin =  (bounds[  dirIsNeg[0]].x - ray.o.x) * invDir.x;
`\refvar{Float}{}` tMax =  (bounds[1-dirIsNeg[0]].x - ray.o.x) * invDir.x;
`\refvar{Float}{}` tyMin = (bounds[  dirIsNeg[1]].y - ray.o.y) * invDir.y;
`\refvar{Float}{}` tyMax = (bounds[1-dirIsNeg[1]].y - ray.o.y) * invDir.y;
`\refcode{Update tMax and tyMax to ensure robust bounds intersection}{}`
if (tMin > tyMax || tyMin > tMax) 
    return false;
if (tyMin > tMin) tMin = tyMin; 
if (tyMax < tMax) tMax = tyMax;
\end{lstlisting}

代码片\refcode{Check for ray intersection against z slab}{}类似,这里就不介绍了
\sidenote{译者注:笔者还是补充上来了。}。
\begin{lstlisting}
`\initcode{Check for ray intersection against z slab}{=}`
`\refvar{Float}{}` tzMin = (bounds[  dirIsNeg[2]].z - ray.o.z) * invDir.z;
`\refvar{Float}{}` tzMax = (bounds[1-dirIsNeg[2]].z - ray.o.z) * invDir.z;
`\refcode{Update tzMax to ensure robust bounds intersection}{}`
if (tMin > tzMax || tzMin > tMax)
    return false;
if (tzMin > tMin)
    tMin = tzMin;
if (tzMax < tMax)
    tMax = tzMax;
\end{lstlisting}

相交测试是遍历\refsec{层次包围盒}介绍的加速结构\refvar{BVHAccel}{}的核心。
因为遍历BVH树时要执行如此多的光线-边界框相交测试,
我们发现比起没有接收预计算方向导数和符号的\refvar{Bounds3::IntersectP}{()},
该优化方法在整个渲染时间中提供了大约15\%的性能提升。

\subsection{相交测试}\label{sub:相交测试}
\refvar{Shape}{}实现必须提供一个(也可能两个)
方法的实现以测试光线与其形状的相交性。
第一个是\refvar{Shape::Intersect}{()},
如果有的话就返回关于单个光线-形状相交处的几何信息,
它在沿光线的参数范围{\ttfamily (0, \refvar{tMax}{})}内对应于首次相交。
\begin{lstlisting}
`\refcode{Shape Interface}{+=}\lastnext{ShapeInterface}`
virtual bool `\initvar[Shape::Intersect]{Intersect}{}`(const `\refvar{Ray}{}` &ray, `\refvar{Float}{}` *tHit,
    `\refvar{SurfaceInteraction}{}` *isect, bool testAlphaTexture = true) const = 0;
\end{lstlisting}

在阅读(和编写)相交例程时要记住一些重点:
\begin{itemize}
    \item \refvar{Ray}{}结构含有成员\refvar[tMax]{Ray::tMax}{}定义光线终点。
          相交例程必须忽略发生在该点后的任何相交。
    \item 如果找到相交处,其沿光线的参数距离应该存于{\ttfamily tHit}指针
          传入相交例程。如果沿光线有多次相交,则应报告最近的那个。
    \item 关于相交处的信息存于结构\refvar{SurfaceInteraction}{},
          它完全刻画了曲面的局部几何属性。该类在整个pbrt中大量使用,
          它用于将光线追踪器的几何部分与着色和照明部分干净地隔离开。
          类\refvar{SurfaceInteraction}{}已在\refsec{交互作用}定义
          \footnote{几乎所有光线追踪器都使用这个通用习惯来返回
              关于与形状相交的几何信息。作为一种优化,许多光线追踪器在找到相交处时
              都只部分初始化相交处信息,并只存储够用的信息,如果确实需要剩余值则稍后再计算。
              在稍后会找到与另一个形状更近的相交处的情况下该方法节省了工作。
              按我们的经验,计算所有信息的额外工作并不多,而对于有着复杂场景数据管理算法
              (例如当占用太多内存时从主内存剔除几何体并写入磁盘)的渲染器而言,
              延迟的方法可能会失败,因为形状不再在内存中了。}。
    \item 传入相交例程的光线在世界空间中,所以如果相交测试需要,
          形状应负责将它们变换到物体空间。返回的相交信息应在世界空间中。
\end{itemize}

一些形状实现支持用纹理切掉它们的一些表面;
参数{\ttfamily testAlphaTexture}表示这些形状是否应该为当前的相交测试执行此操作。
第二个相交测试方法\refvar{Shape::IntersectP}{()},
是一个决定相交是否发生的断言函数,不返回关于相交的任何信息。
类\refvar{Shape}{}提供了方法\refvar[Shape::IntersectP]{IntersectP}{()}的
一个默认实现,即调用方法\refvar{Shape::Intersect}{()}并
只是忽略算得的关于交点的额外信息。
因为这非常浪费,pbrt中几乎所有形状实现都
为\refvar[Shape::IntersectP]{IntersectP}{()}提供了
一个更高效的实现,即确定是否存在相交而不计算其所有细节。
\begin{lstlisting}
`\refcode{Shape Interface}{+=}\lastnext{ShapeInterface}`
virtual bool `\initvar[Shape::IntersectP]{IntersectP}{}`(const `\refvar{Ray}{}` &ray,
        bool testAlphaTexture = true) const {
    `\refvar{Float}{}` tHit = ray.`\refvar{tMax}{}`;
    `\refvar{SurfaceInteraction}{}` isect;
    return `\refvar[Shape::Intersect]{Intersect}{}`(ray, &tHit, &isect, testAlphaTexture);
}
\end{lstlisting}

\subsection{表面积}\label{sub:表面积}
为了合理地将\refvar{Shape}{}用作面光源,需要能够计算物体空间中形状的表面积。
\begin{lstlisting}
`\refcode{Shape Interface}{+=}\lastnext{ShapeInterface}`
virtual `\refvar{Float}{}` `\initvar[Shape::Area]{Area}{}`() const = 0;
\end{lstlisting}

\subsection{面性}\label{sub:面性}
许多渲染系统,尤其是基于扫描线或缓冲区算法的,
支持“单面”\sidenote{译者注:原文one-sided。此外“面性”原文sidedness。}形状概念——
形状可从前面看到但从后面看就消失了。
特别地,若几何体封闭且总是从外面看,
则其背面部分可以去掉而不改变结果图像。
该优化能极大提升这类隐藏表面去除算法的速度。
然而,当光线追踪使用该技术时,提升性能的潜力下降了,
因为它经常需要在确定曲面法线以进行背面测试之前执行光线-物体相交。
而且,若单面物体实际上并不封闭则该功能会导致物理矛盾的场景描述。
例如,当阴影射线从光源追踪到表面上一点时,另一表面可能阻挡光线,
但若阴影射线从另一方向追踪时则不会这样。
由于这些原因,pbrt不支持该功能。