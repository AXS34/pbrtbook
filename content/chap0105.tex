\section{如何继续阅读本书}\label{sec:如何继续阅读本书}
我们编写这本书时预设了大致从前往后的阅读顺序。
一般我们尽量减少向后引用还没有介绍的内容和接口,
但我们也假设读者在文中任何地方都已了解前文内容。
然而,读者可能希望跳过一些深入高级话题的章节
(尤其是第一次阅读时);
每个高级章节的标题都标注了星号。

因为系统的模块化本质,
所以要求读者应熟悉
像\refvar{Point3f}{}、\refvar{Ray}{}和
\refvar{Spectrum}{}等底层类;
\reftab{1.1}列出的抽象基类定义的接口;以及
\refvar{SamplerIntegrator::Render}{()}
中的渲染循环。例如,有了这些知识,
不关心基于透视投影矩阵的相机模型具体是如何
把\refvar{CameraSample}{}映射为光线的读者
可以跳过相机的实现而只记住方法
\refvar[GenerateRayDifferential]{Camera::GenerateRayDifferential}{()}
会以某种方式返回一个
\refvar{CameraSample}{}
到\refvar{RayDifferential}{}中。

本书剩下的内容分为四个主要部分,每部分都有几章。
首先,第\refchap{几何与变换}到第\refchap{图元和相交加速}定义了
系统的主要几何功能。
第\refchap{几何与变换}有诸如
\refvar{Point3f}{}、
\refvar{Ray}{}和
\refvar{Bounds3f}{}等底层类。
第\refchap{形状}定义\refvar{Shape}{}接口,
给出了大量形状的实现,
展示了如何进行光线-形状相交测试。
第\refchap{图元和相交加速}有加速结构体的实现,
通过跳过测试已表明一定不会有光线相交的图元来加快光线追踪。

第二部分涵盖了成像过程。
首先,第\refchap{颜色和辐射度学}介绍了用于度量光的物理单位
和表示波长变化分布(即颜色)的类\refvar{Spectrum}{}。
第\refchap{相机模型}{}定义了\refvar{Camera}{}接口和
一些不同的相机实现。
将样本放置于图像平面的类\refvar{Sampler}{}是第\refchap{采样与重构}的主题,
将胶片上的辐射值转换为适合显示的图像的整个流程将在\refsec{胶片与成像管道}解释。

本书第三部分是关于光以及它怎样从表面和介质散射的。
第\refchap{反射模型}包括一组构建块\sidenote{译者注:原文building-block}类,
定义了各种类型的表面反射。
第\refchap{材质}描述的材质使用这些反射函数
实现大量不同的表面材质,例如塑料、玻璃和金属。
第\refchap{纹理}介绍了纹理,
描述了材质属性(颜色、粗糙度等)随表面的变化,
第\refchap{体积散射}抽象描述了光是如何在介质中被散射和吸收的。
最后第\refchap{光源}有光源的接口和许多光源的实现。

最后一部分将本书剩余的思想整合起来实现许多有趣的光传输算法。
\refchap{蒙特卡罗积分}介绍了蒙特卡罗积分理论,
即估计复杂积分值的统计手段,
并描述了把蒙特卡罗应用到光照和光散射的底层例程。
第\refchap{光传输I:表面反射}、\refchap{光传输II:体积渲染}
和\refchap{光传输III:双向方法}利用蒙特卡罗积分
计算比\refvar{WhittedIntegrator}{}更精确的光传输方程近似,
运用了路径追踪、双向路径追踪、Metropolis光传输和光子映射等技术。

第\refchap{回顾与未来}是本书最后一章,
提供了简短的回顾、关于系统设计决策的讨论
以及许多比练习题更深远的项目的建议。
附录描述了实用函数和如何创建场景解析输入文件的细节。

\subsection{关于习题}\label{sub:关于习题}
每章最后会有关于那章内容的习题。
每道题都标记了三种难度级别之一:

\circleone 只花一两小时的习题;

\circletwo 适合作为课程作业的阅读和/或实现任务且需10到20小时完成;

\circlethree 可能需要40小时或以上的推荐结课作业。