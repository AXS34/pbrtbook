\section{相机模型}\label{sec:相机模型}

\begin{lstlisting}
`\initcode{Camera Declarations}{=}\initnext{CameraDeclarations}`
class `\initvar{Camera}{}` {
public:
    `\refcode{Camera Interface}{}`
    `\refcode{Camera Public Data}{}`
};
\end{lstlisting}
\begin{lstlisting}
`\initcode{Camera Interface}{=}\initnext{CameraInterface}`
`\refvar{Camera}{}`(const `\refvar{AnimatedTransform}{}` &CameraToWorld, `\refvar{Float}{}` shutterOpen,
    `\refvar{Float}{}` shutterClose, `\refvar{Film}{}` *film, const `\refvar{Medium}{}` *medium);
\end{lstlisting}
\begin{lstlisting}
`\initcode{Camera Public Data}{=}`
`\refvar{AnimatedTransform}{}` `\initvar{CameraToWorld}{}`;
const `\refvar{Float}{}` `\initvar{shutterOpen}{}`, `\initvar{shutterClose}{}`;
`\refvar{Film}{}` *`\initvar{film}{}`;
const `\refvar{Medium}{}` *`\initvar[Camera::medium]{medium}{}`;
\end{lstlisting}
相机子类首先要实现的方法是\refvar[GenerateRay]{Camera::GenerateRay}{()},
它计算给定样本相应的光线。
规范化返回光线的方向分量很重要——系统许多其他部分会依赖此行为。
\begin{lstlisting}
`\refcode{Camera Interface}{+=}\lastnext{CameraInterface}`
virtual `\refvar{Float}{}` `\initvar{GenerateRay}{}`(const `\refvar{CameraSample}{}` &sample,
    `\refvar{Ray}{}` *ray) const = 0;
\end{lstlisting}
\begin{lstlisting}
`\refcode{Camera Declarations}{+=}\lastnext{CameraDeclarations}`
struct `\initvar{CameraSample}{}` {
    `\refvar{Point2f}{}` `\initvar{pFilm}{}`;
    `\refvar{Point2f}{}` `\initvar{pLens}{}`;
    `\refvar{Float}{}` `\initvar[CameraSample::time]{time}{}`;
};
\end{lstlisting}
\begin{lstlisting}
`\initcode{Camera Method Definitions}{=}`
`\refvar{Float}{}` `\refvar{Camera}{}`::`\initvar{GenerateRayDifferential}{}`(const `\refvar{CameraSample}{}` &sample,
    `\refvar{RayDifferential}{}` *rd) const {
    `\refvar{Float}{}` wt = `\refvar{GenerateRay}{}`(sample, rd);
    `\refcode{Find camera ray after shifting one pixel in the $x$ direction}{}`
    `\refcode{Find camera ray after shifting one pixel in the $y$ direction}{}`
    rd->`\refvar{hasDifferentials}{}` = true;
    return wt;
}
\end{lstlisting}