\section{边界框}\label{sec:边界框}

系统的许多部分都在与坐标轴对齐的空间区域上操作。
例如,pbrt的多线程是通过把图像划分为可以独立处理的矩形图块实现的,
\refsec{层次包围盒}中的包围盒层次使用3D框来限定场景中的几何图元。
模板类\refvar{Bounds2}{}和\refvar{Bounds3}{}用于表示这类区域范围。
两者都由表示其坐标范围类型的{\ttfamily T}来参数化。
\begin{lstlisting}
`\initcode{Bounds Declarations}{=}\initnext{BoundsDeclarations}`
template <typename T> class `\initvar{Bounds2}{}` {
public:
    `\refcode{Bounds2 Public Methods}{}`
    `\refcode{Bounds2 Public Data}{}`
};
\end{lstlisting}
\begin{lstlisting}
`\refcode{Bounds Declarations}{+=}\lastnext{BoundsDeclarations}`
template <typename T> class `\initvar{Bounds3}{}` {
public:
    `\refcode{Bounds3 Public Methods}{}`
    `\refcode{Bounds3 Public Data}{}`
};
\end{lstlisting}
\begin{lstlisting}
`\refcode{Bounds Declarations}{+=}\lastcode{BoundsDeclarations}`
typedef `\refvar{Bounds2}{}`<`\refvar{Float}{}`> `\initvar{Bounds2f}{}`;
typedef `\refvar{Bounds2}{}`<int>   `\initvar{Bounds2i}{}`;
typedef `\refvar{Bounds3}{}`<`\refvar{Float}{}`> `\initvar{Bounds3f}{}`;
typedef `\refvar{Bounds3}{}`<int>   `\initvar{Bounds3i}{}`;
\end{lstlisting}

这类\keyindex{边界框}{bounding box}{}有若干种可能的表示;
pbrt使用\keyindex{轴对齐的边界框}{axis-aligned bounding box}{bounding box边界框}(AABB),
框边相互垂直且与坐标系统的轴对齐。
另一种可能的选择是\keyindex{定向边界框}{oriented bounding box}{bounding box边界框}(OBB),
不同侧的框边依然互相垂直但不一定对齐坐标系统。
3D的AABB可以用其一个顶点和三个长度表示,
每一个表示沿$x,y$和$z$坐标轴张开的距离。
也可以用框的两个对角顶点描述它。
我们为pbrt的类\refvar{Bounds2}{}和\refvar{Bounds3}{}选择了两点表示法;
它们存储了具有最小坐标值和最大坐标值的两个顶点的位置。
边界框的2D图示及其表示如\reffig{2.8}所示。
\begin{figure}[htbp]
    \centering%LaTeX with PSTricks extensions
%%Creator: Inkscape 1.0.1 (3bc2e813f5, 2020-09-07)
%%Please note this file requires PSTricks extensions
\psset{xunit=.5pt,yunit=.5pt,runit=.5pt}
\begin{pspicture}(664.88000488,292.75)
{
\newrgbcolor{curcolor}{0 0 0}
\pscustom[linewidth=1,linecolor=curcolor]
{
\newpath
\moveto(166.28999329,201.25)
\lineto(355.73999023,201.25)
\lineto(355.73999023,11.80000305)
\lineto(166.28999329,11.80000305)
\closepath
}
}
{
\newrgbcolor{curcolor}{0 0 0}
\pscustom[linewidth=1,linecolor=curcolor]
{
\newpath
\moveto(308.98999023,281.17000008)
\lineto(498.43998718,281.17000008)
\lineto(498.43998718,91.72000313)
\lineto(308.98999023,91.72000313)
\closepath
}
}
{
\newrgbcolor{curcolor}{0 0 0}
\pscustom[linewidth=1,linecolor=curcolor]
{
\newpath
\moveto(166.29,201.25)
\curveto(166.05,201.25)(308.99,281.17)(308.99,281.17)
}
}
{
\newrgbcolor{curcolor}{0 0 0}
\pscustom[linewidth=1,linecolor=curcolor]
{
\newpath
\moveto(355.73001099,201.25)
\lineto(498.44000244,281.17000008)
}
}
{
\newrgbcolor{curcolor}{0 0 0}
\pscustom[linewidth=1,linecolor=curcolor]
{
\newpath
\moveto(355.73001099,11.79998779)
\lineto(498.44000244,91.72000122)
}
}
{
\newrgbcolor{curcolor}{0 0 0}
\pscustom[linewidth=1,linecolor=curcolor]
{
\newpath
\moveto(166.28999329,11.79998779)
\lineto(308.98999023,91.72000122)
}
}
{
\newrgbcolor{curcolor}{0 0 0}
\pscustom[linestyle=none,fillstyle=solid,fillcolor=curcolor]
{
\newpath
\moveto(504.89001322,280)
\curveto(504.89001322,285.54181338)(498.19021164,288.31615943)(494.27203282,284.39798061)
\curveto(490.353854,280.47980179)(493.12820005,273.78000021)(498.67001343,273.78000021)
\curveto(504.21182681,273.78000021)(506.98617285,280.47980179)(503.06799404,284.39798061)
\curveto(499.14981522,288.31615943)(492.45001364,285.54181338)(492.45001364,280)
\curveto(492.45001364,274.45818662)(499.14981522,271.68384057)(503.06799404,275.60201939)
\curveto(506.98617285,279.52019821)(504.21182681,286.21999979)(498.67001343,286.21999979)
\curveto(493.12820005,286.21999979)(490.353854,279.52019821)(494.27203282,275.60201939)
\curveto(498.19021164,271.68384057)(504.89001322,274.45818662)(504.89001322,280)
\closepath
}
}
{
\newrgbcolor{curcolor}{0 0 0}
\pscustom[linestyle=none,fillstyle=solid,fillcolor=curcolor]
{
\newpath
\moveto(172.39999247,11.72000122)
\curveto(172.39999247,17.2618146)(165.70019089,20.03616065)(161.78201207,16.11798183)
\curveto(157.86383325,12.19980301)(160.63817929,5.50000143)(166.17999268,5.50000143)
\curveto(171.72180606,5.50000143)(174.4961521,12.19980301)(170.57797328,16.11798183)
\curveto(166.65979447,20.03616065)(159.95999289,17.2618146)(159.95999289,11.72000122)
\curveto(159.95999289,6.17818784)(166.65979447,3.40384179)(170.57797328,7.32202061)
\curveto(174.4961521,11.24019943)(171.72180606,17.94000101)(166.17999268,17.94000101)
\curveto(160.63817929,17.94000101)(157.86383325,11.24019943)(161.78201207,7.32202061)
\curveto(165.70019089,3.40384179)(172.39999247,6.17818784)(172.39999247,11.72000122)
\closepath
}
}
{
\newrgbcolor{curcolor}{0 0 0}
\pscustom[linestyle=none,fillstyle=solid,fillcolor=curcolor]
{
\newpath
\moveto(522.20314677,276.26744742)
\curveto(522.20314677,275.59296704)(522.07554237,274.99747985)(521.82033358,274.48098586)
\curveto(521.56512478,273.96449188)(521.22180819,273.53914389)(520.7903838,273.20494189)
\curveto(520.27996621,272.8038995)(519.71789923,272.51830871)(519.10418284,272.34816951)
\curveto(518.49654285,272.17803032)(517.72180187,272.09296072)(516.77995989,272.09296072)
\lineto(511.96745121,272.09296072)
\lineto(511.96745121,285.6645998)
\lineto(515.98698971,285.6645998)
\curveto(516.97744289,285.6645998)(517.71876367,285.6281414)(518.21095206,285.55522461)
\curveto(518.70314045,285.48230781)(519.17406144,285.33039781)(519.62371503,285.09949462)
\curveto(520.12197982,284.83820942)(520.48352561,284.50096923)(520.7083524,284.08777404)
\curveto(520.9331792,283.68065525)(521.04559259,283.19150506)(521.04559259,282.62032347)
\curveto(521.04559259,281.97622509)(520.8815298,281.4263109)(520.55340421,280.97058091)
\curveto(520.22527861,280.52092732)(519.78777782,280.15938153)(519.24090184,279.88594354)
\lineto(519.24090184,279.81302674)
\curveto(520.15843822,279.62465834)(520.8815298,279.22057775)(521.41017659,278.60078497)
\curveto(521.93882337,277.98706858)(522.20314677,277.2092894)(522.20314677,276.26744742)
\closepath
\moveto(519.16798504,282.38334388)
\curveto(519.16798504,282.71146947)(519.11329744,282.98794567)(519.00392224,283.21277246)
\curveto(518.89454705,283.43759925)(518.71833145,283.61989125)(518.47527546,283.75964845)
\curveto(518.18968466,283.92371124)(517.84332987,284.02397184)(517.43621108,284.06043024)
\curveto(517.02909229,284.10296504)(516.5247511,284.12423244)(515.92318751,284.12423244)
\lineto(513.77214197,284.12423244)
\lineto(513.77214197,280.20495453)
\lineto(516.10547951,280.20495453)
\curveto(516.6705847,280.20495453)(517.12023829,280.23229833)(517.45444028,280.28698593)
\curveto(517.78864227,280.34774993)(518.09853866,280.46927792)(518.38412946,280.65156992)
\curveto(518.66972025,280.83386192)(518.87024145,281.06780331)(518.98569304,281.3533941)
\curveto(519.10722104,281.6450613)(519.16798504,281.98837789)(519.16798504,282.38334388)
\closepath
\moveto(520.32553921,276.19453062)
\curveto(520.32553921,276.74140661)(520.24350781,277.1758692)(520.07944502,277.49791839)
\curveto(519.91538222,277.81996759)(519.61763863,278.09340558)(519.18621424,278.31823237)
\curveto(518.89454705,278.47014237)(518.53907765,278.56736477)(518.11980606,278.60989957)
\curveto(517.70661087,278.65851077)(517.20226968,278.68281637)(516.6067825,278.68281637)
\lineto(513.77214197,278.68281637)
\lineto(513.77214197,273.63332808)
\lineto(516.16016711,273.63332808)
\curveto(516.95009909,273.63332808)(517.59723568,273.67282468)(518.10157686,273.75181788)
\curveto(518.60591805,273.83688748)(519.01911324,273.98879748)(519.34116243,274.20754787)
\curveto(519.68144083,274.44452747)(519.93057322,274.71492726)(520.08855962,275.01874725)
\curveto(520.24654601,275.32256724)(520.32553921,275.71449504)(520.32553921,276.19453062)
\closepath
}
}
{
\newrgbcolor{curcolor}{0 0 0}
\pscustom[linestyle=none,fillstyle=solid,fillcolor=curcolor]
{
\newpath
\moveto(533.3047291,277.1789074)
\curveto(533.3047291,275.52005024)(532.87938111,274.21058607)(532.02868513,273.25051489)
\curveto(531.17798915,272.29044372)(530.03866417,271.81040813)(528.61071021,271.81040813)
\curveto(527.17060344,271.81040813)(526.02520207,272.29044372)(525.17450609,273.25051489)
\curveto(524.32988651,274.21058607)(523.90757672,275.52005024)(523.90757672,277.1789074)
\curveto(523.90757672,278.83776456)(524.32988651,280.14722873)(525.17450609,281.10729991)
\curveto(526.02520207,282.07344749)(527.17060344,282.55652128)(528.61071021,282.55652128)
\curveto(530.03866417,282.55652128)(531.17798915,282.07344749)(532.02868513,281.10729991)
\curveto(532.87938111,280.14722873)(533.3047291,278.83776456)(533.3047291,277.1789074)
\closepath
\moveto(531.53649674,277.1789074)
\curveto(531.53649674,278.49748617)(531.27824974,279.47578655)(530.76175576,280.11380853)
\curveto(530.24526177,280.75790692)(529.52824658,281.07995611)(528.61071021,281.07995611)
\curveto(527.68102103,281.07995611)(526.95792944,280.75790692)(526.44143546,280.11380853)
\curveto(525.93101787,279.47578655)(525.67580907,278.49748617)(525.67580907,277.1789074)
\curveto(525.67580907,275.90286343)(525.93405607,274.93367765)(526.45055006,274.27135007)
\curveto(526.96704404,273.61509888)(527.68709743,273.28697329)(528.61071021,273.28697329)
\curveto(529.52217018,273.28697329)(530.23614717,273.61206068)(530.75264116,274.26223547)
\curveto(531.27521154,274.91848665)(531.53649674,275.89071063)(531.53649674,277.1789074)
\closepath
}
}
{
\newrgbcolor{curcolor}{0 0 0}
\pscustom[linestyle=none,fillstyle=solid,fillcolor=curcolor]
{
\newpath
\moveto(544.3971982,272.09296072)
\lineto(542.68365344,272.09296072)
\lineto(542.68365344,273.22317109)
\curveto(542.10639545,272.7674411)(541.55344307,272.41804811)(541.02479628,272.17499212)
\curveto(540.49614949,271.93193612)(539.91281511,271.81040813)(539.27479312,271.81040813)
\curveto(538.20534675,271.81040813)(537.37287997,272.13549552)(536.77739278,272.7856703)
\curveto(536.18190559,273.44192149)(535.884162,274.40199267)(535.884162,275.66588384)
\lineto(535.884162,282.27396868)
\lineto(537.59770676,282.27396868)
\lineto(537.59770676,276.47708322)
\curveto(537.59770676,275.96058923)(537.62201236,275.51701204)(537.67062356,275.14635165)
\curveto(537.71923476,274.78176766)(537.82253356,274.46883306)(537.98051995,274.20754787)
\curveto(538.14458275,273.94018628)(538.35725674,273.74574148)(538.61854194,273.62421348)
\curveto(538.87982713,273.50268549)(539.25960212,273.44192149)(539.75786691,273.44192149)
\curveto(540.2014441,273.44192149)(540.68451789,273.55737309)(541.20708828,273.78827628)
\curveto(541.73573506,274.01917948)(542.22792345,274.31388487)(542.68365344,274.67239246)
\lineto(542.68365344,282.27396868)
\lineto(544.3971982,282.27396868)
\closepath
}
}
{
\newrgbcolor{curcolor}{0 0 0}
\pscustom[linestyle=none,fillstyle=solid,fillcolor=curcolor]
{
\newpath
\moveto(556.28263407,272.09296072)
\lineto(554.56908931,272.09296072)
\lineto(554.56908931,277.88984618)
\curveto(554.56908931,278.35772897)(554.54174551,278.79522976)(554.48705791,279.20234855)
\curveto(554.43237032,279.61554374)(554.33210972,279.93759294)(554.18627612,280.16849613)
\curveto(554.03436613,280.42370493)(553.81561573,280.61207332)(553.53002494,280.73360132)
\curveto(553.24443414,280.86120571)(552.87377375,280.92500791)(552.41804376,280.92500791)
\curveto(551.95016097,280.92500791)(551.46101079,280.80955632)(550.9505932,280.57865312)
\curveto(550.44017561,280.34774993)(549.95102542,280.05304453)(549.48314263,279.69453694)
\lineto(549.48314263,272.09296072)
\lineto(547.76959787,272.09296072)
\lineto(547.76959787,282.27396868)
\lineto(549.48314263,282.27396868)
\lineto(549.48314263,281.14375831)
\curveto(550.01786582,281.5873355)(550.57081821,281.93369029)(551.14199979,282.18282268)
\curveto(551.71318138,282.43195508)(552.29955397,282.55652128)(552.90111755,282.55652128)
\curveto(554.00094593,282.55652128)(554.83948911,282.22535748)(555.41674709,281.5630299)
\curveto(555.99400508,280.90070231)(556.28263407,279.94670754)(556.28263407,278.70104557)
\closepath
}
}
{
\newrgbcolor{curcolor}{0 0 0}
\pscustom[linestyle=none,fillstyle=solid,fillcolor=curcolor]
{
\newpath
\moveto(567.83994817,272.09296072)
\lineto(566.12640341,272.09296072)
\lineto(566.12640341,273.1593689)
\curveto(565.63421502,272.73402091)(565.12075923,272.40285711)(564.58603604,272.16587752)
\curveto(564.05131286,271.92889792)(563.47101667,271.81040813)(562.84514748,271.81040813)
\curveto(561.62986751,271.81040813)(560.66371994,272.27829092)(559.94670475,273.21405649)
\curveto(559.23576597,274.14982207)(558.88029658,275.44713344)(558.88029658,277.1059906)
\curveto(558.88029658,277.96883938)(559.00182457,278.73750396)(559.24488057,279.41198435)
\curveto(559.49401296,280.08646473)(559.82821495,280.66068452)(560.24748654,281.13464371)
\curveto(560.66068174,281.5964501)(561.14071732,281.94888129)(561.68759331,282.19193728)
\curveto(562.2405457,282.43499328)(562.81172728,282.55652128)(563.40113807,282.55652128)
\curveto(563.93586126,282.55652128)(564.40982045,282.49879548)(564.82301564,282.38334388)
\curveto(565.23621083,282.27396868)(565.67067342,282.10079129)(566.12640341,281.86381169)
\lineto(566.12640341,286.27527799)
\lineto(567.83994817,286.27527799)
\closepath
\moveto(566.12640341,274.59947566)
\lineto(566.12640341,280.44193412)
\curveto(565.66459702,280.64853172)(565.25140183,280.79132712)(564.88681784,280.87032031)
\curveto(564.52223384,280.94931351)(564.12422965,280.98881011)(563.69280526,280.98881011)
\curveto(562.73273409,280.98881011)(561.9853369,280.65460812)(561.45061372,279.98620414)
\curveto(560.91589053,279.31780015)(560.64852894,278.36988177)(560.64852894,277.142449)
\curveto(560.64852894,275.93324543)(560.85512653,275.01267085)(561.26832172,274.38072527)
\curveto(561.68151691,273.75485608)(562.3438445,273.44192149)(563.25530447,273.44192149)
\curveto(563.74141646,273.44192149)(564.23360485,273.54825849)(564.73186964,273.76093248)
\curveto(565.23013443,273.97968288)(565.69497902,274.25919727)(566.12640341,274.59947566)
\closepath
}
}
{
\newrgbcolor{curcolor}{0 0 0}
\pscustom[linestyle=none,fillstyle=solid,fillcolor=curcolor]
{
\newpath
\moveto(578.48579753,275.02786185)
\curveto(578.48579753,274.09817267)(578.09994613,273.33558449)(577.32824335,272.74009731)
\curveto(576.56261697,272.14461012)(575.51443799,271.84686653)(574.18370643,271.84686653)
\curveto(573.43023284,271.84686653)(572.73752326,271.93497432)(572.10557767,272.11118992)
\curveto(571.47970849,272.29348192)(570.9540999,272.49096491)(570.52875191,272.70363891)
\lineto(570.52875191,274.62681946)
\lineto(570.61989791,274.62681946)
\curveto(571.1606975,274.21970067)(571.76226108,273.89461328)(572.42458867,273.65155728)
\curveto(573.08691625,273.41457769)(573.72190004,273.29608789)(574.32954002,273.29608789)
\curveto(575.08301361,273.29608789)(575.67242439,273.41761589)(576.09777238,273.66067188)
\curveto(576.52312037,273.90372788)(576.73579437,274.28654107)(576.73579437,274.80911146)
\curveto(576.73579437,275.21015385)(576.62034277,275.51397384)(576.38943957,275.72057144)
\curveto(576.15853638,275.92716903)(575.71495919,276.10338463)(575.05870801,276.24921822)
\curveto(574.81565201,276.30390582)(574.49664102,276.36770802)(574.10167503,276.44062482)
\curveto(573.71278544,276.51354162)(573.35731605,276.59253481)(573.03526685,276.67760441)
\curveto(572.14203607,276.91458401)(571.50705229,277.2609388)(571.1303155,277.71666879)
\curveto(570.75965511,278.17847518)(570.57432491,278.74358036)(570.57432491,279.41198435)
\curveto(570.57432491,279.83125594)(570.65939451,280.22622193)(570.8295337,280.59688232)
\curveto(571.0057493,280.96754271)(571.27007269,281.2987065)(571.62250389,281.5903737)
\curveto(571.96278228,281.87596449)(572.39420667,282.10079129)(572.91677706,282.26485408)
\curveto(573.44542384,282.43499328)(574.03483463,282.52006288)(574.68500941,282.52006288)
\curveto(575.2926494,282.52006288)(575.90636579,282.44410788)(576.52615857,282.29219788)
\curveto(577.15202776,282.14636428)(577.67155994,281.96711049)(578.08475513,281.75443649)
\lineto(578.08475513,279.92240194)
\lineto(577.99360914,279.92240194)
\curveto(577.55610835,280.24445113)(577.02442336,280.51485092)(576.39855417,280.73360132)
\curveto(575.77268499,280.95842811)(575.1589686,281.07084151)(574.55740502,281.07084151)
\curveto(573.93153583,281.07084151)(573.40288904,280.94931351)(572.97146465,280.70625752)
\curveto(572.54004026,280.46927792)(572.32432807,280.11380853)(572.32432807,279.63984934)
\curveto(572.32432807,279.22057775)(572.45497067,278.90460496)(572.71625586,278.69193097)
\curveto(572.97146465,278.47925697)(573.38465984,278.30607957)(573.95584143,278.17239878)
\curveto(574.27181422,278.09948198)(574.62424542,278.02656518)(575.01313501,277.95364838)
\curveto(575.408101,277.88073158)(575.73622659,277.81389119)(575.99751178,277.75312719)
\curveto(576.79352016,277.57083519)(577.40723655,277.2579006)(577.83866094,276.81432341)
\curveto(578.27008533,276.36466982)(578.48579753,275.76918263)(578.48579753,275.02786185)
\closepath
}
}
{
\newrgbcolor{curcolor}{0 0 0}
\pscustom[linestyle=none,fillstyle=solid,fillcolor=curcolor]
{
\newpath
\moveto(588.71238128,278.62812877)
\curveto(589.00404847,278.36684357)(589.24406626,278.03871798)(589.43243466,277.64375199)
\curveto(589.62080305,277.248786)(589.71498725,276.73836841)(589.71498725,276.11249923)
\curveto(589.71498725,275.49270644)(589.60257385,274.92456305)(589.37774706,274.40806907)
\curveto(589.15292027,273.89157508)(588.83694747,273.44192149)(588.42982868,273.0591083)
\curveto(587.97409869,272.63376031)(587.43633731,272.31778751)(586.81654452,272.11118992)
\curveto(586.20282813,271.91066872)(585.52834775,271.81040813)(584.79310337,271.81040813)
\curveto(584.03962978,271.81040813)(583.298309,271.90155412)(582.56914102,272.08384612)
\curveto(581.83997304,272.26006172)(581.24144765,272.45450651)(580.77356486,272.66718051)
\lineto(580.77356486,274.57213186)
\lineto(580.91028386,274.57213186)
\curveto(581.42677785,274.23185347)(582.03441783,273.94930088)(582.73320382,273.72447408)
\curveto(583.4319898,273.49964729)(584.10647018,273.38723389)(584.75664497,273.38723389)
\curveto(585.13945816,273.38723389)(585.54657695,273.45103609)(585.97800134,273.57864049)
\curveto(586.40942573,273.70624488)(586.75881872,273.89461328)(587.02618031,274.14374567)
\curveto(587.30569471,274.41110727)(587.5122923,274.70581266)(587.6459731,275.02786185)
\curveto(587.7857303,275.34991104)(587.8556089,275.75702983)(587.8556089,276.24921822)
\curveto(587.8556089,276.73533021)(587.7766157,277.1363726)(587.6186293,277.45234539)
\curveto(587.4667193,277.77439459)(587.25404531,278.02656518)(586.98060732,278.20885718)
\curveto(586.70716932,278.39722557)(586.37600553,278.52482997)(585.98711594,278.59167037)
\curveto(585.59822635,278.66458717)(585.17895476,278.70104557)(584.72930117,278.70104557)
\lineto(583.90898719,278.70104557)
\lineto(583.90898719,280.21406913)
\lineto(584.54700917,280.21406913)
\curveto(585.47062195,280.21406913)(586.20586633,280.40547573)(586.75274232,280.78828892)
\curveto(587.30569471,281.17717851)(587.5821709,281.74228369)(587.5821709,282.48360448)
\curveto(587.5821709,282.81173007)(587.5122923,283.09732086)(587.37253511,283.34037686)
\curveto(587.23277791,283.58950925)(587.03833311,283.79306865)(586.78920072,283.95105504)
\curveto(586.52791553,284.10904144)(586.24840113,284.21841664)(585.95065754,284.27918063)
\curveto(585.65291395,284.33994463)(585.31567375,284.37032663)(584.93893696,284.37032663)
\curveto(584.36167898,284.37032663)(583.74796259,284.26702784)(583.09778781,284.06043024)
\curveto(582.44761302,283.85383264)(581.83389664,283.56216545)(581.25663865,283.18542866)
\lineto(581.16549265,283.18542866)
\lineto(581.16549265,285.09038002)
\curveto(581.59691704,285.30305401)(582.17113683,285.49749881)(582.88815201,285.6737144)
\curveto(583.61124359,285.8560064)(584.31002958,285.9471524)(584.98450996,285.9471524)
\curveto(585.64683755,285.9471524)(586.23017193,285.8863884)(586.73451312,285.7648604)
\curveto(587.23885431,285.6433324)(587.6945843,285.44888761)(588.10170309,285.18152601)
\curveto(588.53920388,284.88985882)(588.87036767,284.53742763)(589.09519447,284.12423244)
\curveto(589.32002126,283.71103725)(589.43243466,283.22796346)(589.43243466,282.67501107)
\curveto(589.43243466,281.92153749)(589.16507306,281.26224811)(588.63034988,280.69714292)
\curveto(588.10170309,280.13811413)(587.4758339,279.78568294)(586.75274232,279.63984934)
\lineto(586.75274232,279.51224495)
\curveto(587.04440951,279.46363375)(587.37861151,279.36033495)(587.7553483,279.20234855)
\curveto(588.13208509,279.05043856)(588.45109608,278.85903196)(588.71238128,278.62812877)
\closepath
}
}
{
\newrgbcolor{curcolor}{0 0 0}
\pscustom[linestyle=none,fillstyle=solid,fillcolor=curcolor]
{
\newpath
\moveto(596.45067796,279.67630774)
\lineto(594.27228861,279.67630774)
\lineto(594.27228861,282.27396868)
\lineto(596.45067796,282.27396868)
\closepath
\moveto(596.45067796,272.09296072)
\lineto(594.27228861,272.09296072)
\lineto(594.27228861,274.69062166)
\lineto(596.45067796,274.69062166)
\closepath
}
}
{
\newrgbcolor{curcolor}{0 0 0}
\pscustom[linestyle=none,fillstyle=solid,fillcolor=curcolor]
{
\newpath
\moveto(604.92725572,279.67630774)
\lineto(602.74886637,279.67630774)
\lineto(602.74886637,282.27396868)
\lineto(604.92725572,282.27396868)
\closepath
\moveto(604.92725572,272.09296072)
\lineto(602.74886637,272.09296072)
\lineto(602.74886637,274.69062166)
\lineto(604.92725572,274.69062166)
\closepath
}
}
{
\newrgbcolor{curcolor}{0 0 0}
\pscustom[linestyle=none,fillstyle=solid,fillcolor=curcolor]
{
\newpath
\moveto(618.71764515,277.3065118)
\curveto(618.71764515,276.48012142)(618.59915536,275.72360964)(618.36217576,275.03697645)
\curveto(618.12519617,274.35641967)(617.79099418,273.77916168)(617.35956979,273.30520249)
\curveto(616.95852739,272.8555489)(616.48456821,272.50615591)(615.93769222,272.25702352)
\curveto(615.39689263,272.01396752)(614.82267284,271.89243952)(614.21503286,271.89243952)
\curveto(613.68638607,271.89243952)(613.20635048,271.95016532)(612.77492609,272.06561692)
\curveto(612.3495781,272.18106852)(611.91511551,272.36032231)(611.47153832,272.60337831)
\lineto(611.47153832,268.33774561)
\lineto(609.75799356,268.33774561)
\lineto(609.75799356,282.27396868)
\lineto(611.47153832,282.27396868)
\lineto(611.47153832,281.20756051)
\curveto(611.92726831,281.5903737)(612.4376859,281.90938469)(613.00279109,282.16459348)
\curveto(613.57397267,282.42587868)(614.18161266,282.55652128)(614.82571104,282.55652128)
\curveto(616.05314382,282.55652128)(617.00713859,282.09167669)(617.68769538,281.16198751)
\curveto(618.37432856,280.23837473)(618.71764515,278.95321616)(618.71764515,277.3065118)
\closepath
\moveto(616.94941279,277.2609388)
\curveto(616.94941279,278.48837157)(616.739777,279.40590795)(616.32050541,280.01354793)
\curveto(615.90123382,280.62118792)(615.25713543,280.92500791)(614.38821025,280.92500791)
\curveto(613.89602187,280.92500791)(613.40079528,280.81867092)(612.90253049,280.60599692)
\curveto(612.4042657,280.39332293)(611.92726831,280.11380853)(611.47153832,279.76745374)
\lineto(611.47153832,273.99791208)
\curveto(611.95765031,273.77916168)(612.3738837,273.63028988)(612.72023849,273.55129669)
\curveto(613.07266969,273.47230349)(613.47067388,273.43280689)(613.91425107,273.43280689)
\curveto(614.86824584,273.43280689)(615.61260483,273.75485608)(616.14732801,274.39895447)
\curveto(616.6820512,275.04305285)(616.94941279,275.99704763)(616.94941279,277.2609388)
\closepath
}
}
{
\newrgbcolor{curcolor}{0 0 0}
\pscustom[linestyle=none,fillstyle=solid,fillcolor=curcolor]
{
\newpath
\moveto(633.61089801,272.09296072)
\lineto(631.80620725,272.09296072)
\lineto(631.80620725,283.78699225)
\lineto(628.03276294,275.82994663)
\lineto(626.95724017,275.82994663)
\lineto(623.21113966,283.78699225)
\lineto(623.21113966,272.09296072)
\lineto(621.5249387,272.09296072)
\lineto(621.5249387,285.6645998)
\lineto(623.98588064,285.6645998)
\lineto(627.60437675,278.10859658)
\lineto(631.10438307,285.6645998)
\lineto(633.61089801,285.6645998)
\closepath
}
}
{
\newrgbcolor{curcolor}{0 0 0}
\pscustom[linestyle=none,fillstyle=solid,fillcolor=curcolor]
{
\newpath
\moveto(645.03148858,272.09296072)
\lineto(643.32705842,272.09296072)
\lineto(643.32705842,273.17759809)
\curveto(643.17514842,273.0742993)(642.96855083,272.9284657)(642.70726563,272.74009731)
\curveto(642.45205684,272.55780531)(642.20292444,272.41197171)(641.95986845,272.30259652)
\curveto(641.67427766,272.16283932)(641.34615206,272.04738772)(640.97549167,271.95624172)
\curveto(640.60483128,271.85901933)(640.17036869,271.81040813)(639.6721039,271.81040813)
\curveto(638.75456752,271.81040813)(637.97678834,272.11422812)(637.33876636,272.72186811)
\curveto(636.70074437,273.32950809)(636.38173338,274.10424907)(636.38173338,275.04609105)
\curveto(636.38173338,275.81779383)(636.54579618,276.44062482)(636.87392177,276.91458401)
\curveto(637.20812376,277.3946196)(637.68208295,277.77135639)(638.29579934,278.04479438)
\curveto(638.91559212,278.31823237)(639.6599511,278.50356257)(640.52887628,278.60078497)
\curveto(641.39780146,278.69800737)(642.33052884,278.77092416)(643.32705842,278.81953536)
\lineto(643.32705842,279.08385876)
\curveto(643.32705842,279.47274835)(643.25717982,279.79479754)(643.11742262,280.05000633)
\curveto(642.98374183,280.30521513)(642.78929703,280.50573632)(642.53408824,280.65156992)
\curveto(642.29103224,280.79132712)(641.99936505,280.88551131)(641.65908666,280.93412251)
\curveto(641.31880826,280.98273371)(640.96333887,281.00703931)(640.59267848,281.00703931)
\curveto(640.14302489,281.00703931)(639.6417219,280.94627531)(639.08876952,280.82474732)
\curveto(638.53581713,280.70929572)(637.96463554,280.53915652)(637.37522476,280.31432973)
\lineto(637.28407876,280.31432973)
\lineto(637.28407876,282.05521829)
\curveto(637.61828075,282.14636428)(638.10135454,282.24662488)(638.73330013,282.35600008)
\curveto(639.36524571,282.46537528)(639.9880767,282.52006288)(640.60179308,282.52006288)
\curveto(641.31880826,282.52006288)(641.94163925,282.45929888)(642.47028604,282.33777088)
\curveto(643.00500923,282.22231928)(643.46681561,282.02179809)(643.85570521,281.73620729)
\curveto(644.2385184,281.4566929)(644.53018559,281.09514711)(644.73070678,280.65156992)
\curveto(644.93122798,280.20799273)(645.03148858,279.65807854)(645.03148858,279.00182736)
\closepath
\moveto(643.32705842,274.59947566)
\lineto(643.32705842,277.4341162)
\curveto(642.80448803,277.4037342)(642.18773344,277.3581612)(641.47679466,277.2973972)
\curveto(640.77193228,277.2366332)(640.21290349,277.1485254)(639.7997083,277.0330738)
\curveto(639.30751991,276.89331661)(638.90951572,276.67456621)(638.60569573,276.37682262)
\curveto(638.30187574,276.08515543)(638.14996574,275.68107484)(638.14996574,275.16458085)
\curveto(638.14996574,274.58124646)(638.32618133,274.14070747)(638.67861253,273.84296388)
\curveto(639.03104372,273.55129669)(639.56880511,273.40546309)(640.29189669,273.40546309)
\curveto(640.89346027,273.40546309)(641.44337446,273.52091469)(641.94163925,273.75181788)
\curveto(642.43990404,273.98879748)(642.90171043,274.27135007)(643.32705842,274.59947566)
\closepath
}
}
{
\newrgbcolor{curcolor}{0 0 0}
\pscustom[linestyle=none,fillstyle=solid,fillcolor=curcolor]
{
\newpath
\moveto(657.14479434,272.09296072)
\lineto(654.9846342,272.09296072)
\lineto(652.09530606,276.00312403)
\lineto(649.18774873,272.09296072)
\lineto(647.19165138,272.09296072)
\lineto(651.16561688,277.1697928)
\lineto(647.22810978,282.27396868)
\lineto(649.38826993,282.27396868)
\lineto(652.25936886,278.42760757)
\lineto(655.13958239,282.27396868)
\lineto(657.14479434,282.27396868)
\lineto(653.14348504,277.2609388)
\closepath
}
}
{
\newrgbcolor{curcolor}{0 0 0}
\pscustom[linestyle=none,fillstyle=solid,fillcolor=curcolor]
{
\newpath
\moveto(22.26811677,15.48351442)
\curveto(22.26811677,14.80903404)(22.14051237,14.21354685)(21.88530358,13.69705286)
\curveto(21.63009478,13.18055888)(21.28677819,12.75521089)(20.8553538,12.42100889)
\curveto(20.34493621,12.0199665)(19.78286923,11.73437571)(19.16915284,11.56423651)
\curveto(18.56151285,11.39409732)(17.78677187,11.30902772)(16.84492989,11.30902772)
\lineto(12.03242121,11.30902772)
\lineto(12.03242121,24.8806668)
\lineto(16.05195971,24.8806668)
\curveto(17.04241289,24.8806668)(17.78373367,24.8442084)(18.27592206,24.77129161)
\curveto(18.76811045,24.69837481)(19.23903144,24.54646481)(19.68868503,24.31556162)
\curveto(20.18694982,24.05427642)(20.54849561,23.71703623)(20.7733224,23.30384104)
\curveto(20.9981492,22.89672225)(21.11056259,22.40757206)(21.11056259,21.83639047)
\curveto(21.11056259,21.19229209)(20.9464998,20.6423779)(20.61837421,20.18664791)
\curveto(20.29024861,19.73699432)(19.85274782,19.37544853)(19.30587184,19.10201054)
\lineto(19.30587184,19.02909374)
\curveto(20.22340822,18.84072534)(20.9464998,18.43664475)(21.47514659,17.81685197)
\curveto(22.00379337,17.20313558)(22.26811677,16.4253564)(22.26811677,15.48351442)
\closepath
\moveto(19.23295504,21.59941088)
\curveto(19.23295504,21.92753647)(19.17826744,22.20401267)(19.06889224,22.42883946)
\curveto(18.95951705,22.65366625)(18.78330145,22.83595825)(18.54024546,22.97571545)
\curveto(18.25465466,23.13977824)(17.90829987,23.24003884)(17.50118108,23.27649724)
\curveto(17.09406229,23.31903204)(16.5897211,23.34029944)(15.98815751,23.34029944)
\lineto(13.83711197,23.34029944)
\lineto(13.83711197,19.42102153)
\lineto(16.17044951,19.42102153)
\curveto(16.7355547,19.42102153)(17.18520829,19.44836533)(17.51941028,19.50305293)
\curveto(17.85361227,19.56381693)(18.16350866,19.68534492)(18.44909946,19.86763692)
\curveto(18.73469025,20.04992892)(18.93521145,20.28387031)(19.05066304,20.5694611)
\curveto(19.17219104,20.8611283)(19.23295504,21.20444489)(19.23295504,21.59941088)
\closepath
\moveto(20.39050921,15.41059762)
\curveto(20.39050921,15.95747361)(20.30847781,16.3919362)(20.14441502,16.71398539)
\curveto(19.98035222,17.03603459)(19.68260863,17.30947258)(19.25118424,17.53429937)
\curveto(18.95951705,17.68620937)(18.60404765,17.78343177)(18.18477606,17.82596657)
\curveto(17.77158087,17.87457777)(17.26723968,17.89888337)(16.6717525,17.89888337)
\lineto(13.83711197,17.89888337)
\lineto(13.83711197,12.84939508)
\lineto(16.22513711,12.84939508)
\curveto(17.01506909,12.84939508)(17.66220568,12.88889168)(18.16654686,12.96788488)
\curveto(18.67088805,13.05295448)(19.08408324,13.20486448)(19.40613243,13.42361487)
\curveto(19.74641083,13.66059447)(19.99554322,13.93099426)(20.15352962,14.23481425)
\curveto(20.31151601,14.53863424)(20.39050921,14.93056204)(20.39050921,15.41059762)
\closepath
}
}
{
\newrgbcolor{curcolor}{0 0 0}
\pscustom[linestyle=none,fillstyle=solid,fillcolor=curcolor]
{
\newpath
\moveto(33.3696991,16.3949744)
\curveto(33.3696991,14.73611724)(32.94435111,13.42665307)(32.09365513,12.46658189)
\curveto(31.24295915,11.50651072)(30.10363417,11.02647513)(28.67568021,11.02647513)
\curveto(27.23557344,11.02647513)(26.09017207,11.50651072)(25.23947609,12.46658189)
\curveto(24.39485651,13.42665307)(23.97254672,14.73611724)(23.97254672,16.3949744)
\curveto(23.97254672,18.05383156)(24.39485651,19.36329573)(25.23947609,20.32336691)
\curveto(26.09017207,21.28951449)(27.23557344,21.77258828)(28.67568021,21.77258828)
\curveto(30.10363417,21.77258828)(31.24295915,21.28951449)(32.09365513,20.32336691)
\curveto(32.94435111,19.36329573)(33.3696991,18.05383156)(33.3696991,16.3949744)
\closepath
\moveto(31.60146674,16.3949744)
\curveto(31.60146674,17.71355317)(31.34321974,18.69185355)(30.82672576,19.32987553)
\curveto(30.31023177,19.97397392)(29.59321658,20.29602311)(28.67568021,20.29602311)
\curveto(27.74599103,20.29602311)(27.02289944,19.97397392)(26.50640546,19.32987553)
\curveto(25.99598787,18.69185355)(25.74077907,17.71355317)(25.74077907,16.3949744)
\curveto(25.74077907,15.11893043)(25.99902607,14.14974465)(26.51552006,13.48741707)
\curveto(27.03201404,12.83116588)(27.75206743,12.50304029)(28.67568021,12.50304029)
\curveto(29.58714018,12.50304029)(30.30111717,12.82812768)(30.81761116,13.47830247)
\curveto(31.34018154,14.13455365)(31.60146674,15.10677763)(31.60146674,16.3949744)
\closepath
}
}
{
\newrgbcolor{curcolor}{0 0 0}
\pscustom[linestyle=none,fillstyle=solid,fillcolor=curcolor]
{
\newpath
\moveto(44.4621682,11.30902772)
\lineto(42.74862344,11.30902772)
\lineto(42.74862344,12.43923809)
\curveto(42.17136545,11.9835081)(41.61841307,11.63411511)(41.08976628,11.39105912)
\curveto(40.56111949,11.14800312)(39.97778511,11.02647513)(39.33976312,11.02647513)
\curveto(38.27031675,11.02647513)(37.43784997,11.35156252)(36.84236278,12.0017373)
\curveto(36.24687559,12.65798849)(35.949132,13.61805967)(35.949132,14.88195084)
\lineto(35.949132,21.49003568)
\lineto(37.66267676,21.49003568)
\lineto(37.66267676,15.69315022)
\curveto(37.66267676,15.17665623)(37.68698236,14.73307904)(37.73559356,14.36241865)
\curveto(37.78420476,13.99783466)(37.88750356,13.68490006)(38.04548995,13.42361487)
\curveto(38.20955275,13.15625328)(38.42222674,12.96180848)(38.68351194,12.84028048)
\curveto(38.94479713,12.71875249)(39.32457212,12.65798849)(39.82283691,12.65798849)
\curveto(40.2664141,12.65798849)(40.74948789,12.77344009)(41.27205828,13.00434328)
\curveto(41.80070506,13.23524648)(42.29289345,13.52995187)(42.74862344,13.88845946)
\lineto(42.74862344,21.49003568)
\lineto(44.4621682,21.49003568)
\closepath
}
}
{
\newrgbcolor{curcolor}{0 0 0}
\pscustom[linestyle=none,fillstyle=solid,fillcolor=curcolor]
{
\newpath
\moveto(56.34760407,11.30902772)
\lineto(54.63405931,11.30902772)
\lineto(54.63405931,17.10591318)
\curveto(54.63405931,17.57379597)(54.60671551,18.01129676)(54.55202791,18.41841555)
\curveto(54.49734032,18.83161074)(54.39707972,19.15365994)(54.25124612,19.38456313)
\curveto(54.09933613,19.63977193)(53.88058573,19.82814032)(53.59499494,19.94966832)
\curveto(53.30940414,20.07727271)(52.93874375,20.14107491)(52.48301376,20.14107491)
\curveto(52.01513097,20.14107491)(51.52598079,20.02562332)(51.0155632,19.79472012)
\curveto(50.50514561,19.56381693)(50.01599542,19.26911153)(49.54811263,18.91060394)
\lineto(49.54811263,11.30902772)
\lineto(47.83456787,11.30902772)
\lineto(47.83456787,21.49003568)
\lineto(49.54811263,21.49003568)
\lineto(49.54811263,20.35982531)
\curveto(50.08283582,20.8034025)(50.63578821,21.14975729)(51.20696979,21.39888968)
\curveto(51.77815138,21.64802208)(52.36452397,21.77258828)(52.96608755,21.77258828)
\curveto(54.06591593,21.77258828)(54.90445911,21.44142448)(55.48171709,20.7790969)
\curveto(56.05897508,20.11676931)(56.34760407,19.16277454)(56.34760407,17.91711257)
\closepath
}
}
{
\newrgbcolor{curcolor}{0 0 0}
\pscustom[linestyle=none,fillstyle=solid,fillcolor=curcolor]
{
\newpath
\moveto(67.90491817,11.30902772)
\lineto(66.19137341,11.30902772)
\lineto(66.19137341,12.3754359)
\curveto(65.69918502,11.95008791)(65.18572923,11.61892411)(64.65100604,11.38194452)
\curveto(64.11628286,11.14496492)(63.53598667,11.02647513)(62.91011748,11.02647513)
\curveto(61.69483751,11.02647513)(60.72868994,11.49435792)(60.01167475,12.43012349)
\curveto(59.30073597,13.36588907)(58.94526658,14.66320044)(58.94526658,16.3220576)
\curveto(58.94526658,17.18490638)(59.06679457,17.95357096)(59.30985057,18.62805135)
\curveto(59.55898296,19.30253173)(59.89318495,19.87675152)(60.31245654,20.35071071)
\curveto(60.72565174,20.8125171)(61.20568732,21.16494829)(61.75256331,21.40800428)
\curveto(62.3055157,21.65106028)(62.87669728,21.77258828)(63.46610807,21.77258828)
\curveto(64.00083126,21.77258828)(64.47479045,21.71486248)(64.88798564,21.59941088)
\curveto(65.30118083,21.49003568)(65.73564342,21.31685829)(66.19137341,21.07987869)
\lineto(66.19137341,25.49134499)
\lineto(67.90491817,25.49134499)
\closepath
\moveto(66.19137341,13.81554266)
\lineto(66.19137341,19.65800112)
\curveto(65.72956702,19.86459872)(65.31637183,20.00739412)(64.95178784,20.08638731)
\curveto(64.58720384,20.16538051)(64.18919965,20.20487711)(63.75777526,20.20487711)
\curveto(62.79770409,20.20487711)(62.0503069,19.87067512)(61.51558372,19.20227114)
\curveto(60.98086053,18.53386715)(60.71349894,17.58594877)(60.71349894,16.358516)
\curveto(60.71349894,15.14931243)(60.92009653,14.22873785)(61.33329172,13.59679227)
\curveto(61.74648691,12.97092308)(62.4088145,12.65798849)(63.32027447,12.65798849)
\curveto(63.80638646,12.65798849)(64.29857485,12.76432549)(64.79683964,12.97699948)
\curveto(65.29510443,13.19574988)(65.75994902,13.47526427)(66.19137341,13.81554266)
\closepath
}
}
{
\newrgbcolor{curcolor}{0 0 0}
\pscustom[linestyle=none,fillstyle=solid,fillcolor=curcolor]
{
\newpath
\moveto(78.55076753,14.24392885)
\curveto(78.55076753,13.31423967)(78.16491613,12.55165149)(77.39321335,11.95616431)
\curveto(76.62758697,11.36067712)(75.57940799,11.06293353)(74.24867643,11.06293353)
\curveto(73.49520284,11.06293353)(72.80249326,11.15104132)(72.17054767,11.32725692)
\curveto(71.54467849,11.50954892)(71.0190699,11.70703191)(70.59372191,11.91970591)
\lineto(70.59372191,13.84288646)
\lineto(70.68486791,13.84288646)
\curveto(71.2256675,13.43576767)(71.82723108,13.11068028)(72.48955867,12.86762428)
\curveto(73.15188625,12.63064469)(73.78687004,12.51215489)(74.39451002,12.51215489)
\curveto(75.14798361,12.51215489)(75.73739439,12.63368289)(76.16274238,12.87673888)
\curveto(76.58809037,13.11979488)(76.80076437,13.50260807)(76.80076437,14.02517846)
\curveto(76.80076437,14.42622085)(76.68531277,14.73004084)(76.45440957,14.93663844)
\curveto(76.22350638,15.14323603)(75.77992919,15.31945163)(75.12367801,15.46528522)
\curveto(74.88062201,15.51997282)(74.56161102,15.58377502)(74.16664503,15.65669182)
\curveto(73.77775544,15.72960862)(73.42228605,15.80860181)(73.10023685,15.89367141)
\curveto(72.20700607,16.13065101)(71.57202229,16.4770058)(71.1952855,16.93273579)
\curveto(70.82462511,17.39454218)(70.63929491,17.95964736)(70.63929491,18.62805135)
\curveto(70.63929491,19.04732294)(70.72436451,19.44228893)(70.8945037,19.81294932)
\curveto(71.0707193,20.18360971)(71.33504269,20.5147735)(71.68747389,20.8064407)
\curveto(72.02775228,21.09203149)(72.45917667,21.31685829)(72.98174706,21.48092108)
\curveto(73.51039384,21.65106028)(74.09980463,21.73612988)(74.74997941,21.73612988)
\curveto(75.3576194,21.73612988)(75.97133579,21.66017488)(76.59112857,21.50826488)
\curveto(77.21699776,21.36243128)(77.73652994,21.18317749)(78.14972513,20.97050349)
\lineto(78.14972513,19.13846894)
\lineto(78.05857914,19.13846894)
\curveto(77.62107835,19.46051813)(77.08939336,19.73091792)(76.46352417,19.94966832)
\curveto(75.83765499,20.17449511)(75.2239386,20.28690851)(74.62237502,20.28690851)
\curveto(73.99650583,20.28690851)(73.46785904,20.16538051)(73.03643465,19.92232452)
\curveto(72.60501026,19.68534492)(72.38929807,19.32987553)(72.38929807,18.85591634)
\curveto(72.38929807,18.43664475)(72.51994067,18.12067196)(72.78122586,17.90799797)
\curveto(73.03643465,17.69532397)(73.44962984,17.52214657)(74.02081143,17.38846578)
\curveto(74.33678422,17.31554898)(74.68921542,17.24263218)(75.07810501,17.16971538)
\curveto(75.473071,17.09679858)(75.80119659,17.02995819)(76.06248178,16.96919419)
\curveto(76.85849016,16.78690219)(77.47220655,16.4739676)(77.90363094,16.03039041)
\curveto(78.33505533,15.58073682)(78.55076753,14.98524963)(78.55076753,14.24392885)
\closepath
}
}
{
\newrgbcolor{curcolor}{0 0 0}
\pscustom[linestyle=none,fillstyle=solid,fillcolor=curcolor]
{
\newpath
\moveto(88.77735128,17.84419577)
\curveto(89.06901847,17.58291057)(89.30903626,17.25478498)(89.49740466,16.85981899)
\curveto(89.68577305,16.464853)(89.77995725,15.95443541)(89.77995725,15.32856623)
\curveto(89.77995725,14.70877344)(89.66754385,14.14063005)(89.44271706,13.62413607)
\curveto(89.21789027,13.10764208)(88.90191747,12.65798849)(88.49479868,12.2751753)
\curveto(88.03906869,11.84982731)(87.50130731,11.53385451)(86.88151452,11.32725692)
\curveto(86.26779813,11.12673572)(85.59331775,11.02647513)(84.85807337,11.02647513)
\curveto(84.10459978,11.02647513)(83.363279,11.11762112)(82.63411102,11.29991312)
\curveto(81.90494304,11.47612872)(81.30641765,11.67057351)(80.83853486,11.88324751)
\lineto(80.83853486,13.78819886)
\lineto(80.97525386,13.78819886)
\curveto(81.49174785,13.44792047)(82.09938783,13.16536788)(82.79817382,12.94054108)
\curveto(83.4969598,12.71571429)(84.17144018,12.60330089)(84.82161497,12.60330089)
\curveto(85.20442816,12.60330089)(85.61154695,12.66710309)(86.04297134,12.79470749)
\curveto(86.47439573,12.92231188)(86.82378872,13.11068028)(87.09115031,13.35981267)
\curveto(87.37066471,13.62717427)(87.5772623,13.92187966)(87.7109431,14.24392885)
\curveto(87.8507003,14.56597804)(87.9205789,14.97309683)(87.9205789,15.46528522)
\curveto(87.9205789,15.95139721)(87.8415857,16.3524396)(87.6835993,16.66841239)
\curveto(87.5316893,16.99046159)(87.31901531,17.24263218)(87.04557732,17.42492418)
\curveto(86.77213932,17.61329257)(86.44097553,17.74089697)(86.05208594,17.80773737)
\curveto(85.66319635,17.88065417)(85.24392476,17.91711257)(84.79427117,17.91711257)
\lineto(83.97395719,17.91711257)
\lineto(83.97395719,19.43013613)
\lineto(84.61197917,19.43013613)
\curveto(85.53559195,19.43013613)(86.27083633,19.62154273)(86.81771232,20.00435592)
\curveto(87.37066471,20.39324551)(87.6471409,20.95835069)(87.6471409,21.69967148)
\curveto(87.6471409,22.02779707)(87.5772623,22.31338786)(87.43750511,22.55644386)
\curveto(87.29774791,22.80557625)(87.10330311,23.00913565)(86.85417072,23.16712204)
\curveto(86.59288553,23.32510844)(86.31337113,23.43448364)(86.01562754,23.49524763)
\curveto(85.71788395,23.55601163)(85.38064375,23.58639363)(85.00390696,23.58639363)
\curveto(84.42664898,23.58639363)(83.81293259,23.48309484)(83.16275781,23.27649724)
\curveto(82.51258302,23.06989964)(81.89886664,22.77823245)(81.32160865,22.40149566)
\lineto(81.23046265,22.40149566)
\lineto(81.23046265,24.30644702)
\curveto(81.66188704,24.51912101)(82.23610683,24.71356581)(82.95312201,24.8897814)
\curveto(83.67621359,25.0720734)(84.37499958,25.1632194)(85.04947996,25.1632194)
\curveto(85.71180755,25.1632194)(86.29514193,25.1024554)(86.79948312,24.9809274)
\curveto(87.30382431,24.8593994)(87.7595543,24.66495461)(88.16667309,24.39759301)
\curveto(88.60417388,24.10592582)(88.93533767,23.75349463)(89.16016447,23.34029944)
\curveto(89.38499126,22.92710425)(89.49740466,22.44403046)(89.49740466,21.89107807)
\curveto(89.49740466,21.13760449)(89.23004306,20.47831511)(88.69531988,19.91320992)
\curveto(88.16667309,19.35418113)(87.5408039,19.00174994)(86.81771232,18.85591634)
\lineto(86.81771232,18.72831195)
\curveto(87.10937951,18.67970075)(87.44358151,18.57640195)(87.8203183,18.41841555)
\curveto(88.19705509,18.26650556)(88.51606608,18.07509896)(88.77735128,17.84419577)
\closepath
}
}
{
\newrgbcolor{curcolor}{0 0 0}
\pscustom[linestyle=none,fillstyle=solid,fillcolor=curcolor]
{
\newpath
\moveto(96.51564796,18.89237474)
\lineto(94.33725861,18.89237474)
\lineto(94.33725861,21.49003568)
\lineto(96.51564796,21.49003568)
\closepath
\moveto(96.51564796,11.30902772)
\lineto(94.33725861,11.30902772)
\lineto(94.33725861,13.90668866)
\lineto(96.51564796,13.90668866)
\closepath
}
}
{
\newrgbcolor{curcolor}{0 0 0}
\pscustom[linestyle=none,fillstyle=solid,fillcolor=curcolor]
{
\newpath
\moveto(104.99222572,18.89237474)
\lineto(102.81383637,18.89237474)
\lineto(102.81383637,21.49003568)
\lineto(104.99222572,21.49003568)
\closepath
\moveto(104.99222572,11.30902772)
\lineto(102.81383637,11.30902772)
\lineto(102.81383637,13.90668866)
\lineto(104.99222572,13.90668866)
\closepath
}
}
{
\newrgbcolor{curcolor}{0 0 0}
\pscustom[linestyle=none,fillstyle=solid,fillcolor=curcolor]
{
\newpath
\moveto(118.78261515,16.5225788)
\curveto(118.78261515,15.69618842)(118.66412536,14.93967664)(118.42714576,14.25304345)
\curveto(118.19016617,13.57248667)(117.85596418,12.99522868)(117.42453979,12.52126949)
\curveto(117.02349739,12.0716159)(116.54953821,11.72222291)(116.00266222,11.47309052)
\curveto(115.46186263,11.23003452)(114.88764284,11.10850652)(114.28000286,11.10850652)
\curveto(113.75135607,11.10850652)(113.27132048,11.16623232)(112.83989609,11.28168392)
\curveto(112.4145481,11.39713552)(111.98008551,11.57638931)(111.53650832,11.81944531)
\lineto(111.53650832,7.55381261)
\lineto(109.82296356,7.55381261)
\lineto(109.82296356,21.49003568)
\lineto(111.53650832,21.49003568)
\lineto(111.53650832,20.42362751)
\curveto(111.99223831,20.8064407)(112.5026559,21.12545169)(113.06776109,21.38066048)
\curveto(113.63894267,21.64194568)(114.24658266,21.77258828)(114.89068104,21.77258828)
\curveto(116.11811382,21.77258828)(117.07210859,21.30774369)(117.75266538,20.37805451)
\curveto(118.43929856,19.45444173)(118.78261515,18.16928316)(118.78261515,16.5225788)
\closepath
\moveto(117.01438279,16.4770058)
\curveto(117.01438279,17.70443857)(116.804747,18.62197495)(116.38547541,19.22961493)
\curveto(115.96620382,19.83725492)(115.32210543,20.14107491)(114.45318025,20.14107491)
\curveto(113.96099187,20.14107491)(113.46576528,20.03473792)(112.96750049,19.82206392)
\curveto(112.4692357,19.60938993)(111.99223831,19.32987553)(111.53650832,18.98352074)
\lineto(111.53650832,13.21397908)
\curveto(112.02262031,12.99522868)(112.4388537,12.84635688)(112.78520849,12.76736369)
\curveto(113.13763969,12.68837049)(113.53564388,12.64887389)(113.97922107,12.64887389)
\curveto(114.93321584,12.64887389)(115.67757483,12.97092308)(116.21229801,13.61502147)
\curveto(116.7470212,14.25911985)(117.01438279,15.21311463)(117.01438279,16.4770058)
\closepath
}
}
{
\newrgbcolor{curcolor}{0 0 0}
\pscustom[linestyle=none,fillstyle=solid,fillcolor=curcolor]
{
\newpath
\moveto(133.67586801,11.30902772)
\lineto(131.87117725,11.30902772)
\lineto(131.87117725,23.00305925)
\lineto(128.09773294,15.04601363)
\lineto(127.02221017,15.04601363)
\lineto(123.27610966,23.00305925)
\lineto(123.27610966,11.30902772)
\lineto(121.5899087,11.30902772)
\lineto(121.5899087,24.8806668)
\lineto(124.05085064,24.8806668)
\lineto(127.66934675,17.32466358)
\lineto(131.16935307,24.8806668)
\lineto(133.67586801,24.8806668)
\closepath
}
}
{
\newrgbcolor{curcolor}{0 0 0}
\pscustom[linestyle=none,fillstyle=solid,fillcolor=curcolor]
{
\newpath
\moveto(139.02613512,23.19446584)
\lineto(137.09383997,23.19446584)
\lineto(137.09383997,24.9718128)
\lineto(139.02613512,24.9718128)
\closepath
\moveto(138.91675992,11.30902772)
\lineto(137.20321516,11.30902772)
\lineto(137.20321516,21.49003568)
\lineto(138.91675992,21.49003568)
\closepath
}
}
{
\newrgbcolor{curcolor}{0 0 0}
\pscustom[linestyle=none,fillstyle=solid,fillcolor=curcolor]
{
\newpath
\moveto(150.8204358,11.30902772)
\lineto(149.10689104,11.30902772)
\lineto(149.10689104,17.10591318)
\curveto(149.10689104,17.57379597)(149.07954724,18.01129676)(149.02485964,18.41841555)
\curveto(148.97017204,18.83161074)(148.86991144,19.15365994)(148.72407785,19.38456313)
\curveto(148.57216785,19.63977193)(148.35341746,19.82814032)(148.06782666,19.94966832)
\curveto(147.78223587,20.07727271)(147.41157548,20.14107491)(146.95584549,20.14107491)
\curveto(146.4879627,20.14107491)(145.99881251,20.02562332)(145.48839492,19.79472012)
\curveto(144.97797734,19.56381693)(144.48882715,19.26911153)(144.02094436,18.91060394)
\lineto(144.02094436,11.30902772)
\lineto(142.3073996,11.30902772)
\lineto(142.3073996,21.49003568)
\lineto(144.02094436,21.49003568)
\lineto(144.02094436,20.35982531)
\curveto(144.55566755,20.8034025)(145.10861993,21.14975729)(145.67980152,21.39888968)
\curveto(146.25098311,21.64802208)(146.83735569,21.77258828)(147.43891928,21.77258828)
\curveto(148.53874765,21.77258828)(149.37729083,21.44142448)(149.95454882,20.7790969)
\curveto(150.53180681,20.11676931)(150.8204358,19.16277454)(150.8204358,17.91711257)
\closepath
}
}
\end{pspicture}

    \caption{轴对齐的边界框。
        类\protect\refvar{Bounds2}{}和\protect\refvar{Bounds3}{}只
        存储该框坐标最小和最大的点;框的其他顶点隐含在该表示中。}
    \label{fig:2.8}
\end{figure}

默认构造函数通过把范围设置为不可用的配置来创建一个空框,
违反了{\ttfamily pMin.x <= pMax.x}的性质(其他维度也这样)。
通过初始化具有最大和最小表示数字的两个顶点,
任何涉及空框的操作(例如\refvar[Union1]{Union}{()})都会产生正确结果。
\begin{lstlisting}
`\initcode{Bounds3 Public Methods}{=}\initnext{Bounds3PublicMethods}`
`\refvar{Bounds3}{}`() {
    T minNum = std::numeric_limits<T>::lowest();
    T maxNum = std::numeric_limits<T>::max();
    `\refvar{pMin}{}` = `\refvar{Point3}{}`<T>(maxNum, maxNum, maxNum);
    `\refvar{pMax}{}` = `\refvar{Point3}{}`<T>(minNum, minNum, minNum);
}
\end{lstlisting}

\begin{lstlisting}
`\initcode{Bounds3 Public Data}{=}`
`\refvar{Point3}{}`<T> `\initvar{pMin}{}`, `\initvar{pMax}{}`;
\end{lstlisting}

初始化\refvar{Bounds3}{}来包围单个点也很有用:
\begin{lstlisting}
`\refcode{Bounds3 Public Methods}{+=}\lastnext{Bounds3PublicMethods}`
`\refvar{Bounds3}{}`(const `\refvar{Point3}{}`<T> &p) : `\refvar{pMin}{}`(p), `\refvar{pMax}{}`(p) { }
\end{lstlisting}

如果调用者传入两个顶点({\ttfamily p1}和{\ttfamily p2})来定义框,
构造函数需要寻找它们逐元素最小和最大值,
因为所选{\ttfamily p1}和{\ttfamily p2}不需要满足{\ttfamily p1.x <= p2.x}等。
\begin{lstlisting}
`\refcode{Bounds3 Public Methods}{+=}\lastnext{Bounds3PublicMethods}`
`\refvar{Bounds3}{}`(const `\refvar{Point3}{}`<T> &p1, const `\refvar{Point3}{}`<T> &p2)
    : `\refvar{pMin}{}`(std::min(p1.x, p2.x), std::min(p1.y, p2.y),
           std::min(p1.z, p2.z)),
      `\refvar{pMax}{}`(std::max(p1.x, p2.x), std::max(p1.y, p2.y),
           std::max(p1.z, p2.z)) {
}
\end{lstlisting}

有时,在框的两个顶点之间使用数组索引来进行选择也很有用。
这些方法的实现基于{\ttfamily i}的值在\refvar{pMin}{}和\refvar{pMax}{}之间做选择。
\begin{lstlisting}
`\refcode{Bounds3 Public Methods}{+=}\lastnext{Bounds3PublicMethods}`
const `\refvar{Point3}{}`<T> &operator[](int i) const;
`\refvar{Point3}{}`<T> &operator[](int i);
\end{lstlisting}

方法\refvar{Corner}{()}返回边界框8个顶点之一的坐标。
\begin{lstlisting}
`\refcode{Bounds3 Public Methods}{+=}\lastnext{Bounds3PublicMethods}`
`\refvar{Point3}{}`<T> `\initvar{Corner}{}`(int corner) const {
    return `\refvar{Point3}{}`<T>((*this)[(corner & 1)].x,
                     (*this)[(corner & 2) ? 1 : 0].y,
                     (*this)[(corner & 4) ? 1 : 0].z);
}
\end{lstlisting}

给定一个边界框和一点,函数\refvar[Union1]{Union}{()}返回包含该点和原始框的新边界框。
\begin{lstlisting}
`\refcode{Geometry Inline Functions}{+=}\lastnext{GeometryInlineFunctions}`
template <typename T> `\refvar{Bounds3}{}` <T>
`\initvar[Union1]{Union}{}`(const `\refvar{Bounds3}{}`<T> &b, const `\refvar{Point3}{}`<T> &p) {
    return `\refvar{Bounds3}{}`<T>(`\refvar{Point3}{}`<T>(std::min(b.pMin.x, p.x),
                                std::min(b.pMin.y, p.y),
                                std::min(b.pMin.z, p.z)),
                      `\refvar{Point3}{}`<T>(std::max(b.pMax.x, p.x),
                                std::max(b.pMax.y, p.y),
                                std::max(b.pMax.z, p.z)));
}
\end{lstlisting}

同样也可以构造一个新框包围另两个边界框所围的空间。
该函数的定义与上述接收\refvar{Point3}{}的方法\refvar[Union1]{Union}{()}类似;
不同点是{\ttfamily std::min()}和{\ttfamily std::max()}测试
分别用的是第二个框的\refvar{pMin}{}和\refvar{pMax}{}。
\begin{lstlisting}
`\refcode{Geometry Inline Functions}{+=}\lastnext{GeometryInlineFunctions}`
template <typename T> `\refvar{Bounds3}{}`<T>
`\initvar[Union2]{Union}{}`(const `\refvar{Bounds3}{}`<T> &b1, const `\refvar{Bounds3}{}`<T> &b2) {
    return `\refvar{Bounds3}{}`<T>(`\refvar{Point3}{}`<T>(std::min(b1.pMin.x, b2.pMin.x),
                                std::min(b1.pMin.y, b2.pMin.y),
                                std::min(b1.pMin.z, b2.pMin.z)),
                      `\refvar{Point3}{}`<T>(std::max(b1.pMax.x, b2.pMax.x),
                                std::max(b1.pMax.y, b2.pMax.y),
                                std::max(b1.pMax.z, b2.pMax.z)));
}
\end{lstlisting}

两个边界框的交集可以通过计算它们最小坐标的最大值和最大坐标的最小值得到(见\reffig{2.9})。
\begin{figure}[htbp]
    \centering%LaTeX with PSTricks extensions
%%Creator: Inkscape 1.0.1 (3bc2e813f5, 2020-09-07)
%%Please note this file requires PSTricks extensions
\psset{xunit=.5pt,yunit=.5pt,runit=.5pt}
\begin{pspicture}(377.16000366,249.58000183)
{
\newrgbcolor{curcolor}{0 0.44313726 0.73725492}
\pscustom[linestyle=none,fillstyle=solid,fillcolor=curcolor]
{
\newpath
\moveto(74.56999969,142.76000214)
\lineto(205.8299942,142.76000214)
\lineto(205.8299942,104.88000107)
\lineto(74.56999969,104.88000107)
\closepath
}
}
{
\newrgbcolor{curcolor}{0 0 0}
\pscustom[linewidth=1,linecolor=curcolor]
{
\newpath
\moveto(21.25,234.8200016)
\lineto(205.67999268,234.8200016)
\lineto(205.67999268,104.87999916)
\lineto(21.25,104.87999916)
\closepath
}
}
{
\newrgbcolor{curcolor}{0 0 0}
\pscustom[linewidth=1,linecolor=curcolor]
{
\newpath
\moveto(74.75,142.61000061)
\lineto(366.97000122,142.61000061)
\lineto(366.97000122,9.88000488)
\lineto(74.75,9.88000488)
\closepath
}
}
{
\newrgbcolor{curcolor}{1 1 1}
\pscustom[linestyle=none,fillstyle=solid,fillcolor=curcolor]
{
\newpath
\moveto(27.61999941,104.88000488)
\curveto(27.61999941,110.42181826)(20.92019783,113.19616431)(17.00201901,109.27798549)
\curveto(13.08384019,105.35980667)(15.85818624,98.66000509)(21.39999962,98.66000509)
\curveto(26.941813,98.66000509)(29.71615905,105.35980667)(25.79798023,109.27798549)
\curveto(21.87980141,113.19616431)(15.17999983,110.42181826)(15.17999983,104.88000488)
\curveto(15.17999983,99.3381915)(21.87980141,96.56384546)(25.79798023,100.48202427)
\curveto(29.71615905,104.40020309)(26.941813,111.10000467)(21.39999962,111.10000467)
\curveto(15.85818624,111.10000467)(13.08384019,104.40020309)(17.00201901,100.48202427)
\curveto(20.92019783,96.56384546)(27.61999941,99.3381915)(27.61999941,104.88000488)
\closepath
}
}
{
\newrgbcolor{curcolor}{0 0 0}
\pscustom[linewidth=1,linecolor=curcolor]
{
\newpath
\moveto(27.61999941,104.88000488)
\curveto(27.61999941,110.42181826)(20.92019783,113.19616431)(17.00201901,109.27798549)
\curveto(13.08384019,105.35980667)(15.85818624,98.66000509)(21.39999962,98.66000509)
\curveto(26.941813,98.66000509)(29.71615905,105.35980667)(25.79798023,109.27798549)
\curveto(21.87980141,113.19616431)(15.17999983,110.42181826)(15.17999983,104.88000488)
\curveto(15.17999983,99.3381915)(21.87980141,96.56384546)(25.79798023,100.48202427)
\curveto(29.71615905,104.40020309)(26.941813,111.10000467)(21.39999962,111.10000467)
\curveto(15.85818624,111.10000467)(13.08384019,104.40020309)(17.00201901,100.48202427)
\curveto(20.92019783,96.56384546)(27.61999941,99.3381915)(27.61999941,104.88000488)
\closepath
}
}
{
\newrgbcolor{curcolor}{1 1 1}
\pscustom[linestyle=none,fillstyle=solid,fillcolor=curcolor]
{
\newpath
\moveto(81.09999704,9.80999756)
\curveto(81.09999704,15.35181094)(74.40019546,18.12615699)(70.48201665,14.20797817)
\curveto(66.56383783,10.28979935)(69.33818387,3.58999777)(74.87999725,3.58999777)
\curveto(80.42181064,3.58999777)(83.19615668,10.28979935)(79.27797786,14.20797817)
\curveto(75.35979904,18.12615699)(68.65999746,15.35181094)(68.65999746,9.80999756)
\curveto(68.65999746,4.26818418)(75.35979904,1.49383813)(79.27797786,5.41201695)
\curveto(83.19615668,9.33019577)(80.42181064,16.02999735)(74.87999725,16.02999735)
\curveto(69.33818387,16.02999735)(66.56383783,9.33019577)(70.48201665,5.41201695)
\curveto(74.40019546,1.49383813)(81.09999704,4.26818418)(81.09999704,9.80999756)
\closepath
}
}
{
\newrgbcolor{curcolor}{0 0 0}
\pscustom[linewidth=1,linecolor=curcolor]
{
\newpath
\moveto(81.09999704,9.80999756)
\curveto(81.09999704,15.35181094)(74.40019546,18.12615699)(70.48201665,14.20797817)
\curveto(66.56383783,10.28979935)(69.33818387,3.58999777)(74.87999725,3.58999777)
\curveto(80.42181064,3.58999777)(83.19615668,10.28979935)(79.27797786,14.20797817)
\curveto(75.35979904,18.12615699)(68.65999746,15.35181094)(68.65999746,9.80999756)
\curveto(68.65999746,4.26818418)(75.35979904,1.49383813)(79.27797786,5.41201695)
\curveto(83.19615668,9.33019577)(80.42181064,16.02999735)(74.87999725,16.02999735)
\curveto(69.33818387,16.02999735)(66.56383783,9.33019577)(70.48201665,5.41201695)
\curveto(74.40019546,1.49383813)(81.09999704,4.26818418)(81.09999704,9.80999756)
\closepath
}
}
{
\newrgbcolor{curcolor}{1 1 1}
\pscustom[linestyle=none,fillstyle=solid,fillcolor=curcolor]
{
\newpath
\moveto(212.10999918,234.67000198)
\curveto(212.10999918,240.21181537)(205.4101976,242.98616141)(201.49201878,239.06798259)
\curveto(197.57383996,235.14980377)(200.34818601,228.45000219)(205.88999939,228.45000219)
\curveto(211.43181277,228.45000219)(214.20615882,235.14980377)(210.28798,239.06798259)
\curveto(206.36980118,242.98616141)(199.6699996,240.21181537)(199.6699996,234.67000198)
\curveto(199.6699996,229.1281886)(206.36980118,226.35384256)(210.28798,230.27202138)
\curveto(214.20615882,234.19020019)(211.43181277,240.89000177)(205.88999939,240.89000177)
\curveto(200.34818601,240.89000177)(197.57383996,234.19020019)(201.49201878,230.27202138)
\curveto(205.4101976,226.35384256)(212.10999918,229.1281886)(212.10999918,234.67000198)
\closepath
}
}
{
\newrgbcolor{curcolor}{0 0 0}
\pscustom[linewidth=1,linecolor=curcolor]
{
\newpath
\moveto(212.10999918,234.67000198)
\curveto(212.10999918,240.21181537)(205.4101976,242.98616141)(201.49201878,239.06798259)
\curveto(197.57383996,235.14980377)(200.34818601,228.45000219)(205.88999939,228.45000219)
\curveto(211.43181277,228.45000219)(214.20615882,235.14980377)(210.28798,239.06798259)
\curveto(206.36980118,242.98616141)(199.6699996,240.21181537)(199.6699996,234.67000198)
\curveto(199.6699996,229.1281886)(206.36980118,226.35384256)(210.28798,230.27202138)
\curveto(214.20615882,234.19020019)(211.43181277,240.89000177)(205.88999939,240.89000177)
\curveto(200.34818601,240.89000177)(197.57383996,234.19020019)(201.49201878,230.27202138)
\curveto(205.4101976,226.35384256)(212.10999918,229.1281886)(212.10999918,234.67000198)
\closepath
}
}
{
\newrgbcolor{curcolor}{1 1 1}
\pscustom[linestyle=none,fillstyle=solid,fillcolor=curcolor]
{
\newpath
\moveto(373.30999613,142.52999878)
\curveto(373.30999613,148.07181216)(366.61019455,150.84615821)(362.69201573,146.92797939)
\curveto(358.77383691,143.00980057)(361.54818296,136.30999899)(367.08999634,136.30999899)
\curveto(372.63180972,136.30999899)(375.40615576,143.00980057)(371.48797695,146.92797939)
\curveto(367.56979813,150.84615821)(360.86999655,148.07181216)(360.86999655,142.52999878)
\curveto(360.86999655,136.9881854)(367.56979813,134.21383935)(371.48797695,138.13201817)
\curveto(375.40615576,142.05019699)(372.63180972,148.74999857)(367.08999634,148.74999857)
\curveto(361.54818296,148.74999857)(358.77383691,142.05019699)(362.69201573,138.13201817)
\curveto(366.61019455,134.21383935)(373.30999613,136.9881854)(373.30999613,142.52999878)
\closepath
}
}
{
\newrgbcolor{curcolor}{0 0 0}
\pscustom[linewidth=1,linecolor=curcolor]
{
\newpath
\moveto(373.30999613,142.52999878)
\curveto(373.30999613,148.07181216)(366.61019455,150.84615821)(362.69201573,146.92797939)
\curveto(358.77383691,143.00980057)(361.54818296,136.30999899)(367.08999634,136.30999899)
\curveto(372.63180972,136.30999899)(375.40615576,143.00980057)(371.48797695,146.92797939)
\curveto(367.56979813,150.84615821)(360.86999655,148.07181216)(360.86999655,142.52999878)
\curveto(360.86999655,136.9881854)(367.56979813,134.21383935)(371.48797695,138.13201817)
\curveto(375.40615576,142.05019699)(372.63180972,148.74999857)(367.08999634,148.74999857)
\curveto(361.54818296,148.74999857)(358.77383691,142.05019699)(362.69201573,138.13201817)
\curveto(366.61019455,134.21383935)(373.30999613,136.9881854)(373.30999613,142.52999878)
\closepath
}
}
{
\newrgbcolor{curcolor}{0 0 0}
\pscustom[linestyle=none,fillstyle=solid,fillcolor=curcolor]
{
\newpath
\moveto(80.52999735,104.47000122)
\curveto(80.52999735,110.0118146)(73.83019577,112.78616065)(69.91201695,108.86798183)
\curveto(65.99383813,104.94980301)(68.76818418,98.25000143)(74.30999756,98.25000143)
\curveto(79.85181094,98.25000143)(82.62615699,104.94980301)(78.70797817,108.86798183)
\curveto(74.78979935,112.78616065)(68.08999777,110.0118146)(68.08999777,104.47000122)
\curveto(68.08999777,98.92818784)(74.78979935,96.15384179)(78.70797817,100.07202061)
\curveto(82.62615699,103.99019943)(79.85181094,110.69000101)(74.30999756,110.69000101)
\curveto(68.76818418,110.69000101)(65.99383813,103.99019943)(69.91201695,100.07202061)
\curveto(73.83019577,96.15384179)(80.52999735,98.92818784)(80.52999735,104.47000122)
\closepath
}
}
{
\newrgbcolor{curcolor}{0 0 0}
\pscustom[linestyle=none,fillstyle=solid,fillcolor=curcolor]
{
\newpath
\moveto(212.66999674,142.70999908)
\curveto(212.66999674,148.25181247)(205.97019516,151.02615851)(202.05201634,147.10797969)
\curveto(198.13383752,143.18980087)(200.90818357,136.48999929)(206.44999695,136.48999929)
\curveto(211.99181033,136.48999929)(214.76615638,143.18980087)(210.84797756,147.10797969)
\curveto(206.92979874,151.02615851)(200.22999716,148.25181247)(200.22999716,142.70999908)
\curveto(200.22999716,137.1681857)(206.92979874,134.39383966)(210.84797756,138.31201848)
\curveto(214.76615638,142.23019729)(211.99181033,148.92999887)(206.44999695,148.92999887)
\curveto(200.90818357,148.92999887)(198.13383752,142.23019729)(202.05201634,138.31201848)
\curveto(205.97019516,134.39383966)(212.66999674,137.1681857)(212.66999674,142.70999908)
\closepath
}
}
\end{pspicture}

    \caption{两个边界框的交集。给定两个边界框,
        用空心圆表示\protect\refvar{pMin}{}和\protect\refvar{pMax}{},
        它们相交区域(涂色区域)边界框的最小点(左下角实心圆)的坐标由
        两个边界框每个维度最小点坐标的最大值给定。
        同样,其最大点(右上角实心圆)由框的最大坐标的最小值给定。}
    \label{fig:2.9}
\end{figure}

\begin{lstlisting}
`\refcode{Geometry Inline Functions}{+=}\lastnext{GeometryInlineFunctions}`
template <typename T> `\refvar{Bounds3}{}`<T>
`\initvar[Bounds3::Intersect]{Intersect}{}`(const `\refvar{Bounds3}{}`<T> &b1, const `\refvar{Bounds3}{}`<T> &b2) {
    return `\refvar{Bounds3}{}`<T>(`\refvar{Point3}{}`<T>(std::max(b1.pMin.x, b2.pMin.x),
                                std::max(b1.pMin.y, b2.pMin.y),
                                std::max(b1.pMin.z, b2.pMin.z)),
                      `\refvar{Point3}{}`<T>(std::min(b1.pMax.x, b2.pMax.x),
                                std::min(b1.pMax.y, b2.pMax.y),
                                std::min(b1.pMax.z, b2.pMax.z)));
}
\end{lstlisting}

我们可以通过检查两个边界框在$x,y$和$z$上的范围是否全都重合来判断它们是否重合:
\begin{lstlisting}
`\refcode{Geometry Inline Functions}{+=}\lastnext{GeometryInlineFunctions}`
template <typename T>
bool `\initvar{Overlaps}{}`(const `\refvar{Bounds3}{}`<T> &b1, const `\refvar{Bounds3}{}`<T> &b2) {
    bool x = (b1.pMax.x >= b2.pMin.x) && (b1.pMin.x <= b2.pMax.x);
    bool y = (b1.pMax.y >= b2.pMin.y) && (b1.pMin.y <= b2.pMax.y);
    bool z = (b1.pMax.z >= b2.pMin.z) && (b1.pMin.z <= b2.pMax.z);
    return (x && y && z);
}
\end{lstlisting}

三个1D包含测试确认给定点是否在边界框内:
\begin{lstlisting}
`\refcode{Geometry Inline Functions}{+=}\lastnext{GeometryInlineFunctions}`
template <typename T>
bool `\initvar{Inside}{}`(const `\refvar{Point3}{}`<T> &p, const `\refvar{Bounds3}{}`<T> &b) {
    return (p.x >= b.pMin.x && p.x <= b.pMax.x &&
            p.y >= b.pMin.y && p.y <= b.pMax.y &&
            p.z >= b.pMin.z && p.z <= b.pMax.z);
}
\end{lstlisting}

\refvar{Inside}{()}的变体\refvar{InsideExclusive}{()}不认为上界上的点在边界内。
它对整数型边界最有用。
\begin{lstlisting}
`\refcode{Geometry Inline Functions}{+=}\lastnext{GeometryInlineFunctions}`
template <typename T>
bool `\initvar{InsideExclusive}{}`(const `\refvar{Point3}{}`<T> &p, const `\refvar{Bounds3}{}`<T> &b) {
    return (p.x >= b.pMin.x && p.x < b.pMax.x &&
            p.y >= b.pMin.y && p.y < b.pMax.y &&
            p.z >= b.pMin.z && p.z < b.pMax.z);
}
\end{lstlisting}

函数\refvar{Expand}{()}用常数因子在所有维度上填补边界框。
\begin{lstlisting}
`\refcode{Geometry Inline Functions}{+=}\lastnext{GeometryInlineFunctions}`
template <typename T, typename U> inline `\refvar{Bounds3}{}`<T>
`\initvar{Expand}{}`(const `\refvar{Bounds3}{}`<T> &b, U delta) {
    return `\refvar{Bounds3}{}`<T>(b.pMin - `\refvar{Vector3}{}`<T>(delta, delta, delta),
                      b.pMax + `\refvar{Vector3}{}`<T>(delta, delta, delta));
}
\end{lstlisting}

\refvar{Diagonal}{()}返回沿框的对角线从最小点指向最大点的向量。
\begin{lstlisting}
`\refcode{Bounds3 Public Methods}{+=}\lastnext{Bounds3PublicMethods}`
`\refvar{Vector3}{}`<T> `\initvar{Diagonal}{}`() const { return `\refvar{pMax}{}` - `\refvar{pMin}{}`; }
\end{lstlisting}

计算框的六个面表面积以及内含体积的方法也很常用。
\begin{lstlisting}
`\refcode{Bounds3 Public Methods}{+=}\lastnext{Bounds3PublicMethods}`
T `\initvar{SurfaceArea}{}`() const {
    `\refvar{Vector3}{}`<T> d = `\refvar{Diagonal}{}`();
    return 2 * (d.x * d.y + d.x * d.z + d.y * d.z);
}
\end{lstlisting}

\begin{lstlisting}
`\refcode{Bounds3 Public Methods}{+=}\lastnext{Bounds3PublicMethods}`
T `\initvar{Volume}{}`() const {
    `\refvar{Vector3}{}`<T> d = `\refvar{Diagonal}{}`();
    return d.x * d.y * d.z;
}
\end{lstlisting}

方法\refvar[MaximumExtent]{Bounds3::MaximumExtent}{()}返回
三轴中最长者的索引。
例如在构建一些光线-交点加速结构需要决定要细分哪个轴时这很有用。
\begin{lstlisting}
`\refcode{Bounds3 Public Methods}{+=}\lastnext{Bounds3PublicMethods}`
int `\initvar{MaximumExtent}{}`() const {
    `\refvar{Vector3}{}`<T> d = `\refvar{Diagonal}{}`();
    if (d.x > d.y && d.x > d.z)
        return 0;
    else if (d.y > d.z)
        return 1;
    else
        return 2;
}
\end{lstlisting}

方法\refvar[Bounds3::Lerp]{Lerp}{()}用每个维度的给定量在框的两个顶点之间线性插值。
\begin{lstlisting}
`\refcode{Bounds3 Public Methods}{+=}\lastnext{Bounds3PublicMethods}`
`\refvar{Point3}{}`<T> `\initvar[Bounds3::Lerp]{Lerp}{}`(const `\refvar{Point3}{}` &t) const {
    return `\refvar{Point3}{}`<T>(::`\refvar{Lerp}{}`(t.x, pMin.x, pMax.x),
                     ::`\refvar{Lerp}{}`(t.y, pMin.y, pMax.y),
                     ::`\refvar{Lerp}{}`(t.z, pMin.z, pMax.z));
}
\end{lstlisting}

\refvar{Offset}{()}返回一点相对于框顶点的连续位置,
最小顶点处的偏移量为$(0,0,0)$,最大顶点处的偏移量为$(1,1,1)$,以此类推。
\begin{lstlisting}
`\refcode{Bounds3 Public Methods}{+=}\lastnext{Bounds3PublicMethods}`
`\refvar{Vector3}{}`<T> `\initvar{Offset}{}`(const `\refvar{Point3}{}`<T> &p) const {
    `\refvar{Vector3}{}`<T> o = p - pMin;
    if (pMax.x > pMin.x) o.x /= pMax.x - pMin.x;
    if (pMax.y > pMin.y) o.y /= pMax.y - pMin.y;
    if (pMax.z > pMin.z) o.z /= pMax.z - pMin.z;
    return o;
}
\end{lstlisting}

\refvar{Bounds3}{}也提供方法返回包围边界框的球体的球心和半径。
尽管它也很有用,但一般这样给出的范围会比直接包围\refvar{Bounds3}{}所含内容的球体松弛得多。
\begin{lstlisting}
`\refcode{Bounds3 Public Methods}{+=}\lastcode{Bounds3PublicMethods}`
void `\initvar{BoundingSphere}{}`(`\refvar{Point3}{}`<T> *center, `\refvar{Float}{}` *radius) const {
    *center = (`\refvar{pMin}{}` + `\refvar{pMax}{}`) / 2;
    *radius = `\refvar{Inside}{}`(*center, *this) ? `\refvar{Distance}{}`(*center, `\refvar{pMax}{}`) : 0;
}
\end{lstlisting}

最后,对于整数框,有一个迭代器类\sidenote{译者注:详见类{\ttfamily Bounds2iIterator}。}
满足C++前向迭代器\sidenote{译者注:原文forward iterator,是一种C++迭代器。}
(即它只能增长)的要求。
其细节比较乏味无趣,所以本书不介绍它的代码。
有了它的定义就可以用范围{\ttfamily for}循环编写代码
遍历边界框内的整数坐标了:

{\ttfamily\indent \refvar{Bounds2i}{} b = ...;}\newline
{\ttfamily\indent for (\refvar{Point2i}{} p : b) \{}\newline
{\ttfamily\indent \qquad//  …}\newline
{\ttfamily\indent \}}

这样实现后,每个维度的迭代会增长到最大范围但不取等于最大值的点
\sidenote{译者注:类似于左闭右开区间。}。