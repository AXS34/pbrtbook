\section{包围盒层次}\label{sec:包围盒层次}

\keyindex{包围盒层次}{bounding volume hierarchy}{}(BVH)是一种
基于图元细分的光线相交加速方法,把图元划分为不相交集合的层次
(相反,空间细分一般把空间划分为不相交集合的层次)。
\reffig{4.2}展示了简单场景的包围盒层次\sidenote{译者注:包围盒是边界框的近义词。}。
图元存于\keyindex{叶子}{leaf}{}中,只要它不与节点的边界相交,
该节点下的子树就可以跳过。
\begin{figure}[htbp]
    \centering%LaTeX with PSTricks extensions
%%Creator: Inkscape 1.0.1 (3bc2e813f5, 2020-09-07)
%%Please note this file requires PSTricks extensions
\psset{xunit=.5pt,yunit=.5pt,runit=.5pt}
\begin{pspicture}(486.5,540.53997803)
{
\newrgbcolor{curcolor}{1 1 1}
\pscustom[linestyle=none,fillstyle=solid,fillcolor=curcolor]
{
\newpath
\moveto(180.18,305.85997803)
\curveto(196.22939482,305.85997803)(209.24,318.8705832)(209.24,334.91997803)
\curveto(209.25213323,346.69122324)(202.16899365,357.30966361)(191.29629426,361.81967461)
\curveto(180.42361715,366.32967637)(167.90482511,363.84214428)(159.58132943,355.5186486)
\curveto(141.2463808,337.18369997)(154.24731849,305.83324788)(180.18,305.85997803)
\closepath
}
}
{
\newrgbcolor{curcolor}{0 0 0}
\pscustom[linestyle=none,fillstyle=solid,fillcolor=curcolor]
{
\newpath
\moveto(180.18,363.47997803)
\curveto(164.40674755,363.47997803)(151.62,350.69323048)(151.62,334.91997803)
\curveto(151.58435925,309.41906014)(182.41250553,296.62810038)(200.44219159,314.65778644)
\curveto(208.6268344,322.84242924)(211.07179876,335.15289509)(206.63507228,345.84337979)
\curveto(202.19833667,356.53388644)(191.75487179,363.49615537)(180.18,363.47997803)
\moveto(180.18,364.47997803)
\curveto(196.5055372,364.47997803)(209.74,351.24551523)(209.74,334.91997803)
\curveto(209.73109046,308.59673601)(177.90714815,295.42494182)(159.29605597,314.036034)
\curveto(140.68496379,332.64712618)(153.85675798,364.47106848)(180.18,364.47997803)
\closepath
}
}
{
\newrgbcolor{curcolor}{1 1 1}
\pscustom[linestyle=none,fillstyle=solid,fillcolor=curcolor]
{
\newpath
\moveto(224.76,419.38997803)
\lineto(290.16,398.63997803)
\lineto(239.49,352.36997803)
\closepath
}
}
{
\newrgbcolor{curcolor}{0 0 0}
\pscustom[linestyle=none,fillstyle=solid,fillcolor=curcolor]
{
\newpath
\moveto(225.43,418.64997803)
\lineto(257.31,408.53997803)
\lineto(289.19,398.42997803)
\lineto(264.49,375.87997803)
\lineto(239.79,353.32997803)
\lineto(232.61,385.99997803)
\lineto(225.43,418.65997803)
\moveto(224.08,420.13997803)
\lineto(231.63,385.77997803)
\lineto(239.19,351.42997803)
\lineto(265.19,375.13997803)
\lineto(291.19,398.85997803)
\lineto(257.66,409.49997803)
\lineto(224.13,420.13997803)
\closepath
}
}
{
\newrgbcolor{curcolor}{1 1 1}
\pscustom[linestyle=none,fillstyle=solid,fillcolor=curcolor]
{
\newpath
\moveto(304.78,536.08997803)
\lineto(383.02,495.40997803)
\lineto(413.91,415.93997803)
\closepath
}
}
{
\newrgbcolor{curcolor}{0 0 0}
\pscustom[linestyle=none,fillstyle=solid,fillcolor=curcolor]
{
\newpath
\moveto(307,534.34997803)
\lineto(342.48,515.90997803)
\lineto(382.59,495.05997803)
\lineto(398.42,454.32997803)
\lineto(412.42,418.26997803)
\lineto(359.68,476.37997803)
\lineto(307,534.34997803)
\moveto(302.5,537.81997803)
\lineto(358.91,475.70997803)
\lineto(415.32,413.59997803)
\lineto(399.32,454.68997803)
\lineto(383.32,495.76997803)
\lineto(343,516.78997803)
\lineto(302.56,537.78997803)
\closepath
}
}
{
\newrgbcolor{curcolor}{0 0 0}
\pscustom[linestyle=none,fillstyle=solid,fillcolor=curcolor]
{
\newpath
\moveto(416.22,538.79997803)
\lineto(414.22,538.79997803)
\lineto(414.22,537.79997803)
\lineto(416.22,537.79997803)
\closepath
\moveto(412.22,538.79997803)
\lineto(410.22,538.79997803)
\lineto(410.22,537.79997803)
\lineto(412.22,537.79997803)
\closepath
\moveto(408.22,538.79997803)
\lineto(406.22,538.79997803)
\lineto(406.22,537.79997803)
\lineto(408.22,537.79997803)
\closepath
\moveto(404.22,538.79997803)
\lineto(402.22,538.79997803)
\lineto(402.22,537.79997803)
\lineto(404.22,537.79997803)
\closepath
\moveto(400.22,538.79997803)
\lineto(398.22,538.79997803)
\lineto(398.22,537.79997803)
\lineto(400.22,537.79997803)
\closepath
\moveto(396.22,538.79997803)
\lineto(394.22,538.79997803)
\lineto(394.22,537.79997803)
\lineto(396.22,537.79997803)
\closepath
\moveto(392.22,538.79997803)
\lineto(390.22,538.79997803)
\lineto(390.22,537.79997803)
\lineto(392.22,537.79997803)
\closepath
\moveto(388.22,538.79997803)
\lineto(386.22,538.79997803)
\lineto(386.22,537.79997803)
\lineto(388.22,537.79997803)
\closepath
\moveto(384.22,538.79997803)
\lineto(382.22,538.79997803)
\lineto(382.22,537.79997803)
\lineto(384.22,537.79997803)
\closepath
\moveto(380.22,538.79997803)
\lineto(378.22,538.79997803)
\lineto(378.22,537.79997803)
\lineto(380.22,537.79997803)
\closepath
\moveto(376.22,538.79997803)
\lineto(374.22,538.79997803)
\lineto(374.22,537.79997803)
\lineto(376.22,537.79997803)
\closepath
\moveto(372.22,538.79997803)
\lineto(370.22,538.79997803)
\lineto(370.22,537.79997803)
\lineto(372.22,537.79997803)
\closepath
\moveto(368.22,538.79997803)
\lineto(366.22,538.79997803)
\lineto(366.22,537.79997803)
\lineto(368.22,537.79997803)
\closepath
\moveto(364.22,538.79997803)
\lineto(362.22,538.79997803)
\lineto(362.22,537.79997803)
\lineto(364.22,537.79997803)
\closepath
\moveto(360.22,538.79997803)
\lineto(358.22,538.79997803)
\lineto(358.22,537.79997803)
\lineto(360.22,537.79997803)
\closepath
\moveto(356.22,538.79997803)
\lineto(354.22,538.79997803)
\lineto(354.22,537.79997803)
\lineto(356.22,537.79997803)
\closepath
\moveto(352.22,538.79997803)
\lineto(350.22,538.79997803)
\lineto(350.22,537.79997803)
\lineto(352.22,537.79997803)
\closepath
\moveto(348.22,538.79997803)
\lineto(346.22,538.79997803)
\lineto(346.22,537.79997803)
\lineto(348.22,537.79997803)
\closepath
\moveto(344.22,538.79997803)
\lineto(342.22,538.79997803)
\lineto(342.22,537.79997803)
\lineto(344.22,537.79997803)
\closepath
\moveto(340.22,538.79997803)
\lineto(338.22,538.79997803)
\lineto(338.22,537.79997803)
\lineto(340.22,537.79997803)
\closepath
\moveto(336.22,538.79997803)
\lineto(334.22,538.79997803)
\lineto(334.22,537.79997803)
\lineto(336.22,537.79997803)
\closepath
\moveto(332.22,538.79997803)
\lineto(330.22,538.79997803)
\lineto(330.22,537.79997803)
\lineto(332.22,537.79997803)
\closepath
\moveto(328.22,538.79997803)
\lineto(326.22,538.79997803)
\lineto(326.22,537.79997803)
\lineto(328.22,537.79997803)
\closepath
\moveto(324.22,538.79997803)
\lineto(322.22,538.79997803)
\lineto(322.22,537.79997803)
\lineto(324.22,537.79997803)
\closepath
\moveto(320.22,538.79997803)
\lineto(318.22,538.79997803)
\lineto(318.22,537.79997803)
\lineto(320.22,537.79997803)
\closepath
\moveto(316.22,538.79997803)
\lineto(314.22,538.79997803)
\lineto(314.22,537.79997803)
\lineto(316.22,537.79997803)
\closepath
\moveto(312.22,538.79997803)
\lineto(310.22,538.79997803)
\lineto(310.22,537.79997803)
\lineto(312.22,537.79997803)
\closepath
\moveto(308.22,538.79997803)
\lineto(306.22,538.79997803)
\lineto(306.22,537.79997803)
\lineto(308.22,537.79997803)
\closepath
\moveto(304.22,538.79997803)
\lineto(302.22,538.79997803)
\lineto(302.22,537.79997803)
\lineto(304.22,537.79997803)
\closepath
\moveto(302.46,537.53997803)
\lineto(301.46,537.53997803)
\lineto(301.46,535.53997803)
\lineto(302.46,535.53997803)
\closepath
\moveto(302.46,533.53997803)
\lineto(301.46,533.53997803)
\lineto(301.46,531.53997803)
\lineto(302.46,531.53997803)
\closepath
\moveto(302.46,529.53997803)
\lineto(301.46,529.53997803)
\lineto(301.46,527.53997803)
\lineto(302.46,527.53997803)
\closepath
\moveto(302.46,525.53997803)
\lineto(301.46,525.53997803)
\lineto(301.46,523.53997803)
\lineto(302.46,523.53997803)
\closepath
\moveto(302.46,521.53997803)
\lineto(301.46,521.53997803)
\lineto(301.46,519.53997803)
\lineto(302.46,519.53997803)
\closepath
\moveto(302.46,517.53997803)
\lineto(301.46,517.53997803)
\lineto(301.46,515.53997803)
\lineto(302.46,515.53997803)
\closepath
\moveto(302.46,513.53997803)
\lineto(301.46,513.53997803)
\lineto(301.46,511.53997803)
\lineto(302.46,511.53997803)
\closepath
\moveto(302.46,509.53997803)
\lineto(301.46,509.53997803)
\lineto(301.46,507.53997803)
\lineto(302.46,507.53997803)
\closepath
\moveto(302.46,505.53997803)
\lineto(301.46,505.53997803)
\lineto(301.46,503.53997803)
\lineto(302.46,503.53997803)
\closepath
\moveto(302.46,501.53997803)
\lineto(301.46,501.53997803)
\lineto(301.46,499.53997803)
\lineto(302.46,499.53997803)
\closepath
\moveto(302.46,497.53997803)
\lineto(301.46,497.53997803)
\lineto(301.46,495.53997803)
\lineto(302.46,495.53997803)
\closepath
\moveto(302.46,493.53997803)
\lineto(301.46,493.53997803)
\lineto(301.46,491.53997803)
\lineto(302.46,491.53997803)
\closepath
\moveto(302.46,489.53997803)
\lineto(301.46,489.53997803)
\lineto(301.46,487.53997803)
\lineto(302.46,487.53997803)
\closepath
\moveto(302.46,485.53997803)
\lineto(301.46,485.53997803)
\lineto(301.46,483.53997803)
\lineto(302.46,483.53997803)
\closepath
\moveto(302.46,481.53997803)
\lineto(301.46,481.53997803)
\lineto(301.46,479.53997803)
\lineto(302.46,479.53997803)
\closepath
\moveto(302.46,477.53997803)
\lineto(301.46,477.53997803)
\lineto(301.46,475.53997803)
\lineto(302.46,475.53997803)
\closepath
\moveto(302.46,473.53997803)
\lineto(301.46,473.53997803)
\lineto(301.46,471.53997803)
\lineto(302.46,471.53997803)
\closepath
\moveto(302.46,469.53997803)
\lineto(301.46,469.53997803)
\lineto(301.46,467.53997803)
\lineto(302.46,467.53997803)
\closepath
\moveto(302.46,465.53997803)
\lineto(301.46,465.53997803)
\lineto(301.46,463.53997803)
\lineto(302.46,463.53997803)
\closepath
\moveto(302.46,461.53997803)
\lineto(301.46,461.53997803)
\lineto(301.46,459.53997803)
\lineto(302.46,459.53997803)
\closepath
\moveto(302.46,457.53997803)
\lineto(301.46,457.53997803)
\lineto(301.46,455.53997803)
\lineto(302.46,455.53997803)
\closepath
\moveto(302.46,453.53997803)
\lineto(301.46,453.53997803)
\lineto(301.46,451.53997803)
\lineto(302.46,451.53997803)
\closepath
\moveto(302.46,449.53997803)
\lineto(301.46,449.53997803)
\lineto(301.46,447.53997803)
\lineto(302.46,447.53997803)
\closepath
\moveto(302.46,445.53997803)
\lineto(301.46,445.53997803)
\lineto(301.46,443.53997803)
\lineto(302.46,443.53997803)
\closepath
\moveto(302.46,441.53997803)
\lineto(301.46,441.53997803)
\lineto(301.46,439.53997803)
\lineto(302.46,439.53997803)
\closepath
\moveto(302.46,437.53997803)
\lineto(301.46,437.53997803)
\lineto(301.46,435.53997803)
\lineto(302.46,435.53997803)
\closepath
\moveto(302.46,433.53997803)
\lineto(301.46,433.53997803)
\lineto(301.46,431.53997803)
\lineto(302.46,431.53997803)
\closepath
\moveto(302.46,429.53997803)
\lineto(301.46,429.53997803)
\lineto(301.46,427.53997803)
\lineto(302.46,427.53997803)
\closepath
\moveto(302.46,425.53997803)
\lineto(301.46,425.53997803)
\lineto(301.46,423.53997803)
\lineto(302.46,423.53997803)
\closepath
\moveto(302.46,421.53997803)
\lineto(301.46,421.53997803)
\lineto(301.46,419.53997803)
\lineto(302.46,419.53997803)
\closepath
\moveto(302.46,417.53997803)
\lineto(301.46,417.53997803)
\lineto(301.46,415.53997803)
\lineto(302.46,415.53997803)
\closepath
\moveto(303.94,415.01997803)
\lineto(301.94,415.01997803)
\lineto(301.94,414.01997803)
\lineto(303.94,414.01997803)
\closepath
\moveto(307.94,415.01997803)
\lineto(305.94,415.01997803)
\lineto(305.94,414.01997803)
\lineto(307.94,414.01997803)
\closepath
\moveto(311.94,415.01997803)
\lineto(309.94,415.01997803)
\lineto(309.94,414.01997803)
\lineto(311.94,414.01997803)
\closepath
\moveto(315.94,415.01997803)
\lineto(313.94,415.01997803)
\lineto(313.94,414.01997803)
\lineto(315.94,414.01997803)
\closepath
\moveto(319.94,415.01997803)
\lineto(317.94,415.01997803)
\lineto(317.94,414.01997803)
\lineto(319.94,414.01997803)
\closepath
\moveto(323.94,415.01997803)
\lineto(321.94,415.01997803)
\lineto(321.94,414.01997803)
\lineto(323.94,414.01997803)
\closepath
\moveto(327.94,415.01997803)
\lineto(325.94,415.01997803)
\lineto(325.94,414.01997803)
\lineto(327.94,414.01997803)
\closepath
\moveto(331.94,415.01997803)
\lineto(329.94,415.01997803)
\lineto(329.94,414.01997803)
\lineto(331.94,414.01997803)
\closepath
\moveto(335.94,415.01997803)
\lineto(333.94,415.01997803)
\lineto(333.94,414.01997803)
\lineto(335.94,414.01997803)
\closepath
\moveto(339.94,415.01997803)
\lineto(337.94,415.01997803)
\lineto(337.94,414.01997803)
\lineto(339.94,414.01997803)
\closepath
\moveto(343.94,415.01997803)
\lineto(341.94,415.01997803)
\lineto(341.94,414.01997803)
\lineto(343.94,414.01997803)
\closepath
\moveto(347.94,415.01997803)
\lineto(345.94,415.01997803)
\lineto(345.94,414.01997803)
\lineto(347.94,414.01997803)
\closepath
\moveto(351.94,415.01997803)
\lineto(349.94,415.01997803)
\lineto(349.94,414.01997803)
\lineto(351.94,414.01997803)
\closepath
\moveto(355.94,415.01997803)
\lineto(353.94,415.01997803)
\lineto(353.94,414.01997803)
\lineto(355.94,414.01997803)
\closepath
\moveto(359.94,415.01997803)
\lineto(357.94,415.01997803)
\lineto(357.94,414.01997803)
\lineto(359.94,414.01997803)
\closepath
\moveto(363.94,415.01997803)
\lineto(361.94,415.01997803)
\lineto(361.94,414.01997803)
\lineto(363.94,414.01997803)
\closepath
\moveto(367.94,415.01997803)
\lineto(365.94,415.01997803)
\lineto(365.94,414.01997803)
\lineto(367.94,414.01997803)
\closepath
\moveto(371.94,415.01997803)
\lineto(369.94,415.01997803)
\lineto(369.94,414.01997803)
\lineto(371.94,414.01997803)
\closepath
\moveto(375.94,415.01997803)
\lineto(373.94,415.01997803)
\lineto(373.94,414.01997803)
\lineto(375.94,414.01997803)
\closepath
\moveto(379.94,415.01997803)
\lineto(377.94,415.01997803)
\lineto(377.94,414.01997803)
\lineto(379.94,414.01997803)
\closepath
\moveto(383.94,415.01997803)
\lineto(381.94,415.01997803)
\lineto(381.94,414.01997803)
\lineto(383.94,414.01997803)
\closepath
\moveto(387.94,415.01997803)
\lineto(385.94,415.01997803)
\lineto(385.94,414.01997803)
\lineto(387.94,414.01997803)
\closepath
\moveto(391.94,415.01997803)
\lineto(389.94,415.01997803)
\lineto(389.94,414.01997803)
\lineto(391.94,414.01997803)
\closepath
\moveto(395.94,415.01997803)
\lineto(393.94,415.01997803)
\lineto(393.94,414.01997803)
\lineto(395.94,414.01997803)
\closepath
\moveto(399.94,415.01997803)
\lineto(397.94,415.01997803)
\lineto(397.94,414.01997803)
\lineto(399.94,414.01997803)
\closepath
\moveto(403.94,415.01997803)
\lineto(401.94,415.01997803)
\lineto(401.94,414.01997803)
\lineto(403.94,414.01997803)
\closepath
\moveto(407.94,415.01997803)
\lineto(405.94,415.01997803)
\lineto(405.94,414.01997803)
\lineto(407.94,414.01997803)
\closepath
\moveto(411.94,415.01997803)
\lineto(409.94,415.01997803)
\lineto(409.94,414.01997803)
\lineto(411.94,414.01997803)
\closepath
\moveto(415.94,415.01997803)
\lineto(413.94,415.01997803)
\lineto(413.94,414.01997803)
\lineto(415.94,414.01997803)
\closepath
\moveto(416.22,417.72997803)
\lineto(415.22,417.72997803)
\lineto(415.22,415.72997803)
\lineto(416.22,415.72997803)
\closepath
\moveto(416.22,421.72997803)
\lineto(415.22,421.72997803)
\lineto(415.22,419.72997803)
\lineto(416.22,419.72997803)
\closepath
\moveto(416.22,425.72997803)
\lineto(415.22,425.72997803)
\lineto(415.22,423.72997803)
\lineto(416.22,423.72997803)
\closepath
\moveto(416.22,429.72997803)
\lineto(415.22,429.72997803)
\lineto(415.22,427.72997803)
\lineto(416.22,427.72997803)
\closepath
\moveto(416.22,433.72997803)
\lineto(415.22,433.72997803)
\lineto(415.22,431.72997803)
\lineto(416.22,431.72997803)
\closepath
\moveto(416.22,437.72997803)
\lineto(415.22,437.72997803)
\lineto(415.22,435.72997803)
\lineto(416.22,435.72997803)
\closepath
\moveto(416.22,441.72997803)
\lineto(415.22,441.72997803)
\lineto(415.22,439.72997803)
\lineto(416.22,439.72997803)
\closepath
\moveto(416.22,445.72997803)
\lineto(415.22,445.72997803)
\lineto(415.22,443.72997803)
\lineto(416.22,443.72997803)
\closepath
\moveto(416.22,449.72997803)
\lineto(415.22,449.72997803)
\lineto(415.22,447.72997803)
\lineto(416.22,447.72997803)
\closepath
\moveto(416.22,453.72997803)
\lineto(415.22,453.72997803)
\lineto(415.22,451.72997803)
\lineto(416.22,451.72997803)
\closepath
\moveto(416.22,457.72997803)
\lineto(415.22,457.72997803)
\lineto(415.22,455.72997803)
\lineto(416.22,455.72997803)
\closepath
\moveto(416.22,461.72997803)
\lineto(415.22,461.72997803)
\lineto(415.22,459.72997803)
\lineto(416.22,459.72997803)
\closepath
\moveto(416.22,465.72997803)
\lineto(415.22,465.72997803)
\lineto(415.22,463.72997803)
\lineto(416.22,463.72997803)
\closepath
\moveto(416.22,469.72997803)
\lineto(415.22,469.72997803)
\lineto(415.22,467.72997803)
\lineto(416.22,467.72997803)
\closepath
\moveto(416.22,473.72997803)
\lineto(415.22,473.72997803)
\lineto(415.22,471.72997803)
\lineto(416.22,471.72997803)
\closepath
\moveto(416.22,477.72997803)
\lineto(415.22,477.72997803)
\lineto(415.22,475.72997803)
\lineto(416.22,475.72997803)
\closepath
\moveto(416.22,481.72997803)
\lineto(415.22,481.72997803)
\lineto(415.22,479.72997803)
\lineto(416.22,479.72997803)
\closepath
\moveto(416.22,485.72997803)
\lineto(415.22,485.72997803)
\lineto(415.22,483.72997803)
\lineto(416.22,483.72997803)
\closepath
\moveto(416.22,489.72997803)
\lineto(415.22,489.72997803)
\lineto(415.22,487.72997803)
\lineto(416.22,487.72997803)
\closepath
\moveto(416.22,493.72997803)
\lineto(415.22,493.72997803)
\lineto(415.22,491.72997803)
\lineto(416.22,491.72997803)
\closepath
\moveto(416.22,497.72997803)
\lineto(415.22,497.72997803)
\lineto(415.22,495.72997803)
\lineto(416.22,495.72997803)
\closepath
\moveto(416.22,501.72997803)
\lineto(415.22,501.72997803)
\lineto(415.22,499.72997803)
\lineto(416.22,499.72997803)
\closepath
\moveto(416.22,505.72997803)
\lineto(415.22,505.72997803)
\lineto(415.22,503.72997803)
\lineto(416.22,503.72997803)
\closepath
\moveto(416.22,509.72997803)
\lineto(415.22,509.72997803)
\lineto(415.22,507.72997803)
\lineto(416.22,507.72997803)
\closepath
\moveto(416.22,513.72997803)
\lineto(415.22,513.72997803)
\lineto(415.22,511.72997803)
\lineto(416.22,511.72997803)
\closepath
\moveto(416.22,517.72997803)
\lineto(415.22,517.72997803)
\lineto(415.22,515.72997803)
\lineto(416.22,515.72997803)
\closepath
\moveto(416.22,521.72997803)
\lineto(415.22,521.72997803)
\lineto(415.22,519.72997803)
\lineto(416.22,519.72997803)
\closepath
\moveto(416.22,525.72997803)
\lineto(415.22,525.72997803)
\lineto(415.22,523.72997803)
\lineto(416.22,523.72997803)
\closepath
\moveto(416.22,529.72997803)
\lineto(415.22,529.72997803)
\lineto(415.22,527.72997803)
\lineto(416.22,527.72997803)
\closepath
\moveto(416.22,533.72997803)
\lineto(415.22,533.72997803)
\lineto(415.22,531.72997803)
\lineto(416.22,531.72997803)
\closepath
\moveto(416.22,537.72997803)
\lineto(415.22,537.72997803)
\lineto(415.22,535.72997803)
\lineto(416.22,535.72997803)
\closepath
}
}
{
\newrgbcolor{curcolor}{0 0 0}
\pscustom[linestyle=none,fillstyle=solid,fillcolor=curcolor]
{
\newpath
\moveto(292.33,420.98997803)
\lineto(290.33,420.98997803)
\lineto(290.33,419.98997803)
\lineto(291.33,419.98997803)
\lineto(291.33,418.42997803)
\lineto(292.33,418.42997803)
\closepath
\moveto(288.33,420.98997803)
\lineto(286.33,420.98997803)
\lineto(286.33,419.98997803)
\lineto(288.33,419.98997803)
\closepath
\moveto(284.33,420.98997803)
\lineto(282.33,420.98997803)
\lineto(282.33,419.98997803)
\lineto(284.33,419.98997803)
\closepath
\moveto(280.33,420.98997803)
\lineto(278.33,420.98997803)
\lineto(278.33,419.98997803)
\lineto(280.33,419.98997803)
\closepath
\moveto(276.33,420.98997803)
\lineto(274.33,420.98997803)
\lineto(274.33,419.98997803)
\lineto(276.33,419.98997803)
\closepath
\moveto(272.33,420.98997803)
\lineto(270.33,420.98997803)
\lineto(270.33,419.98997803)
\lineto(272.33,419.98997803)
\closepath
\moveto(268.33,420.98997803)
\lineto(266.33,420.98997803)
\lineto(266.33,419.98997803)
\lineto(268.33,419.98997803)
\closepath
\moveto(264.33,420.98997803)
\lineto(262.33,420.98997803)
\lineto(262.33,419.98997803)
\lineto(264.33,419.98997803)
\closepath
\moveto(260.33,420.98997803)
\lineto(258.33,420.98997803)
\lineto(258.33,419.98997803)
\lineto(260.33,419.98997803)
\closepath
\moveto(256.33,420.98997803)
\lineto(254.33,420.98997803)
\lineto(254.33,419.98997803)
\lineto(256.33,419.98997803)
\closepath
\moveto(252.33,420.98997803)
\lineto(250.33,420.98997803)
\lineto(250.33,419.98997803)
\lineto(252.33,419.98997803)
\closepath
\moveto(248.33,420.98997803)
\lineto(246.33,420.98997803)
\lineto(246.33,419.98997803)
\lineto(248.33,419.98997803)
\closepath
\moveto(244.33,420.98997803)
\lineto(242.33,420.98997803)
\lineto(242.33,419.98997803)
\lineto(244.33,419.98997803)
\closepath
\moveto(240.33,420.98997803)
\lineto(238.33,420.98997803)
\lineto(238.33,419.98997803)
\lineto(240.33,419.98997803)
\closepath
\moveto(236.33,420.98997803)
\lineto(234.33,420.98997803)
\lineto(234.33,419.98997803)
\lineto(236.33,419.98997803)
\closepath
\moveto(232.33,420.98997803)
\lineto(230.33,420.98997803)
\lineto(230.33,419.98997803)
\lineto(232.33,419.98997803)
\closepath
\moveto(228.33,420.98997803)
\lineto(226.33,420.98997803)
\lineto(226.33,419.98997803)
\lineto(228.33,419.98997803)
\closepath
\moveto(224.33,420.98997803)
\lineto(222.33,420.98997803)
\lineto(222.33,419.98997803)
\lineto(224.33,419.98997803)
\closepath
\moveto(222.47,419.84997803)
\lineto(221.47,419.84997803)
\lineto(221.47,417.84997803)
\lineto(222.47,417.84997803)
\closepath
\moveto(222.47,415.84997803)
\lineto(221.47,415.84997803)
\lineto(221.47,413.84997803)
\lineto(222.47,413.84997803)
\closepath
\moveto(222.47,411.84997803)
\lineto(221.47,411.84997803)
\lineto(221.47,409.84997803)
\lineto(222.47,409.84997803)
\closepath
\moveto(222.47,407.84997803)
\lineto(221.47,407.84997803)
\lineto(221.47,405.84997803)
\lineto(222.47,405.84997803)
\closepath
\moveto(222.47,403.84997803)
\lineto(221.47,403.84997803)
\lineto(221.47,401.84997803)
\lineto(222.47,401.84997803)
\closepath
\moveto(222.47,399.84997803)
\lineto(221.47,399.84997803)
\lineto(221.47,397.84997803)
\lineto(222.47,397.84997803)
\closepath
\moveto(222.47,395.84997803)
\lineto(221.47,395.84997803)
\lineto(221.47,393.84997803)
\lineto(222.47,393.84997803)
\closepath
\moveto(222.47,391.84997803)
\lineto(221.47,391.84997803)
\lineto(221.47,389.84997803)
\lineto(222.47,389.84997803)
\closepath
\moveto(222.47,387.84997803)
\lineto(221.47,387.84997803)
\lineto(221.47,385.84997803)
\lineto(222.47,385.84997803)
\closepath
\moveto(222.47,383.84997803)
\lineto(221.47,383.84997803)
\lineto(221.47,381.84997803)
\lineto(222.47,381.84997803)
\closepath
\moveto(222.47,379.84997803)
\lineto(221.47,379.84997803)
\lineto(221.47,377.84997803)
\lineto(222.47,377.84997803)
\closepath
\moveto(222.47,375.84997803)
\lineto(221.47,375.84997803)
\lineto(221.47,373.84997803)
\lineto(222.47,373.84997803)
\closepath
\moveto(222.47,371.84997803)
\lineto(221.47,371.84997803)
\lineto(221.47,369.84997803)
\lineto(222.47,369.84997803)
\closepath
\moveto(222.47,367.84997803)
\lineto(221.47,367.84997803)
\lineto(221.47,365.84997803)
\lineto(222.47,365.84997803)
\closepath
\moveto(222.47,363.84997803)
\lineto(221.47,363.84997803)
\lineto(221.47,361.84997803)
\lineto(222.47,361.84997803)
\closepath
\moveto(222.47,359.84997803)
\lineto(221.47,359.84997803)
\lineto(221.47,357.84997803)
\lineto(222.47,357.84997803)
\closepath
\moveto(222.47,355.84997803)
\lineto(221.47,355.84997803)
\lineto(221.47,353.84997803)
\lineto(222.47,353.84997803)
\closepath
\moveto(222.47,351.84997803)
\lineto(221.47,351.84997803)
\lineto(221.47,350.53997803)
\lineto(222.47,350.53997803)
\lineto(222.47,351.81997803)
\closepath
\moveto(226.19,351.56997803)
\lineto(224.19,351.56997803)
\lineto(224.19,350.56997803)
\lineto(226.19,350.56997803)
\closepath
\moveto(230.19,351.56997803)
\lineto(228.19,351.56997803)
\lineto(228.19,350.56997803)
\lineto(230.19,350.56997803)
\closepath
\moveto(234.19,351.56997803)
\lineto(232.19,351.56997803)
\lineto(232.19,350.56997803)
\lineto(234.19,350.56997803)
\closepath
\moveto(238.19,351.56997803)
\lineto(236.19,351.56997803)
\lineto(236.19,350.56997803)
\lineto(238.19,350.56997803)
\closepath
\moveto(242.19,351.56997803)
\lineto(240.19,351.56997803)
\lineto(240.19,350.56997803)
\lineto(242.19,350.56997803)
\closepath
\moveto(246.19,351.56997803)
\lineto(244.19,351.56997803)
\lineto(244.19,350.56997803)
\lineto(246.19,350.56997803)
\closepath
\moveto(250.19,351.56997803)
\lineto(248.19,351.56997803)
\lineto(248.19,350.56997803)
\lineto(250.19,350.56997803)
\closepath
\moveto(254.19,351.56997803)
\lineto(252.19,351.56997803)
\lineto(252.19,350.56997803)
\lineto(254.19,350.56997803)
\closepath
\moveto(258.19,351.56997803)
\lineto(256.19,351.56997803)
\lineto(256.19,350.56997803)
\lineto(258.19,350.56997803)
\closepath
\moveto(262.19,351.56997803)
\lineto(260.19,351.56997803)
\lineto(260.19,350.56997803)
\lineto(262.19,350.56997803)
\closepath
\moveto(266.19,351.56997803)
\lineto(264.19,351.56997803)
\lineto(264.19,350.56997803)
\lineto(266.19,350.56997803)
\closepath
\moveto(270.19,351.56997803)
\lineto(268.19,351.56997803)
\lineto(268.19,350.56997803)
\lineto(270.19,350.56997803)
\closepath
\moveto(274.19,351.56997803)
\lineto(272.19,351.56997803)
\lineto(272.19,350.56997803)
\lineto(274.19,350.56997803)
\closepath
\moveto(278.19,351.56997803)
\lineto(276.19,351.56997803)
\lineto(276.19,350.56997803)
\lineto(278.19,350.56997803)
\closepath
\moveto(282.19,351.56997803)
\lineto(280.19,351.56997803)
\lineto(280.19,350.56997803)
\lineto(282.19,350.56997803)
\closepath
\moveto(286.19,351.56997803)
\lineto(284.19,351.56997803)
\lineto(284.19,350.56997803)
\lineto(286.19,350.56997803)
\closepath
\moveto(290.19,351.56997803)
\lineto(288.19,351.56997803)
\lineto(288.19,350.56997803)
\lineto(290.19,350.56997803)
\closepath
\moveto(292.33,352.42997803)
\lineto(291.33,352.42997803)
\lineto(291.33,350.53997803)
\lineto(292.33,350.53997803)
\lineto(292.33,352.39997803)
\closepath
\moveto(292.33,356.42997803)
\lineto(291.33,356.42997803)
\lineto(291.33,354.42997803)
\lineto(292.33,354.42997803)
\closepath
\moveto(292.33,360.42997803)
\lineto(291.33,360.42997803)
\lineto(291.33,358.42997803)
\lineto(292.33,358.42997803)
\closepath
\moveto(292.33,364.42997803)
\lineto(291.33,364.42997803)
\lineto(291.33,362.42997803)
\lineto(292.33,362.42997803)
\closepath
\moveto(292.33,368.42997803)
\lineto(291.33,368.42997803)
\lineto(291.33,366.42997803)
\lineto(292.33,366.42997803)
\closepath
\moveto(292.33,372.42997803)
\lineto(291.33,372.42997803)
\lineto(291.33,370.42997803)
\lineto(292.33,370.42997803)
\closepath
\moveto(292.33,376.42997803)
\lineto(291.33,376.42997803)
\lineto(291.33,374.42997803)
\lineto(292.33,374.42997803)
\closepath
\moveto(292.33,380.42997803)
\lineto(291.33,380.42997803)
\lineto(291.33,378.42997803)
\lineto(292.33,378.42997803)
\closepath
\moveto(292.33,384.42997803)
\lineto(291.33,384.42997803)
\lineto(291.33,382.42997803)
\lineto(292.33,382.42997803)
\closepath
\moveto(292.33,388.42997803)
\lineto(291.33,388.42997803)
\lineto(291.33,386.42997803)
\lineto(292.33,386.42997803)
\closepath
\moveto(292.33,392.42997803)
\lineto(291.33,392.42997803)
\lineto(291.33,390.42997803)
\lineto(292.33,390.42997803)
\closepath
\moveto(292.33,396.42997803)
\lineto(291.33,396.42997803)
\lineto(291.33,394.42997803)
\lineto(292.33,394.42997803)
\closepath
\moveto(292.33,400.42997803)
\lineto(291.33,400.42997803)
\lineto(291.33,398.42997803)
\lineto(292.33,398.42997803)
\closepath
\moveto(292.33,404.42997803)
\lineto(291.33,404.42997803)
\lineto(291.33,402.42997803)
\lineto(292.33,402.42997803)
\closepath
\moveto(292.33,408.42997803)
\lineto(291.33,408.42997803)
\lineto(291.33,406.42997803)
\lineto(292.33,406.42997803)
\closepath
\moveto(292.33,412.42997803)
\lineto(291.33,412.42997803)
\lineto(291.33,410.42997803)
\lineto(292.33,410.42997803)
\closepath
\moveto(292.33,416.42997803)
\lineto(291.33,416.42997803)
\lineto(291.33,414.42997803)
\lineto(292.33,414.42997803)
\closepath
}
}
{
\newrgbcolor{curcolor}{0 0 0}
\pscustom[linestyle=none,fillstyle=solid,fillcolor=curcolor]
{
\newpath
\moveto(206.61,366.21997803)
\lineto(204.61,366.21997803)
\lineto(204.61,365.21997803)
\lineto(206.61,365.21997803)
\closepath
\moveto(202.61,366.21997803)
\lineto(200.61,366.21997803)
\lineto(200.61,365.21997803)
\lineto(202.61,365.21997803)
\closepath
\moveto(198.61,366.21997803)
\lineto(196.61,366.21997803)
\lineto(196.61,365.21997803)
\lineto(198.61,365.21997803)
\closepath
\moveto(194.61,366.21997803)
\lineto(192.61,366.21997803)
\lineto(192.61,365.21997803)
\lineto(194.61,365.21997803)
\closepath
\moveto(190.61,366.21997803)
\lineto(188.61,366.21997803)
\lineto(188.61,365.21997803)
\lineto(190.61,365.21997803)
\closepath
\moveto(186.61,366.21997803)
\lineto(184.61,366.21997803)
\lineto(184.61,365.21997803)
\lineto(186.61,365.21997803)
\closepath
\moveto(182.61,366.21997803)
\lineto(180.61,366.21997803)
\lineto(180.61,365.21997803)
\lineto(182.61,365.21997803)
\closepath
\moveto(178.61,366.21997803)
\lineto(176.61,366.21997803)
\lineto(176.61,365.21997803)
\lineto(178.61,365.21997803)
\closepath
\moveto(174.61,366.21997803)
\lineto(172.61,366.21997803)
\lineto(172.61,365.21997803)
\lineto(174.61,365.21997803)
\closepath
\moveto(170.61,366.21997803)
\lineto(168.61,366.21997803)
\lineto(168.61,365.21997803)
\lineto(170.61,365.21997803)
\closepath
\moveto(166.61,366.21997803)
\lineto(164.61,366.21997803)
\lineto(164.61,365.21997803)
\lineto(166.61,365.21997803)
\closepath
\moveto(162.61,366.21997803)
\lineto(160.61,366.21997803)
\lineto(160.61,365.21997803)
\lineto(162.61,365.21997803)
\closepath
\moveto(158.61,366.21997803)
\lineto(156.61,366.21997803)
\lineto(156.61,365.21997803)
\lineto(158.61,365.21997803)
\closepath
\moveto(154.61,366.21997803)
\lineto(152.61,366.21997803)
\lineto(152.61,365.21997803)
\lineto(154.61,365.21997803)
\closepath
\moveto(150.61,366.21997803)
\lineto(148.88,366.21997803)
\lineto(148.88,365.21997803)
\lineto(150.61,365.21997803)
\closepath
\moveto(149.88,363.94997803)
\lineto(148.88,363.94997803)
\lineto(148.88,361.94997803)
\lineto(149.88,361.94997803)
\closepath
\moveto(149.88,359.94997803)
\lineto(148.88,359.94997803)
\lineto(148.88,357.94997803)
\lineto(149.88,357.94997803)
\closepath
\moveto(149.88,355.94997803)
\lineto(148.88,355.94997803)
\lineto(148.88,353.94997803)
\lineto(149.88,353.94997803)
\closepath
\moveto(149.88,351.94997803)
\lineto(148.88,351.94997803)
\lineto(148.88,349.94997803)
\lineto(149.88,349.94997803)
\closepath
\moveto(149.88,347.94997803)
\lineto(148.88,347.94997803)
\lineto(148.88,345.94997803)
\lineto(149.88,345.94997803)
\closepath
\moveto(149.88,343.94997803)
\lineto(148.88,343.94997803)
\lineto(148.88,341.94997803)
\lineto(149.88,341.94997803)
\closepath
\moveto(149.88,339.94997803)
\lineto(148.88,339.94997803)
\lineto(148.88,337.94997803)
\lineto(149.88,337.94997803)
\closepath
\moveto(149.88,335.94997803)
\lineto(148.88,335.94997803)
\lineto(148.88,333.94997803)
\lineto(149.88,333.94997803)
\closepath
\moveto(149.88,331.94997803)
\lineto(148.88,331.94997803)
\lineto(148.88,329.94997803)
\lineto(149.88,329.94997803)
\closepath
\moveto(149.88,327.94997803)
\lineto(148.88,327.94997803)
\lineto(148.88,325.94997803)
\lineto(149.88,325.94997803)
\closepath
\moveto(149.88,323.94997803)
\lineto(148.88,323.94997803)
\lineto(148.88,321.94997803)
\lineto(149.88,321.94997803)
\closepath
\moveto(149.88,319.94997803)
\lineto(148.88,319.94997803)
\lineto(148.88,317.94997803)
\lineto(149.88,317.94997803)
\closepath
\moveto(149.88,315.94997803)
\lineto(148.88,315.94997803)
\lineto(148.88,313.94997803)
\lineto(149.88,313.94997803)
\closepath
\moveto(149.88,311.94997803)
\lineto(148.88,311.94997803)
\lineto(148.88,309.94997803)
\lineto(149.88,309.94997803)
\closepath
\moveto(149.88,307.94997803)
\lineto(148.88,307.94997803)
\lineto(148.88,305.94997803)
\lineto(149.88,305.94997803)
\closepath
\moveto(150.55,304.61997803)
\lineto(148.88,304.61997803)
\lineto(148.88,303.94997803)
\lineto(148.88,303.94997803)
\lineto(148.88,303.94997803)
\lineto(148.88,303.94997803)
\lineto(148.88,303.61997803)
\lineto(150.55,303.61997803)
\closepath
\moveto(154.55,304.61997803)
\lineto(152.55,304.61997803)
\lineto(152.55,303.61997803)
\lineto(154.55,303.61997803)
\closepath
\moveto(158.55,304.61997803)
\lineto(156.55,304.61997803)
\lineto(156.55,303.61997803)
\lineto(158.55,303.61997803)
\closepath
\moveto(162.55,304.61997803)
\lineto(160.55,304.61997803)
\lineto(160.55,303.61997803)
\lineto(162.55,303.61997803)
\closepath
\moveto(166.55,304.61997803)
\lineto(164.55,304.61997803)
\lineto(164.55,303.61997803)
\lineto(166.55,303.61997803)
\closepath
\moveto(170.55,304.61997803)
\lineto(168.55,304.61997803)
\lineto(168.55,303.61997803)
\lineto(170.55,303.61997803)
\closepath
\moveto(174.55,304.61997803)
\lineto(172.55,304.61997803)
\lineto(172.55,303.61997803)
\lineto(174.55,303.61997803)
\closepath
\moveto(178.55,304.61997803)
\lineto(176.55,304.61997803)
\lineto(176.55,303.61997803)
\lineto(178.55,303.61997803)
\closepath
\moveto(182.55,304.61997803)
\lineto(180.55,304.61997803)
\lineto(180.55,303.61997803)
\lineto(182.55,303.61997803)
\closepath
\moveto(186.55,304.61997803)
\lineto(184.55,304.61997803)
\lineto(184.55,303.61997803)
\lineto(186.55,303.61997803)
\closepath
\moveto(190.55,304.61997803)
\lineto(188.55,304.61997803)
\lineto(188.55,303.61997803)
\lineto(190.55,303.61997803)
\closepath
\moveto(194.55,304.61997803)
\lineto(192.55,304.61997803)
\lineto(192.55,303.61997803)
\lineto(194.55,303.61997803)
\closepath
\moveto(198.55,304.61997803)
\lineto(196.55,304.61997803)
\lineto(196.55,303.61997803)
\lineto(198.55,303.61997803)
\closepath
\moveto(202.55,304.61997803)
\lineto(200.55,304.61997803)
\lineto(200.55,303.61997803)
\lineto(202.55,303.61997803)
\closepath
\moveto(206.55,304.61997803)
\lineto(204.55,304.61997803)
\lineto(204.55,303.61997803)
\lineto(206.55,303.61997803)
\closepath
\moveto(210.55,304.61997803)
\lineto(208.55,304.61997803)
\lineto(208.55,303.61997803)
\lineto(210.55,303.61997803)
\closepath
\moveto(210.61,307.56997803)
\lineto(209.61,307.56997803)
\lineto(209.61,305.56997803)
\lineto(210.61,305.56997803)
\closepath
\moveto(210.61,311.56997803)
\lineto(209.61,311.56997803)
\lineto(209.61,309.56997803)
\lineto(210.61,309.56997803)
\closepath
\moveto(210.61,315.56997803)
\lineto(209.61,315.56997803)
\lineto(209.61,313.56997803)
\lineto(210.61,313.56997803)
\closepath
\moveto(210.61,319.56997803)
\lineto(209.61,319.56997803)
\lineto(209.61,317.56997803)
\lineto(210.61,317.56997803)
\closepath
\moveto(210.61,323.56997803)
\lineto(209.61,323.56997803)
\lineto(209.61,321.56997803)
\lineto(210.61,321.56997803)
\closepath
\moveto(210.61,327.56997803)
\lineto(209.61,327.56997803)
\lineto(209.61,325.56997803)
\lineto(210.61,325.56997803)
\closepath
\moveto(210.61,331.56997803)
\lineto(209.61,331.56997803)
\lineto(209.61,329.56997803)
\lineto(210.61,329.56997803)
\closepath
\moveto(210.61,335.56997803)
\lineto(209.61,335.56997803)
\lineto(209.61,333.56997803)
\lineto(210.61,333.56997803)
\closepath
\moveto(210.61,339.56997803)
\lineto(209.61,339.56997803)
\lineto(209.61,337.56997803)
\lineto(210.61,337.56997803)
\closepath
\moveto(210.61,343.56997803)
\lineto(209.61,343.56997803)
\lineto(209.61,341.56997803)
\lineto(210.61,341.56997803)
\closepath
\moveto(210.61,347.56997803)
\lineto(209.61,347.56997803)
\lineto(209.61,345.56997803)
\lineto(210.61,345.56997803)
\closepath
\moveto(210.61,351.56997803)
\lineto(209.61,351.56997803)
\lineto(209.61,349.56997803)
\lineto(210.61,349.56997803)
\closepath
\moveto(210.61,355.56997803)
\lineto(209.61,355.56997803)
\lineto(209.61,353.56997803)
\lineto(210.61,353.56997803)
\closepath
\moveto(210.61,359.56997803)
\lineto(209.61,359.56997803)
\lineto(209.61,357.56997803)
\lineto(210.61,357.56997803)
\closepath
\moveto(210.61,363.56997803)
\lineto(209.61,363.56997803)
\lineto(209.61,361.56997803)
\lineto(210.61,361.56997803)
\closepath
\moveto(210.61,366.21997803)
\lineto(208.61,366.21997803)
\lineto(208.61,365.21997803)
\lineto(210.61,365.21997803)
\closepath
}
}
{
\newrgbcolor{curcolor}{0 0 0}
\pscustom[linestyle=none,fillstyle=solid,fillcolor=curcolor]
{
\newpath
\moveto(294.7,421.59997803)
\lineto(294.7,302.82997803)
\lineto(148.14,302.82997803)
\lineto(148.14,421.59997803)
\lineto(294.7,421.59997803)
\moveto(295.7,422.59997803)
\lineto(147.14,422.59997803)
\lineto(147.14,301.82997803)
\lineto(295.7,301.82997803)
\closepath
}
}
{
\newrgbcolor{curcolor}{0 0 0}
\pscustom[linestyle=none,fillstyle=solid,fillcolor=curcolor]
{
\newpath
\moveto(417,539.53997803)
\lineto(417,301.14997803)
\lineto(145.53,301.14997803)
\lineto(145.53,539.53997803)
\lineto(417,539.53997803)
\moveto(418,540.53997803)
\lineto(144.53,540.53997803)
\lineto(144.53,300.14997803)
\lineto(418,300.14997803)
\closepath
}
}
{
\newrgbcolor{curcolor}{1 1 1}
\pscustom[linestyle=none,fillstyle=solid,fillcolor=curcolor]
{
\newpath
\moveto(375.05,152.63997803)
\lineto(453.3,111.95997803)
\lineto(484.19,32.47997803)
\closepath
}
}
{
\newrgbcolor{curcolor}{0 0 0}
\pscustom[linestyle=none,fillstyle=solid,fillcolor=curcolor]
{
\newpath
\moveto(377.3,150.89997803)
\lineto(412.79,132.45997803)
\lineto(452.9,111.59997803)
\lineto(468.73,70.86997803)
\lineto(482.73,34.81997803)
\lineto(430,92.92997803)
\lineto(377.34,150.92997803)
\moveto(372.84,154.39997803)
\lineto(429.26,92.28997803)
\lineto(485.67,30.17997803)
\lineto(469.67,71.26997803)
\lineto(453.67,112.34997803)
\lineto(413.23,133.34997803)
\lineto(372.78,154.34997803)
\closepath
}
}
{
\newrgbcolor{curcolor}{0 0 0}
\pscustom[linestyle=none,fillstyle=solid,fillcolor=curcolor]
{
\newpath
\moveto(486.5,155.34997803)
\lineto(484.5,155.34997803)
\lineto(484.5,154.34997803)
\lineto(486.5,154.34997803)
\closepath
\moveto(482.5,155.34997803)
\lineto(480.5,155.34997803)
\lineto(480.5,154.34997803)
\lineto(482.5,154.34997803)
\closepath
\moveto(478.5,155.34997803)
\lineto(476.5,155.34997803)
\lineto(476.5,154.34997803)
\lineto(478.5,154.34997803)
\closepath
\moveto(474.5,155.34997803)
\lineto(472.5,155.34997803)
\lineto(472.5,154.34997803)
\lineto(474.5,154.34997803)
\closepath
\moveto(470.5,155.34997803)
\lineto(468.5,155.34997803)
\lineto(468.5,154.34997803)
\lineto(470.5,154.34997803)
\closepath
\moveto(466.5,155.34997803)
\lineto(464.5,155.34997803)
\lineto(464.5,154.34997803)
\lineto(466.5,154.34997803)
\closepath
\moveto(462.5,155.34997803)
\lineto(460.5,155.34997803)
\lineto(460.5,154.34997803)
\lineto(462.5,154.34997803)
\closepath
\moveto(458.5,155.34997803)
\lineto(456.5,155.34997803)
\lineto(456.5,154.34997803)
\lineto(458.5,154.34997803)
\closepath
\moveto(454.5,155.34997803)
\lineto(452.5,155.34997803)
\lineto(452.5,154.34997803)
\lineto(454.5,154.34997803)
\closepath
\moveto(450.5,155.34997803)
\lineto(448.5,155.34997803)
\lineto(448.5,154.34997803)
\lineto(450.5,154.34997803)
\closepath
\moveto(446.5,155.34997803)
\lineto(444.5,155.34997803)
\lineto(444.5,154.34997803)
\lineto(446.5,154.34997803)
\closepath
\moveto(442.5,155.34997803)
\lineto(440.5,155.34997803)
\lineto(440.5,154.34997803)
\lineto(442.5,154.34997803)
\closepath
\moveto(438.5,155.34997803)
\lineto(436.5,155.34997803)
\lineto(436.5,154.34997803)
\lineto(438.5,154.34997803)
\closepath
\moveto(434.5,155.34997803)
\lineto(432.5,155.34997803)
\lineto(432.5,154.34997803)
\lineto(434.5,154.34997803)
\closepath
\moveto(430.5,155.34997803)
\lineto(428.5,155.34997803)
\lineto(428.5,154.34997803)
\lineto(430.5,154.34997803)
\closepath
\moveto(426.5,155.34997803)
\lineto(424.5,155.34997803)
\lineto(424.5,154.34997803)
\lineto(426.5,154.34997803)
\closepath
\moveto(422.5,155.34997803)
\lineto(420.5,155.34997803)
\lineto(420.5,154.34997803)
\lineto(422.5,154.34997803)
\closepath
\moveto(418.5,155.34997803)
\lineto(416.5,155.34997803)
\lineto(416.5,154.34997803)
\lineto(418.5,154.34997803)
\closepath
\moveto(414.5,155.34997803)
\lineto(412.5,155.34997803)
\lineto(412.5,154.34997803)
\lineto(414.5,154.34997803)
\closepath
\moveto(410.5,155.34997803)
\lineto(408.5,155.34997803)
\lineto(408.5,154.34997803)
\lineto(410.5,154.34997803)
\closepath
\moveto(406.5,155.34997803)
\lineto(404.5,155.34997803)
\lineto(404.5,154.34997803)
\lineto(406.5,154.34997803)
\closepath
\moveto(402.5,155.34997803)
\lineto(400.5,155.34997803)
\lineto(400.5,154.34997803)
\lineto(402.5,154.34997803)
\closepath
\moveto(398.5,155.34997803)
\lineto(396.5,155.34997803)
\lineto(396.5,154.34997803)
\lineto(398.5,154.34997803)
\closepath
\moveto(394.5,155.34997803)
\lineto(392.5,155.34997803)
\lineto(392.5,154.34997803)
\lineto(394.5,154.34997803)
\closepath
\moveto(390.5,155.34997803)
\lineto(388.5,155.34997803)
\lineto(388.5,154.34997803)
\lineto(390.5,154.34997803)
\closepath
\moveto(386.5,155.34997803)
\lineto(384.5,155.34997803)
\lineto(384.5,154.34997803)
\lineto(386.5,154.34997803)
\closepath
\moveto(382.5,155.34997803)
\lineto(380.5,155.34997803)
\lineto(380.5,154.34997803)
\lineto(382.5,154.34997803)
\closepath
\moveto(378.5,155.34997803)
\lineto(376.5,155.34997803)
\lineto(376.5,154.34997803)
\lineto(378.5,154.34997803)
\closepath
\moveto(374.5,155.34997803)
\lineto(372.5,155.34997803)
\lineto(372.5,154.34997803)
\lineto(374.5,154.34997803)
\closepath
\moveto(372.74,154.10997803)
\lineto(371.74,154.10997803)
\lineto(371.74,152.10997803)
\lineto(372.74,152.10997803)
\closepath
\moveto(372.74,150.10997803)
\lineto(371.74,150.10997803)
\lineto(371.74,148.10997803)
\lineto(372.74,148.10997803)
\closepath
\moveto(372.74,146.10997803)
\lineto(371.74,146.10997803)
\lineto(371.74,144.10997803)
\lineto(372.74,144.10997803)
\closepath
\moveto(372.74,142.10997803)
\lineto(371.74,142.10997803)
\lineto(371.74,140.10997803)
\lineto(372.74,140.10997803)
\closepath
\moveto(372.74,138.10997803)
\lineto(371.74,138.10997803)
\lineto(371.74,136.10997803)
\lineto(372.74,136.10997803)
\closepath
\moveto(372.74,134.10997803)
\lineto(371.74,134.10997803)
\lineto(371.74,132.10997803)
\lineto(372.74,132.10997803)
\closepath
\moveto(372.74,130.10997803)
\lineto(371.74,130.10997803)
\lineto(371.74,128.10997803)
\lineto(372.74,128.10997803)
\closepath
\moveto(372.74,126.10997803)
\lineto(371.74,126.10997803)
\lineto(371.74,124.10997803)
\lineto(372.74,124.10997803)
\closepath
\moveto(372.74,122.10997803)
\lineto(371.74,122.10997803)
\lineto(371.74,120.10997803)
\lineto(372.74,120.10997803)
\closepath
\moveto(372.74,118.10997803)
\lineto(371.74,118.10997803)
\lineto(371.74,116.10997803)
\lineto(372.74,116.10997803)
\closepath
\moveto(372.74,114.10997803)
\lineto(371.74,114.10997803)
\lineto(371.74,112.10997803)
\lineto(372.74,112.10997803)
\closepath
\moveto(372.74,110.10997803)
\lineto(371.74,110.10997803)
\lineto(371.74,108.10997803)
\lineto(372.74,108.10997803)
\closepath
\moveto(372.74,106.10997803)
\lineto(371.74,106.10997803)
\lineto(371.74,104.10997803)
\lineto(372.74,104.10997803)
\closepath
\moveto(372.74,102.10997803)
\lineto(371.74,102.10997803)
\lineto(371.74,100.10997803)
\lineto(372.74,100.10997803)
\closepath
\moveto(372.74,98.10997803)
\lineto(371.74,98.10997803)
\lineto(371.74,96.10997803)
\lineto(372.74,96.10997803)
\closepath
\moveto(372.74,94.10997803)
\lineto(371.74,94.10997803)
\lineto(371.74,92.10997803)
\lineto(372.74,92.10997803)
\closepath
\moveto(372.74,90.10997803)
\lineto(371.74,90.10997803)
\lineto(371.74,88.10997803)
\lineto(372.74,88.10997803)
\closepath
\moveto(372.74,86.10997803)
\lineto(371.74,86.10997803)
\lineto(371.74,84.10997803)
\lineto(372.74,84.10997803)
\closepath
\moveto(372.74,82.10997803)
\lineto(371.74,82.10997803)
\lineto(371.74,80.10997803)
\lineto(372.74,80.10997803)
\closepath
\moveto(372.74,78.10997803)
\lineto(371.74,78.10997803)
\lineto(371.74,76.10997803)
\lineto(372.74,76.10997803)
\closepath
\moveto(372.74,74.10997803)
\lineto(371.74,74.10997803)
\lineto(371.74,72.10997803)
\lineto(372.74,72.10997803)
\closepath
\moveto(372.74,70.10997803)
\lineto(371.74,70.10997803)
\lineto(371.74,68.10997803)
\lineto(372.74,68.10997803)
\closepath
\moveto(372.74,66.10997803)
\lineto(371.74,66.10997803)
\lineto(371.74,64.10997803)
\lineto(372.74,64.10997803)
\closepath
\moveto(372.74,62.10997803)
\lineto(371.74,62.10997803)
\lineto(371.74,60.10997803)
\lineto(372.74,60.10997803)
\closepath
\moveto(372.74,58.10997803)
\lineto(371.74,58.10997803)
\lineto(371.74,56.10997803)
\lineto(372.74,56.10997803)
\closepath
\moveto(372.74,54.10997803)
\lineto(371.74,54.10997803)
\lineto(371.74,52.10997803)
\lineto(372.74,52.10997803)
\closepath
\moveto(372.74,50.10997803)
\lineto(371.74,50.10997803)
\lineto(371.74,48.10997803)
\lineto(372.74,48.10997803)
\closepath
\moveto(372.74,46.10997803)
\lineto(371.74,46.10997803)
\lineto(371.74,44.10997803)
\lineto(372.74,44.10997803)
\closepath
\moveto(372.74,42.10997803)
\lineto(371.74,42.10997803)
\lineto(371.74,40.10997803)
\lineto(372.74,40.10997803)
\closepath
\moveto(372.74,38.10997803)
\lineto(371.74,38.10997803)
\lineto(371.74,36.10997803)
\lineto(372.74,36.10997803)
\closepath
\moveto(372.74,34.10997803)
\lineto(371.74,34.10997803)
\lineto(371.74,32.10997803)
\lineto(372.74,32.10997803)
\closepath
\moveto(374.21,31.58997803)
\lineto(372.21,31.58997803)
\lineto(372.21,30.58997803)
\lineto(374.21,30.58997803)
\closepath
\moveto(378.21,31.58997803)
\lineto(376.21,31.58997803)
\lineto(376.21,30.58997803)
\lineto(378.21,30.58997803)
\closepath
\moveto(382.21,31.58997803)
\lineto(380.21,31.58997803)
\lineto(380.21,30.58997803)
\lineto(382.21,30.58997803)
\closepath
\moveto(386.21,31.58997803)
\lineto(384.21,31.58997803)
\lineto(384.21,30.58997803)
\lineto(386.21,30.58997803)
\closepath
\moveto(390.21,31.58997803)
\lineto(388.21,31.58997803)
\lineto(388.21,30.58997803)
\lineto(390.21,30.58997803)
\closepath
\moveto(394.21,31.58997803)
\lineto(392.21,31.58997803)
\lineto(392.21,30.58997803)
\lineto(394.21,30.58997803)
\closepath
\moveto(398.21,31.58997803)
\lineto(396.21,31.58997803)
\lineto(396.21,30.58997803)
\lineto(398.21,30.58997803)
\closepath
\moveto(402.21,31.58997803)
\lineto(400.21,31.58997803)
\lineto(400.21,30.58997803)
\lineto(402.21,30.58997803)
\closepath
\moveto(406.21,31.58997803)
\lineto(404.21,31.58997803)
\lineto(404.21,30.58997803)
\lineto(406.21,30.58997803)
\closepath
\moveto(410.21,31.58997803)
\lineto(408.21,31.58997803)
\lineto(408.21,30.58997803)
\lineto(410.21,30.58997803)
\closepath
\moveto(414.21,31.58997803)
\lineto(412.21,31.58997803)
\lineto(412.21,30.58997803)
\lineto(414.21,30.58997803)
\closepath
\moveto(418.21,31.58997803)
\lineto(416.21,31.58997803)
\lineto(416.21,30.58997803)
\lineto(418.21,30.58997803)
\closepath
\moveto(422.21,31.58997803)
\lineto(420.21,31.58997803)
\lineto(420.21,30.58997803)
\lineto(422.21,30.58997803)
\closepath
\moveto(426.21,31.58997803)
\lineto(424.21,31.58997803)
\lineto(424.21,30.58997803)
\lineto(426.21,30.58997803)
\closepath
\moveto(430.21,31.58997803)
\lineto(428.21,31.58997803)
\lineto(428.21,30.58997803)
\lineto(430.21,30.58997803)
\closepath
\moveto(434.21,31.58997803)
\lineto(432.21,31.58997803)
\lineto(432.21,30.58997803)
\lineto(434.21,30.58997803)
\closepath
\moveto(438.21,31.58997803)
\lineto(436.21,31.58997803)
\lineto(436.21,30.58997803)
\lineto(438.21,30.58997803)
\closepath
\moveto(442.21,31.58997803)
\lineto(440.21,31.58997803)
\lineto(440.21,30.58997803)
\lineto(442.21,30.58997803)
\closepath
\moveto(446.21,31.58997803)
\lineto(444.21,31.58997803)
\lineto(444.21,30.58997803)
\lineto(446.21,30.58997803)
\closepath
\moveto(450.21,31.58997803)
\lineto(448.21,31.58997803)
\lineto(448.21,30.58997803)
\lineto(450.21,30.58997803)
\closepath
\moveto(454.21,31.58997803)
\lineto(452.21,31.58997803)
\lineto(452.21,30.58997803)
\lineto(454.21,30.58997803)
\closepath
\moveto(458.21,31.58997803)
\lineto(456.21,31.58997803)
\lineto(456.21,30.58997803)
\lineto(458.21,30.58997803)
\closepath
\moveto(462.21,31.58997803)
\lineto(460.21,31.58997803)
\lineto(460.21,30.58997803)
\lineto(462.21,30.58997803)
\closepath
\moveto(466.21,31.58997803)
\lineto(464.21,31.58997803)
\lineto(464.21,30.58997803)
\lineto(466.21,30.58997803)
\closepath
\moveto(470.21,31.58997803)
\lineto(468.21,31.58997803)
\lineto(468.21,30.58997803)
\lineto(470.21,30.58997803)
\closepath
\moveto(474.21,31.58997803)
\lineto(472.21,31.58997803)
\lineto(472.21,30.58997803)
\lineto(474.21,30.58997803)
\closepath
\moveto(478.21,31.58997803)
\lineto(476.21,31.58997803)
\lineto(476.21,30.58997803)
\lineto(478.21,30.58997803)
\closepath
\moveto(482.21,31.58997803)
\lineto(480.21,31.58997803)
\lineto(480.21,30.58997803)
\lineto(482.21,30.58997803)
\closepath
\moveto(486.21,31.58997803)
\lineto(484.21,31.58997803)
\lineto(484.21,30.58997803)
\lineto(486.21,30.58997803)
\closepath
\moveto(486.5,34.29997803)
\lineto(485.5,34.29997803)
\lineto(485.5,32.29997803)
\lineto(486.5,32.29997803)
\closepath
\moveto(486.5,38.29997803)
\lineto(485.5,38.29997803)
\lineto(485.5,36.29997803)
\lineto(486.5,36.29997803)
\closepath
\moveto(486.5,42.29997803)
\lineto(485.5,42.29997803)
\lineto(485.5,40.29997803)
\lineto(486.5,40.29997803)
\closepath
\moveto(486.5,46.29997803)
\lineto(485.5,46.29997803)
\lineto(485.5,44.29997803)
\lineto(486.5,44.29997803)
\closepath
\moveto(486.5,50.29997803)
\lineto(485.5,50.29997803)
\lineto(485.5,48.29997803)
\lineto(486.5,48.29997803)
\closepath
\moveto(486.5,54.29997803)
\lineto(485.5,54.29997803)
\lineto(485.5,52.29997803)
\lineto(486.5,52.29997803)
\closepath
\moveto(486.5,58.29997803)
\lineto(485.5,58.29997803)
\lineto(485.5,56.29997803)
\lineto(486.5,56.29997803)
\closepath
\moveto(486.5,62.29997803)
\lineto(485.5,62.29997803)
\lineto(485.5,60.29997803)
\lineto(486.5,60.29997803)
\closepath
\moveto(486.5,66.29997803)
\lineto(485.5,66.29997803)
\lineto(485.5,64.29997803)
\lineto(486.5,64.29997803)
\closepath
\moveto(486.5,70.29997803)
\lineto(485.5,70.29997803)
\lineto(485.5,68.29997803)
\lineto(486.5,68.29997803)
\closepath
\moveto(486.5,74.29997803)
\lineto(485.5,74.29997803)
\lineto(485.5,72.29997803)
\lineto(486.5,72.29997803)
\closepath
\moveto(486.5,78.29997803)
\lineto(485.5,78.29997803)
\lineto(485.5,76.29997803)
\lineto(486.5,76.29997803)
\closepath
\moveto(486.5,82.29997803)
\lineto(485.5,82.29997803)
\lineto(485.5,80.29997803)
\lineto(486.5,80.29997803)
\closepath
\moveto(486.5,86.29997803)
\lineto(485.5,86.29997803)
\lineto(485.5,84.29997803)
\lineto(486.5,84.29997803)
\closepath
\moveto(486.5,90.29997803)
\lineto(485.5,90.29997803)
\lineto(485.5,88.29997803)
\lineto(486.5,88.29997803)
\closepath
\moveto(486.5,94.29997803)
\lineto(485.5,94.29997803)
\lineto(485.5,92.29997803)
\lineto(486.5,92.29997803)
\closepath
\moveto(486.5,98.29997803)
\lineto(485.5,98.29997803)
\lineto(485.5,96.29997803)
\lineto(486.5,96.29997803)
\closepath
\moveto(486.5,102.29997803)
\lineto(485.5,102.29997803)
\lineto(485.5,100.29997803)
\lineto(486.5,100.29997803)
\closepath
\moveto(486.5,106.29997803)
\lineto(485.5,106.29997803)
\lineto(485.5,104.29997803)
\lineto(486.5,104.29997803)
\closepath
\moveto(486.5,110.29997803)
\lineto(485.5,110.29997803)
\lineto(485.5,108.29997803)
\lineto(486.5,108.29997803)
\closepath
\moveto(486.5,114.29997803)
\lineto(485.5,114.29997803)
\lineto(485.5,112.29997803)
\lineto(486.5,112.29997803)
\closepath
\moveto(486.5,118.29997803)
\lineto(485.5,118.29997803)
\lineto(485.5,116.29997803)
\lineto(486.5,116.29997803)
\closepath
\moveto(486.5,122.29997803)
\lineto(485.5,122.29997803)
\lineto(485.5,120.29997803)
\lineto(486.5,120.29997803)
\closepath
\moveto(486.5,126.29997803)
\lineto(485.5,126.29997803)
\lineto(485.5,124.29997803)
\lineto(486.5,124.29997803)
\closepath
\moveto(486.5,130.29997803)
\lineto(485.5,130.29997803)
\lineto(485.5,128.29997803)
\lineto(486.5,128.29997803)
\closepath
\moveto(486.5,134.29997803)
\lineto(485.5,134.29997803)
\lineto(485.5,132.29997803)
\lineto(486.5,132.29997803)
\closepath
\moveto(486.5,138.29997803)
\lineto(485.5,138.29997803)
\lineto(485.5,136.29997803)
\lineto(486.5,136.29997803)
\closepath
\moveto(486.5,142.29997803)
\lineto(485.5,142.29997803)
\lineto(485.5,140.29997803)
\lineto(486.5,140.29997803)
\closepath
\moveto(486.5,146.29997803)
\lineto(485.5,146.29997803)
\lineto(485.5,144.29997803)
\lineto(486.5,144.29997803)
\closepath
\moveto(486.5,150.29997803)
\lineto(485.5,150.29997803)
\lineto(485.5,148.29997803)
\lineto(486.5,148.29997803)
\closepath
\moveto(486.5,154.29997803)
\lineto(485.5,154.29997803)
\lineto(485.5,152.29997803)
\lineto(486.5,152.29997803)
\closepath
}
}
{
\newrgbcolor{curcolor}{1 1 1}
\pscustom[linestyle=none,fillstyle=solid,fillcolor=curcolor]
{
\newpath
\moveto(183.12,82.65997803)
\lineto(248.53,61.90997803)
\lineto(197.86,15.64997803)
\closepath
}
}
{
\newrgbcolor{curcolor}{0 0 0}
\pscustom[linestyle=none,fillstyle=solid,fillcolor=curcolor]
{
\newpath
\moveto(183.8,81.91997803)
\lineto(215.68,71.80997803)
\lineto(247.55,61.69997803)
\lineto(222.86,39.14997803)
\lineto(198.16,16.59997803)
\lineto(191,49.25997803)
\lineto(183.82,81.91997803)
\moveto(182.47,83.39997803)
\lineto(190,49.04997803)
\lineto(197.55,14.68997803)
\lineto(223.55,38.40997803)
\lineto(249.55,62.12997803)
\lineto(216,72.75997803)
\closepath
}
}
{
\newrgbcolor{curcolor}{0 0 0}
\pscustom[linestyle=none,fillstyle=solid,fillcolor=curcolor]
{
\newpath
\moveto(250.7,84.26997803)
\lineto(248.7,84.26997803)
\lineto(248.7,83.26997803)
\lineto(249.7,83.26997803)
\lineto(249.7,81.70997803)
\lineto(250.7,81.70997803)
\closepath
\moveto(246.7,84.26997803)
\lineto(244.7,84.26997803)
\lineto(244.7,83.26997803)
\lineto(246.7,83.26997803)
\closepath
\moveto(242.7,84.26997803)
\lineto(240.7,84.26997803)
\lineto(240.7,83.26997803)
\lineto(242.7,83.26997803)
\closepath
\moveto(238.7,84.26997803)
\lineto(236.7,84.26997803)
\lineto(236.7,83.26997803)
\lineto(238.7,83.26997803)
\closepath
\moveto(234.7,84.26997803)
\lineto(232.7,84.26997803)
\lineto(232.7,83.26997803)
\lineto(234.7,83.26997803)
\closepath
\moveto(230.7,84.26997803)
\lineto(228.7,84.26997803)
\lineto(228.7,83.26997803)
\lineto(230.7,83.26997803)
\closepath
\moveto(226.7,84.26997803)
\lineto(224.7,84.26997803)
\lineto(224.7,83.26997803)
\lineto(226.7,83.26997803)
\closepath
\moveto(222.7,84.26997803)
\lineto(220.7,84.26997803)
\lineto(220.7,83.26997803)
\lineto(222.7,83.26997803)
\closepath
\moveto(218.7,84.26997803)
\lineto(216.7,84.26997803)
\lineto(216.7,83.26997803)
\lineto(218.7,83.26997803)
\closepath
\moveto(214.7,84.26997803)
\lineto(212.7,84.26997803)
\lineto(212.7,83.26997803)
\lineto(214.7,83.26997803)
\closepath
\moveto(210.7,84.26997803)
\lineto(208.7,84.26997803)
\lineto(208.7,83.26997803)
\lineto(210.7,83.26997803)
\closepath
\moveto(206.7,84.26997803)
\lineto(204.7,84.26997803)
\lineto(204.7,83.26997803)
\lineto(206.7,83.26997803)
\closepath
\moveto(202.7,84.26997803)
\lineto(200.7,84.26997803)
\lineto(200.7,83.26997803)
\lineto(202.7,83.26997803)
\closepath
\moveto(198.7,84.26997803)
\lineto(196.7,84.26997803)
\lineto(196.7,83.26997803)
\lineto(198.7,83.26997803)
\closepath
\moveto(194.7,84.26997803)
\lineto(192.7,84.26997803)
\lineto(192.7,83.26997803)
\lineto(194.7,83.26997803)
\closepath
\moveto(190.7,84.26997803)
\lineto(188.7,84.26997803)
\lineto(188.7,83.26997803)
\lineto(190.7,83.26997803)
\closepath
\moveto(186.7,84.26997803)
\lineto(184.7,84.26997803)
\lineto(184.7,83.26997803)
\lineto(186.7,83.26997803)
\closepath
\moveto(182.7,84.26997803)
\lineto(180.7,84.26997803)
\lineto(180.7,83.26997803)
\lineto(182.7,83.26997803)
\closepath
\moveto(180.84,83.12997803)
\lineto(179.84,83.12997803)
\lineto(179.84,81.12997803)
\lineto(180.84,81.12997803)
\closepath
\moveto(180.84,79.12997803)
\lineto(179.84,79.12997803)
\lineto(179.84,77.12997803)
\lineto(180.84,77.12997803)
\closepath
\moveto(180.84,75.12997803)
\lineto(179.84,75.12997803)
\lineto(179.84,73.12997803)
\lineto(180.84,73.12997803)
\closepath
\moveto(180.84,71.12997803)
\lineto(179.84,71.12997803)
\lineto(179.84,69.12997803)
\lineto(180.84,69.12997803)
\closepath
\moveto(180.84,67.12997803)
\lineto(179.84,67.12997803)
\lineto(179.84,65.12997803)
\lineto(180.84,65.12997803)
\closepath
\moveto(180.84,63.12997803)
\lineto(179.84,63.12997803)
\lineto(179.84,61.12997803)
\lineto(180.84,61.12997803)
\closepath
\moveto(180.84,59.12997803)
\lineto(179.84,59.12997803)
\lineto(179.84,57.12997803)
\lineto(180.84,57.12997803)
\closepath
\moveto(180.84,55.12997803)
\lineto(179.84,55.12997803)
\lineto(179.84,53.12997803)
\lineto(180.84,53.12997803)
\closepath
\moveto(180.84,51.12997803)
\lineto(179.84,51.12997803)
\lineto(179.84,49.12997803)
\lineto(180.84,49.12997803)
\closepath
\moveto(180.84,47.12997803)
\lineto(179.84,47.12997803)
\lineto(179.84,45.12997803)
\lineto(180.84,45.12997803)
\closepath
\moveto(180.84,43.12997803)
\lineto(179.84,43.12997803)
\lineto(179.84,41.12997803)
\lineto(180.84,41.12997803)
\closepath
\moveto(180.84,39.12997803)
\lineto(179.84,39.12997803)
\lineto(179.84,37.12997803)
\lineto(180.84,37.12997803)
\closepath
\moveto(180.84,35.12997803)
\lineto(179.84,35.12997803)
\lineto(179.84,33.12997803)
\lineto(180.84,33.12997803)
\closepath
\moveto(180.84,31.12997803)
\lineto(179.84,31.12997803)
\lineto(179.84,29.12997803)
\lineto(180.84,29.12997803)
\closepath
\moveto(180.84,27.12997803)
\lineto(179.84,27.12997803)
\lineto(179.84,25.12997803)
\lineto(180.84,25.12997803)
\closepath
\moveto(180.84,23.12997803)
\lineto(179.84,23.12997803)
\lineto(179.84,21.12997803)
\lineto(180.84,21.12997803)
\closepath
\moveto(180.84,19.12997803)
\lineto(179.84,19.12997803)
\lineto(179.84,17.12997803)
\lineto(180.84,17.12997803)
\closepath
\moveto(180.84,15.12997803)
\lineto(179.84,15.12997803)
\lineto(179.84,13.84997803)
\lineto(180.84,13.84997803)
\closepath
\moveto(184.56,14.84997803)
\lineto(182.56,14.84997803)
\lineto(182.56,13.84997803)
\lineto(184.56,13.84997803)
\closepath
\moveto(188.56,14.84997803)
\lineto(186.56,14.84997803)
\lineto(186.56,13.84997803)
\lineto(188.56,13.84997803)
\closepath
\moveto(192.56,14.84997803)
\lineto(190.56,14.84997803)
\lineto(190.56,13.84997803)
\lineto(192.56,13.84997803)
\closepath
\moveto(196.56,14.84997803)
\lineto(194.56,14.84997803)
\lineto(194.56,13.84997803)
\lineto(196.56,13.84997803)
\closepath
\moveto(200.56,14.84997803)
\lineto(198.56,14.84997803)
\lineto(198.56,13.84997803)
\lineto(200.56,13.84997803)
\closepath
\moveto(204.56,14.84997803)
\lineto(202.56,14.84997803)
\lineto(202.56,13.84997803)
\lineto(204.56,13.84997803)
\closepath
\moveto(208.56,14.84997803)
\lineto(206.56,14.84997803)
\lineto(206.56,13.84997803)
\lineto(208.56,13.84997803)
\closepath
\moveto(212.56,14.84997803)
\lineto(210.56,14.84997803)
\lineto(210.56,13.84997803)
\lineto(212.56,13.84997803)
\closepath
\moveto(216.56,14.84997803)
\lineto(214.56,14.84997803)
\lineto(214.56,13.84997803)
\lineto(216.56,13.84997803)
\closepath
\moveto(220.56,14.84997803)
\lineto(218.56,14.84997803)
\lineto(218.56,13.84997803)
\lineto(220.56,13.84997803)
\closepath
\moveto(224.56,14.84997803)
\lineto(222.56,14.84997803)
\lineto(222.56,13.84997803)
\lineto(224.56,13.84997803)
\closepath
\moveto(228.56,14.84997803)
\lineto(226.56,14.84997803)
\lineto(226.56,13.84997803)
\lineto(228.56,13.84997803)
\closepath
\moveto(232.56,14.84997803)
\lineto(230.56,14.84997803)
\lineto(230.56,13.84997803)
\lineto(232.56,13.84997803)
\closepath
\moveto(236.56,14.84997803)
\lineto(234.56,14.84997803)
\lineto(234.56,13.84997803)
\lineto(236.56,13.84997803)
\closepath
\moveto(240.56,14.84997803)
\lineto(238.56,14.84997803)
\lineto(238.56,13.84997803)
\lineto(240.56,13.84997803)
\closepath
\moveto(244.56,14.84997803)
\lineto(242.56,14.84997803)
\lineto(242.56,13.84997803)
\lineto(244.56,13.84997803)
\closepath
\moveto(248.56,14.84997803)
\lineto(246.56,14.84997803)
\lineto(246.56,13.84997803)
\lineto(248.56,13.84997803)
\closepath
\moveto(250.7,15.70997803)
\lineto(249.7,15.70997803)
\lineto(249.7,13.84997803)
\lineto(250.7,13.84997803)
\closepath
\moveto(250.7,19.70997803)
\lineto(249.7,19.70997803)
\lineto(249.7,17.70997803)
\lineto(250.7,17.70997803)
\closepath
\moveto(250.7,23.70997803)
\lineto(249.7,23.70997803)
\lineto(249.7,21.70997803)
\lineto(250.7,21.70997803)
\closepath
\moveto(250.7,27.70997803)
\lineto(249.7,27.70997803)
\lineto(249.7,25.70997803)
\lineto(250.7,25.70997803)
\closepath
\moveto(250.7,31.70997803)
\lineto(249.7,31.70997803)
\lineto(249.7,29.70997803)
\lineto(250.7,29.70997803)
\closepath
\moveto(250.7,35.70997803)
\lineto(249.7,35.70997803)
\lineto(249.7,33.70997803)
\lineto(250.7,33.70997803)
\closepath
\moveto(250.7,39.70997803)
\lineto(249.7,39.70997803)
\lineto(249.7,37.70997803)
\lineto(250.7,37.70997803)
\closepath
\moveto(250.7,43.70997803)
\lineto(249.7,43.70997803)
\lineto(249.7,41.70997803)
\lineto(250.7,41.70997803)
\closepath
\moveto(250.7,47.70997803)
\lineto(249.7,47.70997803)
\lineto(249.7,45.70997803)
\lineto(250.7,45.70997803)
\closepath
\moveto(250.7,51.70997803)
\lineto(249.7,51.70997803)
\lineto(249.7,49.70997803)
\lineto(250.7,49.70997803)
\closepath
\moveto(250.7,55.70997803)
\lineto(249.7,55.70997803)
\lineto(249.7,53.70997803)
\lineto(250.7,53.70997803)
\closepath
\moveto(250.7,59.70997803)
\lineto(249.7,59.70997803)
\lineto(249.7,57.70997803)
\lineto(250.7,57.70997803)
\closepath
\moveto(250.7,63.70997803)
\lineto(249.7,63.70997803)
\lineto(249.7,61.70997803)
\lineto(250.7,61.70997803)
\closepath
\moveto(250.7,67.70997803)
\lineto(249.7,67.70997803)
\lineto(249.7,65.70997803)
\lineto(250.7,65.70997803)
\closepath
\moveto(250.7,71.70997803)
\lineto(249.7,71.70997803)
\lineto(249.7,69.70997803)
\lineto(250.7,69.70997803)
\closepath
\moveto(250.7,75.70997803)
\lineto(249.7,75.70997803)
\lineto(249.7,73.70997803)
\lineto(250.7,73.70997803)
\closepath
\moveto(250.7,79.70997803)
\lineto(249.7,79.70997803)
\lineto(249.7,77.70997803)
\lineto(250.7,77.70997803)
\closepath
}
}
{
\newrgbcolor{curcolor}{1 1 1}
\pscustom[linestyle=none,fillstyle=solid,fillcolor=curcolor]
{
\newpath
\moveto(93.3999958,52.9699707)
\curveto(93.3999958,64.72363064)(86.31971502,75.32003162)(75.46094384,79.81770797)
\curveto(64.60219481,84.31537515)(52.10357434,81.8285064)(43.79251749,73.51744955)
\curveto(25.48668498,55.21161704)(38.44850126,23.90997124)(64.33999634,23.90997124)
\curveto(76.09365628,23.90997124)(86.69005726,30.99025202)(91.18773361,41.8490232)
\curveto(95.68540078,52.70777224)(93.19853204,65.2063927)(84.88747519,73.51744955)
\curveto(76.57641834,81.8285064)(64.07779787,84.31537515)(53.21904884,79.81770797)
\curveto(42.36027766,75.32003162)(35.27999687,64.72363064)(35.27999687,52.9699707)
\curveto(35.27999687,27.07847563)(66.58164268,14.11665934)(84.88747519,32.42249185)
\curveto(93.19853204,40.7335487)(95.68540078,53.23216917)(91.18773361,64.0909182)
\curveto(86.69005726,74.94968938)(76.09365628,82.02997017)(64.33999634,82.02997017)
\curveto(38.44850126,82.02997017)(25.48668498,50.72832437)(43.79251749,32.42249185)
\curveto(62.09835,14.11665934)(93.3999958,27.07847563)(93.3999958,52.9699707)
\closepath
}
}
{
\newrgbcolor{curcolor}{0 0 0}
\pscustom[linestyle=none,fillstyle=solid,fillcolor=curcolor]
{
\newpath
\moveto(64.34,81.53997803)
\curveto(48.56674755,81.53997803)(35.78,68.75323048)(35.78,52.97997803)
\curveto(35.75326983,27.49278025)(66.56515246,14.71486075)(84.58513487,32.73484316)
\curveto(92.76563229,40.91534058)(95.21037585,53.21908404)(90.77776001,63.90492781)
\curveto(86.34513507,74.5907935)(75.90901415,81.55211125)(64.34,81.53997803)
\moveto(64.34,82.53997803)
\curveto(80.66944324,82.53997896)(93.90552417,69.29942033)(93.9,52.96997803)
\curveto(93.89109115,26.63541353)(62.05139779,13.46025376)(43.43545169,32.08249754)
\curveto(24.81950559,50.70474132)(38.005434,82.53997652)(64.34,82.53997803)
\closepath
}
}
{
\newrgbcolor{curcolor}{0 0 0}
\pscustom[linestyle=none,fillstyle=solid,fillcolor=curcolor]
{
\newpath
\moveto(90.77,84.26997803)
\lineto(88.77,84.26997803)
\lineto(88.77,83.26997803)
\lineto(90.77,83.26997803)
\closepath
\moveto(86.77,84.26997803)
\lineto(84.77,84.26997803)
\lineto(84.77,83.26997803)
\lineto(86.77,83.26997803)
\closepath
\moveto(82.77,84.26997803)
\lineto(80.77,84.26997803)
\lineto(80.77,83.26997803)
\lineto(82.77,83.26997803)
\closepath
\moveto(78.77,84.26997803)
\lineto(76.77,84.26997803)
\lineto(76.77,83.26997803)
\lineto(78.77,83.26997803)
\closepath
\moveto(74.77,84.26997803)
\lineto(72.77,84.26997803)
\lineto(72.77,83.26997803)
\lineto(74.77,83.26997803)
\closepath
\moveto(70.77,84.26997803)
\lineto(68.77,84.26997803)
\lineto(68.77,83.26997803)
\lineto(70.77,83.26997803)
\closepath
\moveto(66.77,84.26997803)
\lineto(64.77,84.26997803)
\lineto(64.77,83.26997803)
\lineto(66.77,83.26997803)
\closepath
\moveto(62.77,84.26997803)
\lineto(60.77,84.26997803)
\lineto(60.77,83.26997803)
\lineto(62.77,83.26997803)
\closepath
\moveto(58.77,84.26997803)
\lineto(56.77,84.26997803)
\lineto(56.77,83.26997803)
\lineto(58.77,83.26997803)
\closepath
\moveto(54.77,84.26997803)
\lineto(52.77,84.26997803)
\lineto(52.77,83.26997803)
\lineto(54.77,83.26997803)
\closepath
\moveto(50.77,84.26997803)
\lineto(48.77,84.26997803)
\lineto(48.77,83.26997803)
\lineto(50.77,83.26997803)
\closepath
\moveto(46.77,84.26997803)
\lineto(44.77,84.26997803)
\lineto(44.77,83.26997803)
\lineto(46.77,83.26997803)
\closepath
\moveto(42.77,84.26997803)
\lineto(40.77,84.26997803)
\lineto(40.77,83.26997803)
\lineto(42.77,83.26997803)
\closepath
\moveto(38.77,84.26997803)
\lineto(36.77,84.26997803)
\lineto(36.77,83.26997803)
\lineto(38.77,83.26997803)
\closepath
\moveto(34.77,84.26997803)
\lineto(33,84.26997803)
\lineto(33,83.26997803)
\lineto(34.73,83.26997803)
\lineto(34.73,84.26997803)
\closepath
\moveto(34,81.99997803)
\lineto(33,81.99997803)
\lineto(33,79.99997803)
\lineto(34,79.99997803)
\closepath
\moveto(34,77.99997803)
\lineto(33,77.99997803)
\lineto(33,75.99997803)
\lineto(34,75.99997803)
\closepath
\moveto(34,73.99997803)
\lineto(33,73.99997803)
\lineto(33,71.99997803)
\lineto(34,71.99997803)
\closepath
\moveto(34,69.99997803)
\lineto(33,69.99997803)
\lineto(33,67.99997803)
\lineto(34,67.99997803)
\closepath
\moveto(34,65.99997803)
\lineto(33,65.99997803)
\lineto(33,63.99997803)
\lineto(34,63.99997803)
\closepath
\moveto(34,61.99997803)
\lineto(33,61.99997803)
\lineto(33,59.99997803)
\lineto(34,59.99997803)
\closepath
\moveto(34,57.99997803)
\lineto(33,57.99997803)
\lineto(33,55.99997803)
\lineto(34,55.99997803)
\closepath
\moveto(34,53.99997803)
\lineto(33,53.99997803)
\lineto(33,51.99997803)
\lineto(34,51.99997803)
\closepath
\moveto(34,49.99997803)
\lineto(33,49.99997803)
\lineto(33,47.99997803)
\lineto(34,47.99997803)
\closepath
\moveto(34,45.99997803)
\lineto(33,45.99997803)
\lineto(33,43.99997803)
\lineto(34,43.99997803)
\closepath
\moveto(34,41.99997803)
\lineto(33,41.99997803)
\lineto(33,39.99997803)
\lineto(34,39.99997803)
\closepath
\moveto(34,37.99997803)
\lineto(33,37.99997803)
\lineto(33,35.99997803)
\lineto(34,35.99997803)
\closepath
\moveto(34,33.99997803)
\lineto(33,33.99997803)
\lineto(33,31.99997803)
\lineto(34,31.99997803)
\closepath
\moveto(34,29.99997803)
\lineto(33,29.99997803)
\lineto(33,27.99997803)
\lineto(34,27.99997803)
\closepath
\moveto(34,25.99997803)
\lineto(33,25.99997803)
\lineto(33,23.99997803)
\lineto(34,23.99997803)
\closepath
\moveto(34.68,22.66997803)
\lineto(33,22.66997803)
\lineto(33,21.99997803)
\lineto(33,21.99997803)
\lineto(33,21.99997803)
\lineto(33,21.99997803)
\lineto(33,21.66997803)
\lineto(34.68,21.66997803)
\closepath
\moveto(38.68,22.66997803)
\lineto(36.68,22.66997803)
\lineto(36.68,21.66997803)
\lineto(38.68,21.66997803)
\closepath
\moveto(42.68,22.66997803)
\lineto(40.68,22.66997803)
\lineto(40.68,21.66997803)
\lineto(42.68,21.66997803)
\closepath
\moveto(46.68,22.66997803)
\lineto(44.68,22.66997803)
\lineto(44.68,21.66997803)
\lineto(46.68,21.66997803)
\closepath
\moveto(50.68,22.66997803)
\lineto(48.68,22.66997803)
\lineto(48.68,21.66997803)
\lineto(50.68,21.66997803)
\closepath
\moveto(54.68,22.66997803)
\lineto(52.68,22.66997803)
\lineto(52.68,21.66997803)
\lineto(54.68,21.66997803)
\closepath
\moveto(58.68,22.66997803)
\lineto(56.68,22.66997803)
\lineto(56.68,21.66997803)
\lineto(58.68,21.66997803)
\closepath
\moveto(62.68,22.66997803)
\lineto(60.68,22.66997803)
\lineto(60.68,21.66997803)
\lineto(62.68,21.66997803)
\closepath
\moveto(66.68,22.66997803)
\lineto(64.68,22.66997803)
\lineto(64.68,21.66997803)
\lineto(66.68,21.66997803)
\closepath
\moveto(70.68,22.66997803)
\lineto(68.68,22.66997803)
\lineto(68.68,21.66997803)
\lineto(70.68,21.66997803)
\closepath
\moveto(74.68,22.66997803)
\lineto(72.68,22.66997803)
\lineto(72.68,21.66997803)
\lineto(74.68,21.66997803)
\closepath
\moveto(78.68,22.66997803)
\lineto(76.68,22.66997803)
\lineto(76.68,21.66997803)
\lineto(78.68,21.66997803)
\closepath
\moveto(82.68,22.66997803)
\lineto(80.68,22.66997803)
\lineto(80.68,21.66997803)
\lineto(82.68,21.66997803)
\closepath
\moveto(86.68,22.66997803)
\lineto(84.68,22.66997803)
\lineto(84.68,21.66997803)
\lineto(86.68,21.66997803)
\closepath
\moveto(90.68,22.66997803)
\lineto(88.68,22.66997803)
\lineto(88.68,21.66997803)
\lineto(90.68,21.66997803)
\closepath
\moveto(94.68,22.66997803)
\lineto(92.68,22.66997803)
\lineto(92.68,21.66997803)
\lineto(94.68,21.66997803)
\closepath
\moveto(94.68,25.66997803)
\lineto(93.68,25.66997803)
\lineto(93.68,23.66997803)
\lineto(94.68,23.66997803)
\closepath
\moveto(94.68,29.66997803)
\lineto(93.68,29.66997803)
\lineto(93.68,27.66997803)
\lineto(94.68,27.66997803)
\closepath
\moveto(94.68,33.66997803)
\lineto(93.68,33.66997803)
\lineto(93.68,31.66997803)
\lineto(94.68,31.66997803)
\closepath
\moveto(94.68,37.66997803)
\lineto(93.68,37.66997803)
\lineto(93.68,35.66997803)
\lineto(94.68,35.66997803)
\closepath
\moveto(94.68,41.66997803)
\lineto(93.68,41.66997803)
\lineto(93.68,39.66997803)
\lineto(94.68,39.66997803)
\closepath
\moveto(94.68,45.66997803)
\lineto(93.68,45.66997803)
\lineto(93.68,43.66997803)
\lineto(94.68,43.66997803)
\closepath
\moveto(94.68,49.66997803)
\lineto(93.68,49.66997803)
\lineto(93.68,47.66997803)
\lineto(94.68,47.66997803)
\closepath
\moveto(94.68,53.66997803)
\lineto(93.68,53.66997803)
\lineto(93.68,51.66997803)
\lineto(94.68,51.66997803)
\closepath
\moveto(94.68,57.66997803)
\lineto(93.68,57.66997803)
\lineto(93.68,55.66997803)
\lineto(94.68,55.66997803)
\closepath
\moveto(94.68,61.66997803)
\lineto(93.68,61.66997803)
\lineto(93.68,59.66997803)
\lineto(94.68,59.66997803)
\closepath
\moveto(94.68,65.66997803)
\lineto(93.68,65.66997803)
\lineto(93.68,63.66997803)
\lineto(94.68,63.66997803)
\closepath
\moveto(94.68,69.66997803)
\lineto(93.68,69.66997803)
\lineto(93.68,67.66997803)
\lineto(94.68,67.66997803)
\closepath
\moveto(94.68,73.66997803)
\lineto(93.68,73.66997803)
\lineto(93.68,71.66997803)
\lineto(94.68,71.66997803)
\closepath
\moveto(94.68,77.66997803)
\lineto(93.68,77.66997803)
\lineto(93.68,75.66997803)
\lineto(94.68,75.66997803)
\closepath
\moveto(94.68,81.66997803)
\lineto(93.68,81.66997803)
\lineto(93.68,79.66997803)
\lineto(94.68,79.66997803)
\closepath
\moveto(94.68,84.31997803)
\lineto(92.68,84.31997803)
\lineto(92.68,83.31997803)
\lineto(94.68,83.31997803)
\closepath
}
}
{
\newrgbcolor{curcolor}{0 0 0}
\pscustom[linewidth=1,linecolor=curcolor]
{
\newpath
\moveto(283.38000488,227.83996582)
\lineto(141.07000732,154.77996826)
}
}
{
\newrgbcolor{curcolor}{0 0 0}
\pscustom[linewidth=1,linecolor=curcolor]
{
\newpath
\moveto(281.70001221,227.83996582)
\lineto(430.83999634,154.66998291)
}
}
{
\newrgbcolor{curcolor}{0 0 0}
\pscustom[linewidth=1,linecolor=curcolor]
{
\newpath
\moveto(140.42999268,153.83996582)
\lineto(70.22000122,83.6199646)
}
}
{
\newrgbcolor{curcolor}{0 0 0}
\pscustom[linewidth=1,linecolor=curcolor]
{
\newpath
\moveto(141.27000427,153.83996582)
\lineto(210.22999573,84.87997437)
}
}
{
\newrgbcolor{curcolor}{0 0.44313726 0.73725492}
\pscustom[linestyle=none,fillstyle=solid,fillcolor=curcolor]
{
\newpath
\moveto(157.25000381,156.35998535)
\curveto(157.25000381,170.5976346)(140.03733022,177.72529537)(129.97101223,167.65897739)
\curveto(119.90469425,157.59265941)(127.03235503,140.37998581)(141.27000427,140.37998581)
\curveto(155.50765352,140.37998581)(162.63531429,157.59265941)(152.56899631,167.65897739)
\curveto(142.50267833,177.72529537)(125.29000473,170.5976346)(125.29000473,156.35998535)
\curveto(125.29000473,142.12233611)(142.50267833,134.99467533)(152.56899631,145.06099331)
\curveto(162.63531429,155.12731129)(155.50765352,172.33998489)(141.27000427,172.33998489)
\curveto(127.03235503,172.33998489)(119.90469425,155.12731129)(129.97101223,145.06099331)
\curveto(140.03733022,134.99467533)(157.25000381,142.12233611)(157.25000381,156.35998535)
\closepath
}
}
{
\newrgbcolor{curcolor}{0 0 0}
\pscustom[linewidth=1,linecolor=curcolor]
{
\newpath
\moveto(157.25000381,156.35998535)
\curveto(157.25000381,170.5976346)(140.03733022,177.72529537)(129.97101223,167.65897739)
\curveto(119.90469425,157.59265941)(127.03235503,140.37998581)(141.27000427,140.37998581)
\curveto(155.50765352,140.37998581)(162.63531429,157.59265941)(152.56899631,167.65897739)
\curveto(142.50267833,177.72529537)(125.29000473,170.5976346)(125.29000473,156.35998535)
\curveto(125.29000473,142.12233611)(142.50267833,134.99467533)(152.56899631,145.06099331)
\curveto(162.63531429,155.12731129)(155.50765352,172.33998489)(141.27000427,172.33998489)
\curveto(127.03235503,172.33998489)(119.90469425,155.12731129)(129.97101223,145.06099331)
\curveto(140.03733022,134.99467533)(157.25000381,142.12233611)(157.25000381,156.35998535)
\closepath
}
}
{
\newrgbcolor{curcolor}{0 0.44313726 0.73725492}
\pscustom[linestyle=none,fillstyle=solid,fillcolor=curcolor]
{
\newpath
\moveto(300.20000076,227.83996582)
\curveto(300.20000076,242.07761506)(282.98732716,249.20527584)(272.92100918,239.13895786)
\curveto(262.8546912,229.07263988)(269.98235198,211.85996628)(284.22000122,211.85996628)
\curveto(298.45765046,211.85996628)(305.58531124,229.07263988)(295.51899326,239.13895786)
\curveto(285.45267528,249.20527584)(268.24000168,242.07761506)(268.24000168,227.83996582)
\curveto(268.24000168,213.60231658)(285.45267528,206.4746558)(295.51899326,216.54097378)
\curveto(305.58531124,226.60729176)(298.45765046,243.81996536)(284.22000122,243.81996536)
\curveto(269.98235198,243.81996536)(262.8546912,226.60729176)(272.92100918,216.54097378)
\curveto(282.98732716,206.4746558)(300.20000076,213.60231658)(300.20000076,227.83996582)
\closepath
}
}
{
\newrgbcolor{curcolor}{0 0 0}
\pscustom[linewidth=1,linecolor=curcolor]
{
\newpath
\moveto(300.20000076,227.83996582)
\curveto(300.20000076,242.07761506)(282.98732716,249.20527584)(272.92100918,239.13895786)
\curveto(262.8546912,229.07263988)(269.98235198,211.85996628)(284.22000122,211.85996628)
\curveto(298.45765046,211.85996628)(305.58531124,229.07263988)(295.51899326,239.13895786)
\curveto(285.45267528,249.20527584)(268.24000168,242.07761506)(268.24000168,227.83996582)
\curveto(268.24000168,213.60231658)(285.45267528,206.4746558)(295.51899326,216.54097378)
\curveto(305.58531124,226.60729176)(298.45765046,243.81996536)(284.22000122,243.81996536)
\curveto(269.98235198,243.81996536)(262.8546912,226.60729176)(272.92100918,216.54097378)
\curveto(282.98732716,206.4746558)(300.20000076,213.60231658)(300.20000076,227.83996582)
\closepath
}
}
{
\newrgbcolor{curcolor}{0 0 0}
\pscustom[linestyle=none,fillstyle=solid,fillcolor=curcolor]
{
\newpath
\moveto(7.70704171,300.66165347)
\lineto(7.70704171,300.32441328)
\curveto(6.78950534,300.78621967)(6.02387895,301.32701926)(5.41016257,301.94681204)
\curveto(4.53516099,302.82789002)(3.8606806,303.8669544)(3.38672142,305.06400517)
\curveto(2.91276223,306.26105594)(2.67578263,307.50367971)(2.67578263,308.79187648)
\curveto(2.67578263,310.67556044)(3.14062722,312.3921434)(4.0703164,313.94162536)
\curveto(5.00000558,315.49718373)(6.21224735,316.6091649)(7.70704171,317.27756888)
\lineto(7.70704171,316.89475569)
\curveto(6.95964453,316.4815605)(6.34592815,315.91645532)(5.86589256,315.19944013)
\curveto(5.38585697,314.48242495)(5.02734938,313.57400317)(4.79036978,312.4741748)
\curveto(4.55339019,311.37434642)(4.43490039,310.22590685)(4.43490039,309.02885608)
\curveto(4.43490039,307.72850651)(4.53516099,306.54664673)(4.73568218,305.48327676)
\curveto(4.89366858,304.64473358)(5.08507518,303.97329139)(5.30990197,303.46895021)
\curveto(5.53472877,302.95853262)(5.83551056,302.46938243)(6.21224735,302.00149964)
\curveto(6.59506054,301.53361685)(7.09332533,301.08700146)(7.70704171,300.66165347)
\closepath
}
}
{
\newrgbcolor{curcolor}{0 0 0}
\pscustom[linestyle=none,fillstyle=solid,fillcolor=curcolor]
{
\newpath
\moveto(13.44012489,305.51973516)
\curveto(12.58335251,304.85740757)(12.04559112,304.47459438)(11.82684072,304.37129559)
\curveto(11.49871513,304.21938559)(11.14932214,304.14343059)(10.77866175,304.14343059)
\curveto(10.20140376,304.14343059)(9.72440637,304.34091359)(9.34766958,304.73587958)
\curveto(8.97700919,305.13084557)(8.79167899,305.65037776)(8.79167899,306.29447614)
\curveto(8.79167899,306.70159493)(8.88282499,307.05402612)(9.06511699,307.35176972)
\curveto(9.31424938,307.76496491)(9.74567377,308.1538545)(10.35939016,308.51843849)
\curveto(10.97918294,308.88302248)(12.00609452,309.32659967)(13.44012489,309.84917006)
\lineto(13.44012489,310.17729565)
\curveto(13.44012489,311.00976243)(13.30644409,311.58094402)(13.03908249,311.89084041)
\curveto(12.7777973,312.2007368)(12.39498411,312.355685)(11.89064292,312.355685)
\curveto(11.50782973,312.355685)(11.20400974,312.2523862)(10.97918294,312.04578861)
\curveto(10.74827975,311.83919101)(10.63282815,311.60221142)(10.63282815,311.33484982)
\lineto(10.65105735,310.80620303)
\curveto(10.65105735,310.52668864)(10.57814055,310.31097645)(10.43230696,310.15906645)
\curveto(10.29254976,310.00715645)(10.10721956,309.93120146)(9.87631637,309.93120146)
\curveto(9.65148957,309.93120146)(9.46615938,310.01019465)(9.32032578,310.16818105)
\curveto(9.18056858,310.32616745)(9.11068999,310.54187964)(9.11068999,310.81531763)
\curveto(9.11068999,311.33788802)(9.37805158,311.81792361)(9.91277477,312.2554244)
\curveto(10.44749796,312.69292519)(11.19793334,312.91167559)(12.16408092,312.91167559)
\curveto(12.9054017,312.91167559)(13.51304168,312.78710939)(13.98700087,312.53797699)
\curveto(14.34550846,312.3496086)(14.60983186,312.05490321)(14.77997105,311.65386082)
\curveto(14.88934625,311.39257562)(14.94403385,310.85785243)(14.94403385,310.04969125)
\lineto(14.94403385,307.21505072)
\curveto(14.94403385,306.41904234)(14.95922485,305.92989215)(14.98960685,305.74760015)
\curveto(15.01998885,305.57138456)(15.06860005,305.45289476)(15.13544045,305.39213076)
\curveto(15.20835724,305.33136676)(15.29038864,305.30098476)(15.38153464,305.30098476)
\curveto(15.47875704,305.30098476)(15.56382664,305.32225216)(15.63674343,305.36478696)
\curveto(15.76434783,305.44378016)(16.01044203,305.66556876)(16.37502602,306.03015275)
\lineto(16.37502602,305.51973516)
\curveto(15.69446923,304.60827518)(15.04429445,304.15254519)(14.42450166,304.15254519)
\curveto(14.12675807,304.15254519)(13.88977847,304.25584399)(13.71356288,304.46244158)
\curveto(13.53734728,304.66903918)(13.44620129,305.02147037)(13.44012489,305.51973516)
\closepath
\moveto(13.44012489,306.11218414)
\lineto(13.44012489,309.29317947)
\curveto(12.52258851,308.92859548)(11.93013952,308.67034848)(11.66277793,308.51843849)
\curveto(11.18274234,308.25107689)(10.83942575,307.9715625)(10.63282815,307.67989531)
\curveto(10.42623056,307.38822811)(10.32293176,307.06921712)(10.32293176,306.72286233)
\curveto(10.32293176,306.28536154)(10.45357436,305.92077755)(10.71485955,305.62911036)
\curveto(10.97614474,305.34351956)(11.27692654,305.20072417)(11.61720493,305.20072417)
\curveto(12.07901132,305.20072417)(12.6866513,305.50454416)(13.44012489,306.11218414)
\closepath
}
}
{
\newrgbcolor{curcolor}{0 0 0}
\pscustom[linestyle=none,fillstyle=solid,fillcolor=curcolor]
{
\newpath
\moveto(16.83075631,316.89475569)
\lineto(16.83075631,317.27756888)
\curveto(17.75436909,316.82183889)(18.52303367,316.28407751)(19.13675005,315.66428472)
\curveto(20.00567523,314.77713034)(20.67711742,313.73502777)(21.15107661,312.53797699)
\curveto(21.6250358,311.34700262)(21.86201539,310.10437885)(21.86201539,308.81010568)
\curveto(21.86201539,306.92642173)(21.3971708,305.20983877)(20.46748162,303.6603568)
\curveto(19.54386885,302.10479844)(18.33162707,300.99281726)(16.83075631,300.32441328)
\lineto(16.83075631,300.66165347)
\curveto(17.57815349,301.08092506)(18.19186988,301.64906845)(18.67190547,302.36608363)
\curveto(19.15801745,303.07702242)(19.51652505,303.98240599)(19.74742824,305.08223437)
\curveto(19.98440783,306.18813914)(20.10289763,307.33961692)(20.10289763,308.53666769)
\curveto(20.10289763,309.83094086)(20.00263703,311.01280063)(19.80211584,312.08224701)
\curveto(19.65020584,312.92079019)(19.45879925,313.59223237)(19.22789605,314.09657356)
\curveto(19.00306926,314.60091475)(18.70228746,315.08702673)(18.32555067,315.55490952)
\curveto(17.94881388,316.02279231)(17.45054909,316.4694077)(16.83075631,316.89475569)
\closepath
}
}
{
\newrgbcolor{curcolor}{0 0 0}
\pscustom[linestyle=none,fillstyle=solid,fillcolor=curcolor]
{
\newpath
\moveto(7.62499697,2.68418817)
\lineto(7.62499697,2.34694798)
\curveto(6.7074606,2.80875437)(5.94183421,3.34955396)(5.32811783,3.96934674)
\curveto(4.45311625,4.85042472)(3.77863586,5.8894891)(3.30467668,7.08653987)
\curveto(2.83071749,8.28359064)(2.59373789,9.52621441)(2.59373789,10.81441118)
\curveto(2.59373789,12.69809514)(3.05858248,14.4146781)(3.98827166,15.96416006)
\curveto(4.91796084,17.51971843)(6.13020261,18.6316996)(7.62499697,19.30010358)
\lineto(7.62499697,18.91729039)
\curveto(6.87759979,18.5040952)(6.26388341,17.93899002)(5.78384782,17.22197483)
\curveto(5.30381223,16.50495965)(4.94530464,15.59653787)(4.70832504,14.4967095)
\curveto(4.47134545,13.39688112)(4.35285565,12.24844155)(4.35285565,11.05139078)
\curveto(4.35285565,9.75104121)(4.45311625,8.56918143)(4.65363744,7.50581146)
\curveto(4.81162384,6.66726828)(5.00303044,5.99582609)(5.22785723,5.49148491)
\curveto(5.45268403,4.98106732)(5.75346582,4.49191713)(6.13020261,4.02403434)
\curveto(6.5130158,3.55615155)(7.01128059,3.10953616)(7.62499697,2.68418817)
\closepath
}
}
{
\newrgbcolor{curcolor}{0 0 0}
\pscustom[linestyle=none,fillstyle=solid,fillcolor=curcolor]
{
\newpath
\moveto(10.9153674,13.24800933)
\curveto(11.72352858,14.3721433)(12.59549196,14.93421029)(13.53125754,14.93421029)
\curveto(14.38802992,14.93421029)(15.1354271,14.56658809)(15.77344909,13.83134371)
\curveto(16.41147107,13.10217573)(16.73048207,12.10260795)(16.73048207,10.83264038)
\curveto(16.73048207,9.34999882)(16.23829368,8.15598624)(15.2539169,7.25060267)
\curveto(14.40929732,6.47282348)(13.46745534,6.08393389)(12.42839097,6.08393389)
\curveto(11.94227898,6.08393389)(11.44705239,6.17204169)(10.9427112,6.34825729)
\curveto(10.44444641,6.52447288)(9.93402883,6.78879628)(9.41145844,7.14122747)
\lineto(9.41145844,15.79098267)
\curveto(9.41145844,16.73890104)(9.38715284,17.32223543)(9.33854164,17.54098582)
\curveto(9.29600684,17.75973622)(9.22612824,17.90860802)(9.12890584,17.98760121)
\curveto(9.03168345,18.06659441)(8.91015545,18.10609101)(8.76432185,18.10609101)
\curveto(8.59418266,18.10609101)(8.38150866,18.05747981)(8.12629987,17.96025741)
\lineto(7.99869547,18.27926841)
\lineto(10.50521041,19.30010358)
\lineto(10.9153674,19.30010358)
\closepath
\moveto(10.9153674,12.66467494)
\lineto(10.9153674,7.66987426)
\curveto(11.2252638,7.36605426)(11.54427479,7.13515107)(11.87240038,6.97716467)
\curveto(12.20660237,6.82525468)(12.54688076,6.74929968)(12.89323556,6.74929968)
\curveto(13.44618794,6.74929968)(13.95964373,7.05311967)(14.43360292,7.66075966)
\curveto(14.91363851,8.26839964)(15.1536563,9.15251582)(15.1536563,10.31310819)
\curveto(15.1536563,11.38255457)(14.91363851,12.20286855)(14.43360292,12.77405014)
\curveto(13.95964373,13.35130812)(13.41884414,13.63993712)(12.81120416,13.63993712)
\curveto(12.48915497,13.63993712)(12.16710577,13.55790572)(11.84505658,13.39384292)
\curveto(11.60200059,13.27231492)(11.29210419,13.02925893)(10.9153674,12.66467494)
\closepath
}
}
{
\newrgbcolor{curcolor}{0 0 0}
\pscustom[linestyle=none,fillstyle=solid,fillcolor=curcolor]
{
\newpath
\moveto(17.79689024,18.91729039)
\lineto(17.79689024,19.30010358)
\curveto(18.72050302,18.84437359)(19.4891676,18.30661221)(20.10288399,17.68681942)
\curveto(20.97180917,16.79966504)(21.64325135,15.75756247)(22.11721054,14.56051169)
\curveto(22.59116973,13.36953732)(22.82814932,12.12691355)(22.82814932,10.83264038)
\curveto(22.82814932,8.94895643)(22.36330473,7.23237347)(21.43361556,5.6828915)
\curveto(20.51000278,4.12733314)(19.29776101,3.01535196)(17.79689024,2.34694798)
\lineto(17.79689024,2.68418817)
\curveto(18.54428742,3.10345976)(19.15800381,3.67160315)(19.6380394,4.38861833)
\curveto(20.12415139,5.09955712)(20.48265898,6.00494069)(20.71356217,7.10476907)
\curveto(20.95054177,8.21067384)(21.06903156,9.36215162)(21.06903156,10.55920239)
\curveto(21.06903156,11.85347556)(20.96877097,13.03533533)(20.76824977,14.10478171)
\curveto(20.61633978,14.94332489)(20.42493318,15.61476707)(20.19402999,16.11910826)
\curveto(19.96920319,16.62344945)(19.6684214,17.10956143)(19.29168461,17.57744422)
\curveto(18.91494782,18.04532701)(18.41668303,18.4919424)(17.79689024,18.91729039)
\closepath
}
}
\end{pspicture}

    \caption{简单场景的包围盒层次。(a)一小部分图元,边界框用虚线表示。
        图元基于邻近度聚合;这里,球体和等边三角形在被框住整个场景的边界框
        围住之前都被另一个边界框包围了(都用实线表示)。(b)相应的包围盒层次。
        根节点持有整个场景。这里它有两个孩子,一个保存包围了球体和等边三角形的边界框
        (又把这些图元作为其孩子),另一个保存持有瘦三角形的边界框。}
    \label{fig:4.2}
\end{figure}

图元细分的一个性质是每个图元只在层次中出现一次。
相反,一个图元可能与空间细分的多个空间区域重合,
因此在光线穿过它们时要多次测试相交
\footnote{\protect\keyindex{邮箱}{mailboxing}{}技术可用于
    让使用空间细分的加速器避免这样的多次相交,但它的实现在存在多进程时会很棘手。
    “扩展阅读”一节有关于邮箱的更多信息。}。
该性质还意味着表示图元细分层次所需的内存量是有界的。
对于每个叶子中保存单个图元的二叉BVH,节点总数为$2n-1$,其中$n$是图元数量。
有$n$个叶子节点和$n-1$个内部节点
\sidenote{译者注:这些结论利用了二叉BVH的前提:每个节点要么是叶子节点,要么是有两个孩子的内部节点。}。
如果叶子保存了多个图元,则需要的节点更少。

构建BVH比kd树更高效,kd树分发光线相交测试通常比BVH稍快但构建时间长得多。
另一方面,BVH通常数值更稳定,比起kd树更不容易因为舍入误差错过相交。

BVH加速器{\refvar{BVHAccel}{}}定义在\href{https://github.com/mmp/pbrt-v3/tree/master/src/accelerators/bvh.h}{\ttfamily accelerators/bvh.h}
和\href{https://github.com/mmp/pbrt-v3/tree/master/src/accelerators/bvh.cpp}{\ttfamily accelerators/bvh.cpp}
中。除了要保存的图元以及任何叶子节点中的最大图元数目,
其构造函数还接收一个描述当划分图元以构建树时要用四个算法中哪一个的枚举值。
应该用默认值\refvar{SAH}{},它表示\refsub{表面积启发法}讨论的基于“表面积启发法”的算法。
另一个是\refsub{线性包围盒层次}讨论的\refvar{HLBVH}{},
它能更高效地构造(且更易并行化),但建立的树不如\refvar{SAH}{}高效。
剩下的两种方法使用的计算量甚至更少,但创建的树的质量非常低。
\begin{lstlisting}
`\initcode{BVHAccel Public Types}{=}`
enum class `\initvar{SplitMethod}{}` { `\initvar{SAH}{}`, `\initvar{HLBVH}{}`, `\initvar{Middle}{}`, `\initvar{EqualCounts}{}` };
\end{lstlisting}
\begin{lstlisting}
`\initcode{BVHAccel Method Definitions}{=}\initnext{BVHAccelMethodDefinitions}`
`\initvar{BVHAccel}{}`::`\refvar{BVHAccel}{}`(const std::vector<std::shared_ptr<`\refvar{Primitive}{}`>> &p,
         int maxPrimsInNode, `\refvar{SplitMethod}{}` splitMethod)
     : `\refvar{maxPrimsInNode}{}`(std::min(255, maxPrimsInNode)), `\refvar[BVHAccel::primitives]{primitives}{}`(p),
       `\refvar{splitMethod}{}`(splitMethod) {
    if (primitives.size() == 0)
        return;
    `\refcode{Build BVH from primitives}{}`
}
\end{lstlisting}
\begin{lstlisting}
`\initcode{BVHAccel Private Data}{=}\initnext{BVHAccelPrivateData}`
const int `\initvar{maxPrimsInNode}{}`;
const `\refvar{SplitMethod}{}` `\initvar{splitMethod}{}`;
std::vector<std::shared_ptr<`\refvar{Primitive}{}`>> `\initvar[BVHAccel::primitives]{primitives}{}`;
\end{lstlisting}

\subsection{BVH构建}\label{sub:BVH构建}
这里的实现中构建BVH有三个阶段。
首先,计算关于每个图元的边界信息并保存到将于树构建期间使用的数组中。
接着,用选择的编码于\refvar{SplitMethod}{}的算法构建树。
结果是\keyindex{二叉树}{binary tree}{}每个内部节点
都有指针指向其孩子且每个叶子节点都有指向一个或多个图元的引用。
最后,该树转化为更紧实(且因此更高效)的无指针表示以供渲染时使用
(虽然在构建树期间直接计算无指针表示也可以,但用该方法实现更简单)。
\begin{lstlisting}
`\initcode{Build BVH from primitives}{=}`
`\refcode{Initialize primitiveInfo array for primitives}{}`
`\refcode{Build BVH tree for primitives using primitiveInfo}{}`
`\refcode{Compute representation of depth-first traversal of BVH tree}{}`
\end{lstlisting}

对于每个要存于BVH的图元,我们在结构体\refvar{BVHPrimitiveInfo}{}的一个实例中
存储其边界框的形心、完整边界框以及它在\refvar{primitives}{}数组中的索引。
\begin{lstlisting}
`\initcode{Initialize primitiveInfo array for primitives}{=}`
std::vector<`\refvar{BVHPrimitiveInfo}{}`> primitiveInfo(`\refvar[BVHAccel::primitives]{primitives}{}`.size());
for (size_t i = 0; i < `\refvar[BVHAccel::primitives]{primitives}{}`.size(); ++i)
    primitiveInfo[i] = { i, `\refvar[BVHAccel::primitives]{primitives}{}`[i]->`\refvar[Primitive::WorldBound]{WorldBound}{}`() };
\end{lstlisting}
\begin{lstlisting}
`\initcode{BVHAccel Local Declarations}{=}\initnext{BVHAccelLocalDeclarations}`
struct `\initvar{BVHPrimitiveInfo}{}` {
    `\refvar{BVHPrimitiveInfo}{}`(size_t primitiveNumber, const `\refvar{Bounds3f}{}` &bounds)
        : `\refvar{primitiveNumber}{}`(primitiveNumber), `\refvar[BVHPrimitiveInfo::bounds]{bounds}{}`(bounds),
          `\refvar{centroid}{}`(.5f * bounds.`\refvar{pMin}{}` + .5f * bounds.`\refvar{pMax}{}`) { }
    size_t `\initvar{primitiveNumber}{}`;
    `\refvar{Bounds3f}{}` `\initvar[BVHPrimitiveInfo::bounds]{bounds}{}`;
    `\refvar{Point3f}{}` `\initvar{centroid}{}`;
};
\end{lstlisting}

现在可以开始层次构建了。如果选择HLBVH构建算法,则调用\refvar{HLBVHBuild}{()}
构建树。其他三种构建算法都由\refvar{recursiveBuild}{()}负责。
初始调用这些函数时传递了所有要存于树中的图元。
它们返回一个指向树根的指针,用结构体\refvar{BVHBuildNode}{}表示。
树节点应该用提供的\refvar{MemoryArena}{}分配内存,
创建的总数应存于{\ttfamily *totalNodes}中。

树构建过程的一个重要副作用是通过参数{\ttfamily orderedPrims}返回指向图元的新指针数组;
该数组保存了有序的图元这样叶子节点的图元在数组中占有连续的范围。
在树构建后它与原始的\refvar[BVHAccel::primitives]{primitives}{}数组交换。
\begin{lstlisting}
`\initcode{Build BVH tree for primitives using primitiveInfo}{=}`
`\refvar{MemoryArena}{}` arena(1024 * 1024);
int totalNodes = 0;
std::vector<std::shared_ptr<`\refvar{Primitive}{}`>> orderedPrims;
`\refvar{BVHBuildNode}{}` *root;
if (splitMethod == `\refvar{SplitMethod}{}`::`\refvar{HLBVH}{}`)
    root = `\refvar{HLBVHBuild}{}`(arena, primitiveInfo, &totalNodes, orderedPrims);
else
    root = `\refvar{recursiveBuild}{}`(arena, primitiveInfo, 0, `\refvar[BVHAccel::primitives]{primitives}{}`.size(),
                          &totalNodes, orderedPrims);
`\refvar[BVHAccel::primitives]{primitives}{}`.swap(orderedPrims);
\end{lstlisting}

每个\refvar{BVHBuildNode}{}表示一个BVH节点。
所有节点存储一个\refvar{Bounds3f}{}以表示该节点下所有孩子的边界。
每个内部节点在\refvar[BVHBuildNode::children]{children}{}中存有指向其两个孩子的指针。
内部节点也记录图元沿哪个坐标轴划分分给它们的两个孩子;
该信息用于提高遍历算法的性能。
叶子节点需要记录哪个或哪些图元保存在它们中;
数组\refvar{BVHAccel::primitives}{}中从偏移量\refvar{firstPrimOffset}{}起
直到但不包括{\ttfamily\refvar{firstPrimOffset}{}+\refvar{nPrimitives}{}}的元素是叶子中的元素
(因此需要记录图元数组,这样就可以利用该表示,
而不是例如在每个叶子节点中保存一个大小可变的图元索引数组)。
\begin{lstlisting}
`\refcode{BVHAccel Local Declarations}{+=}\lastnext{BVHAccelLocalDeclarations}`
struct `\initvar{BVHBuildNode}{}` {
    `\refcode{BVHBuildNode Public Methods}{}`
    `\refvar{Bounds3f}{}` `\initvar[BVHBuildNode::bounds]{bounds}{}`;
    `\refvar{BVHBuildNode}{}` *`\initvar[BVHBuildNode::children]{children}{}`[2];
    int `\initvar{splitAxis}{}`, `\initvar{firstPrimOffset}{}`, `\initvar{nPrimitives}{}`;
};
\end{lstlisting}

我们将通过其孩子指针是否有值{\ttfamily nullptr}来分别区分叶子和内部节点。
\begin{lstlisting}
`\initcode{BVHBuildNode Public Methods}{=}\initnext{BVHBuildNodePublicMethods}`
void `\initvar{InitLeaf}{}`(int first, int n, const `\refvar{Bounds3f}{}` &b) {
    `\refvar{firstPrimOffset}{}` = first;
    `\refvar{nPrimitives}{}` = n;
    `\refvar[BVHBuildNode::bounds]{bounds}{}` = b;
    `\refvar[BVHBuildNode::children]{children}{}`[0] = `\refvar[BVHBuildNode::children]{children}{}`[1] = nullptr;
}
\end{lstlisting}

方法\refvar{InitInterior}{()}要求已创建两个孩子节点,这样它们的指针才能传入。
该要求让计算内部节点的边界更加容易了,因为孩子的边界可以立刻获得。
\begin{lstlisting}
`\refcode{BVHBuildNode Public Methods}{+=}\lastcode{BVHBuildNodePublicMethods}`
void `\initvar{InitInterior}{}`(int axis, `\refvar{BVHBuildNode}{}` *c0, `\refvar{BVHBuildNode}{}` *c1) {
    `\refvar[BVHBuildNode::children]{children}{}`[0] = c0;
    `\refvar[BVHBuildNode::children]{children}{}`[1] = c1;
    `\refvar[BVHBuildNode::bounds]{bounds}{}` = `\refvar[Union2]{Union}{}`(c0->`\refvar[BVHBuildNode::bounds]{bounds}{}`, c1->`\refvar[BVHBuildNode::bounds]{bounds}{}`);
    `\refvar{splitAxis}{}` = axis;
    `\refvar{nPrimitives}{}` = 0;
}
\end{lstlisting}

除了用于分配节点和\refvar{BVHPrimitiveInfo}{}结构体数组的\refvar{MemoryArena}{}外,\linebreak
\refvar{recursiveBuild}{()}接收范围参数{\ttfamily[start,end)}。
它负责为从{\ttfamily primitiveInfo [start]}直到并包括{\ttfamily primitiveInfo[end-1]}的
范围表示的图元子集返回一个BVH。
如果该范围只含有单个图元,则递归触底并创建一个叶子节点。
否则,该方法用划分算法之一来划分数组该范围内的元素并相应地重新排列它们,
这样范围{\ttfamily[start,mid)}和{\ttfamily[mid,end)}表示分开的子集。
如果划分成功,则这两个图元集合又传入将会为当前节点的两个孩子返回节点指针的递归调用。

{\ttfamily totalNodes}跟踪已创建的BVH节点总数;
利用该数目使得之后可以分配数目恰好正确的更紧实的\refvar{LinearBVHNode}{}。
最终,数组{\ttfamily orderedPrims}用于保存图元引用就像图元存于树的叶子节点一样。
该数组初始化为空;当创建一个叶子节点时,\refvar{recursiveBuild}{()}把
与之重合的图元添加到数列末尾,让叶子节点可以只存储对该数组的偏移量以及
表示与之重合的图元集的图元数量。
回想当完成树构建时,用这里创建的有序图元数组代替\refvar{BVHAccel::primitives}{}。
\begin{lstlisting}
`\refcode{BVHAccel Method Definitions}{+=}\lastnext{BVHAccelMethodDefinitions}`
`\refvar{BVHBuildNode}{}` *`\refvar{BVHAccel}{}`::`\initvar{recursiveBuild}{}`(`\refvar{MemoryArena}{}` &arena,
        std::vector<`\refvar{BVHPrimitiveInfo}{}`> &primitiveInfo, int start,
        int end, int *totalNodes,
        std::vector<std::shared_ptr<`\refvar{Primitive}{}`>> &orderedPrims) {
    `\refvar{BVHBuildNode}{}` *node = arena.`\refvar[MemoryArena:Alloc2]{Alloc}{}`<`\refvar{BVHBuildNode}{}`>();
    (*totalNodes)++;
    `\refcode{Compute bounds of all primitives in BVH node}{}`
    int nPrimitives = end - start;
    if (nPrimitives == 1) {
        `\refcode{Create leaf BVHBuildNode}{}`
    } else {
        `\refcode{Compute bound of primitive centroids, choose split dimension dim}{}`
        `\refcode{Partition primitives into two sets and build children}{}`
    }
    return node;
}
\end{lstlisting}
\begin{lstlisting}
`\initcode{Compute bounds of all primitives in BVH node}{=}`
`\refvar{Bounds3f}{}` bounds;
for (int i = start; i < end; ++i)
    bounds = `\refvar[Union2]{Union}{}`(bounds, primitiveInfo[i].`\refvar[BVHPrimitiveInfo::bounds]{bounds}{}`);
\end{lstlisting}

在叶子节点处,与该叶子重合的图元被添到{\ttfamily orderedPrims}数组末尾并初始化一个叶子节点对象。
\begin{lstlisting}
`\initcode{Create leaf BVHBuildNode}{=}`
int firstPrimOffset = orderedPrims.size();
for (int i = start; i < end; ++i) {
    int primNum = primitiveInfo[i].`\refvar{primitiveNumber}{}`;
    orderedPrims.push_back(`\refvar[BVHAccel::primitives]{primitives}{}`[primNum]);
}
node->`\refvar{InitLeaf}{}`(firstPrimOffset, nPrimitives, bounds);
return node;
\end{lstlisting}

对于内部节点,一组图元必须在两个子树之间划分。
给定$n$个图元,有$2^{n-1}-1$种\sidenote{译者注:原文误写为$2(n-1)-2$,已修正。}
可能的方法将它们划分到两个非空组。
实际中构建BVH时,一般考虑沿一个坐标轴划分,这意味着大约有$3n$个候选划分
(沿每个轴方向,每个图元可能放到第一分区或第二分区)。

这里我们就选择三个坐标轴的一个用来划分图元。
当为当前图元集合投影边界框形心时,我们选择有最大范围的轴
(另一种是尝试所有三个轴并选择给出最好结果的那个,但实际中本方法更好)。
该方法在许多场景下给出良好划分;\reffig{4.3}说明了该策略。
\begin{figure}[htbp]
    \centering%LaTeX with PSTricks extensions
%%Creator: Inkscape 1.0.1 (3bc2e813f5, 2020-09-07)
%%Please note this file requires PSTricks extensions
\psset{xunit=.5pt,yunit=.5pt,runit=.5pt}
\begin{pspicture}(363.91000366,352.60998535)
{
\newrgbcolor{curcolor}{0 0 0}
\pscustom[linewidth=1,linecolor=curcolor]
{
\newpath
\moveto(37.66,323.48998535)
\lineto(37.66,44.58998535)
\lineto(316.56,44.58998535)
}
}
{
\newrgbcolor{curcolor}{0 0 0}
\pscustom[linestyle=none,fillstyle=solid,fillcolor=curcolor]
{
\newpath
\moveto(32.16,318.57998535)
\lineto(37.66,322.83998535)
\lineto(43.17,318.57998535)
\lineto(37.66,331.58998535)
\closepath
}
}
{
\newrgbcolor{curcolor}{0.65098041 0.65098041 0.65098041}
\pscustom[linestyle=none,fillstyle=solid,fillcolor=curcolor]
{
\newpath
\moveto(33.36,320.12998535)
\lineto(37.66,330.27998535)
\lineto(37.66,323.46998535)
\closepath
}
}
{
\newrgbcolor{curcolor}{0.40000001 0.40000001 0.40000001}
\pscustom[linestyle=none,fillstyle=solid,fillcolor=curcolor]
{
\newpath
\moveto(41.96,320.12998535)
\lineto(37.66,330.27998535)
\lineto(37.66,323.46998535)
\closepath
}
}
{
\newrgbcolor{curcolor}{0 0 0}
\pscustom[linestyle=none,fillstyle=solid,fillcolor=curcolor]
{
\newpath
\moveto(311.65,39.08998535)
\lineto(315.91,44.58998535)
\lineto(311.65,50.08998535)
\lineto(324.66,44.58998535)
\closepath
}
}
{
\newrgbcolor{curcolor}{0.65098041 0.65098041 0.65098041}
\pscustom[linestyle=none,fillstyle=solid,fillcolor=curcolor]
{
\newpath
\moveto(313.21,40.28998535)
\lineto(323.35,44.58998535)
\lineto(316.54,44.58998535)
\closepath
}
}
{
\newrgbcolor{curcolor}{0.40000001 0.40000001 0.40000001}
\pscustom[linestyle=none,fillstyle=solid,fillcolor=curcolor]
{
\newpath
\moveto(313.21,48.88998535)
\lineto(323.35,44.58998535)
\lineto(316.54,44.58998535)
\closepath
}
}
{
\newrgbcolor{curcolor}{0 0 0}
\pscustom[linewidth=1,linecolor=curcolor]
{
\newpath
\moveto(196.75,199.3099823)
\lineto(224.45000076,199.3099823)
\lineto(224.45000076,182.62998199)
\lineto(196.75,182.62998199)
\closepath
}
}
{
\newrgbcolor{curcolor}{0 0 0}
\pscustom[linestyle=none,fillstyle=solid,fillcolor=curcolor]
{
\newpath
\moveto(216.16000366,191.08998108)
\curveto(216.16000366,195.54481514)(210.77431089,197.77499692)(207.62464936,194.62533539)
\curveto(204.47498782,191.47567385)(206.7051696,186.08998108)(211.16000366,186.08998108)
\curveto(215.61483772,186.08998108)(217.8450195,191.47567385)(214.69535797,194.62533539)
\curveto(211.54569643,197.77499692)(206.16000366,195.54481514)(206.16000366,191.08998108)
\curveto(206.16000366,186.63514702)(211.54569643,184.40496524)(214.69535797,187.55462677)
\curveto(217.8450195,190.70428831)(215.61483772,196.08998108)(211.16000366,196.08998108)
\curveto(206.7051696,196.08998108)(204.47498782,190.70428831)(207.62464936,187.55462677)
\curveto(210.77431089,184.40496524)(216.16000366,186.63514702)(216.16000366,191.08998108)
\closepath
}
}
{
\newrgbcolor{curcolor}{0 0 0}
\pscustom[linewidth=1,linecolor=curcolor]
{
\newpath
\moveto(175.50999451,114.46998596)
\lineto(294.67999268,114.46998596)
\lineto(294.67999268,87.46998596)
\lineto(175.50999451,87.46998596)
\closepath
}
}
{
\newrgbcolor{curcolor}{0 0 0}
\pscustom[linestyle=none,fillstyle=solid,fillcolor=curcolor]
{
\newpath
\moveto(240.66000366,101.08998108)
\curveto(240.66000366,105.54481514)(235.27431089,107.77499692)(232.12464936,104.62533539)
\curveto(228.97498782,101.47567385)(231.2051696,96.08998108)(235.66000366,96.08998108)
\curveto(240.11483772,96.08998108)(242.3450195,101.47567385)(239.19535797,104.62533539)
\curveto(236.04569643,107.77499692)(230.66000366,105.54481514)(230.66000366,101.08998108)
\curveto(230.66000366,96.63514702)(236.04569643,94.40496524)(239.19535797,97.55462677)
\curveto(242.3450195,100.70428831)(240.11483772,106.08998108)(235.66000366,106.08998108)
\curveto(231.2051696,106.08998108)(228.97498782,100.70428831)(232.12464936,97.55462677)
\curveto(235.27431089,94.40496524)(240.66000366,96.63514702)(240.66000366,101.08998108)
\closepath
}
}
{
\newrgbcolor{curcolor}{0 0 0}
\pscustom[linewidth=1,linecolor=curcolor]
{
\newpath
\moveto(94.87999725,125.21998596)
\lineto(198.31999969,125.21998596)
\lineto(198.31999969,55.7299881)
\lineto(94.87999725,55.7299881)
\closepath
}
}
{
\newrgbcolor{curcolor}{0 0 0}
\pscustom[linestyle=none,fillstyle=solid,fillcolor=curcolor]
{
\newpath
\moveto(152.16000366,90.58999634)
\curveto(152.16000366,95.0448304)(146.77431089,97.27501218)(143.62464936,94.12535064)
\curveto(140.47498782,90.97568911)(142.7051696,85.58999634)(147.16000366,85.58999634)
\curveto(151.61483772,85.58999634)(153.8450195,90.97568911)(150.69535797,94.12535064)
\curveto(147.54569643,97.27501218)(142.16000366,95.0448304)(142.16000366,90.58999634)
\curveto(142.16000366,86.13516228)(147.54569643,83.9049805)(150.69535797,87.05464203)
\curveto(153.8450195,90.20430357)(151.61483772,95.58999634)(147.16000366,95.58999634)
\curveto(142.7051696,95.58999634)(140.47498782,90.20430357)(143.62464936,87.05464203)
\curveto(146.77431089,83.9049805)(152.16000366,86.13516228)(152.16000366,90.58999634)
\closepath
}
}
{
\newrgbcolor{curcolor}{0 0 0}
\pscustom[linewidth=1,linecolor=curcolor]
{
\newpath
\moveto(100.34999847,316.86998367)
\lineto(174.84999847,316.86998367)
\lineto(174.84999847,244.05998611)
\lineto(100.34999847,244.05998611)
\closepath
}
}
{
\newrgbcolor{curcolor}{0 0 0}
\pscustom[linestyle=none,fillstyle=solid,fillcolor=curcolor]
{
\newpath
\moveto(143.16000366,280.58998871)
\curveto(143.16000366,285.04482277)(137.77431089,287.27500455)(134.62464936,284.12534301)
\curveto(131.47498782,280.97568148)(133.7051696,275.58998871)(138.16000366,275.58998871)
\curveto(142.61483772,275.58998871)(144.8450195,280.97568148)(141.69535797,284.12534301)
\curveto(138.54569643,287.27500455)(133.16000366,285.04482277)(133.16000366,280.58998871)
\curveto(133.16000366,276.13515465)(138.54569643,273.90497287)(141.69535797,277.0546344)
\curveto(144.8450195,280.20429594)(142.61483772,285.58998871)(138.16000366,285.58998871)
\curveto(133.7051696,285.58998871)(131.47498782,280.20429594)(134.62464936,277.0546344)
\curveto(137.77431089,273.90497287)(143.16000366,276.13515465)(143.16000366,280.58998871)
\closepath
}
}
{
\newrgbcolor{curcolor}{0 0 0}
\pscustom[linewidth=1,linecolor=curcolor]
{
\newpath
\moveto(73.34999847,229.96998596)
\lineto(147.84999847,229.96998596)
\lineto(147.84999847,202.96998596)
\lineto(73.34999847,202.96998596)
\closepath
}
}
{
\newrgbcolor{curcolor}{0 0 0}
\pscustom[linestyle=none,fillstyle=solid,fillcolor=curcolor]
{
\newpath
\moveto(116.16000366,216.58998108)
\curveto(116.16000366,221.04481514)(110.77431089,223.27499692)(107.62464936,220.12533539)
\curveto(104.47498782,216.97567385)(106.7051696,211.58998108)(111.16000366,211.58998108)
\curveto(115.61483772,211.58998108)(117.8450195,216.97567385)(114.69535797,220.12533539)
\curveto(111.54569643,223.27499692)(106.16000366,221.04481514)(106.16000366,216.58998108)
\curveto(106.16000366,212.13514702)(111.54569643,209.90496524)(114.69535797,213.05462677)
\curveto(117.8450195,216.20428831)(115.61483772,221.58998108)(111.16000366,221.58998108)
\curveto(106.7051696,221.58998108)(104.47498782,216.20428831)(107.62464936,213.05462677)
\curveto(110.77431089,209.90496524)(116.16000366,212.13514702)(116.16000366,216.58998108)
\closepath
}
}
{
\newrgbcolor{curcolor}{0 0 0}
\pscustom[linewidth=1,linecolor=curcolor,linestyle=dashed,dash=2]
{
\newpath
\moveto(137.66000366,281.08998871)
\lineto(37.65999985,281.08998871)
}
}
{
\newrgbcolor{curcolor}{0 0 0}
\pscustom[linewidth=1,linecolor=curcolor,linestyle=dashed,dash=2]
{
\newpath
\moveto(235.16000366,102.08998108)
\lineto(235.16000366,45.58999634)
}
}
{
\newrgbcolor{curcolor}{0 0 0}
\pscustom[linewidth=1,linecolor=curcolor,linestyle=dashed,dash=2]
{
\newpath
\moveto(146.66000366,91.08999634)
\lineto(37.65999985,91.08999634)
}
}
{
\newrgbcolor{curcolor}{0 0 0}
\pscustom[linewidth=1,linecolor=curcolor,linestyle=dashed,dash=2]
{
\newpath
\moveto(110.66000366,217.08998108)
\lineto(110.66000366,44.58999634)
}
}
{
\newrgbcolor{curcolor}{0 0 0}
\pscustom[linewidth=1,linecolor=curcolor]
{
\newpath
\moveto(110.64,40.13998535)
\lineto(110.64,35.47998535)
\lineto(235.07,35.47998535)
\lineto(235.07,40.52998535)
}
}
{
\newrgbcolor{curcolor}{0 0 0}
\pscustom[linewidth=1,linecolor=curcolor]
{
\newpath
\moveto(33.2,281.22998535)
\lineto(28.33,281.22998535)
\lineto(28.33,90.60998535)
\lineto(32.99,90.60998535)
}
}
{
\newrgbcolor{curcolor}{0 0 0}
\pscustom[linestyle=none,fillstyle=solid,fillcolor=curcolor]
{
\newpath
\moveto(42.709806,347.87659135)
\curveto(42.803556,348.15784135)(42.803556,348.18909135)(42.803556,348.34534135)
\curveto(42.803556,348.68909135)(42.522306,348.87659135)(42.209806,348.87659135)
\curveto(42.022306,348.87659135)(41.709806,348.75159135)(41.522306,348.47034135)
\curveto(41.491056,348.34534135)(41.303556,347.75159135)(41.241056,347.37659135)
\curveto(41.084806,346.87659135)(40.959806,346.31409135)(40.834806,345.78284135)
\lineto(39.928556,342.18909135)
\curveto(39.866056,341.90784135)(38.991056,340.50159135)(37.678556,340.50159135)
\curveto(36.678556,340.50159135)(36.459806,341.37659135)(36.459806,342.12659135)
\curveto(36.459806,343.03284135)(36.803556,344.28284135)(37.459806,346.03284135)
\curveto(37.772306,346.84534135)(37.866056,347.06409135)(37.866056,347.47034135)
\curveto(37.866056,348.34534135)(37.241056,349.09534135)(36.241056,349.09534135)
\curveto(34.334806,349.09534135)(33.616056,346.18909135)(33.616056,346.03284135)
\curveto(33.616056,345.81409135)(33.803556,345.81409135)(33.834806,345.81409135)
\curveto(34.053556,345.81409135)(34.053556,345.87659135)(34.147306,346.18909135)
\curveto(34.709806,348.06409135)(35.491056,348.65784135)(36.178556,348.65784135)
\curveto(36.334806,348.65784135)(36.678556,348.65784135)(36.678556,348.03284135)
\curveto(36.678556,347.53284135)(36.459806,347.00159135)(36.334806,346.62659135)
\curveto(35.522306,344.50159135)(35.178556,343.37659135)(35.178556,342.43909135)
\curveto(35.178556,340.65784135)(36.428556,340.06409135)(37.616056,340.06409135)
\curveto(38.397306,340.06409135)(39.053556,340.40784135)(39.616056,340.97034135)
\curveto(39.366056,339.93909135)(39.116056,338.93909135)(38.334806,337.87659135)
\curveto(37.803556,337.22034135)(37.053556,336.62659135)(36.147306,336.62659135)
\curveto(35.866056,336.62659135)(34.959806,336.68909135)(34.616056,337.47034135)
\curveto(34.928556,337.47034135)(35.209806,337.47034135)(35.459806,337.72034135)
\curveto(35.678556,337.87659135)(35.866056,338.15784135)(35.866056,338.53284135)
\curveto(35.866056,339.15784135)(35.334806,339.22034135)(35.147306,339.22034135)
\curveto(34.678556,339.22034135)(34.022306,338.90784135)(34.022306,337.93909135)
\curveto(34.022306,336.93909135)(34.897306,336.18909135)(36.147306,336.18909135)
\curveto(38.178556,336.18909135)(40.241056,338.00159135)(40.803556,340.25159135)
\closepath
\moveto(42.709806,347.87659135)
}
}
{
\newrgbcolor{curcolor}{0 0 0}
\pscustom[linestyle=none,fillstyle=solid,fillcolor=curcolor]
{
\newpath
\moveto(336.691076,46.58935135)
\curveto(336.816076,47.08935135)(337.284826,48.93310135)(338.659826,48.93310135)
\curveto(338.753576,48.93310135)(339.253576,48.93310135)(339.659826,48.68310135)
\curveto(339.097326,48.55810135)(338.722326,48.08935135)(338.722326,47.58935135)
\curveto(338.722326,47.27685135)(338.941076,46.90185135)(339.472326,46.90185135)
\curveto(339.909826,46.90185135)(340.534826,47.24560135)(340.534826,48.05810135)
\curveto(340.534826,49.08935135)(339.378576,49.37060135)(338.691076,49.37060135)
\curveto(337.534826,49.37060135)(336.847326,48.30810135)(336.597326,47.87060135)
\curveto(336.097326,49.18310135)(335.034826,49.37060135)(334.441076,49.37060135)
\curveto(332.378576,49.37060135)(331.222326,46.80810135)(331.222326,46.30810135)
\curveto(331.222326,46.08935135)(331.441076,46.08935135)(331.472326,46.08935135)
\curveto(331.628576,46.08935135)(331.691076,46.15185135)(331.722326,46.30810135)
\curveto(332.409826,48.43310135)(333.722326,48.93310135)(334.409826,48.93310135)
\curveto(334.784826,48.93310135)(335.472326,48.74560135)(335.472326,47.58935135)
\curveto(335.472326,46.96435135)(335.128576,45.65185135)(334.409826,42.83935135)
\curveto(334.097326,41.62060135)(333.378576,40.77685135)(332.503576,40.77685135)
\curveto(332.378576,40.77685135)(331.941076,40.77685135)(331.503576,41.02685135)
\curveto(332.003576,41.15185135)(332.441076,41.55810135)(332.441076,42.12060135)
\curveto(332.441076,42.65185135)(332.003576,42.80810135)(331.722326,42.80810135)
\curveto(331.097326,42.80810135)(330.628576,42.30810135)(330.628576,41.65185135)
\curveto(330.628576,40.74560135)(331.597326,40.33935135)(332.472326,40.33935135)
\curveto(333.816076,40.33935135)(334.534826,41.74560135)(334.566076,41.83935135)
\curveto(334.816076,41.12060135)(335.534826,40.33935135)(336.722326,40.33935135)
\curveto(338.784826,40.33935135)(339.909826,42.90185135)(339.909826,43.40185135)
\curveto(339.909826,43.62060135)(339.753576,43.62060135)(339.691076,43.62060135)
\curveto(339.503576,43.62060135)(339.472326,43.52685135)(339.409826,43.40185135)
\curveto(338.753576,41.24560135)(337.409826,40.77685135)(336.784826,40.77685135)
\curveto(336.003576,40.77685135)(335.691076,41.40185135)(335.691076,42.08935135)
\curveto(335.691076,42.52685135)(335.784826,42.96435135)(336.003576,43.83935135)
\closepath
\moveto(336.691076,46.58935135)
}
}
\end{pspicture}

    \caption{选择沿哪个轴划分图元。\protect\refvar{BVHAccel}{}基于
        图元边界框形心在哪个轴有最大范围来选择划分图元所沿的轴。
        这里在二维中,它们沿$y$轴的范围最大(轴上的实心点),所以图元会在$y$上划分。}
    \label{fig:4.3}
\end{figure}

这里划分的一般目标是选择图元的一个划分使得
得到的两个图元集合的边界框没有太多重合——
如果有大量重合则在遍历树时它会更频繁地需要遍历两个子树,
比起本应可以更高效剪除一些图元它需要更多计算量。
待会儿在讨论表面积启发法时会更严谨地表述该求取高效图元划分的思想。
\begin{lstlisting}
`\initcode{Compute bound of primitive centroids, choose split dimension dim}{=}`
`\refvar{Bounds3f}{}` centroidBounds;
for (int i = start; i < end; ++i)
    centroidBounds = `\refvar[Union2]{Union}{}`(centroidBounds, primitiveInfo[i].`\refvar{centroid}{}`);
int dim = centroidBounds.`\refvar{MaximumExtent}{}`();
\end{lstlisting}

如果所有形心点都在同一位置(即形心边界为零体积),
则递归停止并用该图元创建一个叶子结点;
这里没有划分方法能对那种(非常)情况有效。
否则用选择的方法划分图元并传入两个对\refvar{recursiveBuild}{()}的递归调用。
\begin{lstlisting}
`\initcode{Partition primitives into two sets and build children}{=}`
int mid = (start + end) / 2;
if (centroidBounds.`\refvar{pMax}{}`[dim] == centroidBounds.`\refvar{pMin}{}`[dim]) {
    `\refcode{Create leaf BVHBuildNode}{}`
} else {
    `\refcode{Partition primitives based on splitMethod}{}`
    node->`\refvar{InitInterior}{}`(dim,
                       `\refvar{recursiveBuild}{}`(arena, primitiveInfo, start, mid,
                                      totalNodes, orderedPrims),
                       `\refvar{recursiveBuild}{}`(arena, primitiveInfo, mid, end,
                                      totalNodes, orderedPrims));
}
\end{lstlisting}

代码片\refcode{Partition primitives based on splitMethod}{}只是用
\refvar[splitMethod]{BVHAccel::splitMethod}{}
的值决定该用哪个图元划分方案。接下来的几页将介绍这三个方案。
\begin{lstlisting}
`\initcode{Partition primitives based on splitMethod}{=}`
switch (`\refvar{splitMethod}{}`) {
case `\refvar{SplitMethod}{}`::`\refvar{Middle}{}`: {
    `\refcode{Partition primitives through node's midpoint}{}`
}
case `\refvar{SplitMethod}{}`::`\refvar{EqualCounts}{}`: {
    `\refcode{Partition primitives into equally sized subsets}{}`
    break;
}
case `\refvar{SplitMethod}{}`::`\refvar{SAH}{}`:
default: {
    `\refcode{Partition primitives using approximate SAH}{}`
    break;
}
}
\end{lstlisting}

\refvar{Middle}{}是一个简单的\refvar{SplitMethod}{},
它首先计算图元形心沿划分轴的中点。
该方法在代码片\refcode{Partition primitives through node's midpoint}{}中实现。
图元按其形心在中点之上还是之下分为两个集合。
该划分用C++标准库函数{\ttfamily std::partition()}很容易完成,
它接收数组中的一系列元素和比较函数,并对数组元素排序使得
对于判定函数而言所有返回{\ttfamily true}的元素都出现在返回{\ttfamily false}的范围之前
\footnote{在调用{\ttfamily std::partition()}时,
注意数组{\ttfamily primitiveInfo}索引的特殊表达式即{\ttfamily \&primitiveInfo[end-1]+1}。
这样写代码有些晦涩的理由。在C和C++程序语言中,
计算数组末尾后下一个元素的指针是合法的,
这样遍历数组元素能持续到当前指针等于末端点。
为此,我们这里想就写成表达式{\ttfamily \&primitiveInfo[end]}。
然而{\ttfamily primitiveInfo}分配为C++的{\ttfamily vector};
一些{\ttfamily vector}实现在传给其{\ttfamily []}操作符
的偏移量是在数组末端之后时会报运行时错误。
因为我们不会尝试引用数组末端后下一个元素的值而只是想计算其地址,
所以该操作事实上是安全的。
因此我们最终用这里的表达式计算同一地址,并且也满足任何{\ttfamily vector}错误检查。}。
{\ttfamily std::partition()}返回指向第一个对于判定有{\ttfamily false}值的元素的指针,
它转化为对数组{\ttfamily primitiveInfo}的偏移量,这样我们就可以将其传入递归调用。
\reffig{4.4}说明了该方法,包括其有效和无效的情况。

如果图元都有巨大的重叠边界框,则该划分方法可能无法把图元分为两组。
这种情况下,执行往下进入{\ttfamily \refvar{SplitMethod}{}::\refvar{EqualCounts}{}}方法再试一次。
\begin{lstlisting}
`\initcode{Partition primitives through node's midpoint}{=}`
`\refvar{Float}{}` pmid = (centroidBounds.`\refvar{pMin}{}`[dim] + centroidBounds.`\refvar{pMax}{}`[dim]) / 2;
`\refvar{BVHPrimitiveInfo}{}` *midPtr =
    std::partition(&primitiveInfo[start], &primitiveInfo[end-1]+1,
        [dim, pmid](const `\refvar{BVHPrimitiveInfo}{}` &pi) {
            return pi.`\refvar{centroid}{}`[dim] < pmid;
        });
mid = midPtr - &primitiveInfo[0];
if (mid != start && mid != end)
    break;
\end{lstlisting}

当\refvar{splitMethod}{}是{\ttfamily\refvar{SplitMethod}{}::\refvar{EqualCounts}{}}时,
则运行代码片\refcode{Partition primitives into equally sized subsets}{}。
它把图元划分为两个数量相等的子集使得$n$个中前一半的$\displaystyle\frac{n}{2}$个
沿所选轴的形心坐标最小,另一半的则有最大形心坐标值。
尽管该方法有时能起效,但也有\reffig{4.4}(b)效果不好的情况。
\begin{figure}[htb]
    \centering%LaTeX with PSTricks extensions
%%Creator: Inkscape 1.0.1 (3bc2e813f5, 2020-09-07)
%%Please note this file requires PSTricks extensions
\psset{xunit=.5pt,yunit=.5pt,runit=.5pt}
\begin{pspicture}(429.88000488,709.7800293)
{
\newrgbcolor{curcolor}{0 0 0}
\pscustom[linewidth=1,linecolor=curcolor]
{
\newpath
\moveto(343.63000488,578.50003052)
\lineto(371.33000565,578.50003052)
\lineto(371.33000565,561.82003021)
\lineto(343.63000488,561.82003021)
\closepath
}
}
{
\newrgbcolor{curcolor}{0 0 0}
\pscustom[linestyle=none,fillstyle=solid,fillcolor=curcolor]
{
\newpath
\moveto(363.04998779,570.2800293)
\curveto(363.04998779,574.73486336)(357.66429502,576.96504514)(354.51463349,573.8153836)
\curveto(351.36497195,570.66572207)(353.59515373,565.2800293)(358.04998779,565.2800293)
\curveto(362.50482185,565.2800293)(364.73500364,570.66572207)(361.5853421,573.8153836)
\curveto(358.43568056,576.96504514)(353.04998779,574.73486336)(353.04998779,570.2800293)
\curveto(353.04998779,565.82519524)(358.43568056,563.59501345)(361.5853421,566.74467499)
\curveto(364.73500364,569.89433653)(362.50482185,575.2800293)(358.04998779,575.2800293)
\curveto(353.59515373,575.2800293)(351.36497195,569.89433653)(354.51463349,566.74467499)
\curveto(357.66429502,563.59501345)(363.04998779,565.82519524)(363.04998779,570.2800293)
\closepath
}
}
{
\newrgbcolor{curcolor}{0 0 0}
\pscustom[linewidth=1,linecolor=curcolor]
{
\newpath
\moveto(219.77000427,575.23002625)
\lineto(338.94000244,575.23002625)
\lineto(338.94000244,548.23002625)
\lineto(219.77000427,548.23002625)
\closepath
}
}
{
\newrgbcolor{curcolor}{0 0 0}
\pscustom[linestyle=none,fillstyle=solid,fillcolor=curcolor]
{
\newpath
\moveto(284.92001343,561.85003662)
\curveto(284.92001343,566.30487068)(279.53432066,568.53505246)(276.38465912,565.38539093)
\curveto(273.23499759,562.23572939)(275.46517937,556.85003662)(279.92001343,556.85003662)
\curveto(284.37484749,556.85003662)(286.60502927,562.23572939)(283.45536773,565.38539093)
\curveto(280.3057062,568.53505246)(274.92001343,566.30487068)(274.92001343,561.85003662)
\curveto(274.92001343,557.39520256)(280.3057062,555.16502078)(283.45536773,558.31468232)
\curveto(286.60502927,561.46434385)(284.37484749,566.85003662)(279.92001343,566.85003662)
\curveto(275.46517937,566.85003662)(273.23499759,561.46434385)(276.38465912,558.31468232)
\curveto(279.53432066,555.16502078)(284.92001343,557.39520256)(284.92001343,561.85003662)
\closepath
}
}
{
\newrgbcolor{curcolor}{0 0 0}
\pscustom[linewidth=1,linecolor=curcolor]
{
\newpath
\moveto(235.61999512,658.95002747)
\lineto(339.05999756,658.95002747)
\lineto(339.05999756,589.4600296)
\lineto(235.61999512,589.4600296)
\closepath
}
}
{
\newrgbcolor{curcolor}{0 0 0}
\pscustom[linestyle=none,fillstyle=solid,fillcolor=curcolor]
{
\newpath
\moveto(291.97000122,624.24002838)
\curveto(291.97000122,628.69486244)(286.58430845,630.92504422)(283.43464691,627.77538269)
\curveto(280.28498538,624.62572115)(282.51516716,619.24002838)(286.97000122,619.24002838)
\curveto(291.42483528,619.24002838)(293.65501706,624.62572115)(290.50535553,627.77538269)
\curveto(287.35569399,630.92504422)(281.97000122,628.69486244)(281.97000122,624.24002838)
\curveto(281.97000122,619.78519432)(287.35569399,617.55501254)(290.50535553,620.70467408)
\curveto(293.65501706,623.85433561)(291.42483528,629.24002838)(286.97000122,629.24002838)
\curveto(282.51516716,629.24002838)(280.28498538,623.85433561)(283.43464691,620.70467408)
\curveto(286.58430845,617.55501254)(291.97000122,619.78519432)(291.97000122,624.24002838)
\closepath
}
}
{
\newrgbcolor{curcolor}{0 0 0}
\pscustom[linewidth=1,linecolor=curcolor]
{
\newpath
\moveto(82.61000061,666.43003082)
\lineto(157.11000061,666.43003082)
\lineto(157.11000061,593.62003326)
\lineto(82.61000061,593.62003326)
\closepath
}
}
{
\newrgbcolor{curcolor}{0 0 0}
\pscustom[linestyle=none,fillstyle=solid,fillcolor=curcolor]
{
\newpath
\moveto(125.43000031,630.14002991)
\curveto(125.43000031,634.59486397)(120.04430754,636.82504575)(116.894646,633.67538421)
\curveto(113.74498446,630.52572268)(115.97516625,625.14002991)(120.43000031,625.14002991)
\curveto(124.88483436,625.14002991)(127.11501615,630.52572268)(123.96535461,633.67538421)
\curveto(120.81569308,636.82504575)(115.43000031,634.59486397)(115.43000031,630.14002991)
\curveto(115.43000031,625.68519585)(120.81569308,623.45501407)(123.96535461,626.6046756)
\curveto(127.11501615,629.75433714)(124.88483436,635.14002991)(120.43000031,635.14002991)
\curveto(115.97516625,635.14002991)(113.74498446,629.75433714)(116.894646,626.6046756)
\curveto(120.04430754,623.45501407)(125.43000031,625.68519585)(125.43000031,630.14002991)
\closepath
}
}
{
\newrgbcolor{curcolor}{0 0 0}
\pscustom[linewidth=1,linecolor=curcolor]
{
\newpath
\moveto(31.56999969,582.88002777)
\lineto(106.06999969,582.88002777)
\lineto(106.06999969,555.88002777)
\lineto(31.56999969,555.88002777)
\closepath
}
}
{
\newrgbcolor{curcolor}{0 0 0}
\pscustom[linestyle=none,fillstyle=solid,fillcolor=curcolor]
{
\newpath
\moveto(74.38999939,569.49003601)
\curveto(74.38999939,573.94487007)(69.00430662,576.17505185)(65.85464508,573.02539032)
\curveto(62.70498355,569.87572878)(64.93516533,564.49003601)(69.38999939,564.49003601)
\curveto(73.84483345,564.49003601)(76.07501523,569.87572878)(72.9253537,573.02539032)
\curveto(69.77569216,576.17505185)(64.38999939,573.94487007)(64.38999939,569.49003601)
\curveto(64.38999939,565.03520195)(69.77569216,562.80502017)(72.9253537,565.9546817)
\curveto(76.07501523,569.10434324)(73.84483345,574.49003601)(69.38999939,574.49003601)
\curveto(64.93516533,574.49003601)(62.70498355,569.10434324)(65.85464508,565.9546817)
\curveto(69.00430662,562.80502017)(74.38999939,565.03520195)(74.38999939,569.49003601)
\closepath
}
}
{
\newrgbcolor{curcolor}{0 0 0}
\pscustom[linewidth=1,linecolor=curcolor]
{
\newpath
\moveto(24.30999947,497.7900238)
\lineto(429.45001221,497.7900238)
}
}
{
\newrgbcolor{curcolor}{0 0 0}
\pscustom[linewidth=1,linecolor=curcolor,linestyle=dashed,dash=2]
{
\newpath
\moveto(69.30999756,569.62002563)
\lineto(69.30999756,497.64002991)
}
}
{
\newrgbcolor{curcolor}{0 0 0}
\pscustom[linewidth=1,linecolor=curcolor,linestyle=dashed,dash=2]
{
\newpath
\moveto(357.86999512,571.27003479)
\lineto(357.86999512,498.07002258)
}
}
{
\newrgbcolor{curcolor}{0 0.44313726 0.73725492}
\pscustom[linewidth=2,linecolor=curcolor]
{
\newpath
\moveto(213.47000122,498.60003662)
\lineto(213.47000122,709.7800293)
}
}
{
\newrgbcolor{curcolor}{0 0 0}
\pscustom[linewidth=1,linecolor=curcolor,linestyle=dashed,dash=2]
{
\newpath
\moveto(29.89999962,668.36003113)
\lineto(158.64000511,668.36003113)
\lineto(158.64000511,554.30003357)
\lineto(29.89999962,554.30003357)
\closepath
}
}
{
\newrgbcolor{curcolor}{0 0 0}
\pscustom[linewidth=1,linecolor=curcolor,linestyle=dashed,dash=2]
{
\newpath
\moveto(218.36000061,661.05002975)
\lineto(372.66999817,661.05002975)
\lineto(372.66999817,546.65002823)
\lineto(218.36000061,546.65002823)
\closepath
}
}
{
\newrgbcolor{curcolor}{0 0 0}
\pscustom[linewidth=1,linecolor=curcolor]
{
\newpath
\moveto(343.39001465,331.49002075)
\lineto(371.09001541,331.49002075)
\lineto(371.09001541,314.81002045)
\lineto(343.39001465,314.81002045)
\closepath
}
}
{
\newrgbcolor{curcolor}{0 0 0}
\pscustom[linestyle=none,fillstyle=solid,fillcolor=curcolor]
{
\newpath
\moveto(362.80999756,323.27001953)
\curveto(362.80999756,327.72485359)(357.42430479,329.95503537)(354.27464325,326.80537384)
\curveto(351.12498172,323.6557123)(353.3551635,318.27001953)(357.80999756,318.27001953)
\curveto(362.26483162,318.27001953)(364.4950134,323.6557123)(361.34535186,326.80537384)
\curveto(358.19569033,329.95503537)(352.80999756,327.72485359)(352.80999756,323.27001953)
\curveto(352.80999756,318.81518547)(358.19569033,316.58500369)(361.34535186,319.73466523)
\curveto(364.4950134,322.88432676)(362.26483162,328.27001953)(357.80999756,328.27001953)
\curveto(353.3551635,328.27001953)(351.12498172,322.88432676)(354.27464325,319.73466523)
\curveto(357.42430479,316.58500369)(362.80999756,318.81518547)(362.80999756,323.27001953)
\closepath
}
}
{
\newrgbcolor{curcolor}{0 0 0}
\pscustom[linewidth=1,linecolor=curcolor]
{
\newpath
\moveto(194.66999817,439.29003906)
\lineto(338.81999207,439.29003906)
\lineto(338.81999207,342.45004272)
\lineto(194.66999817,342.45004272)
\closepath
}
}
{
\newrgbcolor{curcolor}{0 0 0}
\pscustom[linestyle=none,fillstyle=solid,fillcolor=curcolor]
{
\newpath
\moveto(272.01998901,391.1300354)
\curveto(272.01998901,395.58486946)(266.63429624,397.81505124)(263.48463471,394.66538971)
\curveto(260.33497317,391.51572817)(262.56515495,386.1300354)(267.01998901,386.1300354)
\curveto(271.47482307,386.1300354)(273.70500486,391.51572817)(270.55534332,394.66538971)
\curveto(267.40568178,397.81505124)(262.01998901,395.58486946)(262.01998901,391.1300354)
\curveto(262.01998901,386.67520134)(267.40568178,384.44501956)(270.55534332,387.59468109)
\curveto(273.70500486,390.74434263)(271.47482307,396.1300354)(267.01998901,396.1300354)
\curveto(262.56515495,396.1300354)(260.33497317,390.74434263)(263.48463471,387.59468109)
\curveto(266.63429624,384.44501956)(272.01998901,386.67520134)(272.01998901,391.1300354)
\closepath
}
}
{
\newrgbcolor{curcolor}{0 0 0}
\pscustom[linewidth=1,linecolor=curcolor]
{
\newpath
\moveto(50.22999954,419.42004395)
\lineto(105.70999908,419.42004395)
\lineto(105.70999908,365.20004272)
\lineto(50.22999954,365.20004272)
\closepath
}
}
{
\newrgbcolor{curcolor}{0 0 0}
\pscustom[linestyle=none,fillstyle=solid,fillcolor=curcolor]
{
\newpath
\moveto(83.25,392.34002686)
\curveto(83.25,396.79486091)(77.86430723,399.0250427)(74.71464569,395.87538116)
\curveto(71.56498416,392.72571963)(73.79516594,387.34002686)(78.25,387.34002686)
\curveto(82.70483406,387.34002686)(84.93501584,392.72571963)(81.78535431,395.87538116)
\curveto(78.63569277,399.0250427)(73.25,396.79486091)(73.25,392.34002686)
\curveto(73.25,387.8851928)(78.63569277,385.65501101)(81.78535431,388.80467255)
\curveto(84.93501584,391.95433409)(82.70483406,397.34002686)(78.25,397.34002686)
\curveto(73.79516594,397.34002686)(71.56498416,391.95433409)(74.71464569,388.80467255)
\curveto(77.86430723,385.65501101)(83.25,387.8851928)(83.25,392.34002686)
\closepath
}
}
{
\newrgbcolor{curcolor}{0 0 0}
\pscustom[linewidth=1,linecolor=curcolor]
{
\newpath
\moveto(31.32999992,335.87002563)
\lineto(105.82999992,335.87002563)
\lineto(105.82999992,308.87002563)
\lineto(31.32999992,308.87002563)
\closepath
}
}
{
\newrgbcolor{curcolor}{0 0 0}
\pscustom[linestyle=none,fillstyle=solid,fillcolor=curcolor]
{
\newpath
\moveto(74.13999939,322.4800415)
\curveto(74.13999939,326.93487556)(68.75430662,329.16505735)(65.60464508,326.01539581)
\curveto(62.45498355,322.86573427)(64.68516533,317.4800415)(69.13999939,317.4800415)
\curveto(73.59483345,317.4800415)(75.82501523,322.86573427)(72.6753537,326.01539581)
\curveto(69.52569216,329.16505735)(64.13999939,326.93487556)(64.13999939,322.4800415)
\curveto(64.13999939,318.02520745)(69.52569216,315.79502566)(72.6753537,318.9446872)
\curveto(75.82501523,322.09434873)(73.59483345,327.4800415)(69.13999939,327.4800415)
\curveto(64.68516533,327.4800415)(62.45498355,322.09434873)(65.60464508,318.9446872)
\curveto(68.75430662,315.79502566)(74.13999939,318.02520745)(74.13999939,322.4800415)
\closepath
}
}
{
\newrgbcolor{curcolor}{0 0 0}
\pscustom[linewidth=1,linecolor=curcolor]
{
\newpath
\moveto(24.29999924,250.7800293)
\lineto(429.88000488,250.7800293)
}
}
{
\newrgbcolor{curcolor}{0 0 0}
\pscustom[linewidth=1,linecolor=curcolor,linestyle=dashed,dash=2]
{
\newpath
\moveto(69.06999969,322.61001587)
\lineto(69.06999969,250.6300354)
}
}
{
\newrgbcolor{curcolor}{0 0 0}
\pscustom[linewidth=1,linecolor=curcolor,linestyle=dashed,dash=2]
{
\newpath
\moveto(357.63000488,324.26004028)
\lineto(357.63000488,251.06002808)
}
}
{
\newrgbcolor{curcolor}{0 0.44313726 0.73725492}
\pscustom[linewidth=2,linecolor=curcolor]
{
\newpath
\moveto(213.22999573,251.59002686)
\lineto(213.22999573,462.77003479)
}
}
{
\newrgbcolor{curcolor}{0 0 0}
\pscustom[linewidth=1,linecolor=curcolor,linestyle=dashed,dash=2]
{
\newpath
\moveto(29.65999985,421.35003662)
\lineto(221.01000595,421.35003662)
\lineto(221.01000595,281.24003601)
\lineto(29.65999985,281.24003601)
\closepath
}
}
{
\newrgbcolor{curcolor}{0 0 0}
\pscustom[linewidth=1,linecolor=curcolor]
{
\newpath
\moveto(192.05999756,299.6300354)
\lineto(219.75999832,299.6300354)
\lineto(219.75999832,282.9500351)
\lineto(192.05999756,282.9500351)
\closepath
}
}
{
\newrgbcolor{curcolor}{0 0 0}
\pscustom[linestyle=none,fillstyle=solid,fillcolor=curcolor]
{
\newpath
\moveto(211.47999573,291.41003418)
\curveto(211.47999573,295.86486824)(206.09430296,298.09505002)(202.94464142,294.94538849)
\curveto(199.79497989,291.79572695)(202.02516167,286.41003418)(206.47999573,286.41003418)
\curveto(210.93482979,286.41003418)(213.16501157,291.79572695)(210.01535003,294.94538849)
\curveto(206.8656885,298.09505002)(201.47999573,295.86486824)(201.47999573,291.41003418)
\curveto(201.47999573,286.95520012)(206.8656885,284.72501834)(210.01535003,287.87467987)
\curveto(213.16501157,291.02434141)(210.93482979,296.41003418)(206.47999573,296.41003418)
\curveto(202.02516167,296.41003418)(199.79497989,291.02434141)(202.94464142,287.87467987)
\curveto(206.09430296,284.72501834)(211.47999573,286.95520012)(211.47999573,291.41003418)
\closepath
}
}
{
\newrgbcolor{curcolor}{0 0 0}
\pscustom[linewidth=1,linecolor=curcolor,linestyle=dashed,dash=2]
{
\newpath
\moveto(193.24000549,440.99002075)
\lineto(372.73001099,440.99002075)
\lineto(372.73001099,313.47002411)
\lineto(193.24000549,313.47002411)
\closepath
}
}
{
\newrgbcolor{curcolor}{0 0 0}
\pscustom[linewidth=1,linecolor=curcolor]
{
\newpath
\moveto(343.8500061,83.63000488)
\lineto(371.55000687,83.63000488)
\lineto(371.55000687,66.95000458)
\lineto(343.8500061,66.95000458)
\closepath
}
}
{
\newrgbcolor{curcolor}{0 0 0}
\pscustom[linestyle=none,fillstyle=solid,fillcolor=curcolor]
{
\newpath
\moveto(363.26998901,75.41003418)
\curveto(363.26998901,79.86486824)(357.88429624,82.09505002)(354.73463471,78.94538849)
\curveto(351.58497317,75.79572695)(353.81515495,70.41003418)(358.26998901,70.41003418)
\curveto(362.72482307,70.41003418)(364.95500486,75.79572695)(361.80534332,78.94538849)
\curveto(358.65568178,82.09505002)(353.26998901,79.86486824)(353.26998901,75.41003418)
\curveto(353.26998901,70.95520012)(358.65568178,68.72501834)(361.80534332,71.87467987)
\curveto(364.95500486,75.02434141)(362.72482307,80.41003418)(358.26998901,80.41003418)
\curveto(353.81515495,80.41003418)(351.58497317,75.02434141)(354.73463471,71.87467987)
\curveto(357.88429624,68.72501834)(363.26998901,70.95520012)(363.26998901,75.41003418)
\closepath
}
}
{
\newrgbcolor{curcolor}{0 0 0}
\pscustom[linewidth=1,linecolor=curcolor]
{
\newpath
\moveto(195.13000488,191.42004395)
\lineto(339.27999878,191.42004395)
\lineto(339.27999878,94.58004761)
\lineto(195.13000488,94.58004761)
\closepath
}
}
{
\newrgbcolor{curcolor}{0 0 0}
\pscustom[linestyle=none,fillstyle=solid,fillcolor=curcolor]
{
\newpath
\moveto(272.48001099,143.26000977)
\curveto(272.48001099,147.71484382)(267.09431822,149.94502561)(263.94465668,146.79536407)
\curveto(260.79499514,143.64570254)(263.02517693,138.26000977)(267.48001099,138.26000977)
\curveto(271.93484505,138.26000977)(274.16502683,143.64570254)(271.01536529,146.79536407)
\curveto(267.86570376,149.94502561)(262.48001099,147.71484382)(262.48001099,143.26000977)
\curveto(262.48001099,138.80517571)(267.86570376,136.57499392)(271.01536529,139.72465546)
\curveto(274.16502683,142.874317)(271.93484505,148.26000977)(267.48001099,148.26000977)
\curveto(263.02517693,148.26000977)(260.79499514,142.874317)(263.94465668,139.72465546)
\curveto(267.09431822,136.57499392)(272.48001099,138.80517571)(272.48001099,143.26000977)
\closepath
}
}
{
\newrgbcolor{curcolor}{0 0 0}
\pscustom[linewidth=1,linecolor=curcolor]
{
\newpath
\moveto(50.68999863,171.55004883)
\lineto(106.16999817,171.55004883)
\lineto(106.16999817,117.33004761)
\lineto(50.68999863,117.33004761)
\closepath
}
}
{
\newrgbcolor{curcolor}{0 0 0}
\pscustom[linestyle=none,fillstyle=solid,fillcolor=curcolor]
{
\newpath
\moveto(83.70999908,144.4800415)
\curveto(83.70999908,148.93487556)(78.32430631,151.16505735)(75.17464478,148.01539581)
\curveto(72.02498324,144.86573427)(74.25516503,139.4800415)(78.70999908,139.4800415)
\curveto(83.16483314,139.4800415)(85.39501493,144.86573427)(82.24535339,148.01539581)
\curveto(79.09569185,151.16505735)(73.70999908,148.93487556)(73.70999908,144.4800415)
\curveto(73.70999908,140.02520745)(79.09569185,137.79502566)(82.24535339,140.9446872)
\curveto(85.39501493,144.09434873)(83.16483314,149.4800415)(78.70999908,149.4800415)
\curveto(74.25516503,149.4800415)(72.02498324,144.09434873)(75.17464478,140.9446872)
\curveto(78.32430631,137.79502566)(83.70999908,140.02520745)(83.70999908,144.4800415)
\closepath
}
}
{
\newrgbcolor{curcolor}{0 0 0}
\pscustom[linewidth=1,linecolor=curcolor]
{
\newpath
\moveto(31.79000092,88)
\lineto(106.29000092,88)
\lineto(106.29000092,61)
\lineto(31.79000092,61)
\closepath
}
}
{
\newrgbcolor{curcolor}{0 0 0}
\pscustom[linestyle=none,fillstyle=solid,fillcolor=curcolor]
{
\newpath
\moveto(74.59999847,74.62005615)
\curveto(74.59999847,79.07489021)(69.2143057,81.30507199)(66.06464417,78.15541046)
\curveto(62.91498263,75.00574892)(65.14516442,69.62005615)(69.59999847,69.62005615)
\curveto(74.05483253,69.62005615)(76.28501432,75.00574892)(73.13535278,78.15541046)
\curveto(69.98569124,81.30507199)(64.59999847,79.07489021)(64.59999847,74.62005615)
\curveto(64.59999847,70.16522209)(69.98569124,67.93504031)(73.13535278,71.08470185)
\curveto(76.28501432,74.23436338)(74.05483253,79.62005615)(69.59999847,79.62005615)
\curveto(65.14516442,79.62005615)(62.91498263,74.23436338)(66.06464417,71.08470185)
\curveto(69.2143057,67.93504031)(74.59999847,70.16522209)(74.59999847,74.62005615)
\closepath
}
}
{
\newrgbcolor{curcolor}{0 0 0}
\pscustom[linewidth=1,linecolor=curcolor]
{
\newpath
\moveto(25.36000061,2.92004395)
\lineto(429.73001099,2.92004395)
}
}
{
\newrgbcolor{curcolor}{0 0 0}
\pscustom[linewidth=1,linecolor=curcolor,linestyle=dashed,dash=2]
{
\newpath
\moveto(69.52999878,74.75)
\lineto(69.52999878,2.77001953)
}
}
{
\newrgbcolor{curcolor}{0 0 0}
\pscustom[linewidth=1,linecolor=curcolor,linestyle=dashed,dash=2]
{
\newpath
\moveto(358.08999634,76.40002441)
\lineto(358.08999634,3.20001221)
}
}
{
\newrgbcolor{curcolor}{0 0.44313726 0.73725492}
\pscustom[linewidth=2,linecolor=curcolor]
{
\newpath
\moveto(152.55000305,3.72003174)
\lineto(152.55000305,214.91003418)
}
}
{
\newrgbcolor{curcolor}{0 0 0}
\pscustom[linewidth=1,linecolor=curcolor,linestyle=dashed,dash=2]
{
\newpath
\moveto(30.12000084,173.4800415)
\lineto(108.70999718,173.4800415)
\lineto(108.70999718,58.74004364)
\lineto(30.12000084,58.74004364)
\closepath
}
}
{
\newrgbcolor{curcolor}{0 0 0}
\pscustom[linewidth=1,linecolor=curcolor]
{
\newpath
\moveto(192.52000427,51.77001953)
\lineto(220.22000504,51.77001953)
\lineto(220.22000504,35.09001923)
\lineto(192.52000427,35.09001923)
\closepath
}
}
{
\newrgbcolor{curcolor}{0 0 0}
\pscustom[linestyle=none,fillstyle=solid,fillcolor=curcolor]
{
\newpath
\moveto(211.80000305,43.55004883)
\curveto(211.80000305,48.00488289)(206.41431028,50.23506467)(203.26464875,47.08540313)
\curveto(200.11498721,43.9357416)(202.34516899,38.55004883)(206.80000305,38.55004883)
\curveto(211.25483711,38.55004883)(213.48501889,43.9357416)(210.33535736,47.08540313)
\curveto(207.18569582,50.23506467)(201.80000305,48.00488289)(201.80000305,43.55004883)
\curveto(201.80000305,39.09521477)(207.18569582,36.86503299)(210.33535736,40.01469452)
\curveto(213.48501889,43.16435606)(211.25483711,48.55004883)(206.80000305,48.55004883)
\curveto(202.34516899,48.55004883)(200.11498721,43.16435606)(203.26464875,40.01469452)
\curveto(206.41431028,36.86503299)(211.80000305,39.09521477)(211.80000305,43.55004883)
\closepath
}
}
{
\newrgbcolor{curcolor}{0 0 0}
\pscustom[linewidth=1,linecolor=curcolor,linestyle=dashed,dash=2]
{
\newpath
\moveto(190.3500061,193.13000488)
\lineto(373.19000244,193.13000488)
\lineto(373.19000244,33.23001099)
\lineto(190.3500061,33.23001099)
\closepath
}
}
{
\newrgbcolor{curcolor}{0 0 0}
\pscustom[linestyle=none,fillstyle=solid,fillcolor=curcolor]
{
\newpath
\moveto(8.30486109,498.10690382)
\lineto(8.30486109,497.76966362)
\curveto(7.38732471,498.23147001)(6.62169833,498.7722696)(6.00798194,499.39206239)
\curveto(5.13298036,500.27314037)(4.45849998,501.31220474)(3.98454079,502.50925551)
\curveto(3.5105816,503.70630629)(3.27360201,504.94893006)(3.27360201,506.23712683)
\curveto(3.27360201,508.12081078)(3.7384466,509.83739374)(4.66813577,511.38687571)
\curveto(5.59782495,512.94243407)(6.81006672,514.05441524)(8.30486109,514.72281923)
\lineto(8.30486109,514.34000604)
\curveto(7.55746391,513.92681085)(6.94374752,513.36170566)(6.46371193,512.64469048)
\curveto(5.98367634,511.92767529)(5.62516875,511.01925351)(5.38818916,509.91942514)
\curveto(5.15120956,508.81959677)(5.03271977,507.67115719)(5.03271977,506.47410642)
\curveto(5.03271977,505.17375685)(5.13298036,503.99189708)(5.33350156,502.9285271)
\curveto(5.49148796,502.08998392)(5.68289455,501.41854174)(5.90772135,500.91420055)
\curveto(6.13254814,500.40378296)(6.43332993,499.91463277)(6.81006672,499.44674999)
\curveto(7.19287992,498.9788672)(7.6911447,498.53225181)(8.30486109,498.10690382)
\closepath
}
}
{
\newrgbcolor{curcolor}{0 0 0}
\pscustom[linestyle=none,fillstyle=solid,fillcolor=curcolor]
{
\newpath
\moveto(14.03794426,502.9649855)
\curveto(13.18117188,502.30265792)(12.64341049,501.91984473)(12.4246601,501.81654593)
\curveto(12.09653451,501.66463593)(11.74714151,501.58868094)(11.37648112,501.58868094)
\curveto(10.79922314,501.58868094)(10.32222575,501.78616393)(9.94548896,502.18112992)
\curveto(9.57482856,502.57609591)(9.38949837,503.0956281)(9.38949837,503.73972649)
\curveto(9.38949837,504.14684528)(9.48064437,504.49927647)(9.66293636,504.79702006)
\curveto(9.91206876,505.21021525)(10.34349315,505.59910484)(10.95720953,505.96368883)
\curveto(11.57700232,506.32827282)(12.60391389,506.77185001)(14.03794426,507.2944204)
\lineto(14.03794426,507.62254599)
\curveto(14.03794426,508.45501277)(13.90426346,509.02619436)(13.63690187,509.33609075)
\curveto(13.37561668,509.64598715)(12.99280348,509.80093534)(12.4884623,509.80093534)
\curveto(12.10564911,509.80093534)(11.80182911,509.69763655)(11.57700232,509.49103895)
\curveto(11.34609912,509.28444136)(11.23064753,509.04746176)(11.23064753,508.78010017)
\lineto(11.24887673,508.25145338)
\curveto(11.24887673,507.97193899)(11.17595993,507.75622679)(11.03012633,507.60431679)
\curveto(10.89036913,507.4524068)(10.70503894,507.3764518)(10.47413574,507.3764518)
\curveto(10.24930895,507.3764518)(10.06397875,507.455445)(9.91814516,507.61343139)
\curveto(9.77838796,507.77141779)(9.70850936,507.98712999)(9.70850936,508.26056798)
\curveto(9.70850936,508.78313837)(9.97587096,509.26317396)(10.51059414,509.70067475)
\curveto(11.04531733,510.13817554)(11.79575271,510.35692593)(12.76190029,510.35692593)
\curveto(13.50322107,510.35692593)(14.11086106,510.23235973)(14.58482025,509.98322734)
\curveto(14.94332784,509.79485894)(15.20765123,509.50015355)(15.37779043,509.09911116)
\curveto(15.48716563,508.83782597)(15.54185322,508.30310278)(15.54185322,507.4949416)
\lineto(15.54185322,504.66030106)
\curveto(15.54185322,503.86429268)(15.55704422,503.37514249)(15.58742622,503.1928505)
\curveto(15.61780822,503.0166349)(15.66641942,502.8981451)(15.73325982,502.83738111)
\curveto(15.80617662,502.77661711)(15.88820802,502.74623511)(15.97935401,502.74623511)
\curveto(16.07657641,502.74623511)(16.16164601,502.76750251)(16.23456281,502.81003731)
\curveto(16.36216721,502.88903051)(16.6082614,503.1108191)(16.97284539,503.47540309)
\lineto(16.97284539,502.9649855)
\curveto(16.29228861,502.05352552)(15.64211382,501.59779554)(15.02232104,501.59779554)
\curveto(14.72457744,501.59779554)(14.48759785,501.70109433)(14.31138225,501.90769193)
\curveto(14.13516666,502.11428952)(14.04402066,502.46672071)(14.03794426,502.9649855)
\closepath
\moveto(14.03794426,503.55743449)
\lineto(14.03794426,506.73842981)
\curveto(13.12040788,506.37384582)(12.5279589,506.11559883)(12.2605973,505.96368883)
\curveto(11.78056171,505.69632724)(11.43724512,505.41681285)(11.23064753,505.12514565)
\curveto(11.02404993,504.83347846)(10.92075113,504.51446747)(10.92075113,504.16811268)
\curveto(10.92075113,503.73061189)(11.05139373,503.36602789)(11.31267892,503.0743607)
\curveto(11.57396412,502.78876991)(11.87474591,502.64597451)(12.2150243,502.64597451)
\curveto(12.67683069,502.64597451)(13.28447068,502.9497945)(14.03794426,503.55743449)
\closepath
}
}
{
\newrgbcolor{curcolor}{0 0 0}
\pscustom[linestyle=none,fillstyle=solid,fillcolor=curcolor]
{
\newpath
\moveto(17.42857568,514.34000604)
\lineto(17.42857568,514.72281923)
\curveto(18.35218846,514.26708924)(19.12085304,513.72932785)(19.73456943,513.10953507)
\curveto(20.60349461,512.22238069)(21.27493679,511.18027811)(21.74889598,509.98322734)
\curveto(22.22285517,508.79225297)(22.45983477,507.5496292)(22.45983477,506.25535603)
\curveto(22.45983477,504.37167207)(21.99499018,502.65508911)(21.065301,501.10560715)
\curveto(20.14168822,499.55004878)(18.92944645,498.43806761)(17.42857568,497.76966362)
\lineto(17.42857568,498.10690382)
\curveto(18.17597287,498.52617541)(18.78968925,499.09431879)(19.26972484,499.81133398)
\curveto(19.75583683,500.52227276)(20.11434442,501.42765634)(20.34524762,502.52748471)
\curveto(20.58222721,503.63338949)(20.70071701,504.78486726)(20.70071701,505.98191803)
\curveto(20.70071701,507.2761912)(20.60045641,508.45805097)(20.39993521,509.52749735)
\curveto(20.24802522,510.36604053)(20.05661862,511.03748271)(19.82571543,511.5418239)
\curveto(19.60088863,512.04616509)(19.30010684,512.53227708)(18.92337005,513.00015987)
\curveto(18.54663326,513.46804266)(18.04836847,513.91465805)(17.42857568,514.34000604)
\closepath
}
}
{
\newrgbcolor{curcolor}{0 0 0}
\pscustom[linestyle=none,fillstyle=solid,fillcolor=curcolor]
{
\newpath
\moveto(7.14240709,250.19051382)
\lineto(7.14240709,249.85327362)
\curveto(6.22487071,250.31508001)(5.45924433,250.8558796)(4.84552794,251.47567239)
\curveto(3.97052636,252.35675037)(3.29604598,253.39581474)(2.82208679,254.59286551)
\curveto(2.3481276,255.78991629)(2.11114801,257.03254006)(2.11114801,258.32073683)
\curveto(2.11114801,260.20442078)(2.5759926,261.92100374)(3.50568177,263.47048571)
\curveto(4.43537095,265.02604407)(5.64761272,266.13802524)(7.14240709,266.80642923)
\lineto(7.14240709,266.42361604)
\curveto(6.39500991,266.01042085)(5.78129352,265.44531566)(5.30125793,264.72830048)
\curveto(4.82122234,264.01128529)(4.46271475,263.10286351)(4.22573516,262.00303514)
\curveto(3.98875556,260.90320677)(3.87026577,259.75476719)(3.87026577,258.55771642)
\curveto(3.87026577,257.25736685)(3.97052636,256.07550708)(4.17104756,255.0121371)
\curveto(4.32903396,254.17359392)(4.52044055,253.50215174)(4.74526735,252.99781055)
\curveto(4.97009414,252.48739296)(5.27087593,251.99824277)(5.64761272,251.53035999)
\curveto(6.03042592,251.0624772)(6.5286907,250.61586181)(7.14240709,250.19051382)
\closepath
}
}
{
\newrgbcolor{curcolor}{0 0 0}
\pscustom[linestyle=none,fillstyle=solid,fillcolor=curcolor]
{
\newpath
\moveto(10.43277752,260.75433497)
\curveto(11.2409387,261.87846894)(12.11290208,262.44053593)(13.04866766,262.44053593)
\curveto(13.90544004,262.44053593)(14.65283722,262.07291374)(15.2908592,261.33766936)
\curveto(15.92888119,260.60850137)(16.24789218,259.6089336)(16.24789218,258.33896603)
\curveto(16.24789218,256.85632446)(15.75570379,255.66231189)(14.77132702,254.75692831)
\curveto(13.92670744,253.97914913)(12.98486546,253.59025954)(11.94580108,253.59025954)
\curveto(11.45968909,253.59025954)(10.9644625,253.67836734)(10.46012132,253.85458293)
\curveto(9.96185653,254.03079853)(9.45143894,254.29512192)(8.92886855,254.64755311)
\lineto(8.92886855,263.29730831)
\curveto(8.92886855,264.24522669)(8.90456295,264.82856107)(8.85595175,265.04731147)
\curveto(8.81341696,265.26606186)(8.74353836,265.41493366)(8.64631596,265.49392686)
\curveto(8.54909356,265.57292006)(8.42756556,265.61241666)(8.28173197,265.61241666)
\curveto(8.11159277,265.61241666)(7.89891878,265.56380546)(7.64370998,265.46658306)
\lineto(7.51610559,265.78559405)
\lineto(10.02262053,266.80642923)
\lineto(10.43277752,266.80642923)
\closepath
\moveto(10.43277752,260.17100058)
\lineto(10.43277752,255.1761999)
\curveto(10.74267391,254.87237991)(11.0616849,254.64147671)(11.38981049,254.48349032)
\curveto(11.72401249,254.33158032)(12.06429088,254.25562532)(12.41064567,254.25562532)
\curveto(12.96359806,254.25562532)(13.47705385,254.55944531)(13.95101304,255.1670853)
\curveto(14.43104862,255.77472529)(14.67106642,256.65884147)(14.67106642,257.81943384)
\curveto(14.67106642,258.88888021)(14.43104862,259.70919419)(13.95101304,260.28037578)
\curveto(13.47705385,260.85763377)(12.93625426,261.14626276)(12.32861427,261.14626276)
\curveto(12.00656508,261.14626276)(11.68451589,261.06423136)(11.3624667,260.90016857)
\curveto(11.1194107,260.77864057)(10.80951431,260.53558457)(10.43277752,260.17100058)
\closepath
}
}
{
\newrgbcolor{curcolor}{0 0 0}
\pscustom[linestyle=none,fillstyle=solid,fillcolor=curcolor]
{
\newpath
\moveto(17.31430036,266.42361604)
\lineto(17.31430036,266.80642923)
\curveto(18.23791313,266.35069924)(19.00657772,265.81293785)(19.6202941,265.19314507)
\curveto(20.48921928,264.30599069)(21.16066147,263.26388811)(21.63462066,262.06683734)
\curveto(22.10857984,260.87586297)(22.34555944,259.6332392)(22.34555944,258.33896603)
\curveto(22.34555944,256.45528207)(21.88071485,254.73869911)(20.95102567,253.18921715)
\curveto(20.02741289,251.63365878)(18.81517112,250.52167761)(17.31430036,249.85327362)
\lineto(17.31430036,250.19051382)
\curveto(18.06169754,250.60978541)(18.67541392,251.17792879)(19.15544951,251.89494398)
\curveto(19.6415615,252.60588276)(20.00006909,253.51126634)(20.23097229,254.61109471)
\curveto(20.46795188,255.71699949)(20.58644168,256.86847726)(20.58644168,258.06552803)
\curveto(20.58644168,259.3598012)(20.48618108,260.54166097)(20.28565989,261.61110735)
\curveto(20.13374989,262.44965053)(19.94234329,263.12109271)(19.7114401,263.6254339)
\curveto(19.48661331,264.12977509)(19.18583151,264.61588708)(18.80909472,265.08376987)
\curveto(18.43235793,265.55165266)(17.93409314,265.99826805)(17.31430036,266.42361604)
\closepath
}
}
{
\newrgbcolor{curcolor}{0 0 0}
\pscustom[linestyle=none,fillstyle=solid,fillcolor=curcolor]
{
\newpath
\moveto(8.81444409,2.72894382)
\lineto(8.81444409,2.39170362)
\curveto(7.89690771,2.85351001)(7.13128133,3.3943096)(6.51756494,4.01410239)
\curveto(5.64256336,4.89518037)(4.96808298,5.93424474)(4.49412379,7.13129551)
\curveto(4.0201646,8.32834629)(3.78318501,9.57097006)(3.78318501,10.85916683)
\curveto(3.78318501,12.74285078)(4.2480296,14.45943374)(5.17771877,16.00891571)
\curveto(6.10740795,17.56447407)(7.31964972,18.67645524)(8.81444409,19.34485923)
\lineto(8.81444409,18.96204604)
\curveto(8.06704691,18.54885085)(7.45333052,17.98374566)(6.97329493,17.26673048)
\curveto(6.49325934,16.54971529)(6.13475175,15.64129351)(5.89777216,14.54146514)
\curveto(5.66079256,13.44163677)(5.54230277,12.29319719)(5.54230277,11.09614642)
\curveto(5.54230277,9.79579685)(5.64256336,8.61393708)(5.84308456,7.5505671)
\curveto(6.00107096,6.71202392)(6.19247755,6.04058174)(6.41730435,5.53624055)
\curveto(6.64213114,5.02582296)(6.94291293,4.53667277)(7.31964972,4.06878999)
\curveto(7.70246292,3.6009072)(8.2007277,3.15429181)(8.81444409,2.72894382)
\closepath
}
}
{
\newrgbcolor{curcolor}{0 0 0}
\pscustom[linestyle=none,fillstyle=solid,fillcolor=curcolor]
{
\newpath
\moveto(16.9082086,9.55577906)
\curveto(16.68338181,8.45595068)(16.24284282,7.6082929)(15.58659164,7.01280572)
\curveto(14.93034045,6.42339493)(14.20421067,6.12868954)(13.40820229,6.12868954)
\curveto(12.46028391,6.12868954)(11.63389353,6.52669373)(10.92903114,7.32270211)
\curveto(10.22416876,8.11871049)(9.87173757,9.19423327)(9.87173757,10.54927043)
\curveto(9.87173757,11.8617728)(10.26062716,12.92818098)(11.03840634,13.74849496)
\curveto(11.82226192,14.56880894)(12.7610657,14.97896593)(13.85481768,14.97896593)
\curveto(14.67513166,14.97896593)(15.34961204,14.76021554)(15.87825883,14.32271475)
\curveto(16.40690562,13.89129036)(16.67122901,13.44163677)(16.67122901,12.97375398)
\curveto(16.67122901,12.74285078)(16.59527401,12.55448239)(16.44336402,12.40864879)
\curveto(16.29753042,12.26889159)(16.09093282,12.199013)(15.82357123,12.199013)
\curveto(15.46506364,12.199013)(15.19466385,12.31446459)(15.01237185,12.54536779)
\curveto(14.90907305,12.67297218)(14.83919445,12.91602818)(14.80273605,13.27453577)
\curveto(14.77235405,13.63304336)(14.65082606,13.90648136)(14.43815206,14.09484975)
\curveto(14.22547807,14.27714175)(13.93077267,14.36828774)(13.55403588,14.36828774)
\curveto(12.9463959,14.36828774)(12.45724571,14.14346095)(12.08658532,13.69380736)
\curveto(11.59439693,13.09832017)(11.34830274,12.31142639)(11.34830274,11.33312602)
\curveto(11.34830274,10.33659644)(11.59135873,9.45551846)(12.07747072,8.68989208)
\curveto(12.56965911,7.9303421)(13.23198669,7.5505671)(14.06445347,7.5505671)
\curveto(14.65994066,7.5505671)(15.19466385,7.7541265)(15.66862303,8.16124529)
\curveto(16.00282503,8.44075968)(16.32791242,8.94813907)(16.64388521,9.68338345)
\closepath
}
}
{
\newrgbcolor{curcolor}{0 0 0}
\pscustom[linestyle=none,fillstyle=solid,fillcolor=curcolor]
{
\newpath
\moveto(17.93815868,18.96204604)
\lineto(17.93815868,19.34485923)
\curveto(18.86177146,18.88912924)(19.63043604,18.35136785)(20.24415243,17.73157507)
\curveto(21.11307761,16.84442069)(21.78451979,15.80231811)(22.25847898,14.60526734)
\curveto(22.73243817,13.41429297)(22.96941777,12.1716692)(22.96941777,10.87739603)
\curveto(22.96941777,8.99371207)(22.50457318,7.27712911)(21.574884,5.72764715)
\curveto(20.65127122,4.17208878)(19.43902945,3.06010761)(17.93815868,2.39170362)
\lineto(17.93815868,2.72894382)
\curveto(18.68555587,3.14821541)(19.29927225,3.71635879)(19.77930784,4.43337398)
\curveto(20.26541983,5.14431276)(20.62392742,6.04969634)(20.85483062,7.14952471)
\curveto(21.09181021,8.25542949)(21.21030001,9.40690726)(21.21030001,10.60395803)
\curveto(21.21030001,11.8982312)(21.11003941,13.08009097)(20.90951821,14.14953735)
\curveto(20.75760822,14.98808053)(20.56620162,15.65952271)(20.33529843,16.1638639)
\curveto(20.11047163,16.66820509)(19.80968984,17.15431708)(19.43295305,17.62219987)
\curveto(19.05621626,18.09008266)(18.55795147,18.53669805)(17.93815868,18.96204604)
\closepath
}
}
\end{pspicture}

    \caption{一轴上基于形心中点划分图元。(a)对于一些图元分布,
        例如这里所示,沿所选轴(粗蓝线)基于形心中点的划分效果很好。
        (b)对于像这个的分布,中点是次优选项;所得两个边界框大量重叠。
        (c)若来自(b)的同一组图元换为用这里展示的线分开,所得边界框
        更小且根本不会重叠,使渲染时性能更好。}
    \label{fig:4.4}
\end{figure}

该方案也易于调用标准库的{\ttfamily std::nth\_element()}实现。
它接收起点、中点和终点指针以及一个比较函数。
它对数组排序使得中点指针处元素的位置是,如果数组完全排序,
则所有中点之前的元素都比中点元素小且所有后面的元素都比它大。
对于$n$个元素该排序可以在$O(n)$时间内完成,
比完全排序数组的$O(n\log{n})$更高效。
\begin{lstlisting}
`\initcode{Partition primitives into equally sized subsets}{}`
mid = (start + end) / 2;
std::nth_element(&primitiveInfo[start], &primitiveInfo[mid], 
                 &primitiveInfo[end-1]+1,
    [dim](const BVHPrimitiveInfo &a, const BVHPrimitiveInfo &b) { 
        return a.`\refvar{centroid}{}`[dim] < b.`\refvar{centroid}{}`[dim];
    });
\end{lstlisting}

\subsection{表面积启发法}\label{sub:表面积启发法}
上述两个图元划分方法对一些图元分布效果不错,
但实际中它们经常选择性能较差的划分,导致光线要访问树的更多节点,
因此带来渲染时不必要的低效光线-图元相交计算。
当下光线追踪大部分最好的构建加速结构算法都基于“\keyindex{表面积启发法}{surface area heuristic}{}”(SAH),
它提供了全面的开销模型来回答问题,例如
“大量图元划分中哪个会为光线-图元相交测试带来更好的BVH?”,
或者“在空间划分方案里大量划分空间的可选位置中哪个会带来更好的加速结构?”

SAH模型估计执行光线相交测试的开销,包括穿行树的节点花的时间和
为特定的图元划分进行光线-图元相交测试花的时间。
然后构建加速结构的算法可以遵循最小化总开销的目标。
通常用贪婪算法独立地为正在构建的每个层次节点最小化开销。

SAH开销模型背后的思想很简单:构建自适应加速结构(图元划分或空间划分)的任意点处,
我们只用为当前区域和几何体创建一个叶子结点。
这种情况下,任何穿过该区域的光线都要对所有重合的图元测试,
且带来的开销为
\begin{align*}
    \sum\limits_{i=1}^{N}{t_{\mathrm{isect}}(i)}\, ,
\end{align*}
其中$N$是图元数量,$t_{\mathrm{isect}}(i)$是对第$i$个图元计算光线-物体相交的时间。

另一选项是划分空间。这种情况下,光线会带来开销
\begin{align}\label{eq:4.1}
    c(A,B)=t_{\mathrm{trav}}+p_A\sum\limits_{i=1}^{N_A}{t_\mathrm{isect}(a_i)}+p_B\sum\limits_{i=1}^{N_B}{t_{\mathrm{isect}}(b_i)}\, ,
\end{align}
其中$t_{\mathrm{trav}}$是穿行内部节点并确定光线穿过哪个子树所花的时间,分别地,
$p_A$和$p_B$是光线穿过每个子节点(假设二分划分)的概率,
$a_i$和$b_i$是两个子节点中图元的索引,
$N_A$和$N_B$是与两个子节点区域重合的图元数量。
怎样划分图元的选项会影响两个概率值以及划分出的两边的图元集合。

pbrt中,我们将作出简化假设即所有图元的$t_{\mathrm{trav}}(i)$都相同;
该假设可能和实际差不了多少,且它引入的任何误差看起来并不太影响加速器的性能。
另一种可能是向\refvar{Primitive}{}添加一个方法返回估计的相交测试所需的CPU周期数。

概率$p_A$和$p_B$可用来自几何概型的思想计算。
可以证明当凸体$A$包含于另一凸体$B$中,
穿过$B$的均匀分布的随机光线也穿过$A$的条件概率是它们表面积$s_A$与$s_B$的比:
\begin{align*}
    \displaystyle p(A|B)=\frac{s_A}{s_B}\, .
\end{align*}

因为我们对光线穿过节点的开销感兴趣,我们可以直接利用该结果。
因此,如果我们考虑细化一空间区域$A$使得有两个新子区域边界为$B$和$C$(\reffig{4.5}),
则穿过$A$的光线也会穿过两个子区域之一的概率很容易计算。
\begin{figure}[htbp]
    \centering%LaTeX with PSTricks extensions
%%Creator: Inkscape 1.0.1 (3bc2e813f5, 2020-09-07)
%%Please note this file requires PSTricks extensions
\psset{xunit=.5pt,yunit=.5pt,runit=.5pt}
\begin{pspicture}(552.39001465,175.75999451)
{
\newrgbcolor{curcolor}{0 0 0}
\pscustom[linewidth=1,linecolor=curcolor]
{
\newpath
\moveto(296.30999756,174.83999449)
\lineto(551.88999939,174.83999449)
\lineto(551.88999939,0.49999815)
\lineto(296.30999756,0.49999815)
\closepath
}
}
{
\newrgbcolor{curcolor}{0 0 0}
\pscustom[linewidth=1,linecolor=curcolor]
{
\newpath
\moveto(0.5,175.25999451)
\lineto(256.08000183,175.25999451)
\lineto(256.08000183,0.91999817)
\lineto(0.5,0.91999817)
\closepath
}
}
{
\newrgbcolor{curcolor}{0 0 0}
\pscustom[linewidth=1,linecolor=curcolor,linestyle=dashed,dash=2]
{
\newpath
\moveto(297.1499939,173.80999446)
\lineto(454.09999084,173.80999446)
\lineto(454.09999084,64.95999599)
\lineto(297.1499939,64.95999599)
\closepath
}
}
{
\newrgbcolor{curcolor}{0 0 0}
\pscustom[linewidth=1,linecolor=curcolor,linestyle=dashed,dash=2]
{
\newpath
\moveto(351.1499939,82.67999268)
\lineto(550.28999329,82.67999268)
\lineto(550.28999329,1.66999054)
\lineto(351.1499939,1.66999054)
\closepath
}
}
{
\newrgbcolor{curcolor}{0 0 0}
\pscustom[linestyle=none,fillstyle=solid,fillcolor=curcolor]
{
\newpath
\moveto(129.3839105,85.19464552)
\lineto(124.59874561,85.19464552)
\lineto(123.76020243,83.24412117)
\curveto(123.55360484,82.76408558)(123.45030604,82.40557798)(123.45030604,82.16859839)
\curveto(123.45030604,81.98022999)(123.53841384,81.813129)(123.71462943,81.6672954)
\curveto(123.89692143,81.52753821)(124.28581102,81.43639221)(124.88129821,81.39385741)
\lineto(124.88129821,81.05661722)
\lineto(120.9893641,81.05661722)
\lineto(120.9893641,81.39385741)
\curveto(121.50585808,81.48500341)(121.84006008,81.6034932)(121.99197007,81.7493268)
\curveto(122.30186647,82.04099399)(122.64518306,82.63344298)(123.02191985,83.52667376)
\lineto(127.36958395,93.69856712)
\lineto(127.68859494,93.69856712)
\lineto(131.99068604,83.41729856)
\curveto(132.33704083,82.59090818)(132.64997542,82.05314679)(132.92948982,81.8040144)
\curveto(133.21508061,81.5609584)(133.6100466,81.42423941)(134.11438779,81.39385741)
\lineto(134.11438779,81.05661722)
\lineto(129.2380769,81.05661722)
\lineto(129.2380769,81.39385741)
\curveto(129.73026529,81.41816301)(130.06142908,81.50019441)(130.23156828,81.6399516)
\curveto(130.40778388,81.7797088)(130.49589167,81.949848)(130.49589167,82.15036919)
\curveto(130.49589167,82.41773078)(130.37436368,82.84004057)(130.13130768,83.41729856)
\closepath
\moveto(129.12870171,85.8691259)
\lineto(127.03234376,90.86392659)
\lineto(124.88129821,85.8691259)
\closepath
}
}
{
\newrgbcolor{curcolor}{0 0 0}
\pscustom[linestyle=none,fillstyle=solid,fillcolor=curcolor]
{
\newpath
\moveto(380.5523089,122.51601247)
\curveto(381.40908128,122.33372047)(382.05014146,122.04205328)(382.47548945,121.64101089)
\curveto(383.06490024,121.0819821)(383.35960563,120.39838712)(383.35960563,119.59022594)
\curveto(383.35960563,118.97650955)(383.16516084,118.38709877)(382.77627125,117.82199358)
\curveto(382.38738166,117.26296479)(381.85265847,116.8528078)(381.17210168,116.59152261)
\curveto(380.4976213,116.33631381)(379.46463332,116.20870942)(378.07313776,116.20870942)
\lineto(372.23979389,116.20870942)
\lineto(372.23979389,116.54594961)
\lineto(372.70463848,116.54594961)
\curveto(373.22113247,116.54594961)(373.59179286,116.7100124)(373.81661966,117.038138)
\curveto(373.95637685,117.25081199)(374.02625545,117.70350378)(374.02625545,118.39621337)
\lineto(374.02625545,126.38060278)
\curveto(374.02625545,127.14622916)(373.93814765,127.62930295)(373.76193206,127.82982414)
\curveto(373.52495246,128.09718574)(373.17252127,128.23086654)(372.70463848,128.23086654)
\lineto(372.23979389,128.23086654)
\lineto(372.23979389,128.56810673)
\lineto(377.58094937,128.56810673)
\curveto(378.57747894,128.56810673)(379.37652553,128.49518993)(379.97808911,128.34935633)
\curveto(380.88954909,128.13060594)(381.58529687,127.74171635)(382.06533246,127.18268756)
\curveto(382.54536805,126.62973517)(382.78538585,125.99171319)(382.78538585,125.2686216)
\curveto(382.78538585,124.64882882)(382.59701745,124.09283823)(382.22028066,123.60064984)
\curveto(381.84354387,123.11453785)(381.28755328,122.75299206)(380.5523089,122.51601247)
\closepath
\moveto(375.77625861,123.00820086)
\curveto(376.0010854,122.96566606)(376.2562942,122.93224586)(376.54188499,122.90794026)
\curveto(376.83355219,122.88971106)(377.15256318,122.88059646)(377.49891797,122.88059646)
\curveto(378.38607235,122.88059646)(379.05143813,122.97478066)(379.49501532,123.16314905)
\curveto(379.94466891,123.35759385)(380.2879855,123.65229924)(380.5249651,124.04726523)
\curveto(380.76194469,124.44223122)(380.88043449,124.87365561)(380.88043449,125.3415384)
\curveto(380.88043449,126.06462999)(380.5857291,126.68138457)(379.99631831,127.19180216)
\curveto(379.40690753,127.70221975)(378.54709695,127.95742854)(377.41688657,127.95742854)
\curveto(376.80924659,127.95742854)(376.2623706,127.89058814)(375.77625861,127.75690735)
\closepath
\moveto(375.77625861,117.1019402)
\curveto(376.48112099,116.9378774)(377.17686878,116.855846)(377.86350196,116.855846)
\curveto(378.96333034,116.855846)(379.80187352,117.1019402)(380.3791315,117.59412858)
\curveto(380.95638949,118.09239337)(381.24501848,118.70610976)(381.24501848,119.43527774)
\curveto(381.24501848,119.91531333)(381.11437589,120.37711972)(380.85309069,120.82069691)
\curveto(380.5918055,121.2642741)(380.16645751,121.61366709)(379.57704672,121.86887588)
\curveto(378.98763593,122.12408468)(378.25846795,122.25168907)(377.38954277,122.25168907)
\curveto(377.01280598,122.25168907)(376.69075679,122.24561268)(376.42339519,122.23345988)
\curveto(376.1560336,122.22130708)(375.94032141,122.20003968)(375.77625861,122.16965768)
\closepath
}
}
{
\newrgbcolor{curcolor}{0 0 0}
\pscustom[linestyle=none,fillstyle=solid,fillcolor=curcolor]
{
\newpath
\moveto(456.98759904,49.00519332)
\lineto(457.27015163,44.80336282)
\lineto(456.98759904,44.80336282)
\curveto(456.61086225,46.06117759)(456.07310086,46.96656117)(455.37431487,47.51951356)
\curveto(454.67552889,48.07246594)(453.83698571,48.34894214)(452.85868533,48.34894214)
\curveto(452.03837135,48.34894214)(451.29705057,48.13930634)(450.63472299,47.72003475)
\curveto(449.9723954,47.30683956)(449.44982501,46.64451198)(449.06701182,45.733052)
\curveto(448.69027503,44.82159202)(448.50190664,43.68834344)(448.50190664,42.33330628)
\curveto(448.50190664,41.2152487)(448.68116043,40.24606293)(449.03966802,39.42574894)
\curveto(449.39817561,38.60543496)(449.935937,37.97652758)(450.65295219,37.53902679)
\curveto(451.37604377,37.101526)(452.19939595,36.8827756)(453.12300873,36.8827756)
\curveto(453.92509351,36.8827756)(454.63299409,37.0529148)(455.24671048,37.39319319)
\curveto(455.86042686,37.73954798)(456.53490725,38.42314297)(457.27015163,39.44397814)
\lineto(457.55270422,39.26168615)
\curveto(456.93291144,38.16185777)(456.20981986,37.35673479)(455.38342947,36.84631721)
\curveto(454.55703909,36.33589962)(453.57570052,36.08069082)(452.43941374,36.08069082)
\curveto(450.39166699,36.08069082)(448.80572663,36.84024081)(447.68159265,38.35934077)
\curveto(446.84304947,39.48955114)(446.42377788,40.82028271)(446.42377788,42.35153548)
\curveto(446.42377788,43.58504465)(446.70025408,44.71829322)(447.25320646,45.7512812)
\curveto(447.80615885,46.78426917)(448.56570883,47.58331575)(449.53185641,48.14842094)
\curveto(450.50408039,48.71960253)(451.56441216,49.00519332)(452.71285174,49.00519332)
\curveto(453.60608252,49.00519332)(454.4871605,48.78644293)(455.35608568,48.34894214)
\curveto(455.61129447,48.21526134)(455.79358646,48.14842094)(455.90296166,48.14842094)
\curveto(456.06702446,48.14842094)(456.20981986,48.20614674)(456.33134785,48.32159834)
\curveto(456.48933425,48.48566113)(456.60174765,48.71352613)(456.66858804,49.00519332)
\closepath
}
}
\end{pspicture}

    \caption{如果边界层次一个表面积为$s_A$的节点分为两个表面积为$s_B$和$s_C$的孩子,
        则穿过$A$的光线也穿过$B$和$C$的概率分别为$\displaystyle\frac{s_B}{s_A}$和$\displaystyle\frac{s_C}{s_A}$。}
    \label{fig:4.5}
\end{figure}

当\refvar{splitMethod}{}值为{\ttfamily\refvar{SplitMethod}{}::\refvar{SAH}{}}时,SAH用于构建BVH;
通过考虑大量候选划分来寻找沿所选轴给出最小SAH估计开销的图元划分
(这是默认\refvar{SplitMethod}{},且它为渲染创建最高效的树)。
然而,一旦它细分少量图元,则实现切换为划分成等量的子集。
这时应用SAH所增加的计算量是不划算的。
\begin{lstlisting}
`\initcode{Partition primitives using approximate SAH}{=}`
if (nPrimitives <= 4) {
    `\refcode{Partition primitives into equally sized subsets}{}`
} else {
    `\refcode{Allocate BucketInfo for SAH partition buckets}{}`
    `\refcode{Initialize BucketInfo for SAH partition buckets}{}`
    `\refcode{Compute costs for splitting after each bucket}{}`
    `\refcode{Find bucket to split at that minimizes SAH metric}{}`
    `\refcode{Either create leaf or split primitives at selected SAH bucket}{}`
}
\end{lstlisting}

比起穷举沿该轴所有$2n$个可能的划分,这里的实现是
将沿该轴的范围分为少量相等的较大范围,并为每个计算SAH以选择最好的。
然后它只考虑在该范围边界内的划分。
该方法比考虑所有划分更高效且通常仍产出几乎一样高效的划分。
\reffig{4.6}说明了该思想。
\begin{figure}[htbp]
    \centering%LaTeX with PSTricks extensions
%%Creator: Inkscape 1.0.1 (3bc2e813f5, 2020-09-07)
%%Please note this file requires PSTricks extensions
\psset{xunit=.5pt,yunit=.5pt,runit=.5pt}
\begin{pspicture}(394.22000122,242.08999634)
{
\newrgbcolor{curcolor}{0 0 0}
\pscustom[linewidth=1,linecolor=curcolor]
{
\newpath
\moveto(345.51998901,107.17999268)
\lineto(373.21998978,107.17999268)
\lineto(373.21998978,90.49999237)
\lineto(345.51998901,90.49999237)
\closepath
}
}
{
\newrgbcolor{curcolor}{0 0 0}
\pscustom[linestyle=none,fillstyle=solid,fillcolor=curcolor]
{
\newpath
\moveto(364.94000244,98.95999146)
\curveto(364.94000244,103.41482551)(359.55430967,105.6450073)(356.40464814,102.49534576)
\curveto(353.2549866,99.34568422)(355.48516838,93.95999146)(359.94000244,93.95999146)
\curveto(364.3948365,93.95999146)(366.62501828,99.34568422)(363.47535675,102.49534576)
\curveto(360.32569521,105.6450073)(354.94000244,103.41482551)(354.94000244,98.95999146)
\curveto(354.94000244,94.5051574)(360.32569521,92.27497561)(363.47535675,95.42463715)
\curveto(366.62501828,98.57429869)(364.3948365,103.95999146)(359.94000244,103.95999146)
\curveto(355.48516838,103.95999146)(353.2549866,98.57429869)(356.40464814,95.42463715)
\curveto(359.55430967,92.27497561)(364.94000244,94.5051574)(364.94000244,98.95999146)
\closepath
}
}
{
\newrgbcolor{curcolor}{0 0 0}
\pscustom[linewidth=1,linecolor=curcolor]
{
\newpath
\moveto(213.8500061,203.52999496)
\lineto(323.90000916,203.52999496)
\lineto(323.90000916,129.58999252)
\lineto(213.8500061,129.58999252)
\closepath
}
}
{
\newrgbcolor{curcolor}{0 0 0}
\pscustom[linestyle=none,fillstyle=solid,fillcolor=curcolor]
{
\newpath
\moveto(274.1499939,166.81999969)
\curveto(274.1499939,171.27483375)(268.76430113,173.50501554)(265.61463959,170.355354)
\curveto(262.46497805,167.20569246)(264.69515984,161.81999969)(269.1499939,161.81999969)
\curveto(273.60482796,161.81999969)(275.83500974,167.20569246)(272.6853482,170.355354)
\curveto(269.53568667,173.50501554)(264.1499939,171.27483375)(264.1499939,166.81999969)
\curveto(264.1499939,162.36516564)(269.53568667,160.13498385)(272.6853482,163.28464539)
\curveto(275.83500974,166.43430692)(273.60482796,171.81999969)(269.1499939,171.81999969)
\curveto(264.69515984,171.81999969)(262.46497805,166.43430692)(265.61463959,163.28464539)
\curveto(268.76430113,160.13498385)(274.1499939,162.36516564)(274.1499939,166.81999969)
\closepath
}
}
{
\newrgbcolor{curcolor}{0 0 0}
\pscustom[linewidth=1,linecolor=curcolor]
{
\newpath
\moveto(7.5,194.63999557)
\lineto(62.97999954,194.63999557)
\lineto(62.97999954,140.41999435)
\lineto(7.5,140.41999435)
\closepath
}
}
{
\newrgbcolor{curcolor}{0 0 0}
\pscustom[linestyle=none,fillstyle=solid,fillcolor=curcolor]
{
\newpath
\moveto(40.24000168,167.01999664)
\curveto(40.24000168,171.4748307)(34.85430891,173.70501249)(31.70464737,170.55535095)
\curveto(28.55498584,167.40568941)(30.78516762,162.01999664)(35.24000168,162.01999664)
\curveto(39.69483574,162.01999664)(41.92501752,167.40568941)(38.77535598,170.55535095)
\curveto(35.62569445,173.70501249)(30.24000168,171.4748307)(30.24000168,167.01999664)
\curveto(30.24000168,162.56516258)(35.62569445,160.3349808)(38.77535598,163.48464234)
\curveto(41.92501752,166.63430387)(39.69483574,172.01999664)(35.24000168,172.01999664)
\curveto(30.78516762,172.01999664)(28.55498584,166.63430387)(31.70464737,163.48464234)
\curveto(34.85430891,160.3349808)(40.24000168,162.56516258)(40.24000168,167.01999664)
\closepath
}
}
{
\newrgbcolor{curcolor}{0 0 0}
\pscustom[linewidth=1,linecolor=curcolor]
{
\newpath
\moveto(67.26999664,117.34999847)
\lineto(141.76999664,117.34999847)
\lineto(141.76999664,90.34999847)
\lineto(67.26999664,90.34999847)
\closepath
}
}
{
\newrgbcolor{curcolor}{0 0 0}
\pscustom[linestyle=none,fillstyle=solid,fillcolor=curcolor]
{
\newpath
\moveto(110.08000183,103.97000122)
\curveto(110.08000183,108.42483528)(104.69430906,110.65501706)(101.54464753,107.50535553)
\curveto(98.39498599,104.35569399)(100.62516777,98.97000122)(105.08000183,98.97000122)
\curveto(109.53483589,98.97000122)(111.76501767,104.35569399)(108.61535614,107.50535553)
\curveto(105.4656946,110.65501706)(100.08000183,108.42483528)(100.08000183,103.97000122)
\curveto(100.08000183,99.51516716)(105.4656946,97.28498538)(108.61535614,100.43464691)
\curveto(111.76501767,103.58430845)(109.53483589,108.97000122)(105.08000183,108.97000122)
\curveto(100.62516777,108.97000122)(98.39498599,103.58430845)(101.54464753,100.43464691)
\curveto(104.69430906,97.28498538)(110.08000183,99.51516716)(110.08000183,103.97000122)
\closepath
}
}
{
\newrgbcolor{curcolor}{0 0 0}
\pscustom[linewidth=1,linecolor=curcolor]
{
\newpath
\moveto(181.77000427,100.76998901)
\lineto(234.32000351,100.76998901)
\lineto(234.32000351,33.20999146)
\lineto(181.77000427,33.20999146)
\closepath
}
}
{
\newrgbcolor{curcolor}{0 0 0}
\pscustom[linestyle=none,fillstyle=solid,fillcolor=curcolor]
{
\newpath
\moveto(213.61000061,67.11000061)
\curveto(213.61000061,71.56483467)(208.22430784,73.79501645)(205.0746463,70.64535492)
\curveto(201.92498477,67.49569338)(204.15516655,62.11000061)(208.61000061,62.11000061)
\curveto(213.06483467,62.11000061)(215.29501645,67.49569338)(212.14535492,70.64535492)
\curveto(208.99569338,73.79501645)(203.61000061,71.56483467)(203.61000061,67.11000061)
\curveto(203.61000061,62.65516655)(208.99569338,60.42498477)(212.14535492,63.5746463)
\curveto(215.29501645,66.72430784)(213.06483467,72.11000061)(208.61000061,72.11000061)
\curveto(204.15516655,72.11000061)(201.92498477,66.72430784)(205.0746463,63.5746463)
\curveto(208.22430784,60.42498477)(213.61000061,62.65516655)(213.61000061,67.11000061)
\closepath
}
}
{
\newrgbcolor{curcolor}{0 0 0}
\pscustom[linewidth=1,linecolor=curcolor]
{
\newpath
\moveto(116.62000275,79.69999695)
\lineto(191.12000275,79.69999695)
\lineto(191.12000275,52.69999695)
\lineto(116.62000275,52.69999695)
\closepath
}
}
{
\newrgbcolor{curcolor}{0 0 0}
\pscustom[linestyle=none,fillstyle=solid,fillcolor=curcolor]
{
\newpath
\moveto(159.42999268,66.31999207)
\curveto(159.42999268,70.77482612)(154.04429991,73.00500791)(150.89463837,69.85534637)
\curveto(147.74497683,66.70568484)(149.97515862,61.31999207)(154.42999268,61.31999207)
\curveto(158.88482673,61.31999207)(161.11500852,66.70568484)(157.96534698,69.85534637)
\curveto(154.81568545,73.00500791)(149.42999268,70.77482612)(149.42999268,66.31999207)
\curveto(149.42999268,61.86515801)(154.81568545,59.63497622)(157.96534698,62.78463776)
\curveto(161.11500852,65.9342993)(158.88482673,71.31999207)(154.42999268,71.31999207)
\curveto(149.97515862,71.31999207)(147.74497683,65.9342993)(150.89463837,62.78463776)
\curveto(154.04429991,59.63497622)(159.42999268,61.86515801)(159.42999268,66.31999207)
\closepath
}
}
{
\newrgbcolor{curcolor}{0 0 0}
\pscustom[linewidth=1,linecolor=curcolor]
{
\newpath
\moveto(0,0.69000244)
\lineto(394.22000122,0.69000244)
}
}
{
\newrgbcolor{curcolor}{0 0 0}
\pscustom[linewidth=1,linecolor=curcolor,linestyle=dashed,dash=2]
{
\newpath
\moveto(359.92999268,99.09999084)
\lineto(359.92999268,1.22000122)
}
}
{
\newrgbcolor{curcolor}{0 0 0}
\pscustom[linewidth=1,linecolor=curcolor,linestyle=dashed,dash=2]
{
\newpath
\moveto(35.25,167.67999268)
\lineto(35.25,1.22000122)
}
}
{
\newrgbcolor{curcolor}{0 0.44313726 0.73725492}
\pscustom[linewidth=1,linecolor=curcolor]
{
\newpath
\moveto(304.57000732,241.84999634)
\lineto(304.57000732,0.8999939)
}
}
{
\newrgbcolor{curcolor}{0 0.44313726 0.73725492}
\pscustom[linewidth=1,linecolor=curcolor]
{
\newpath
\moveto(251.30999756,242.08999634)
\lineto(251.30999756,1.12998962)
}
}
{
\newrgbcolor{curcolor}{0 0.44313726 0.73725492}
\pscustom[linewidth=1,linecolor=curcolor]
{
\newpath
\moveto(197.08000183,241.46999633)
\lineto(197.08000183,0.50999451)
}
}
{
\newrgbcolor{curcolor}{0 0.44313726 0.73725492}
\pscustom[linewidth=1,linecolor=curcolor]
{
\newpath
\moveto(143.77000427,242.08999634)
\lineto(143.77000427,1.12998962)
}
}
{
\newrgbcolor{curcolor}{0 0.44313726 0.73725492}
\pscustom[linewidth=1,linecolor=curcolor]
{
\newpath
\moveto(89.22000122,241.77999634)
\lineto(89.22000122,0.81999207)
}
}
{
\newrgbcolor{curcolor}{0 0 0}
\pscustom[linewidth=1,linecolor=curcolor,linestyle=dashed,dash=2]
{
\newpath
\moveto(105.26000214,103.75)
\lineto(105.26000214,0.55999756)
}
}
{
\newrgbcolor{curcolor}{0 0 0}
\pscustom[linewidth=1,linecolor=curcolor,linestyle=dashed,dash=2]
{
\newpath
\moveto(154.63999939,66.58000183)
\lineto(154.63999939,0.83000183)
}
}
{
\newrgbcolor{curcolor}{0 0 0}
\pscustom[linewidth=1,linecolor=curcolor,linestyle=dashed,dash=2]
{
\newpath
\moveto(208.38999939,66.86000061)
\lineto(208.38999939,0)
}
}
{
\newrgbcolor{curcolor}{0 0 0}
\pscustom[linewidth=1,linecolor=curcolor,linestyle=dashed,dash=2]
{
\newpath
\moveto(269.35998535,167)
\lineto(269.35998535,0.27999878)
}
}
\end{pspicture}

    \caption{用表面积启发法为BVH选择划分平面。图元边界形心的投影范围被投影在选定的划分轴上。
        每个图元都基于其边界形心被放在沿轴的一个桶中。
        然后实现会估计沿每个桶的边界(粗蓝线)的平面划分图元的开销;
        谁的表面积启发法开销最小就选谁。}
    \label{fig:4.6}
\end{figure}
\begin{lstlisting}
`\initcode{Allocate BucketInfo for SAH partition buckets}{=}`
constexpr int nBuckets = 12;
struct `\initvar{BucketInfo}{}` {
    int `\initvar[BucketInfo::count]{count}{}` = 0;
    `\refvar{Bounds3f}{}` `\initvar[BucketInfo::bounds]{bounds}{}`;
};
`\refvar{BucketInfo}{}` buckets[nBuckets];
\end{lstlisting}

对于该范围内的每个图元,我们确定包含其形心的桶\protect\sidenote{译者注:原文bucket。}并更新桶的边界以包含图元的边界。
\begin{lstlisting}
`\initcode{Initialize BucketInfo for SAH partition buckets}{=}`
for (int i = start; i < end; ++i) {
    int b = nBuckets * 
        centroidBounds.`\refvar{Offset}{}`(primitiveInfo[i].`\refvar{centroid}{}`)[dim];
    if (b == nBuckets) b = nBuckets - 1;
    buckets[b].`\refvar[BucketInfo::count]{count}{}`++;
    buckets[b].`\refvar[BucketInfo::bounds]{bounds}{}` = `\refvar[Union2]{Union}{}`(buckets[b].`\refvar[BucketInfo::bounds]{bounds}{}`, primitiveInfo[i].`\refvar[BVHPrimitiveInfo::bounds]{bounds}{}`);
}
\end{lstlisting}

对于每个桶,我们现在都有图元数量以及全部相应边界框的边界。
我们想用SAH估计在每个桶边界处作划分的开销。
下面的代码片遍历所有桶并初始化数组{\ttfamily cost[i]}来
保存估计的在第{\ttfamily i}个桶后划分的SAH开销
(不考虑在最后一个桶之后划分,因为根据定义它并不划分图元)。

我们任意设置估计的相交开销为1,然后设置估计的遍历开销
为$\displaystyle\frac{1}{8}$(两者之一总是可以设为1,
因为是估计的遍历和相交开销的相对量级而不是绝对量级决定其影响)。
尽管遍历节点即光线-边界框相交的绝对计算量仅稍低于光线与形状相交所需的计算量,
但pbrt中光线-图元相交测试经过了两次虚函数调用,增加了大量开销,
所以这里我们估计其开销大于光线-框相交的八倍。

基于对桶前向和后向扫描而增量式地计算、存储和计数边界的线性时间实现是可能的,
但这里的计算关于桶的数量有$O(n^2)$复杂度。
更加高度优化解决该低效问题的渲染器是值得的,
但这里对于小的$n$,性能影响通常是可接受的。
\begin{lstlisting}
`\initcode{Compute costs for splitting after each bucket}{=}`
`\refvar{Float}{}` cost[nBuckets - 1];
for (int i = 0; i < nBuckets - 1; ++i) {
    `\refvar{Bounds3f}{}` b0, b1;
    int count0 = 0, count1 = 0;
    for (int j = 0; j <= i; ++j) {
        b0 = `\refvar[Union2]{Union}{}`(b0, buckets[j].`\refvar[BucketInfo::bounds]{bounds}{}`);
        count0 += buckets[j].`\refvar[BucketInfo::count]{count}{}`;
    }
    for (int j = i+1; j < nBuckets; ++j) {
        b1 = `\refvar[Union2]{Union}{}`(b1, buckets[j].`\refvar[BucketInfo::bounds]{bounds}{}`);
        count1 += buckets[j].`\refvar[BucketInfo::count]{count}{}`;
    }
    cost[i] = .125f + (count0 * b0.`\refvar{SurfaceArea}{}`() +
                       count1 * b1.`\refvar{SurfaceArea}{}`()) / bounds.`\refvar{SurfaceArea}{}`();
}
\end{lstlisting}

有了所有开销,对数组{\ttfamily cost}的线性扫描找到最小开销的划分。
\begin{lstlisting}
`\initcode{Find bucket to split at that minimizes SAH metric}{=}`
`\refvar{Float}{}` minCost = cost[0];
int minCostSplitBucket = 0;
for (int i = 1; i < nBuckets - 1; ++i) {
    if (cost[i] < minCost) {
        minCost = cost[i];
        minCostSplitBucket = i;
    }
}
\end{lstlisting}

如果为划分选的桶边界有比用存在的图元构建节点更低的估计开销,
或者出现一个节点的图元超过了允许的最大数量,
则用函数{\ttfamily std::partition()}来完成
在数组{\ttfamily primitiveInfo}中记录节点的工作。
回想之前它的用法即该函数确保数组的所有对于给定判定函数返回{\ttfamily true}的元素
都出现在返回{\ttfamily false}的之前,
并且它返回指向第一个判定函数返回{\ttfamily false}的元素的指针。
因为我们之前任意设置估计的相交开销为1,
所以只是创建叶子结点的估计开销等于图元的数量{\ttfamily nPrimitives}。
\begin{lstlisting}
`\initcode{Either create leaf or split primitives at selected SAH bucket}{=}`
`\refvar{Float}{}` leafCost = nPrimitives;
if (nPrimitives > maxPrimsInNode || minCost < leafCost) {
    `\refvar{BVHPrimitiveInfo}{}` *pmid = std::partition(&primitiveInfo[start],
        &primitiveInfo[end-1]+1, 
        [=](const `\refvar{BVHPrimitiveInfo}{}` &pi) {
            int b = nBuckets * centroidBounds.`\refvar{Offset}{}`(pi.centroid)[dim];
            if (b == nBuckets) b = nBuckets - 1;
            return b <= minCostSplitBucket;
        });
    mid = pmid - &primitiveInfo[0];
} else {
    `\refcode{Create leaf BVHBuildNode}{}`
}
\end{lstlisting}

\subsection{线性包围盒层次}\label{sub:线性包围盒层次}