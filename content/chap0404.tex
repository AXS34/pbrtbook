\section{kd树加速器}\label{sec:kd树加速器}

\keyindex{二叉空间划分}{binary space partitioning}{}(BSP)树用平面自适应地细分空间。
一个BSP树从包含整个场景的边界框开始。
如果框内图元的数量大于某个阈值,则用平面将该框分为两半。
然后图元和与之重合的任意一半关联,
同时位于两半里的图元就都与它们关联
(相比之下,BVH中划分后每个图元只能分配到两个组中的一个)。

划分过程递归进行,直到结果树中的每个叶子区域都包含足够少的图元或者达到最大深度。
因为划分平面可以放置于整个框内的任意位置,
且3D空间的不同部分可以精确到不同程度,
所以BSP易于处理分布不均的几何体。

BSP树的两个变种是\keyindex{kd树}{kd-tree}{tree树}和\keyindex{八叉树}{octree}{tree树}。
kd树\sidenote{译者注:“kd”是k个维度的缩写。}简单地限制划分平面垂直于一个坐标轴;
这让树的遍历和构建都更高效,而在如何划分空间上牺牲一些灵活性。
八叉树每一步用三个垂直于轴的平面同时将该框分为八个区域
(通常在每个方向沿范围中心划分)。

本节中,我们将在类\refvar{KdTreeAccel}{}中为光线相交加速实现一个kd树。
该类源码可在文件\href{https://github.com/mmp/pbrt-v3/tree/master/src/accelerators/kdtreeaccel.h}{\ttfamily accelerators/kdtreeaccel.h}
和\href{https://github.com/mmp/pbrt-v3/tree/master/src/accelerators/kdtreeaccel.cpp}{\ttfamily accelerators/kdtreeaccel.cpp}
中找到。

\begin{lstlisting}
`\initcode{KdTreeAccel Declarations}{=}\initnext{KdTreeAccelDeclarations}`
class `\initvar{KdTreeAccel}{}` : public `\refvar{Aggregate}{}` {
public:
    `\refcode{KdTreeAccel Public Methods}{}`
private:
    `\refcode{KdTreeAccel Private Methods}{}`
    `\refcode{KdTreeAccel Private Data}{}`
};
\end{lstlisting}

除了要保存的图元外,\refvar{KdTreeAccel}{}构造函数
还接收一些参数用于在构建树时指导要作出的决定;
这些参数存于成员变量中(\refvar{isectCost}{}、\refvar{traversalCost}{}、
\refvar{maxPrims}{}、{\ttfamily maxDepth}和\refvar{emptyBonus}{})留待后用。
见\reffig{4.14}中构建树的图示。
\begin{figure}[htbp]
    \centering%LaTeX with PSTricks extensions
%%Creator: Inkscape 1.0.1 (3bc2e813f5, 2020-09-07)
%%Please note this file requires PSTricks extensions
\psset{xunit=.5pt,yunit=.5pt,runit=.5pt}
\begin{pspicture}(300.01998901,255.22999573)
{
\newrgbcolor{curcolor}{0 0 0}
\pscustom[linewidth=1,linecolor=curcolor]
{
\newpath
\moveto(162.36999512,109.19000244)
\lineto(271.04999542,109.19000244)
\lineto(271.04999542,0.51000214)
\lineto(162.36999512,0.51000214)
\closepath
}
}
{
\newrgbcolor{curcolor}{0 0 0}
\pscustom[linewidth=1,linecolor=curcolor]
{
\newpath
\moveto(189.00999546,89.62998962)
\curveto(189.00999546,99.98302387)(180.09128852,105.16596629)(174.87544919,97.84615295)
\curveto(169.65960985,90.52633961)(173.35279076,78.00998974)(180.72999573,78.00998974)
\curveto(188.10720069,78.00998974)(191.80038161,90.52633961)(186.58454227,97.84615295)
\curveto(181.36870293,105.16596629)(172.44999599,99.98302387)(172.44999599,89.62998962)
\curveto(172.44999599,79.27695537)(181.36870293,74.09401296)(186.58454227,81.4138263)
\curveto(191.80038161,88.73363964)(188.10720069,101.24998951)(180.72999573,101.24998951)
\curveto(173.35279076,101.24998951)(169.65960985,88.73363964)(174.87544919,81.4138263)
\curveto(180.09128852,74.09401296)(189.00999546,79.27695537)(189.00999546,89.62998962)
\closepath
}
}
{
\newrgbcolor{curcolor}{0 0 0}
\pscustom[linewidth=1,linecolor=curcolor]
{
\newpath
\moveto(189.80999756,68.94000244)
\lineto(207.16999817,68.94000244)
\lineto(207.16999817,56.12000275)
\lineto(189.80999756,56.12000275)
\closepath
}
}
{
\newrgbcolor{curcolor}{0 0 0}
\pscustom[linewidth=1,linecolor=curcolor]
{
\newpath
\moveto(173.72853784,14.46987784)
\lineto(206.60367651,40.47971259)
\lineto(214.7193409,30.22191607)
\lineto(181.84420223,4.21208132)
\closepath
}
}
{
\newrgbcolor{curcolor}{0 0 0}
\pscustom[linewidth=1,linecolor=curcolor]
{
\newpath
\moveto(239.21,99.23999573)
\lineto(252.72,93.42999573)
\lineto(266.23,87.60999573)
\lineto(254.43,78.81999573)
\lineto(242.64,70.02999573)
\lineto(240.92,84.63999573)
\closepath
}
}
{
\newrgbcolor{curcolor}{0 0 0}
\pscustom[linewidth=1,linecolor=curcolor,linestyle=dashed,dash=2]
{
\newpath
\moveto(238.27999878,109.19999695)
\lineto(238.27999878,0.5)
}
}
{
\newrgbcolor{curcolor}{0 0 0}
\pscustom[linewidth=1,linecolor=curcolor,linestyle=dashed,dash=2]
{
\newpath
\moveto(162.69000244,41.63999939)
\lineto(237.83999634,41.63999939)
}
}
{
\newrgbcolor{curcolor}{0 0 0}
\pscustom[linewidth=1,linecolor=curcolor,linestyle=dashed,dash=2]
{
\newpath
\moveto(208.8500061,41.86000061)
\lineto(208.8500061,108.86999512)
}
}
{
\newrgbcolor{curcolor}{0 0 0}
\pscustom[linewidth=1,linecolor=curcolor]
{
\newpath
\moveto(17.13999939,109.19000244)
\lineto(125.81999969,109.19000244)
\lineto(125.81999969,0.51000214)
\lineto(17.13999939,0.51000214)
\closepath
}
}
{
\newrgbcolor{curcolor}{0 0 0}
\pscustom[linewidth=1,linecolor=curcolor]
{
\newpath
\moveto(43.77999973,89.62998962)
\curveto(43.77999973,99.98302387)(34.86129279,105.16596629)(29.64545346,97.84615295)
\curveto(24.42961412,90.52633961)(28.12279504,78.00998974)(35.5,78.00998974)
\curveto(42.87720496,78.00998974)(46.57038588,90.52633961)(41.35454654,97.84615295)
\curveto(36.13870721,105.16596629)(27.22000027,99.98302387)(27.22000027,89.62998962)
\curveto(27.22000027,79.27695537)(36.13870721,74.09401296)(41.35454654,81.4138263)
\curveto(46.57038588,88.73363964)(42.87720496,101.24998951)(35.5,101.24998951)
\curveto(28.12279504,101.24998951)(24.42961412,88.73363964)(29.64545346,81.4138263)
\curveto(34.86129279,74.09401296)(43.77999973,79.27695537)(43.77999973,89.62998962)
\closepath
}
}
{
\newrgbcolor{curcolor}{0 0 0}
\pscustom[linewidth=1,linecolor=curcolor]
{
\newpath
\moveto(44.58000183,68.94000244)
\lineto(61.94000244,68.94000244)
\lineto(61.94000244,56.12000275)
\lineto(44.58000183,56.12000275)
\closepath
}
}
{
\newrgbcolor{curcolor}{0 0 0}
\pscustom[linewidth=1,linecolor=curcolor]
{
\newpath
\moveto(28.48622023,14.46374212)
\lineto(61.36135891,40.47357687)
\lineto(69.4770233,30.21578035)
\lineto(36.60188463,4.2059456)
\closepath
}
}
{
\newrgbcolor{curcolor}{0 0 0}
\pscustom[linewidth=1,linecolor=curcolor]
{
\newpath
\moveto(93.98,99.23999573)
\lineto(107.49,93.42999573)
\lineto(120.99,87.60999573)
\lineto(109.2,78.81999573)
\lineto(97.41,70.02999573)
\lineto(95.69,84.63999573)
\closepath
}
}
{
\newrgbcolor{curcolor}{0 0 0}
\pscustom[linewidth=1,linecolor=curcolor,linestyle=dashed,dash=2]
{
\newpath
\moveto(93.05000305,109.19999695)
\lineto(93.05000305,0.5)
}
}
{
\newrgbcolor{curcolor}{0 0 0}
\pscustom[linewidth=1,linecolor=curcolor,linestyle=dashed,dash=2]
{
\newpath
\moveto(17.45999908,41.63999939)
\lineto(92.59999847,41.63999939)
}
}
{
\newrgbcolor{curcolor}{0 0 0}
\pscustom[linewidth=1,linecolor=curcolor]
{
\newpath
\moveto(162.36999512,254.71999574)
\lineto(271.04999542,254.71999574)
\lineto(271.04999542,146.03999543)
\lineto(162.36999512,146.03999543)
\closepath
}
}
{
\newrgbcolor{curcolor}{0 0 0}
\pscustom[linewidth=1,linecolor=curcolor]
{
\newpath
\moveto(189.00999546,235.15999603)
\curveto(189.00999546,245.51303028)(180.09128852,250.6959727)(174.87544919,243.37615936)
\curveto(169.65960985,236.05634602)(173.35279076,223.53999615)(180.72999573,223.53999615)
\curveto(188.10720069,223.53999615)(191.80038161,236.05634602)(186.58454227,243.37615936)
\curveto(181.36870293,250.6959727)(172.44999599,245.51303028)(172.44999599,235.15999603)
\curveto(172.44999599,224.80696178)(181.36870293,219.62401937)(186.58454227,226.94383271)
\curveto(191.80038161,234.26364604)(188.10720069,246.77999592)(180.72999573,246.77999592)
\curveto(173.35279076,246.77999592)(169.65960985,234.26364604)(174.87544919,226.94383271)
\curveto(180.09128852,219.62401937)(189.00999546,224.80696178)(189.00999546,235.15999603)
\closepath
}
}
{
\newrgbcolor{curcolor}{0 0 0}
\pscustom[linewidth=1,linecolor=curcolor]
{
\newpath
\moveto(189.80999756,214.46999741)
\lineto(207.16999817,214.46999741)
\lineto(207.16999817,201.64999771)
\lineto(189.80999756,201.64999771)
\closepath
}
}
{
\newrgbcolor{curcolor}{0 0 0}
\pscustom[linewidth=1,linecolor=curcolor]
{
\newpath
\moveto(173.72246,159.99963334)
\lineto(206.59759867,186.00946809)
\lineto(214.71326307,175.75167157)
\lineto(181.83812439,149.74183682)
\closepath
}
}
{
\newrgbcolor{curcolor}{0 0 0}
\pscustom[linewidth=1,linecolor=curcolor]
{
\newpath
\moveto(239.21,244.76999573)
\lineto(252.72,238.94999573)
\lineto(266.23,233.12999573)
\lineto(254.43,224.34999573)
\lineto(242.64,215.55999573)
\lineto(240.92,230.15999573)
\closepath
}
}
{
\newrgbcolor{curcolor}{0 0 0}
\pscustom[linewidth=1,linecolor=curcolor,linestyle=dashed,dash=2]
{
\newpath
\moveto(238.27999878,254.72999573)
\lineto(238.27999878,146.01999664)
}
}
{
\newrgbcolor{curcolor}{0 0 0}
\pscustom[linewidth=1,linecolor=curcolor]
{
\newpath
\moveto(17.13999939,254.72999573)
\lineto(125.81999969,254.72999573)
\lineto(125.81999969,146.04999542)
\lineto(17.13999939,146.04999542)
\closepath
}
}
{
\newrgbcolor{curcolor}{0 0 0}
\pscustom[linewidth=1,linecolor=curcolor]
{
\newpath
\moveto(43.77999973,235.16999626)
\curveto(43.77999973,245.52303051)(34.86129279,250.70597293)(29.64545346,243.38615959)
\curveto(24.42961412,236.06634625)(28.12279504,223.54999638)(35.5,223.54999638)
\curveto(42.87720496,223.54999638)(46.57038588,236.06634625)(41.35454654,243.38615959)
\curveto(36.13870721,250.70597293)(27.22000027,245.52303051)(27.22000027,235.16999626)
\curveto(27.22000027,224.81696201)(36.13870721,219.6340196)(41.35454654,226.95383294)
\curveto(46.57038588,234.27364627)(42.87720496,246.78999615)(35.5,246.78999615)
\curveto(28.12279504,246.78999615)(24.42961412,234.27364627)(29.64545346,226.95383294)
\curveto(34.86129279,219.6340196)(43.77999973,224.81696201)(43.77999973,235.16999626)
\closepath
}
}
{
\newrgbcolor{curcolor}{0 0 0}
\pscustom[linewidth=1,linecolor=curcolor]
{
\newpath
\moveto(44.58000183,214.47999573)
\lineto(61.94000244,214.47999573)
\lineto(61.94000244,201.65999603)
\lineto(44.58000183,201.65999603)
\closepath
}
}
{
\newrgbcolor{curcolor}{0 0 0}
\pscustom[linewidth=1,linecolor=curcolor]
{
\newpath
\moveto(28.47773521,160.0091797)
\lineto(61.35287388,186.01901445)
\lineto(69.46853827,175.76121793)
\lineto(36.5933996,149.75138318)
\closepath
}
}
{
\newrgbcolor{curcolor}{0 0 0}
\pscustom[linewidth=1,linecolor=curcolor]
{
\newpath
\moveto(93.98,244.78999573)
\lineto(107.49,238.96999573)
\lineto(120.99,233.14999573)
\lineto(109.2,224.35999573)
\lineto(97.41,215.56999573)
\lineto(95.69,230.17999573)
\closepath
}
}
{
\newrgbcolor{curcolor}{0 0 0}
\pscustom[linewidth=1,linecolor=curcolor]
{
\newpath
\moveto(4.13,175.02999573)
\lineto(4.13,131.26999573)
\lineto(48.29,131.26999573)
}
}
{
\newrgbcolor{curcolor}{0 0 0}
\pscustom[linestyle=none,fillstyle=solid,fillcolor=curcolor]
{
\newpath
\moveto(0,171.33999573)
\lineto(4.13,174.53999573)
\lineto(8.26,171.33999573)
\lineto(4.13,181.09999573)
\closepath
}
}
{
\newrgbcolor{curcolor}{0.65098041 0.65098041 0.65098041}
\pscustom[linestyle=none,fillstyle=solid,fillcolor=curcolor]
{
\newpath
\moveto(0.9,172.50999573)
\lineto(4.13,180.11999573)
\lineto(4.13,175.00999573)
\closepath
}
}
{
\newrgbcolor{curcolor}{0.40000001 0.40000001 0.40000001}
\pscustom[linestyle=none,fillstyle=solid,fillcolor=curcolor]
{
\newpath
\moveto(7.35,172.50999573)
\lineto(4.13,180.11999573)
\lineto(4.13,175.00999573)
\closepath
}
}
{
\newrgbcolor{curcolor}{0 0 0}
\pscustom[linestyle=none,fillstyle=solid,fillcolor=curcolor]
{
\newpath
\moveto(44.61,127.13999573)
\lineto(47.8,131.26999573)
\lineto(44.61,135.38999573)
\lineto(54.37,131.26999573)
\closepath
}
}
{
\newrgbcolor{curcolor}{0.65098041 0.65098041 0.65098041}
\pscustom[linestyle=none,fillstyle=solid,fillcolor=curcolor]
{
\newpath
\moveto(45.78,128.03999573)
\lineto(53.38,131.26999573)
\lineto(48.28,131.26999573)
\closepath
}
}
{
\newrgbcolor{curcolor}{0.40000001 0.40000001 0.40000001}
\pscustom[linestyle=none,fillstyle=solid,fillcolor=curcolor]
{
\newpath
\moveto(45.78,134.48999573)
\lineto(53.38,131.26999573)
\lineto(48.28,131.26999573)
\closepath
}
}
{
\newrgbcolor{curcolor}{0 0 0}
\pscustom[linewidth=1,linecolor=curcolor]
{
\newpath
\moveto(134.3500061,199.18999481)
\lineto(149.58999634,199.18999481)
}
}
{
\newrgbcolor{curcolor}{0 0 0}
\pscustom[linestyle=none,fillstyle=solid,fillcolor=curcolor]
{
\newpath
\moveto(147.14,196.43999573)
\lineto(149.27,199.18999573)
\lineto(147.14,201.93999573)
\lineto(153.65,199.18999573)
\closepath
}
}
{
\newrgbcolor{curcolor}{0.65098041 0.65098041 0.65098041}
\pscustom[linestyle=none,fillstyle=solid,fillcolor=curcolor]
{
\newpath
\moveto(147.92,197.03999573)
\lineto(152.99,199.18999573)
\lineto(149.58,199.18999573)
\closepath
}
}
{
\newrgbcolor{curcolor}{0.40000001 0.40000001 0.40000001}
\pscustom[linestyle=none,fillstyle=solid,fillcolor=curcolor]
{
\newpath
\moveto(147.92,201.33999573)
\lineto(152.99,199.18999573)
\lineto(149.58,199.18999573)
\closepath
}
}
{
\newrgbcolor{curcolor}{0 0 0}
\pscustom[linewidth=1,linecolor=curcolor]
{
\newpath
\moveto(280.73001099,199.1099968)
\lineto(295.97000122,199.1099968)
}
}
{
\newrgbcolor{curcolor}{0 0 0}
\pscustom[linestyle=none,fillstyle=solid,fillcolor=curcolor]
{
\newpath
\moveto(293.51,196.35999573)
\lineto(295.64,199.10999573)
\lineto(293.51,201.85999573)
\lineto(300.02,199.10999573)
\closepath
}
}
{
\newrgbcolor{curcolor}{0.65098041 0.65098041 0.65098041}
\pscustom[linestyle=none,fillstyle=solid,fillcolor=curcolor]
{
\newpath
\moveto(294.29,196.95999573)
\lineto(299.36,199.10999573)
\lineto(295.96,199.10999573)
\closepath
}
}
{
\newrgbcolor{curcolor}{0.40000001 0.40000001 0.40000001}
\pscustom[linestyle=none,fillstyle=solid,fillcolor=curcolor]
{
\newpath
\moveto(294.29,201.25999573)
\lineto(299.36,199.10999573)
\lineto(295.96,199.10999573)
\closepath
}
}
{
\newrgbcolor{curcolor}{0 0 0}
\pscustom[linewidth=1,linecolor=curcolor]
{
\newpath
\moveto(134.83000183,55.22000122)
\lineto(150.07000732,55.22000122)
}
}
{
\newrgbcolor{curcolor}{0 0 0}
\pscustom[linestyle=none,fillstyle=solid,fillcolor=curcolor]
{
\newpath
\moveto(147.61,52.46999573)
\lineto(149.74,55.21999573)
\lineto(147.61,57.96999573)
\lineto(154.12,55.21999573)
\closepath
}
}
{
\newrgbcolor{curcolor}{0.65098041 0.65098041 0.65098041}
\pscustom[linestyle=none,fillstyle=solid,fillcolor=curcolor]
{
\newpath
\moveto(148.39,53.06999573)
\lineto(153.46,55.21999573)
\lineto(150.06,55.21999573)
\closepath
}
}
{
\newrgbcolor{curcolor}{0.40000001 0.40000001 0.40000001}
\pscustom[linestyle=none,fillstyle=solid,fillcolor=curcolor]
{
\newpath
\moveto(148.39,57.36999573)
\lineto(153.46,55.21999573)
\lineto(150.06,55.21999573)
\closepath
}
}
{
\newrgbcolor{curcolor}{0 0 0}
\pscustom[linestyle=none,fillstyle=solid,fillcolor=curcolor]
{
\newpath
\moveto(9.06889958,192.57739907)
\curveto(9.13921208,192.78833657)(9.13921208,192.81177407)(9.13921208,192.92896157)
\curveto(9.13921208,193.18677407)(8.92827458,193.32739907)(8.69389958,193.32739907)
\curveto(8.55327458,193.32739907)(8.31889958,193.23364907)(8.17827458,193.02271157)
\curveto(8.15483708,192.92896157)(8.01421208,192.48364907)(7.96733708,192.20239907)
\curveto(7.85014958,191.82739907)(7.75639958,191.40552407)(7.66264958,191.00708657)
\lineto(6.98296208,188.31177407)
\curveto(6.93608708,188.10083657)(6.27983708,187.04614907)(5.29546208,187.04614907)
\curveto(4.54546208,187.04614907)(4.38139958,187.70239907)(4.38139958,188.26489907)
\curveto(4.38139958,188.94458657)(4.63921208,189.88208657)(5.13139958,191.19458657)
\curveto(5.36577458,191.80396157)(5.43608708,191.96802407)(5.43608708,192.27271157)
\curveto(5.43608708,192.92896157)(4.96733708,193.49146157)(4.21733708,193.49146157)
\curveto(2.78764958,193.49146157)(2.24858708,191.31177407)(2.24858708,191.19458657)
\curveto(2.24858708,191.03052407)(2.38921208,191.03052407)(2.41264958,191.03052407)
\curveto(2.57671208,191.03052407)(2.57671208,191.07739907)(2.64702458,191.31177407)
\curveto(3.06889958,192.71802407)(3.65483708,193.16333657)(4.17046208,193.16333657)
\curveto(4.28764958,193.16333657)(4.54546208,193.16333657)(4.54546208,192.69458657)
\curveto(4.54546208,192.31958657)(4.38139958,191.92114907)(4.28764958,191.63989907)
\curveto(3.67827458,190.04614907)(3.42046208,189.20239907)(3.42046208,188.49927407)
\curveto(3.42046208,187.16333657)(4.35796208,186.71802407)(5.24858708,186.71802407)
\curveto(5.83452458,186.71802407)(6.32671208,186.97583657)(6.74858708,187.39771157)
\curveto(6.56108708,186.62427407)(6.37358708,185.87427407)(5.78764958,185.07739907)
\curveto(5.38921208,184.58521157)(4.82671208,184.13989907)(4.14702458,184.13989907)
\curveto(3.93608708,184.13989907)(3.25639958,184.18677407)(2.99858708,184.77271157)
\curveto(3.23296208,184.77271157)(3.44389958,184.77271157)(3.63139958,184.96021157)
\curveto(3.79546208,185.07739907)(3.93608708,185.28833657)(3.93608708,185.56958657)
\curveto(3.93608708,186.03833657)(3.53764958,186.08521157)(3.39702458,186.08521157)
\curveto(3.04546208,186.08521157)(2.55327458,185.85083657)(2.55327458,185.12427407)
\curveto(2.55327458,184.37427407)(3.20952458,183.81177407)(4.14702458,183.81177407)
\curveto(5.67046208,183.81177407)(7.21733708,185.17114907)(7.63921208,186.85864907)
\closepath
\moveto(9.06889958,192.57739907)
}
}
{
\newrgbcolor{curcolor}{0 0 0}
\pscustom[linestyle=none,fillstyle=solid,fillcolor=curcolor]
{
\newpath
\moveto(62.45644408,132.90413867)
\curveto(62.55019408,133.27913867)(62.90175658,134.66195117)(63.93300658,134.66195117)
\curveto(64.00331908,134.66195117)(64.37831908,134.66195117)(64.68300658,134.47445117)
\curveto(64.26113158,134.38070117)(63.97988158,134.02913867)(63.97988158,133.65413867)
\curveto(63.97988158,133.41976367)(64.14394408,133.13851367)(64.54238158,133.13851367)
\curveto(64.87050658,133.13851367)(65.33925658,133.39632617)(65.33925658,134.00570117)
\curveto(65.33925658,134.77913867)(64.47206908,134.99007617)(63.95644408,134.99007617)
\curveto(63.08925658,134.99007617)(62.57363158,134.19320117)(62.38613158,133.86507617)
\curveto(62.01113158,134.84945117)(61.21425658,134.99007617)(60.76894408,134.99007617)
\curveto(59.22206908,134.99007617)(58.35488158,133.06820117)(58.35488158,132.69320117)
\curveto(58.35488158,132.52913867)(58.51894408,132.52913867)(58.54238158,132.52913867)
\curveto(58.65956908,132.52913867)(58.70644408,132.57601367)(58.72988158,132.69320117)
\curveto(59.24550658,134.28695117)(60.22988158,134.66195117)(60.74550658,134.66195117)
\curveto(61.02675658,134.66195117)(61.54238158,134.52132617)(61.54238158,133.65413867)
\curveto(61.54238158,133.18538867)(61.28456908,132.20101367)(60.74550658,130.09163867)
\curveto(60.51113158,129.17757617)(59.97206908,128.54476367)(59.31581908,128.54476367)
\curveto(59.22206908,128.54476367)(58.89394408,128.54476367)(58.56581908,128.73226367)
\curveto(58.94081908,128.82601367)(59.26894408,129.13070117)(59.26894408,129.55257617)
\curveto(59.26894408,129.95101367)(58.94081908,130.06820117)(58.72988158,130.06820117)
\curveto(58.26113158,130.06820117)(57.90956908,129.69320117)(57.90956908,129.20101367)
\curveto(57.90956908,128.52132617)(58.63613158,128.21663867)(59.29238158,128.21663867)
\curveto(60.30019408,128.21663867)(60.83925658,129.27132617)(60.86269408,129.34163867)
\curveto(61.05019408,128.80257617)(61.58925658,128.21663867)(62.47988158,128.21663867)
\curveto(64.02675658,128.21663867)(64.87050658,130.13851367)(64.87050658,130.51351367)
\curveto(64.87050658,130.67757617)(64.75331908,130.67757617)(64.70644408,130.67757617)
\curveto(64.56581908,130.67757617)(64.54238158,130.60726367)(64.49550658,130.51351367)
\curveto(64.00331908,128.89632617)(62.99550658,128.54476367)(62.52675658,128.54476367)
\curveto(61.94081908,128.54476367)(61.70644408,129.01351367)(61.70644408,129.52913867)
\curveto(61.70644408,129.85726367)(61.77675658,130.18538867)(61.94081908,130.84163867)
\closepath
\moveto(62.45644408,132.90413867)
}
}
\end{pspicture}

    \caption{通过沿坐标轴之一递归地划分场景几何边界框来构建kd树。这里,第一次划分沿$x$轴;
        它摆放后使三角形刚好单独在右边区域而其余图元则在左边。
        然后再用轴对齐的划分平面细化若干次左边的区域。
        细化标准的细节——每一步用哪个轴划分空间、沿轴上哪个位置放置平面
        以及何时结束细分——在实践中均会极大影响树的性能。}
    \label{fig:4.14}
\end{figure}

\begin{lstlisting}
`\initcode{KdTreeAccel Method Definitions}{=}\initnext{KdTreeAccelMethodDefinitions}`
`\refvar{KdTreeAccel}{}`::`\refvar{KdTreeAccel}{}`(
        const std::vector<std::shared_ptr<`\refvar{Primitive}{}`>> &p,
        int isectCost, int traversalCost, `\refvar{Float}{}` emptyBonus,
        int maxPrims, int maxDepth)
    : `\refvar{isectCost}{}`(isectCost), `\refvar{traversalCost}{}`(traversalCost),
      `\refvar{maxPrims}{}`(maxPrims), `\refvar{emptyBonus}{}`(emptyBonus), `\refvar[KdTreeAccel::primitives]{primitives}{}`(p) {
    `\refcode{Build kd-tree for accelerator}{}`
}
\end{lstlisting}

\begin{lstlisting}
`\initcode{KdTreeAccel Private Data}{=}\initnext{KdTreeAccelPrivateData}`
const int `\initvar{isectCost}{}`, `\initvar{traversalCost}{}`, `\initvar{maxPrims}{}`;
const `\refvar{Float}{}` `\initvar{emptyBonus}{}`;
std::vector<std::shared_ptr<`\refvar{Primitive}{}`>> `\initvar[KdTreeAccel::primitives]{primitives}{}`;
\end{lstlisting}

\subsection{树状表示}\label{sub:树状表示}
kd树是二叉树,每个内部节点总是有两个孩子且树的叶子存有与之重合的图元。
每个内部节点必须提供三块信息的访问渠道:
\begin{itemize}
    \item 划分轴:该节点划分了$x,y$和$z$中的哪一个轴;
    \item 划分位置:划分平面沿该轴的位置;
    \item 孩子:关于如何到达其下两个子节点的信息。
\end{itemize}
每个叶子节点只需要记录哪个图元与之重合。

为了保证所有内部节点和许多叶子节点只用8字节内存
(假设\refvar{Float}{}占4字节)而麻烦一下是值得的,
因为这样做保证了八个节点契合一个64字节的缓存行。
因为树中经常有许多节点且每条光线通常都要访问许多节点,
最小化节点表示的大小能极大提高缓存性能。
我们最初的实现使用了16字节节点表示;
当我们把大小减少到8字节时我们得到了几乎20\%的提速。

叶子和内部节点都用下面的结构体\refvar{KdAccelNode}{}表示。
每个{\ttfamily union}成员后的注释都说明了特定域是用于内部节点、叶子节点还是两者都是。
\begin{lstlisting}
`\initcode{KdTreeAccel Local Declarations}{=}\initnext{KdTreeAccelLocalDeclarations}`
struct `\initvar{KdAccelNode}{}` {
    `\refcode{KdAccelNode Methods}{}`
    union {
        `\refvar{Float}{}` `\initvar[KdAccelNode::split]{split}{}`;                  // Interior
        int `\initvar{onePrimitive}{}`;             // Leaf
        int `\initvar{primitiveIndicesOffset}{}`;   // Leaf
    };
    union {
        int `\initvar[KdAccelNode::flags]{flags}{}`;         // Both
        int `\initvar{nPrims}{}`;        // Leaf
        int `\initvar{aboveChild}{}`;    // Interior
    };
};
\end{lstlisting}

变量\refvar{KdAccelNode::flags}{}的低两位用于区分用$x,y$和$z$划分的内部节点
(这些数位分别取值0,1和2)以及叶子节点(这些数位取值3)。
在8字节中保存叶子节点相对简单:\refvar{KdAccelNode::flags}{}的低2位
用于表示这是一个叶子,\refvar[nPrims]{KdAccelNode::nPrims}{}的高30位
可用于记录有多少个图元与之重合。
然后,如果只有一个图元与\refvar{KdAccelNode}{}叶子重合,
则指向数组\refvar{KdTreeAccel::primitives}{}
的整数索引会指出该\refvar{Primitive}{}。如果重合的图元多于一个,
则它们的索引保存于\refvar[primitiveIndices]{KdTreeAccel::primitiveIndices}{}的一段中。
该叶子第一个索引的偏移量存于\refvar[primitiveIndicesOffset]{KdAccelNode::primitiveIndicesOffset}{}且后面直接跟着剩下的索引。
\begin{lstlisting}
`\refcode{KdTreeAccel Private Data}{+=}\lastnext{KdTreeAccelPrivateData}`
std::vector<int> `\initvar{primitiveIndices}{}`;
\end{lstlisting}

叶子节点很容易初始化,不过我们要注意细节,
因为\refvar[KdAccelNode::flags]{flags}{}和\refvar{nPrims}{}共享同一存储;
我们需要注意在初始化其中一个时不要搞乱了另一个。
此外,在保存图元数量前必须向左移两位,
这样\refvar{KdAccelNode::flags}{}的低两位可以都设为1以表示这是一个叶子节点。
\begin{lstlisting}
`\refcode{KdTreeAccel Method Definitions}{+=}\lastnext{KdTreeAccelMethodDefinitions}`
void `\refvar{KdAccelNode}{}`::`\initvar[KdAccelNode::InitLeaf]{InitLeaf}{}`(int *primNums, int np,
        std::vector<int> *primitiveIndices) {
    `\refvar[KdAccelNode::flags]{flags}{}` = 3;
    `\refvar{nPrims}{}` |= (np << 2);
    `\refcode{Store primitive ids for leaf node}{}`
}
\end{lstlisting}

对于有零或一个重合图元的叶子节点,
因为有\refvar[onePrimitive]{KdAccelNode::onePrimitive}{}
域了,所以不再需要额外分配内存。
对于有多个重合图元的情况,则在数组{\ttfamily primitiveIndices}中分配存储。
\begin{lstlisting}
`\initcode{Store primitive ids for leaf node}{=}`
if (np == 0)
    `\refvar{onePrimitive}{}` = 0;
else if (np == 1)
    `\refvar{onePrimitive}{}` = primNums[0];
else {
    `\refvar{primitiveIndicesOffset}{}` = primitiveIndices->size();
    for (int i = 0; i < np; ++i)
        primitiveIndices->push_back(primNums[i]);
}
\end{lstlisting}

让内部节点减少到8字节也相当简单。
一个\refvar{Float}{}(当\refvar{Float}{}定义为{\ttfamily float}时其大小为32位)
保存了节点沿所选划分轴分割空间的位置,并且如之前所述,
\refvar{KdAccelNode::flags}{}低两位用于记录该节点是沿哪个轴划分的。
剩下的就是存储足够的信息使我们遍历树时能找到该节点的两个孩子。

我们排布节点的方式是只存储一个孩子指针,而不是存储两个指针或偏移量:
所有节点都分配到单个连续内存块,
内部节点的对应划分平面下方空间的孩子在数组中的保存位置总是紧跟其父亲
(通过在内存中保持至少一个孩子挨着其父亲,这样的排布也提高了缓存性能)。
另一个对应于划分平面上方的孩子,则在数组其他某处出现;
单个整数偏移量\refvar[aboveChild]{KdAccelNode::aboveChild}{}保存了它在节点数组中的位置。
该表示和\refsub{为遍历而压实的BVH}中BVH节点用的类似。

有了所有这些约定,初始化内部节点的代码就很简单了。
就像方法\refvar[KdAccelNode::InitLeaf]{InitLeaf}{()}
那样,在设置\refvar{aboveChild}{}前为\refvar[KdAccelNode::flags]{flags}{}赋值、
计算移位的\refvar{aboveChild}{}逻辑或值很重要,
这样才不会搞乱保存在\refvar[KdAccelNode::flags]{flags}{}中的数位。
\begin{lstlisting}
`\initcode{KdAccelNode Methods}{=}\initnext{KdAccelNodeMethods}`
void `\initvar[KdAccelNode::InitInterior]{InitInterior}{}`(int axis, int ac, `\refvar{Float}{}` s) {
    `\refvar[KdAccelNode::split]{split}{}` = s;
    `\refvar[KdAccelNode::flags]{flags}{}` = axis;
    `\refvar{aboveChild}{}` |= (ac << 2);
}
\end{lstlisting}

最后,我们将提供一些方法从节点中提取各种值,
这样调用者就不需要了解其内存表示的细节了。
\begin{lstlisting}
`\refcode{KdAccelNode Methods}{+=}\lastcode{KdAccelNodeMethods}`
`\refvar{Float}{}` `\initvar{SplitPos}{()}` const { return `\refvar[KdAccelNode::split]{split}{}`; }
int `\initvar[KdAccelNode::nPrimitives]{nPrimitives}{()}` const { return `\refvar{nPrims}{}` >> 2; }
int `\initvar[KdAccelNode::SplitAxis]{SplitAxis}{()}` const { return `\refvar[KdAccelNode::flags]{flags}{}` & 3; }
bool `\initvar{IsLeaf}{()}` const { return (`\refvar[KdAccelNode::flags]{flags}{}` & 3) == 3; }
int `\initvar{AboveChild}{()}` const { return `\refvar{aboveChild}{}` >> 2; }
\end{lstlisting}

\subsection{树的构建}\label{sub:树的构建}
kd树是用递归自顶向下算法构建的。
每一步中,我们有一个轴对齐空间区域和与该区域重合的图元集。
要么该区域分为两个子区域且转化为内部节点,
要么用重合的图元创建一个叶子节点,结束递归。

正如讨论\refvar{KdAccelNode}{}时所提到的,
所有树节点都保存于连续数组中。\newline
\refvar[nextFreeNode]{KdTreeAccel::nextFreeNode}{}记录了该数组中下一个有效节点,
\refvar[nAllocedNodes]{KdTreeAccel::\newline nAllocedNodes}{}记录了已经分配的总数。
通过一开始设置两者为0且不分配任何节点,这里的实现保证了当初始化树的第一个节点时能立即完成分配。

如果没有为构造函数给定,则还有必要确定树的最大深度。
尽管树的构建过程通常会自然地在合理的深度结束,
但限制最大深度很重要,这样极端情况下树所用的内存数量才不会无限增长。
我们已经发现值$8+1.3\log(N)$为大量场景给出了合理的最大深度\sidenote{译者注:对数以2为底。}。

\begin{lstlisting}
`\initcode{Build kd-tree for accelerator}{=}`
`\refvar{nextFreeNode}{}` = `\refvar{nAllocedNodes}{}` = 0;
if (maxDepth <= 0)
    maxDepth = std::round(8 + 1.3f * `\refvar{Log2Int}{}`(`\refvar[KdTreeAccel::primitives]{primitives}{}`.size()));
`\refcode{Compute bounds for kd-tree construction}{}`
`\refcode{Allocate working memory for kd-tree construction}{}`
`\refcode{Initialize primNums for kd-tree construction}{}`
`\refcode{Start recursive construction of kd-tree}{}`
\end{lstlisting}

\begin{lstlisting}
`\refcode{KdTreeAccel Private Data}{+=}\lastnext{KdTreeAccelPrivateData}`
`\refvar{KdAccelNode}{}` *`\initvar[KdTreeAccel::nodes]{nodes}{}`;
int `\initvar{nAllocedNodes}{}`, `\initvar{nextFreeNode}{}`;
\end{lstlisting}

因为构建例程会一路重复使用图元边界框,
所以在开始构建树前它们被保存在{\ttfamily vector}中,
这样就不需重复调用可能更慢的方法\refvar{Primitive::WorldBound}{()}。
\begin{lstlisting}
`\initcode{Compute bounds for kd-tree construction}{=}`
std::vector<`\refvar{Bounds3f}{}`> primBounds;
for (const std::shared_ptr<`\refvar{Primitive}{}`> &prim : `\refvar[KdTreeAccel::primitives]{primitives}{}`) {
    `\refvar{Bounds3f}{}` b = prim->`\refvar[Primitive::WorldBound]{WorldBound}{}`();
    `\refvar[KdTreeAccel::bounds]{bounds}{}` = `\refvar[Union2]{Union}{}`(bounds, b);
    primBounds.push_back(b);
}
\end{lstlisting}

\begin{lstlisting}
`\refcode{KdTreeAccel Private Data}{+=}\lastcode{KdTreeAccelPrivateData}`
`\refvar{Bounds3f}{}` `\initvar[KdTreeAccel::bounds]{bounds}{}`;
\end{lstlisting}

树构建例程的参数之一是图元索引数组,表示哪个图元与当前节点重合。
因为(当递归开始时)所有图元都和根节点重合,
所以我们从初始化值为零到{\ttfamily primitives.size()-1}的数组开始。
\begin{lstlisting}
`\initcode{Initialize primNums for kd-tree construction}{=}`
std::unique_ptr<int[]> primNums(new int[`\refvar[KdTreeAccel::primitives]{primitives}{}`.size()]);
for (size_t i = 0; i < `\refvar[KdTreeAccel::primitives]{primitives}{}`.size(); ++i)
    primNums[i] = i;
\end{lstlisting}

每个树节点都会调用\refvar{KdTreeAccel::buildTree}{()}。
它负责决定该节点应该是内部节点还是叶子并适当更新数据结构。
最后三个参数{\ttfamily edges}、{\ttfamily prims0}、{\ttfamily prims1}是
指向分配于代码片\refcode{Allocate working memory for kd-tree construction}{}的数据的指针,
稍后几页会对此作定义和介绍。
\begin{lstlisting}
`\initcode{Start recursive construction of kd-tree}{=}`
`\refvar[KdTreeAccel::buildTree]{buildTree}{}`(0, `\refvar[KdTreeAccel::bounds]{bounds}{}`, primBounds, primNums.get(), `\refvar[KdTreeAccel::primitives]{primitives}{}`.size(), 
          maxDepth, edges, prims0.get(), prims1.get());
\end{lstlisting}

\refvar{KdTreeAccel::buildTree}{()}的主要参数是供创建的节点使用的相对于
\refvar{KdAccelNode}{}数组的偏移量{\ttfamily nodeNum}、
给出该节点覆盖的空间区域边界框的{\ttfamily nodeBounds},
以及与之重合的图元索引{\ttfamily primNums}。
其余参数稍后在快用到它们时阐述。
\begin{lstlisting}
`\refcode{KdTreeAccel Method Definitions}{+=}\lastnext{KdTreeAccelMethodDefinitions}`
void `\refvar{KdTreeAccel}{}::\initvar[KdTreeAccel::buildTree]{buildTree}{}`(int nodeNum, const `\refvar{Bounds3f}{}` &nodeBounds,
        const std::vector<`\refvar{Bounds3f}{}`> &allPrimBounds, int *primNums,
        int nPrimitives, int depth,
        const std::unique_ptr<`\refvar{BoundEdge}{}`[]> edges[3], 
        int *prims0, int *prims1, int badRefines) {
    `\refcode{Get next free node from nodes array}{}`
    `\refcode{Initialize leaf node if termination criteria met}{}`
    `\refcode{Initialize interior node and continue recursion}{}`
}
\end{lstlisting}

如果所有分配的节点都已经用完了,则重新分配两倍数量的节点内存并复制旧值。
第一次调用\refvar{KdTreeAccel::buildTree}{()}时,
\refvar[nAllocedNodes]{KdTreeAccel::nAllocedNodes}{}
为0并分配树节点的一个初始块。
\begin{lstlisting}
`\initcode{Get next free node from nodes array}{=}`
if (`\refvar{nextFreeNode}{}` == `\refvar{nAllocedNodes}{}`) {
    int nNewAllocNodes = std::max(2 * `\refvar{nAllocedNodes}{}`, 512);
    `\refvar{KdAccelNode}{}` *n = `\refvar{AllocAligned}{}`<`\refvar{KdAccelNode}{}`>(nNewAllocNodes);
    if (`\refvar{nAllocedNodes}{}` > 0) {
        memcpy(n, `\refvar[KdTreeAccel::nodes]{nodes}{}`, `\refvar{nAllocedNodes}{}` * sizeof(`\refvar{KdAccelNode}{}`));
        `\refvar{FreeAligned}{}`(`\refvar[KdTreeAccel::nodes]{nodes}{}`);
    }
    `\refvar[KdTreeAccel::nodes]{nodes}{}` = n;
    `\refvar{nAllocedNodes}{}` = nNewAllocNodes;
}
++`\refvar{nextFreeNode}{}`;
\end{lstlisting}

当区域内有足够少量的图元或达到最大深度时就创建叶子节点(停止递归)。
参数{\ttfamily depth}一开始为树的最大深度,且每一层递减。
\begin{lstlisting}
`\initcode{Initialize leaf node if termination criteria met}{=}`
if (nPrimitives <= `\refvar{maxPrims}{}` || depth == 0) {
    `\refvar[KdTreeAccel::nodes]{nodes}{}`[nodeNum].`\refvar[KdAccelNode::InitLeaf]{InitLeaf}{}`(primNums, nPrimitives, &`\refvar{primitiveIndices}{}`);
    return;
}
\end{lstlisting}

若这是个内部节点,则需要选择一个划分平面,按该平面划分图元并递归。
\begin{lstlisting}
`\initcode{Initialize interior node and continue recursion}{=}`
`\refcode{Choose split axis position for interior node}{}`
`\refcode{Create leaf if no good splits were found}{}`
`\refcode{Classify primitives with respect to split}{}`
`\refcode{Recursively initialize children nodes}{}`
\end{lstlisting}

我们的实现选择用\refsub{表面积启发法}介绍的SAH来划分。
SAH适用于kd树和BVH;为节点中一系列候选划分平面计算估计的开销,
并选择给出最少开销的划分。

在这里的实现中,相交开销$t_{\text{isect}}$和遍历开销$t_{\text{trav}}$可由用户设置;
它们的默认值分别是80和1。
重要的是,这两个值的比例决定了树构建算法的表现
\footnote{该方法的许多其他实现似乎给这些开销使用了接近得多的值,
    有时甚至接近相等值(例如,见Hurley等\parencite*{hurley2002fast})。
    在pbrt中这里所用的值为大量测试场景给出了最好的性能。
    我们怀疑这一矛盾是因为pbrt中光线-图元相交测试需要两次虚函数调用
    以及一次光线从世界到物体空间的变换这一事实,
    此外还有执行实际相交测试的开销。
    只支持三角图元的高度优化的光线追踪器不会有此类任何额外开销。
    见\refsub{只有三角形}关于这一平衡设计的更多讨论。}。
比起BVH所用的值,这些值之间更大的比例反映的事实是
访问kd树的节点比访问BVH节点的开销更少。

针对用于BVH树的SAH的一点修改是,对于kd树值得稍微偏好选择
其中一个孩子没有与之重合的图元的划分,
因为光线穿过这些区域可以立即进行到下一个kd树节点而无需任何光线-图元相交测试。
因此,未划分和划分后区域的改进开销分别为
\begin{align*}
    t_{\text{isect}}N \quad \text{和} \quad t_{\text{trav}}+(1-b_{\mathrm{e}})(p_BN_Bt_{\text{isect}}+p_AN_At_{\text{isect}})\, ,
\end{align*}
其中$b_{\mathrm{e}}$是为零的“补贴”\sidenote{译者注:原文bonus。}值,
除非两个区域之一完全为空,此时取值0到1。

有了为开销模型计算概率的方法,唯一要解决的问题是
怎么生成候选划分位置以及怎么为每个候选者高效计算开销。
可以证明该模型最小开销能于在某一图元边界框的一个面上划分时取得——
不需要考虑在中间位置的划分(为了帮助你自己理解,
考虑一下开销函数在面的边界之间时的特性)。
这里,我们将考虑该区域内三个坐标轴之一或以上的所有边界框面。

利用精心构造的算法可以把检查所有这些候选者的开销维持在相对低的水平。
为了计算这些开销,我们将扫掠边界框在每个轴上的投影并追踪开销最低的那些(\reffig{4.15})。
每个边界框在每个轴上有两处边界,每处都用结构体\refvar{BoundEdge}{}的实例表示。
该结构体记录了边界沿轴的位置,它表示边界框的开始或结束
(沿轴从低到高),以及哪个图元与之关联。
\begin{figure}[htbp]
    \centering%LaTeX with PSTricks extensions
%%Creator: Inkscape 1.0.1 (3bc2e813f5, 2020-09-07)
%%Please note this file requires PSTricks extensions
\psset{xunit=.5pt,yunit=.5pt,runit=.5pt}
\begin{pspicture}(375.70999146,171.55999756)
{
\newrgbcolor{curcolor}{0 0 0}
\pscustom[linewidth=1,linecolor=curcolor]
{
\newpath
\moveto(0,20.94000244)
\lineto(352.42001343,20.94000244)
}
}
{
\newrgbcolor{curcolor}{0 0 0}
\pscustom[linestyle=none,fillstyle=solid,fillcolor=curcolor]
{
\newpath
\moveto(347.51,15.42999756)
\lineto(351.77,20.93999756)
\lineto(347.51,26.43999756)
\lineto(360.53,20.93999756)
\closepath
}
}
{
\newrgbcolor{curcolor}{0.65098041 0.65098041 0.65098041}
\pscustom[linestyle=none,fillstyle=solid,fillcolor=curcolor]
{
\newpath
\moveto(349.07,16.63999756)
\lineto(359.21,20.93999756)
\lineto(352.4,20.93999756)
\closepath
}
}
{
\newrgbcolor{curcolor}{0.40000001 0.40000001 0.40000001}
\pscustom[linestyle=none,fillstyle=solid,fillcolor=curcolor]
{
\newpath
\moveto(349.07,25.23999756)
\lineto(359.21,20.93999756)
\lineto(352.4,20.93999756)
\closepath
}
}
{
\newrgbcolor{curcolor}{0 0 0}
\pscustom[linewidth=1,linecolor=curcolor]
{
\newpath
\moveto(15.44999981,171.05999756)
\lineto(152.00000286,171.05999756)
\lineto(152.00000286,74.27999878)
\lineto(15.44999981,74.27999878)
\closepath
}
}
{
\newrgbcolor{curcolor}{0 0 0}
\pscustom[linewidth=1,linecolor=curcolor]
{
\newpath
\moveto(123.36000061,125.98999786)
\lineto(249.09000397,125.98999786)
\lineto(249.09000397,36.22000122)
\lineto(123.36000061,36.22000122)
\closepath
}
}
{
\newrgbcolor{curcolor}{0 0 0}
\pscustom[linewidth=1,linecolor=curcolor]
{
\newpath
\moveto(290.11999512,164.49999762)
\lineto(344.79999542,164.49999762)
\lineto(344.79999542,109.81999731)
\lineto(290.11999512,109.81999731)
\closepath
}
}
{
\newrgbcolor{curcolor}{0 0 0}
\pscustom[linewidth=1,linecolor=curcolor,linestyle=dashed,dash=2]
{
\newpath
\moveto(15.5,73.80999756)
\lineto(15.5,20.73999023)
}
}
{
\newrgbcolor{curcolor}{0 0 0}
\pscustom[linewidth=1,linecolor=curcolor,linestyle=dashed,dash=2]
{
\newpath
\moveto(123.09999847,35.86000061)
\lineto(123.09999847,21.25)
}
}
{
\newrgbcolor{curcolor}{0 0 0}
\pscustom[linewidth=1,linecolor=curcolor,linestyle=dashed,dash=2]
{
\newpath
\moveto(152.05000305,74.1499939)
\lineto(152.05000305,21.25)
}
}
{
\newrgbcolor{curcolor}{0 0 0}
\pscustom[linewidth=1,linecolor=curcolor,linestyle=dashed,dash=2]
{
\newpath
\moveto(248.83000183,36.02000427)
\lineto(248.83000183,21.41000366)
}
}
{
\newrgbcolor{curcolor}{0 0 0}
\pscustom[linewidth=1,linecolor=curcolor,linestyle=dashed,dash=2]
{
\newpath
\moveto(289.47000122,108.48999786)
\lineto(289.47000122,20.98999023)
}
}
{
\newrgbcolor{curcolor}{0 0 0}
\pscustom[linewidth=1,linecolor=curcolor,linestyle=dashed,dash=2]
{
\newpath
\moveto(344.1499939,108.56999588)
\lineto(344.1499939,21.08000183)
}
}
{
\newrgbcolor{curcolor}{0 0 0}
\pscustom[linestyle=none,fillstyle=solid,fillcolor=curcolor]
{
\newpath
\moveto(78.17413888,122.29249266)
\curveto(77.54913888,121.24249266)(76.92413888,121.01749266)(76.22413888,120.96749266)
\curveto(76.02413888,120.94249266)(75.87413888,120.94249266)(75.87413888,120.64249266)
\curveto(75.87413888,120.54249266)(75.97413888,120.46749266)(76.09913888,120.46749266)
\curveto(76.52413888,120.46749266)(77.02413888,120.51749266)(77.44913888,120.51749266)
\curveto(77.99913888,120.51749266)(78.54913888,120.46749266)(79.04913888,120.46749266)
\curveto(79.14913888,120.46749266)(79.34913888,120.46749266)(79.34913888,120.76749266)
\curveto(79.34913888,120.94249266)(79.22413888,120.96749266)(79.09913888,120.96749266)
\curveto(78.74913888,120.99249266)(78.34913888,121.11749266)(78.34913888,121.51749266)
\curveto(78.34913888,121.71749266)(78.44913888,121.89249266)(78.57413888,122.11749266)
\lineto(79.79913888,124.14249266)
\lineto(83.79913888,124.14249266)
\curveto(83.82413888,123.81749266)(84.04913888,121.64249266)(84.04913888,121.49249266)
\curveto(84.04913888,121.01749266)(83.22413888,120.96749266)(82.89913888,120.96749266)
\curveto(82.67413888,120.96749266)(82.52413888,120.96749266)(82.52413888,120.64249266)
\curveto(82.52413888,120.46749266)(82.69913888,120.46749266)(82.74913888,120.46749266)
\curveto(83.39913888,120.46749266)(84.07413888,120.51749266)(84.72413888,120.51749266)
\curveto(85.12413888,120.51749266)(86.14913888,120.46749266)(86.54913888,120.46749266)
\curveto(86.62413888,120.46749266)(86.82413888,120.46749266)(86.82413888,120.79249266)
\curveto(86.82413888,120.96749266)(86.67413888,120.96749266)(86.44913888,120.96749266)
\curveto(85.47413888,120.96749266)(85.47413888,121.06749266)(85.42413888,121.54249266)
\lineto(84.44913888,131.49249266)
\curveto(84.42413888,131.81749266)(84.42413888,131.89249266)(84.14913888,131.89249266)
\curveto(83.89913888,131.89249266)(83.82413888,131.76749266)(83.72413888,131.61749266)
\closepath
\moveto(80.09913888,124.64249266)
\lineto(83.22413888,129.91749266)
\lineto(83.74913888,124.64249266)
\closepath
\moveto(80.09913888,124.64249266)
}
}
{
\newrgbcolor{curcolor}{0 0 0}
\pscustom[linestyle=none,fillstyle=solid,fillcolor=curcolor]
{
\newpath
\moveto(184.43915488,75.26607566)
\curveto(184.28915488,74.64107566)(184.23915488,74.51607566)(182.98915488,74.51607566)
\curveto(182.71415488,74.51607566)(182.56415488,74.51607566)(182.56415488,74.19107566)
\curveto(182.56415488,74.01607566)(182.71415488,74.01607566)(182.98915488,74.01607566)
\lineto(188.68915488,74.01607566)
\curveto(191.21415488,74.01607566)(193.08915488,75.89107566)(193.08915488,77.46607566)
\curveto(193.08915488,78.61607566)(192.16415488,79.54107566)(190.61415488,79.71607566)
\curveto(192.26415488,80.01607566)(193.93915488,81.19107566)(193.93915488,82.71607566)
\curveto(193.93915488,83.89107566)(192.88915488,84.91607566)(190.98915488,84.91607566)
\lineto(185.61415488,84.91607566)
\curveto(185.31415488,84.91607566)(185.16415488,84.91607566)(185.16415488,84.59107566)
\curveto(185.16415488,84.41607566)(185.31415488,84.41607566)(185.61415488,84.41607566)
\curveto(185.63915488,84.41607566)(185.93915488,84.41607566)(186.21415488,84.39107566)
\curveto(186.48915488,84.34107566)(186.63915488,84.34107566)(186.63915488,84.11607566)
\curveto(186.63915488,84.06607566)(186.61415488,84.01607566)(186.58915488,83.81607566)
\closepath
\moveto(186.83915488,79.86607566)
\lineto(187.83915488,83.81607566)
\curveto(187.98915488,84.36607566)(188.01415488,84.41607566)(188.68915488,84.41607566)
\lineto(190.76415488,84.41607566)
\curveto(192.16415488,84.41607566)(192.48915488,83.46607566)(192.48915488,82.76607566)
\curveto(192.48915488,81.36607566)(191.11415488,79.86607566)(189.18915488,79.86607566)
\closepath
\moveto(186.13915488,74.51607566)
\lineto(185.78915488,74.51607566)
\curveto(185.61415488,74.54107566)(185.56415488,74.56607566)(185.56415488,74.69107566)
\curveto(185.56415488,74.74107566)(185.56415488,74.76607566)(185.66415488,75.04107566)
\lineto(186.76415488,79.49107566)
\lineto(189.76415488,79.49107566)
\curveto(191.28915488,79.49107566)(191.61415488,78.31607566)(191.61415488,77.64107566)
\curveto(191.61415488,76.06607566)(190.18915488,74.51607566)(188.28915488,74.51607566)
\closepath
\moveto(186.13915488,74.51607566)
}
}
{
\newrgbcolor{curcolor}{0 0 0}
\pscustom[linestyle=none,fillstyle=solid,fillcolor=curcolor]
{
\newpath
\moveto(324.14643488,141.72082066)
\curveto(324.14643488,141.77082066)(324.12143488,141.89582066)(323.97143488,141.89582066)
\curveto(323.92143488,141.89582066)(323.89643488,141.87082066)(323.72143488,141.69582066)
\lineto(322.62143488,140.47082066)
\curveto(322.47143488,140.69582066)(321.74643488,141.89582066)(319.97143488,141.89582066)
\curveto(316.39643488,141.89582066)(312.82143488,138.37082066)(312.82143488,134.67082066)
\curveto(312.82143488,132.04582066)(314.69643488,130.29582066)(317.14643488,130.29582066)
\curveto(318.52143488,130.29582066)(319.74643488,130.92082066)(320.59643488,131.67082066)
\curveto(322.07143488,132.97082066)(322.34643488,134.42082066)(322.34643488,134.47082066)
\curveto(322.34643488,134.64582066)(322.17143488,134.64582066)(322.14643488,134.64582066)
\curveto(322.04643488,134.64582066)(321.97143488,134.59582066)(321.94643488,134.47082066)
\curveto(321.79643488,134.02082066)(321.42143488,132.87082066)(320.32143488,131.94582066)
\curveto(319.22143488,131.07082066)(318.22143488,130.79582066)(317.39643488,130.79582066)
\curveto(315.97143488,130.79582066)(314.27143488,131.62082066)(314.27143488,134.09582066)
\curveto(314.27143488,135.02082066)(314.59643488,137.59582066)(316.19643488,139.47082066)
\curveto(317.17143488,140.59582066)(318.67143488,141.39582066)(320.09643488,141.39582066)
\curveto(321.72143488,141.39582066)(322.67143488,140.17082066)(322.67143488,138.32082066)
\curveto(322.67143488,137.67082066)(322.62143488,137.67082066)(322.62143488,137.49582066)
\curveto(322.62143488,137.34582066)(322.79643488,137.34582066)(322.84643488,137.34582066)
\curveto(323.04643488,137.34582066)(323.04643488,137.37082066)(323.14643488,137.67082066)
\closepath
\moveto(324.14643488,141.72082066)
}
}
{
\newrgbcolor{curcolor}{0 0 0}
\pscustom[linestyle=none,fillstyle=solid,fillcolor=curcolor]
{
\newpath
\moveto(367.92731488,22.18139966)
\curveto(368.02731488,22.58139966)(368.40231488,24.05639966)(369.50231488,24.05639966)
\curveto(369.57731488,24.05639966)(369.97731488,24.05639966)(370.30231488,23.85639966)
\curveto(369.85231488,23.75639966)(369.55231488,23.38139966)(369.55231488,22.98139966)
\curveto(369.55231488,22.73139966)(369.72731488,22.43139966)(370.15231488,22.43139966)
\curveto(370.50231488,22.43139966)(371.00231488,22.70639966)(371.00231488,23.35639966)
\curveto(371.00231488,24.18139966)(370.07731488,24.40639966)(369.52731488,24.40639966)
\curveto(368.60231488,24.40639966)(368.05231488,23.55639966)(367.85231488,23.20639966)
\curveto(367.45231488,24.25639966)(366.60231488,24.40639966)(366.12731488,24.40639966)
\curveto(364.47731488,24.40639966)(363.55231488,22.35639966)(363.55231488,21.95639966)
\curveto(363.55231488,21.78139966)(363.72731488,21.78139966)(363.75231488,21.78139966)
\curveto(363.87731488,21.78139966)(363.92731488,21.83139966)(363.95231488,21.95639966)
\curveto(364.50231488,23.65639966)(365.55231488,24.05639966)(366.10231488,24.05639966)
\curveto(366.40231488,24.05639966)(366.95231488,23.90639966)(366.95231488,22.98139966)
\curveto(366.95231488,22.48139966)(366.67731488,21.43139966)(366.10231488,19.18139966)
\curveto(365.85231488,18.20639966)(365.27731488,17.53139966)(364.57731488,17.53139966)
\curveto(364.47731488,17.53139966)(364.12731488,17.53139966)(363.77731488,17.73139966)
\curveto(364.17731488,17.83139966)(364.52731488,18.15639966)(364.52731488,18.60639966)
\curveto(364.52731488,19.03139966)(364.17731488,19.15639966)(363.95231488,19.15639966)
\curveto(363.45231488,19.15639966)(363.07731488,18.75639966)(363.07731488,18.23139966)
\curveto(363.07731488,17.50639966)(363.85231488,17.18139966)(364.55231488,17.18139966)
\curveto(365.62731488,17.18139966)(366.20231488,18.30639966)(366.22731488,18.38139966)
\curveto(366.42731488,17.80639966)(367.00231488,17.18139966)(367.95231488,17.18139966)
\curveto(369.60231488,17.18139966)(370.50231488,19.23139966)(370.50231488,19.63139966)
\curveto(370.50231488,19.80639966)(370.37731488,19.80639966)(370.32731488,19.80639966)
\curveto(370.17731488,19.80639966)(370.15231488,19.73139966)(370.10231488,19.63139966)
\curveto(369.57731488,17.90639966)(368.50231488,17.53139966)(368.00231488,17.53139966)
\curveto(367.37731488,17.53139966)(367.12731488,18.03139966)(367.12731488,18.58139966)
\curveto(367.12731488,18.93139966)(367.20231488,19.28139966)(367.37731488,19.98139966)
\closepath
\moveto(367.92731488,22.18139966)
}
}
{
\newrgbcolor{curcolor}{0 0 0}
\pscustom[linestyle=none,fillstyle=solid,fillcolor=curcolor]
{
\newpath
\moveto(15.31661488,13.54462966)
\curveto(15.01661488,14.14462966)(14.56661488,14.56962966)(13.84161488,14.56962966)
\curveto(11.99161488,14.56962966)(10.01661488,12.21962966)(10.01661488,9.89462966)
\curveto(10.01661488,8.39462966)(10.89161488,7.34462966)(12.11661488,7.34462966)
\curveto(12.44161488,7.34462966)(13.24161488,7.41962966)(14.19161488,8.54462966)
\curveto(14.31661488,7.86962966)(14.89161488,7.34462966)(15.64161488,7.34462966)
\curveto(16.21661488,7.34462966)(16.56661488,7.71962966)(16.84161488,8.21962966)
\curveto(17.09161488,8.79462966)(17.31661488,9.76962966)(17.31661488,9.79462966)
\curveto(17.31661488,9.96962966)(17.16661488,9.96962966)(17.11661488,9.96962966)
\curveto(16.96661488,9.96962966)(16.94161488,9.89462966)(16.89161488,9.66962966)
\curveto(16.61661488,8.64462966)(16.34161488,7.69462966)(15.69161488,7.69462966)
\curveto(15.24161488,7.69462966)(15.21661488,8.11962966)(15.21661488,8.41962966)
\curveto(15.21661488,8.76962966)(15.24161488,8.91962966)(15.41661488,9.61962966)
\curveto(15.59161488,10.26962966)(15.61661488,10.44462966)(15.76661488,11.04462966)
\lineto(16.34161488,13.26962966)
\curveto(16.44161488,13.71962966)(16.44161488,13.74462966)(16.44161488,13.81962966)
\curveto(16.44161488,14.09462966)(16.26661488,14.24462966)(15.99161488,14.24462966)
\curveto(15.59161488,14.24462966)(15.36661488,13.89462966)(15.31661488,13.54462966)
\closepath
\moveto(14.29161488,9.41962966)
\curveto(14.19161488,9.11962966)(14.19161488,9.09462966)(13.96661488,8.81962966)
\curveto(13.26661488,7.94462966)(12.61661488,7.69462966)(12.16661488,7.69462966)
\curveto(11.36661488,7.69462966)(11.14161488,8.56962966)(11.14161488,9.19462966)
\curveto(11.14161488,9.99462966)(11.64161488,11.94462966)(12.01661488,12.69462966)
\curveto(12.51661488,13.61962966)(13.21661488,14.21962966)(13.86661488,14.21962966)
\curveto(14.89161488,14.21962966)(15.11661488,12.91962966)(15.11661488,12.81962966)
\curveto(15.11661488,12.71962966)(15.09161488,12.61962966)(15.06661488,12.54462966)
\closepath
\moveto(14.29161488,9.41962966)
}
}
{
\newrgbcolor{curcolor}{0 0 0}
\pscustom[linestyle=none,fillstyle=solid,fillcolor=curcolor]
{
\newpath
\moveto(23.54221303,8.67922439)
\curveto(23.54221303,9.90422439)(23.39221303,10.80422439)(22.89221303,11.57922439)
\curveto(22.54221303,12.07922439)(21.84221303,12.52922439)(20.96721303,12.52922439)
\curveto(18.36721303,12.52922439)(18.36721303,9.47922439)(18.36721303,8.67922439)
\curveto(18.36721303,7.87922439)(18.36721303,4.90422439)(20.96721303,4.90422439)
\curveto(23.54221303,4.90422439)(23.54221303,7.87922439)(23.54221303,8.67922439)
\closepath
\moveto(20.96721303,5.22922439)
\curveto(20.44221303,5.22922439)(19.76721303,5.52922439)(19.54221303,6.42922439)
\curveto(19.39221303,7.07922439)(19.39221303,8.00422439)(19.39221303,8.82922439)
\curveto(19.39221303,9.65422439)(19.39221303,10.50422439)(19.54221303,11.10422439)
\curveto(19.79221303,11.97922439)(20.49221303,12.22922439)(20.96721303,12.22922439)
\curveto(21.56721303,12.22922439)(22.14221303,11.85422439)(22.34221303,11.20422439)
\curveto(22.51721303,10.60422439)(22.54221303,9.80422439)(22.54221303,8.82922439)
\curveto(22.54221303,8.00422439)(22.54221303,7.17922439)(22.39221303,6.47922439)
\curveto(22.16721303,5.45422439)(21.41721303,5.22922439)(20.96721303,5.22922439)
\closepath
\moveto(20.96721303,5.22922439)
}
}
{
\newrgbcolor{curcolor}{0 0 0}
\pscustom[linestyle=none,fillstyle=solid,fillcolor=curcolor]
{
\newpath
\moveto(152.27998299,12.51996566)
\curveto(151.97998299,13.11996566)(151.52998299,13.54496566)(150.80498299,13.54496566)
\curveto(148.95498299,13.54496566)(146.97998299,11.19496566)(146.97998299,8.86996566)
\curveto(146.97998299,7.36996566)(147.85498299,6.31996566)(149.07998299,6.31996566)
\curveto(149.40498299,6.31996566)(150.20498299,6.39496566)(151.15498299,7.51996566)
\curveto(151.27998299,6.84496566)(151.85498299,6.31996566)(152.60498299,6.31996566)
\curveto(153.17998299,6.31996566)(153.52998299,6.69496566)(153.80498299,7.19496566)
\curveto(154.05498299,7.76996566)(154.27998299,8.74496566)(154.27998299,8.76996566)
\curveto(154.27998299,8.94496566)(154.12998299,8.94496566)(154.07998299,8.94496566)
\curveto(153.92998299,8.94496566)(153.90498299,8.86996566)(153.85498299,8.64496566)
\curveto(153.57998299,7.61996566)(153.30498299,6.66996566)(152.65498299,6.66996566)
\curveto(152.20498299,6.66996566)(152.17998299,7.09496566)(152.17998299,7.39496566)
\curveto(152.17998299,7.74496566)(152.20498299,7.89496566)(152.37998299,8.59496566)
\curveto(152.55498299,9.24496566)(152.57998299,9.41996566)(152.72998299,10.01996566)
\lineto(153.30498299,12.24496566)
\curveto(153.40498299,12.69496566)(153.40498299,12.71996566)(153.40498299,12.79496566)
\curveto(153.40498299,13.06996566)(153.22998299,13.21996566)(152.95498299,13.21996566)
\curveto(152.55498299,13.21996566)(152.32998299,12.86996566)(152.27998299,12.51996566)
\closepath
\moveto(151.25498299,8.39496566)
\curveto(151.15498299,8.09496566)(151.15498299,8.06996566)(150.92998299,7.79496566)
\curveto(150.22998299,6.91996566)(149.57998299,6.66996566)(149.12998299,6.66996566)
\curveto(148.32998299,6.66996566)(148.10498299,7.54496566)(148.10498299,8.16996566)
\curveto(148.10498299,8.96996566)(148.60498299,10.91996566)(148.97998299,11.66996566)
\curveto(149.47998299,12.59496566)(150.17998299,13.19496566)(150.82998299,13.19496566)
\curveto(151.85498299,13.19496566)(152.07998299,11.89496566)(152.07998299,11.79496566)
\curveto(152.07998299,11.69496566)(152.05498299,11.59496566)(152.02998299,11.51996566)
\closepath
\moveto(151.25498299,8.39496566)
}
}
{
\newrgbcolor{curcolor}{0 0 0}
\pscustom[linestyle=none,fillstyle=solid,fillcolor=curcolor]
{
\newpath
\moveto(158.48058114,11.20456039)
\curveto(158.48058114,11.50456039)(158.48058114,11.50456039)(158.15558114,11.50456039)
\curveto(157.43058114,10.80456039)(156.43058114,10.80456039)(155.98058114,10.80456039)
\lineto(155.98058114,10.40456039)
\curveto(156.23058114,10.40456039)(156.98058114,10.40456039)(157.58058114,10.70456039)
\lineto(157.58058114,5.02956039)
\curveto(157.58058114,4.65456039)(157.58058114,4.50456039)(156.48058114,4.50456039)
\lineto(156.05558114,4.50456039)
\lineto(156.05558114,4.10456039)
\curveto(156.25558114,4.10456039)(157.63058114,4.15456039)(158.03058114,4.15456039)
\curveto(158.38058114,4.15456039)(159.78058114,4.10456039)(160.03058114,4.10456039)
\lineto(160.03058114,4.50456039)
\lineto(159.60558114,4.50456039)
\curveto(158.48058114,4.50456039)(158.48058114,4.65456039)(158.48058114,5.02956039)
\closepath
\moveto(158.48058114,11.20456039)
}
}
{
\newrgbcolor{curcolor}{0 0 0}
\pscustom[linestyle=none,fillstyle=solid,fillcolor=curcolor]
{
\newpath
\moveto(248.83924488,16.37030166)
\curveto(248.83924488,16.37030166)(248.83924488,16.54530166)(248.63924488,16.54530166)
\curveto(248.28924488,16.54530166)(247.11424488,16.42030166)(246.68924488,16.37030166)
\curveto(246.56424488,16.37030166)(246.38924488,16.34530166)(246.38924488,16.07030166)
\curveto(246.38924488,15.87030166)(246.53924488,15.87030166)(246.78924488,15.87030166)
\curveto(247.53924488,15.87030166)(247.56424488,15.77030166)(247.56424488,15.59530166)
\curveto(247.56424488,15.49530166)(247.43924488,14.94530166)(247.36424488,14.62030166)
\lineto(246.03924488,9.42030166)
\curveto(245.86424488,8.62030166)(245.78924488,8.34530166)(245.78924488,7.79530166)
\curveto(245.78924488,6.29530166)(246.63924488,5.29530166)(247.81424488,5.29530166)
\curveto(249.68924488,5.29530166)(251.66424488,7.67030166)(251.66424488,9.97030166)
\curveto(251.66424488,11.42030166)(250.81424488,12.52030166)(249.53924488,12.52030166)
\curveto(248.81424488,12.52030166)(248.13924488,12.04530166)(247.66424488,11.57030166)
\closepath
\moveto(247.36424488,10.34530166)
\curveto(247.43924488,10.69530166)(247.43924488,10.72030166)(247.58924488,10.89530166)
\curveto(248.36424488,11.92030166)(249.08924488,12.17030166)(249.51424488,12.17030166)
\curveto(250.08924488,12.17030166)(250.51424488,11.69530166)(250.51424488,10.67030166)
\curveto(250.51424488,9.72030166)(249.98924488,7.89530166)(249.68924488,7.29530166)
\curveto(249.16424488,6.22030166)(248.43924488,5.64530166)(247.81424488,5.64530166)
\curveto(247.26424488,5.64530166)(246.73924488,6.07030166)(246.73924488,7.24530166)
\curveto(246.73924488,7.57030166)(246.73924488,7.87030166)(246.98924488,8.87030166)
\closepath
\moveto(247.36424488,10.34530166)
}
}
{
\newrgbcolor{curcolor}{0 0 0}
\pscustom[linestyle=none,fillstyle=solid,fillcolor=curcolor]
{
\newpath
\moveto(255.60583424,10.17989639)
\curveto(255.60583424,10.47989639)(255.60583424,10.47989639)(255.28083424,10.47989639)
\curveto(254.55583424,9.77989639)(253.55583424,9.77989639)(253.10583424,9.77989639)
\lineto(253.10583424,9.37989639)
\curveto(253.35583424,9.37989639)(254.10583424,9.37989639)(254.70583424,9.67989639)
\lineto(254.70583424,4.00489639)
\curveto(254.70583424,3.62989639)(254.70583424,3.47989639)(253.60583424,3.47989639)
\lineto(253.18083424,3.47989639)
\lineto(253.18083424,3.07989639)
\curveto(253.38083424,3.07989639)(254.75583424,3.12989639)(255.15583424,3.12989639)
\curveto(255.50583424,3.12989639)(256.90583424,3.07989639)(257.15583424,3.07989639)
\lineto(257.15583424,3.47989639)
\lineto(256.73083424,3.47989639)
\curveto(255.60583424,3.47989639)(255.60583424,3.62989639)(255.60583424,4.00489639)
\closepath
\moveto(255.60583424,10.17989639)
}
}
{
\newrgbcolor{curcolor}{0 0 0}
\pscustom[linestyle=none,fillstyle=solid,fillcolor=curcolor]
{
\newpath
\moveto(123.14717488,16.02874766)
\curveto(123.14717488,16.02874766)(123.14717488,16.20374766)(122.94717488,16.20374766)
\curveto(122.59717488,16.20374766)(121.42217488,16.07874766)(120.99717488,16.02874766)
\curveto(120.87217488,16.02874766)(120.69717488,16.00374766)(120.69717488,15.72874766)
\curveto(120.69717488,15.52874766)(120.84717488,15.52874766)(121.09717488,15.52874766)
\curveto(121.84717488,15.52874766)(121.87217488,15.42874766)(121.87217488,15.25374766)
\curveto(121.87217488,15.15374766)(121.74717488,14.60374766)(121.67217488,14.27874766)
\lineto(120.34717488,9.07874766)
\curveto(120.17217488,8.27874766)(120.09717488,8.00374766)(120.09717488,7.45374766)
\curveto(120.09717488,5.95374766)(120.94717488,4.95374766)(122.12217488,4.95374766)
\curveto(123.99717488,4.95374766)(125.97217488,7.32874766)(125.97217488,9.62874766)
\curveto(125.97217488,11.07874766)(125.12217488,12.17874766)(123.84717488,12.17874766)
\curveto(123.12217488,12.17874766)(122.44717488,11.70374766)(121.97217488,11.22874766)
\closepath
\moveto(121.67217488,10.00374766)
\curveto(121.74717488,10.35374766)(121.74717488,10.37874766)(121.89717488,10.55374766)
\curveto(122.67217488,11.57874766)(123.39717488,11.82874766)(123.82217488,11.82874766)
\curveto(124.39717488,11.82874766)(124.82217488,11.35374766)(124.82217488,10.32874766)
\curveto(124.82217488,9.37874766)(124.29717488,7.55374766)(123.99717488,6.95374766)
\curveto(123.47217488,5.87874766)(122.74717488,5.30374766)(122.12217488,5.30374766)
\curveto(121.57217488,5.30374766)(121.04717488,5.72874766)(121.04717488,6.90374766)
\curveto(121.04717488,7.22874766)(121.04717488,7.52874766)(121.29717488,8.52874766)
\closepath
\moveto(121.67217488,10.00374766)
}
}
{
\newrgbcolor{curcolor}{0 0 0}
\pscustom[linestyle=none,fillstyle=solid,fillcolor=curcolor]
{
\newpath
\moveto(131.93876424,6.28834239)
\curveto(131.93876424,7.51334239)(131.78876424,8.41334239)(131.28876424,9.18834239)
\curveto(130.93876424,9.68834239)(130.23876424,10.13834239)(129.36376424,10.13834239)
\curveto(126.76376424,10.13834239)(126.76376424,7.08834239)(126.76376424,6.28834239)
\curveto(126.76376424,5.48834239)(126.76376424,2.51334239)(129.36376424,2.51334239)
\curveto(131.93876424,2.51334239)(131.93876424,5.48834239)(131.93876424,6.28834239)
\closepath
\moveto(129.36376424,2.83834239)
\curveto(128.83876424,2.83834239)(128.16376424,3.13834239)(127.93876424,4.03834239)
\curveto(127.78876424,4.68834239)(127.78876424,5.61334239)(127.78876424,6.43834239)
\curveto(127.78876424,7.26334239)(127.78876424,8.11334239)(127.93876424,8.71334239)
\curveto(128.18876424,9.58834239)(128.88876424,9.83834239)(129.36376424,9.83834239)
\curveto(129.96376424,9.83834239)(130.53876424,9.46334239)(130.73876424,8.81334239)
\curveto(130.91376424,8.21334239)(130.93876424,7.41334239)(130.93876424,6.43834239)
\curveto(130.93876424,5.61334239)(130.93876424,4.78834239)(130.78876424,4.08834239)
\curveto(130.56376424,3.06334239)(129.81376424,2.83834239)(129.36376424,2.83834239)
\closepath
\moveto(129.36376424,2.83834239)
}
}
{
\newrgbcolor{curcolor}{0 0 0}
\pscustom[linestyle=none,fillstyle=solid,fillcolor=curcolor]
{
\newpath
\moveto(291.32612488,12.88651966)
\curveto(291.05112488,12.88651966)(290.85112488,12.88651966)(290.62612488,12.68651966)
\curveto(290.35112488,12.43651966)(290.32612488,12.16151966)(290.32612488,12.06151966)
\curveto(290.32612488,11.66151966)(290.62612488,11.48651966)(290.92612488,11.48651966)
\curveto(291.37612488,11.48651966)(291.80112488,11.88651966)(291.80112488,12.51151966)
\curveto(291.80112488,13.28651966)(291.05112488,13.88651966)(289.92612488,13.88651966)
\curveto(287.77612488,13.88651966)(285.65112488,11.61151966)(285.65112488,9.36151966)
\curveto(285.65112488,7.91151966)(286.57612488,6.66151966)(288.25112488,6.66151966)
\curveto(290.52612488,6.66151966)(291.85112488,8.36151966)(291.85112488,8.53651966)
\curveto(291.85112488,8.63651966)(291.77612488,8.76151966)(291.67612488,8.76151966)
\curveto(291.57612488,8.76151966)(291.55112488,8.71151966)(291.45112488,8.58651966)
\curveto(290.20112488,7.01151966)(288.45112488,7.01151966)(288.27612488,7.01151966)
\curveto(287.27612488,7.01151966)(286.82612488,7.78651966)(286.82612488,8.76151966)
\curveto(286.82612488,9.41151966)(287.15112488,10.96151966)(287.70112488,11.93651966)
\curveto(288.20112488,12.86151966)(289.07612488,13.53651966)(289.95112488,13.53651966)
\curveto(290.47612488,13.53651966)(291.10112488,13.33651966)(291.32612488,12.88651966)
\closepath
\moveto(291.32612488,12.88651966)
}
}
{
\newrgbcolor{curcolor}{0 0 0}
\pscustom[linestyle=none,fillstyle=solid,fillcolor=curcolor]
{
\newpath
\moveto(297.65031922,7.99611439)
\curveto(297.65031922,9.22111439)(297.50031922,10.12111439)(297.00031922,10.89611439)
\curveto(296.65031922,11.39611439)(295.95031922,11.84611439)(295.07531922,11.84611439)
\curveto(292.47531922,11.84611439)(292.47531922,8.79611439)(292.47531922,7.99611439)
\curveto(292.47531922,7.19611439)(292.47531922,4.22111439)(295.07531922,4.22111439)
\curveto(297.65031922,4.22111439)(297.65031922,7.19611439)(297.65031922,7.99611439)
\closepath
\moveto(295.07531922,4.54611439)
\curveto(294.55031922,4.54611439)(293.87531922,4.84611439)(293.65031922,5.74611439)
\curveto(293.50031922,6.39611439)(293.50031922,7.32111439)(293.50031922,8.14611439)
\curveto(293.50031922,8.97111439)(293.50031922,9.82111439)(293.65031922,10.42111439)
\curveto(293.90031922,11.29611439)(294.60031922,11.54611439)(295.07531922,11.54611439)
\curveto(295.67531922,11.54611439)(296.25031922,11.17111439)(296.45031922,10.52111439)
\curveto(296.62531922,9.92111439)(296.65031922,9.12111439)(296.65031922,8.14611439)
\curveto(296.65031922,7.32111439)(296.65031922,6.49611439)(296.50031922,5.79611439)
\curveto(296.27531922,4.77111439)(295.52531922,4.54611439)(295.07531922,4.54611439)
\closepath
\moveto(295.07531922,4.54611439)
}
}
{
\newrgbcolor{curcolor}{0 0 0}
\pscustom[linestyle=none,fillstyle=solid,fillcolor=curcolor]
{
\newpath
\moveto(344.60863488,12.34656766)
\curveto(344.33363488,12.34656766)(344.13363488,12.34656766)(343.90863488,12.14656766)
\curveto(343.63363488,11.89656766)(343.60863488,11.62156766)(343.60863488,11.52156766)
\curveto(343.60863488,11.12156766)(343.90863488,10.94656766)(344.20863488,10.94656766)
\curveto(344.65863488,10.94656766)(345.08363488,11.34656766)(345.08363488,11.97156766)
\curveto(345.08363488,12.74656766)(344.33363488,13.34656766)(343.20863488,13.34656766)
\curveto(341.05863488,13.34656766)(338.93363488,11.07156766)(338.93363488,8.82156766)
\curveto(338.93363488,7.37156766)(339.85863488,6.12156766)(341.53363488,6.12156766)
\curveto(343.80863488,6.12156766)(345.13363488,7.82156766)(345.13363488,7.99656766)
\curveto(345.13363488,8.09656766)(345.05863488,8.22156766)(344.95863488,8.22156766)
\curveto(344.85863488,8.22156766)(344.83363488,8.17156766)(344.73363488,8.04656766)
\curveto(343.48363488,6.47156766)(341.73363488,6.47156766)(341.55863488,6.47156766)
\curveto(340.55863488,6.47156766)(340.10863488,7.24656766)(340.10863488,8.22156766)
\curveto(340.10863488,8.87156766)(340.43363488,10.42156766)(340.98363488,11.39656766)
\curveto(341.48363488,12.32156766)(342.35863488,12.99656766)(343.23363488,12.99656766)
\curveto(343.75863488,12.99656766)(344.38363488,12.79656766)(344.60863488,12.34656766)
\closepath
\moveto(344.60863488,12.34656766)
}
}
{
\newrgbcolor{curcolor}{0 0 0}
\pscustom[linestyle=none,fillstyle=solid,fillcolor=curcolor]
{
\newpath
\moveto(348.90782922,11.00616239)
\curveto(348.90782922,11.30616239)(348.90782922,11.30616239)(348.58282922,11.30616239)
\curveto(347.85782922,10.60616239)(346.85782922,10.60616239)(346.40782922,10.60616239)
\lineto(346.40782922,10.20616239)
\curveto(346.65782922,10.20616239)(347.40782922,10.20616239)(348.00782922,10.50616239)
\lineto(348.00782922,4.83116239)
\curveto(348.00782922,4.45616239)(348.00782922,4.30616239)(346.90782922,4.30616239)
\lineto(346.48282922,4.30616239)
\lineto(346.48282922,3.90616239)
\curveto(346.68282922,3.90616239)(348.05782922,3.95616239)(348.45782922,3.95616239)
\curveto(348.80782922,3.95616239)(350.20782922,3.90616239)(350.45782922,3.90616239)
\lineto(350.45782922,4.30616239)
\lineto(350.03282922,4.30616239)
\curveto(348.90782922,4.30616239)(348.90782922,4.45616239)(348.90782922,4.83116239)
\closepath
\moveto(348.90782922,11.00616239)
}
}
\end{pspicture}

    \caption{给定我们要考虑的可能划分所沿的轴,图元的边界框被投影到该轴上,
    这带来了一个高效算法以追踪特定划分平面两侧会各有多少图元。
    例如这里,在$a_1$处划分会让$A$完全留在划分平面下方,$B$横跨之,而$C$完全在其上方。
    轴上每一个点$a_0,a_1,b_0,b_1,c_0$和$c_1$都由结构体\refvar{BoundEdge}{}的一个实例表示。}
    \label{fig:4.15}
\end{figure}
\begin{lstlisting}
`\refcode{KdTreeAccel Local Declarations}{+=}\lastnext{KdTreeAccelLocalDeclarations}`
enum class `\initvar{EdgeType}{}` { `\initvar[EdgeType::Start]{Start}{}`, `\initvar[EdgeType::End]{End}{}` };
\end{lstlisting}
\begin{lstlisting}
`\refcode{KdTreeAccel Local Declarations}{+=}\lastcode{KdTreeAccelLocalDeclarations}`
struct `\initvar{BoundEdge}{}` {
    `\refcode{BoundEdge Public Methods}{}`
    `\refvar{Float}{}` `\initvar[BoundEdge::t]{t}{}`;
    int `\initvar[BoundEdge::primNum]{primNum}{}`;
    `\refvar{EdgeType}{}` `\initvar[BoundEdge::type]{type}{}`;
};
\end{lstlisting}
\begin{lstlisting}
`\initcode{BoundEdge Public Methods}{=}`
`\refvar{BoundEdge}{}`(`\refvar{Float}{}` t, int primNum, bool starting)
    : `\refvar[BoundEdge::t]{t}{}`(t), `\refvar[BoundEdge::primNum]{primNum}{}`(primNum) {
    `\refvar[BoundEdge::type]{type}{}` = starting ? `\refvar{EdgeType::Start}{}` : `\refvar{EdgeType::End}{}`; 
}
\end{lstlisting}

对于任意树节点至多需要为{\ttfamily 2*\refvar[KdTreeAccel::primitives]{primitives}{}.size()}个\refvar{BoundEdge}{}计算开销,
所以一次分配全部三轴上所有边界的内存然后再为每个创建的节点复用。
\begin{lstlisting}
`\initcode{Allocate working memory for kd-tree construction}{=}`
std::unique_ptr<`\refvar{BoundEdge}{}`[]> edges[3];
for (int i = 0; i < 3; ++i)
    edges[i].reset(new `\refvar{BoundEdge}{}`[2 * `\refvar[KdTreeAccel::primitives]{primitives}{}`.size()]);
\end{lstlisting}

在为创建的叶子确定估计的开销后,\refvar{KdTreeAccel::buildTree}{()}选择
一个轴尝试沿其划分并为每个候选划分计算开销函数。
{\ttfamily bestAxis}和{\ttfamily bestOffset}记录了该轴
以及目前给出最低开销{\ttfamily bestCost}的边界框边界索引。
{\ttfamily invTotalSA}初始化为节点表面积的倒数;
当计算光线穿过每个候选孩子节点的概率时会用到它的值。
\begin{lstlisting}
`\initcode{Choose split axis position for interior node}{=}`
int bestAxis = -1, bestOffset = -1;
`\refvar{Float}{}` bestCost = `\refvar{Infinity}{}`;
`\refvar{Float}{}` oldCost = `\refvar{isectCost}{}` * `\refvar{Float}{}`(nPrimitives);
`\refvar{Float}{}` totalSA = nodeBounds.`\refvar{SurfaceArea}{}`();
`\refvar{Float}{}` invTotalSA = 1 / totalSA;
`\refvar{Vector3f}{}` d = nodeBounds.`\refvar{pMax}{}` - nodeBounds.`\refvar{pMin}{}`;
`\refcode{Choose which axis to split along}{}`
int retries = 0;
retrySplit:
`\refcode{Initialize edges for axis}{}`
`\refcode{Compute cost of all splits for axis to find best}{}`
\end{lstlisting}