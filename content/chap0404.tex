\section{kd树加速器}\label{sec:kd树加速器}

\keyindex{二叉空间划分}{binary space partitioning}{}(BSP)树用平面自适应地细分空间。
一个BSP树从包含整个场景的边界框开始。
如果框内图元的数量大于某个阈值,则用平面将该框分为两半。
然后图元和与之重合的任意一半关联,
同时位于两半里的图元就都与它们关联
(相比之下,BVH中划分后每个图元只能分配到两个组中的一个)。

划分过程递归进行,直到结果树中的每个叶子区域都包含足够少的图元或者达到最大深度。
因为划分平面可以放置于整个框内的任意位置,
且3D空间的不同部分可以精确到不同程度,
所以BSP易于处理分布不均的几何体。

BSP树的两个变种是\keyindex{kd树}{kd-tree}{tree树}和\keyindex{八叉树}{octree}{tree树}。
kd树\sidenote{译者注:“kd”是k个维度的缩写。}简单地限制划分平面垂直于一个坐标轴;
这让树的遍历和构建都更高效,而在如何划分空间上牺牲一些灵活性。
八叉树每一步用三个垂直于轴的平面同时将该框分为八个区域
(通常在每个方向沿范围中心划分)。

本节中,我们将在类\refvar{KdTreeAccel}{}中为光线相交加速实现一个kd树。
该类源码可在文件\href{https://github.com/mmp/pbrt-v3/tree/master/src/accelerators/kdtreeaccel.h}{\ttfamily accelerators/kdtreeaccel.h}
和\href{https://github.com/mmp/pbrt-v3/tree/master/src/accelerators/kdtreeaccel.cpp}{\ttfamily accelerators/kdtreeaccel.cpp}
中找到。

\begin{lstlisting}
`\initcode{KdTreeAccel Declarations}{=}\initnext{KdTreeAccelDeclarations}`
class `\initvar{KdTreeAccel}{}` : public `\refvar{Aggregate}{}` {
public:
    `\refcode{KdTreeAccel Public Methods}{}`
private:
    `\refcode{KdTreeAccel Private Methods}{}`
    `\refcode{KdTreeAccel Private Data}{}`
};
\end{lstlisting}

除了要保存的图元外,\refvar{KdTreeAccel}{}构造函数
还接收一些参数用于在构建树时指导要作出的决定;
这些参数存于成员变量中(\refvar{isectCost}{}、\refvar{traversalCost}{}、
\refvar{maxPrims}{}、{\ttfamily maxDepth}和\refvar{emptyBonus}{})留待后用。
见\reffig{4.14}中构建树的图示。
\begin{figure}[htbp]
    \centering%LaTeX with PSTricks extensions
%%Creator: Inkscape 1.0.1 (3bc2e813f5, 2020-09-07)
%%Please note this file requires PSTricks extensions
\psset{xunit=.5pt,yunit=.5pt,runit=.5pt}
\begin{pspicture}(300.01998901,255.22999573)
{
\newrgbcolor{curcolor}{0 0 0}
\pscustom[linewidth=1,linecolor=curcolor]
{
\newpath
\moveto(162.36999512,109.19000244)
\lineto(271.04999542,109.19000244)
\lineto(271.04999542,0.51000214)
\lineto(162.36999512,0.51000214)
\closepath
}
}
{
\newrgbcolor{curcolor}{0 0 0}
\pscustom[linewidth=1,linecolor=curcolor]
{
\newpath
\moveto(189.00999546,89.62998962)
\curveto(189.00999546,99.98302387)(180.09128852,105.16596629)(174.87544919,97.84615295)
\curveto(169.65960985,90.52633961)(173.35279076,78.00998974)(180.72999573,78.00998974)
\curveto(188.10720069,78.00998974)(191.80038161,90.52633961)(186.58454227,97.84615295)
\curveto(181.36870293,105.16596629)(172.44999599,99.98302387)(172.44999599,89.62998962)
\curveto(172.44999599,79.27695537)(181.36870293,74.09401296)(186.58454227,81.4138263)
\curveto(191.80038161,88.73363964)(188.10720069,101.24998951)(180.72999573,101.24998951)
\curveto(173.35279076,101.24998951)(169.65960985,88.73363964)(174.87544919,81.4138263)
\curveto(180.09128852,74.09401296)(189.00999546,79.27695537)(189.00999546,89.62998962)
\closepath
}
}
{
\newrgbcolor{curcolor}{0 0 0}
\pscustom[linewidth=1,linecolor=curcolor]
{
\newpath
\moveto(189.80999756,68.94000244)
\lineto(207.16999817,68.94000244)
\lineto(207.16999817,56.12000275)
\lineto(189.80999756,56.12000275)
\closepath
}
}
{
\newrgbcolor{curcolor}{0 0 0}
\pscustom[linewidth=1,linecolor=curcolor]
{
\newpath
\moveto(173.72853784,14.46987784)
\lineto(206.60367651,40.47971259)
\lineto(214.7193409,30.22191607)
\lineto(181.84420223,4.21208132)
\closepath
}
}
{
\newrgbcolor{curcolor}{0 0 0}
\pscustom[linewidth=1,linecolor=curcolor]
{
\newpath
\moveto(239.21,99.23999573)
\lineto(252.72,93.42999573)
\lineto(266.23,87.60999573)
\lineto(254.43,78.81999573)
\lineto(242.64,70.02999573)
\lineto(240.92,84.63999573)
\closepath
}
}
{
\newrgbcolor{curcolor}{0 0 0}
\pscustom[linewidth=1,linecolor=curcolor,linestyle=dashed,dash=2]
{
\newpath
\moveto(238.27999878,109.19999695)
\lineto(238.27999878,0.5)
}
}
{
\newrgbcolor{curcolor}{0 0 0}
\pscustom[linewidth=1,linecolor=curcolor,linestyle=dashed,dash=2]
{
\newpath
\moveto(162.69000244,41.63999939)
\lineto(237.83999634,41.63999939)
}
}
{
\newrgbcolor{curcolor}{0 0 0}
\pscustom[linewidth=1,linecolor=curcolor,linestyle=dashed,dash=2]
{
\newpath
\moveto(208.8500061,41.86000061)
\lineto(208.8500061,108.86999512)
}
}
{
\newrgbcolor{curcolor}{0 0 0}
\pscustom[linewidth=1,linecolor=curcolor]
{
\newpath
\moveto(17.13999939,109.19000244)
\lineto(125.81999969,109.19000244)
\lineto(125.81999969,0.51000214)
\lineto(17.13999939,0.51000214)
\closepath
}
}
{
\newrgbcolor{curcolor}{0 0 0}
\pscustom[linewidth=1,linecolor=curcolor]
{
\newpath
\moveto(43.77999973,89.62998962)
\curveto(43.77999973,99.98302387)(34.86129279,105.16596629)(29.64545346,97.84615295)
\curveto(24.42961412,90.52633961)(28.12279504,78.00998974)(35.5,78.00998974)
\curveto(42.87720496,78.00998974)(46.57038588,90.52633961)(41.35454654,97.84615295)
\curveto(36.13870721,105.16596629)(27.22000027,99.98302387)(27.22000027,89.62998962)
\curveto(27.22000027,79.27695537)(36.13870721,74.09401296)(41.35454654,81.4138263)
\curveto(46.57038588,88.73363964)(42.87720496,101.24998951)(35.5,101.24998951)
\curveto(28.12279504,101.24998951)(24.42961412,88.73363964)(29.64545346,81.4138263)
\curveto(34.86129279,74.09401296)(43.77999973,79.27695537)(43.77999973,89.62998962)
\closepath
}
}
{
\newrgbcolor{curcolor}{0 0 0}
\pscustom[linewidth=1,linecolor=curcolor]
{
\newpath
\moveto(44.58000183,68.94000244)
\lineto(61.94000244,68.94000244)
\lineto(61.94000244,56.12000275)
\lineto(44.58000183,56.12000275)
\closepath
}
}
{
\newrgbcolor{curcolor}{0 0 0}
\pscustom[linewidth=1,linecolor=curcolor]
{
\newpath
\moveto(28.48622023,14.46374212)
\lineto(61.36135891,40.47357687)
\lineto(69.4770233,30.21578035)
\lineto(36.60188463,4.2059456)
\closepath
}
}
{
\newrgbcolor{curcolor}{0 0 0}
\pscustom[linewidth=1,linecolor=curcolor]
{
\newpath
\moveto(93.98,99.23999573)
\lineto(107.49,93.42999573)
\lineto(120.99,87.60999573)
\lineto(109.2,78.81999573)
\lineto(97.41,70.02999573)
\lineto(95.69,84.63999573)
\closepath
}
}
{
\newrgbcolor{curcolor}{0 0 0}
\pscustom[linewidth=1,linecolor=curcolor,linestyle=dashed,dash=2]
{
\newpath
\moveto(93.05000305,109.19999695)
\lineto(93.05000305,0.5)
}
}
{
\newrgbcolor{curcolor}{0 0 0}
\pscustom[linewidth=1,linecolor=curcolor,linestyle=dashed,dash=2]
{
\newpath
\moveto(17.45999908,41.63999939)
\lineto(92.59999847,41.63999939)
}
}
{
\newrgbcolor{curcolor}{0 0 0}
\pscustom[linewidth=1,linecolor=curcolor]
{
\newpath
\moveto(162.36999512,254.71999574)
\lineto(271.04999542,254.71999574)
\lineto(271.04999542,146.03999543)
\lineto(162.36999512,146.03999543)
\closepath
}
}
{
\newrgbcolor{curcolor}{0 0 0}
\pscustom[linewidth=1,linecolor=curcolor]
{
\newpath
\moveto(189.00999546,235.15999603)
\curveto(189.00999546,245.51303028)(180.09128852,250.6959727)(174.87544919,243.37615936)
\curveto(169.65960985,236.05634602)(173.35279076,223.53999615)(180.72999573,223.53999615)
\curveto(188.10720069,223.53999615)(191.80038161,236.05634602)(186.58454227,243.37615936)
\curveto(181.36870293,250.6959727)(172.44999599,245.51303028)(172.44999599,235.15999603)
\curveto(172.44999599,224.80696178)(181.36870293,219.62401937)(186.58454227,226.94383271)
\curveto(191.80038161,234.26364604)(188.10720069,246.77999592)(180.72999573,246.77999592)
\curveto(173.35279076,246.77999592)(169.65960985,234.26364604)(174.87544919,226.94383271)
\curveto(180.09128852,219.62401937)(189.00999546,224.80696178)(189.00999546,235.15999603)
\closepath
}
}
{
\newrgbcolor{curcolor}{0 0 0}
\pscustom[linewidth=1,linecolor=curcolor]
{
\newpath
\moveto(189.80999756,214.46999741)
\lineto(207.16999817,214.46999741)
\lineto(207.16999817,201.64999771)
\lineto(189.80999756,201.64999771)
\closepath
}
}
{
\newrgbcolor{curcolor}{0 0 0}
\pscustom[linewidth=1,linecolor=curcolor]
{
\newpath
\moveto(173.72246,159.99963334)
\lineto(206.59759867,186.00946809)
\lineto(214.71326307,175.75167157)
\lineto(181.83812439,149.74183682)
\closepath
}
}
{
\newrgbcolor{curcolor}{0 0 0}
\pscustom[linewidth=1,linecolor=curcolor]
{
\newpath
\moveto(239.21,244.76999573)
\lineto(252.72,238.94999573)
\lineto(266.23,233.12999573)
\lineto(254.43,224.34999573)
\lineto(242.64,215.55999573)
\lineto(240.92,230.15999573)
\closepath
}
}
{
\newrgbcolor{curcolor}{0 0 0}
\pscustom[linewidth=1,linecolor=curcolor,linestyle=dashed,dash=2]
{
\newpath
\moveto(238.27999878,254.72999573)
\lineto(238.27999878,146.01999664)
}
}
{
\newrgbcolor{curcolor}{0 0 0}
\pscustom[linewidth=1,linecolor=curcolor]
{
\newpath
\moveto(17.13999939,254.72999573)
\lineto(125.81999969,254.72999573)
\lineto(125.81999969,146.04999542)
\lineto(17.13999939,146.04999542)
\closepath
}
}
{
\newrgbcolor{curcolor}{0 0 0}
\pscustom[linewidth=1,linecolor=curcolor]
{
\newpath
\moveto(43.77999973,235.16999626)
\curveto(43.77999973,245.52303051)(34.86129279,250.70597293)(29.64545346,243.38615959)
\curveto(24.42961412,236.06634625)(28.12279504,223.54999638)(35.5,223.54999638)
\curveto(42.87720496,223.54999638)(46.57038588,236.06634625)(41.35454654,243.38615959)
\curveto(36.13870721,250.70597293)(27.22000027,245.52303051)(27.22000027,235.16999626)
\curveto(27.22000027,224.81696201)(36.13870721,219.6340196)(41.35454654,226.95383294)
\curveto(46.57038588,234.27364627)(42.87720496,246.78999615)(35.5,246.78999615)
\curveto(28.12279504,246.78999615)(24.42961412,234.27364627)(29.64545346,226.95383294)
\curveto(34.86129279,219.6340196)(43.77999973,224.81696201)(43.77999973,235.16999626)
\closepath
}
}
{
\newrgbcolor{curcolor}{0 0 0}
\pscustom[linewidth=1,linecolor=curcolor]
{
\newpath
\moveto(44.58000183,214.47999573)
\lineto(61.94000244,214.47999573)
\lineto(61.94000244,201.65999603)
\lineto(44.58000183,201.65999603)
\closepath
}
}
{
\newrgbcolor{curcolor}{0 0 0}
\pscustom[linewidth=1,linecolor=curcolor]
{
\newpath
\moveto(28.47773521,160.0091797)
\lineto(61.35287388,186.01901445)
\lineto(69.46853827,175.76121793)
\lineto(36.5933996,149.75138318)
\closepath
}
}
{
\newrgbcolor{curcolor}{0 0 0}
\pscustom[linewidth=1,linecolor=curcolor]
{
\newpath
\moveto(93.98,244.78999573)
\lineto(107.49,238.96999573)
\lineto(120.99,233.14999573)
\lineto(109.2,224.35999573)
\lineto(97.41,215.56999573)
\lineto(95.69,230.17999573)
\closepath
}
}
{
\newrgbcolor{curcolor}{0 0 0}
\pscustom[linewidth=1,linecolor=curcolor]
{
\newpath
\moveto(4.13,175.02999573)
\lineto(4.13,131.26999573)
\lineto(48.29,131.26999573)
}
}
{
\newrgbcolor{curcolor}{0 0 0}
\pscustom[linestyle=none,fillstyle=solid,fillcolor=curcolor]
{
\newpath
\moveto(0,171.33999573)
\lineto(4.13,174.53999573)
\lineto(8.26,171.33999573)
\lineto(4.13,181.09999573)
\closepath
}
}
{
\newrgbcolor{curcolor}{0.65098041 0.65098041 0.65098041}
\pscustom[linestyle=none,fillstyle=solid,fillcolor=curcolor]
{
\newpath
\moveto(0.9,172.50999573)
\lineto(4.13,180.11999573)
\lineto(4.13,175.00999573)
\closepath
}
}
{
\newrgbcolor{curcolor}{0.40000001 0.40000001 0.40000001}
\pscustom[linestyle=none,fillstyle=solid,fillcolor=curcolor]
{
\newpath
\moveto(7.35,172.50999573)
\lineto(4.13,180.11999573)
\lineto(4.13,175.00999573)
\closepath
}
}
{
\newrgbcolor{curcolor}{0 0 0}
\pscustom[linestyle=none,fillstyle=solid,fillcolor=curcolor]
{
\newpath
\moveto(44.61,127.13999573)
\lineto(47.8,131.26999573)
\lineto(44.61,135.38999573)
\lineto(54.37,131.26999573)
\closepath
}
}
{
\newrgbcolor{curcolor}{0.65098041 0.65098041 0.65098041}
\pscustom[linestyle=none,fillstyle=solid,fillcolor=curcolor]
{
\newpath
\moveto(45.78,128.03999573)
\lineto(53.38,131.26999573)
\lineto(48.28,131.26999573)
\closepath
}
}
{
\newrgbcolor{curcolor}{0.40000001 0.40000001 0.40000001}
\pscustom[linestyle=none,fillstyle=solid,fillcolor=curcolor]
{
\newpath
\moveto(45.78,134.48999573)
\lineto(53.38,131.26999573)
\lineto(48.28,131.26999573)
\closepath
}
}
{
\newrgbcolor{curcolor}{0 0 0}
\pscustom[linewidth=1,linecolor=curcolor]
{
\newpath
\moveto(134.3500061,199.18999481)
\lineto(149.58999634,199.18999481)
}
}
{
\newrgbcolor{curcolor}{0 0 0}
\pscustom[linestyle=none,fillstyle=solid,fillcolor=curcolor]
{
\newpath
\moveto(147.14,196.43999573)
\lineto(149.27,199.18999573)
\lineto(147.14,201.93999573)
\lineto(153.65,199.18999573)
\closepath
}
}
{
\newrgbcolor{curcolor}{0.65098041 0.65098041 0.65098041}
\pscustom[linestyle=none,fillstyle=solid,fillcolor=curcolor]
{
\newpath
\moveto(147.92,197.03999573)
\lineto(152.99,199.18999573)
\lineto(149.58,199.18999573)
\closepath
}
}
{
\newrgbcolor{curcolor}{0.40000001 0.40000001 0.40000001}
\pscustom[linestyle=none,fillstyle=solid,fillcolor=curcolor]
{
\newpath
\moveto(147.92,201.33999573)
\lineto(152.99,199.18999573)
\lineto(149.58,199.18999573)
\closepath
}
}
{
\newrgbcolor{curcolor}{0 0 0}
\pscustom[linewidth=1,linecolor=curcolor]
{
\newpath
\moveto(280.73001099,199.1099968)
\lineto(295.97000122,199.1099968)
}
}
{
\newrgbcolor{curcolor}{0 0 0}
\pscustom[linestyle=none,fillstyle=solid,fillcolor=curcolor]
{
\newpath
\moveto(293.51,196.35999573)
\lineto(295.64,199.10999573)
\lineto(293.51,201.85999573)
\lineto(300.02,199.10999573)
\closepath
}
}
{
\newrgbcolor{curcolor}{0.65098041 0.65098041 0.65098041}
\pscustom[linestyle=none,fillstyle=solid,fillcolor=curcolor]
{
\newpath
\moveto(294.29,196.95999573)
\lineto(299.36,199.10999573)
\lineto(295.96,199.10999573)
\closepath
}
}
{
\newrgbcolor{curcolor}{0.40000001 0.40000001 0.40000001}
\pscustom[linestyle=none,fillstyle=solid,fillcolor=curcolor]
{
\newpath
\moveto(294.29,201.25999573)
\lineto(299.36,199.10999573)
\lineto(295.96,199.10999573)
\closepath
}
}
{
\newrgbcolor{curcolor}{0 0 0}
\pscustom[linewidth=1,linecolor=curcolor]
{
\newpath
\moveto(134.83000183,55.22000122)
\lineto(150.07000732,55.22000122)
}
}
{
\newrgbcolor{curcolor}{0 0 0}
\pscustom[linestyle=none,fillstyle=solid,fillcolor=curcolor]
{
\newpath
\moveto(147.61,52.46999573)
\lineto(149.74,55.21999573)
\lineto(147.61,57.96999573)
\lineto(154.12,55.21999573)
\closepath
}
}
{
\newrgbcolor{curcolor}{0.65098041 0.65098041 0.65098041}
\pscustom[linestyle=none,fillstyle=solid,fillcolor=curcolor]
{
\newpath
\moveto(148.39,53.06999573)
\lineto(153.46,55.21999573)
\lineto(150.06,55.21999573)
\closepath
}
}
{
\newrgbcolor{curcolor}{0.40000001 0.40000001 0.40000001}
\pscustom[linestyle=none,fillstyle=solid,fillcolor=curcolor]
{
\newpath
\moveto(148.39,57.36999573)
\lineto(153.46,55.21999573)
\lineto(150.06,55.21999573)
\closepath
}
}
{
\newrgbcolor{curcolor}{0 0 0}
\pscustom[linestyle=none,fillstyle=solid,fillcolor=curcolor]
{
\newpath
\moveto(9.06889958,192.57739907)
\curveto(9.13921208,192.78833657)(9.13921208,192.81177407)(9.13921208,192.92896157)
\curveto(9.13921208,193.18677407)(8.92827458,193.32739907)(8.69389958,193.32739907)
\curveto(8.55327458,193.32739907)(8.31889958,193.23364907)(8.17827458,193.02271157)
\curveto(8.15483708,192.92896157)(8.01421208,192.48364907)(7.96733708,192.20239907)
\curveto(7.85014958,191.82739907)(7.75639958,191.40552407)(7.66264958,191.00708657)
\lineto(6.98296208,188.31177407)
\curveto(6.93608708,188.10083657)(6.27983708,187.04614907)(5.29546208,187.04614907)
\curveto(4.54546208,187.04614907)(4.38139958,187.70239907)(4.38139958,188.26489907)
\curveto(4.38139958,188.94458657)(4.63921208,189.88208657)(5.13139958,191.19458657)
\curveto(5.36577458,191.80396157)(5.43608708,191.96802407)(5.43608708,192.27271157)
\curveto(5.43608708,192.92896157)(4.96733708,193.49146157)(4.21733708,193.49146157)
\curveto(2.78764958,193.49146157)(2.24858708,191.31177407)(2.24858708,191.19458657)
\curveto(2.24858708,191.03052407)(2.38921208,191.03052407)(2.41264958,191.03052407)
\curveto(2.57671208,191.03052407)(2.57671208,191.07739907)(2.64702458,191.31177407)
\curveto(3.06889958,192.71802407)(3.65483708,193.16333657)(4.17046208,193.16333657)
\curveto(4.28764958,193.16333657)(4.54546208,193.16333657)(4.54546208,192.69458657)
\curveto(4.54546208,192.31958657)(4.38139958,191.92114907)(4.28764958,191.63989907)
\curveto(3.67827458,190.04614907)(3.42046208,189.20239907)(3.42046208,188.49927407)
\curveto(3.42046208,187.16333657)(4.35796208,186.71802407)(5.24858708,186.71802407)
\curveto(5.83452458,186.71802407)(6.32671208,186.97583657)(6.74858708,187.39771157)
\curveto(6.56108708,186.62427407)(6.37358708,185.87427407)(5.78764958,185.07739907)
\curveto(5.38921208,184.58521157)(4.82671208,184.13989907)(4.14702458,184.13989907)
\curveto(3.93608708,184.13989907)(3.25639958,184.18677407)(2.99858708,184.77271157)
\curveto(3.23296208,184.77271157)(3.44389958,184.77271157)(3.63139958,184.96021157)
\curveto(3.79546208,185.07739907)(3.93608708,185.28833657)(3.93608708,185.56958657)
\curveto(3.93608708,186.03833657)(3.53764958,186.08521157)(3.39702458,186.08521157)
\curveto(3.04546208,186.08521157)(2.55327458,185.85083657)(2.55327458,185.12427407)
\curveto(2.55327458,184.37427407)(3.20952458,183.81177407)(4.14702458,183.81177407)
\curveto(5.67046208,183.81177407)(7.21733708,185.17114907)(7.63921208,186.85864907)
\closepath
\moveto(9.06889958,192.57739907)
}
}
{
\newrgbcolor{curcolor}{0 0 0}
\pscustom[linestyle=none,fillstyle=solid,fillcolor=curcolor]
{
\newpath
\moveto(62.45644408,132.90413867)
\curveto(62.55019408,133.27913867)(62.90175658,134.66195117)(63.93300658,134.66195117)
\curveto(64.00331908,134.66195117)(64.37831908,134.66195117)(64.68300658,134.47445117)
\curveto(64.26113158,134.38070117)(63.97988158,134.02913867)(63.97988158,133.65413867)
\curveto(63.97988158,133.41976367)(64.14394408,133.13851367)(64.54238158,133.13851367)
\curveto(64.87050658,133.13851367)(65.33925658,133.39632617)(65.33925658,134.00570117)
\curveto(65.33925658,134.77913867)(64.47206908,134.99007617)(63.95644408,134.99007617)
\curveto(63.08925658,134.99007617)(62.57363158,134.19320117)(62.38613158,133.86507617)
\curveto(62.01113158,134.84945117)(61.21425658,134.99007617)(60.76894408,134.99007617)
\curveto(59.22206908,134.99007617)(58.35488158,133.06820117)(58.35488158,132.69320117)
\curveto(58.35488158,132.52913867)(58.51894408,132.52913867)(58.54238158,132.52913867)
\curveto(58.65956908,132.52913867)(58.70644408,132.57601367)(58.72988158,132.69320117)
\curveto(59.24550658,134.28695117)(60.22988158,134.66195117)(60.74550658,134.66195117)
\curveto(61.02675658,134.66195117)(61.54238158,134.52132617)(61.54238158,133.65413867)
\curveto(61.54238158,133.18538867)(61.28456908,132.20101367)(60.74550658,130.09163867)
\curveto(60.51113158,129.17757617)(59.97206908,128.54476367)(59.31581908,128.54476367)
\curveto(59.22206908,128.54476367)(58.89394408,128.54476367)(58.56581908,128.73226367)
\curveto(58.94081908,128.82601367)(59.26894408,129.13070117)(59.26894408,129.55257617)
\curveto(59.26894408,129.95101367)(58.94081908,130.06820117)(58.72988158,130.06820117)
\curveto(58.26113158,130.06820117)(57.90956908,129.69320117)(57.90956908,129.20101367)
\curveto(57.90956908,128.52132617)(58.63613158,128.21663867)(59.29238158,128.21663867)
\curveto(60.30019408,128.21663867)(60.83925658,129.27132617)(60.86269408,129.34163867)
\curveto(61.05019408,128.80257617)(61.58925658,128.21663867)(62.47988158,128.21663867)
\curveto(64.02675658,128.21663867)(64.87050658,130.13851367)(64.87050658,130.51351367)
\curveto(64.87050658,130.67757617)(64.75331908,130.67757617)(64.70644408,130.67757617)
\curveto(64.56581908,130.67757617)(64.54238158,130.60726367)(64.49550658,130.51351367)
\curveto(64.00331908,128.89632617)(62.99550658,128.54476367)(62.52675658,128.54476367)
\curveto(61.94081908,128.54476367)(61.70644408,129.01351367)(61.70644408,129.52913867)
\curveto(61.70644408,129.85726367)(61.77675658,130.18538867)(61.94081908,130.84163867)
\closepath
\moveto(62.45644408,132.90413867)
}
}
\end{pspicture}

    \caption{通过沿坐标轴之一递归地划分场景几何边界框来构建kd树。这里,第一次划分沿$x$轴;
        它摆放后使三角形刚好单独在右边区域而其余图元则在左边。
        然后再用轴对齐的划分平面细化若干次左边的区域。
        细化标准的细节——每一步用哪个轴划分空间、沿轴上哪个位置放置平面
        以及何时结束细分——在实践中均会极大影响树的性能。}
    \label{fig:4.14}
\end{figure}

\begin{lstlisting}
`\initcode{KdTreeAccel Method Definitions}{=}\initnext{KdTreeAccelMethodDefinitions}`
`\refvar{KdTreeAccel}{}`::`\refvar{KdTreeAccel}{}`(
        const std::vector<std::shared_ptr<`\refvar{Primitive}{}`>> &p,
        int isectCost, int traversalCost, `\refvar{Float}{}` emptyBonus,
        int maxPrims, int maxDepth)
    : `\refvar{isectCost}{}`(isectCost), `\refvar{traversalCost}{}`(traversalCost),
      `\refvar{maxPrims}{}`(maxPrims), `\refvar{emptyBonus}{}`(emptyBonus), `\refvar[KdTreeAccel::primitives]{primitives}{}`(p) {
    `\refcode{Build kd-tree for accelerator}{}`
}
\end{lstlisting}

\begin{lstlisting}
`\initcode{KdTreeAccel Private Data}{=}\initnext{KdTreeAccelPrivateData}`
const int `\initvar{isectCost}{}`, `\initvar{traversalCost}{}`, `\initvar{maxPrims}{}`;
const `\refvar{Float}{}` `\initvar{emptyBonus}{}`;
std::vector<std::shared_ptr<`\refvar{Primitive}{}`>> `\initvar[KdTreeAccel::primitives]{primitives}{}`;
\end{lstlisting}

\subsection{树状表示}\label{sub:树状表示}
kd树是二叉树,每个内部节点总是有两个孩子且树的叶子存有与之重合的图元。
每个内部节点必须提供三块信息的访问渠道:
\begin{itemize}
    \item 划分轴:该节点划分了$x,y$和$z$中的哪一个轴;
    \item 划分位置:划分平面沿该轴的位置;
    \item 孩子:关于如何到达其下两个子节点的信息。
\end{itemize}
每个叶子节点只需要记录哪个图元与之重合。

为了保证所有内部节点和许多叶子节点只用8字节内存
(假设\refvar{Float}{}占4字节)而麻烦一下是值得的,
因为这样做保证了八个节点契合一个64字节的缓存行。
因为树中经常有许多节点且每条光线通常都要访问许多节点,
最小化节点表示的大小能极大提高缓存性能。
我们最初的实现使用了16字节节点表示;
当我们把大小减少到8字节时我们得到了几乎20\%的提速。

叶子和内部节点都用下面的结构体\refvar{KdAccelNode}{}表示。
每个{\ttfamily union}成员后的注释都说明了特定域是用于内部节点、叶子节点还是两者都是。
\begin{lstlisting}
`\initcode{KdTreeAccel Local Declarations}{=}\initnext{KdTreeAccelLocalDeclarations}`
struct `\initvar{KdAccelNode}{}` {
    `\refcode{KdAccelNode Methods}{}`
    union {
        `\refvar{Float}{}` `\initvar[KdAccelNode::split]{split}{}`;                  // Interior
        int `\initvar{onePrimitive}{}`;             // Leaf
        int `\initvar{primitiveIndicesOffset}{}`;   // Leaf
    };
    union {
        int `\initvar[KdAccelNode::flags]{flags}{}`;         // Both
        int `\initvar{nPrims}{}`;        // Leaf
        int `\initvar{aboveChild}{}`;    // Interior
    };
};
\end{lstlisting}

变量\refvar{KdAccelNode::flags}{}的低两位用于区分用$x,y$和$z$划分的内部节点
(这些数位分别取值0,1和2)以及叶子节点(这些数位取值3)。
在8字节中保存叶子节点相对简单:\refvar{KdAccelNode::flags}{}的低2位
用于表示这是一个叶子,\refvar[nPrims]{KdAccelNode::nPrims}{}的高30位
可用于记录有多少个图元与之重合。
然后,如果只有一个图元与\refvar{KdAccelNode}{}叶子重合,
则指向数组\refvar{KdTreeAccel::primitives}{}
的整数索引会指出该\refvar{Primitive}{}。如果重合的图元多于一个,
则它们的索引保存于\refvar[primitiveIndices]{KdTreeAccel::primitiveIndices}{}的一段中。
该叶子第一个索引的偏移量存于\refvar[primitiveIndicesOffset]{KdAccelNode::primitiveIndicesOffset}{}且后面直接跟着剩下的索引。
\begin{lstlisting}
`\refcode{KdTreeAccel Private Data}{+=}\lastnext{KdTreeAccelPrivateData}`
std::vector<int> `\initvar{primitiveIndices}{}`;
\end{lstlisting}

叶子节点很容易初始化,不过我们要注意细节,
因为\refvar[KdAccelNode::flags]{flags}{}和\refvar{nPrims}{}共享同一存储;
我们需要注意在初始化其中一个时不要搞乱了另一个。
此外,在保存图元数量前必须向左移两位,
这样\refvar{KdAccelNode::flags}{}的低两位可以都设为1以表示这是一个叶子节点。
\begin{lstlisting}
`\refcode{KdTreeAccel Method Definitions}{+=}\lastnext{KdTreeAccelMethodDefinitions}`
void `\refvar{KdAccelNode}{}`::`\initvar[KdAccelNode::InitLeaf]{InitLeaf}{}`(int *primNums, int np,
        std::vector<int> *primitiveIndices) {
    `\refvar[KdAccelNode::flags]{flags}{}` = 3;
    `\refvar{nPrims}{}` |= (np << 2);
    `\refcode{Store primitive ids for leaf node}{}`
}
\end{lstlisting}

对于有零或一个重合图元的叶子节点,
因为有\refvar[onePrimitive]{KdAccelNode::onePrimitive}{}
域了,所以不再需要额外分配内存。
对于有多个重合图元的情况,则在数组{\ttfamily primitiveIndices}中分配存储。
\begin{lstlisting}
`\initcode{Store primitive ids for leaf node}{=}`
if (np == 0)
    `\refvar{onePrimitive}{}` = 0;
else if (np == 1)
    `\refvar{onePrimitive}{}` = primNums[0];
else {
    `\refvar{primitiveIndicesOffset}{}` = primitiveIndices->size();
    for (int i = 0; i < np; ++i)
        primitiveIndices->push_back(primNums[i]);
}
\end{lstlisting}

让内部节点减少到8字节也相当简单。
一个\refvar{Float}{}(当\refvar{Float}{}定义为{\ttfamily float}时其大小为32位)
保存了节点沿所选划分轴分割空间的位置,并且如之前所述,
\refvar{KdAccelNode::flags}{}低两位用于记录该节点是沿哪个轴划分的。
剩下的就是存储足够的信息使我们遍历树时能找到该节点的两个孩子。

我们排布节点的方式是只存储一个孩子指针,而不是存储两个指针或偏移量:
所有节点都分配到单个连续内存块,
内部节点的对应划分平面下方空间的孩子在数组中的保存位置总是紧跟其父亲
(通过在内存中保持至少一个孩子挨着其父亲,这样的排布也提高了缓存性能)。
另一个对应于划分平面上方的孩子,则在数组其他某处出现;
单个整数偏移量\refvar[aboveChild]{KdAccelNode::aboveChild}{}保存了它在节点数组中的位置。
该表示和\refsub{为遍历而压实的BVH}中BVH节点用的类似。

有了所有这些约定,初始化内部节点的代码就很简单了。
就像方法\refvar[KdAccelNode::InitLeaf]{InitLeaf}{()}
那样,在设置\refvar{aboveChild}{}前为\refvar[KdAccelNode::flags]{flags}{}赋值、
计算移位的\refvar{aboveChild}{}逻辑或值很重要,
这样才不会搞乱保存在\refvar[KdAccelNode::flags]{flags}{}中的数位。
\begin{lstlisting}
`\initcode{KdAccelNode Methods}{=}\initnext{KdAccelNodeMethods}`
void `\initvar[KdAccelNode::InitInterior]{InitInterior}{}`(int axis, int ac, `\refvar{Float}{}` s) {
    `\refvar[KdAccelNode::split]{split}{}` = s;
    `\refvar[KdAccelNode::flags]{flags}{}` = axis;
    `\refvar{aboveChild}{}` |= (ac << 2);
}
\end{lstlisting}

最后,我们将提供一些方法从节点中提取各种值,
这样调用者就不需要了解其内存表示的细节了。
\begin{lstlisting}
`\refcode{KdAccelNode Methods}{+=}\lastcode{KdAccelNodeMethods}`
`\refvar{Float}{}` `\initvar{SplitPos}{()}` const { return `\refvar[KdAccelNode::split]{split}{}`; }
int `\initvar[KdAccelNode::nPrimitives]{nPrimitives}{()}` const { return `\refvar{nPrims}{}` >> 2; }
int `\initvar[KdAccelNode::SplitAxis]{SplitAxis}{()}` const { return `\refvar[KdAccelNode::flags]{flags}{}` & 3; }
bool `\initvar{IsLeaf}{()}` const { return (`\refvar[KdAccelNode::flags]{flags}{}` & 3) == 3; }
int `\initvar{AboveChild}{()}` const { return `\refvar{aboveChild}{}` >> 2; }
\end{lstlisting}

\subsection{树的构建}\label{sub:树的构建}
kd树是用递归自顶向下算法构建的。
每一步中,我们有一个轴对齐空间区域和与该区域重合的图元集。
要么该区域分为两个子区域且转化为内部节点,
要么用重合的图元创建一个叶子节点,结束递归。

正如讨论\refvar{KdAccelNode}{}时所提到的,
所有树节点都保存于连续数组中。\newline
\refvar[nextFreeNode]{KdTreeAccel::nextFreeNode}{}记录了该数组中下一个有效节点,
\refvar[nAllocedNodes]{KdTreeAccel::\newline nAllocedNodes}{}记录了已经分配的总数。
通过一开始设置两者为0且不分配任何节点,这里的实现保证了当初始化树的第一个节点时能立即完成分配。

如果没有为构造函数给定,则还有必要确定树的最大深度。
尽管树的构建过程通常会自然地在合理的深度结束,
但限制最大深度很重要,这样极端情况下树所用的内存数量才不会无限增长。
我们已经发现值$8+1.3\log_2N$为大量场景给出了合理的最大深度。

\begin{lstlisting}
`\initcode{Build kd-tree for accelerator}{=}`
`\refvar{nextFreeNode}{}` = `\refvar{nAllocedNodes}{}` = 0;
if (maxDepth <= 0)
    maxDepth = std::round(8 + 1.3f * `\refvar{Log2Int}{}`(`\refvar[KdTreeAccel::primitives]{primitives}{}`.size()));
`\refcode{Compute bounds for kd-tree construction}{}`
`\refcode{Allocate working memory for kd-tree construction}{}`
`\refcode{Initialize primNums for kd-tree construction}{}`
`\refcode{Start recursive construction of kd-tree}{}`
\end{lstlisting}

\begin{lstlisting}
`\refcode{KdTreeAccel Private Data}{+=}\lastnext{KdTreeAccelPrivateData}`
`\refvar{KdAccelNode}{}` *`\initvar[KdTreeAccel::nodes]{nodes}{}`;
int `\initvar{nAllocedNodes}{}`, `\initvar{nextFreeNode}{}`;
\end{lstlisting}

因为构建例程会一路重复使用图元边界框,
所以在开始构建树前它们被保存在{\ttfamily vector}中,
这样就不需重复调用可能更慢的方法\refvar{Primitive::WorldBound}{()}。
\begin{lstlisting}
`\initcode{Compute bounds for kd-tree construction}{=}`
std::vector<`\refvar{Bounds3f}{}`> primBounds;
for (const std::shared_ptr<`\refvar{Primitive}{}`> &prim : `\refvar[KdTreeAccel::primitives]{primitives}{}`) {
    `\refvar{Bounds3f}{}` b = prim->`\refvar[Primitive::WorldBound]{WorldBound}{}`();
    `\refvar[KdTreeAccel::bounds]{bounds}{}` = `\refvar[Union2]{Union}{}`(`\refvar[KdTreeAccel::bounds]{bounds}{}`, b);
    primBounds.push_back(b);
}
\end{lstlisting}

\begin{lstlisting}
`\refcode{KdTreeAccel Private Data}{+=}\lastcode{KdTreeAccelPrivateData}`
`\refvar{Bounds3f}{}` `\initvar[KdTreeAccel::bounds]{bounds}{}`;
\end{lstlisting}

树构建例程的参数之一是图元索引数组,表示哪个图元与当前节点重合。
因为(当递归开始时)所有图元都和根节点重合,
所以我们从初始化值为零到{\ttfamily primitives.size()-1}的数组开始。
\begin{lstlisting}
`\initcode{Initialize primNums for kd-tree construction}{=}`
std::unique_ptr<int[]> primNums(new int[`\refvar[KdTreeAccel::primitives]{primitives}{}`.size()]);
for (size_t i = 0; i < `\refvar[KdTreeAccel::primitives]{primitives}{}`.size(); ++i)
    primNums[i] = i;
\end{lstlisting}

每个树节点都会调用\refvar{KdTreeAccel::buildTree}{()}。
它负责决定该节点应该是内部节点还是叶子并适当更新数据结构。
最后三个参数{\ttfamily edges}、{\ttfamily prims0}、{\ttfamily prims1}是
指向分配于代码片\refcode{Allocate working memory for kd-tree construction}{}的数据的指针,
稍后几页会对此作定义和介绍。
\begin{lstlisting}
`\initcode{Start recursive construction of kd-tree}{=}`
`\refvar[KdTreeAccel::buildTree]{buildTree}{}`(0, `\refvar[KdTreeAccel::bounds]{bounds}{}`, primBounds, primNums.get(), `\refvar[KdTreeAccel::primitives]{primitives}{}`.size(), 
          maxDepth, edges, prims0.get(), prims1.get());
\end{lstlisting}

\refvar{KdTreeAccel::buildTree}{()}的主要参数是供创建的节点使用的相对于
\refvar{KdAccelNode}{}数组的偏移量{\ttfamily nodeNum}、
给出该节点覆盖的空间区域边界框的{\ttfamily nodeBounds},
以及与之重合的图元索引{\ttfamily primNums}。
其余参数稍后在快用到它们时阐述。
\begin{lstlisting}
`\refcode{KdTreeAccel Method Definitions}{+=}\lastnext{KdTreeAccelMethodDefinitions}`
void `\refvar{KdTreeAccel}{}::\initvar[KdTreeAccel::buildTree]{buildTree}{}`(int nodeNum, const `\refvar{Bounds3f}{}` &nodeBounds,
        const std::vector<`\refvar{Bounds3f}{}`> &allPrimBounds, int *primNums,
        int nPrimitives, int depth,
        const std::unique_ptr<`\refvar{BoundEdge}{}`[]> edges[3], 
        int *prims0, int *prims1, int badRefines) {
    `\refcode{Get next free node from nodes array}{}`
    `\refcode{Initialize leaf node if termination criteria met}{}`
    `\refcode{Initialize interior node and continue recursion}{}`
}
\end{lstlisting}

如果所有分配的节点都已经用完了,则重新分配两倍数量的节点内存并复制旧值。
第一次调用\refvar{KdTreeAccel::buildTree}{()}时,
\refvar[nAllocedNodes]{KdTreeAccel::nAllocedNodes}{}
为0并分配树节点的一个初始块。
\begin{lstlisting}
`\initcode{Get next free node from nodes array}{=}`
if (`\refvar{nextFreeNode}{}` == `\refvar{nAllocedNodes}{}`) {
    int nNewAllocNodes = std::max(2 * `\refvar{nAllocedNodes}{}`, 512);
    `\refvar{KdAccelNode}{}` *n = `\refvar{AllocAligned}{}`<`\refvar{KdAccelNode}{}`>(nNewAllocNodes);
    if (`\refvar{nAllocedNodes}{}` > 0) {
        memcpy(n, `\refvar[KdTreeAccel::nodes]{nodes}{}`, `\refvar{nAllocedNodes}{}` * sizeof(`\refvar{KdAccelNode}{}`));
        `\refvar{FreeAligned}{}`(`\refvar[KdTreeAccel::nodes]{nodes}{}`);
    }
    `\refvar[KdTreeAccel::nodes]{nodes}{}` = n;
    `\refvar{nAllocedNodes}{}` = nNewAllocNodes;
}
++`\refvar{nextFreeNode}{}`;
\end{lstlisting}

当区域内有足够少量的图元或达到最大深度时就创建叶子节点(停止递归)。
参数{\ttfamily depth}一开始为树的最大深度,且每一层递减。
\begin{lstlisting}
`\initcode{Initialize leaf node if termination criteria met}{=}`
if (nPrimitives <= `\refvar{maxPrims}{}` || depth == 0) {
    `\refvar[KdTreeAccel::nodes]{nodes}{}`[nodeNum].`\refvar[KdAccelNode::InitLeaf]{InitLeaf}{}`(primNums, nPrimitives, &`\refvar{primitiveIndices}{}`);
    return;
}
\end{lstlisting}

若这是个内部节点,则需要选择一个划分平面,按该平面划分图元并递归。
\begin{lstlisting}
`\initcode{Initialize interior node and continue recursion}{=}`
`\refcode{Choose split axis position for interior node}{}`
`\refcode{Create leaf if no good splits were found}{}`
`\refcode{Classify primitives with respect to split}{}`
`\refcode{Recursively initialize children nodes}{}`
\end{lstlisting}

我们的实现选择用\refsub{表面积启发法}介绍的SAH来划分。
SAH适用于kd树和BVH;为节点中一系列候选划分平面计算估计的开销,
并选择给出最少开销的划分。

在这里的实现中,相交开销$t_{\text{isect}}$和遍历开销$t_{\text{trav}}$可由用户设置;
它们的默认值分别是80和1。
重要的是,这两个值的比例决定了树构建算法的表现
\footnote{该方法的许多其他实现似乎给这些开销使用了接近得多的值,
    有时甚至接近相等值(例如,见Hurley等\parencite*{hurley2002fast})。
    在pbrt中这里所用的值为大量测试场景给出了最好的性能。
    我们怀疑这一矛盾是因为pbrt中光线-图元相交测试需要两次虚函数调用
    以及一次光线从世界到物体空间的变换这一事实,
    此外还有执行实际相交测试的开销。
    只支持三角图元的高度优化的光线追踪器不会有此类任何额外开销。
    见\refsub{只有三角形}关于这一平衡设计的更多讨论。}。
比起BVH所用的值,这些值之间更大的比例反映的事实是
访问kd树的节点比访问BVH节点的开销更少。

针对用于BVH树的SAH的一点修改是,对于kd树值得稍微偏好选择
使其中一个孩子没有与之重合的图元的划分,
因为光线穿过这些区域可以立即进行到下一个kd树节点而无需任何光线-图元相交测试。
因此,未划分和划分后区域的改进开销分别为
\begin{align*}
    t_{\text{isect}}N \quad \text{和} \quad t_{\text{trav}}+(1-b_{\mathrm{e}})(p_BN_Bt_{\text{isect}}+p_AN_At_{\text{isect}})\, ,
\end{align*}
其中$b_{\mathrm{e}}$是为零的“补贴”\sidenote{译者注:原文bonus。}值,
除非两个区域之一完全为空时取值0到1。

有了为开销模型计算概率的方法,唯一要解决的问题是
怎么生成候选划分位置以及怎么为每个候选者高效计算开销。
可以证明该模型最小开销能于在某一图元边界框的一个面上划分时取得——
不需要考虑在中间位置的划分(为了帮助你自己理解,
考虑一下开销函数在面的边界之间时的特性)。
这里,我们将考虑该区域内三个坐标轴之一或以上的所有边界框面。

利用精心构造的算法可以把检查所有这些候选者的开销维持在相对低的水平。
为了计算这些开销,我们将扫掠边界框在每个轴上的投影并追踪开销最低的那些(\reffig{4.15})。
每个边界框在每个轴上有两处边界,每处都用结构体\refvar{BoundEdge}{}的实例表示。
该结构体记录了边界沿轴的位置,它表示边界框的开始或结束
(沿轴从低到高),以及哪个图元与之关联。
\begin{figure}[htbp]
    \centering%LaTeX with PSTricks extensions
%%Creator: Inkscape 1.0.1 (3bc2e813f5, 2020-09-07)
%%Please note this file requires PSTricks extensions
\psset{xunit=.5pt,yunit=.5pt,runit=.5pt}
\begin{pspicture}(375.70999146,171.55999756)
{
\newrgbcolor{curcolor}{0 0 0}
\pscustom[linewidth=1,linecolor=curcolor]
{
\newpath
\moveto(0,20.94000244)
\lineto(352.42001343,20.94000244)
}
}
{
\newrgbcolor{curcolor}{0 0 0}
\pscustom[linestyle=none,fillstyle=solid,fillcolor=curcolor]
{
\newpath
\moveto(347.51,15.42999756)
\lineto(351.77,20.93999756)
\lineto(347.51,26.43999756)
\lineto(360.53,20.93999756)
\closepath
}
}
{
\newrgbcolor{curcolor}{0.65098041 0.65098041 0.65098041}
\pscustom[linestyle=none,fillstyle=solid,fillcolor=curcolor]
{
\newpath
\moveto(349.07,16.63999756)
\lineto(359.21,20.93999756)
\lineto(352.4,20.93999756)
\closepath
}
}
{
\newrgbcolor{curcolor}{0.40000001 0.40000001 0.40000001}
\pscustom[linestyle=none,fillstyle=solid,fillcolor=curcolor]
{
\newpath
\moveto(349.07,25.23999756)
\lineto(359.21,20.93999756)
\lineto(352.4,20.93999756)
\closepath
}
}
{
\newrgbcolor{curcolor}{0 0 0}
\pscustom[linewidth=1,linecolor=curcolor]
{
\newpath
\moveto(15.44999981,171.05999756)
\lineto(152.00000286,171.05999756)
\lineto(152.00000286,74.27999878)
\lineto(15.44999981,74.27999878)
\closepath
}
}
{
\newrgbcolor{curcolor}{0 0 0}
\pscustom[linewidth=1,linecolor=curcolor]
{
\newpath
\moveto(123.36000061,125.98999786)
\lineto(249.09000397,125.98999786)
\lineto(249.09000397,36.22000122)
\lineto(123.36000061,36.22000122)
\closepath
}
}
{
\newrgbcolor{curcolor}{0 0 0}
\pscustom[linewidth=1,linecolor=curcolor]
{
\newpath
\moveto(290.11999512,164.49999762)
\lineto(344.79999542,164.49999762)
\lineto(344.79999542,109.81999731)
\lineto(290.11999512,109.81999731)
\closepath
}
}
{
\newrgbcolor{curcolor}{0 0 0}
\pscustom[linewidth=1,linecolor=curcolor,linestyle=dashed,dash=2]
{
\newpath
\moveto(15.5,73.80999756)
\lineto(15.5,20.73999023)
}
}
{
\newrgbcolor{curcolor}{0 0 0}
\pscustom[linewidth=1,linecolor=curcolor,linestyle=dashed,dash=2]
{
\newpath
\moveto(123.09999847,35.86000061)
\lineto(123.09999847,21.25)
}
}
{
\newrgbcolor{curcolor}{0 0 0}
\pscustom[linewidth=1,linecolor=curcolor,linestyle=dashed,dash=2]
{
\newpath
\moveto(152.05000305,74.1499939)
\lineto(152.05000305,21.25)
}
}
{
\newrgbcolor{curcolor}{0 0 0}
\pscustom[linewidth=1,linecolor=curcolor,linestyle=dashed,dash=2]
{
\newpath
\moveto(248.83000183,36.02000427)
\lineto(248.83000183,21.41000366)
}
}
{
\newrgbcolor{curcolor}{0 0 0}
\pscustom[linewidth=1,linecolor=curcolor,linestyle=dashed,dash=2]
{
\newpath
\moveto(289.47000122,108.48999786)
\lineto(289.47000122,20.98999023)
}
}
{
\newrgbcolor{curcolor}{0 0 0}
\pscustom[linewidth=1,linecolor=curcolor,linestyle=dashed,dash=2]
{
\newpath
\moveto(344.1499939,108.56999588)
\lineto(344.1499939,21.08000183)
}
}
{
\newrgbcolor{curcolor}{0 0 0}
\pscustom[linestyle=none,fillstyle=solid,fillcolor=curcolor]
{
\newpath
\moveto(78.17413888,122.29249266)
\curveto(77.54913888,121.24249266)(76.92413888,121.01749266)(76.22413888,120.96749266)
\curveto(76.02413888,120.94249266)(75.87413888,120.94249266)(75.87413888,120.64249266)
\curveto(75.87413888,120.54249266)(75.97413888,120.46749266)(76.09913888,120.46749266)
\curveto(76.52413888,120.46749266)(77.02413888,120.51749266)(77.44913888,120.51749266)
\curveto(77.99913888,120.51749266)(78.54913888,120.46749266)(79.04913888,120.46749266)
\curveto(79.14913888,120.46749266)(79.34913888,120.46749266)(79.34913888,120.76749266)
\curveto(79.34913888,120.94249266)(79.22413888,120.96749266)(79.09913888,120.96749266)
\curveto(78.74913888,120.99249266)(78.34913888,121.11749266)(78.34913888,121.51749266)
\curveto(78.34913888,121.71749266)(78.44913888,121.89249266)(78.57413888,122.11749266)
\lineto(79.79913888,124.14249266)
\lineto(83.79913888,124.14249266)
\curveto(83.82413888,123.81749266)(84.04913888,121.64249266)(84.04913888,121.49249266)
\curveto(84.04913888,121.01749266)(83.22413888,120.96749266)(82.89913888,120.96749266)
\curveto(82.67413888,120.96749266)(82.52413888,120.96749266)(82.52413888,120.64249266)
\curveto(82.52413888,120.46749266)(82.69913888,120.46749266)(82.74913888,120.46749266)
\curveto(83.39913888,120.46749266)(84.07413888,120.51749266)(84.72413888,120.51749266)
\curveto(85.12413888,120.51749266)(86.14913888,120.46749266)(86.54913888,120.46749266)
\curveto(86.62413888,120.46749266)(86.82413888,120.46749266)(86.82413888,120.79249266)
\curveto(86.82413888,120.96749266)(86.67413888,120.96749266)(86.44913888,120.96749266)
\curveto(85.47413888,120.96749266)(85.47413888,121.06749266)(85.42413888,121.54249266)
\lineto(84.44913888,131.49249266)
\curveto(84.42413888,131.81749266)(84.42413888,131.89249266)(84.14913888,131.89249266)
\curveto(83.89913888,131.89249266)(83.82413888,131.76749266)(83.72413888,131.61749266)
\closepath
\moveto(80.09913888,124.64249266)
\lineto(83.22413888,129.91749266)
\lineto(83.74913888,124.64249266)
\closepath
\moveto(80.09913888,124.64249266)
}
}
{
\newrgbcolor{curcolor}{0 0 0}
\pscustom[linestyle=none,fillstyle=solid,fillcolor=curcolor]
{
\newpath
\moveto(184.43915488,75.26607566)
\curveto(184.28915488,74.64107566)(184.23915488,74.51607566)(182.98915488,74.51607566)
\curveto(182.71415488,74.51607566)(182.56415488,74.51607566)(182.56415488,74.19107566)
\curveto(182.56415488,74.01607566)(182.71415488,74.01607566)(182.98915488,74.01607566)
\lineto(188.68915488,74.01607566)
\curveto(191.21415488,74.01607566)(193.08915488,75.89107566)(193.08915488,77.46607566)
\curveto(193.08915488,78.61607566)(192.16415488,79.54107566)(190.61415488,79.71607566)
\curveto(192.26415488,80.01607566)(193.93915488,81.19107566)(193.93915488,82.71607566)
\curveto(193.93915488,83.89107566)(192.88915488,84.91607566)(190.98915488,84.91607566)
\lineto(185.61415488,84.91607566)
\curveto(185.31415488,84.91607566)(185.16415488,84.91607566)(185.16415488,84.59107566)
\curveto(185.16415488,84.41607566)(185.31415488,84.41607566)(185.61415488,84.41607566)
\curveto(185.63915488,84.41607566)(185.93915488,84.41607566)(186.21415488,84.39107566)
\curveto(186.48915488,84.34107566)(186.63915488,84.34107566)(186.63915488,84.11607566)
\curveto(186.63915488,84.06607566)(186.61415488,84.01607566)(186.58915488,83.81607566)
\closepath
\moveto(186.83915488,79.86607566)
\lineto(187.83915488,83.81607566)
\curveto(187.98915488,84.36607566)(188.01415488,84.41607566)(188.68915488,84.41607566)
\lineto(190.76415488,84.41607566)
\curveto(192.16415488,84.41607566)(192.48915488,83.46607566)(192.48915488,82.76607566)
\curveto(192.48915488,81.36607566)(191.11415488,79.86607566)(189.18915488,79.86607566)
\closepath
\moveto(186.13915488,74.51607566)
\lineto(185.78915488,74.51607566)
\curveto(185.61415488,74.54107566)(185.56415488,74.56607566)(185.56415488,74.69107566)
\curveto(185.56415488,74.74107566)(185.56415488,74.76607566)(185.66415488,75.04107566)
\lineto(186.76415488,79.49107566)
\lineto(189.76415488,79.49107566)
\curveto(191.28915488,79.49107566)(191.61415488,78.31607566)(191.61415488,77.64107566)
\curveto(191.61415488,76.06607566)(190.18915488,74.51607566)(188.28915488,74.51607566)
\closepath
\moveto(186.13915488,74.51607566)
}
}
{
\newrgbcolor{curcolor}{0 0 0}
\pscustom[linestyle=none,fillstyle=solid,fillcolor=curcolor]
{
\newpath
\moveto(324.14643488,141.72082066)
\curveto(324.14643488,141.77082066)(324.12143488,141.89582066)(323.97143488,141.89582066)
\curveto(323.92143488,141.89582066)(323.89643488,141.87082066)(323.72143488,141.69582066)
\lineto(322.62143488,140.47082066)
\curveto(322.47143488,140.69582066)(321.74643488,141.89582066)(319.97143488,141.89582066)
\curveto(316.39643488,141.89582066)(312.82143488,138.37082066)(312.82143488,134.67082066)
\curveto(312.82143488,132.04582066)(314.69643488,130.29582066)(317.14643488,130.29582066)
\curveto(318.52143488,130.29582066)(319.74643488,130.92082066)(320.59643488,131.67082066)
\curveto(322.07143488,132.97082066)(322.34643488,134.42082066)(322.34643488,134.47082066)
\curveto(322.34643488,134.64582066)(322.17143488,134.64582066)(322.14643488,134.64582066)
\curveto(322.04643488,134.64582066)(321.97143488,134.59582066)(321.94643488,134.47082066)
\curveto(321.79643488,134.02082066)(321.42143488,132.87082066)(320.32143488,131.94582066)
\curveto(319.22143488,131.07082066)(318.22143488,130.79582066)(317.39643488,130.79582066)
\curveto(315.97143488,130.79582066)(314.27143488,131.62082066)(314.27143488,134.09582066)
\curveto(314.27143488,135.02082066)(314.59643488,137.59582066)(316.19643488,139.47082066)
\curveto(317.17143488,140.59582066)(318.67143488,141.39582066)(320.09643488,141.39582066)
\curveto(321.72143488,141.39582066)(322.67143488,140.17082066)(322.67143488,138.32082066)
\curveto(322.67143488,137.67082066)(322.62143488,137.67082066)(322.62143488,137.49582066)
\curveto(322.62143488,137.34582066)(322.79643488,137.34582066)(322.84643488,137.34582066)
\curveto(323.04643488,137.34582066)(323.04643488,137.37082066)(323.14643488,137.67082066)
\closepath
\moveto(324.14643488,141.72082066)
}
}
{
\newrgbcolor{curcolor}{0 0 0}
\pscustom[linestyle=none,fillstyle=solid,fillcolor=curcolor]
{
\newpath
\moveto(367.92731488,22.18139966)
\curveto(368.02731488,22.58139966)(368.40231488,24.05639966)(369.50231488,24.05639966)
\curveto(369.57731488,24.05639966)(369.97731488,24.05639966)(370.30231488,23.85639966)
\curveto(369.85231488,23.75639966)(369.55231488,23.38139966)(369.55231488,22.98139966)
\curveto(369.55231488,22.73139966)(369.72731488,22.43139966)(370.15231488,22.43139966)
\curveto(370.50231488,22.43139966)(371.00231488,22.70639966)(371.00231488,23.35639966)
\curveto(371.00231488,24.18139966)(370.07731488,24.40639966)(369.52731488,24.40639966)
\curveto(368.60231488,24.40639966)(368.05231488,23.55639966)(367.85231488,23.20639966)
\curveto(367.45231488,24.25639966)(366.60231488,24.40639966)(366.12731488,24.40639966)
\curveto(364.47731488,24.40639966)(363.55231488,22.35639966)(363.55231488,21.95639966)
\curveto(363.55231488,21.78139966)(363.72731488,21.78139966)(363.75231488,21.78139966)
\curveto(363.87731488,21.78139966)(363.92731488,21.83139966)(363.95231488,21.95639966)
\curveto(364.50231488,23.65639966)(365.55231488,24.05639966)(366.10231488,24.05639966)
\curveto(366.40231488,24.05639966)(366.95231488,23.90639966)(366.95231488,22.98139966)
\curveto(366.95231488,22.48139966)(366.67731488,21.43139966)(366.10231488,19.18139966)
\curveto(365.85231488,18.20639966)(365.27731488,17.53139966)(364.57731488,17.53139966)
\curveto(364.47731488,17.53139966)(364.12731488,17.53139966)(363.77731488,17.73139966)
\curveto(364.17731488,17.83139966)(364.52731488,18.15639966)(364.52731488,18.60639966)
\curveto(364.52731488,19.03139966)(364.17731488,19.15639966)(363.95231488,19.15639966)
\curveto(363.45231488,19.15639966)(363.07731488,18.75639966)(363.07731488,18.23139966)
\curveto(363.07731488,17.50639966)(363.85231488,17.18139966)(364.55231488,17.18139966)
\curveto(365.62731488,17.18139966)(366.20231488,18.30639966)(366.22731488,18.38139966)
\curveto(366.42731488,17.80639966)(367.00231488,17.18139966)(367.95231488,17.18139966)
\curveto(369.60231488,17.18139966)(370.50231488,19.23139966)(370.50231488,19.63139966)
\curveto(370.50231488,19.80639966)(370.37731488,19.80639966)(370.32731488,19.80639966)
\curveto(370.17731488,19.80639966)(370.15231488,19.73139966)(370.10231488,19.63139966)
\curveto(369.57731488,17.90639966)(368.50231488,17.53139966)(368.00231488,17.53139966)
\curveto(367.37731488,17.53139966)(367.12731488,18.03139966)(367.12731488,18.58139966)
\curveto(367.12731488,18.93139966)(367.20231488,19.28139966)(367.37731488,19.98139966)
\closepath
\moveto(367.92731488,22.18139966)
}
}
{
\newrgbcolor{curcolor}{0 0 0}
\pscustom[linestyle=none,fillstyle=solid,fillcolor=curcolor]
{
\newpath
\moveto(15.31661488,13.54462966)
\curveto(15.01661488,14.14462966)(14.56661488,14.56962966)(13.84161488,14.56962966)
\curveto(11.99161488,14.56962966)(10.01661488,12.21962966)(10.01661488,9.89462966)
\curveto(10.01661488,8.39462966)(10.89161488,7.34462966)(12.11661488,7.34462966)
\curveto(12.44161488,7.34462966)(13.24161488,7.41962966)(14.19161488,8.54462966)
\curveto(14.31661488,7.86962966)(14.89161488,7.34462966)(15.64161488,7.34462966)
\curveto(16.21661488,7.34462966)(16.56661488,7.71962966)(16.84161488,8.21962966)
\curveto(17.09161488,8.79462966)(17.31661488,9.76962966)(17.31661488,9.79462966)
\curveto(17.31661488,9.96962966)(17.16661488,9.96962966)(17.11661488,9.96962966)
\curveto(16.96661488,9.96962966)(16.94161488,9.89462966)(16.89161488,9.66962966)
\curveto(16.61661488,8.64462966)(16.34161488,7.69462966)(15.69161488,7.69462966)
\curveto(15.24161488,7.69462966)(15.21661488,8.11962966)(15.21661488,8.41962966)
\curveto(15.21661488,8.76962966)(15.24161488,8.91962966)(15.41661488,9.61962966)
\curveto(15.59161488,10.26962966)(15.61661488,10.44462966)(15.76661488,11.04462966)
\lineto(16.34161488,13.26962966)
\curveto(16.44161488,13.71962966)(16.44161488,13.74462966)(16.44161488,13.81962966)
\curveto(16.44161488,14.09462966)(16.26661488,14.24462966)(15.99161488,14.24462966)
\curveto(15.59161488,14.24462966)(15.36661488,13.89462966)(15.31661488,13.54462966)
\closepath
\moveto(14.29161488,9.41962966)
\curveto(14.19161488,9.11962966)(14.19161488,9.09462966)(13.96661488,8.81962966)
\curveto(13.26661488,7.94462966)(12.61661488,7.69462966)(12.16661488,7.69462966)
\curveto(11.36661488,7.69462966)(11.14161488,8.56962966)(11.14161488,9.19462966)
\curveto(11.14161488,9.99462966)(11.64161488,11.94462966)(12.01661488,12.69462966)
\curveto(12.51661488,13.61962966)(13.21661488,14.21962966)(13.86661488,14.21962966)
\curveto(14.89161488,14.21962966)(15.11661488,12.91962966)(15.11661488,12.81962966)
\curveto(15.11661488,12.71962966)(15.09161488,12.61962966)(15.06661488,12.54462966)
\closepath
\moveto(14.29161488,9.41962966)
}
}
{
\newrgbcolor{curcolor}{0 0 0}
\pscustom[linestyle=none,fillstyle=solid,fillcolor=curcolor]
{
\newpath
\moveto(23.54221303,8.67922439)
\curveto(23.54221303,9.90422439)(23.39221303,10.80422439)(22.89221303,11.57922439)
\curveto(22.54221303,12.07922439)(21.84221303,12.52922439)(20.96721303,12.52922439)
\curveto(18.36721303,12.52922439)(18.36721303,9.47922439)(18.36721303,8.67922439)
\curveto(18.36721303,7.87922439)(18.36721303,4.90422439)(20.96721303,4.90422439)
\curveto(23.54221303,4.90422439)(23.54221303,7.87922439)(23.54221303,8.67922439)
\closepath
\moveto(20.96721303,5.22922439)
\curveto(20.44221303,5.22922439)(19.76721303,5.52922439)(19.54221303,6.42922439)
\curveto(19.39221303,7.07922439)(19.39221303,8.00422439)(19.39221303,8.82922439)
\curveto(19.39221303,9.65422439)(19.39221303,10.50422439)(19.54221303,11.10422439)
\curveto(19.79221303,11.97922439)(20.49221303,12.22922439)(20.96721303,12.22922439)
\curveto(21.56721303,12.22922439)(22.14221303,11.85422439)(22.34221303,11.20422439)
\curveto(22.51721303,10.60422439)(22.54221303,9.80422439)(22.54221303,8.82922439)
\curveto(22.54221303,8.00422439)(22.54221303,7.17922439)(22.39221303,6.47922439)
\curveto(22.16721303,5.45422439)(21.41721303,5.22922439)(20.96721303,5.22922439)
\closepath
\moveto(20.96721303,5.22922439)
}
}
{
\newrgbcolor{curcolor}{0 0 0}
\pscustom[linestyle=none,fillstyle=solid,fillcolor=curcolor]
{
\newpath
\moveto(152.27998299,12.51996566)
\curveto(151.97998299,13.11996566)(151.52998299,13.54496566)(150.80498299,13.54496566)
\curveto(148.95498299,13.54496566)(146.97998299,11.19496566)(146.97998299,8.86996566)
\curveto(146.97998299,7.36996566)(147.85498299,6.31996566)(149.07998299,6.31996566)
\curveto(149.40498299,6.31996566)(150.20498299,6.39496566)(151.15498299,7.51996566)
\curveto(151.27998299,6.84496566)(151.85498299,6.31996566)(152.60498299,6.31996566)
\curveto(153.17998299,6.31996566)(153.52998299,6.69496566)(153.80498299,7.19496566)
\curveto(154.05498299,7.76996566)(154.27998299,8.74496566)(154.27998299,8.76996566)
\curveto(154.27998299,8.94496566)(154.12998299,8.94496566)(154.07998299,8.94496566)
\curveto(153.92998299,8.94496566)(153.90498299,8.86996566)(153.85498299,8.64496566)
\curveto(153.57998299,7.61996566)(153.30498299,6.66996566)(152.65498299,6.66996566)
\curveto(152.20498299,6.66996566)(152.17998299,7.09496566)(152.17998299,7.39496566)
\curveto(152.17998299,7.74496566)(152.20498299,7.89496566)(152.37998299,8.59496566)
\curveto(152.55498299,9.24496566)(152.57998299,9.41996566)(152.72998299,10.01996566)
\lineto(153.30498299,12.24496566)
\curveto(153.40498299,12.69496566)(153.40498299,12.71996566)(153.40498299,12.79496566)
\curveto(153.40498299,13.06996566)(153.22998299,13.21996566)(152.95498299,13.21996566)
\curveto(152.55498299,13.21996566)(152.32998299,12.86996566)(152.27998299,12.51996566)
\closepath
\moveto(151.25498299,8.39496566)
\curveto(151.15498299,8.09496566)(151.15498299,8.06996566)(150.92998299,7.79496566)
\curveto(150.22998299,6.91996566)(149.57998299,6.66996566)(149.12998299,6.66996566)
\curveto(148.32998299,6.66996566)(148.10498299,7.54496566)(148.10498299,8.16996566)
\curveto(148.10498299,8.96996566)(148.60498299,10.91996566)(148.97998299,11.66996566)
\curveto(149.47998299,12.59496566)(150.17998299,13.19496566)(150.82998299,13.19496566)
\curveto(151.85498299,13.19496566)(152.07998299,11.89496566)(152.07998299,11.79496566)
\curveto(152.07998299,11.69496566)(152.05498299,11.59496566)(152.02998299,11.51996566)
\closepath
\moveto(151.25498299,8.39496566)
}
}
{
\newrgbcolor{curcolor}{0 0 0}
\pscustom[linestyle=none,fillstyle=solid,fillcolor=curcolor]
{
\newpath
\moveto(158.48058114,11.20456039)
\curveto(158.48058114,11.50456039)(158.48058114,11.50456039)(158.15558114,11.50456039)
\curveto(157.43058114,10.80456039)(156.43058114,10.80456039)(155.98058114,10.80456039)
\lineto(155.98058114,10.40456039)
\curveto(156.23058114,10.40456039)(156.98058114,10.40456039)(157.58058114,10.70456039)
\lineto(157.58058114,5.02956039)
\curveto(157.58058114,4.65456039)(157.58058114,4.50456039)(156.48058114,4.50456039)
\lineto(156.05558114,4.50456039)
\lineto(156.05558114,4.10456039)
\curveto(156.25558114,4.10456039)(157.63058114,4.15456039)(158.03058114,4.15456039)
\curveto(158.38058114,4.15456039)(159.78058114,4.10456039)(160.03058114,4.10456039)
\lineto(160.03058114,4.50456039)
\lineto(159.60558114,4.50456039)
\curveto(158.48058114,4.50456039)(158.48058114,4.65456039)(158.48058114,5.02956039)
\closepath
\moveto(158.48058114,11.20456039)
}
}
{
\newrgbcolor{curcolor}{0 0 0}
\pscustom[linestyle=none,fillstyle=solid,fillcolor=curcolor]
{
\newpath
\moveto(248.83924488,16.37030166)
\curveto(248.83924488,16.37030166)(248.83924488,16.54530166)(248.63924488,16.54530166)
\curveto(248.28924488,16.54530166)(247.11424488,16.42030166)(246.68924488,16.37030166)
\curveto(246.56424488,16.37030166)(246.38924488,16.34530166)(246.38924488,16.07030166)
\curveto(246.38924488,15.87030166)(246.53924488,15.87030166)(246.78924488,15.87030166)
\curveto(247.53924488,15.87030166)(247.56424488,15.77030166)(247.56424488,15.59530166)
\curveto(247.56424488,15.49530166)(247.43924488,14.94530166)(247.36424488,14.62030166)
\lineto(246.03924488,9.42030166)
\curveto(245.86424488,8.62030166)(245.78924488,8.34530166)(245.78924488,7.79530166)
\curveto(245.78924488,6.29530166)(246.63924488,5.29530166)(247.81424488,5.29530166)
\curveto(249.68924488,5.29530166)(251.66424488,7.67030166)(251.66424488,9.97030166)
\curveto(251.66424488,11.42030166)(250.81424488,12.52030166)(249.53924488,12.52030166)
\curveto(248.81424488,12.52030166)(248.13924488,12.04530166)(247.66424488,11.57030166)
\closepath
\moveto(247.36424488,10.34530166)
\curveto(247.43924488,10.69530166)(247.43924488,10.72030166)(247.58924488,10.89530166)
\curveto(248.36424488,11.92030166)(249.08924488,12.17030166)(249.51424488,12.17030166)
\curveto(250.08924488,12.17030166)(250.51424488,11.69530166)(250.51424488,10.67030166)
\curveto(250.51424488,9.72030166)(249.98924488,7.89530166)(249.68924488,7.29530166)
\curveto(249.16424488,6.22030166)(248.43924488,5.64530166)(247.81424488,5.64530166)
\curveto(247.26424488,5.64530166)(246.73924488,6.07030166)(246.73924488,7.24530166)
\curveto(246.73924488,7.57030166)(246.73924488,7.87030166)(246.98924488,8.87030166)
\closepath
\moveto(247.36424488,10.34530166)
}
}
{
\newrgbcolor{curcolor}{0 0 0}
\pscustom[linestyle=none,fillstyle=solid,fillcolor=curcolor]
{
\newpath
\moveto(255.60583424,10.17989639)
\curveto(255.60583424,10.47989639)(255.60583424,10.47989639)(255.28083424,10.47989639)
\curveto(254.55583424,9.77989639)(253.55583424,9.77989639)(253.10583424,9.77989639)
\lineto(253.10583424,9.37989639)
\curveto(253.35583424,9.37989639)(254.10583424,9.37989639)(254.70583424,9.67989639)
\lineto(254.70583424,4.00489639)
\curveto(254.70583424,3.62989639)(254.70583424,3.47989639)(253.60583424,3.47989639)
\lineto(253.18083424,3.47989639)
\lineto(253.18083424,3.07989639)
\curveto(253.38083424,3.07989639)(254.75583424,3.12989639)(255.15583424,3.12989639)
\curveto(255.50583424,3.12989639)(256.90583424,3.07989639)(257.15583424,3.07989639)
\lineto(257.15583424,3.47989639)
\lineto(256.73083424,3.47989639)
\curveto(255.60583424,3.47989639)(255.60583424,3.62989639)(255.60583424,4.00489639)
\closepath
\moveto(255.60583424,10.17989639)
}
}
{
\newrgbcolor{curcolor}{0 0 0}
\pscustom[linestyle=none,fillstyle=solid,fillcolor=curcolor]
{
\newpath
\moveto(123.14717488,16.02874766)
\curveto(123.14717488,16.02874766)(123.14717488,16.20374766)(122.94717488,16.20374766)
\curveto(122.59717488,16.20374766)(121.42217488,16.07874766)(120.99717488,16.02874766)
\curveto(120.87217488,16.02874766)(120.69717488,16.00374766)(120.69717488,15.72874766)
\curveto(120.69717488,15.52874766)(120.84717488,15.52874766)(121.09717488,15.52874766)
\curveto(121.84717488,15.52874766)(121.87217488,15.42874766)(121.87217488,15.25374766)
\curveto(121.87217488,15.15374766)(121.74717488,14.60374766)(121.67217488,14.27874766)
\lineto(120.34717488,9.07874766)
\curveto(120.17217488,8.27874766)(120.09717488,8.00374766)(120.09717488,7.45374766)
\curveto(120.09717488,5.95374766)(120.94717488,4.95374766)(122.12217488,4.95374766)
\curveto(123.99717488,4.95374766)(125.97217488,7.32874766)(125.97217488,9.62874766)
\curveto(125.97217488,11.07874766)(125.12217488,12.17874766)(123.84717488,12.17874766)
\curveto(123.12217488,12.17874766)(122.44717488,11.70374766)(121.97217488,11.22874766)
\closepath
\moveto(121.67217488,10.00374766)
\curveto(121.74717488,10.35374766)(121.74717488,10.37874766)(121.89717488,10.55374766)
\curveto(122.67217488,11.57874766)(123.39717488,11.82874766)(123.82217488,11.82874766)
\curveto(124.39717488,11.82874766)(124.82217488,11.35374766)(124.82217488,10.32874766)
\curveto(124.82217488,9.37874766)(124.29717488,7.55374766)(123.99717488,6.95374766)
\curveto(123.47217488,5.87874766)(122.74717488,5.30374766)(122.12217488,5.30374766)
\curveto(121.57217488,5.30374766)(121.04717488,5.72874766)(121.04717488,6.90374766)
\curveto(121.04717488,7.22874766)(121.04717488,7.52874766)(121.29717488,8.52874766)
\closepath
\moveto(121.67217488,10.00374766)
}
}
{
\newrgbcolor{curcolor}{0 0 0}
\pscustom[linestyle=none,fillstyle=solid,fillcolor=curcolor]
{
\newpath
\moveto(131.93876424,6.28834239)
\curveto(131.93876424,7.51334239)(131.78876424,8.41334239)(131.28876424,9.18834239)
\curveto(130.93876424,9.68834239)(130.23876424,10.13834239)(129.36376424,10.13834239)
\curveto(126.76376424,10.13834239)(126.76376424,7.08834239)(126.76376424,6.28834239)
\curveto(126.76376424,5.48834239)(126.76376424,2.51334239)(129.36376424,2.51334239)
\curveto(131.93876424,2.51334239)(131.93876424,5.48834239)(131.93876424,6.28834239)
\closepath
\moveto(129.36376424,2.83834239)
\curveto(128.83876424,2.83834239)(128.16376424,3.13834239)(127.93876424,4.03834239)
\curveto(127.78876424,4.68834239)(127.78876424,5.61334239)(127.78876424,6.43834239)
\curveto(127.78876424,7.26334239)(127.78876424,8.11334239)(127.93876424,8.71334239)
\curveto(128.18876424,9.58834239)(128.88876424,9.83834239)(129.36376424,9.83834239)
\curveto(129.96376424,9.83834239)(130.53876424,9.46334239)(130.73876424,8.81334239)
\curveto(130.91376424,8.21334239)(130.93876424,7.41334239)(130.93876424,6.43834239)
\curveto(130.93876424,5.61334239)(130.93876424,4.78834239)(130.78876424,4.08834239)
\curveto(130.56376424,3.06334239)(129.81376424,2.83834239)(129.36376424,2.83834239)
\closepath
\moveto(129.36376424,2.83834239)
}
}
{
\newrgbcolor{curcolor}{0 0 0}
\pscustom[linestyle=none,fillstyle=solid,fillcolor=curcolor]
{
\newpath
\moveto(291.32612488,12.88651966)
\curveto(291.05112488,12.88651966)(290.85112488,12.88651966)(290.62612488,12.68651966)
\curveto(290.35112488,12.43651966)(290.32612488,12.16151966)(290.32612488,12.06151966)
\curveto(290.32612488,11.66151966)(290.62612488,11.48651966)(290.92612488,11.48651966)
\curveto(291.37612488,11.48651966)(291.80112488,11.88651966)(291.80112488,12.51151966)
\curveto(291.80112488,13.28651966)(291.05112488,13.88651966)(289.92612488,13.88651966)
\curveto(287.77612488,13.88651966)(285.65112488,11.61151966)(285.65112488,9.36151966)
\curveto(285.65112488,7.91151966)(286.57612488,6.66151966)(288.25112488,6.66151966)
\curveto(290.52612488,6.66151966)(291.85112488,8.36151966)(291.85112488,8.53651966)
\curveto(291.85112488,8.63651966)(291.77612488,8.76151966)(291.67612488,8.76151966)
\curveto(291.57612488,8.76151966)(291.55112488,8.71151966)(291.45112488,8.58651966)
\curveto(290.20112488,7.01151966)(288.45112488,7.01151966)(288.27612488,7.01151966)
\curveto(287.27612488,7.01151966)(286.82612488,7.78651966)(286.82612488,8.76151966)
\curveto(286.82612488,9.41151966)(287.15112488,10.96151966)(287.70112488,11.93651966)
\curveto(288.20112488,12.86151966)(289.07612488,13.53651966)(289.95112488,13.53651966)
\curveto(290.47612488,13.53651966)(291.10112488,13.33651966)(291.32612488,12.88651966)
\closepath
\moveto(291.32612488,12.88651966)
}
}
{
\newrgbcolor{curcolor}{0 0 0}
\pscustom[linestyle=none,fillstyle=solid,fillcolor=curcolor]
{
\newpath
\moveto(297.65031922,7.99611439)
\curveto(297.65031922,9.22111439)(297.50031922,10.12111439)(297.00031922,10.89611439)
\curveto(296.65031922,11.39611439)(295.95031922,11.84611439)(295.07531922,11.84611439)
\curveto(292.47531922,11.84611439)(292.47531922,8.79611439)(292.47531922,7.99611439)
\curveto(292.47531922,7.19611439)(292.47531922,4.22111439)(295.07531922,4.22111439)
\curveto(297.65031922,4.22111439)(297.65031922,7.19611439)(297.65031922,7.99611439)
\closepath
\moveto(295.07531922,4.54611439)
\curveto(294.55031922,4.54611439)(293.87531922,4.84611439)(293.65031922,5.74611439)
\curveto(293.50031922,6.39611439)(293.50031922,7.32111439)(293.50031922,8.14611439)
\curveto(293.50031922,8.97111439)(293.50031922,9.82111439)(293.65031922,10.42111439)
\curveto(293.90031922,11.29611439)(294.60031922,11.54611439)(295.07531922,11.54611439)
\curveto(295.67531922,11.54611439)(296.25031922,11.17111439)(296.45031922,10.52111439)
\curveto(296.62531922,9.92111439)(296.65031922,9.12111439)(296.65031922,8.14611439)
\curveto(296.65031922,7.32111439)(296.65031922,6.49611439)(296.50031922,5.79611439)
\curveto(296.27531922,4.77111439)(295.52531922,4.54611439)(295.07531922,4.54611439)
\closepath
\moveto(295.07531922,4.54611439)
}
}
{
\newrgbcolor{curcolor}{0 0 0}
\pscustom[linestyle=none,fillstyle=solid,fillcolor=curcolor]
{
\newpath
\moveto(344.60863488,12.34656766)
\curveto(344.33363488,12.34656766)(344.13363488,12.34656766)(343.90863488,12.14656766)
\curveto(343.63363488,11.89656766)(343.60863488,11.62156766)(343.60863488,11.52156766)
\curveto(343.60863488,11.12156766)(343.90863488,10.94656766)(344.20863488,10.94656766)
\curveto(344.65863488,10.94656766)(345.08363488,11.34656766)(345.08363488,11.97156766)
\curveto(345.08363488,12.74656766)(344.33363488,13.34656766)(343.20863488,13.34656766)
\curveto(341.05863488,13.34656766)(338.93363488,11.07156766)(338.93363488,8.82156766)
\curveto(338.93363488,7.37156766)(339.85863488,6.12156766)(341.53363488,6.12156766)
\curveto(343.80863488,6.12156766)(345.13363488,7.82156766)(345.13363488,7.99656766)
\curveto(345.13363488,8.09656766)(345.05863488,8.22156766)(344.95863488,8.22156766)
\curveto(344.85863488,8.22156766)(344.83363488,8.17156766)(344.73363488,8.04656766)
\curveto(343.48363488,6.47156766)(341.73363488,6.47156766)(341.55863488,6.47156766)
\curveto(340.55863488,6.47156766)(340.10863488,7.24656766)(340.10863488,8.22156766)
\curveto(340.10863488,8.87156766)(340.43363488,10.42156766)(340.98363488,11.39656766)
\curveto(341.48363488,12.32156766)(342.35863488,12.99656766)(343.23363488,12.99656766)
\curveto(343.75863488,12.99656766)(344.38363488,12.79656766)(344.60863488,12.34656766)
\closepath
\moveto(344.60863488,12.34656766)
}
}
{
\newrgbcolor{curcolor}{0 0 0}
\pscustom[linestyle=none,fillstyle=solid,fillcolor=curcolor]
{
\newpath
\moveto(348.90782922,11.00616239)
\curveto(348.90782922,11.30616239)(348.90782922,11.30616239)(348.58282922,11.30616239)
\curveto(347.85782922,10.60616239)(346.85782922,10.60616239)(346.40782922,10.60616239)
\lineto(346.40782922,10.20616239)
\curveto(346.65782922,10.20616239)(347.40782922,10.20616239)(348.00782922,10.50616239)
\lineto(348.00782922,4.83116239)
\curveto(348.00782922,4.45616239)(348.00782922,4.30616239)(346.90782922,4.30616239)
\lineto(346.48282922,4.30616239)
\lineto(346.48282922,3.90616239)
\curveto(346.68282922,3.90616239)(348.05782922,3.95616239)(348.45782922,3.95616239)
\curveto(348.80782922,3.95616239)(350.20782922,3.90616239)(350.45782922,3.90616239)
\lineto(350.45782922,4.30616239)
\lineto(350.03282922,4.30616239)
\curveto(348.90782922,4.30616239)(348.90782922,4.45616239)(348.90782922,4.83116239)
\closepath
\moveto(348.90782922,11.00616239)
}
}
\end{pspicture}

    \caption{给定我们要考虑的可能划分所沿的轴,图元的边界框被投影到该轴上,
    这带来了一个高效算法以追踪特定划分平面两侧会各有多少图元。
    例如这里,在$a_1$处划分会让$A$完全留在划分平面下方,$B$横跨之,而$C$完全在其上方。
    轴上每一个点$a_0,a_1,b_0,b_1,c_0$和$c_1$都由结构体\refvar{BoundEdge}{}的一个实例表示。}
    \label{fig:4.15}
\end{figure}
\begin{lstlisting}
`\refcode{KdTreeAccel Local Declarations}{+=}\lastnext{KdTreeAccelLocalDeclarations}`
enum class `\initvar{EdgeType}{}` { `\initvar[EdgeType::Start]{Start}{}`, `\initvar[EdgeType::End]{End}{}` };
\end{lstlisting}
\begin{lstlisting}
`\refcode{KdTreeAccel Local Declarations}{+=}\lastcode{KdTreeAccelLocalDeclarations}`
struct `\initvar{BoundEdge}{}` {
    `\refcode{BoundEdge Public Methods}{}`
    `\refvar{Float}{}` `\initvar[BoundEdge::t]{t}{}`;
    int `\initvar[BoundEdge::primNum]{primNum}{}`;
    `\refvar{EdgeType}{}` `\initvar[BoundEdge::type]{type}{}`;
};
\end{lstlisting}
\begin{lstlisting}
`\initcode{BoundEdge Public Methods}{=}`
`\refvar{BoundEdge}{}`(`\refvar{Float}{}` t, int primNum, bool starting)
    : `\refvar[BoundEdge::t]{t}{}`(t), `\refvar[BoundEdge::primNum]{primNum}{}`(primNum) {
    `\refvar[BoundEdge::type]{type}{}` = starting ? `\refvar{EdgeType::Start}{}` : `\refvar{EdgeType::End}{}`; 
}
\end{lstlisting}

对于任意树节点至多需要为{\ttfamily 2*\refvar[KdTreeAccel::primitives]{primitives}{}.size()}个\refvar{BoundEdge}{}计算开销,
所以一次分配全部三轴上所有边界的内存然后再为每个创建的节点复用。
\begin{lstlisting}
`\initcode{Allocate working memory for kd-tree construction}{=}\initnext{Allocateworkingmemoryforkdtreeconstruction}`
std::unique_ptr<`\refvar{BoundEdge}{}`[]> edges[3];
for (int i = 0; i < 3; ++i)
    edges[i].reset(new `\refvar{BoundEdge}{}`[2 * `\refvar[KdTreeAccel::primitives]{primitives}{}`.size()]);
\end{lstlisting}

在为创建的叶子确定估计的开销后,\refvar{KdTreeAccel::buildTree}{()}选择
一个轴尝试沿其划分并为每个候选划分计算开销函数。
{\ttfamily bestAxis}和{\ttfamily bestOffset}记录了该轴
以及目前给出最低开销{\ttfamily bestCost}的边界框边界索引。
{\ttfamily invTotalSA}初始化为节点表面积的倒数;
当计算光线穿过每个候选孩子节点的概率时会用到它的值。
\begin{lstlisting}
`\initcode{Choose split axis position for interior node}{=}`
int bestAxis = -1, bestOffset = -1;
`\refvar{Float}{}` bestCost = `\refvar{Infinity}{}`;
`\refvar{Float}{}` oldCost = `\refvar{isectCost}{}` * `\refvar{Float}{}`(nPrimitives);
`\refvar{Float}{}` totalSA = nodeBounds.`\refvar{SurfaceArea}{}`();
`\refvar{Float}{}` invTotalSA = 1 / totalSA;
`\refvar{Vector3f}{}` d = nodeBounds.`\refvar{pMax}{}` - nodeBounds.`\refvar{pMin}{}`;
`\refcode{Choose which axis to split along}{}`
int retries = 0;
retrySplit:
`\refcode{Initialize edges for axis}{}`
`\refcode{Compute cost of all splits for axis to find best}{}`
\end{lstlisting}

该方法首先尝试沿具有最大空间范围的轴寻找一个划分;
如果成功,该选项则有助于给出形状上趋于方形的空间区域。
这在直觉上是合理的方法。
如果沿该轴没有成功找到好的划分,则回退并依次尝试其他的。
\begin{lstlisting}
`\initcode{Choose which axis to split along}{=}`
int axis = nodeBounds.`\refvar{MaximumExtent}{}`();
\end{lstlisting}

首先用重合图元的边界框初始化该轴的数组{\ttfamily edges}。
然后该数组沿该轴从低到高存储,这样它就能从头到尾扫掠框的边界。
\begin{lstlisting}
`\initcode{Initialize edges for axis}{=}`
for (int i = 0; i < nPrimitives; ++i) {
    int pn = primNums[i];
    const `\refvar{Bounds3f}{}` &bounds = allPrimBounds[pn];
    edges[axis][2 * i] =     `\refvar{BoundEdge}{}`(bounds.`\refvar{pMin}{}`[axis], pn, true);
    edges[axis][2 * i + 1] = `\refvar{BoundEdge}{}`(bounds.`\refvar{pMax}{}`[axis], pn, false);
}
`\refcode{Sort edges for axis}{}`
\end{lstlisting}

C++标准库例程{\ttfamily sort()}要求被排序的结构要定义顺序;
这用值\refvar{BoundEdge::t}{}
来完成。然而,一个细微之处是如果值\refvar{BoundEdge::t}{}相同,
则需要通过比较节点类型来打破平局;
这是必要的,因为{\ttfamily sort()}所取决的事实是
{\ttfamily a < b}和{\ttfamily b < a}都为{\ttfamily false}的唯一时刻是{\ttfamily a == b}。
\begin{lstlisting}
`\initcode{Sort edges for axis}{=}`
std::sort(&edges[axis][0], &edges[axis][2*nPrimitives],
    [](const `\refvar{BoundEdge}{}` &e0, const `\refvar{BoundEdge}{}` &e1) -> bool {
        if (e0.`\refvar[BoundEdge::t]{t}{}` == e1.`\refvar[BoundEdge::t]{t}{}`)
            return (int)e0.`\refvar[BoundEdge::type]{type}{}` < (int)e1.`\refvar[BoundEdge::type]{type}{}`;
        else return e0.`\refvar[BoundEdge::t]{t}{}` < e1.`\refvar[BoundEdge::t]{t}{}`; 
    });
\end{lstlisting}

有了排好序的边界数组,我们想为它们中的每一处划分快速计算开销函数。
光线穿过每个孩子节点的概率很容易用其表面积计算,
划分处每侧的图元数量由变量{\ttfamily nBelow}和{\ttfamily nAbove}跟踪。
我们想保持它们的值是最新的,这样如果我们在某次循环中选择在{\ttfamily edgeT}处划分,
{\ttfamily nBelow}会给出最终在划分平面之下的图元数量而{\ttfamily nAbove}则给出之上的数量
\footnote{当多个边界框面投影到轴上同一点时,这一特性在这些点处可能不成立。
然而此处的实现只会高估数量,而且更重要的是,
它会于在这些点的每一个上进行的多次循环中的某一次取到正确的值,
所以无论如何最终算法的功能是正确的。}。

在第一个边界处,依据定义所有图元必须在该边界之上,
所以{\ttfamily nAbove}初始化为{\ttfamily nPrimitives}且{\ttfamily nBelow}设为0。
当循环考虑在边界框范围尾部的划分时,
{\ttfamily nAbove}需要递减,因为该框以前一定在划分平面之上,
如果在该点完成划分则它不会再在平面之上。
同样,计算划分开销后,如果划分候选项在边界框范围的起点处,
则所有后续划分中该框都会在下侧。
在循环体开头和结尾的测试为这两种情况更新了图元数量。
\begin{lstlisting}
`\initcode{Compute cost of all splits for axis to find best}{=}`
int nBelow = 0, nAbove = nPrimitives;
for (int i = 0; i < 2 * nPrimitives; ++i) {
    if (edges[axis][i].`\refvar[BoundEdge::type]{type}{}` == `\refvar{EdgeType::End}{}`) --nAbove;
    `\refvar{Float}{}` edgeT = edges[axis][i].`\refvar[BoundEdge::t]{t}{}`;
    if (edgeT > nodeBounds.`\refvar{pMin}{}`[axis] &&
        edgeT < nodeBounds.`\refvar{pMax}{}`[axis]) {
        `\refcode{Compute cost for split at ith edge}{}`
    }
    if (edges[axis][i].`\refvar[BoundEdge::type]{type}{}` == `\refvar{EdgeType::Start}{}`) ++nBelow;
}
\end{lstlisting}

{\ttfamily belowSA}和{\ttfamily aboveSA}持有两个候选孩子边框的表面积;
通过把六个面的面积加在一起很容易将其算出来。
\begin{lstlisting}
`\initcode{Compute child surface areas for split at edgeT}{=}`
int otherAxis0 = (axis + 1) % 3, otherAxis1 = (axis + 2) % 3;
`\refvar{Float}{}` belowSA = 2 * (d[otherAxis0] * d[otherAxis1] +
                     (edgeT - nodeBounds.`\refvar{pMin}{}`[axis]) * 
                     (d[otherAxis0] + d[otherAxis1]));
`\refvar{Float}{}` aboveSA = 2 * (d[otherAxis0] * d[otherAxis1] +
                     (nodeBounds.`\refvar{pMax}{}`[axis] - edgeT) * 
                     (d[otherAxis0] + d[otherAxis1]));
\end{lstlisting}

有了所有这些信息,就可以计算特定划分的开销了。
\begin{lstlisting}
`\initcode{Compute cost for split at ith edge}{=}`
`\refcode{Compute child surface areas for split at edgeT}{}`
`\refvar{Float}{}` pBelow = belowSA * invTotalSA; 
`\refvar{Float}{}` pAbove = aboveSA * invTotalSA;
`\refvar{Float}{}` eb = (nAbove == 0 || nBelow == 0) ? emptyBonus : 0;
`\refvar{Float}{}` cost = `\refvar{traversalCost}{}` + 
             `\refvar{isectCost}{}` * (1 - eb) * (pBelow * nBelow + pAbove * nAbove);
`\refcode{Update best split if this is lowest cost so far}{}`
\end{lstlisting}

如果为该候选划分计算的开销是目前最好的,则记录该划分的细节。
\begin{lstlisting}
`\initcode{Update best split if this is lowest cost so far}{=}`
if (cost < bestCost)  {
    bestCost = cost;
    bestAxis = axis;
    bestOffset = i;
}
\end{lstlisting}

可能在之前的测试中没有找到可行的划分(\reffig{4.16}展示了一种可能发生的情况)。
这种情况下,沿当前轴不存在可以把该节点划分开的单个候选位置。
这时,依次尝试另外两轴的划分。
(当{\ttfamily retries}等于2时)如果它们也没有找到划分,
则没有有用的方式细化该节点,因为两个孩子都仍会有同样多的重合图元。
当这种条件发生时,所能做的就是放弃并构建一个叶子节点。
\begin{figure}[htbp]
    \centering%LaTeX with PSTricks extensions
%%Creator: Inkscape 1.0.1 (3bc2e813f5, 2020-09-07)
%%Please note this file requires PSTricks extensions
\psset{xunit=.5pt,yunit=.5pt,runit=.5pt}
\begin{pspicture}(239.02999878,189.1000061)
{
\newrgbcolor{curcolor}{0 0 0}
\pscustom[linewidth=1,linecolor=curcolor]
{
\newpath
\moveto(42.56999969,156.40000534)
\lineto(173.01000214,156.40000534)
\lineto(173.01000214,46.560009)
\lineto(42.56999969,46.560009)
\closepath
}
}
{
\newrgbcolor{curcolor}{0 0 0}
\pscustom[linewidth=1,linecolor=curcolor,linestyle=dashed,dash=2]
{
\newpath
\moveto(25.44000053,173.75000572)
\lineto(189.19000053,173.75000572)
\lineto(189.19000053,29.52000999)
\lineto(25.44000053,29.52000999)
\closepath
}
}
{
\newrgbcolor{curcolor}{0 0 0}
\pscustom[linewidth=1,linecolor=curcolor,linestyle=dashed,dash=2]
{
\newpath
\moveto(37.36999893,161.28000641)
\lineto(182.67999649,161.28000641)
\lineto(182.67999649,13.80001068)
\lineto(37.36999893,13.80001068)
\closepath
}
}
{
\newrgbcolor{curcolor}{0 0 0}
\pscustom[linewidth=1,linecolor=curcolor,linestyle=dashed,dash=2]
{
\newpath
\moveto(30.31999969,188.6000061)
\lineto(194.60999298,188.6000061)
\lineto(194.60999298,0.5)
\lineto(30.31999969,0.5)
\closepath
}
}
{
\newrgbcolor{curcolor}{0 0 0}
\pscustom[linewidth=1,linecolor=curcolor,linestyle=dashed,dash=2]
{
\newpath
\moveto(0.5,180.80000591)
\lineto(238.52999878,180.80000591)
\lineto(238.52999878,24.10000896)
\lineto(0.5,24.10000896)
\closepath
}
}
\end{pspicture}

    \caption{如果多个边界框(虚线)如上所示与一个kd树节点(实线)重合,
    则不可能有划分位置能让其两侧的图元比总和更少。}
    \label{fig:4.16}
\end{figure}

也可能最佳划分的开销仍高于根本不划分该节点的开销。
如果它差得多且图元也不太多,就立即创建叶子节点。
否则,{\ttfamily badRefines}保持追踪目前在树的当前节点以上已经做了多少次不良划分。
允许稍微差些的细化是值得的,因为要考虑的图元子集更小后,之后的划分可能会找到更好的结果。
\begin{lstlisting}
`\initcode{Create leaf if no good splits were found}{=}`
if (bestAxis == -1 && retries < 2) {
    ++retries;
    axis = (axis + 1) % 3;
    goto retrySplit;
}
if (bestCost > oldCost) ++badRefines;
if ((bestCost > 4 * oldCost && nPrimitives < 16) || 
    bestAxis == -1 || badRefines == 3) {
    nodes[nodeNum].`\refvar[KdAccelNode::InitLeaf]{InitLeaf}{}`(primNums, nPrimitives, &`\refvar{primitiveIndices}{}`);
    return; 
}
\end{lstlisting}

选好划分位置后,按照之前代码中跟踪{\ttfamily nBelow}和{\ttfamily nAbove}的同样方式,
边界框的边界可用于把图元分为在划分处上方、下方或同在两侧。
注意下面的循环中跳过了数组中的项{\ttfamily bestOffset};
这是必要的,这样边界框被用作划分处的图元不会被错误地分类为同时位于划分处的两侧。
\begin{lstlisting}
`\initcode{Classify primitives with respect to split}{=}`
int n0 = 0, n1 = 0;
for (int i = 0; i < bestOffset; ++i)
    if (edges[bestAxis][i].`\refvar[BoundEdge::type]{type}{}` == `\refvar{EdgeType::Start}{}`)
        prims0[n0++] = edges[bestAxis][i].`\refvar[BoundEdge::primNum]{primNum}{}`;
for (int i = bestOffset + 1; i < 2 * nPrimitives; ++i)
    if (edges[bestAxis][i].`\refvar[BoundEdge::type]{type}{}` == `\refvar{EdgeType::End}{}`)
        prims1[n1++] = edges[bestAxis][i].`\refvar[BoundEdge::primNum]{primNum}{}`;
\end{lstlisting}

回想在kd树节点数组中该节点的“下方”孩子的节点序数是当前节点序数加一。
在递归从树的这一侧返回后,偏移量\refvar{nextFreeNode}{}被用于“上方”孩子。
这里唯一的重要细节是内存{\ttfamily prims0}被直接传入给两个孩子复用,
而{\ttfamily prims1}指针则首先向前推进
\sidenote{译者注:这段内容较难,笔者的理解是:由于\refvar{KdTreeAccel::buildTree}{()}在构建
过程中会原位修改传入的{\ttfamily prims0}和{\ttfamily prims1},所以需要保护现场。
左子树构建完成后,{\ttfamily prims0}的内容就没有用处了,可在右子树构建时被覆盖;
但构建左子树时不能变动还未构建的右子树所需的{\ttfamily prims1},
所以需要挪到新存储位置{\ttfamily prims1 + nPrimitives}。}。
这是必要的,因为当前对\refvar{KdTreeAccel::buildTree}{()}的调用取决于
下文中它首次递归调用\refvar{KdTreeAccel::buildTree}{()}时贮藏的{\ttfamily prims1}值,
毕竟它必须作为参数传给第二次调用。
然而,在首次递归调用立即使用之后就没有相应的必要贮藏{\ttfamily edges}值或贮藏{\ttfamily prims0}了。
\begin{lstlisting}
`\initcode{Recursively initialize children nodes}{=}`
`\refvar{Float}{}` tSplit = edges[bestAxis][bestOffset].`\refvar[BoundEdge::t]{t}{}`;
`\refvar{Bounds3f}{}` bounds0 = nodeBounds, bounds1 = nodeBounds;
bounds0.`\refvar{pMax}{}`[bestAxis] = bounds1.`\refvar{pMin}{}`[bestAxis] = tSplit;
`\refvar[KdTreeAccel::buildTree]{buildTree}{}`(nodeNum + 1, bounds0, allPrimBounds, prims0, n0,
          depth - 1, edges, prims0, prims1 + nPrimitives, badRefines);
int aboveChild = `\refvar{nextFreeNode}{}`;
nodes[nodeNum].`\refvar[KdAccelNode::InitInterior]{InitInterior}{}`(bestAxis, aboveChild, tSplit);
`\refvar[KdTreeAccel::buildTree]{buildTree}{}`(aboveChild, bounds1, allPrimBounds, prims1, n1, 
          depth - 1, edges, prims0, prims1 + nPrimitives, badRefines);
\end{lstlisting}

因此,整数数组{\ttfamily prims1}比数组{\ttfamily prims0}需要
多得多的空间存储最坏情况下重合图元可能的数目,
后者只需要一次处理单个层级的图元。
\begin{lstlisting}
`\refcode{Allocate working memory for kd-tree construction}{+=}\lastcode{Allocateworkingmemoryforkdtreeconstruction}`
std::unique_ptr<int[]> prims0(new int[`\refvar[KdTreeAccel::primitives]{primitives}{}`.size()]);
std::unique_ptr<int[]> prims1(new int[(maxDepth+1) * `\refvar[KdTreeAccel::primitives]{primitives}{}`.size()]);
\end{lstlisting}

\subsection{遍历}\label{sub:遍历2}
\reffig{4.17}展示了光线遍历树的基本过程
\sidenote{译者注:原文中该图题注将左右孩子写反了,已修正。}。
让光线与树的整体边框相交给出了初始的{\ttfamily tMin}和{\ttfamily tMax}值,即图中标出的点。
像本章的\refvar{BVHAccel}{}那样,如果光线错开了整体图元边框,
则该方法可立即返回{\ttfamily false}。
否则,它从根开始下沉到树中。
在每个内部节点处,它确定光线首先进入两个孩子中的哪个并按顺序处理两个孩子。
当光线退出树或者找到最近相交处时遍历结束。
\begin{figure}[htbp]
    \centering%LaTeX with PSTricks extensions
%%Creator: Inkscape 1.0.1 (3bc2e813f5, 2020-09-07)
%%Please note this file requires PSTricks extensions
\psset{xunit=.5pt,yunit=.5pt,runit=.5pt}
\begin{pspicture}(330.70001221,286.44000244)
{
\newrgbcolor{curcolor}{0.80000001 0.80000001 0.80000001}
\pscustom[linestyle=none,fillstyle=solid,fillcolor=curcolor]
{
\newpath
\moveto(21.20999908,134.83000183)
\lineto(74.77999878,134.83000183)
\lineto(74.77999878,24.41000366)
\lineto(21.20999908,24.41000366)
\closepath
}
}
{
\newrgbcolor{curcolor}{0 0 0}
\pscustom[linewidth=1,linecolor=curcolor]
{
\newpath
\moveto(20.32999992,285.94000244)
\lineto(129.44000053,285.94000244)
\lineto(129.44000053,176.83000183)
\lineto(20.32999992,176.83000183)
\closepath
}
}
{
\newrgbcolor{curcolor}{0 0 0}
\pscustom[linewidth=1,linecolor=curcolor]
{
\newpath
\moveto(0.23,192.83000183)
\lineto(145.17999268,266.28000259)
}
}
{
\newrgbcolor{curcolor}{0 0 0}
\pscustom[linestyle=none,fillstyle=solid,fillcolor=curcolor]
{
\newpath
\moveto(143.29,259.15000244)
\lineto(144.6,265.99000244)
\lineto(138.31,268.97000244)
\lineto(152.41,269.94000244)
\closepath
}
}
{
\newrgbcolor{curcolor}{0.65098041 0.65098041 0.65098041}
\pscustom[linestyle=none,fillstyle=solid,fillcolor=curcolor]
{
\newpath
\moveto(144.13,260.93000244)
\lineto(151.24,269.35000244)
\lineto(145.16,266.27000244)
\closepath
}
}
{
\newrgbcolor{curcolor}{0.40000001 0.40000001 0.40000001}
\pscustom[linestyle=none,fillstyle=solid,fillcolor=curcolor]
{
\newpath
\moveto(140.25,268.60000244)
\lineto(151.24,269.35000244)
\lineto(145.16,266.27000244)
\closepath
}
}
{
\newrgbcolor{curcolor}{0 0 0}
\pscustom[linestyle=none,fillstyle=solid,fillcolor=curcolor]
{
\newpath
\moveto(132.99999762,258.48000336)
\curveto(132.99999762,261.76767094)(129.02535629,263.41354512)(126.70090604,261.08909488)
\curveto(124.37645579,258.76464463)(126.02232997,254.7900033)(129.30999756,254.7900033)
\curveto(132.59766514,254.7900033)(134.24353933,258.76464463)(131.91908908,261.08909488)
\curveto(129.59463883,263.41354512)(125.6199975,261.76767094)(125.6199975,258.48000336)
\curveto(125.6199975,255.19233577)(129.59463883,253.54646159)(131.91908908,255.87091184)
\curveto(134.24353933,258.19536209)(132.59766514,262.17000341)(129.30999756,262.17000341)
\curveto(126.02232997,262.17000341)(124.37645579,258.19536209)(126.70090604,255.87091184)
\curveto(129.02535629,253.54646159)(132.99999762,255.19233577)(132.99999762,258.48000336)
\closepath
}
}
{
\newrgbcolor{curcolor}{0 0 0}
\pscustom[linestyle=none,fillstyle=solid,fillcolor=curcolor]
{
\newpath
\moveto(24.12000036,202.95999908)
\curveto(24.12000036,206.24766667)(20.14535904,207.89354085)(17.82090879,205.5690906)
\curveto(15.49645854,203.24464035)(17.14233272,199.26999903)(20.43000031,199.26999903)
\curveto(23.71766789,199.26999903)(25.36354207,203.24464035)(23.03909182,205.5690906)
\curveto(20.71464157,207.89354085)(16.74000025,206.24766667)(16.74000025,202.95999908)
\curveto(16.74000025,199.6723315)(20.71464157,198.02645732)(23.03909182,200.35090757)
\curveto(25.36354207,202.67535782)(23.71766789,206.64999914)(20.43000031,206.64999914)
\curveto(17.14233272,206.64999914)(15.49645854,202.67535782)(17.82090879,200.35090757)
\curveto(20.14535904,198.02645732)(24.12000036,199.6723315)(24.12000036,202.95999908)
\closepath
}
}
{
\newrgbcolor{curcolor}{0 0 0}
\pscustom[linewidth=1,linecolor=curcolor]
{
\newpath
\moveto(192.1000061,285.94000244)
\lineto(301.21000671,285.94000244)
\lineto(301.21000671,176.83000183)
\lineto(192.1000061,176.83000183)
\closepath
}
}
{
\newrgbcolor{curcolor}{0 0 0}
\pscustom[linewidth=1,linecolor=curcolor]
{
\newpath
\moveto(172,192.83000183)
\lineto(316.95999146,266.28000259)
}
}
{
\newrgbcolor{curcolor}{0 0 0}
\pscustom[linestyle=none,fillstyle=solid,fillcolor=curcolor]
{
\newpath
\moveto(315.06,259.15000244)
\lineto(316.38,265.99000244)
\lineto(310.09,268.97000244)
\lineto(324.18,269.94000244)
\closepath
}
}
{
\newrgbcolor{curcolor}{0.65098041 0.65098041 0.65098041}
\pscustom[linestyle=none,fillstyle=solid,fillcolor=curcolor]
{
\newpath
\moveto(315.91,260.93000244)
\lineto(323.01,269.35000244)
\lineto(316.94,266.27000244)
\closepath
}
}
{
\newrgbcolor{curcolor}{0.40000001 0.40000001 0.40000001}
\pscustom[linestyle=none,fillstyle=solid,fillcolor=curcolor]
{
\newpath
\moveto(312.02,268.60000244)
\lineto(323.01,269.35000244)
\lineto(316.94,266.27000244)
\closepath
}
}
{
\newrgbcolor{curcolor}{0 0 0}
\pscustom[linestyle=none,fillstyle=solid,fillcolor=curcolor]
{
\newpath
\moveto(304.7799964,258.48000336)
\curveto(304.7799964,261.76767094)(300.80535507,263.41354512)(298.48090482,261.08909488)
\curveto(296.15645457,258.76464463)(297.80232875,254.7900033)(301.08999634,254.7900033)
\curveto(304.37766392,254.7900033)(306.02353811,258.76464463)(303.69908786,261.08909488)
\curveto(301.37463761,263.41354512)(297.39999628,261.76767094)(297.39999628,258.48000336)
\curveto(297.39999628,255.19233577)(301.37463761,253.54646159)(303.69908786,255.87091184)
\curveto(306.02353811,258.19536209)(304.37766392,262.17000341)(301.08999634,262.17000341)
\curveto(297.80232875,262.17000341)(296.15645457,258.19536209)(298.48090482,255.87091184)
\curveto(300.80535507,253.54646159)(304.7799964,255.19233577)(304.7799964,258.48000336)
\closepath
}
}
{
\newrgbcolor{curcolor}{0 0 0}
\pscustom[linestyle=none,fillstyle=solid,fillcolor=curcolor]
{
\newpath
\moveto(195.90000677,202.95999908)
\curveto(195.90000677,206.24766667)(191.92536545,207.89354085)(189.6009152,205.5690906)
\curveto(187.27646495,203.24464035)(188.92233913,199.26999903)(192.21000671,199.26999903)
\curveto(195.4976743,199.26999903)(197.14354848,203.24464035)(194.81909823,205.5690906)
\curveto(192.49464798,207.89354085)(188.52000666,206.24766667)(188.52000666,202.95999908)
\curveto(188.52000666,199.6723315)(192.49464798,198.02645732)(194.81909823,200.35090757)
\curveto(197.14354848,202.67535782)(195.4976743,206.64999914)(192.21000671,206.64999914)
\curveto(188.92233913,206.64999914)(187.27646495,202.67535782)(189.6009152,200.35090757)
\curveto(191.92536545,198.02645732)(195.90000677,199.6723315)(195.90000677,202.95999908)
\closepath
}
}
{
\newrgbcolor{curcolor}{0 0 0}
\pscustom[linewidth=1,linecolor=curcolor,linestyle=dashed,dash=2]
{
\newpath
\moveto(246,285.59000242)
\lineto(246,176.70999908)
}
}
{
\newrgbcolor{curcolor}{0 0 0}
\pscustom[linestyle=none,fillstyle=solid,fillcolor=curcolor]
{
\newpath
\moveto(249.46999884,230.07000351)
\curveto(249.46999884,233.3576711)(245.49535751,235.00354528)(243.17090726,232.67909503)
\curveto(240.84645701,230.35464478)(242.49233119,226.38000345)(245.77999878,226.38000345)
\curveto(249.06766637,226.38000345)(250.71354055,230.35464478)(248.3890903,232.67909503)
\curveto(246.06464005,235.00354528)(242.08999872,233.3576711)(242.08999872,230.07000351)
\curveto(242.08999872,226.78233592)(246.06464005,225.13646174)(248.3890903,227.46091199)
\curveto(250.71354055,229.78536224)(249.06766637,233.76000357)(245.77999878,233.76000357)
\curveto(242.49233119,233.76000357)(240.84645701,229.78536224)(243.17090726,227.46091199)
\curveto(245.49535751,225.13646174)(249.46999884,226.78233592)(249.46999884,230.07000351)
\closepath
}
}
{
\newrgbcolor{curcolor}{0 0 0}
\pscustom[linewidth=1,linecolor=curcolor]
{
\newpath
\moveto(21.12999916,134.13000488)
\lineto(130.23999977,134.13000488)
\lineto(130.23999977,25.02000427)
\lineto(21.12999916,25.02000427)
\closepath
}
}
{
\newrgbcolor{curcolor}{0 0 0}
\pscustom[linewidth=1,linecolor=curcolor]
{
\newpath
\moveto(1.02999997,41.02000427)
\lineto(145.97999573,114.47000122)
}
}
{
\newrgbcolor{curcolor}{0 0 0}
\pscustom[linestyle=none,fillstyle=solid,fillcolor=curcolor]
{
\newpath
\moveto(144.09,107.34000244)
\lineto(145.4,114.18000244)
\lineto(139.11,117.16000244)
\lineto(153.21,118.13000244)
\closepath
}
}
{
\newrgbcolor{curcolor}{0.65098041 0.65098041 0.65098041}
\pscustom[linestyle=none,fillstyle=solid,fillcolor=curcolor]
{
\newpath
\moveto(144.94,109.12000244)
\lineto(152.04,117.54000244)
\lineto(145.96,114.46000244)
\closepath
}
}
{
\newrgbcolor{curcolor}{0.40000001 0.40000001 0.40000001}
\pscustom[linestyle=none,fillstyle=solid,fillcolor=curcolor]
{
\newpath
\moveto(141.05,116.79000244)
\lineto(152.04,117.54000244)
\lineto(145.96,114.46000244)
\closepath
}
}
{
\newrgbcolor{curcolor}{0 0 0}
\pscustom[linestyle=none,fillstyle=solid,fillcolor=curcolor]
{
\newpath
\moveto(133.80000067,106.66999817)
\curveto(133.80000067,109.95766576)(129.82535934,111.60353994)(127.50090909,109.27908969)
\curveto(125.17645884,106.95463944)(126.82233302,102.97999811)(130.11000061,102.97999811)
\curveto(133.3976682,102.97999811)(135.04354238,106.95463944)(132.71909213,109.27908969)
\curveto(130.39464188,111.60353994)(126.42000055,109.95766576)(126.42000055,106.66999817)
\curveto(126.42000055,103.38233058)(130.39464188,101.7364564)(132.71909213,104.06090665)
\curveto(135.04354238,106.3853569)(133.3976682,110.35999823)(130.11000061,110.35999823)
\curveto(126.82233302,110.35999823)(125.17645884,106.3853569)(127.50090909,104.06090665)
\curveto(129.82535934,101.7364564)(133.80000067,103.38233058)(133.80000067,106.66999817)
\closepath
}
}
{
\newrgbcolor{curcolor}{0 0 0}
\pscustom[linewidth=1,linecolor=curcolor,linestyle=dashed,dash=2]
{
\newpath
\moveto(75.01999664,133.77999878)
\lineto(75.01999664,24.8999939)
}
}
{
\newrgbcolor{curcolor}{0 0 0}
\pscustom[linestyle=none,fillstyle=solid,fillcolor=curcolor]
{
\newpath
\moveto(78.49999762,78.26000977)
\curveto(78.49999762,81.54767735)(74.52535629,83.19355153)(72.20090604,80.86910128)
\curveto(69.87645579,78.54465103)(71.52232997,74.57000971)(74.80999756,74.57000971)
\curveto(78.09766514,74.57000971)(79.74353933,78.54465103)(77.41908908,80.86910128)
\curveto(75.09463883,83.19355153)(71.1199975,81.54767735)(71.1199975,78.26000977)
\curveto(71.1199975,74.97234218)(75.09463883,73.326468)(77.41908908,75.65091825)
\curveto(79.74353933,77.9753685)(78.09766514,81.95000982)(74.80999756,81.95000982)
\curveto(71.52232997,81.95000982)(69.87645579,77.9753685)(72.20090604,75.65091825)
\curveto(74.52535629,73.326468)(78.49999762,74.97234218)(78.49999762,78.26000977)
\closepath
}
}
{
\newrgbcolor{curcolor}{0.80000001 0.80000001 0.80000001}
\pscustom[linestyle=none,fillstyle=solid,fillcolor=curcolor]
{
\newpath
\moveto(192.11000061,134.58000183)
\lineto(245.68000031,134.58000183)
\lineto(245.68000031,24.16000366)
\lineto(192.11000061,24.16000366)
\closepath
}
}
{
\newrgbcolor{curcolor}{0 0 0}
\pscustom[linewidth=1,linecolor=curcolor]
{
\newpath
\moveto(192.02999878,133.86999512)
\lineto(301.13999939,133.86999512)
\lineto(301.13999939,24.75999451)
\lineto(192.02999878,24.75999451)
\closepath
}
}
{
\newrgbcolor{curcolor}{0 0 0}
\pscustom[linewidth=1,linecolor=curcolor]
{
\newpath
\moveto(171.92999268,40.76000977)
\lineto(316.88000488,114.21000671)
}
}
{
\newrgbcolor{curcolor}{0 0 0}
\pscustom[linestyle=none,fillstyle=solid,fillcolor=curcolor]
{
\newpath
\moveto(314.99,107.08000244)
\lineto(316.3,113.92000244)
\lineto(310.01,116.90000244)
\lineto(324.11,117.88000244)
\closepath
}
}
{
\newrgbcolor{curcolor}{0.65098041 0.65098041 0.65098041}
\pscustom[linestyle=none,fillstyle=solid,fillcolor=curcolor]
{
\newpath
\moveto(315.84,108.86000244)
\lineto(322.94,117.28000244)
\lineto(316.87,114.20000244)
\closepath
}
}
{
\newrgbcolor{curcolor}{0.40000001 0.40000001 0.40000001}
\pscustom[linestyle=none,fillstyle=solid,fillcolor=curcolor]
{
\newpath
\moveto(311.95,116.53000244)
\lineto(322.94,117.28000244)
\lineto(316.87,114.20000244)
\closepath
}
}
{
\newrgbcolor{curcolor}{0 0 0}
\pscustom[linestyle=none,fillstyle=solid,fillcolor=curcolor]
{
\newpath
\moveto(304.41000128,107.61000061)
\curveto(304.41000128,110.8976682)(300.43535995,112.54354238)(298.1109097,110.21909213)
\curveto(295.78645945,107.89464188)(297.43233363,103.92000055)(300.72000122,103.92000055)
\curveto(304.00766881,103.92000055)(305.65354299,107.89464188)(303.32909274,110.21909213)
\curveto(301.00464249,112.54354238)(297.03000116,110.8976682)(297.03000116,107.61000061)
\curveto(297.03000116,104.32233302)(301.00464249,102.67645884)(303.32909274,105.00090909)
\curveto(305.65354299,107.32535934)(304.00766881,111.30000067)(300.72000122,111.30000067)
\curveto(297.43233363,111.30000067)(295.78645945,107.32535934)(298.1109097,105.00090909)
\curveto(300.43535995,102.67645884)(304.41000128,104.32233302)(304.41000128,107.61000061)
\closepath
}
}
{
\newrgbcolor{curcolor}{0 0 0}
\pscustom[linewidth=1,linecolor=curcolor,linestyle=dashed,dash=2]
{
\newpath
\moveto(245.92999268,133.52000427)
\lineto(245.92999268,24.6499939)
}
}
{
\newrgbcolor{curcolor}{0 0 0}
\pscustom[linestyle=none,fillstyle=solid,fillcolor=curcolor]
{
\newpath
\moveto(249.40000677,78)
\curveto(249.40000677,81.28766759)(245.42536545,82.93354177)(243.1009152,80.60909152)
\curveto(240.77646495,78.28464127)(242.42233913,74.30999994)(245.71000671,74.30999994)
\curveto(248.9976743,74.30999994)(250.64354848,78.28464127)(248.31909823,80.60909152)
\curveto(245.99464798,82.93354177)(242.02000666,81.28766759)(242.02000666,78)
\curveto(242.02000666,74.71233241)(245.99464798,73.06645823)(248.31909823,75.39090848)
\curveto(250.64354848,77.71535873)(248.9976743,81.69000006)(245.71000671,81.69000006)
\curveto(242.42233913,81.69000006)(240.77646495,77.71535873)(243.1009152,75.39090848)
\curveto(245.42536545,73.06645823)(249.40000677,74.71233241)(249.40000677,78)
\closepath
}
}
{
\newrgbcolor{curcolor}{0 0 0}
\pscustom[linewidth=1,linecolor=curcolor,linestyle=dashed,dash=2]
{
\newpath
\moveto(245.24000549,95.5)
\lineto(299.58999634,95.5)
}
}
{
\newrgbcolor{curcolor}{0 0 0}
\pscustom[linestyle=none,fillstyle=solid,fillcolor=curcolor]
{
\newpath
\moveto(282.98998785,95.26000977)
\curveto(282.98998785,98.54767735)(279.01534652,100.19355153)(276.69089627,97.86910128)
\curveto(274.36644602,95.54465103)(276.01232021,91.57000971)(279.29998779,91.57000971)
\curveto(282.58765538,91.57000971)(284.23352956,95.54465103)(281.90907931,97.86910128)
\curveto(279.58462906,100.19355153)(275.60998774,98.54767735)(275.60998774,95.26000977)
\curveto(275.60998774,91.97234218)(279.58462906,90.326468)(281.90907931,92.65091825)
\curveto(284.23352956,94.9753685)(282.58765538,98.95000982)(279.29998779,98.95000982)
\curveto(276.01232021,98.95000982)(274.36644602,94.9753685)(276.69089627,92.65091825)
\curveto(279.01534652,90.326468)(282.98998785,91.97234218)(282.98998785,95.26000977)
\closepath
}
}
{
\newrgbcolor{curcolor}{0 0 0}
\pscustom[linestyle=none,fillstyle=solid,fillcolor=curcolor]
{
\newpath
\moveto(28.41202488,196.26353154)
\lineto(29.91202488,196.26353154)
\curveto(30.23702488,196.26353154)(30.41202488,196.26353154)(30.41202488,196.58853154)
\curveto(30.41202488,196.76353154)(30.23702488,196.76353154)(29.96202488,196.76353154)
\lineto(28.56202488,196.76353154)
\curveto(29.13702488,199.03853154)(29.21202488,199.33853154)(29.21202488,199.43853154)
\curveto(29.21202488,199.71353154)(29.01202488,199.86353154)(28.73702488,199.86353154)
\curveto(28.68702488,199.86353154)(28.23702488,199.86353154)(28.11202488,199.28853154)
\lineto(27.48702488,196.76353154)
\lineto(25.98702488,196.76353154)
\curveto(25.66202488,196.76353154)(25.51202488,196.76353154)(25.51202488,196.46353154)
\curveto(25.51202488,196.26353154)(25.63702488,196.26353154)(25.96202488,196.26353154)
\lineto(27.36202488,196.26353154)
\curveto(26.21202488,191.73853154)(26.13702488,191.46353154)(26.13702488,191.18853154)
\curveto(26.13702488,190.31353154)(26.73702488,189.71353154)(27.61202488,189.71353154)
\curveto(29.23702488,189.71353154)(30.13702488,192.03853154)(30.13702488,192.16353154)
\curveto(30.13702488,192.33853154)(30.01202488,192.33853154)(29.96202488,192.33853154)
\curveto(29.81202488,192.33853154)(29.78702488,192.28853154)(29.71202488,192.11353154)
\curveto(29.03702488,190.43853154)(28.18702488,190.06353154)(27.63702488,190.06353154)
\curveto(27.31202488,190.06353154)(27.13702488,190.26353154)(27.13702488,190.78853154)
\curveto(27.13702488,191.18853154)(27.18702488,191.28853154)(27.23702488,191.56353154)
\closepath
\moveto(28.41202488,196.26353154)
}
}
{
\newrgbcolor{curcolor}{0 0 0}
\pscustom[linestyle=none,fillstyle=solid,fillcolor=curcolor]
{
\newpath
\moveto(40.0688242,190.87312627)
\curveto(40.0688242,191.84812627)(39.5688242,192.42312627)(38.3938242,192.42312627)
\curveto(37.4938242,192.42312627)(36.8938242,191.94812627)(36.5938242,191.37312627)
\curveto(36.3688242,192.17312627)(35.7688242,192.42312627)(34.9688242,192.42312627)
\curveto(34.0438242,192.42312627)(33.4688242,191.92312627)(33.1438242,191.32312627)
\lineto(33.1438242,192.42312627)
\lineto(31.4938242,192.29812627)
\lineto(31.4938242,191.89812627)
\curveto(32.2438242,191.89812627)(32.3438242,191.82312627)(32.3438242,191.27312627)
\lineto(32.3438242,188.37312627)
\curveto(32.3438242,187.89812627)(32.2188242,187.89812627)(31.4938242,187.89812627)
\lineto(31.4938242,187.49812627)
\curveto(31.5188242,187.49812627)(32.2938242,187.54812627)(32.7688242,187.54812627)
\curveto(33.1688242,187.54812627)(33.9438242,187.49812627)(34.0438242,187.49812627)
\lineto(34.0438242,187.89812627)
\curveto(33.3188242,187.89812627)(33.2188242,187.89812627)(33.2188242,188.37312627)
\lineto(33.2188242,190.39812627)
\curveto(33.2188242,191.54812627)(34.1438242,192.09812627)(34.8688242,192.09812627)
\curveto(35.6438242,192.09812627)(35.7438242,191.49812627)(35.7438242,190.92312627)
\lineto(35.7438242,188.37312627)
\curveto(35.7438242,187.89812627)(35.6438242,187.89812627)(34.9188242,187.89812627)
\lineto(34.9188242,187.49812627)
\curveto(34.9438242,187.49812627)(35.7188242,187.54812627)(36.1938242,187.54812627)
\curveto(36.5938242,187.54812627)(37.3688242,187.49812627)(37.4688242,187.49812627)
\lineto(37.4688242,187.89812627)
\curveto(36.7438242,187.89812627)(36.6438242,187.89812627)(36.6438242,188.37312627)
\lineto(36.6438242,190.39812627)
\curveto(36.6438242,191.54812627)(37.5688242,192.09812627)(38.2938242,192.09812627)
\curveto(39.0688242,192.09812627)(39.1688242,191.49812627)(39.1688242,190.92312627)
\lineto(39.1688242,188.37312627)
\curveto(39.1688242,187.89812627)(39.0688242,187.89812627)(38.3438242,187.89812627)
\lineto(38.3438242,187.49812627)
\curveto(38.3688242,187.49812627)(39.1438242,187.54812627)(39.6188242,187.54812627)
\curveto(40.0188242,187.54812627)(40.7938242,187.49812627)(40.8938242,187.49812627)
\lineto(40.8938242,187.89812627)
\curveto(40.1688242,187.89812627)(40.0688242,187.89812627)(40.0688242,188.37312627)
\closepath
\moveto(40.0688242,190.87312627)
}
}
{
\newrgbcolor{curcolor}{0 0 0}
\pscustom[linestyle=none,fillstyle=solid,fillcolor=curcolor]
{
\newpath
\moveto(43.71567479,194.37312627)
\curveto(43.71567479,194.69812627)(43.44067479,195.02312627)(43.06567479,195.02312627)
\curveto(42.74067479,195.02312627)(42.44067479,194.74812627)(42.44067479,194.37312627)
\curveto(42.44067479,193.97312627)(42.76567479,193.72312627)(43.06567479,193.72312627)
\curveto(43.44067479,193.72312627)(43.71567479,193.99812627)(43.71567479,194.37312627)
\closepath
\moveto(42.01567479,192.29812627)
\lineto(42.01567479,191.89812627)
\curveto(42.71567479,191.89812627)(42.81567479,191.82312627)(42.81567479,191.27312627)
\lineto(42.81567479,188.37312627)
\curveto(42.81567479,187.89812627)(42.71567479,187.89812627)(41.99067479,187.89812627)
\lineto(41.99067479,187.49812627)
\curveto(42.01567479,187.49812627)(42.79067479,187.54812627)(43.24067479,187.54812627)
\curveto(43.64067479,187.54812627)(44.04067479,187.52312627)(44.41567479,187.49812627)
\lineto(44.41567479,187.89812627)
\curveto(43.76567479,187.89812627)(43.66567479,187.89812627)(43.66567479,188.37312627)
\lineto(43.66567479,192.42312627)
\closepath
\moveto(42.01567479,192.29812627)
}
}
{
\newrgbcolor{curcolor}{0 0 0}
\pscustom[linestyle=none,fillstyle=solid,fillcolor=curcolor]
{
\newpath
\moveto(50.72418309,190.87312627)
\curveto(50.72418309,191.84812627)(50.24918309,192.42312627)(49.04918309,192.42312627)
\curveto(48.12418309,192.42312627)(47.54918309,191.92312627)(47.22418309,191.32312627)
\lineto(47.22418309,192.42312627)
\lineto(45.57418309,192.29812627)
\lineto(45.57418309,191.89812627)
\curveto(46.32418309,191.89812627)(46.42418309,191.82312627)(46.42418309,191.27312627)
\lineto(46.42418309,188.37312627)
\curveto(46.42418309,187.89812627)(46.29918309,187.89812627)(45.57418309,187.89812627)
\lineto(45.57418309,187.49812627)
\curveto(45.59918309,187.49812627)(46.37418309,187.54812627)(46.84918309,187.54812627)
\curveto(47.24918309,187.54812627)(48.02418309,187.49812627)(48.12418309,187.49812627)
\lineto(48.12418309,187.89812627)
\curveto(47.39918309,187.89812627)(47.29918309,187.89812627)(47.29918309,188.37312627)
\lineto(47.29918309,190.39812627)
\curveto(47.29918309,191.54812627)(48.22418309,192.09812627)(48.94918309,192.09812627)
\curveto(49.72418309,192.09812627)(49.82418309,191.49812627)(49.82418309,190.92312627)
\lineto(49.82418309,188.37312627)
\curveto(49.82418309,187.89812627)(49.72418309,187.89812627)(48.99918309,187.89812627)
\lineto(48.99918309,187.49812627)
\curveto(49.02418309,187.49812627)(49.79918309,187.54812627)(50.27418309,187.54812627)
\curveto(50.67418309,187.54812627)(51.44918309,187.49812627)(51.54918309,187.49812627)
\lineto(51.54918309,187.89812627)
\curveto(50.82418309,187.89812627)(50.72418309,187.89812627)(50.72418309,188.37312627)
\closepath
\moveto(50.72418309,190.87312627)
}
}
{
\newrgbcolor{curcolor}{0 0 0}
\pscustom[linestyle=none,fillstyle=solid,fillcolor=curcolor]
{
\newpath
\moveto(252.29177488,71.73215054)
\lineto(253.79177488,71.73215054)
\curveto(254.11677488,71.73215054)(254.29177488,71.73215054)(254.29177488,72.05715054)
\curveto(254.29177488,72.23215054)(254.11677488,72.23215054)(253.84177488,72.23215054)
\lineto(252.44177488,72.23215054)
\curveto(253.01677488,74.50715054)(253.09177488,74.80715054)(253.09177488,74.90715054)
\curveto(253.09177488,75.18215054)(252.89177488,75.33215054)(252.61677488,75.33215054)
\curveto(252.56677488,75.33215054)(252.11677488,75.33215054)(251.99177488,74.75715054)
\lineto(251.36677488,72.23215054)
\lineto(249.86677488,72.23215054)
\curveto(249.54177488,72.23215054)(249.39177488,72.23215054)(249.39177488,71.93215054)
\curveto(249.39177488,71.73215054)(249.51677488,71.73215054)(249.84177488,71.73215054)
\lineto(251.24177488,71.73215054)
\curveto(250.09177488,67.20715054)(250.01677488,66.93215054)(250.01677488,66.65715054)
\curveto(250.01677488,65.78215054)(250.61677488,65.18215054)(251.49177488,65.18215054)
\curveto(253.11677488,65.18215054)(254.01677488,67.50715054)(254.01677488,67.63215054)
\curveto(254.01677488,67.80715054)(253.89177488,67.80715054)(253.84177488,67.80715054)
\curveto(253.69177488,67.80715054)(253.66677488,67.75715054)(253.59177488,67.58215054)
\curveto(252.91677488,65.90715054)(252.06677488,65.53215054)(251.51677488,65.53215054)
\curveto(251.19177488,65.53215054)(251.01677488,65.73215054)(251.01677488,66.25715054)
\curveto(251.01677488,66.65715054)(251.06677488,66.75715054)(251.11677488,67.03215054)
\closepath
\moveto(252.29177488,71.73215054)
}
}
{
\newrgbcolor{curcolor}{0 0 0}
\pscustom[linestyle=none,fillstyle=solid,fillcolor=curcolor]
{
\newpath
\moveto(263.9485742,66.34174527)
\curveto(263.9485742,67.31674527)(263.4485742,67.89174527)(262.2735742,67.89174527)
\curveto(261.3735742,67.89174527)(260.7735742,67.41674527)(260.4735742,66.84174527)
\curveto(260.2485742,67.64174527)(259.6485742,67.89174527)(258.8485742,67.89174527)
\curveto(257.9235742,67.89174527)(257.3485742,67.39174527)(257.0235742,66.79174527)
\lineto(257.0235742,67.89174527)
\lineto(255.3735742,67.76674527)
\lineto(255.3735742,67.36674527)
\curveto(256.1235742,67.36674527)(256.2235742,67.29174527)(256.2235742,66.74174527)
\lineto(256.2235742,63.84174527)
\curveto(256.2235742,63.36674527)(256.0985742,63.36674527)(255.3735742,63.36674527)
\lineto(255.3735742,62.96674527)
\curveto(255.3985742,62.96674527)(256.1735742,63.01674527)(256.6485742,63.01674527)
\curveto(257.0485742,63.01674527)(257.8235742,62.96674527)(257.9235742,62.96674527)
\lineto(257.9235742,63.36674527)
\curveto(257.1985742,63.36674527)(257.0985742,63.36674527)(257.0985742,63.84174527)
\lineto(257.0985742,65.86674527)
\curveto(257.0985742,67.01674527)(258.0235742,67.56674527)(258.7485742,67.56674527)
\curveto(259.5235742,67.56674527)(259.6235742,66.96674527)(259.6235742,66.39174527)
\lineto(259.6235742,63.84174527)
\curveto(259.6235742,63.36674527)(259.5235742,63.36674527)(258.7985742,63.36674527)
\lineto(258.7985742,62.96674527)
\curveto(258.8235742,62.96674527)(259.5985742,63.01674527)(260.0735742,63.01674527)
\curveto(260.4735742,63.01674527)(261.2485742,62.96674527)(261.3485742,62.96674527)
\lineto(261.3485742,63.36674527)
\curveto(260.6235742,63.36674527)(260.5235742,63.36674527)(260.5235742,63.84174527)
\lineto(260.5235742,65.86674527)
\curveto(260.5235742,67.01674527)(261.4485742,67.56674527)(262.1735742,67.56674527)
\curveto(262.9485742,67.56674527)(263.0485742,66.96674527)(263.0485742,66.39174527)
\lineto(263.0485742,63.84174527)
\curveto(263.0485742,63.36674527)(262.9485742,63.36674527)(262.2235742,63.36674527)
\lineto(262.2235742,62.96674527)
\curveto(262.2485742,62.96674527)(263.0235742,63.01674527)(263.4985742,63.01674527)
\curveto(263.8985742,63.01674527)(264.6735742,62.96674527)(264.7735742,62.96674527)
\lineto(264.7735742,63.36674527)
\curveto(264.0485742,63.36674527)(263.9485742,63.36674527)(263.9485742,63.84174527)
\closepath
\moveto(263.9485742,66.34174527)
}
}
{
\newrgbcolor{curcolor}{0 0 0}
\pscustom[linestyle=none,fillstyle=solid,fillcolor=curcolor]
{
\newpath
\moveto(267.59542479,69.84174527)
\curveto(267.59542479,70.16674527)(267.32042479,70.49174527)(266.94542479,70.49174527)
\curveto(266.62042479,70.49174527)(266.32042479,70.21674527)(266.32042479,69.84174527)
\curveto(266.32042479,69.44174527)(266.64542479,69.19174527)(266.94542479,69.19174527)
\curveto(267.32042479,69.19174527)(267.59542479,69.46674527)(267.59542479,69.84174527)
\closepath
\moveto(265.89542479,67.76674527)
\lineto(265.89542479,67.36674527)
\curveto(266.59542479,67.36674527)(266.69542479,67.29174527)(266.69542479,66.74174527)
\lineto(266.69542479,63.84174527)
\curveto(266.69542479,63.36674527)(266.59542479,63.36674527)(265.87042479,63.36674527)
\lineto(265.87042479,62.96674527)
\curveto(265.89542479,62.96674527)(266.67042479,63.01674527)(267.12042479,63.01674527)
\curveto(267.52042479,63.01674527)(267.92042479,62.99174527)(268.29542479,62.96674527)
\lineto(268.29542479,63.36674527)
\curveto(267.64542479,63.36674527)(267.54542479,63.36674527)(267.54542479,63.84174527)
\lineto(267.54542479,67.89174527)
\closepath
\moveto(265.89542479,67.76674527)
}
}
{
\newrgbcolor{curcolor}{0 0 0}
\pscustom[linestyle=none,fillstyle=solid,fillcolor=curcolor]
{
\newpath
\moveto(274.60393309,66.34174527)
\curveto(274.60393309,67.31674527)(274.12893309,67.89174527)(272.92893309,67.89174527)
\curveto(272.00393309,67.89174527)(271.42893309,67.39174527)(271.10393309,66.79174527)
\lineto(271.10393309,67.89174527)
\lineto(269.45393309,67.76674527)
\lineto(269.45393309,67.36674527)
\curveto(270.20393309,67.36674527)(270.30393309,67.29174527)(270.30393309,66.74174527)
\lineto(270.30393309,63.84174527)
\curveto(270.30393309,63.36674527)(270.17893309,63.36674527)(269.45393309,63.36674527)
\lineto(269.45393309,62.96674527)
\curveto(269.47893309,62.96674527)(270.25393309,63.01674527)(270.72893309,63.01674527)
\curveto(271.12893309,63.01674527)(271.90393309,62.96674527)(272.00393309,62.96674527)
\lineto(272.00393309,63.36674527)
\curveto(271.27893309,63.36674527)(271.17893309,63.36674527)(271.17893309,63.84174527)
\lineto(271.17893309,65.86674527)
\curveto(271.17893309,67.01674527)(272.10393309,67.56674527)(272.82893309,67.56674527)
\curveto(273.60393309,67.56674527)(273.70393309,66.96674527)(273.70393309,66.39174527)
\lineto(273.70393309,63.84174527)
\curveto(273.70393309,63.36674527)(273.60393309,63.36674527)(272.87893309,63.36674527)
\lineto(272.87893309,62.96674527)
\curveto(272.90393309,62.96674527)(273.67893309,63.01674527)(274.15393309,63.01674527)
\curveto(274.55393309,63.01674527)(275.32893309,62.96674527)(275.42893309,62.96674527)
\lineto(275.42893309,63.36674527)
\curveto(274.70393309,63.36674527)(274.60393309,63.36674527)(274.60393309,63.84174527)
\closepath
\moveto(274.60393309,66.34174527)
}
}
{
\newrgbcolor{curcolor}{0 0 0}
\pscustom[linestyle=none,fillstyle=solid,fillcolor=curcolor]
{
\newpath
\moveto(83.02134388,72.19337854)
\lineto(84.52134388,72.19337854)
\curveto(84.84634388,72.19337854)(85.02134388,72.19337854)(85.02134388,72.51837854)
\curveto(85.02134388,72.69337854)(84.84634388,72.69337854)(84.57134388,72.69337854)
\lineto(83.17134388,72.69337854)
\curveto(83.74634388,74.96837854)(83.82134388,75.26837854)(83.82134388,75.36837854)
\curveto(83.82134388,75.64337854)(83.62134388,75.79337854)(83.34634388,75.79337854)
\curveto(83.29634388,75.79337854)(82.84634388,75.79337854)(82.72134388,75.21837854)
\lineto(82.09634388,72.69337854)
\lineto(80.59634388,72.69337854)
\curveto(80.27134388,72.69337854)(80.12134388,72.69337854)(80.12134388,72.39337854)
\curveto(80.12134388,72.19337854)(80.24634388,72.19337854)(80.57134388,72.19337854)
\lineto(81.97134388,72.19337854)
\curveto(80.82134388,67.66837854)(80.74634388,67.39337854)(80.74634388,67.11837854)
\curveto(80.74634388,66.24337854)(81.34634388,65.64337854)(82.22134388,65.64337854)
\curveto(83.84634388,65.64337854)(84.74634388,67.96837854)(84.74634388,68.09337854)
\curveto(84.74634388,68.26837854)(84.62134388,68.26837854)(84.57134388,68.26837854)
\curveto(84.42134388,68.26837854)(84.39634388,68.21837854)(84.32134388,68.04337854)
\curveto(83.64634388,66.36837854)(82.79634388,65.99337854)(82.24634388,65.99337854)
\curveto(81.92134388,65.99337854)(81.74634388,66.19337854)(81.74634388,66.71837854)
\curveto(81.74634388,67.11837854)(81.79634388,67.21837854)(81.84634388,67.49337854)
\closepath
\moveto(83.02134388,72.19337854)
}
}
{
\newrgbcolor{curcolor}{0 0 0}
\pscustom[linestyle=none,fillstyle=solid,fillcolor=curcolor]
{
\newpath
\moveto(94.6781432,66.80297327)
\curveto(94.6781432,67.77797327)(94.1781432,68.35297327)(93.0031432,68.35297327)
\curveto(92.1031432,68.35297327)(91.5031432,67.87797327)(91.2031432,67.30297327)
\curveto(90.9781432,68.10297327)(90.3781432,68.35297327)(89.5781432,68.35297327)
\curveto(88.6531432,68.35297327)(88.0781432,67.85297327)(87.7531432,67.25297327)
\lineto(87.7531432,68.35297327)
\lineto(86.1031432,68.22797327)
\lineto(86.1031432,67.82797327)
\curveto(86.8531432,67.82797327)(86.9531432,67.75297327)(86.9531432,67.20297327)
\lineto(86.9531432,64.30297327)
\curveto(86.9531432,63.82797327)(86.8281432,63.82797327)(86.1031432,63.82797327)
\lineto(86.1031432,63.42797327)
\curveto(86.1281432,63.42797327)(86.9031432,63.47797327)(87.3781432,63.47797327)
\curveto(87.7781432,63.47797327)(88.5531432,63.42797327)(88.6531432,63.42797327)
\lineto(88.6531432,63.82797327)
\curveto(87.9281432,63.82797327)(87.8281432,63.82797327)(87.8281432,64.30297327)
\lineto(87.8281432,66.32797327)
\curveto(87.8281432,67.47797327)(88.7531432,68.02797327)(89.4781432,68.02797327)
\curveto(90.2531432,68.02797327)(90.3531432,67.42797327)(90.3531432,66.85297327)
\lineto(90.3531432,64.30297327)
\curveto(90.3531432,63.82797327)(90.2531432,63.82797327)(89.5281432,63.82797327)
\lineto(89.5281432,63.42797327)
\curveto(89.5531432,63.42797327)(90.3281432,63.47797327)(90.8031432,63.47797327)
\curveto(91.2031432,63.47797327)(91.9781432,63.42797327)(92.0781432,63.42797327)
\lineto(92.0781432,63.82797327)
\curveto(91.3531432,63.82797327)(91.2531432,63.82797327)(91.2531432,64.30297327)
\lineto(91.2531432,66.32797327)
\curveto(91.2531432,67.47797327)(92.1781432,68.02797327)(92.9031432,68.02797327)
\curveto(93.6781432,68.02797327)(93.7781432,67.42797327)(93.7781432,66.85297327)
\lineto(93.7781432,64.30297327)
\curveto(93.7781432,63.82797327)(93.6781432,63.82797327)(92.9531432,63.82797327)
\lineto(92.9531432,63.42797327)
\curveto(92.9781432,63.42797327)(93.7531432,63.47797327)(94.2281432,63.47797327)
\curveto(94.6281432,63.47797327)(95.4031432,63.42797327)(95.5031432,63.42797327)
\lineto(95.5031432,63.82797327)
\curveto(94.7781432,63.82797327)(94.6781432,63.82797327)(94.6781432,64.30297327)
\closepath
\moveto(94.6781432,66.80297327)
}
}
{
\newrgbcolor{curcolor}{0 0 0}
\pscustom[linestyle=none,fillstyle=solid,fillcolor=curcolor]
{
\newpath
\moveto(98.32499379,70.30297327)
\curveto(98.32499379,70.62797327)(98.04999379,70.95297327)(97.67499379,70.95297327)
\curveto(97.34999379,70.95297327)(97.04999379,70.67797327)(97.04999379,70.30297327)
\curveto(97.04999379,69.90297327)(97.37499379,69.65297327)(97.67499379,69.65297327)
\curveto(98.04999379,69.65297327)(98.32499379,69.92797327)(98.32499379,70.30297327)
\closepath
\moveto(96.62499379,68.22797327)
\lineto(96.62499379,67.82797327)
\curveto(97.32499379,67.82797327)(97.42499379,67.75297327)(97.42499379,67.20297327)
\lineto(97.42499379,64.30297327)
\curveto(97.42499379,63.82797327)(97.32499379,63.82797327)(96.59999379,63.82797327)
\lineto(96.59999379,63.42797327)
\curveto(96.62499379,63.42797327)(97.39999379,63.47797327)(97.84999379,63.47797327)
\curveto(98.24999379,63.47797327)(98.64999379,63.45297327)(99.02499379,63.42797327)
\lineto(99.02499379,63.82797327)
\curveto(98.37499379,63.82797327)(98.27499379,63.82797327)(98.27499379,64.30297327)
\lineto(98.27499379,68.35297327)
\closepath
\moveto(96.62499379,68.22797327)
}
}
{
\newrgbcolor{curcolor}{0 0 0}
\pscustom[linestyle=none,fillstyle=solid,fillcolor=curcolor]
{
\newpath
\moveto(105.33350209,66.80297327)
\curveto(105.33350209,67.77797327)(104.85850209,68.35297327)(103.65850209,68.35297327)
\curveto(102.73350209,68.35297327)(102.15850209,67.85297327)(101.83350209,67.25297327)
\lineto(101.83350209,68.35297327)
\lineto(100.18350209,68.22797327)
\lineto(100.18350209,67.82797327)
\curveto(100.93350209,67.82797327)(101.03350209,67.75297327)(101.03350209,67.20297327)
\lineto(101.03350209,64.30297327)
\curveto(101.03350209,63.82797327)(100.90850209,63.82797327)(100.18350209,63.82797327)
\lineto(100.18350209,63.42797327)
\curveto(100.20850209,63.42797327)(100.98350209,63.47797327)(101.45850209,63.47797327)
\curveto(101.85850209,63.47797327)(102.63350209,63.42797327)(102.73350209,63.42797327)
\lineto(102.73350209,63.82797327)
\curveto(102.00850209,63.82797327)(101.90850209,63.82797327)(101.90850209,64.30297327)
\lineto(101.90850209,66.32797327)
\curveto(101.90850209,67.47797327)(102.83350209,68.02797327)(103.55850209,68.02797327)
\curveto(104.33350209,68.02797327)(104.43350209,67.42797327)(104.43350209,66.85297327)
\lineto(104.43350209,64.30297327)
\curveto(104.43350209,63.82797327)(104.33350209,63.82797327)(103.60850209,63.82797327)
\lineto(103.60850209,63.42797327)
\curveto(103.63350209,63.42797327)(104.40850209,63.47797327)(104.88350209,63.47797327)
\curveto(105.28350209,63.47797327)(106.05850209,63.42797327)(106.15850209,63.42797327)
\lineto(106.15850209,63.82797327)
\curveto(105.43350209,63.82797327)(105.33350209,63.82797327)(105.33350209,64.30297327)
\closepath
\moveto(105.33350209,66.80297327)
}
}
{
\newrgbcolor{curcolor}{0 0 0}
\pscustom[linestyle=none,fillstyle=solid,fillcolor=curcolor]
{
\newpath
\moveto(200.63431488,197.64721354)
\lineto(202.13431488,197.64721354)
\curveto(202.45931488,197.64721354)(202.63431488,197.64721354)(202.63431488,197.97221354)
\curveto(202.63431488,198.14721354)(202.45931488,198.14721354)(202.18431488,198.14721354)
\lineto(200.78431488,198.14721354)
\curveto(201.35931488,200.42221354)(201.43431488,200.72221354)(201.43431488,200.82221354)
\curveto(201.43431488,201.09721354)(201.23431488,201.24721354)(200.95931488,201.24721354)
\curveto(200.90931488,201.24721354)(200.45931488,201.24721354)(200.33431488,200.67221354)
\lineto(199.70931488,198.14721354)
\lineto(198.20931488,198.14721354)
\curveto(197.88431488,198.14721354)(197.73431488,198.14721354)(197.73431488,197.84721354)
\curveto(197.73431488,197.64721354)(197.85931488,197.64721354)(198.18431488,197.64721354)
\lineto(199.58431488,197.64721354)
\curveto(198.43431488,193.12221354)(198.35931488,192.84721354)(198.35931488,192.57221354)
\curveto(198.35931488,191.69721354)(198.95931488,191.09721354)(199.83431488,191.09721354)
\curveto(201.45931488,191.09721354)(202.35931488,193.42221354)(202.35931488,193.54721354)
\curveto(202.35931488,193.72221354)(202.23431488,193.72221354)(202.18431488,193.72221354)
\curveto(202.03431488,193.72221354)(202.00931488,193.67221354)(201.93431488,193.49721354)
\curveto(201.25931488,191.82221354)(200.40931488,191.44721354)(199.85931488,191.44721354)
\curveto(199.53431488,191.44721354)(199.35931488,191.64721354)(199.35931488,192.17221354)
\curveto(199.35931488,192.57221354)(199.40931488,192.67221354)(199.45931488,192.94721354)
\closepath
\moveto(200.63431488,197.64721354)
}
}
{
\newrgbcolor{curcolor}{0 0 0}
\pscustom[linestyle=none,fillstyle=solid,fillcolor=curcolor]
{
\newpath
\moveto(212.2911142,192.25680827)
\curveto(212.2911142,193.23180827)(211.7911142,193.80680827)(210.6161142,193.80680827)
\curveto(209.7161142,193.80680827)(209.1161142,193.33180827)(208.8161142,192.75680827)
\curveto(208.5911142,193.55680827)(207.9911142,193.80680827)(207.1911142,193.80680827)
\curveto(206.2661142,193.80680827)(205.6911142,193.30680827)(205.3661142,192.70680827)
\lineto(205.3661142,193.80680827)
\lineto(203.7161142,193.68180827)
\lineto(203.7161142,193.28180827)
\curveto(204.4661142,193.28180827)(204.5661142,193.20680827)(204.5661142,192.65680827)
\lineto(204.5661142,189.75680827)
\curveto(204.5661142,189.28180827)(204.4411142,189.28180827)(203.7161142,189.28180827)
\lineto(203.7161142,188.88180827)
\curveto(203.7411142,188.88180827)(204.5161142,188.93180827)(204.9911142,188.93180827)
\curveto(205.3911142,188.93180827)(206.1661142,188.88180827)(206.2661142,188.88180827)
\lineto(206.2661142,189.28180827)
\curveto(205.5411142,189.28180827)(205.4411142,189.28180827)(205.4411142,189.75680827)
\lineto(205.4411142,191.78180827)
\curveto(205.4411142,192.93180827)(206.3661142,193.48180827)(207.0911142,193.48180827)
\curveto(207.8661142,193.48180827)(207.9661142,192.88180827)(207.9661142,192.30680827)
\lineto(207.9661142,189.75680827)
\curveto(207.9661142,189.28180827)(207.8661142,189.28180827)(207.1411142,189.28180827)
\lineto(207.1411142,188.88180827)
\curveto(207.1661142,188.88180827)(207.9411142,188.93180827)(208.4161142,188.93180827)
\curveto(208.8161142,188.93180827)(209.5911142,188.88180827)(209.6911142,188.88180827)
\lineto(209.6911142,189.28180827)
\curveto(208.9661142,189.28180827)(208.8661142,189.28180827)(208.8661142,189.75680827)
\lineto(208.8661142,191.78180827)
\curveto(208.8661142,192.93180827)(209.7911142,193.48180827)(210.5161142,193.48180827)
\curveto(211.2911142,193.48180827)(211.3911142,192.88180827)(211.3911142,192.30680827)
\lineto(211.3911142,189.75680827)
\curveto(211.3911142,189.28180827)(211.2911142,189.28180827)(210.5661142,189.28180827)
\lineto(210.5661142,188.88180827)
\curveto(210.5911142,188.88180827)(211.3661142,188.93180827)(211.8411142,188.93180827)
\curveto(212.2411142,188.93180827)(213.0161142,188.88180827)(213.1161142,188.88180827)
\lineto(213.1161142,189.28180827)
\curveto(212.3911142,189.28180827)(212.2911142,189.28180827)(212.2911142,189.75680827)
\closepath
\moveto(212.2911142,192.25680827)
}
}
{
\newrgbcolor{curcolor}{0 0 0}
\pscustom[linestyle=none,fillstyle=solid,fillcolor=curcolor]
{
\newpath
\moveto(215.93796479,195.75680827)
\curveto(215.93796479,196.08180827)(215.66296479,196.40680827)(215.28796479,196.40680827)
\curveto(214.96296479,196.40680827)(214.66296479,196.13180827)(214.66296479,195.75680827)
\curveto(214.66296479,195.35680827)(214.98796479,195.10680827)(215.28796479,195.10680827)
\curveto(215.66296479,195.10680827)(215.93796479,195.38180827)(215.93796479,195.75680827)
\closepath
\moveto(214.23796479,193.68180827)
\lineto(214.23796479,193.28180827)
\curveto(214.93796479,193.28180827)(215.03796479,193.20680827)(215.03796479,192.65680827)
\lineto(215.03796479,189.75680827)
\curveto(215.03796479,189.28180827)(214.93796479,189.28180827)(214.21296479,189.28180827)
\lineto(214.21296479,188.88180827)
\curveto(214.23796479,188.88180827)(215.01296479,188.93180827)(215.46296479,188.93180827)
\curveto(215.86296479,188.93180827)(216.26296479,188.90680827)(216.63796479,188.88180827)
\lineto(216.63796479,189.28180827)
\curveto(215.98796479,189.28180827)(215.88796479,189.28180827)(215.88796479,189.75680827)
\lineto(215.88796479,193.80680827)
\closepath
\moveto(214.23796479,193.68180827)
}
}
{
\newrgbcolor{curcolor}{0 0 0}
\pscustom[linestyle=none,fillstyle=solid,fillcolor=curcolor]
{
\newpath
\moveto(222.94647309,192.25680827)
\curveto(222.94647309,193.23180827)(222.47147309,193.80680827)(221.27147309,193.80680827)
\curveto(220.34647309,193.80680827)(219.77147309,193.30680827)(219.44647309,192.70680827)
\lineto(219.44647309,193.80680827)
\lineto(217.79647309,193.68180827)
\lineto(217.79647309,193.28180827)
\curveto(218.54647309,193.28180827)(218.64647309,193.20680827)(218.64647309,192.65680827)
\lineto(218.64647309,189.75680827)
\curveto(218.64647309,189.28180827)(218.52147309,189.28180827)(217.79647309,189.28180827)
\lineto(217.79647309,188.88180827)
\curveto(217.82147309,188.88180827)(218.59647309,188.93180827)(219.07147309,188.93180827)
\curveto(219.47147309,188.93180827)(220.24647309,188.88180827)(220.34647309,188.88180827)
\lineto(220.34647309,189.28180827)
\curveto(219.62147309,189.28180827)(219.52147309,189.28180827)(219.52147309,189.75680827)
\lineto(219.52147309,191.78180827)
\curveto(219.52147309,192.93180827)(220.44647309,193.48180827)(221.17147309,193.48180827)
\curveto(221.94647309,193.48180827)(222.04647309,192.88180827)(222.04647309,192.30680827)
\lineto(222.04647309,189.75680827)
\curveto(222.04647309,189.28180827)(221.94647309,189.28180827)(221.22147309,189.28180827)
\lineto(221.22147309,188.88180827)
\curveto(221.24647309,188.88180827)(222.02147309,188.93180827)(222.49647309,188.93180827)
\curveto(222.89647309,188.93180827)(223.67147309,188.88180827)(223.77147309,188.88180827)
\lineto(223.77147309,189.28180827)
\curveto(223.04647309,189.28180827)(222.94647309,189.28180827)(222.94647309,189.75680827)
\closepath
\moveto(222.94647309,192.25680827)
}
}
{
\newrgbcolor{curcolor}{0 0 0}
\pscustom[linestyle=none,fillstyle=solid,fillcolor=curcolor]
{
\newpath
\moveto(305.42516488,99.22130054)
\lineto(306.92516488,99.22130054)
\curveto(307.25016488,99.22130054)(307.42516488,99.22130054)(307.42516488,99.54630054)
\curveto(307.42516488,99.72130054)(307.25016488,99.72130054)(306.97516488,99.72130054)
\lineto(305.57516488,99.72130054)
\curveto(306.15016488,101.99630054)(306.22516488,102.29630054)(306.22516488,102.39630054)
\curveto(306.22516488,102.67130054)(306.02516488,102.82130054)(305.75016488,102.82130054)
\curveto(305.70016488,102.82130054)(305.25016488,102.82130054)(305.12516488,102.24630054)
\lineto(304.50016488,99.72130054)
\lineto(303.00016488,99.72130054)
\curveto(302.67516488,99.72130054)(302.52516488,99.72130054)(302.52516488,99.42130054)
\curveto(302.52516488,99.22130054)(302.65016488,99.22130054)(302.97516488,99.22130054)
\lineto(304.37516488,99.22130054)
\curveto(303.22516488,94.69630054)(303.15016488,94.42130054)(303.15016488,94.14630054)
\curveto(303.15016488,93.27130054)(303.75016488,92.67130054)(304.62516488,92.67130054)
\curveto(306.25016488,92.67130054)(307.15016488,94.99630054)(307.15016488,95.12130054)
\curveto(307.15016488,95.29630054)(307.02516488,95.29630054)(306.97516488,95.29630054)
\curveto(306.82516488,95.29630054)(306.80016488,95.24630054)(306.72516488,95.07130054)
\curveto(306.05016488,93.39630054)(305.20016488,93.02130054)(304.65016488,93.02130054)
\curveto(304.32516488,93.02130054)(304.15016488,93.22130054)(304.15016488,93.74630054)
\curveto(304.15016488,94.14630054)(304.20016488,94.24630054)(304.25016488,94.52130054)
\closepath
\moveto(305.42516488,99.22130054)
}
}
{
\newrgbcolor{curcolor}{0 0 0}
\pscustom[linestyle=none,fillstyle=solid,fillcolor=curcolor]
{
\newpath
\moveto(317.0819642,93.83089527)
\curveto(317.0819642,94.80589527)(316.5819642,95.38089527)(315.4069642,95.38089527)
\curveto(314.5069642,95.38089527)(313.9069642,94.90589527)(313.6069642,94.33089527)
\curveto(313.3819642,95.13089527)(312.7819642,95.38089527)(311.9819642,95.38089527)
\curveto(311.0569642,95.38089527)(310.4819642,94.88089527)(310.1569642,94.28089527)
\lineto(310.1569642,95.38089527)
\lineto(308.5069642,95.25589527)
\lineto(308.5069642,94.85589527)
\curveto(309.2569642,94.85589527)(309.3569642,94.78089527)(309.3569642,94.23089527)
\lineto(309.3569642,91.33089527)
\curveto(309.3569642,90.85589527)(309.2319642,90.85589527)(308.5069642,90.85589527)
\lineto(308.5069642,90.45589527)
\curveto(308.5319642,90.45589527)(309.3069642,90.50589527)(309.7819642,90.50589527)
\curveto(310.1819642,90.50589527)(310.9569642,90.45589527)(311.0569642,90.45589527)
\lineto(311.0569642,90.85589527)
\curveto(310.3319642,90.85589527)(310.2319642,90.85589527)(310.2319642,91.33089527)
\lineto(310.2319642,93.35589527)
\curveto(310.2319642,94.50589527)(311.1569642,95.05589527)(311.8819642,95.05589527)
\curveto(312.6569642,95.05589527)(312.7569642,94.45589527)(312.7569642,93.88089527)
\lineto(312.7569642,91.33089527)
\curveto(312.7569642,90.85589527)(312.6569642,90.85589527)(311.9319642,90.85589527)
\lineto(311.9319642,90.45589527)
\curveto(311.9569642,90.45589527)(312.7319642,90.50589527)(313.2069642,90.50589527)
\curveto(313.6069642,90.50589527)(314.3819642,90.45589527)(314.4819642,90.45589527)
\lineto(314.4819642,90.85589527)
\curveto(313.7569642,90.85589527)(313.6569642,90.85589527)(313.6569642,91.33089527)
\lineto(313.6569642,93.35589527)
\curveto(313.6569642,94.50589527)(314.5819642,95.05589527)(315.3069642,95.05589527)
\curveto(316.0819642,95.05589527)(316.1819642,94.45589527)(316.1819642,93.88089527)
\lineto(316.1819642,91.33089527)
\curveto(316.1819642,90.85589527)(316.0819642,90.85589527)(315.3569642,90.85589527)
\lineto(315.3569642,90.45589527)
\curveto(315.3819642,90.45589527)(316.1569642,90.50589527)(316.6319642,90.50589527)
\curveto(317.0319642,90.50589527)(317.8069642,90.45589527)(317.9069642,90.45589527)
\lineto(317.9069642,90.85589527)
\curveto(317.1819642,90.85589527)(317.0819642,90.85589527)(317.0819642,91.33089527)
\closepath
\moveto(317.0819642,93.83089527)
}
}
{
\newrgbcolor{curcolor}{0 0 0}
\pscustom[linestyle=none,fillstyle=solid,fillcolor=curcolor]
{
\newpath
\moveto(323.35381479,93.45589527)
\curveto(323.35381479,94.03089527)(323.35381479,94.45589527)(322.82881479,94.88089527)
\curveto(322.37881479,95.25589527)(321.85381479,95.43089527)(321.17881479,95.43089527)
\curveto(320.12881479,95.43089527)(319.37881479,95.03089527)(319.37881479,94.35589527)
\curveto(319.37881479,93.98089527)(319.62881479,93.80589527)(319.92881479,93.80589527)
\curveto(320.22881479,93.80589527)(320.45381479,94.03089527)(320.45381479,94.33089527)
\curveto(320.45381479,94.50589527)(320.35381479,94.75589527)(320.05381479,94.83089527)
\curveto(320.45381479,95.10589527)(321.10381479,95.10589527)(321.15381479,95.10589527)
\curveto(321.77881479,95.10589527)(322.47881479,94.70589527)(322.47881479,93.75589527)
\lineto(322.47881479,93.43089527)
\curveto(321.85381479,93.40589527)(321.12881479,93.35589527)(320.30381479,93.05589527)
\curveto(319.30381479,92.70589527)(319.00381479,92.08089527)(319.00381479,91.58089527)
\curveto(319.00381479,90.63089527)(320.15381479,90.35589527)(320.95381479,90.35589527)
\curveto(321.82881479,90.35589527)(322.35381479,90.85589527)(322.60381479,91.28089527)
\curveto(322.62881479,90.83089527)(322.92881479,90.40589527)(323.45381479,90.40589527)
\curveto(323.47881479,90.40589527)(324.55381479,90.40589527)(324.55381479,91.45589527)
\lineto(324.55381479,92.08089527)
\lineto(324.17881479,92.08089527)
\lineto(324.17881479,91.48089527)
\curveto(324.17881479,91.35589527)(324.17881479,90.83089527)(323.75381479,90.83089527)
\curveto(323.35381479,90.83089527)(323.35381479,91.35589527)(323.35381479,91.48089527)
\closepath
\moveto(322.47881479,92.03089527)
\curveto(322.47881479,90.95589527)(321.52881479,90.65589527)(321.02881479,90.65589527)
\curveto(320.45381479,90.65589527)(319.92881479,91.03089527)(319.92881479,91.58089527)
\curveto(319.92881479,92.20589527)(320.45381479,93.05589527)(322.47881479,93.13089527)
\closepath
\moveto(322.47881479,92.03089527)
}
}
{
\newrgbcolor{curcolor}{0 0 0}
\pscustom[linestyle=none,fillstyle=solid,fillcolor=curcolor]
{
\newpath
\moveto(328.50834359,92.95589527)
\lineto(328.43334359,93.05589527)
\curveto(328.43334359,93.10589527)(329.15834359,93.88089527)(329.25834359,93.98089527)
\curveto(329.65834359,94.43089527)(330.03334359,94.85589527)(330.93334359,94.85589527)
\lineto(330.93334359,95.25589527)
\curveto(330.60834359,95.23089527)(330.28334359,95.20589527)(329.98334359,95.20589527)
\curveto(329.65834359,95.20589527)(329.20834359,95.23089527)(328.88334359,95.25589527)
\lineto(328.88334359,94.85589527)
\curveto(329.05834359,94.83089527)(329.10834359,94.70589527)(329.10834359,94.60589527)
\curveto(329.10834359,94.58089527)(329.10834359,94.40589527)(328.93334359,94.23089527)
\lineto(328.15834359,93.35589527)
\lineto(327.23334359,94.40589527)
\curveto(327.10834359,94.53089527)(327.10834359,94.58089527)(327.10834359,94.63089527)
\curveto(327.10834359,94.78089527)(327.25834359,94.85589527)(327.40834359,94.85589527)
\lineto(327.40834359,95.25589527)
\curveto(327.00834359,95.23089527)(326.58334359,95.20589527)(326.15834359,95.20589527)
\curveto(325.83334359,95.20589527)(325.40834359,95.23089527)(325.08334359,95.25589527)
\lineto(325.08334359,94.85589527)
\curveto(325.58334359,94.85589527)(325.88334359,94.85589527)(326.18334359,94.53089527)
\lineto(327.55834359,92.93089527)
\curveto(327.58334359,92.90589527)(327.65834359,92.83089527)(327.65834359,92.80589527)
\curveto(327.65834359,92.78089527)(326.80834359,91.85589527)(326.70834359,91.73089527)
\curveto(326.28334359,91.28089527)(325.90834359,90.88089527)(325.03334359,90.85589527)
\lineto(325.03334359,90.45589527)
\curveto(325.35834359,90.48089527)(325.60834359,90.50589527)(325.95834359,90.50589527)
\curveto(326.28334359,90.50589527)(326.73334359,90.48089527)(327.05834359,90.45589527)
\lineto(327.05834359,90.85589527)
\curveto(326.93334359,90.88089527)(326.83334359,90.95589527)(326.83334359,91.10589527)
\curveto(326.83334359,91.33089527)(326.95834359,91.48089527)(327.13334359,91.65589527)
\lineto(327.93334359,92.53089527)
\lineto(328.88334359,91.40589527)
\curveto(329.10834359,91.18089527)(329.10834359,91.15589527)(329.10834359,91.08089527)
\curveto(329.10834359,90.88089527)(328.85834359,90.85589527)(328.80834359,90.85589527)
\lineto(328.80834359,90.45589527)
\curveto(328.90834359,90.45589527)(329.70834359,90.50589527)(330.05834359,90.50589527)
\curveto(330.40834359,90.50589527)(330.78334359,90.48089527)(331.13334359,90.45589527)
\lineto(331.13334359,90.85589527)
\curveto(330.38334359,90.85589527)(330.30834359,90.93089527)(330.03334359,91.20589527)
\closepath
\moveto(328.50834359,92.95589527)
}
}
{
\newrgbcolor{curcolor}{0 0 0}
\pscustom[linestyle=none,fillstyle=solid,fillcolor=curcolor]
{
\newpath
\moveto(137.99964288,98.76007354)
\lineto(139.49964288,98.76007354)
\curveto(139.82464288,98.76007354)(139.99964288,98.76007354)(139.99964288,99.08507354)
\curveto(139.99964288,99.26007354)(139.82464288,99.26007354)(139.54964288,99.26007354)
\lineto(138.14964288,99.26007354)
\curveto(138.72464288,101.53507354)(138.79964288,101.83507354)(138.79964288,101.93507354)
\curveto(138.79964288,102.21007354)(138.59964288,102.36007354)(138.32464288,102.36007354)
\curveto(138.27464288,102.36007354)(137.82464288,102.36007354)(137.69964288,101.78507354)
\lineto(137.07464288,99.26007354)
\lineto(135.57464288,99.26007354)
\curveto(135.24964288,99.26007354)(135.09964288,99.26007354)(135.09964288,98.96007354)
\curveto(135.09964288,98.76007354)(135.22464288,98.76007354)(135.54964288,98.76007354)
\lineto(136.94964288,98.76007354)
\curveto(135.79964288,94.23507354)(135.72464288,93.96007354)(135.72464288,93.68507354)
\curveto(135.72464288,92.81007354)(136.32464288,92.21007354)(137.19964288,92.21007354)
\curveto(138.82464288,92.21007354)(139.72464288,94.53507354)(139.72464288,94.66007354)
\curveto(139.72464288,94.83507354)(139.59964288,94.83507354)(139.54964288,94.83507354)
\curveto(139.39964288,94.83507354)(139.37464288,94.78507354)(139.29964288,94.61007354)
\curveto(138.62464288,92.93507354)(137.77464288,92.56007354)(137.22464288,92.56007354)
\curveto(136.89964288,92.56007354)(136.72464288,92.76007354)(136.72464288,93.28507354)
\curveto(136.72464288,93.68507354)(136.77464288,93.78507354)(136.82464288,94.06007354)
\closepath
\moveto(137.99964288,98.76007354)
}
}
{
\newrgbcolor{curcolor}{0 0 0}
\pscustom[linestyle=none,fillstyle=solid,fillcolor=curcolor]
{
\newpath
\moveto(149.6564422,93.36966827)
\curveto(149.6564422,94.34466827)(149.1564422,94.91966827)(147.9814422,94.91966827)
\curveto(147.0814422,94.91966827)(146.4814422,94.44466827)(146.1814422,93.86966827)
\curveto(145.9564422,94.66966827)(145.3564422,94.91966827)(144.5564422,94.91966827)
\curveto(143.6314422,94.91966827)(143.0564422,94.41966827)(142.7314422,93.81966827)
\lineto(142.7314422,94.91966827)
\lineto(141.0814422,94.79466827)
\lineto(141.0814422,94.39466827)
\curveto(141.8314422,94.39466827)(141.9314422,94.31966827)(141.9314422,93.76966827)
\lineto(141.9314422,90.86966827)
\curveto(141.9314422,90.39466827)(141.8064422,90.39466827)(141.0814422,90.39466827)
\lineto(141.0814422,89.99466827)
\curveto(141.1064422,89.99466827)(141.8814422,90.04466827)(142.3564422,90.04466827)
\curveto(142.7564422,90.04466827)(143.5314422,89.99466827)(143.6314422,89.99466827)
\lineto(143.6314422,90.39466827)
\curveto(142.9064422,90.39466827)(142.8064422,90.39466827)(142.8064422,90.86966827)
\lineto(142.8064422,92.89466827)
\curveto(142.8064422,94.04466827)(143.7314422,94.59466827)(144.4564422,94.59466827)
\curveto(145.2314422,94.59466827)(145.3314422,93.99466827)(145.3314422,93.41966827)
\lineto(145.3314422,90.86966827)
\curveto(145.3314422,90.39466827)(145.2314422,90.39466827)(144.5064422,90.39466827)
\lineto(144.5064422,89.99466827)
\curveto(144.5314422,89.99466827)(145.3064422,90.04466827)(145.7814422,90.04466827)
\curveto(146.1814422,90.04466827)(146.9564422,89.99466827)(147.0564422,89.99466827)
\lineto(147.0564422,90.39466827)
\curveto(146.3314422,90.39466827)(146.2314422,90.39466827)(146.2314422,90.86966827)
\lineto(146.2314422,92.89466827)
\curveto(146.2314422,94.04466827)(147.1564422,94.59466827)(147.8814422,94.59466827)
\curveto(148.6564422,94.59466827)(148.7564422,93.99466827)(148.7564422,93.41966827)
\lineto(148.7564422,90.86966827)
\curveto(148.7564422,90.39466827)(148.6564422,90.39466827)(147.9314422,90.39466827)
\lineto(147.9314422,89.99466827)
\curveto(147.9564422,89.99466827)(148.7314422,90.04466827)(149.2064422,90.04466827)
\curveto(149.6064422,90.04466827)(150.3814422,89.99466827)(150.4814422,89.99466827)
\lineto(150.4814422,90.39466827)
\curveto(149.7564422,90.39466827)(149.6564422,90.39466827)(149.6564422,90.86966827)
\closepath
\moveto(149.6564422,93.36966827)
}
}
{
\newrgbcolor{curcolor}{0 0 0}
\pscustom[linestyle=none,fillstyle=solid,fillcolor=curcolor]
{
\newpath
\moveto(155.92829279,92.99466827)
\curveto(155.92829279,93.56966827)(155.92829279,93.99466827)(155.40329279,94.41966827)
\curveto(154.95329279,94.79466827)(154.42829279,94.96966827)(153.75329279,94.96966827)
\curveto(152.70329279,94.96966827)(151.95329279,94.56966827)(151.95329279,93.89466827)
\curveto(151.95329279,93.51966827)(152.20329279,93.34466827)(152.50329279,93.34466827)
\curveto(152.80329279,93.34466827)(153.02829279,93.56966827)(153.02829279,93.86966827)
\curveto(153.02829279,94.04466827)(152.92829279,94.29466827)(152.62829279,94.36966827)
\curveto(153.02829279,94.64466827)(153.67829279,94.64466827)(153.72829279,94.64466827)
\curveto(154.35329279,94.64466827)(155.05329279,94.24466827)(155.05329279,93.29466827)
\lineto(155.05329279,92.96966827)
\curveto(154.42829279,92.94466827)(153.70329279,92.89466827)(152.87829279,92.59466827)
\curveto(151.87829279,92.24466827)(151.57829279,91.61966827)(151.57829279,91.11966827)
\curveto(151.57829279,90.16966827)(152.72829279,89.89466827)(153.52829279,89.89466827)
\curveto(154.40329279,89.89466827)(154.92829279,90.39466827)(155.17829279,90.81966827)
\curveto(155.20329279,90.36966827)(155.50329279,89.94466827)(156.02829279,89.94466827)
\curveto(156.05329279,89.94466827)(157.12829279,89.94466827)(157.12829279,90.99466827)
\lineto(157.12829279,91.61966827)
\lineto(156.75329279,91.61966827)
\lineto(156.75329279,91.01966827)
\curveto(156.75329279,90.89466827)(156.75329279,90.36966827)(156.32829279,90.36966827)
\curveto(155.92829279,90.36966827)(155.92829279,90.89466827)(155.92829279,91.01966827)
\closepath
\moveto(155.05329279,91.56966827)
\curveto(155.05329279,90.49466827)(154.10329279,90.19466827)(153.60329279,90.19466827)
\curveto(153.02829279,90.19466827)(152.50329279,90.56966827)(152.50329279,91.11966827)
\curveto(152.50329279,91.74466827)(153.02829279,92.59466827)(155.05329279,92.66966827)
\closepath
\moveto(155.05329279,91.56966827)
}
}
{
\newrgbcolor{curcolor}{0 0 0}
\pscustom[linestyle=none,fillstyle=solid,fillcolor=curcolor]
{
\newpath
\moveto(161.08282159,92.49466827)
\lineto(161.00782159,92.59466827)
\curveto(161.00782159,92.64466827)(161.73282159,93.41966827)(161.83282159,93.51966827)
\curveto(162.23282159,93.96966827)(162.60782159,94.39466827)(163.50782159,94.39466827)
\lineto(163.50782159,94.79466827)
\curveto(163.18282159,94.76966827)(162.85782159,94.74466827)(162.55782159,94.74466827)
\curveto(162.23282159,94.74466827)(161.78282159,94.76966827)(161.45782159,94.79466827)
\lineto(161.45782159,94.39466827)
\curveto(161.63282159,94.36966827)(161.68282159,94.24466827)(161.68282159,94.14466827)
\curveto(161.68282159,94.11966827)(161.68282159,93.94466827)(161.50782159,93.76966827)
\lineto(160.73282159,92.89466827)
\lineto(159.80782159,93.94466827)
\curveto(159.68282159,94.06966827)(159.68282159,94.11966827)(159.68282159,94.16966827)
\curveto(159.68282159,94.31966827)(159.83282159,94.39466827)(159.98282159,94.39466827)
\lineto(159.98282159,94.79466827)
\curveto(159.58282159,94.76966827)(159.15782159,94.74466827)(158.73282159,94.74466827)
\curveto(158.40782159,94.74466827)(157.98282159,94.76966827)(157.65782159,94.79466827)
\lineto(157.65782159,94.39466827)
\curveto(158.15782159,94.39466827)(158.45782159,94.39466827)(158.75782159,94.06966827)
\lineto(160.13282159,92.46966827)
\curveto(160.15782159,92.44466827)(160.23282159,92.36966827)(160.23282159,92.34466827)
\curveto(160.23282159,92.31966827)(159.38282159,91.39466827)(159.28282159,91.26966827)
\curveto(158.85782159,90.81966827)(158.48282159,90.41966827)(157.60782159,90.39466827)
\lineto(157.60782159,89.99466827)
\curveto(157.93282159,90.01966827)(158.18282159,90.04466827)(158.53282159,90.04466827)
\curveto(158.85782159,90.04466827)(159.30782159,90.01966827)(159.63282159,89.99466827)
\lineto(159.63282159,90.39466827)
\curveto(159.50782159,90.41966827)(159.40782159,90.49466827)(159.40782159,90.64466827)
\curveto(159.40782159,90.86966827)(159.53282159,91.01966827)(159.70782159,91.19466827)
\lineto(160.50782159,92.06966827)
\lineto(161.45782159,90.94466827)
\curveto(161.68282159,90.71966827)(161.68282159,90.69466827)(161.68282159,90.61966827)
\curveto(161.68282159,90.41966827)(161.43282159,90.39466827)(161.38282159,90.39466827)
\lineto(161.38282159,89.99466827)
\curveto(161.48282159,89.99466827)(162.28282159,90.04466827)(162.63282159,90.04466827)
\curveto(162.98282159,90.04466827)(163.35782159,90.01966827)(163.70782159,89.99466827)
\lineto(163.70782159,90.39466827)
\curveto(162.95782159,90.39466827)(162.88282159,90.46966827)(162.60782159,90.74466827)
\closepath
\moveto(161.08282159,92.49466827)
}
}
{
\newrgbcolor{curcolor}{0 0 0}
\pscustom[linestyle=none,fillstyle=solid,fillcolor=curcolor]
{
\newpath
\moveto(305.42516488,250.50386754)
\lineto(306.92516488,250.50386754)
\curveto(307.25016488,250.50386754)(307.42516488,250.50386754)(307.42516488,250.82886754)
\curveto(307.42516488,251.00386754)(307.25016488,251.00386754)(306.97516488,251.00386754)
\lineto(305.57516488,251.00386754)
\curveto(306.15016488,253.27886754)(306.22516488,253.57886754)(306.22516488,253.67886754)
\curveto(306.22516488,253.95386754)(306.02516488,254.10386754)(305.75016488,254.10386754)
\curveto(305.70016488,254.10386754)(305.25016488,254.10386754)(305.12516488,253.52886754)
\lineto(304.50016488,251.00386754)
\lineto(303.00016488,251.00386754)
\curveto(302.67516488,251.00386754)(302.52516488,251.00386754)(302.52516488,250.70386754)
\curveto(302.52516488,250.50386754)(302.65016488,250.50386754)(302.97516488,250.50386754)
\lineto(304.37516488,250.50386754)
\curveto(303.22516488,245.97886754)(303.15016488,245.70386754)(303.15016488,245.42886754)
\curveto(303.15016488,244.55386754)(303.75016488,243.95386754)(304.62516488,243.95386754)
\curveto(306.25016488,243.95386754)(307.15016488,246.27886754)(307.15016488,246.40386754)
\curveto(307.15016488,246.57886754)(307.02516488,246.57886754)(306.97516488,246.57886754)
\curveto(306.82516488,246.57886754)(306.80016488,246.52886754)(306.72516488,246.35386754)
\curveto(306.05016488,244.67886754)(305.20016488,244.30386754)(304.65016488,244.30386754)
\curveto(304.32516488,244.30386754)(304.15016488,244.50386754)(304.15016488,245.02886754)
\curveto(304.15016488,245.42886754)(304.20016488,245.52886754)(304.25016488,245.80386754)
\closepath
\moveto(305.42516488,250.50386754)
}
}
{
\newrgbcolor{curcolor}{0 0 0}
\pscustom[linestyle=none,fillstyle=solid,fillcolor=curcolor]
{
\newpath
\moveto(317.0819642,245.11346227)
\curveto(317.0819642,246.08846227)(316.5819642,246.66346227)(315.4069642,246.66346227)
\curveto(314.5069642,246.66346227)(313.9069642,246.18846227)(313.6069642,245.61346227)
\curveto(313.3819642,246.41346227)(312.7819642,246.66346227)(311.9819642,246.66346227)
\curveto(311.0569642,246.66346227)(310.4819642,246.16346227)(310.1569642,245.56346227)
\lineto(310.1569642,246.66346227)
\lineto(308.5069642,246.53846227)
\lineto(308.5069642,246.13846227)
\curveto(309.2569642,246.13846227)(309.3569642,246.06346227)(309.3569642,245.51346227)
\lineto(309.3569642,242.61346227)
\curveto(309.3569642,242.13846227)(309.2319642,242.13846227)(308.5069642,242.13846227)
\lineto(308.5069642,241.73846227)
\curveto(308.5319642,241.73846227)(309.3069642,241.78846227)(309.7819642,241.78846227)
\curveto(310.1819642,241.78846227)(310.9569642,241.73846227)(311.0569642,241.73846227)
\lineto(311.0569642,242.13846227)
\curveto(310.3319642,242.13846227)(310.2319642,242.13846227)(310.2319642,242.61346227)
\lineto(310.2319642,244.63846227)
\curveto(310.2319642,245.78846227)(311.1569642,246.33846227)(311.8819642,246.33846227)
\curveto(312.6569642,246.33846227)(312.7569642,245.73846227)(312.7569642,245.16346227)
\lineto(312.7569642,242.61346227)
\curveto(312.7569642,242.13846227)(312.6569642,242.13846227)(311.9319642,242.13846227)
\lineto(311.9319642,241.73846227)
\curveto(311.9569642,241.73846227)(312.7319642,241.78846227)(313.2069642,241.78846227)
\curveto(313.6069642,241.78846227)(314.3819642,241.73846227)(314.4819642,241.73846227)
\lineto(314.4819642,242.13846227)
\curveto(313.7569642,242.13846227)(313.6569642,242.13846227)(313.6569642,242.61346227)
\lineto(313.6569642,244.63846227)
\curveto(313.6569642,245.78846227)(314.5819642,246.33846227)(315.3069642,246.33846227)
\curveto(316.0819642,246.33846227)(316.1819642,245.73846227)(316.1819642,245.16346227)
\lineto(316.1819642,242.61346227)
\curveto(316.1819642,242.13846227)(316.0819642,242.13846227)(315.3569642,242.13846227)
\lineto(315.3569642,241.73846227)
\curveto(315.3819642,241.73846227)(316.1569642,241.78846227)(316.6319642,241.78846227)
\curveto(317.0319642,241.78846227)(317.8069642,241.73846227)(317.9069642,241.73846227)
\lineto(317.9069642,242.13846227)
\curveto(317.1819642,242.13846227)(317.0819642,242.13846227)(317.0819642,242.61346227)
\closepath
\moveto(317.0819642,245.11346227)
}
}
{
\newrgbcolor{curcolor}{0 0 0}
\pscustom[linestyle=none,fillstyle=solid,fillcolor=curcolor]
{
\newpath
\moveto(323.35381479,244.73846227)
\curveto(323.35381479,245.31346227)(323.35381479,245.73846227)(322.82881479,246.16346227)
\curveto(322.37881479,246.53846227)(321.85381479,246.71346227)(321.17881479,246.71346227)
\curveto(320.12881479,246.71346227)(319.37881479,246.31346227)(319.37881479,245.63846227)
\curveto(319.37881479,245.26346227)(319.62881479,245.08846227)(319.92881479,245.08846227)
\curveto(320.22881479,245.08846227)(320.45381479,245.31346227)(320.45381479,245.61346227)
\curveto(320.45381479,245.78846227)(320.35381479,246.03846227)(320.05381479,246.11346227)
\curveto(320.45381479,246.38846227)(321.10381479,246.38846227)(321.15381479,246.38846227)
\curveto(321.77881479,246.38846227)(322.47881479,245.98846227)(322.47881479,245.03846227)
\lineto(322.47881479,244.71346227)
\curveto(321.85381479,244.68846227)(321.12881479,244.63846227)(320.30381479,244.33846227)
\curveto(319.30381479,243.98846227)(319.00381479,243.36346227)(319.00381479,242.86346227)
\curveto(319.00381479,241.91346227)(320.15381479,241.63846227)(320.95381479,241.63846227)
\curveto(321.82881479,241.63846227)(322.35381479,242.13846227)(322.60381479,242.56346227)
\curveto(322.62881479,242.11346227)(322.92881479,241.68846227)(323.45381479,241.68846227)
\curveto(323.47881479,241.68846227)(324.55381479,241.68846227)(324.55381479,242.73846227)
\lineto(324.55381479,243.36346227)
\lineto(324.17881479,243.36346227)
\lineto(324.17881479,242.76346227)
\curveto(324.17881479,242.63846227)(324.17881479,242.11346227)(323.75381479,242.11346227)
\curveto(323.35381479,242.11346227)(323.35381479,242.63846227)(323.35381479,242.76346227)
\closepath
\moveto(322.47881479,243.31346227)
\curveto(322.47881479,242.23846227)(321.52881479,241.93846227)(321.02881479,241.93846227)
\curveto(320.45381479,241.93846227)(319.92881479,242.31346227)(319.92881479,242.86346227)
\curveto(319.92881479,243.48846227)(320.45381479,244.33846227)(322.47881479,244.41346227)
\closepath
\moveto(322.47881479,243.31346227)
}
}
{
\newrgbcolor{curcolor}{0 0 0}
\pscustom[linestyle=none,fillstyle=solid,fillcolor=curcolor]
{
\newpath
\moveto(328.50834359,244.23846227)
\lineto(328.43334359,244.33846227)
\curveto(328.43334359,244.38846227)(329.15834359,245.16346227)(329.25834359,245.26346227)
\curveto(329.65834359,245.71346227)(330.03334359,246.13846227)(330.93334359,246.13846227)
\lineto(330.93334359,246.53846227)
\curveto(330.60834359,246.51346227)(330.28334359,246.48846227)(329.98334359,246.48846227)
\curveto(329.65834359,246.48846227)(329.20834359,246.51346227)(328.88334359,246.53846227)
\lineto(328.88334359,246.13846227)
\curveto(329.05834359,246.11346227)(329.10834359,245.98846227)(329.10834359,245.88846227)
\curveto(329.10834359,245.86346227)(329.10834359,245.68846227)(328.93334359,245.51346227)
\lineto(328.15834359,244.63846227)
\lineto(327.23334359,245.68846227)
\curveto(327.10834359,245.81346227)(327.10834359,245.86346227)(327.10834359,245.91346227)
\curveto(327.10834359,246.06346227)(327.25834359,246.13846227)(327.40834359,246.13846227)
\lineto(327.40834359,246.53846227)
\curveto(327.00834359,246.51346227)(326.58334359,246.48846227)(326.15834359,246.48846227)
\curveto(325.83334359,246.48846227)(325.40834359,246.51346227)(325.08334359,246.53846227)
\lineto(325.08334359,246.13846227)
\curveto(325.58334359,246.13846227)(325.88334359,246.13846227)(326.18334359,245.81346227)
\lineto(327.55834359,244.21346227)
\curveto(327.58334359,244.18846227)(327.65834359,244.11346227)(327.65834359,244.08846227)
\curveto(327.65834359,244.06346227)(326.80834359,243.13846227)(326.70834359,243.01346227)
\curveto(326.28334359,242.56346227)(325.90834359,242.16346227)(325.03334359,242.13846227)
\lineto(325.03334359,241.73846227)
\curveto(325.35834359,241.76346227)(325.60834359,241.78846227)(325.95834359,241.78846227)
\curveto(326.28334359,241.78846227)(326.73334359,241.76346227)(327.05834359,241.73846227)
\lineto(327.05834359,242.13846227)
\curveto(326.93334359,242.16346227)(326.83334359,242.23846227)(326.83334359,242.38846227)
\curveto(326.83334359,242.61346227)(326.95834359,242.76346227)(327.13334359,242.93846227)
\lineto(327.93334359,243.81346227)
\lineto(328.88334359,242.68846227)
\curveto(329.10834359,242.46346227)(329.10834359,242.43846227)(329.10834359,242.36346227)
\curveto(329.10834359,242.16346227)(328.85834359,242.13846227)(328.80834359,242.13846227)
\lineto(328.80834359,241.73846227)
\curveto(328.90834359,241.73846227)(329.70834359,241.78846227)(330.05834359,241.78846227)
\curveto(330.40834359,241.78846227)(330.78334359,241.76346227)(331.13334359,241.73846227)
\lineto(331.13334359,242.13846227)
\curveto(330.38334359,242.13846227)(330.30834359,242.21346227)(330.03334359,242.48846227)
\closepath
\moveto(328.50834359,244.23846227)
}
}
{
\newrgbcolor{curcolor}{0 0 0}
\pscustom[linestyle=none,fillstyle=solid,fillcolor=curcolor]
{
\newpath
\moveto(137.99964288,251.42632154)
\lineto(139.49964288,251.42632154)
\curveto(139.82464288,251.42632154)(139.99964288,251.42632154)(139.99964288,251.75132154)
\curveto(139.99964288,251.92632154)(139.82464288,251.92632154)(139.54964288,251.92632154)
\lineto(138.14964288,251.92632154)
\curveto(138.72464288,254.20132154)(138.79964288,254.50132154)(138.79964288,254.60132154)
\curveto(138.79964288,254.87632154)(138.59964288,255.02632154)(138.32464288,255.02632154)
\curveto(138.27464288,255.02632154)(137.82464288,255.02632154)(137.69964288,254.45132154)
\lineto(137.07464288,251.92632154)
\lineto(135.57464288,251.92632154)
\curveto(135.24964288,251.92632154)(135.09964288,251.92632154)(135.09964288,251.62632154)
\curveto(135.09964288,251.42632154)(135.22464288,251.42632154)(135.54964288,251.42632154)
\lineto(136.94964288,251.42632154)
\curveto(135.79964288,246.90132154)(135.72464288,246.62632154)(135.72464288,246.35132154)
\curveto(135.72464288,245.47632154)(136.32464288,244.87632154)(137.19964288,244.87632154)
\curveto(138.82464288,244.87632154)(139.72464288,247.20132154)(139.72464288,247.32632154)
\curveto(139.72464288,247.50132154)(139.59964288,247.50132154)(139.54964288,247.50132154)
\curveto(139.39964288,247.50132154)(139.37464288,247.45132154)(139.29964288,247.27632154)
\curveto(138.62464288,245.60132154)(137.77464288,245.22632154)(137.22464288,245.22632154)
\curveto(136.89964288,245.22632154)(136.72464288,245.42632154)(136.72464288,245.95132154)
\curveto(136.72464288,246.35132154)(136.77464288,246.45132154)(136.82464288,246.72632154)
\closepath
\moveto(137.99964288,251.42632154)
}
}
{
\newrgbcolor{curcolor}{0 0 0}
\pscustom[linestyle=none,fillstyle=solid,fillcolor=curcolor]
{
\newpath
\moveto(149.6564422,246.03591627)
\curveto(149.6564422,247.01091627)(149.1564422,247.58591627)(147.9814422,247.58591627)
\curveto(147.0814422,247.58591627)(146.4814422,247.11091627)(146.1814422,246.53591627)
\curveto(145.9564422,247.33591627)(145.3564422,247.58591627)(144.5564422,247.58591627)
\curveto(143.6314422,247.58591627)(143.0564422,247.08591627)(142.7314422,246.48591627)
\lineto(142.7314422,247.58591627)
\lineto(141.0814422,247.46091627)
\lineto(141.0814422,247.06091627)
\curveto(141.8314422,247.06091627)(141.9314422,246.98591627)(141.9314422,246.43591627)
\lineto(141.9314422,243.53591627)
\curveto(141.9314422,243.06091627)(141.8064422,243.06091627)(141.0814422,243.06091627)
\lineto(141.0814422,242.66091627)
\curveto(141.1064422,242.66091627)(141.8814422,242.71091627)(142.3564422,242.71091627)
\curveto(142.7564422,242.71091627)(143.5314422,242.66091627)(143.6314422,242.66091627)
\lineto(143.6314422,243.06091627)
\curveto(142.9064422,243.06091627)(142.8064422,243.06091627)(142.8064422,243.53591627)
\lineto(142.8064422,245.56091627)
\curveto(142.8064422,246.71091627)(143.7314422,247.26091627)(144.4564422,247.26091627)
\curveto(145.2314422,247.26091627)(145.3314422,246.66091627)(145.3314422,246.08591627)
\lineto(145.3314422,243.53591627)
\curveto(145.3314422,243.06091627)(145.2314422,243.06091627)(144.5064422,243.06091627)
\lineto(144.5064422,242.66091627)
\curveto(144.5314422,242.66091627)(145.3064422,242.71091627)(145.7814422,242.71091627)
\curveto(146.1814422,242.71091627)(146.9564422,242.66091627)(147.0564422,242.66091627)
\lineto(147.0564422,243.06091627)
\curveto(146.3314422,243.06091627)(146.2314422,243.06091627)(146.2314422,243.53591627)
\lineto(146.2314422,245.56091627)
\curveto(146.2314422,246.71091627)(147.1564422,247.26091627)(147.8814422,247.26091627)
\curveto(148.6564422,247.26091627)(148.7564422,246.66091627)(148.7564422,246.08591627)
\lineto(148.7564422,243.53591627)
\curveto(148.7564422,243.06091627)(148.6564422,243.06091627)(147.9314422,243.06091627)
\lineto(147.9314422,242.66091627)
\curveto(147.9564422,242.66091627)(148.7314422,242.71091627)(149.2064422,242.71091627)
\curveto(149.6064422,242.71091627)(150.3814422,242.66091627)(150.4814422,242.66091627)
\lineto(150.4814422,243.06091627)
\curveto(149.7564422,243.06091627)(149.6564422,243.06091627)(149.6564422,243.53591627)
\closepath
\moveto(149.6564422,246.03591627)
}
}
{
\newrgbcolor{curcolor}{0 0 0}
\pscustom[linestyle=none,fillstyle=solid,fillcolor=curcolor]
{
\newpath
\moveto(155.92829279,245.66091627)
\curveto(155.92829279,246.23591627)(155.92829279,246.66091627)(155.40329279,247.08591627)
\curveto(154.95329279,247.46091627)(154.42829279,247.63591627)(153.75329279,247.63591627)
\curveto(152.70329279,247.63591627)(151.95329279,247.23591627)(151.95329279,246.56091627)
\curveto(151.95329279,246.18591627)(152.20329279,246.01091627)(152.50329279,246.01091627)
\curveto(152.80329279,246.01091627)(153.02829279,246.23591627)(153.02829279,246.53591627)
\curveto(153.02829279,246.71091627)(152.92829279,246.96091627)(152.62829279,247.03591627)
\curveto(153.02829279,247.31091627)(153.67829279,247.31091627)(153.72829279,247.31091627)
\curveto(154.35329279,247.31091627)(155.05329279,246.91091627)(155.05329279,245.96091627)
\lineto(155.05329279,245.63591627)
\curveto(154.42829279,245.61091627)(153.70329279,245.56091627)(152.87829279,245.26091627)
\curveto(151.87829279,244.91091627)(151.57829279,244.28591627)(151.57829279,243.78591627)
\curveto(151.57829279,242.83591627)(152.72829279,242.56091627)(153.52829279,242.56091627)
\curveto(154.40329279,242.56091627)(154.92829279,243.06091627)(155.17829279,243.48591627)
\curveto(155.20329279,243.03591627)(155.50329279,242.61091627)(156.02829279,242.61091627)
\curveto(156.05329279,242.61091627)(157.12829279,242.61091627)(157.12829279,243.66091627)
\lineto(157.12829279,244.28591627)
\lineto(156.75329279,244.28591627)
\lineto(156.75329279,243.68591627)
\curveto(156.75329279,243.56091627)(156.75329279,243.03591627)(156.32829279,243.03591627)
\curveto(155.92829279,243.03591627)(155.92829279,243.56091627)(155.92829279,243.68591627)
\closepath
\moveto(155.05329279,244.23591627)
\curveto(155.05329279,243.16091627)(154.10329279,242.86091627)(153.60329279,242.86091627)
\curveto(153.02829279,242.86091627)(152.50329279,243.23591627)(152.50329279,243.78591627)
\curveto(152.50329279,244.41091627)(153.02829279,245.26091627)(155.05329279,245.33591627)
\closepath
\moveto(155.05329279,244.23591627)
}
}
{
\newrgbcolor{curcolor}{0 0 0}
\pscustom[linestyle=none,fillstyle=solid,fillcolor=curcolor]
{
\newpath
\moveto(161.08282159,245.16091627)
\lineto(161.00782159,245.26091627)
\curveto(161.00782159,245.31091627)(161.73282159,246.08591627)(161.83282159,246.18591627)
\curveto(162.23282159,246.63591627)(162.60782159,247.06091627)(163.50782159,247.06091627)
\lineto(163.50782159,247.46091627)
\curveto(163.18282159,247.43591627)(162.85782159,247.41091627)(162.55782159,247.41091627)
\curveto(162.23282159,247.41091627)(161.78282159,247.43591627)(161.45782159,247.46091627)
\lineto(161.45782159,247.06091627)
\curveto(161.63282159,247.03591627)(161.68282159,246.91091627)(161.68282159,246.81091627)
\curveto(161.68282159,246.78591627)(161.68282159,246.61091627)(161.50782159,246.43591627)
\lineto(160.73282159,245.56091627)
\lineto(159.80782159,246.61091627)
\curveto(159.68282159,246.73591627)(159.68282159,246.78591627)(159.68282159,246.83591627)
\curveto(159.68282159,246.98591627)(159.83282159,247.06091627)(159.98282159,247.06091627)
\lineto(159.98282159,247.46091627)
\curveto(159.58282159,247.43591627)(159.15782159,247.41091627)(158.73282159,247.41091627)
\curveto(158.40782159,247.41091627)(157.98282159,247.43591627)(157.65782159,247.46091627)
\lineto(157.65782159,247.06091627)
\curveto(158.15782159,247.06091627)(158.45782159,247.06091627)(158.75782159,246.73591627)
\lineto(160.13282159,245.13591627)
\curveto(160.15782159,245.11091627)(160.23282159,245.03591627)(160.23282159,245.01091627)
\curveto(160.23282159,244.98591627)(159.38282159,244.06091627)(159.28282159,243.93591627)
\curveto(158.85782159,243.48591627)(158.48282159,243.08591627)(157.60782159,243.06091627)
\lineto(157.60782159,242.66091627)
\curveto(157.93282159,242.68591627)(158.18282159,242.71091627)(158.53282159,242.71091627)
\curveto(158.85782159,242.71091627)(159.30782159,242.68591627)(159.63282159,242.66091627)
\lineto(159.63282159,243.06091627)
\curveto(159.50782159,243.08591627)(159.40782159,243.16091627)(159.40782159,243.31091627)
\curveto(159.40782159,243.53591627)(159.53282159,243.68591627)(159.70782159,243.86091627)
\lineto(160.50782159,244.73591627)
\lineto(161.45782159,243.61091627)
\curveto(161.68282159,243.38591627)(161.68282159,243.36091627)(161.68282159,243.28591627)
\curveto(161.68282159,243.08591627)(161.43282159,243.06091627)(161.38282159,243.06091627)
\lineto(161.38282159,242.66091627)
\curveto(161.48282159,242.66091627)(162.28282159,242.71091627)(162.63282159,242.71091627)
\curveto(162.98282159,242.71091627)(163.35782159,242.68591627)(163.70782159,242.66091627)
\lineto(163.70782159,243.06091627)
\curveto(162.95782159,243.06091627)(162.88282159,243.13591627)(162.60782159,243.41091627)
\closepath
\moveto(161.08282159,245.16091627)
}
}
{
\newrgbcolor{curcolor}{0 0 0}
\pscustom[linestyle=none,fillstyle=solid,fillcolor=curcolor]
{
\newpath
\moveto(257.27303488,111.67443854)
\lineto(258.77303488,111.67443854)
\curveto(259.09803488,111.67443854)(259.27303488,111.67443854)(259.27303488,111.99943854)
\curveto(259.27303488,112.17443854)(259.09803488,112.17443854)(258.82303488,112.17443854)
\lineto(257.42303488,112.17443854)
\curveto(257.99803488,114.44943854)(258.07303488,114.74943854)(258.07303488,114.84943854)
\curveto(258.07303488,115.12443854)(257.87303488,115.27443854)(257.59803488,115.27443854)
\curveto(257.54803488,115.27443854)(257.09803488,115.27443854)(256.97303488,114.69943854)
\lineto(256.34803488,112.17443854)
\lineto(254.84803488,112.17443854)
\curveto(254.52303488,112.17443854)(254.37303488,112.17443854)(254.37303488,111.87443854)
\curveto(254.37303488,111.67443854)(254.49803488,111.67443854)(254.82303488,111.67443854)
\lineto(256.22303488,111.67443854)
\curveto(255.07303488,107.14943854)(254.99803488,106.87443854)(254.99803488,106.59943854)
\curveto(254.99803488,105.72443854)(255.59803488,105.12443854)(256.47303488,105.12443854)
\curveto(258.09803488,105.12443854)(258.99803488,107.44943854)(258.99803488,107.57443854)
\curveto(258.99803488,107.74943854)(258.87303488,107.74943854)(258.82303488,107.74943854)
\curveto(258.67303488,107.74943854)(258.64803488,107.69943854)(258.57303488,107.52443854)
\curveto(257.89803488,105.84943854)(257.04803488,105.47443854)(256.49803488,105.47443854)
\curveto(256.17303488,105.47443854)(255.99803488,105.67443854)(255.99803488,106.19943854)
\curveto(255.99803488,106.59943854)(256.04803488,106.69943854)(256.09803488,106.97443854)
\closepath
\moveto(257.27303488,111.67443854)
}
}
{
\newrgbcolor{curcolor}{0 0 0}
\pscustom[linestyle=none,fillstyle=solid,fillcolor=curcolor]
{
\newpath
\moveto(263.9798342,107.58403327)
\curveto(263.9798342,107.78403327)(263.9798342,107.88403327)(263.8298342,107.88403327)
\curveto(263.7798342,107.88403327)(263.7548342,107.88403327)(263.6048342,107.75903327)
\curveto(263.5798342,107.73403327)(263.4798342,107.63403327)(263.4048342,107.58403327)
\curveto(263.0548342,107.80903327)(262.6548342,107.88403327)(262.2298342,107.88403327)
\curveto(260.6298342,107.88403327)(260.2548342,107.03403327)(260.2548342,106.48403327)
\curveto(260.2548342,106.13403327)(260.4048342,105.83403327)(260.6798342,105.60903327)
\curveto(261.1048342,105.25903327)(261.5298342,105.18403327)(262.2298342,105.05903327)
\curveto(262.7798342,104.95903327)(263.6798342,104.80903327)(263.6798342,104.05903327)
\curveto(263.6798342,103.63403327)(263.3798342,103.10903327)(262.3048342,103.10903327)
\curveto(261.2298342,103.10903327)(260.8548342,103.80903327)(260.6548342,104.55903327)
\curveto(260.6048342,104.70903327)(260.6048342,104.75903327)(260.4548342,104.75903327)
\curveto(260.2548342,104.75903327)(260.2548342,104.68403327)(260.2548342,104.45903327)
\lineto(260.2548342,103.08403327)
\curveto(260.2548342,102.90903327)(260.2548342,102.80903327)(260.4048342,102.80903327)
\curveto(260.5048342,102.80903327)(260.7298342,103.03403327)(260.9548342,103.28403327)
\curveto(261.4298342,102.80903327)(262.0298342,102.80903327)(262.3048342,102.80903327)
\curveto(263.7548342,102.80903327)(264.3048342,103.58403327)(264.3048342,104.35903327)
\curveto(264.3048342,104.78403327)(264.1048342,105.13403327)(263.8048342,105.38403327)
\curveto(263.3798342,105.78403327)(262.8798342,105.88403327)(262.4798342,105.93403327)
\curveto(261.6048342,106.10903327)(260.8798342,106.23403327)(260.8798342,106.83403327)
\curveto(260.8798342,107.18403327)(261.1798342,107.60903327)(262.2298342,107.60903327)
\curveto(263.5298342,107.60903327)(263.5798342,106.70903327)(263.6048342,106.38403327)
\curveto(263.6048342,106.25903327)(263.7548342,106.25903327)(263.7798342,106.25903327)
\curveto(263.9798342,106.25903327)(263.9798342,106.33403327)(263.9798342,106.55903327)
\closepath
\moveto(263.9798342,107.58403327)
}
}
{
\newrgbcolor{curcolor}{0 0 0}
\pscustom[linestyle=none,fillstyle=solid,fillcolor=curcolor]
{
\newpath
\moveto(267.90498068,101.15903327)
\curveto(267.17998068,101.15903327)(267.07998068,101.15903327)(267.07998068,101.63403327)
\lineto(267.07998068,103.43403327)
\curveto(267.12998068,103.38403327)(267.65498068,102.80903327)(268.57998068,102.80903327)
\curveto(270.05498068,102.80903327)(271.30498068,103.90903327)(271.30498068,105.30903327)
\curveto(271.30498068,106.68403327)(270.17998068,107.83403327)(268.75498068,107.83403327)
\curveto(268.12998068,107.83403327)(267.47998068,107.58403327)(267.02998068,107.13403327)
\lineto(267.02998068,107.83403327)
\lineto(265.35498068,107.70903327)
\lineto(265.35498068,107.30903327)
\curveto(266.12998068,107.30903327)(266.17998068,107.23403327)(266.17998068,106.78403327)
\lineto(266.17998068,101.63403327)
\curveto(266.17998068,101.15903327)(266.07998068,101.15903327)(265.35498068,101.15903327)
\lineto(265.35498068,100.73403327)
\curveto(265.37998068,100.73403327)(266.15498068,100.78403327)(266.62998068,100.78403327)
\curveto(267.02998068,100.78403327)(267.80498068,100.75903327)(267.90498068,100.73403327)
\closepath
\moveto(267.07998068,106.63403327)
\curveto(267.40498068,107.18403327)(268.05498068,107.45903327)(268.62998068,107.45903327)
\curveto(269.57998068,107.45903327)(270.30498068,106.48403327)(270.30498068,105.30903327)
\curveto(270.30498068,104.03403327)(269.45498068,103.10903327)(268.52998068,103.10903327)
\curveto(267.52998068,103.10903327)(267.10498068,103.95903327)(267.07998068,104.03403327)
\closepath
\moveto(267.07998068,106.63403327)
}
}
{
\newrgbcolor{curcolor}{0 0 0}
\pscustom[linestyle=none,fillstyle=solid,fillcolor=curcolor]
{
\newpath
\moveto(274.14572775,110.65903327)
\lineto(272.47072775,110.53403327)
\lineto(272.47072775,110.13403327)
\curveto(273.22072775,110.13403327)(273.29572775,110.03403327)(273.29572775,109.50903327)
\lineto(273.29572775,103.78403327)
\curveto(273.29572775,103.30903327)(273.19572775,103.30903327)(272.47072775,103.30903327)
\lineto(272.47072775,102.90903327)
\curveto(272.49572775,102.90903327)(273.27072775,102.95903327)(273.72072775,102.95903327)
\curveto(274.14572775,102.95903327)(274.54572775,102.93403327)(274.97072775,102.90903327)
\lineto(274.97072775,103.30903327)
\curveto(274.24572775,103.30903327)(274.14572775,103.30903327)(274.14572775,103.78403327)
\closepath
\moveto(274.14572775,110.65903327)
}
}
{
\newrgbcolor{curcolor}{0 0 0}
\pscustom[linestyle=none,fillstyle=solid,fillcolor=curcolor]
{
\newpath
\moveto(277.80424826,109.78403327)
\curveto(277.80424826,110.10903327)(277.52924826,110.43403327)(277.15424826,110.43403327)
\curveto(276.82924826,110.43403327)(276.52924826,110.15903327)(276.52924826,109.78403327)
\curveto(276.52924826,109.38403327)(276.85424826,109.13403327)(277.15424826,109.13403327)
\curveto(277.52924826,109.13403327)(277.80424826,109.40903327)(277.80424826,109.78403327)
\closepath
\moveto(276.10424826,107.70903327)
\lineto(276.10424826,107.30903327)
\curveto(276.80424826,107.30903327)(276.90424826,107.23403327)(276.90424826,106.68403327)
\lineto(276.90424826,103.78403327)
\curveto(276.90424826,103.30903327)(276.80424826,103.30903327)(276.07924826,103.30903327)
\lineto(276.07924826,102.90903327)
\curveto(276.10424826,102.90903327)(276.87924826,102.95903327)(277.32924826,102.95903327)
\curveto(277.72924826,102.95903327)(278.12924826,102.93403327)(278.50424826,102.90903327)
\lineto(278.50424826,103.30903327)
\curveto(277.85424826,103.30903327)(277.75424826,103.30903327)(277.75424826,103.78403327)
\lineto(277.75424826,107.83403327)
\closepath
\moveto(276.10424826,107.70903327)
}
}
{
\newrgbcolor{curcolor}{0 0 0}
\pscustom[linestyle=none,fillstyle=solid,fillcolor=curcolor]
{
\newpath
\moveto(281.31276877,107.30903327)
\lineto(283.03776877,107.30903327)
\lineto(283.03776877,107.70903327)
\lineto(281.31276877,107.70903327)
\lineto(281.31276877,109.75903327)
\lineto(280.91276877,109.75903327)
\curveto(280.91276877,108.75903327)(280.46276877,107.65903327)(279.38776877,107.63403327)
\lineto(279.38776877,107.30903327)
\lineto(280.41276877,107.30903327)
\lineto(280.41276877,104.30903327)
\curveto(280.41276877,103.05903327)(281.36276877,102.80903327)(281.98776877,102.80903327)
\curveto(282.73776877,102.80903327)(283.23776877,103.43403327)(283.23776877,104.30903327)
\lineto(283.23776877,104.93403327)
\lineto(282.86276877,104.93403327)
\lineto(282.86276877,104.33403327)
\curveto(282.86276877,103.55903327)(282.51276877,103.15903327)(282.06276877,103.15903327)
\curveto(281.31276877,103.15903327)(281.31276877,104.08403327)(281.31276877,104.28403327)
\closepath
\moveto(281.31276877,107.30903327)
}
}
{
\newrgbcolor{curcolor}{0 0 0}
\pscustom[linestyle=none,fillstyle=solid,fillcolor=curcolor]
{
\newpath
\moveto(256.35057488,224.21390854)
\lineto(257.85057488,224.21390854)
\curveto(258.17557488,224.21390854)(258.35057488,224.21390854)(258.35057488,224.53890854)
\curveto(258.35057488,224.71390854)(258.17557488,224.71390854)(257.90057488,224.71390854)
\lineto(256.50057488,224.71390854)
\curveto(257.07557488,226.98890854)(257.15057488,227.28890854)(257.15057488,227.38890854)
\curveto(257.15057488,227.66390854)(256.95057488,227.81390854)(256.67557488,227.81390854)
\curveto(256.62557488,227.81390854)(256.17557488,227.81390854)(256.05057488,227.23890854)
\lineto(255.42557488,224.71390854)
\lineto(253.92557488,224.71390854)
\curveto(253.60057488,224.71390854)(253.45057488,224.71390854)(253.45057488,224.41390854)
\curveto(253.45057488,224.21390854)(253.57557488,224.21390854)(253.90057488,224.21390854)
\lineto(255.30057488,224.21390854)
\curveto(254.15057488,219.68890854)(254.07557488,219.41390854)(254.07557488,219.13890854)
\curveto(254.07557488,218.26390854)(254.67557488,217.66390854)(255.55057488,217.66390854)
\curveto(257.17557488,217.66390854)(258.07557488,219.98890854)(258.07557488,220.11390854)
\curveto(258.07557488,220.28890854)(257.95057488,220.28890854)(257.90057488,220.28890854)
\curveto(257.75057488,220.28890854)(257.72557488,220.23890854)(257.65057488,220.06390854)
\curveto(256.97557488,218.38890854)(256.12557488,218.01390854)(255.57557488,218.01390854)
\curveto(255.25057488,218.01390854)(255.07557488,218.21390854)(255.07557488,218.73890854)
\curveto(255.07557488,219.13890854)(255.12557488,219.23890854)(255.17557488,219.51390854)
\closepath
\moveto(256.35057488,224.21390854)
}
}
{
\newrgbcolor{curcolor}{0 0 0}
\pscustom[linestyle=none,fillstyle=solid,fillcolor=curcolor]
{
\newpath
\moveto(263.0573742,220.12350327)
\curveto(263.0573742,220.32350327)(263.0573742,220.42350327)(262.9073742,220.42350327)
\curveto(262.8573742,220.42350327)(262.8323742,220.42350327)(262.6823742,220.29850327)
\curveto(262.6573742,220.27350327)(262.5573742,220.17350327)(262.4823742,220.12350327)
\curveto(262.1323742,220.34850327)(261.7323742,220.42350327)(261.3073742,220.42350327)
\curveto(259.7073742,220.42350327)(259.3323742,219.57350327)(259.3323742,219.02350327)
\curveto(259.3323742,218.67350327)(259.4823742,218.37350327)(259.7573742,218.14850327)
\curveto(260.1823742,217.79850327)(260.6073742,217.72350327)(261.3073742,217.59850327)
\curveto(261.8573742,217.49850327)(262.7573742,217.34850327)(262.7573742,216.59850327)
\curveto(262.7573742,216.17350327)(262.4573742,215.64850327)(261.3823742,215.64850327)
\curveto(260.3073742,215.64850327)(259.9323742,216.34850327)(259.7323742,217.09850327)
\curveto(259.6823742,217.24850327)(259.6823742,217.29850327)(259.5323742,217.29850327)
\curveto(259.3323742,217.29850327)(259.3323742,217.22350327)(259.3323742,216.99850327)
\lineto(259.3323742,215.62350327)
\curveto(259.3323742,215.44850327)(259.3323742,215.34850327)(259.4823742,215.34850327)
\curveto(259.5823742,215.34850327)(259.8073742,215.57350327)(260.0323742,215.82350327)
\curveto(260.5073742,215.34850327)(261.1073742,215.34850327)(261.3823742,215.34850327)
\curveto(262.8323742,215.34850327)(263.3823742,216.12350327)(263.3823742,216.89850327)
\curveto(263.3823742,217.32350327)(263.1823742,217.67350327)(262.8823742,217.92350327)
\curveto(262.4573742,218.32350327)(261.9573742,218.42350327)(261.5573742,218.47350327)
\curveto(260.6823742,218.64850327)(259.9573742,218.77350327)(259.9573742,219.37350327)
\curveto(259.9573742,219.72350327)(260.2573742,220.14850327)(261.3073742,220.14850327)
\curveto(262.6073742,220.14850327)(262.6573742,219.24850327)(262.6823742,218.92350327)
\curveto(262.6823742,218.79850327)(262.8323742,218.79850327)(262.8573742,218.79850327)
\curveto(263.0573742,218.79850327)(263.0573742,218.87350327)(263.0573742,219.09850327)
\closepath
\moveto(263.0573742,220.12350327)
}
}
{
\newrgbcolor{curcolor}{0 0 0}
\pscustom[linestyle=none,fillstyle=solid,fillcolor=curcolor]
{
\newpath
\moveto(266.98252068,213.69850327)
\curveto(266.25752068,213.69850327)(266.15752068,213.69850327)(266.15752068,214.17350327)
\lineto(266.15752068,215.97350327)
\curveto(266.20752068,215.92350327)(266.73252068,215.34850327)(267.65752068,215.34850327)
\curveto(269.13252068,215.34850327)(270.38252068,216.44850327)(270.38252068,217.84850327)
\curveto(270.38252068,219.22350327)(269.25752068,220.37350327)(267.83252068,220.37350327)
\curveto(267.20752068,220.37350327)(266.55752068,220.12350327)(266.10752068,219.67350327)
\lineto(266.10752068,220.37350327)
\lineto(264.43252068,220.24850327)
\lineto(264.43252068,219.84850327)
\curveto(265.20752068,219.84850327)(265.25752068,219.77350327)(265.25752068,219.32350327)
\lineto(265.25752068,214.17350327)
\curveto(265.25752068,213.69850327)(265.15752068,213.69850327)(264.43252068,213.69850327)
\lineto(264.43252068,213.27350327)
\curveto(264.45752068,213.27350327)(265.23252068,213.32350327)(265.70752068,213.32350327)
\curveto(266.10752068,213.32350327)(266.88252068,213.29850327)(266.98252068,213.27350327)
\closepath
\moveto(266.15752068,219.17350327)
\curveto(266.48252068,219.72350327)(267.13252068,219.99850327)(267.70752068,219.99850327)
\curveto(268.65752068,219.99850327)(269.38252068,219.02350327)(269.38252068,217.84850327)
\curveto(269.38252068,216.57350327)(268.53252068,215.64850327)(267.60752068,215.64850327)
\curveto(266.60752068,215.64850327)(266.18252068,216.49850327)(266.15752068,216.57350327)
\closepath
\moveto(266.15752068,219.17350327)
}
}
{
\newrgbcolor{curcolor}{0 0 0}
\pscustom[linestyle=none,fillstyle=solid,fillcolor=curcolor]
{
\newpath
\moveto(273.22326775,223.19850327)
\lineto(271.54826775,223.07350327)
\lineto(271.54826775,222.67350327)
\curveto(272.29826775,222.67350327)(272.37326775,222.57350327)(272.37326775,222.04850327)
\lineto(272.37326775,216.32350327)
\curveto(272.37326775,215.84850327)(272.27326775,215.84850327)(271.54826775,215.84850327)
\lineto(271.54826775,215.44850327)
\curveto(271.57326775,215.44850327)(272.34826775,215.49850327)(272.79826775,215.49850327)
\curveto(273.22326775,215.49850327)(273.62326775,215.47350327)(274.04826775,215.44850327)
\lineto(274.04826775,215.84850327)
\curveto(273.32326775,215.84850327)(273.22326775,215.84850327)(273.22326775,216.32350327)
\closepath
\moveto(273.22326775,223.19850327)
}
}
{
\newrgbcolor{curcolor}{0 0 0}
\pscustom[linestyle=none,fillstyle=solid,fillcolor=curcolor]
{
\newpath
\moveto(276.88178826,222.32350327)
\curveto(276.88178826,222.64850327)(276.60678826,222.97350327)(276.23178826,222.97350327)
\curveto(275.90678826,222.97350327)(275.60678826,222.69850327)(275.60678826,222.32350327)
\curveto(275.60678826,221.92350327)(275.93178826,221.67350327)(276.23178826,221.67350327)
\curveto(276.60678826,221.67350327)(276.88178826,221.94850327)(276.88178826,222.32350327)
\closepath
\moveto(275.18178826,220.24850327)
\lineto(275.18178826,219.84850327)
\curveto(275.88178826,219.84850327)(275.98178826,219.77350327)(275.98178826,219.22350327)
\lineto(275.98178826,216.32350327)
\curveto(275.98178826,215.84850327)(275.88178826,215.84850327)(275.15678826,215.84850327)
\lineto(275.15678826,215.44850327)
\curveto(275.18178826,215.44850327)(275.95678826,215.49850327)(276.40678826,215.49850327)
\curveto(276.80678826,215.49850327)(277.20678826,215.47350327)(277.58178826,215.44850327)
\lineto(277.58178826,215.84850327)
\curveto(276.93178826,215.84850327)(276.83178826,215.84850327)(276.83178826,216.32350327)
\lineto(276.83178826,220.37350327)
\closepath
\moveto(275.18178826,220.24850327)
}
}
{
\newrgbcolor{curcolor}{0 0 0}
\pscustom[linestyle=none,fillstyle=solid,fillcolor=curcolor]
{
\newpath
\moveto(280.39030877,219.84850327)
\lineto(282.11530877,219.84850327)
\lineto(282.11530877,220.24850327)
\lineto(280.39030877,220.24850327)
\lineto(280.39030877,222.29850327)
\lineto(279.99030877,222.29850327)
\curveto(279.99030877,221.29850327)(279.54030877,220.19850327)(278.46530877,220.17350327)
\lineto(278.46530877,219.84850327)
\lineto(279.49030877,219.84850327)
\lineto(279.49030877,216.84850327)
\curveto(279.49030877,215.59850327)(280.44030877,215.34850327)(281.06530877,215.34850327)
\curveto(281.81530877,215.34850327)(282.31530877,215.97350327)(282.31530877,216.84850327)
\lineto(282.31530877,217.47350327)
\lineto(281.94030877,217.47350327)
\lineto(281.94030877,216.87350327)
\curveto(281.94030877,216.09850327)(281.59030877,215.69850327)(281.14030877,215.69850327)
\curveto(280.39030877,215.69850327)(280.39030877,216.62350327)(280.39030877,216.82350327)
\closepath
\moveto(280.39030877,219.84850327)
}
}
{
\newrgbcolor{curcolor}{0 0 0}
\pscustom[linestyle=none,fillstyle=solid,fillcolor=curcolor]
{
\newpath
\moveto(202.26772313,276.37526966)
\lineto(205.14272313,276.37526966)
\lineto(211.61928563,268.42995716)
\lineto(211.61928563,274.53933216)
\curveto(211.61928563,275.19037383)(211.54636896,275.59662383)(211.40053563,275.75808216)
\curveto(211.20782729,275.97683216)(210.90313979,276.08620716)(210.48647313,276.08620716)
\lineto(210.11928563,276.08620716)
\lineto(210.11928563,276.37526966)
\lineto(213.80678563,276.37526966)
\lineto(213.80678563,276.08620716)
\lineto(213.43178563,276.08620716)
\curveto(212.98386896,276.08620716)(212.66616063,275.95079049)(212.47866063,275.67995716)
\curveto(212.36407729,275.51329049)(212.30678563,275.13308216)(212.30678563,274.53933216)
\lineto(212.30678563,265.60964466)
\lineto(212.02553563,265.60964466)
\lineto(205.04116063,274.14089466)
\lineto(205.04116063,267.61745716)
\curveto(205.04116063,266.96641549)(205.11147313,266.56016549)(205.25209813,266.39870716)
\curveto(205.45001479,266.17995716)(205.75470229,266.07058216)(206.16616063,266.07058216)
\lineto(206.54116063,266.07058216)
\lineto(206.54116063,265.78151966)
\lineto(202.85366063,265.78151966)
\lineto(202.85366063,266.07058216)
\lineto(203.22084813,266.07058216)
\curveto(203.67397313,266.07058216)(203.99428563,266.20599883)(204.18178563,266.47683216)
\curveto(204.29636896,266.64349883)(204.35366063,267.02370716)(204.35366063,267.61745716)
\lineto(204.35366063,274.98464466)
\curveto(204.04636896,275.34401966)(203.81199396,275.58099883)(203.65053563,275.69558216)
\curveto(203.49428563,275.81016549)(203.26251479,275.91693633)(202.95522313,276.01589466)
\curveto(202.80418146,276.06276966)(202.57501479,276.08620716)(202.26772313,276.08620716)
\closepath
}
}
{
\newrgbcolor{curcolor}{0 0 0}
\pscustom[linestyle=none,fillstyle=solid,fillcolor=curcolor]
{
\newpath
\moveto(215.73647313,270.24245716)
\curveto(215.73126479,269.17995716)(215.98907729,268.34662383)(216.50991063,267.74245716)
\curveto(217.03074396,267.13829049)(217.64272313,266.83620716)(218.34584813,266.83620716)
\curveto(218.81459813,266.83620716)(219.22084813,266.96381133)(219.56459813,267.21901966)
\curveto(219.91355646,267.47943633)(220.20522313,267.92214466)(220.43959813,268.54714466)
\lineto(220.68178563,268.39089466)
\curveto(220.57241063,267.67735299)(220.25470229,267.02631133)(219.72866063,266.43776966)
\curveto(219.20261896,265.85443633)(218.54376479,265.56276966)(217.75209813,265.56276966)
\curveto(216.89272313,265.56276966)(216.15574396,265.89610299)(215.54116063,266.56276966)
\curveto(214.93178563,267.23464466)(214.62709813,268.13568633)(214.62709813,269.26589466)
\curveto(214.62709813,270.48985299)(214.93959813,271.44297799)(215.56459813,272.12526966)
\curveto(216.19480646,272.81276966)(216.98386896,273.15651966)(217.93178563,273.15651966)
\curveto(218.73386896,273.15651966)(219.39272313,272.89089466)(219.90834813,272.35964466)
\curveto(220.42397313,271.83360299)(220.68178563,271.12787383)(220.68178563,270.24245716)
\closepath
\moveto(215.73647313,270.69558216)
\lineto(219.04897313,270.69558216)
\curveto(219.02293146,271.15391549)(218.96824396,271.47683216)(218.88491063,271.66433216)
\curveto(218.75470229,271.95599883)(218.55938979,272.18516549)(218.29897313,272.35183216)
\curveto(218.04376479,272.51849883)(217.77553563,272.60183216)(217.49428563,272.60183216)
\curveto(217.06199396,272.60183216)(216.67397313,272.43256133)(216.33022313,272.09401966)
\curveto(215.99168146,271.76068633)(215.79376479,271.29454049)(215.73647313,270.69558216)
\closepath
}
}
{
\newrgbcolor{curcolor}{0 0 0}
\pscustom[linestyle=none,fillstyle=solid,fillcolor=curcolor]
{
\newpath
\moveto(225.68959813,266.81276966)
\curveto(224.95522313,266.24506133)(224.49428563,265.91693633)(224.30678563,265.82839466)
\curveto(224.02553563,265.69818633)(223.72605646,265.63308216)(223.40834813,265.63308216)
\curveto(222.91355646,265.63308216)(222.50470229,265.80235299)(222.18178563,266.14089466)
\curveto(221.86407729,266.47943633)(221.70522313,266.92474883)(221.70522313,267.47683216)
\curveto(221.70522313,267.82579049)(221.78334813,268.12787383)(221.93959813,268.38308216)
\curveto(222.15313979,268.73724883)(222.52293146,269.07058216)(223.04897313,269.38308216)
\curveto(223.58022313,269.69558216)(224.46043146,270.07579049)(225.68959813,270.52370716)
\lineto(225.68959813,270.80495716)
\curveto(225.68959813,271.51849883)(225.57501479,272.00808216)(225.34584813,272.27370716)
\curveto(225.12188979,272.53933216)(224.79376479,272.67214466)(224.36147313,272.67214466)
\curveto(224.03334813,272.67214466)(223.77293146,272.58360299)(223.58022313,272.40651966)
\curveto(223.38230646,272.22943633)(223.28334813,272.02631133)(223.28334813,271.79714466)
\lineto(223.29897313,271.34401966)
\curveto(223.29897313,271.10443633)(223.23647313,270.91954049)(223.11147313,270.78933216)
\curveto(222.99168146,270.65912383)(222.83282729,270.59401966)(222.63491063,270.59401966)
\curveto(222.44220229,270.59401966)(222.28334813,270.66172799)(222.15834813,270.79714466)
\curveto(222.03855646,270.93256133)(221.97866063,271.11745716)(221.97866063,271.35183216)
\curveto(221.97866063,271.79974883)(222.20782729,272.21120716)(222.66616063,272.58620716)
\curveto(223.12449396,272.96120716)(223.76772313,273.14870716)(224.59584813,273.14870716)
\curveto(225.23126479,273.14870716)(225.75209813,273.04193633)(226.15834813,272.82839466)
\curveto(226.46563979,272.66693633)(226.69220229,272.41433216)(226.83803563,272.07058216)
\curveto(226.93178563,271.84662383)(226.97866063,271.38829049)(226.97866063,270.69558216)
\lineto(226.97866063,268.26589466)
\curveto(226.97866063,267.58360299)(226.99168146,267.16433216)(227.01772313,267.00808216)
\curveto(227.04376479,266.85704049)(227.08543146,266.75547799)(227.14272313,266.70339466)
\curveto(227.20522313,266.65131133)(227.27553563,266.62526966)(227.35366063,266.62526966)
\curveto(227.43699396,266.62526966)(227.50991063,266.64349883)(227.57241063,266.67995716)
\curveto(227.68178563,266.74766549)(227.89272313,266.93776966)(228.20522313,267.25026966)
\lineto(228.20522313,266.81276966)
\curveto(227.62188979,266.03151966)(227.06459813,265.64089466)(226.53334813,265.64089466)
\curveto(226.27813979,265.64089466)(226.07501479,265.72943633)(225.92397313,265.90651966)
\curveto(225.77293146,266.08360299)(225.69480646,266.38568633)(225.68959813,266.81276966)
\closepath
\moveto(225.68959813,267.32058216)
\lineto(225.68959813,270.04714466)
\curveto(224.90313979,269.73464466)(224.39532729,269.51329049)(224.16616063,269.38308216)
\curveto(223.75470229,269.15391549)(223.46043146,268.91433216)(223.28334813,268.66433216)
\curveto(223.10626479,268.41433216)(223.01772313,268.14089466)(223.01772313,267.84401966)
\curveto(223.01772313,267.46901966)(223.12970229,267.15651966)(223.35366063,266.90651966)
\curveto(223.57761896,266.66172799)(223.83543146,266.53933216)(224.12709813,266.53933216)
\curveto(224.52293146,266.53933216)(225.04376479,266.79974883)(225.68959813,267.32058216)
\closepath
}
}
{
\newrgbcolor{curcolor}{0 0 0}
\pscustom[linestyle=none,fillstyle=solid,fillcolor=curcolor]
{
\newpath
\moveto(230.83022313,273.14870716)
\lineto(230.83022313,271.53933216)
\curveto(231.42918146,272.61224883)(232.04376479,273.14870716)(232.67397313,273.14870716)
\curveto(232.96043146,273.14870716)(233.19741063,273.06016549)(233.38491063,272.88308216)
\curveto(233.57241063,272.71120716)(233.66616063,272.51068633)(233.66616063,272.28151966)
\curveto(233.66616063,272.07839466)(233.59845229,271.90651966)(233.46303563,271.76589466)
\curveto(233.32761896,271.62526966)(233.16616063,271.55495716)(232.97866063,271.55495716)
\curveto(232.79636896,271.55495716)(232.59063979,271.64349883)(232.36147313,271.82058216)
\curveto(232.13751479,272.00287383)(231.97084813,272.09401966)(231.86147313,272.09401966)
\curveto(231.76772313,272.09401966)(231.66616063,272.04193633)(231.55678563,271.93776966)
\curveto(231.32241063,271.72422799)(231.08022313,271.37266549)(230.83022313,270.88308216)
\lineto(230.83022313,267.45339466)
\curveto(230.83022313,267.05756133)(230.87970229,266.75808216)(230.97866063,266.55495716)
\curveto(231.04636896,266.41433216)(231.16616063,266.29714466)(231.33803563,266.20339466)
\curveto(231.50991063,266.10964466)(231.75730646,266.06276966)(232.08022313,266.06276966)
\lineto(232.08022313,265.78151966)
\lineto(228.41616063,265.78151966)
\lineto(228.41616063,266.06276966)
\curveto(228.78074396,266.06276966)(229.05157729,266.12006133)(229.22866063,266.23464466)
\curveto(229.35886896,266.31797799)(229.45001479,266.45079049)(229.50209813,266.63308216)
\curveto(229.52813979,266.72162383)(229.54116063,266.97422799)(229.54116063,267.39089466)
\lineto(229.54116063,270.16433216)
\curveto(229.54116063,270.99766549)(229.52293146,271.49245716)(229.48647313,271.64870716)
\curveto(229.45522313,271.81016549)(229.39272313,271.92735299)(229.29897313,272.00026966)
\curveto(229.21043146,272.07318633)(229.09845229,272.10964466)(228.96303563,272.10964466)
\curveto(228.80157729,272.10964466)(228.61928563,272.07058216)(228.41616063,271.99245716)
\lineto(228.33803563,272.27370716)
\lineto(230.50209813,273.14870716)
\closepath
}
}
{
\newrgbcolor{curcolor}{0 0 0}
\pscustom[linestyle=none,fillstyle=solid,fillcolor=curcolor]
{
\newpath
\moveto(264.83173813,275.12516466)
\lineto(264.83173813,270.96110216)
\lineto(266.76142563,270.96110216)
\curveto(267.20413396,270.96110216)(267.52705063,271.05745633)(267.73017563,271.25016466)
\curveto(267.93850896,271.44808133)(268.07652979,271.83610216)(268.14423813,272.41422716)
\lineto(268.43330063,272.41422716)
\lineto(268.43330063,268.82828966)
\lineto(268.14423813,268.82828966)
\curveto(268.13902979,269.23974799)(268.08434229,269.54183133)(267.98017563,269.73453966)
\curveto(267.88121729,269.92724799)(267.74059229,270.07047716)(267.55830063,270.16422716)
\curveto(267.38121729,270.26318549)(267.11559229,270.31266466)(266.76142563,270.31266466)
\lineto(264.83173813,270.31266466)
\lineto(264.83173813,266.98453966)
\curveto(264.83173813,266.44808133)(264.86559229,266.09391466)(264.93330063,265.92203966)
\curveto(264.98538396,265.79183133)(265.09475896,265.67985216)(265.26142563,265.58610216)
\curveto(265.49059229,265.46110216)(265.73017563,265.39860216)(265.98017563,265.39860216)
\lineto(266.36298813,265.39860216)
\lineto(266.36298813,265.10953966)
\lineto(261.81611313,265.10953966)
\lineto(261.81611313,265.39860216)
\lineto(262.19111313,265.39860216)
\curveto(262.62861313,265.39860216)(262.94632146,265.52620633)(263.14423813,265.78141466)
\curveto(263.26923813,265.94808133)(263.33173813,266.34912299)(263.33173813,266.98453966)
\lineto(263.33173813,273.82828966)
\curveto(263.33173813,274.36474799)(263.29788396,274.71891466)(263.23017563,274.89078966)
\curveto(263.17809229,275.02099799)(263.07132146,275.13297716)(262.90986313,275.22672716)
\curveto(262.68590479,275.35172716)(262.44632146,275.41422716)(262.19111313,275.41422716)
\lineto(261.81611313,275.41422716)
\lineto(261.81611313,275.70328966)
\lineto(269.69892563,275.70328966)
\lineto(269.80048813,273.37516466)
\lineto(269.52705063,273.37516466)
\curveto(269.39163396,273.86995633)(269.23277979,274.23193549)(269.05048813,274.46110216)
\curveto(268.87340479,274.69547716)(268.65205063,274.86474799)(268.38642563,274.96891466)
\curveto(268.12600896,275.07308133)(267.71975896,275.12516466)(267.16767563,275.12516466)
\closepath
}
}
{
\newrgbcolor{curcolor}{0 0 0}
\pscustom[linestyle=none,fillstyle=solid,fillcolor=curcolor]
{
\newpath
\moveto(275.01142563,266.14078966)
\curveto(274.27705063,265.57308133)(273.81611313,265.24495633)(273.62861313,265.15641466)
\curveto(273.34736313,265.02620633)(273.04788396,264.96110216)(272.73017563,264.96110216)
\curveto(272.23538396,264.96110216)(271.82652979,265.13037299)(271.50361313,265.46891466)
\curveto(271.18590479,265.80745633)(271.02705063,266.25276883)(271.02705063,266.80485216)
\curveto(271.02705063,267.15381049)(271.10517563,267.45589383)(271.26142563,267.71110216)
\curveto(271.47496729,268.06526883)(271.84475896,268.39860216)(272.37080063,268.71110216)
\curveto(272.90205063,269.02360216)(273.78225896,269.40381049)(275.01142563,269.85172716)
\lineto(275.01142563,270.13297716)
\curveto(275.01142563,270.84651883)(274.89684229,271.33610216)(274.66767563,271.60172716)
\curveto(274.44371729,271.86735216)(274.11559229,272.00016466)(273.68330063,272.00016466)
\curveto(273.35517563,272.00016466)(273.09475896,271.91162299)(272.90205063,271.73453966)
\curveto(272.70413396,271.55745633)(272.60517563,271.35433133)(272.60517563,271.12516466)
\lineto(272.62080063,270.67203966)
\curveto(272.62080063,270.43245633)(272.55830063,270.24756049)(272.43330063,270.11735216)
\curveto(272.31350896,269.98714383)(272.15465479,269.92203966)(271.95673813,269.92203966)
\curveto(271.76402979,269.92203966)(271.60517563,269.98974799)(271.48017563,270.12516466)
\curveto(271.36038396,270.26058133)(271.30048813,270.44547716)(271.30048813,270.67985216)
\curveto(271.30048813,271.12776883)(271.52965479,271.53922716)(271.98798813,271.91422716)
\curveto(272.44632146,272.28922716)(273.08955063,272.47672716)(273.91767563,272.47672716)
\curveto(274.55309229,272.47672716)(275.07392563,272.36995633)(275.48017563,272.15641466)
\curveto(275.78746729,271.99495633)(276.01402979,271.74235216)(276.15986313,271.39860216)
\curveto(276.25361313,271.17464383)(276.30048813,270.71631049)(276.30048813,270.02360216)
\lineto(276.30048813,267.59391466)
\curveto(276.30048813,266.91162299)(276.31350896,266.49235216)(276.33955063,266.33610216)
\curveto(276.36559229,266.18506049)(276.40725896,266.08349799)(276.46455063,266.03141466)
\curveto(276.52705063,265.97933133)(276.59736313,265.95328966)(276.67548813,265.95328966)
\curveto(276.75882146,265.95328966)(276.83173813,265.97151883)(276.89423813,266.00797716)
\curveto(277.00361313,266.07568549)(277.21455063,266.26578966)(277.52705063,266.57828966)
\lineto(277.52705063,266.14078966)
\curveto(276.94371729,265.35953966)(276.38642563,264.96891466)(275.85517563,264.96891466)
\curveto(275.59996729,264.96891466)(275.39684229,265.05745633)(275.24580063,265.23453966)
\curveto(275.09475896,265.41162299)(275.01663396,265.71370633)(275.01142563,266.14078966)
\closepath
\moveto(275.01142563,266.64860216)
\lineto(275.01142563,269.37516466)
\curveto(274.22496729,269.06266466)(273.71715479,268.84131049)(273.48798813,268.71110216)
\curveto(273.07652979,268.48193549)(272.78225896,268.24235216)(272.60517563,267.99235216)
\curveto(272.42809229,267.74235216)(272.33955063,267.46891466)(272.33955063,267.17203966)
\curveto(272.33955063,266.79703966)(272.45152979,266.48453966)(272.67548813,266.23453966)
\curveto(272.89944646,265.98974799)(273.15725896,265.86735216)(273.44892563,265.86735216)
\curveto(273.84475896,265.86735216)(274.36559229,266.12776883)(275.01142563,266.64860216)
\closepath
}
}
{
\newrgbcolor{curcolor}{0 0 0}
\pscustom[linestyle=none,fillstyle=solid,fillcolor=curcolor]
{
\newpath
\moveto(280.15205063,272.47672716)
\lineto(280.15205063,270.86735216)
\curveto(280.75100896,271.94026883)(281.36559229,272.47672716)(281.99580063,272.47672716)
\curveto(282.28225896,272.47672716)(282.51923813,272.38818549)(282.70673813,272.21110216)
\curveto(282.89423813,272.03922716)(282.98798813,271.83870633)(282.98798813,271.60953966)
\curveto(282.98798813,271.40641466)(282.92027979,271.23453966)(282.78486313,271.09391466)
\curveto(282.64944646,270.95328966)(282.48798813,270.88297716)(282.30048813,270.88297716)
\curveto(282.11819646,270.88297716)(281.91246729,270.97151883)(281.68330063,271.14860216)
\curveto(281.45934229,271.33089383)(281.29267563,271.42203966)(281.18330063,271.42203966)
\curveto(281.08955063,271.42203966)(280.98798813,271.36995633)(280.87861313,271.26578966)
\curveto(280.64423813,271.05224799)(280.40205063,270.70068549)(280.15205063,270.21110216)
\lineto(280.15205063,266.78141466)
\curveto(280.15205063,266.38558133)(280.20152979,266.08610216)(280.30048813,265.88297716)
\curveto(280.36819646,265.74235216)(280.48798813,265.62516466)(280.65986313,265.53141466)
\curveto(280.83173813,265.43766466)(281.07913396,265.39078966)(281.40205063,265.39078966)
\lineto(281.40205063,265.10953966)
\lineto(277.73798813,265.10953966)
\lineto(277.73798813,265.39078966)
\curveto(278.10257146,265.39078966)(278.37340479,265.44808133)(278.55048813,265.56266466)
\curveto(278.68069646,265.64599799)(278.77184229,265.77881049)(278.82392563,265.96110216)
\curveto(278.84996729,266.04964383)(278.86298813,266.30224799)(278.86298813,266.71891466)
\lineto(278.86298813,269.49235216)
\curveto(278.86298813,270.32568549)(278.84475896,270.82047716)(278.80830063,270.97672716)
\curveto(278.77705063,271.13818549)(278.71455063,271.25537299)(278.62080063,271.32828966)
\curveto(278.53225896,271.40120633)(278.42027979,271.43766466)(278.28486313,271.43766466)
\curveto(278.12340479,271.43766466)(277.94111313,271.39860216)(277.73798813,271.32047716)
\lineto(277.65986313,271.60172716)
\lineto(279.82392563,272.47672716)
\closepath
}
}
{
\newrgbcolor{curcolor}{0 0 0}
\pscustom[linestyle=none,fillstyle=solid,fillcolor=curcolor]
{
\newpath
\moveto(9.58024025,156.56555176)
\lineto(9.58024025,156.27648926)
\curveto(8.79378192,156.67232259)(8.13753192,157.13586426)(7.61149025,157.66711426)
\curveto(6.86149025,158.42232259)(6.28336525,159.31294759)(5.87711525,160.33898926)
\curveto(5.47086525,161.36503093)(5.26774025,162.43013509)(5.26774025,163.53430176)
\curveto(5.26774025,165.14888509)(5.66617775,166.62023926)(6.46305275,167.94836426)
\curveto(7.25992775,169.28169759)(8.29899025,170.23482259)(9.58024025,170.80773926)
\lineto(9.58024025,170.47961426)
\curveto(8.93961525,170.12544759)(8.41357358,169.64107259)(8.00211525,169.02648926)
\curveto(7.59065692,168.41190593)(7.28336525,167.63326009)(7.08024025,166.69055176)
\curveto(6.87711525,165.74784343)(6.77555275,164.76346843)(6.77555275,163.73742676)
\curveto(6.77555275,162.62284343)(6.86149025,161.60982259)(7.03336525,160.69836426)
\curveto(7.16878192,159.97961426)(7.33284442,159.40409343)(7.52555275,158.97180176)
\curveto(7.71826108,158.53430176)(7.97607358,158.11503093)(8.29899025,157.71398926)
\curveto(8.62711525,157.31294759)(9.05419858,156.93013509)(9.58024025,156.56555176)
\closepath
}
}
{
\newrgbcolor{curcolor}{0 0 0}
\pscustom[linestyle=none,fillstyle=solid,fillcolor=curcolor]
{
\newpath
\moveto(14.49430275,160.72961426)
\curveto(13.75992775,160.16190593)(13.29899025,159.83378093)(13.11149025,159.74523926)
\curveto(12.83024025,159.61503093)(12.53076108,159.54992676)(12.21305275,159.54992676)
\curveto(11.71826108,159.54992676)(11.30940692,159.71919759)(10.98649025,160.05773926)
\curveto(10.66878192,160.39628093)(10.50992775,160.84159343)(10.50992775,161.39367676)
\curveto(10.50992775,161.74263509)(10.58805275,162.04471843)(10.74430275,162.29992676)
\curveto(10.95784442,162.65409343)(11.32763608,162.98742676)(11.85367775,163.29992676)
\curveto(12.38492775,163.61242676)(13.26513608,163.99263509)(14.49430275,164.44055176)
\lineto(14.49430275,164.72180176)
\curveto(14.49430275,165.43534343)(14.37971942,165.92492676)(14.15055275,166.19055176)
\curveto(13.92659442,166.45617676)(13.59846942,166.58898926)(13.16617775,166.58898926)
\curveto(12.83805275,166.58898926)(12.57763608,166.50044759)(12.38492775,166.32336426)
\curveto(12.18701108,166.14628093)(12.08805275,165.94315593)(12.08805275,165.71398926)
\lineto(12.10367775,165.26086426)
\curveto(12.10367775,165.02128093)(12.04117775,164.83638509)(11.91617775,164.70617676)
\curveto(11.79638608,164.57596843)(11.63753192,164.51086426)(11.43961525,164.51086426)
\curveto(11.24690692,164.51086426)(11.08805275,164.57857259)(10.96305275,164.71398926)
\curveto(10.84326108,164.84940593)(10.78336525,165.03430176)(10.78336525,165.26867676)
\curveto(10.78336525,165.71659343)(11.01253192,166.12805176)(11.47086525,166.50305176)
\curveto(11.92919858,166.87805176)(12.57242775,167.06555176)(13.40055275,167.06555176)
\curveto(14.03596942,167.06555176)(14.55680275,166.95878093)(14.96305275,166.74523926)
\curveto(15.27034442,166.58378093)(15.49690692,166.33117676)(15.64274025,165.98742676)
\curveto(15.73649025,165.76346843)(15.78336525,165.30513509)(15.78336525,164.61242676)
\lineto(15.78336525,162.18273926)
\curveto(15.78336525,161.50044759)(15.79638608,161.08117676)(15.82242775,160.92492676)
\curveto(15.84846942,160.77388509)(15.89013608,160.67232259)(15.94742775,160.62023926)
\curveto(16.00992775,160.56815593)(16.08024025,160.54211426)(16.15836525,160.54211426)
\curveto(16.24169858,160.54211426)(16.31461525,160.56034343)(16.37711525,160.59680176)
\curveto(16.48649025,160.66451009)(16.69742775,160.85461426)(17.00992775,161.16711426)
\lineto(17.00992775,160.72961426)
\curveto(16.42659442,159.94836426)(15.86930275,159.55773926)(15.33805275,159.55773926)
\curveto(15.08284442,159.55773926)(14.87971942,159.64628093)(14.72867775,159.82336426)
\curveto(14.57763608,160.00044759)(14.49951108,160.30253093)(14.49430275,160.72961426)
\closepath
\moveto(14.49430275,161.23742676)
\lineto(14.49430275,163.96398926)
\curveto(13.70784442,163.65148926)(13.20003192,163.43013509)(12.97086525,163.29992676)
\curveto(12.55940692,163.07076009)(12.26513608,162.83117676)(12.08805275,162.58117676)
\curveto(11.91096942,162.33117676)(11.82242775,162.05773926)(11.82242775,161.76086426)
\curveto(11.82242775,161.38586426)(11.93440692,161.07336426)(12.15836525,160.82336426)
\curveto(12.38232358,160.57857259)(12.64013608,160.45617676)(12.93180275,160.45617676)
\curveto(13.32763608,160.45617676)(13.84846942,160.71659343)(14.49430275,161.23742676)
\closepath
}
}
{
\newrgbcolor{curcolor}{0 0 0}
\pscustom[linestyle=none,fillstyle=solid,fillcolor=curcolor]
{
\newpath
\moveto(17.40055275,170.47961426)
\lineto(17.40055275,170.80773926)
\curveto(18.19221942,170.41711426)(18.85107358,169.95617676)(19.37711525,169.42492676)
\curveto(20.12190692,168.66451009)(20.69742775,167.77128093)(21.10367775,166.74523926)
\curveto(21.50992775,165.72440593)(21.71305275,164.65930176)(21.71305275,163.54992676)
\curveto(21.71305275,161.93534343)(21.31461525,160.46398926)(20.51774025,159.13586426)
\curveto(19.72607358,157.80253093)(18.68701108,156.84940593)(17.40055275,156.27648926)
\lineto(17.40055275,156.56555176)
\curveto(18.04117775,156.92492676)(18.56721942,157.41190593)(18.97867775,158.02648926)
\curveto(19.39534442,158.63586426)(19.70263608,159.41190593)(19.90055275,160.35461426)
\curveto(20.10367775,161.30253093)(20.20524025,162.28951009)(20.20524025,163.31555176)
\curveto(20.20524025,164.42492676)(20.11930275,165.43794759)(19.94742775,166.35461426)
\curveto(19.81721942,167.07336426)(19.65315692,167.64888509)(19.45524025,168.08117676)
\curveto(19.26253192,168.51346843)(19.00471942,168.93013509)(18.68180275,169.33117676)
\curveto(18.35888608,169.73221843)(17.93180275,170.11503093)(17.40055275,170.47961426)
\closepath
}
}
{
\newrgbcolor{curcolor}{0 0 0}
\pscustom[linestyle=none,fillstyle=solid,fillcolor=curcolor]
{
\newpath
\moveto(181.50323625,157.23819596)
\lineto(181.50323625,156.94913346)
\curveto(180.71677792,157.34496679)(180.06052792,157.80850846)(179.53448625,158.33975846)
\curveto(178.78448625,159.09496679)(178.20636125,159.98559179)(177.80011125,161.01163346)
\curveto(177.39386125,162.03767513)(177.19073625,163.10277929)(177.19073625,164.20694596)
\curveto(177.19073625,165.82152929)(177.58917375,167.29288346)(178.38604875,168.62100846)
\curveto(179.18292375,169.95434179)(180.22198625,170.90746679)(181.50323625,171.48038346)
\lineto(181.50323625,171.15225846)
\curveto(180.86261125,170.79809179)(180.33656958,170.31371679)(179.92511125,169.69913346)
\curveto(179.51365292,169.08455013)(179.20636125,168.30590429)(179.00323625,167.36319596)
\curveto(178.80011125,166.42048763)(178.69854875,165.43611263)(178.69854875,164.41007096)
\curveto(178.69854875,163.29548763)(178.78448625,162.28246679)(178.95636125,161.37100846)
\curveto(179.09177792,160.65225846)(179.25584042,160.07673763)(179.44854875,159.64444596)
\curveto(179.64125708,159.20694596)(179.89906958,158.78767513)(180.22198625,158.38663346)
\curveto(180.55011125,157.98559179)(180.97719458,157.60277929)(181.50323625,157.23819596)
\closepath
}
}
{
\newrgbcolor{curcolor}{0 0 0}
\pscustom[linestyle=none,fillstyle=solid,fillcolor=curcolor]
{
\newpath
\moveto(184.32354875,166.29288346)
\curveto(185.01625708,167.25642513)(185.76365292,167.73819596)(186.56573625,167.73819596)
\curveto(187.30011125,167.73819596)(187.94073625,167.42309179)(188.48761125,166.79288346)
\curveto(189.03448625,166.16788346)(189.30792375,165.31111263)(189.30792375,164.22257096)
\curveto(189.30792375,162.95173763)(188.88604875,161.92830013)(188.04229875,161.15225846)
\curveto(187.31834042,160.48559179)(186.51104875,160.15225846)(185.62042375,160.15225846)
\curveto(185.20375708,160.15225846)(184.77927792,160.22777929)(184.34698625,160.37882096)
\curveto(183.91990292,160.52986263)(183.48240292,160.75642513)(183.03448625,161.05850846)
\lineto(183.03448625,168.47257096)
\curveto(183.03448625,169.28507096)(183.01365292,169.78507096)(182.97198625,169.97257096)
\curveto(182.93552792,170.16007096)(182.87563208,170.28767513)(182.79229875,170.35538346)
\curveto(182.70896542,170.42309179)(182.60479875,170.45694596)(182.47979875,170.45694596)
\curveto(182.33396542,170.45694596)(182.15167375,170.41527929)(181.93292375,170.33194596)
\lineto(181.82354875,170.60538346)
\lineto(183.97198625,171.48038346)
\lineto(184.32354875,171.48038346)
\closepath
\moveto(184.32354875,165.79288346)
\lineto(184.32354875,161.51163346)
\curveto(184.58917375,161.25121679)(184.86261125,161.05330013)(185.14386125,160.91788346)
\curveto(185.43031958,160.78767513)(185.72198625,160.72257096)(186.01886125,160.72257096)
\curveto(186.49281958,160.72257096)(186.93292375,160.98298763)(187.33917375,161.50382096)
\curveto(187.75063208,162.02465429)(187.95636125,162.78246679)(187.95636125,163.77725846)
\curveto(187.95636125,164.69392513)(187.75063208,165.39705013)(187.33917375,165.88663346)
\curveto(186.93292375,166.38142513)(186.46938208,166.62882096)(185.94854875,166.62882096)
\curveto(185.67250708,166.62882096)(185.39646542,166.55850846)(185.12042375,166.41788346)
\curveto(184.91209042,166.31371679)(184.64646542,166.10538346)(184.32354875,165.79288346)
\closepath
}
}
{
\newrgbcolor{curcolor}{0 0 0}
\pscustom[linestyle=none,fillstyle=solid,fillcolor=curcolor]
{
\newpath
\moveto(190.22198625,171.15225846)
\lineto(190.22198625,171.48038346)
\curveto(191.01365292,171.08975846)(191.67250708,170.62882096)(192.19854875,170.09757096)
\curveto(192.94334042,169.33715429)(193.51886125,168.44392513)(193.92511125,167.41788346)
\curveto(194.33136125,166.39705013)(194.53448625,165.33194596)(194.53448625,164.22257096)
\curveto(194.53448625,162.60798763)(194.13604875,161.13663346)(193.33917375,159.80850846)
\curveto(192.54750708,158.47517513)(191.50844458,157.52205013)(190.22198625,156.94913346)
\lineto(190.22198625,157.23819596)
\curveto(190.86261125,157.59757096)(191.38865292,158.08455013)(191.80011125,158.69913346)
\curveto(192.21677792,159.30850846)(192.52406958,160.08455013)(192.72198625,161.02725846)
\curveto(192.92511125,161.97517513)(193.02667375,162.96215429)(193.02667375,163.98819596)
\curveto(193.02667375,165.09757096)(192.94073625,166.11059179)(192.76886125,167.02725846)
\curveto(192.63865292,167.74600846)(192.47459042,168.32152929)(192.27667375,168.75382096)
\curveto(192.08396542,169.18611263)(191.82615292,169.60277929)(191.50323625,170.00382096)
\curveto(191.18031958,170.40486263)(190.75323625,170.78767513)(190.22198625,171.15225846)
\closepath
}
}
{
\newrgbcolor{curcolor}{0 0 0}
\pscustom[linestyle=none,fillstyle=solid,fillcolor=curcolor]
{
\newpath
\moveto(8.54298225,2.26580966)
\lineto(8.54298225,1.97674716)
\curveto(7.75652392,2.37258049)(7.10027392,2.83612216)(6.57423225,3.36737216)
\curveto(5.82423225,4.12258049)(5.24610725,5.01320549)(4.83985725,6.03924716)
\curveto(4.43360725,7.06528883)(4.23048225,8.13039299)(4.23048225,9.23455966)
\curveto(4.23048225,10.84914299)(4.62891975,12.32049716)(5.42579475,13.64862216)
\curveto(6.22266975,14.98195549)(7.26173225,15.93508049)(8.54298225,16.50799716)
\lineto(8.54298225,16.17987216)
\curveto(7.90235725,15.82570549)(7.37631558,15.34133049)(6.96485725,14.72674716)
\curveto(6.55339892,14.11216383)(6.24610725,13.33351799)(6.04298225,12.39080966)
\curveto(5.83985725,11.44810133)(5.73829475,10.46372633)(5.73829475,9.43768466)
\curveto(5.73829475,8.32310133)(5.82423225,7.31008049)(5.99610725,6.39862216)
\curveto(6.13152392,5.67987216)(6.29558642,5.10435133)(6.48829475,4.67205966)
\curveto(6.68100308,4.23455966)(6.93881558,3.81528883)(7.26173225,3.41424716)
\curveto(7.58985725,3.01320549)(8.01694058,2.63039299)(8.54298225,2.26580966)
\closepath
}
}
{
\newrgbcolor{curcolor}{0 0 0}
\pscustom[linestyle=none,fillstyle=solid,fillcolor=curcolor]
{
\newpath
\moveto(15.48048225,8.11737216)
\curveto(15.28777392,7.17466383)(14.91016975,6.44810133)(14.34766975,5.93768466)
\curveto(13.78516975,5.43247633)(13.16277392,5.17987216)(12.48048225,5.17987216)
\curveto(11.66798225,5.17987216)(10.95964892,5.52101799)(10.35548225,6.20330966)
\curveto(9.75131558,6.88560133)(9.44923225,7.80747633)(9.44923225,8.96893466)
\curveto(9.44923225,10.09393466)(9.78256558,11.00799716)(10.44923225,11.71112216)
\curveto(11.12110725,12.41424716)(11.92579475,12.76580966)(12.86329475,12.76580966)
\curveto(13.56641975,12.76580966)(14.14454475,12.57830966)(14.59766975,12.20330966)
\curveto(15.05079475,11.83351799)(15.27735725,11.44810133)(15.27735725,11.04705966)
\curveto(15.27735725,10.84914299)(15.21225308,10.68768466)(15.08204475,10.56268466)
\curveto(14.95704475,10.44289299)(14.77996142,10.38299716)(14.55079475,10.38299716)
\curveto(14.24350308,10.38299716)(14.01173225,10.48195549)(13.85548225,10.67987216)
\curveto(13.76694058,10.78924716)(13.70704475,10.99758049)(13.67579475,11.30487216)
\curveto(13.64975308,11.61216383)(13.54558642,11.84653883)(13.36329475,12.00799716)
\curveto(13.18100308,12.16424716)(12.92839892,12.24237216)(12.60548225,12.24237216)
\curveto(12.08464892,12.24237216)(11.66537808,12.04966383)(11.34766975,11.66424716)
\curveto(10.92579475,11.15383049)(10.71485725,10.47935133)(10.71485725,9.64080966)
\curveto(10.71485725,8.78664299)(10.92319058,8.03143466)(11.33985725,7.37518466)
\curveto(11.76173225,6.72414299)(12.32944058,6.39862216)(13.04298225,6.39862216)
\curveto(13.55339892,6.39862216)(14.01173225,6.57310133)(14.41798225,6.92205966)
\curveto(14.70444058,7.16164299)(14.98308642,7.59653883)(15.25391975,8.22674716)
\closepath
}
}
{
\newrgbcolor{curcolor}{0 0 0}
\pscustom[linestyle=none,fillstyle=solid,fillcolor=curcolor]
{
\newpath
\moveto(16.36329475,16.17987216)
\lineto(16.36329475,16.50799716)
\curveto(17.15496142,16.11737216)(17.81381558,15.65643466)(18.33985725,15.12518466)
\curveto(19.08464892,14.36476799)(19.66016975,13.47153883)(20.06641975,12.44549716)
\curveto(20.47266975,11.42466383)(20.67579475,10.35955966)(20.67579475,9.25018466)
\curveto(20.67579475,7.63560133)(20.27735725,6.16424716)(19.48048225,4.83612216)
\curveto(18.68881558,3.50278883)(17.64975308,2.54966383)(16.36329475,1.97674716)
\lineto(16.36329475,2.26580966)
\curveto(17.00391975,2.62518466)(17.52996142,3.11216383)(17.94141975,3.72674716)
\curveto(18.35808642,4.33612216)(18.66537808,5.11216383)(18.86329475,6.05487216)
\curveto(19.06641975,7.00278883)(19.16798225,7.98976799)(19.16798225,9.01580966)
\curveto(19.16798225,10.12518466)(19.08204475,11.13820549)(18.91016975,12.05487216)
\curveto(18.77996142,12.77362216)(18.61589892,13.34914299)(18.41798225,13.78143466)
\curveto(18.22527392,14.21372633)(17.96746142,14.63039299)(17.64454475,15.03143466)
\curveto(17.32162808,15.43247633)(16.89454475,15.81528883)(16.36329475,16.17987216)
\closepath
}
}
{
\newrgbcolor{curcolor}{0 0 0}
\pscustom[linestyle=none,fillstyle=solid,fillcolor=curcolor]
{
\newpath
\moveto(181.50323625,2.26580966)
\lineto(181.50323625,1.97674716)
\curveto(180.71677792,2.37258049)(180.06052792,2.83612216)(179.53448625,3.36737216)
\curveto(178.78448625,4.12258049)(178.20636125,5.01320549)(177.80011125,6.03924716)
\curveto(177.39386125,7.06528883)(177.19073625,8.13039299)(177.19073625,9.23455966)
\curveto(177.19073625,10.84914299)(177.58917375,12.32049716)(178.38604875,13.64862216)
\curveto(179.18292375,14.98195549)(180.22198625,15.93508049)(181.50323625,16.50799716)
\lineto(181.50323625,16.17987216)
\curveto(180.86261125,15.82570549)(180.33656958,15.34133049)(179.92511125,14.72674716)
\curveto(179.51365292,14.11216383)(179.20636125,13.33351799)(179.00323625,12.39080966)
\curveto(178.80011125,11.44810133)(178.69854875,10.46372633)(178.69854875,9.43768466)
\curveto(178.69854875,8.32310133)(178.78448625,7.31008049)(178.95636125,6.39862216)
\curveto(179.09177792,5.67987216)(179.25584042,5.10435133)(179.44854875,4.67205966)
\curveto(179.64125708,4.23455966)(179.89906958,3.81528883)(180.22198625,3.41424716)
\curveto(180.55011125,3.01320549)(180.97719458,2.63039299)(181.50323625,2.26580966)
\closepath
}
}
{
\newrgbcolor{curcolor}{0 0 0}
\pscustom[linestyle=none,fillstyle=solid,fillcolor=curcolor]
{
\newpath
\moveto(187.41729875,6.20330966)
\curveto(187.06834042,5.83872633)(186.72719458,5.57570549)(186.39386125,5.41424716)
\curveto(186.06052792,5.25799716)(185.70115292,5.17987216)(185.31573625,5.17987216)
\curveto(184.53448625,5.17987216)(183.85219458,5.50539299)(183.26886125,6.15643466)
\curveto(182.68552792,6.81268466)(182.39386125,7.65383049)(182.39386125,8.67987216)
\curveto(182.39386125,9.70591383)(182.71677792,10.64341383)(183.36261125,11.49237216)
\curveto(184.00844458,12.34653883)(184.83917375,12.77362216)(185.85479875,12.77362216)
\curveto(186.48500708,12.77362216)(187.00584042,12.57310133)(187.41729875,12.17205966)
\lineto(187.41729875,13.49237216)
\curveto(187.41729875,14.31008049)(187.39646542,14.81268466)(187.35479875,15.00018466)
\curveto(187.31834042,15.18768466)(187.25844458,15.31528883)(187.17511125,15.38299716)
\curveto(187.09177792,15.45070549)(186.98761125,15.48455966)(186.86261125,15.48455966)
\curveto(186.72719458,15.48455966)(186.54750708,15.44289299)(186.32354875,15.35955966)
\lineto(186.22198625,15.63299716)
\lineto(188.35479875,16.50799716)
\lineto(188.70636125,16.50799716)
\lineto(188.70636125,8.23455966)
\curveto(188.70636125,7.39601799)(188.72459042,6.88299716)(188.76104875,6.69549716)
\curveto(188.80271542,6.51320549)(188.86521542,6.38560133)(188.94854875,6.31268466)
\curveto(189.03709042,6.23976799)(189.13865292,6.20330966)(189.25323625,6.20330966)
\curveto(189.39386125,6.20330966)(189.58136125,6.24758049)(189.81573625,6.33612216)
\lineto(189.90167375,6.06268466)
\lineto(187.77667375,5.17987216)
\lineto(187.41729875,5.17987216)
\closepath
\moveto(187.41729875,6.75018466)
\lineto(187.41729875,10.43768466)
\curveto(187.38604875,10.79185133)(187.29229875,11.11476799)(187.13604875,11.40643466)
\curveto(186.97979875,11.69810133)(186.77146542,11.91685133)(186.51104875,12.06268466)
\curveto(186.25584042,12.21372633)(186.00584042,12.28924716)(185.76104875,12.28924716)
\curveto(185.30271542,12.28924716)(184.89386125,12.08351799)(184.53448625,11.67205966)
\curveto(184.06052792,11.13039299)(183.82354875,10.33872633)(183.82354875,9.29705966)
\curveto(183.82354875,8.24497633)(184.05271542,7.43768466)(184.51104875,6.87518466)
\curveto(184.96938208,6.31789299)(185.47979875,6.03924716)(186.04229875,6.03924716)
\curveto(186.51625708,6.03924716)(186.97459042,6.27622633)(187.41729875,6.75018466)
\closepath
}
}
{
\newrgbcolor{curcolor}{0 0 0}
\pscustom[linestyle=none,fillstyle=solid,fillcolor=curcolor]
{
\newpath
\moveto(190.22198625,16.17987216)
\lineto(190.22198625,16.50799716)
\curveto(191.01365292,16.11737216)(191.67250708,15.65643466)(192.19854875,15.12518466)
\curveto(192.94334042,14.36476799)(193.51886125,13.47153883)(193.92511125,12.44549716)
\curveto(194.33136125,11.42466383)(194.53448625,10.35955966)(194.53448625,9.25018466)
\curveto(194.53448625,7.63560133)(194.13604875,6.16424716)(193.33917375,4.83612216)
\curveto(192.54750708,3.50278883)(191.50844458,2.54966383)(190.22198625,1.97674716)
\lineto(190.22198625,2.26580966)
\curveto(190.86261125,2.62518466)(191.38865292,3.11216383)(191.80011125,3.72674716)
\curveto(192.21677792,4.33612216)(192.52406958,5.11216383)(192.72198625,6.05487216)
\curveto(192.92511125,7.00278883)(193.02667375,7.98976799)(193.02667375,9.01580966)
\curveto(193.02667375,10.12518466)(192.94073625,11.13820549)(192.76886125,12.05487216)
\curveto(192.63865292,12.77362216)(192.47459042,13.34914299)(192.27667375,13.78143466)
\curveto(192.08396542,14.21372633)(191.82615292,14.63039299)(191.50323625,15.03143466)
\curveto(191.18031958,15.43247633)(190.75323625,15.81528883)(190.22198625,16.17987216)
\closepath
}
}
\end{pspicture}

    \caption{光线遍历穿过kd树。(a)光线与树的边框相交,
    给出了要考虑的初始参数范围$[t_{\min},t_{\max}]$。
    (b)因为该范围非空,所以这里需要考虑根节点的两个孩子。
    光线首先进入左侧孩子,标记为“near”,有参数范围$[t_{\min},t_{\text{split}}]$。
    如果近处节点是含有图元的叶子节点,则执行光线-图元相交测试;
    否则处理其孩子节点。(c)如果该节点内没有找到命中处,或者找到的命中处
    超出了$[t_{\min},t_{\text{split}}]$,则处理右边的远处节点。
    (d)继续该过程——按深度优先处理树节点,从前往后遍历——直到
    求得最近相交处或者光线退出该树。}
    \label{fig:4.17}
\end{figure}
\begin{lstlisting}
`\refcode{KdTreeAccel Method Definitions}{+=}\lastcode{KdTreeAccelMethodDefinitions}`
bool `\refvar{KdTreeAccel}{}`::`\initvar[KdTreeAccel::Intersect]{Intersect}{}`(const `\refvar{Ray}{}` &ray,
        `\refvar{SurfaceInteraction}{}` *isect) const {
    `\refcode{Compute initial parametric range of ray inside kd-tree extent}{}`
    `\refcode{Prepare to traverse kd-tree for ray}{}`
    `\refcode{Traverse kd-tree nodes in order for ray}{}`
}
\end{lstlisting}

算法从寻找光线与树重合的整体参数范围$[t_{\min},t_{\max}]$开始,
如果没有重合则立即退出。
\begin{lstlisting}
`\initcode{Compute initial parametric range of ray inside kd-tree extent}{=}`
`\refvar{Float}{}` tMin, tMax;
if (!bounds.`\refvar[Bounds3::IntersectP]{IntersectP}{}`(ray, &tMin, &tMax)) 
    return false;
\end{lstlisting}

结构体\refvar{KdToDo}{}数组用于记录该光线目前要处理的节点;
它排了序使得数组中最后一个活跃项是应该考虑的下一个节点。
该数组中需要的最大项数是kd树的最大深度;
下文所用的数组大小在实践中应该是绰绰有余的。
\begin{lstlisting}
`\initcode{Prepare to traverse kd-tree for ray}{=}`
`\refvar{Vector3f}{}` invDir(1 / ray.`\refvar[Ray::d]{d}{}`.x, 1 / ray.`\refvar[Ray::d]{d}{}`.y, 1 / ray.`\refvar[Ray::d]{d}{}`.z);
constexpr int maxTodo = 64;
`\refvar{KdToDo}{}` todo[maxTodo];
int todoPos = 0;
\end{lstlisting}
\begin{lstlisting}
`\refcode{KdTreeAccel Declarations}{+=}\lastcode{KdTreeAccelDeclarations}`
struct `\initvar{KdToDo}{}` {
    const `\refvar{KdAccelNode}{}` *`\initvar[KdToDo::node]{node}{}`;
    `\refvar{Float}{}` `\initvar[KdToDo::tMin]{tMin}{}`, `\initvar[KdToDo::tMax]{tMax}{}`;
};
\end{lstlisting}

遍历继续穿过节点,循环中每次处理单个叶子或内部节点。
值\refvar[KdToDo::tMin]{tMin}{}和\refvar[KdToDo::tMax]{tMax}{}总是持有
光线与当前节点重合的参数范围。
\begin{lstlisting}
`\initcode{Traverse kd-tree nodes in order for ray}{=}`
bool hit = false;
const `\refvar{KdAccelNode}{}` *node = &`\refvar[KdTreeAccel::nodes]{nodes}{}`[0];
while (node != nullptr) {
    `\refcode{Bail out if we found a hit closer than the current node}{}`
    if (!node->`\refvar{IsLeaf}{}`()) {
        `\refcode{Process kd-tree interior node}{}`
    } else {
        `\refcode{Check for intersections inside leaf node}{}`
        `\refcode{Grab next node to process from todo list}{}`
    }
}
return hit;
\end{lstlisting}

对于和多个节点重合的图元可能以前就找到相交处了。
首次检测到时如果相交处在当前节点之外,
则有必要继续遍历树直到我们遇到一个节点的{\ttfamily tMin}超过相交处。
只有这时才能确定和其他图元不会再有更近的相交处了。
\begin{lstlisting}
`\initcode{Bail out if we found a hit closer than the current node}{=}`
if (ray.`\refvar{tMax}{}` < tMin) break;
\end{lstlisting}

对于内部节点要做的第一件事是让光线与节点的划分平面相交;
有了交点后,我们可以确定需要处理一个还是两个孩子节点以及光线穿过它们的顺序。
\begin{lstlisting}
`\initcode{Process kd-tree interior node}{=}`
`\refcode{Compute parametric distance along ray to split plane}{}`
`\refcode{Get node children pointers for ray}{}`
`\refcode{Advance to next child node, possibly enqueue other child}{}`
\end{lstlisting}

按照在光线-边界框测试中和计算光线与轴对齐平面相交一样的方式计算到划分平面的参数距离。
循环中我们每次用预先计算的值{\ttfamily invDir}保存除数。
\begin{lstlisting}
`\initcode{Compute parametric distance along ray to split plane}{=}`
int axis = node->`\refvar[KdAccelNode::SplitAxis]{SplitAxis}{}`();
`\refvar{Float}{}` tPlane = (node->`\refvar{SplitPos}{}`() - ray.`\refvar[Ray::o]{o}{}`[axis]) * invDir[axis];
\end{lstlisting}

现在需要确定光线遇到孩子节点的顺序使得是沿光线按从前往后的顺序遍历树的。
\reffig{4.18}展示了该计算的几何结构。
射线端点关于划分平面的位置足够区分两种情况,
现在忽略光线实际上没有穿过两节点之一的情况。
射线端点位于划分平面上的罕见情况需仔细处理,
需要改用它的方向来区分两种情况。
\begin{figure}[htbp]
    \centering%LaTeX with PSTricks extensions
%%Creator: Inkscape 1.0.1 (3bc2e813f5, 2020-09-07)
%%Please note this file requires PSTricks extensions
\psset{xunit=.5pt,yunit=.5pt,runit=.5pt}
\begin{pspicture}(209.97000122,125.19000244)
{
\newrgbcolor{curcolor}{0 0 0}
\pscustom[linewidth=1,linecolor=curcolor]
{
\newpath
\moveto(27.87000084,124.69000244)
\lineto(183.61000633,124.69000244)
\lineto(183.61000633,0.70000458)
\lineto(27.87000084,0.70000458)
\closepath
}
}
{
\newrgbcolor{curcolor}{0 0 0}
\pscustom[linewidth=1,linecolor=curcolor]
{
\newpath
\moveto(4.73000002,86.68000412)
\lineto(201.8999939,105.18000221)
}
}
{
\newrgbcolor{curcolor}{0 0 0}
\pscustom[linestyle=none,fillstyle=solid,fillcolor=curcolor]
{
\newpath
\moveto(197.52,99.24000244)
\lineto(201.25,105.12000244)
\lineto(196.5,110.20000244)
\lineto(209.97,105.94000244)
\closepath
}
}
{
\newrgbcolor{curcolor}{0.65098041 0.65098041 0.65098041}
\pscustom[linestyle=none,fillstyle=solid,fillcolor=curcolor]
{
\newpath
\moveto(198.96,100.59000244)
\lineto(208.66,105.82000244)
\lineto(201.88,105.18000244)
\closepath
}
}
{
\newrgbcolor{curcolor}{0.40000001 0.40000001 0.40000001}
\pscustom[linestyle=none,fillstyle=solid,fillcolor=curcolor]
{
\newpath
\moveto(198.16,109.15000244)
\lineto(208.66,105.82000244)
\lineto(201.88,105.18000244)
\closepath
}
}
{
\newrgbcolor{curcolor}{0 0 0}
\pscustom[linewidth=1,linecolor=curcolor]
{
\newpath
\moveto(201.52000427,35.6700058)
\lineto(8.10000038,41.18000031)
}
}
{
\newrgbcolor{curcolor}{0 0 0}
\pscustom[linestyle=none,fillstyle=solid,fillcolor=curcolor]
{
\newpath
\moveto(13.16,46.54000244)
\lineto(8.75,41.16000244)
\lineto(12.85,35.54000244)
\lineto(0,41.41000244)
\closepath
}
}
{
\newrgbcolor{curcolor}{0.65098041 0.65098041 0.65098041}
\pscustom[linestyle=none,fillstyle=solid,fillcolor=curcolor]
{
\newpath
\moveto(11.57,45.39000244)
\lineto(1.31,41.38000244)
\lineto(8.12,41.18000244)
\closepath
}
}
{
\newrgbcolor{curcolor}{0.40000001 0.40000001 0.40000001}
\pscustom[linestyle=none,fillstyle=solid,fillcolor=curcolor]
{
\newpath
\moveto(11.33,36.79000244)
\lineto(1.31,41.38000244)
\lineto(8.12,41.18000244)
\closepath
}
}
{
\newrgbcolor{curcolor}{0 0 0}
\pscustom[linewidth=1,linecolor=curcolor,linestyle=dashed,dash=2]
{
\newpath
\moveto(121.11000061,124.86000243)
\lineto(121.11000061,0.52999878)
}
}
{
\newrgbcolor{curcolor}{0 0 0}
\pscustom[linestyle=none,fillstyle=solid,fillcolor=curcolor]
{
\newpath
\moveto(60.39618952,8.62717504)
\curveto(61.13056452,8.47092504)(61.68004369,8.22092504)(62.04462702,7.87717504)
\curveto(62.54983536,7.39800837)(62.80243952,6.81207087)(62.80243952,6.11936254)
\curveto(62.80243952,5.59332087)(62.63577286,5.08811254)(62.30243952,4.60373754)
\curveto(61.96910619,4.12457087)(61.51077286,3.77300837)(60.92743952,3.54905004)
\curveto(60.34931452,3.33030004)(59.46389786,3.22092504)(58.27118952,3.22092504)
\lineto(53.27118952,3.22092504)
\lineto(53.27118952,3.50998754)
\lineto(53.66962702,3.50998754)
\curveto(54.11233536,3.50998754)(54.43004369,3.65061254)(54.62275202,3.93186254)
\curveto(54.74254369,4.1141542)(54.80243952,4.50217504)(54.80243952,5.09592504)
\lineto(54.80243952,11.93967504)
\curveto(54.80243952,12.59592504)(54.72691869,13.00998754)(54.57587702,13.18186254)
\curveto(54.37275202,13.4110292)(54.07066869,13.52561254)(53.66962702,13.52561254)
\lineto(53.27118952,13.52561254)
\lineto(53.27118952,13.81467504)
\lineto(57.84931452,13.81467504)
\curveto(58.70348119,13.81467504)(59.38837702,13.75217504)(59.90400202,13.62717504)
\curveto(60.68525202,13.43967504)(61.28160619,13.1063417)(61.69306452,12.62717504)
\curveto(62.10452286,12.1532167)(62.31025202,11.6063417)(62.31025202,10.98655004)
\curveto(62.31025202,10.45530004)(62.14879369,9.97873754)(61.82587702,9.55686254)
\curveto(61.50296036,9.14019587)(61.02639786,8.83030004)(60.39618952,8.62717504)
\closepath
\moveto(56.30243952,9.04905004)
\curveto(56.49514786,9.0125917)(56.71389786,8.98394587)(56.95868952,8.96311254)
\curveto(57.20868952,8.94748754)(57.48212702,8.93967504)(57.77900202,8.93967504)
\curveto(58.53941869,8.93967504)(59.10973119,9.0204042)(59.48993952,9.18186254)
\curveto(59.87535619,9.3485292)(60.16962702,9.60113337)(60.37275202,9.93967504)
\curveto(60.57587702,10.2782167)(60.67743952,10.64800837)(60.67743952,11.04905004)
\curveto(60.67743952,11.6688417)(60.42483536,12.19748754)(59.91962702,12.63498754)
\curveto(59.41441869,13.07248754)(58.67743952,13.29123754)(57.70868952,13.29123754)
\curveto(57.18785619,13.29123754)(56.71910619,13.23394587)(56.30243952,13.11936254)
\closepath
\moveto(56.30243952,3.98655004)
\curveto(56.90660619,3.84592504)(57.50296036,3.77561254)(58.09150202,3.77561254)
\curveto(59.03421036,3.77561254)(59.75296036,3.98655004)(60.24775202,4.40842504)
\curveto(60.74254369,4.83550837)(60.98993952,5.36155004)(60.98993952,5.98655004)
\curveto(60.98993952,6.39800837)(60.87796036,6.7938417)(60.65400202,7.17405004)
\curveto(60.43004369,7.55425837)(60.06546036,7.85373754)(59.56025202,8.07248754)
\curveto(59.05504369,8.29123754)(58.43004369,8.40061254)(57.68525202,8.40061254)
\curveto(57.36233536,8.40061254)(57.08629369,8.3954042)(56.85712702,8.38498754)
\curveto(56.62796036,8.37457087)(56.44306452,8.3563417)(56.30243952,8.33030004)
\closepath
}
}
{
\newrgbcolor{curcolor}{0 0 0}
\pscustom[linestyle=none,fillstyle=solid,fillcolor=curcolor]
{
\newpath
\moveto(65.38056452,7.68186254)
\curveto(65.37535619,6.61936254)(65.63316869,5.7860292)(66.15400202,5.18186254)
\curveto(66.67483536,4.57769587)(67.28681452,4.27561254)(67.98993952,4.27561254)
\curveto(68.45868952,4.27561254)(68.86493952,4.4032167)(69.20868952,4.65842504)
\curveto(69.55764786,4.9188417)(69.84931452,5.36155004)(70.08368952,5.98655004)
\lineto(70.32587702,5.83030004)
\curveto(70.21650202,5.11675837)(69.89879369,4.4657167)(69.37275202,3.87717504)
\curveto(68.84671036,3.2938417)(68.18785619,3.00217504)(67.39618952,3.00217504)
\curveto(66.53681452,3.00217504)(65.79983536,3.33550837)(65.18525202,4.00217504)
\curveto(64.57587702,4.67405004)(64.27118952,5.5750917)(64.27118952,6.70530004)
\curveto(64.27118952,7.92925837)(64.58368952,8.88238337)(65.20868952,9.56467504)
\curveto(65.83889786,10.25217504)(66.62796036,10.59592504)(67.57587702,10.59592504)
\curveto(68.37796036,10.59592504)(69.03681452,10.33030004)(69.55243952,9.79905004)
\curveto(70.06806452,9.27300837)(70.32587702,8.5672792)(70.32587702,7.68186254)
\closepath
\moveto(65.38056452,8.13498754)
\lineto(68.69306452,8.13498754)
\curveto(68.66702286,8.59332087)(68.61233536,8.91623754)(68.52900202,9.10373754)
\curveto(68.39879369,9.3954042)(68.20348119,9.62457087)(67.94306452,9.79123754)
\curveto(67.68785619,9.9579042)(67.41962702,10.04123754)(67.13837702,10.04123754)
\curveto(66.70608536,10.04123754)(66.31806452,9.8719667)(65.97431452,9.53342504)
\curveto(65.63577286,9.2000917)(65.43785619,8.73394587)(65.38056452,8.13498754)
\closepath
}
}
{
\newrgbcolor{curcolor}{0 0 0}
\pscustom[linestyle=none,fillstyle=solid,fillcolor=curcolor]
{
\newpath
\moveto(73.73993952,14.33030004)
\lineto(73.73993952,4.83811254)
\curveto(73.73993952,4.39019587)(73.77118952,4.09332087)(73.83368952,3.94748754)
\curveto(73.90139786,3.8016542)(74.00296036,3.68967504)(74.13837702,3.61155004)
\curveto(74.27379369,3.53863337)(74.52639786,3.50217504)(74.89618952,3.50217504)
\lineto(74.89618952,3.22092504)
\lineto(71.38837702,3.22092504)
\lineto(71.38837702,3.50217504)
\curveto(71.71650202,3.50217504)(71.94046036,3.5360292)(72.06025202,3.60373754)
\curveto(72.18004369,3.67144587)(72.27379369,3.78342504)(72.34150202,3.93967504)
\curveto(72.40921036,4.09592504)(72.44306452,4.3954042)(72.44306452,4.83811254)
\lineto(72.44306452,11.33811254)
\curveto(72.44306452,12.1454042)(72.42483536,12.64019587)(72.38837702,12.82248754)
\curveto(72.35191869,13.00998754)(72.29202286,13.1375917)(72.20868952,13.20530004)
\curveto(72.13056452,13.27300837)(72.02900202,13.30686254)(71.90400202,13.30686254)
\curveto(71.76858536,13.30686254)(71.59671036,13.26519587)(71.38837702,13.18186254)
\lineto(71.25556452,13.45530004)
\lineto(73.38837702,14.33030004)
\closepath
}
}
{
\newrgbcolor{curcolor}{0 0 0}
\pscustom[linestyle=none,fillstyle=solid,fillcolor=curcolor]
{
\newpath
\moveto(79.22431452,10.58811254)
\curveto(80.30764786,10.58811254)(81.17743952,10.1766542)(81.83368952,9.35373754)
\curveto(82.39098119,8.65061254)(82.66962702,7.84332087)(82.66962702,6.93186254)
\curveto(82.66962702,6.29123754)(82.51598119,5.64280004)(82.20868952,4.98655004)
\curveto(81.90139786,4.33030004)(81.47691869,3.83550837)(80.93525202,3.50217504)
\curveto(80.39879369,3.1688417)(79.79983536,3.00217504)(79.13837702,3.00217504)
\curveto(78.06025202,3.00217504)(77.20348119,3.43186254)(76.56806452,4.29123754)
\curveto(76.03160619,5.01519587)(75.76337702,5.82769587)(75.76337702,6.72873754)
\curveto(75.76337702,7.38498754)(75.92483536,8.0360292)(76.24775202,8.68186254)
\curveto(76.57587702,9.3329042)(77.00556452,9.81207087)(77.53681452,10.11936254)
\curveto(78.06806452,10.43186254)(78.63056452,10.58811254)(79.22431452,10.58811254)
\closepath
\moveto(78.98212702,10.08030004)
\curveto(78.70608536,10.08030004)(78.42743952,9.9969667)(78.14618952,9.83030004)
\curveto(77.87014786,9.6688417)(77.64618952,9.38238337)(77.47431452,8.97092504)
\curveto(77.30243952,8.5594667)(77.21650202,8.03082087)(77.21650202,7.38498754)
\curveto(77.21650202,6.34332087)(77.42223119,5.44488337)(77.83368952,4.68967504)
\curveto(78.25035619,3.9344667)(78.79723119,3.55686254)(79.47431452,3.55686254)
\curveto(79.97952286,3.55686254)(80.39618952,3.76519587)(80.72431452,4.18186254)
\curveto(81.05243952,4.5985292)(81.21650202,5.31467504)(81.21650202,6.33030004)
\curveto(81.21650202,7.60113337)(80.94306452,8.60113337)(80.39618952,9.33030004)
\curveto(80.02639786,9.83030004)(79.55504369,10.08030004)(78.98212702,10.08030004)
\closepath
}
}
{
\newrgbcolor{curcolor}{0 0 0}
\pscustom[linestyle=none,fillstyle=solid,fillcolor=curcolor]
{
\newpath
\moveto(83.32587702,10.37717504)
\lineto(86.32587702,10.37717504)
\lineto(86.32587702,10.08811254)
\curveto(86.04983536,10.0672792)(85.86754369,10.01780004)(85.77900202,9.93967504)
\curveto(85.69566869,9.86155004)(85.65400202,9.74957087)(85.65400202,9.60373754)
\curveto(85.65400202,9.4422792)(85.69827286,9.2469667)(85.78681452,9.01780004)
\lineto(87.31806452,4.90061254)
\lineto(88.85712702,8.25217504)
\lineto(88.45087702,9.30686254)
\curveto(88.32587702,9.61936254)(88.16181452,9.83550837)(87.95868952,9.95530004)
\curveto(87.84410619,10.0282167)(87.63056452,10.07248754)(87.31806452,10.08811254)
\lineto(87.31806452,10.37717504)
\lineto(90.72431452,10.37717504)
\lineto(90.72431452,10.08811254)
\curveto(90.34931452,10.07248754)(90.08368952,10.0047792)(89.92743952,9.88498754)
\curveto(89.82327286,9.8016542)(89.77118952,9.6688417)(89.77118952,9.48655004)
\curveto(89.77118952,9.38238337)(89.79202286,9.27561254)(89.83368952,9.16623754)
\lineto(91.45868952,5.05686254)
\lineto(92.96650202,9.01780004)
\curveto(93.07066869,9.29905004)(93.12275202,9.52300837)(93.12275202,9.68967504)
\curveto(93.12275202,9.78863337)(93.07066869,9.87717504)(92.96650202,9.95530004)
\curveto(92.86754369,10.03342504)(92.66962702,10.07769587)(92.37275202,10.08811254)
\lineto(92.37275202,10.37717504)
\lineto(94.63056452,10.37717504)
\lineto(94.63056452,10.08811254)
\curveto(94.17743952,10.0204042)(93.84410619,9.71311254)(93.63056452,9.16623754)
\lineto(91.23993952,3.00217504)
\lineto(90.91962702,3.00217504)
\lineto(89.13056452,7.57248754)
\lineto(87.04462702,3.00217504)
\lineto(86.75556452,3.00217504)
\lineto(84.45868952,9.01780004)
\curveto(84.30764786,9.39800837)(84.15921036,9.6532167)(84.01337702,9.78342504)
\curveto(83.86754369,9.9188417)(83.63837702,10.0204042)(83.32587702,10.08811254)
\closepath
}
}
{
\newrgbcolor{curcolor}{0 0 0}
\pscustom[linestyle=none,fillstyle=solid,fillcolor=curcolor]
{
\newpath
\moveto(139.24018863,7.32487004)
\lineto(135.13862613,7.32487004)
\lineto(134.41987613,5.65299504)
\curveto(134.24279279,5.2415367)(134.15425113,4.93424504)(134.15425113,4.73112004)
\curveto(134.15425113,4.5696617)(134.22977196,4.42643254)(134.38081363,4.30143254)
\curveto(134.53706363,4.18164087)(134.87039696,4.10351587)(135.38081363,4.06705754)
\lineto(135.38081363,3.77799504)
\lineto(132.04487613,3.77799504)
\lineto(132.04487613,4.06705754)
\curveto(132.48758446,4.14518254)(132.77404279,4.24674504)(132.90425113,4.37174504)
\curveto(133.16987613,4.62174504)(133.46414696,5.12955754)(133.78706363,5.89518254)
\lineto(137.51362613,14.61393254)
\lineto(137.78706363,14.61393254)
\lineto(141.47456363,5.80143254)
\curveto(141.77143863,5.0930992)(142.03966779,4.6321617)(142.27925113,4.41862004)
\curveto(142.52404279,4.2102867)(142.86258446,4.0930992)(143.29487613,4.06705754)
\lineto(143.29487613,3.77799504)
\lineto(139.11518863,3.77799504)
\lineto(139.11518863,4.06705754)
\curveto(139.53706363,4.08789087)(139.82091779,4.15820337)(139.96675113,4.27799504)
\curveto(140.11779279,4.3977867)(140.19331363,4.54362004)(140.19331363,4.71549504)
\curveto(140.19331363,4.9446617)(140.08914696,5.30664087)(139.88081363,5.80143254)
\closepath
\moveto(139.02143863,7.90299504)
\lineto(137.22456363,12.18424504)
\lineto(135.38081363,7.90299504)
\closepath
}
}
{
\newrgbcolor{curcolor}{0 0 0}
\pscustom[linestyle=none,fillstyle=solid,fillcolor=curcolor]
{
\newpath
\moveto(145.93550113,9.69987004)
\curveto(146.62820946,10.6634117)(147.37560529,11.14518254)(148.17768863,11.14518254)
\curveto(148.91206363,11.14518254)(149.55268863,10.83007837)(150.09956363,10.19987004)
\curveto(150.64643863,9.57487004)(150.91987613,8.7180992)(150.91987613,7.62955754)
\curveto(150.91987613,6.3587242)(150.49800113,5.3352867)(149.65425113,4.55924504)
\curveto(148.93029279,3.89257837)(148.12300113,3.55924504)(147.23237613,3.55924504)
\curveto(146.81570946,3.55924504)(146.39123029,3.63476587)(145.95893863,3.78580754)
\curveto(145.53185529,3.9368492)(145.09435529,4.1634117)(144.64643863,4.46549504)
\lineto(144.64643863,11.87955754)
\curveto(144.64643863,12.69205754)(144.62560529,13.19205754)(144.58393863,13.37955754)
\curveto(144.54748029,13.56705754)(144.48758446,13.6946617)(144.40425113,13.76237004)
\curveto(144.32091779,13.83007837)(144.21675113,13.86393254)(144.09175113,13.86393254)
\curveto(143.94591779,13.86393254)(143.76362613,13.82226587)(143.54487613,13.73893254)
\lineto(143.43550113,14.01237004)
\lineto(145.58393863,14.88737004)
\lineto(145.93550113,14.88737004)
\closepath
\moveto(145.93550113,9.19987004)
\lineto(145.93550113,4.91862004)
\curveto(146.20112613,4.65820337)(146.47456363,4.4602867)(146.75581363,4.32487004)
\curveto(147.04227196,4.1946617)(147.33393863,4.12955754)(147.63081363,4.12955754)
\curveto(148.10477196,4.12955754)(148.54487613,4.3899742)(148.95112613,4.91080754)
\curveto(149.36258446,5.43164087)(149.56831363,6.18945337)(149.56831363,7.18424504)
\curveto(149.56831363,8.1009117)(149.36258446,8.8040367)(148.95112613,9.29362004)
\curveto(148.54487613,9.7884117)(148.08133446,10.03580754)(147.56050113,10.03580754)
\curveto(147.28445946,10.03580754)(147.00841779,9.96549504)(146.73237613,9.82487004)
\curveto(146.52404279,9.72070337)(146.25841779,9.51237004)(145.93550113,9.19987004)
\closepath
}
}
{
\newrgbcolor{curcolor}{0 0 0}
\pscustom[linestyle=none,fillstyle=solid,fillcolor=curcolor]
{
\newpath
\moveto(155.47456363,11.14518254)
\curveto(156.55789696,11.14518254)(157.42768863,10.7337242)(158.08393863,9.91080754)
\curveto(158.64123029,9.20768254)(158.91987613,8.40039087)(158.91987613,7.48893254)
\curveto(158.91987613,6.84830754)(158.76623029,6.19987004)(158.45893863,5.54362004)
\curveto(158.15164696,4.88737004)(157.72716779,4.39257837)(157.18550113,4.05924504)
\curveto(156.64904279,3.7259117)(156.05008446,3.55924504)(155.38862613,3.55924504)
\curveto(154.31050113,3.55924504)(153.45373029,3.98893254)(152.81831363,4.84830754)
\curveto(152.28185529,5.57226587)(152.01362613,6.38476587)(152.01362613,7.28580754)
\curveto(152.01362613,7.94205754)(152.17508446,8.5930992)(152.49800113,9.23893254)
\curveto(152.82612613,9.8899742)(153.25581363,10.36914087)(153.78706363,10.67643254)
\curveto(154.31831363,10.98893254)(154.88081363,11.14518254)(155.47456363,11.14518254)
\closepath
\moveto(155.23237613,10.63737004)
\curveto(154.95633446,10.63737004)(154.67768863,10.5540367)(154.39643863,10.38737004)
\curveto(154.12039696,10.2259117)(153.89643863,9.93945337)(153.72456363,9.52799504)
\curveto(153.55268863,9.1165367)(153.46675113,8.58789087)(153.46675113,7.94205754)
\curveto(153.46675113,6.90039087)(153.67248029,6.00195337)(154.08393863,5.24674504)
\curveto(154.50060529,4.4915367)(155.04748029,4.11393254)(155.72456363,4.11393254)
\curveto(156.22977196,4.11393254)(156.64643863,4.32226587)(156.97456363,4.73893254)
\curveto(157.30268863,5.1555992)(157.46675113,5.87174504)(157.46675113,6.88737004)
\curveto(157.46675113,8.15820337)(157.19331363,9.15820337)(156.64643863,9.88737004)
\curveto(156.27664696,10.38737004)(155.80529279,10.63737004)(155.23237613,10.63737004)
\closepath
}
}
{
\newrgbcolor{curcolor}{0 0 0}
\pscustom[linestyle=none,fillstyle=solid,fillcolor=curcolor]
{
\newpath
\moveto(159.60737613,10.93424504)
\lineto(162.97456363,10.93424504)
\lineto(162.97456363,10.64518254)
\lineto(162.75581363,10.64518254)
\curveto(162.55268863,10.64518254)(162.39643863,10.59570337)(162.28706363,10.49674504)
\curveto(162.18289696,10.3977867)(162.13081363,10.2649742)(162.13081363,10.09830754)
\curveto(162.13081363,9.91601587)(162.18550113,9.69987004)(162.29487613,9.44987004)
\lineto(163.95893863,5.49674504)
\lineto(165.63081363,9.59830754)
\curveto(165.75060529,9.8899742)(165.81050113,10.11132837)(165.81050113,10.26237004)
\curveto(165.81050113,10.3352867)(165.78966779,10.39518254)(165.74800113,10.44205754)
\curveto(165.69070946,10.52018254)(165.61779279,10.57226587)(165.52925113,10.59830754)
\curveto(165.44070946,10.62955754)(165.26102196,10.64518254)(164.99018863,10.64518254)
\lineto(164.99018863,10.93424504)
\lineto(167.32612613,10.93424504)
\lineto(167.32612613,10.64518254)
\curveto(167.05529279,10.6243492)(166.86779279,10.5696617)(166.76362613,10.48112004)
\curveto(166.58133446,10.32487004)(166.41727196,10.06445337)(166.27143863,9.69987004)
\lineto(163.73237613,3.55924504)
\lineto(163.41206363,3.55924504)
\lineto(160.85737613,9.59830754)
\curveto(160.74279279,9.87955754)(160.63341779,10.08007837)(160.52925113,10.19987004)
\curveto(160.42508446,10.32487004)(160.29227196,10.4290367)(160.13081363,10.51237004)
\curveto(160.04227196,10.55924504)(159.86779279,10.60351587)(159.60737613,10.64518254)
\closepath
}
}
{
\newrgbcolor{curcolor}{0 0 0}
\pscustom[linestyle=none,fillstyle=solid,fillcolor=curcolor]
{
\newpath
\moveto(169.17768863,8.23893254)
\curveto(169.17248029,7.17643254)(169.43029279,6.3430992)(169.95112613,5.73893254)
\curveto(170.47195946,5.13476587)(171.08393863,4.83268254)(171.78706363,4.83268254)
\curveto(172.25581363,4.83268254)(172.66206363,4.9602867)(173.00581363,5.21549504)
\curveto(173.35477196,5.4759117)(173.64643863,5.91862004)(173.88081363,6.54362004)
\lineto(174.12300113,6.38737004)
\curveto(174.01362613,5.67382837)(173.69591779,5.0227867)(173.16987613,4.43424504)
\curveto(172.64383446,3.8509117)(171.98498029,3.55924504)(171.19331363,3.55924504)
\curveto(170.33393863,3.55924504)(169.59695946,3.89257837)(168.98237613,4.55924504)
\curveto(168.37300113,5.23112004)(168.06831363,6.1321617)(168.06831363,7.26237004)
\curveto(168.06831363,8.48632837)(168.38081363,9.43945337)(169.00581363,10.12174504)
\curveto(169.63602196,10.80924504)(170.42508446,11.15299504)(171.37300113,11.15299504)
\curveto(172.17508446,11.15299504)(172.83393863,10.88737004)(173.34956363,10.35612004)
\curveto(173.86518863,9.83007837)(174.12300113,9.1243492)(174.12300113,8.23893254)
\closepath
\moveto(169.17768863,8.69205754)
\lineto(172.49018863,8.69205754)
\curveto(172.46414696,9.15039087)(172.40945946,9.47330754)(172.32612613,9.66080754)
\curveto(172.19591779,9.9524742)(172.00060529,10.18164087)(171.74018863,10.34830754)
\curveto(171.48498029,10.5149742)(171.21675113,10.59830754)(170.93550113,10.59830754)
\curveto(170.50320946,10.59830754)(170.11518863,10.4290367)(169.77143863,10.09049504)
\curveto(169.43289696,9.7571617)(169.23498029,9.29101587)(169.17768863,8.69205754)
\closepath
}
}
\end{pspicture}

    \caption{射线端点关于划分平面的位置可用于确定该首先处理该节点的哪个孩子。
    如果射线如$\bm r_1$的端点在划分平面的“下方”一侧,
    则我们在处理上方孩子前应该先处理下方孩子,反之亦然。}
    \label{fig:4.18}
\end{figure}
\begin{lstlisting}
`\initcode{Get node children pointers for ray}{=}`
const `\refvar{KdAccelNode}{}` *firstChild, *secondChild;
int belowFirst = (ray.`\refvar[Ray::o]{o}{}`[axis] <  node->`\refvar{SplitPos}{}`()) ||
                 (ray.`\refvar[Ray::o]{o}{}`[axis] == node->`\refvar{SplitPos}{}`() && ray.`\refvar[Ray::d]{d}{}`[axis] <= 0);
if (belowFirst) {
    firstChild = node + 1;
    secondChild = &`\refvar[KdTreeAccel::nodes]{nodes}{}`[node->`\refvar{AboveChild}{}`()];
} else {
    firstChild = &`\refvar[KdTreeAccel::nodes]{nodes}{}`[node->`\refvar{AboveChild}{}`()];
    secondChild = node + 1;
}
\end{lstlisting}

该节点的两个孩子可能没有必要都处理。
\reffig{4.19}展示了一些光线只穿过一个孩子的配置。
光线绝不会同时错过两个孩子,因为否则当前内部节点就不该被访问到。
\begin{figure}[htbp]
    \centering%LaTeX with PSTricks extensions
%%Creator: Inkscape 1.0.1 (3bc2e813f5, 2020-09-07)
%%Please note this file requires PSTricks extensions
\psset{xunit=.5pt,yunit=.5pt,runit=.5pt}
\begin{pspicture}(422.61999512,228.77000427)
{
\newrgbcolor{curcolor}{0 0 0}
\pscustom[linewidth=1,linecolor=curcolor]
{
\newpath
\moveto(31.90999985,170.9200058)
\lineto(175.79000473,170.9200058)
\lineto(175.79000473,27.04000092)
\lineto(31.90999985,27.04000092)
\closepath
}
}
{
\newrgbcolor{curcolor}{0 0 0}
\pscustom[linewidth=1,linecolor=curcolor,linestyle=dashed,dash=2]
{
\newpath
\moveto(125.94000244,27.55000305)
\lineto(125.94000244,225.90000439)
}
}
{
\newrgbcolor{curcolor}{0 0 0}
\pscustom[linewidth=1,linecolor=curcolor]
{
\newpath
\moveto(104.31999969,69.27000427)
\lineto(8.07999992,77.02000427)
}
}
{
\newrgbcolor{curcolor}{0 0 0}
\pscustom[linestyle=none,fillstyle=solid,fillcolor=curcolor]
{
\newpath
\moveto(13.41,82.11000427)
\lineto(8.73,76.97000427)
\lineto(12.53,71.14000427)
\lineto(0,77.67000427)
\closepath
}
}
{
\newrgbcolor{curcolor}{0.65098041 0.65098041 0.65098041}
\pscustom[linestyle=none,fillstyle=solid,fillcolor=curcolor]
{
\newpath
\moveto(11.76,81.04000427)
\lineto(1.31,77.57000427)
\lineto(8.1,77.02000427)
\closepath
}
}
{
\newrgbcolor{curcolor}{0.40000001 0.40000001 0.40000001}
\pscustom[linestyle=none,fillstyle=solid,fillcolor=curcolor]
{
\newpath
\moveto(11.07,72.47000427)
\lineto(1.31,77.57000427)
\lineto(8.1,77.02000427)
\closepath
}
}
{
\newrgbcolor{curcolor}{0 0 0}
\pscustom[linewidth=1,linecolor=curcolor,linestyle=dashed,dash=2]
{
\newpath
\moveto(159.55000305,64.80999756)
\lineto(104.31999969,69.27000427)
}
}
{
\newrgbcolor{curcolor}{0 0 0}
\pscustom[linewidth=1,linecolor=curcolor]
{
\newpath
\moveto(17.43000031,148.04000092)
\lineto(170.61000061,203.09000397)
}
}
{
\newrgbcolor{curcolor}{0 0 0}
\pscustom[linestyle=none,fillstyle=solid,fillcolor=curcolor]
{
\newpath
\moveto(167.85,196.25000427)
\lineto(170,202.87000427)
\lineto(164.12,206.61000427)
\lineto(178.23,205.83000427)
\closepath
}
}
{
\newrgbcolor{curcolor}{0.65098041 0.65098041 0.65098041}
\pscustom[linestyle=none,fillstyle=solid,fillcolor=curcolor]
{
\newpath
\moveto(168.91,197.91000427)
\lineto(177,205.39000427)
\lineto(170.59,203.09000427)
\closepath
}
}
{
\newrgbcolor{curcolor}{0.40000001 0.40000001 0.40000001}
\pscustom[linestyle=none,fillstyle=solid,fillcolor=curcolor]
{
\newpath
\moveto(166,206.00000427)
\lineto(177,205.39000427)
\lineto(170.59,203.09000427)
\closepath
}
}
{
\newrgbcolor{curcolor}{0 0 0}
\pscustom[linestyle=none,fillstyle=solid,fillcolor=curcolor]
{
\newpath
\moveto(84.85999823,170.95000458)
\curveto(84.85999823,174.23767216)(80.8853569,175.88354635)(78.56090665,173.5590961)
\curveto(76.2364564,171.23464585)(77.88233058,167.26000452)(81.16999817,167.26000452)
\curveto(84.45766576,167.26000452)(86.10353994,171.23464585)(83.77908969,173.5590961)
\curveto(81.45463944,175.88354635)(77.47999811,174.23767216)(77.47999811,170.95000458)
\curveto(77.47999811,167.66233699)(81.45463944,166.01646281)(83.77908969,168.34091306)
\curveto(86.10353994,170.66536331)(84.45766576,174.64000463)(81.16999817,174.64000463)
\curveto(77.88233058,174.64000463)(76.2364564,170.66536331)(78.56090665,168.34091306)
\curveto(80.8853569,166.01646281)(84.85999823,167.66233699)(84.85999823,170.95000458)
\closepath
}
}
{
\newrgbcolor{curcolor}{0 0 0}
\pscustom[linestyle=none,fillstyle=solid,fillcolor=curcolor]
{
\newpath
\moveto(129.73000097,187.12000275)
\curveto(129.73000097,190.40767033)(125.75535965,192.05354451)(123.4309094,189.72909426)
\curveto(121.10645915,187.40464402)(122.75233333,183.43000269)(126.04000092,183.43000269)
\curveto(129.3276685,183.43000269)(130.97354268,187.40464402)(128.64909243,189.72909426)
\curveto(126.32464218,192.05354451)(122.35000086,190.40767033)(122.35000086,187.12000275)
\curveto(122.35000086,183.83233516)(126.32464218,182.18646098)(128.64909243,184.51091123)
\curveto(130.97354268,186.83536148)(129.3276685,190.8100028)(126.04000092,190.8100028)
\curveto(122.75233333,190.8100028)(121.10645915,186.83536148)(123.4309094,184.51091123)
\curveto(125.75535965,182.18646098)(129.73000097,183.83233516)(129.73000097,187.12000275)
\closepath
}
}
{
\newrgbcolor{curcolor}{0 0 0}
\pscustom[linestyle=none,fillstyle=solid,fillcolor=curcolor]
{
\newpath
\moveto(35.73000097,153.12000275)
\curveto(35.73000097,156.40767033)(31.75535965,158.05354451)(29.4309094,155.72909426)
\curveto(27.10645915,153.40464402)(28.75233333,149.43000269)(32.04000092,149.43000269)
\curveto(35.3276685,149.43000269)(36.97354268,153.40464402)(34.64909243,155.72909426)
\curveto(32.32464218,158.05354451)(28.35000086,156.40767033)(28.35000086,153.12000275)
\curveto(28.35000086,149.83233516)(32.32464218,148.18646098)(34.64909243,150.51091123)
\curveto(36.97354268,152.83536148)(35.3276685,156.8100028)(32.04000092,156.8100028)
\curveto(28.75233333,156.8100028)(27.10645915,152.83536148)(29.4309094,150.51091123)
\curveto(31.75535965,148.18646098)(35.73000097,149.83233516)(35.73000097,153.12000275)
\closepath
}
}
{
\newrgbcolor{curcolor}{1 1 1}
\pscustom[linestyle=none,fillstyle=solid,fillcolor=curcolor]
{
\newpath
\moveto(129.88999701,67.66000366)
\curveto(129.88999701,70.94767125)(125.91535568,72.59354543)(123.59090543,70.26909518)
\curveto(121.26645518,67.94464493)(122.91232936,63.9700036)(126.19999695,63.9700036)
\curveto(129.48766453,63.9700036)(131.13353872,67.94464493)(128.80908847,70.26909518)
\curveto(126.48463822,72.59354543)(122.50999689,70.94767125)(122.50999689,67.66000366)
\curveto(122.50999689,64.37233608)(126.48463822,62.72646189)(128.80908847,65.05091214)
\curveto(131.13353872,67.37536239)(129.48766453,71.35000372)(126.19999695,71.35000372)
\curveto(122.91232936,71.35000372)(121.26645518,67.37536239)(123.59090543,65.05091214)
\curveto(125.91535568,62.72646189)(129.88999701,64.37233608)(129.88999701,67.66000366)
\closepath
}
}
{
\newrgbcolor{curcolor}{0 0 0}
\pscustom[linewidth=1,linecolor=curcolor]
{
\newpath
\moveto(129.88999701,67.66000366)
\curveto(129.88999701,70.94767125)(125.91535568,72.59354543)(123.59090543,70.26909518)
\curveto(121.26645518,67.94464493)(122.91232936,63.9700036)(126.19999695,63.9700036)
\curveto(129.48766453,63.9700036)(131.13353872,67.94464493)(128.80908847,70.26909518)
\curveto(126.48463822,72.59354543)(122.50999689,70.94767125)(122.50999689,67.66000366)
\curveto(122.50999689,64.37233608)(126.48463822,62.72646189)(128.80908847,65.05091214)
\curveto(131.13353872,67.37536239)(129.48766453,71.35000372)(126.19999695,71.35000372)
\curveto(122.91232936,71.35000372)(121.26645518,67.37536239)(123.59090543,65.05091214)
\curveto(125.91535568,62.72646189)(129.88999701,64.37233608)(129.88999701,67.66000366)
\closepath
}
}
{
\newrgbcolor{curcolor}{0 0 0}
\pscustom[linewidth=1,linecolor=curcolor]
{
\newpath
\moveto(247.63999939,170.73000336)
\lineto(391.52000427,170.73000336)
\lineto(391.52000427,26.84999847)
\lineto(247.63999939,26.84999847)
\closepath
}
}
{
\newrgbcolor{curcolor}{0 0 0}
\pscustom[linewidth=1,linecolor=curcolor,linestyle=dashed,dash=2]
{
\newpath
\moveto(341.67001343,27.37001038)
\lineto(341.67001343,225.71000433)
}
}
{
\newrgbcolor{curcolor}{0 0 0}
\pscustom[linestyle=none,fillstyle=solid,fillcolor=curcolor]
{
\newpath
\moveto(362.1300025,170.76000595)
\curveto(362.1300025,174.04767354)(358.15536117,175.69354772)(355.83091092,173.36909747)
\curveto(353.50646067,171.04464722)(355.15233486,167.07000589)(358.44000244,167.07000589)
\curveto(361.72767003,167.07000589)(363.37354421,171.04464722)(361.04909396,173.36909747)
\curveto(358.72464371,175.69354772)(354.75000238,174.04767354)(354.75000238,170.76000595)
\curveto(354.75000238,167.47233836)(358.72464371,165.82646418)(361.04909396,168.15091443)
\curveto(363.37354421,170.47536468)(361.72767003,174.45000601)(358.44000244,174.45000601)
\curveto(355.15233486,174.45000601)(353.50646067,170.47536468)(355.83091092,168.15091443)
\curveto(358.15536117,165.82646418)(362.1300025,167.47233836)(362.1300025,170.76000595)
\closepath
}
}
{
\newrgbcolor{curcolor}{0 0 0}
\pscustom[linestyle=none,fillstyle=solid,fillcolor=curcolor]
{
\newpath
\moveto(344.71999884,199.47000504)
\curveto(344.71999884,202.75767262)(340.74535751,204.4035468)(338.42090726,202.07909655)
\curveto(336.09645701,199.7546463)(337.74233119,195.78000498)(341.02999878,195.78000498)
\curveto(344.31766637,195.78000498)(345.96354055,199.7546463)(343.6390903,202.07909655)
\curveto(341.31464005,204.4035468)(337.33999872,202.75767262)(337.33999872,199.47000504)
\curveto(337.33999872,196.18233745)(341.31464005,194.53646327)(343.6390903,196.86091352)
\curveto(345.96354055,199.18536377)(344.31766637,203.16000509)(341.02999878,203.16000509)
\curveto(337.74233119,203.16000509)(336.09645701,199.18536377)(338.42090726,196.86091352)
\curveto(340.74535751,194.53646327)(344.71999884,196.18233745)(344.71999884,199.47000504)
\closepath
}
}
{
\newrgbcolor{curcolor}{0 0 0}
\pscustom[linestyle=none,fillstyle=solid,fillcolor=curcolor]
{
\newpath
\moveto(394.80999517,112.88000488)
\curveto(394.80999517,116.16767247)(390.83535385,117.81354665)(388.5109036,115.4890964)
\curveto(386.18645335,113.16464615)(387.83232753,109.19000483)(391.11999512,109.19000483)
\curveto(394.4076627,109.19000483)(396.05353689,113.16464615)(393.72908664,115.4890964)
\curveto(391.40463639,117.81354665)(387.42999506,116.16767247)(387.42999506,112.88000488)
\curveto(387.42999506,109.5923373)(391.40463639,107.94646311)(393.72908664,110.27091336)
\curveto(396.05353689,112.59536361)(394.4076627,116.57000494)(391.11999512,116.57000494)
\curveto(387.83232753,116.57000494)(386.18645335,112.59536361)(388.5109036,110.27091336)
\curveto(390.83535385,107.94646311)(394.80999517,109.5923373)(394.80999517,112.88000488)
\closepath
}
}
{
\newrgbcolor{curcolor}{0 0 0}
\pscustom[linewidth=1,linecolor=curcolor]
{
\newpath
\moveto(325.54000854,228.52000427)
\lineto(418.60998535,64.71000671)
}
}
{
\newrgbcolor{curcolor}{0 0 0}
\pscustom[linestyle=none,fillstyle=solid,fillcolor=curcolor]
{
\newpath
\moveto(411.4,66.26000427)
\lineto(418.29,65.28000427)
\lineto(420.97,71.70000427)
\lineto(422.62,57.67000427)
\closepath
}
}
{
\newrgbcolor{curcolor}{0.65098041 0.65098041 0.65098041}
\pscustom[linestyle=none,fillstyle=solid,fillcolor=curcolor]
{
\newpath
\moveto(413.22,65.50000427)
\lineto(421.97,58.81000427)
\lineto(418.6,64.73000427)
\closepath
}
}
{
\newrgbcolor{curcolor}{0.40000001 0.40000001 0.40000001}
\pscustom[linestyle=none,fillstyle=solid,fillcolor=curcolor]
{
\newpath
\moveto(420.69,69.75000427)
\lineto(421.97,58.81000427)
\lineto(418.6,64.73000427)
\closepath
}
}
{
\newrgbcolor{curcolor}{0 0 0}
\pscustom[linestyle=none,fillstyle=solid,fillcolor=curcolor]
{
\newpath
\moveto(135.00701488,184.41418537)
\lineto(136.50701488,184.41418537)
\curveto(136.83201488,184.41418537)(137.00701488,184.41418537)(137.00701488,184.73918537)
\curveto(137.00701488,184.91418537)(136.83201488,184.91418537)(136.55701488,184.91418537)
\lineto(135.15701488,184.91418537)
\curveto(135.73201488,187.18918537)(135.80701488,187.48918537)(135.80701488,187.58918537)
\curveto(135.80701488,187.86418537)(135.60701488,188.01418537)(135.33201488,188.01418537)
\curveto(135.28201488,188.01418537)(134.83201488,188.01418537)(134.70701488,187.43918537)
\lineto(134.08201488,184.91418537)
\lineto(132.58201488,184.91418537)
\curveto(132.25701488,184.91418537)(132.10701488,184.91418537)(132.10701488,184.61418537)
\curveto(132.10701488,184.41418537)(132.23201488,184.41418537)(132.55701488,184.41418537)
\lineto(133.95701488,184.41418537)
\curveto(132.80701488,179.88918537)(132.73201488,179.61418537)(132.73201488,179.33918537)
\curveto(132.73201488,178.46418537)(133.33201488,177.86418537)(134.20701488,177.86418537)
\curveto(135.83201488,177.86418537)(136.73201488,180.18918537)(136.73201488,180.31418537)
\curveto(136.73201488,180.48918537)(136.60701488,180.48918537)(136.55701488,180.48918537)
\curveto(136.40701488,180.48918537)(136.38201488,180.43918537)(136.30701488,180.26418537)
\curveto(135.63201488,178.58918537)(134.78201488,178.21418537)(134.23201488,178.21418537)
\curveto(133.90701488,178.21418537)(133.73201488,178.41418537)(133.73201488,178.93918537)
\curveto(133.73201488,179.33918537)(133.78201488,179.43918537)(133.83201488,179.71418537)
\closepath
\moveto(135.00701488,184.41418537)
}
}
{
\newrgbcolor{curcolor}{0 0 0}
\pscustom[linestyle=none,fillstyle=solid,fillcolor=curcolor]
{
\newpath
\moveto(141.7138142,180.3237801)
\curveto(141.7138142,180.5237801)(141.7138142,180.6237801)(141.5638142,180.6237801)
\curveto(141.5138142,180.6237801)(141.4888142,180.6237801)(141.3388142,180.4987801)
\curveto(141.3138142,180.4737801)(141.2138142,180.3737801)(141.1388142,180.3237801)
\curveto(140.7888142,180.5487801)(140.3888142,180.6237801)(139.9638142,180.6237801)
\curveto(138.3638142,180.6237801)(137.9888142,179.7737801)(137.9888142,179.2237801)
\curveto(137.9888142,178.8737801)(138.1388142,178.5737801)(138.4138142,178.3487801)
\curveto(138.8388142,177.9987801)(139.2638142,177.9237801)(139.9638142,177.7987801)
\curveto(140.5138142,177.6987801)(141.4138142,177.5487801)(141.4138142,176.7987801)
\curveto(141.4138142,176.3737801)(141.1138142,175.8487801)(140.0388142,175.8487801)
\curveto(138.9638142,175.8487801)(138.5888142,176.5487801)(138.3888142,177.2987801)
\curveto(138.3388142,177.4487801)(138.3388142,177.4987801)(138.1888142,177.4987801)
\curveto(137.9888142,177.4987801)(137.9888142,177.4237801)(137.9888142,177.1987801)
\lineto(137.9888142,175.8237801)
\curveto(137.9888142,175.6487801)(137.9888142,175.5487801)(138.1388142,175.5487801)
\curveto(138.2388142,175.5487801)(138.4638142,175.7737801)(138.6888142,176.0237801)
\curveto(139.1638142,175.5487801)(139.7638142,175.5487801)(140.0388142,175.5487801)
\curveto(141.4888142,175.5487801)(142.0388142,176.3237801)(142.0388142,177.0987801)
\curveto(142.0388142,177.5237801)(141.8388142,177.8737801)(141.5388142,178.1237801)
\curveto(141.1138142,178.5237801)(140.6138142,178.6237801)(140.2138142,178.6737801)
\curveto(139.3388142,178.8487801)(138.6138142,178.9737801)(138.6138142,179.5737801)
\curveto(138.6138142,179.9237801)(138.9138142,180.3487801)(139.9638142,180.3487801)
\curveto(141.2638142,180.3487801)(141.3138142,179.4487801)(141.3388142,179.1237801)
\curveto(141.3388142,178.9987801)(141.4888142,178.9987801)(141.5138142,178.9987801)
\curveto(141.7138142,178.9987801)(141.7138142,179.0737801)(141.7138142,179.2987801)
\closepath
\moveto(141.7138142,180.3237801)
}
}
{
\newrgbcolor{curcolor}{0 0 0}
\pscustom[linestyle=none,fillstyle=solid,fillcolor=curcolor]
{
\newpath
\moveto(145.63896068,173.8987801)
\curveto(144.91396068,173.8987801)(144.81396068,173.8987801)(144.81396068,174.3737801)
\lineto(144.81396068,176.1737801)
\curveto(144.86396068,176.1237801)(145.38896068,175.5487801)(146.31396068,175.5487801)
\curveto(147.78896068,175.5487801)(149.03896068,176.6487801)(149.03896068,178.0487801)
\curveto(149.03896068,179.4237801)(147.91396068,180.5737801)(146.48896068,180.5737801)
\curveto(145.86396068,180.5737801)(145.21396068,180.3237801)(144.76396068,179.8737801)
\lineto(144.76396068,180.5737801)
\lineto(143.08896068,180.4487801)
\lineto(143.08896068,180.0487801)
\curveto(143.86396068,180.0487801)(143.91396068,179.9737801)(143.91396068,179.5237801)
\lineto(143.91396068,174.3737801)
\curveto(143.91396068,173.8987801)(143.81396068,173.8987801)(143.08896068,173.8987801)
\lineto(143.08896068,173.4737801)
\curveto(143.11396068,173.4737801)(143.88896068,173.5237801)(144.36396068,173.5237801)
\curveto(144.76396068,173.5237801)(145.53896068,173.4987801)(145.63896068,173.4737801)
\closepath
\moveto(144.81396068,179.3737801)
\curveto(145.13896068,179.9237801)(145.78896068,180.1987801)(146.36396068,180.1987801)
\curveto(147.31396068,180.1987801)(148.03896068,179.2237801)(148.03896068,178.0487801)
\curveto(148.03896068,176.7737801)(147.18896068,175.8487801)(146.26396068,175.8487801)
\curveto(145.26396068,175.8487801)(144.83896068,176.6987801)(144.81396068,176.7737801)
\closepath
\moveto(144.81396068,179.3737801)
}
}
{
\newrgbcolor{curcolor}{0 0 0}
\pscustom[linestyle=none,fillstyle=solid,fillcolor=curcolor]
{
\newpath
\moveto(151.87970775,183.3987801)
\lineto(150.20470775,183.2737801)
\lineto(150.20470775,182.8737801)
\curveto(150.95470775,182.8737801)(151.02970775,182.7737801)(151.02970775,182.2487801)
\lineto(151.02970775,176.5237801)
\curveto(151.02970775,176.0487801)(150.92970775,176.0487801)(150.20470775,176.0487801)
\lineto(150.20470775,175.6487801)
\curveto(150.22970775,175.6487801)(151.00470775,175.6987801)(151.45470775,175.6987801)
\curveto(151.87970775,175.6987801)(152.27970775,175.6737801)(152.70470775,175.6487801)
\lineto(152.70470775,176.0487801)
\curveto(151.97970775,176.0487801)(151.87970775,176.0487801)(151.87970775,176.5237801)
\closepath
\moveto(151.87970775,183.3987801)
}
}
{
\newrgbcolor{curcolor}{0 0 0}
\pscustom[linestyle=none,fillstyle=solid,fillcolor=curcolor]
{
\newpath
\moveto(155.53822826,182.5237801)
\curveto(155.53822826,182.8487801)(155.26322826,183.1737801)(154.88822826,183.1737801)
\curveto(154.56322826,183.1737801)(154.26322826,182.8987801)(154.26322826,182.5237801)
\curveto(154.26322826,182.1237801)(154.58822826,181.8737801)(154.88822826,181.8737801)
\curveto(155.26322826,181.8737801)(155.53822826,182.1487801)(155.53822826,182.5237801)
\closepath
\moveto(153.83822826,180.4487801)
\lineto(153.83822826,180.0487801)
\curveto(154.53822826,180.0487801)(154.63822826,179.9737801)(154.63822826,179.4237801)
\lineto(154.63822826,176.5237801)
\curveto(154.63822826,176.0487801)(154.53822826,176.0487801)(153.81322826,176.0487801)
\lineto(153.81322826,175.6487801)
\curveto(153.83822826,175.6487801)(154.61322826,175.6987801)(155.06322826,175.6987801)
\curveto(155.46322826,175.6987801)(155.86322826,175.6737801)(156.23822826,175.6487801)
\lineto(156.23822826,176.0487801)
\curveto(155.58822826,176.0487801)(155.48822826,176.0487801)(155.48822826,176.5237801)
\lineto(155.48822826,180.5737801)
\closepath
\moveto(153.83822826,180.4487801)
}
}
{
\newrgbcolor{curcolor}{0 0 0}
\pscustom[linestyle=none,fillstyle=solid,fillcolor=curcolor]
{
\newpath
\moveto(159.04674877,180.0487801)
\lineto(160.77174877,180.0487801)
\lineto(160.77174877,180.4487801)
\lineto(159.04674877,180.4487801)
\lineto(159.04674877,182.4987801)
\lineto(158.64674877,182.4987801)
\curveto(158.64674877,181.4987801)(158.19674877,180.3987801)(157.12174877,180.3737801)
\lineto(157.12174877,180.0487801)
\lineto(158.14674877,180.0487801)
\lineto(158.14674877,177.0487801)
\curveto(158.14674877,175.7987801)(159.09674877,175.5487801)(159.72174877,175.5487801)
\curveto(160.47174877,175.5487801)(160.97174877,176.1737801)(160.97174877,177.0487801)
\lineto(160.97174877,177.6737801)
\lineto(160.59674877,177.6737801)
\lineto(160.59674877,177.0737801)
\curveto(160.59674877,176.2987801)(160.24674877,175.8987801)(159.79674877,175.8987801)
\curveto(159.04674877,175.8987801)(159.04674877,176.8237801)(159.04674877,177.0237801)
\closepath
\moveto(159.04674877,180.0487801)
}
}
{
\newrgbcolor{curcolor}{0 0 0}
\pscustom[linestyle=none,fillstyle=solid,fillcolor=curcolor]
{
\newpath
\moveto(353.87491488,203.83001837)
\lineto(355.37491488,203.83001837)
\curveto(355.69991488,203.83001837)(355.87491488,203.83001837)(355.87491488,204.15501837)
\curveto(355.87491488,204.33001837)(355.69991488,204.33001837)(355.42491488,204.33001837)
\lineto(354.02491488,204.33001837)
\curveto(354.59991488,206.60501837)(354.67491488,206.90501837)(354.67491488,207.00501837)
\curveto(354.67491488,207.28001837)(354.47491488,207.43001837)(354.19991488,207.43001837)
\curveto(354.14991488,207.43001837)(353.69991488,207.43001837)(353.57491488,206.85501837)
\lineto(352.94991488,204.33001837)
\lineto(351.44991488,204.33001837)
\curveto(351.12491488,204.33001837)(350.97491488,204.33001837)(350.97491488,204.03001837)
\curveto(350.97491488,203.83001837)(351.09991488,203.83001837)(351.42491488,203.83001837)
\lineto(352.82491488,203.83001837)
\curveto(351.67491488,199.30501837)(351.59991488,199.03001837)(351.59991488,198.75501837)
\curveto(351.59991488,197.88001837)(352.19991488,197.28001837)(353.07491488,197.28001837)
\curveto(354.69991488,197.28001837)(355.59991488,199.60501837)(355.59991488,199.73001837)
\curveto(355.59991488,199.90501837)(355.47491488,199.90501837)(355.42491488,199.90501837)
\curveto(355.27491488,199.90501837)(355.24991488,199.85501837)(355.17491488,199.68001837)
\curveto(354.49991488,198.00501837)(353.64991488,197.63001837)(353.09991488,197.63001837)
\curveto(352.77491488,197.63001837)(352.59991488,197.83001837)(352.59991488,198.35501837)
\curveto(352.59991488,198.75501837)(352.64991488,198.85501837)(352.69991488,199.13001837)
\closepath
\moveto(353.87491488,203.83001837)
}
}
{
\newrgbcolor{curcolor}{0 0 0}
\pscustom[linestyle=none,fillstyle=solid,fillcolor=curcolor]
{
\newpath
\moveto(360.5817142,199.7396131)
\curveto(360.5817142,199.9396131)(360.5817142,200.0396131)(360.4317142,200.0396131)
\curveto(360.3817142,200.0396131)(360.3567142,200.0396131)(360.2067142,199.9146131)
\curveto(360.1817142,199.8896131)(360.0817142,199.7896131)(360.0067142,199.7396131)
\curveto(359.6567142,199.9646131)(359.2567142,200.0396131)(358.8317142,200.0396131)
\curveto(357.2317142,200.0396131)(356.8567142,199.1896131)(356.8567142,198.6396131)
\curveto(356.8567142,198.2896131)(357.0067142,197.9896131)(357.2817142,197.7646131)
\curveto(357.7067142,197.4146131)(358.1317142,197.3396131)(358.8317142,197.2146131)
\curveto(359.3817142,197.1146131)(360.2817142,196.9646131)(360.2817142,196.2146131)
\curveto(360.2817142,195.7896131)(359.9817142,195.2646131)(358.9067142,195.2646131)
\curveto(357.8317142,195.2646131)(357.4567142,195.9646131)(357.2567142,196.7146131)
\curveto(357.2067142,196.8646131)(357.2067142,196.9146131)(357.0567142,196.9146131)
\curveto(356.8567142,196.9146131)(356.8567142,196.8396131)(356.8567142,196.6146131)
\lineto(356.8567142,195.2396131)
\curveto(356.8567142,195.0646131)(356.8567142,194.9646131)(357.0067142,194.9646131)
\curveto(357.1067142,194.9646131)(357.3317142,195.1896131)(357.5567142,195.4396131)
\curveto(358.0317142,194.9646131)(358.6317142,194.9646131)(358.9067142,194.9646131)
\curveto(360.3567142,194.9646131)(360.9067142,195.7396131)(360.9067142,196.5146131)
\curveto(360.9067142,196.9396131)(360.7067142,197.2896131)(360.4067142,197.5396131)
\curveto(359.9817142,197.9396131)(359.4817142,198.0396131)(359.0817142,198.0896131)
\curveto(358.2067142,198.2646131)(357.4817142,198.3896131)(357.4817142,198.9896131)
\curveto(357.4817142,199.3396131)(357.7817142,199.7646131)(358.8317142,199.7646131)
\curveto(360.1317142,199.7646131)(360.1817142,198.8646131)(360.2067142,198.5396131)
\curveto(360.2067142,198.4146131)(360.3567142,198.4146131)(360.3817142,198.4146131)
\curveto(360.5817142,198.4146131)(360.5817142,198.4896131)(360.5817142,198.7146131)
\closepath
\moveto(360.5817142,199.7396131)
}
}
{
\newrgbcolor{curcolor}{0 0 0}
\pscustom[linestyle=none,fillstyle=solid,fillcolor=curcolor]
{
\newpath
\moveto(364.50686068,193.3146131)
\curveto(363.78186068,193.3146131)(363.68186068,193.3146131)(363.68186068,193.7896131)
\lineto(363.68186068,195.5896131)
\curveto(363.73186068,195.5396131)(364.25686068,194.9646131)(365.18186068,194.9646131)
\curveto(366.65686068,194.9646131)(367.90686068,196.0646131)(367.90686068,197.4646131)
\curveto(367.90686068,198.8396131)(366.78186068,199.9896131)(365.35686068,199.9896131)
\curveto(364.73186068,199.9896131)(364.08186068,199.7396131)(363.63186068,199.2896131)
\lineto(363.63186068,199.9896131)
\lineto(361.95686068,199.8646131)
\lineto(361.95686068,199.4646131)
\curveto(362.73186068,199.4646131)(362.78186068,199.3896131)(362.78186068,198.9396131)
\lineto(362.78186068,193.7896131)
\curveto(362.78186068,193.3146131)(362.68186068,193.3146131)(361.95686068,193.3146131)
\lineto(361.95686068,192.8896131)
\curveto(361.98186068,192.8896131)(362.75686068,192.9396131)(363.23186068,192.9396131)
\curveto(363.63186068,192.9396131)(364.40686068,192.9146131)(364.50686068,192.8896131)
\closepath
\moveto(363.68186068,198.7896131)
\curveto(364.00686068,199.3396131)(364.65686068,199.6146131)(365.23186068,199.6146131)
\curveto(366.18186068,199.6146131)(366.90686068,198.6396131)(366.90686068,197.4646131)
\curveto(366.90686068,196.1896131)(366.05686068,195.2646131)(365.13186068,195.2646131)
\curveto(364.13186068,195.2646131)(363.70686068,196.1146131)(363.68186068,196.1896131)
\closepath
\moveto(363.68186068,198.7896131)
}
}
{
\newrgbcolor{curcolor}{0 0 0}
\pscustom[linestyle=none,fillstyle=solid,fillcolor=curcolor]
{
\newpath
\moveto(370.74760775,202.8146131)
\lineto(369.07260775,202.6896131)
\lineto(369.07260775,202.2896131)
\curveto(369.82260775,202.2896131)(369.89760775,202.1896131)(369.89760775,201.6646131)
\lineto(369.89760775,195.9396131)
\curveto(369.89760775,195.4646131)(369.79760775,195.4646131)(369.07260775,195.4646131)
\lineto(369.07260775,195.0646131)
\curveto(369.09760775,195.0646131)(369.87260775,195.1146131)(370.32260775,195.1146131)
\curveto(370.74760775,195.1146131)(371.14760775,195.0896131)(371.57260775,195.0646131)
\lineto(371.57260775,195.4646131)
\curveto(370.84760775,195.4646131)(370.74760775,195.4646131)(370.74760775,195.9396131)
\closepath
\moveto(370.74760775,202.8146131)
}
}
{
\newrgbcolor{curcolor}{0 0 0}
\pscustom[linestyle=none,fillstyle=solid,fillcolor=curcolor]
{
\newpath
\moveto(374.40612826,201.9396131)
\curveto(374.40612826,202.2646131)(374.13112826,202.5896131)(373.75612826,202.5896131)
\curveto(373.43112826,202.5896131)(373.13112826,202.3146131)(373.13112826,201.9396131)
\curveto(373.13112826,201.5396131)(373.45612826,201.2896131)(373.75612826,201.2896131)
\curveto(374.13112826,201.2896131)(374.40612826,201.5646131)(374.40612826,201.9396131)
\closepath
\moveto(372.70612826,199.8646131)
\lineto(372.70612826,199.4646131)
\curveto(373.40612826,199.4646131)(373.50612826,199.3896131)(373.50612826,198.8396131)
\lineto(373.50612826,195.9396131)
\curveto(373.50612826,195.4646131)(373.40612826,195.4646131)(372.68112826,195.4646131)
\lineto(372.68112826,195.0646131)
\curveto(372.70612826,195.0646131)(373.48112826,195.1146131)(373.93112826,195.1146131)
\curveto(374.33112826,195.1146131)(374.73112826,195.0896131)(375.10612826,195.0646131)
\lineto(375.10612826,195.4646131)
\curveto(374.45612826,195.4646131)(374.35612826,195.4646131)(374.35612826,195.9396131)
\lineto(374.35612826,199.9896131)
\closepath
\moveto(372.70612826,199.8646131)
}
}
{
\newrgbcolor{curcolor}{0 0 0}
\pscustom[linestyle=none,fillstyle=solid,fillcolor=curcolor]
{
\newpath
\moveto(377.91464877,199.4646131)
\lineto(379.63964877,199.4646131)
\lineto(379.63964877,199.8646131)
\lineto(377.91464877,199.8646131)
\lineto(377.91464877,201.9146131)
\lineto(377.51464877,201.9146131)
\curveto(377.51464877,200.9146131)(377.06464877,199.8146131)(375.98964877,199.7896131)
\lineto(375.98964877,199.4646131)
\lineto(377.01464877,199.4646131)
\lineto(377.01464877,196.4646131)
\curveto(377.01464877,195.2146131)(377.96464877,194.9646131)(378.58964877,194.9646131)
\curveto(379.33964877,194.9646131)(379.83964877,195.5896131)(379.83964877,196.4646131)
\lineto(379.83964877,197.0896131)
\lineto(379.46464877,197.0896131)
\lineto(379.46464877,196.4896131)
\curveto(379.46464877,195.7146131)(379.11464877,195.3146131)(378.66464877,195.3146131)
\curveto(377.91464877,195.3146131)(377.91464877,196.2396131)(377.91464877,196.4396131)
\closepath
\moveto(377.91464877,199.4646131)
}
}
{
\newrgbcolor{curcolor}{0 0 0}
\pscustom[linestyle=none,fillstyle=solid,fillcolor=curcolor]
{
\newpath
\moveto(133.85957188,83.48430237)
\lineto(135.35957188,83.48430237)
\curveto(135.68457188,83.48430237)(135.85957188,83.48430237)(135.85957188,83.80930237)
\curveto(135.85957188,83.98430237)(135.68457188,83.98430237)(135.40957188,83.98430237)
\lineto(134.00957188,83.98430237)
\curveto(134.58457188,86.25930237)(134.65957188,86.55930237)(134.65957188,86.65930237)
\curveto(134.65957188,86.93430237)(134.45957188,87.08430237)(134.18457188,87.08430237)
\curveto(134.13457188,87.08430237)(133.68457188,87.08430237)(133.55957188,86.50930237)
\lineto(132.93457188,83.98430237)
\lineto(131.43457188,83.98430237)
\curveto(131.10957188,83.98430237)(130.95957188,83.98430237)(130.95957188,83.68430237)
\curveto(130.95957188,83.48430237)(131.08457188,83.48430237)(131.40957188,83.48430237)
\lineto(132.80957188,83.48430237)
\curveto(131.65957188,78.95930237)(131.58457188,78.68430237)(131.58457188,78.40930237)
\curveto(131.58457188,77.53430237)(132.18457188,76.93430237)(133.05957188,76.93430237)
\curveto(134.68457188,76.93430237)(135.58457188,79.25930237)(135.58457188,79.38430237)
\curveto(135.58457188,79.55930237)(135.45957188,79.55930237)(135.40957188,79.55930237)
\curveto(135.25957188,79.55930237)(135.23457188,79.50930237)(135.15957188,79.33430237)
\curveto(134.48457188,77.65930237)(133.63457188,77.28430237)(133.08457188,77.28430237)
\curveto(132.75957188,77.28430237)(132.58457188,77.48430237)(132.58457188,78.00930237)
\curveto(132.58457188,78.40930237)(132.63457188,78.50930237)(132.68457188,78.78430237)
\closepath
\moveto(133.85957188,83.48430237)
}
}
{
\newrgbcolor{curcolor}{0 0 0}
\pscustom[linestyle=none,fillstyle=solid,fillcolor=curcolor]
{
\newpath
\moveto(140.5663712,79.3938971)
\curveto(140.5663712,79.5938971)(140.5663712,79.6938971)(140.4163712,79.6938971)
\curveto(140.3663712,79.6938971)(140.3413712,79.6938971)(140.1913712,79.5688971)
\curveto(140.1663712,79.5438971)(140.0663712,79.4438971)(139.9913712,79.3938971)
\curveto(139.6413712,79.6188971)(139.2413712,79.6938971)(138.8163712,79.6938971)
\curveto(137.2163712,79.6938971)(136.8413712,78.8438971)(136.8413712,78.2938971)
\curveto(136.8413712,77.9438971)(136.9913712,77.6438971)(137.2663712,77.4188971)
\curveto(137.6913712,77.0688971)(138.1163712,76.9938971)(138.8163712,76.8688971)
\curveto(139.3663712,76.7688971)(140.2663712,76.6188971)(140.2663712,75.8688971)
\curveto(140.2663712,75.4438971)(139.9663712,74.9188971)(138.8913712,74.9188971)
\curveto(137.8163712,74.9188971)(137.4413712,75.6188971)(137.2413712,76.3688971)
\curveto(137.1913712,76.5188971)(137.1913712,76.5688971)(137.0413712,76.5688971)
\curveto(136.8413712,76.5688971)(136.8413712,76.4938971)(136.8413712,76.2688971)
\lineto(136.8413712,74.8938971)
\curveto(136.8413712,74.7188971)(136.8413712,74.6188971)(136.9913712,74.6188971)
\curveto(137.0913712,74.6188971)(137.3163712,74.8438971)(137.5413712,75.0938971)
\curveto(138.0163712,74.6188971)(138.6163712,74.6188971)(138.8913712,74.6188971)
\curveto(140.3413712,74.6188971)(140.8913712,75.3938971)(140.8913712,76.1688971)
\curveto(140.8913712,76.5938971)(140.6913712,76.9438971)(140.3913712,77.1938971)
\curveto(139.9663712,77.5938971)(139.4663712,77.6938971)(139.0663712,77.7438971)
\curveto(138.1913712,77.9188971)(137.4663712,78.0438971)(137.4663712,78.6438971)
\curveto(137.4663712,78.9938971)(137.7663712,79.4188971)(138.8163712,79.4188971)
\curveto(140.1163712,79.4188971)(140.1663712,78.5188971)(140.1913712,78.1938971)
\curveto(140.1913712,78.0688971)(140.3413712,78.0688971)(140.3663712,78.0688971)
\curveto(140.5663712,78.0688971)(140.5663712,78.1438971)(140.5663712,78.3688971)
\closepath
\moveto(140.5663712,79.3938971)
}
}
{
\newrgbcolor{curcolor}{0 0 0}
\pscustom[linestyle=none,fillstyle=solid,fillcolor=curcolor]
{
\newpath
\moveto(144.49151768,72.9688971)
\curveto(143.76651768,72.9688971)(143.66651768,72.9688971)(143.66651768,73.4438971)
\lineto(143.66651768,75.2438971)
\curveto(143.71651768,75.1938971)(144.24151768,74.6188971)(145.16651768,74.6188971)
\curveto(146.64151768,74.6188971)(147.89151768,75.7188971)(147.89151768,77.1188971)
\curveto(147.89151768,78.4938971)(146.76651768,79.6438971)(145.34151768,79.6438971)
\curveto(144.71651768,79.6438971)(144.06651768,79.3938971)(143.61651768,78.9438971)
\lineto(143.61651768,79.6438971)
\lineto(141.94151768,79.5188971)
\lineto(141.94151768,79.1188971)
\curveto(142.71651768,79.1188971)(142.76651768,79.0438971)(142.76651768,78.5938971)
\lineto(142.76651768,73.4438971)
\curveto(142.76651768,72.9688971)(142.66651768,72.9688971)(141.94151768,72.9688971)
\lineto(141.94151768,72.5438971)
\curveto(141.96651768,72.5438971)(142.74151768,72.5938971)(143.21651768,72.5938971)
\curveto(143.61651768,72.5938971)(144.39151768,72.5688971)(144.49151768,72.5438971)
\closepath
\moveto(143.66651768,78.4438971)
\curveto(143.99151768,78.9938971)(144.64151768,79.2688971)(145.21651768,79.2688971)
\curveto(146.16651768,79.2688971)(146.89151768,78.2938971)(146.89151768,77.1188971)
\curveto(146.89151768,75.8438971)(146.04151768,74.9188971)(145.11651768,74.9188971)
\curveto(144.11651768,74.9188971)(143.69151768,75.7688971)(143.66651768,75.8438971)
\closepath
\moveto(143.66651768,78.4438971)
}
}
{
\newrgbcolor{curcolor}{0 0 0}
\pscustom[linestyle=none,fillstyle=solid,fillcolor=curcolor]
{
\newpath
\moveto(150.73226475,82.4688971)
\lineto(149.05726475,82.3438971)
\lineto(149.05726475,81.9438971)
\curveto(149.80726475,81.9438971)(149.88226475,81.8438971)(149.88226475,81.3188971)
\lineto(149.88226475,75.5938971)
\curveto(149.88226475,75.1188971)(149.78226475,75.1188971)(149.05726475,75.1188971)
\lineto(149.05726475,74.7188971)
\curveto(149.08226475,74.7188971)(149.85726475,74.7688971)(150.30726475,74.7688971)
\curveto(150.73226475,74.7688971)(151.13226475,74.7438971)(151.55726475,74.7188971)
\lineto(151.55726475,75.1188971)
\curveto(150.83226475,75.1188971)(150.73226475,75.1188971)(150.73226475,75.5938971)
\closepath
\moveto(150.73226475,82.4688971)
}
}
{
\newrgbcolor{curcolor}{0 0 0}
\pscustom[linestyle=none,fillstyle=solid,fillcolor=curcolor]
{
\newpath
\moveto(154.39078526,81.5938971)
\curveto(154.39078526,81.9188971)(154.11578526,82.2438971)(153.74078526,82.2438971)
\curveto(153.41578526,82.2438971)(153.11578526,81.9688971)(153.11578526,81.5938971)
\curveto(153.11578526,81.1938971)(153.44078526,80.9438971)(153.74078526,80.9438971)
\curveto(154.11578526,80.9438971)(154.39078526,81.2188971)(154.39078526,81.5938971)
\closepath
\moveto(152.69078526,79.5188971)
\lineto(152.69078526,79.1188971)
\curveto(153.39078526,79.1188971)(153.49078526,79.0438971)(153.49078526,78.4938971)
\lineto(153.49078526,75.5938971)
\curveto(153.49078526,75.1188971)(153.39078526,75.1188971)(152.66578526,75.1188971)
\lineto(152.66578526,74.7188971)
\curveto(152.69078526,74.7188971)(153.46578526,74.7688971)(153.91578526,74.7688971)
\curveto(154.31578526,74.7688971)(154.71578526,74.7438971)(155.09078526,74.7188971)
\lineto(155.09078526,75.1188971)
\curveto(154.44078526,75.1188971)(154.34078526,75.1188971)(154.34078526,75.5938971)
\lineto(154.34078526,79.6438971)
\closepath
\moveto(152.69078526,79.5188971)
}
}
{
\newrgbcolor{curcolor}{0 0 0}
\pscustom[linestyle=none,fillstyle=solid,fillcolor=curcolor]
{
\newpath
\moveto(157.89930577,79.1188971)
\lineto(159.62430577,79.1188971)
\lineto(159.62430577,79.5188971)
\lineto(157.89930577,79.5188971)
\lineto(157.89930577,81.5688971)
\lineto(157.49930577,81.5688971)
\curveto(157.49930577,80.5688971)(157.04930577,79.4688971)(155.97430577,79.4438971)
\lineto(155.97430577,79.1188971)
\lineto(156.99930577,79.1188971)
\lineto(156.99930577,76.1188971)
\curveto(156.99930577,74.8688971)(157.94930577,74.6188971)(158.57430577,74.6188971)
\curveto(159.32430577,74.6188971)(159.82430577,75.2438971)(159.82430577,76.1188971)
\lineto(159.82430577,76.7438971)
\lineto(159.44930577,76.7438971)
\lineto(159.44930577,76.1438971)
\curveto(159.44930577,75.3688971)(159.09930577,74.9688971)(158.64930577,74.9688971)
\curveto(157.89930577,74.9688971)(157.89930577,75.8938971)(157.89930577,76.0938971)
\closepath
\moveto(157.89930577,79.1188971)
}
}
{
\newrgbcolor{curcolor}{0 0 0}
\pscustom[linestyle=none,fillstyle=solid,fillcolor=curcolor]
{
\newpath
\moveto(367.31689488,165.99381137)
\lineto(368.81689488,165.99381137)
\curveto(369.14189488,165.99381137)(369.31689488,165.99381137)(369.31689488,166.31881137)
\curveto(369.31689488,166.49381137)(369.14189488,166.49381137)(368.86689488,166.49381137)
\lineto(367.46689488,166.49381137)
\curveto(368.04189488,168.76881137)(368.11689488,169.06881137)(368.11689488,169.16881137)
\curveto(368.11689488,169.44381137)(367.91689488,169.59381137)(367.64189488,169.59381137)
\curveto(367.59189488,169.59381137)(367.14189488,169.59381137)(367.01689488,169.01881137)
\lineto(366.39189488,166.49381137)
\lineto(364.89189488,166.49381137)
\curveto(364.56689488,166.49381137)(364.41689488,166.49381137)(364.41689488,166.19381137)
\curveto(364.41689488,165.99381137)(364.54189488,165.99381137)(364.86689488,165.99381137)
\lineto(366.26689488,165.99381137)
\curveto(365.11689488,161.46881137)(365.04189488,161.19381137)(365.04189488,160.91881137)
\curveto(365.04189488,160.04381137)(365.64189488,159.44381137)(366.51689488,159.44381137)
\curveto(368.14189488,159.44381137)(369.04189488,161.76881137)(369.04189488,161.89381137)
\curveto(369.04189488,162.06881137)(368.91689488,162.06881137)(368.86689488,162.06881137)
\curveto(368.71689488,162.06881137)(368.69189488,162.01881137)(368.61689488,161.84381137)
\curveto(367.94189488,160.16881137)(367.09189488,159.79381137)(366.54189488,159.79381137)
\curveto(366.21689488,159.79381137)(366.04189488,159.99381137)(366.04189488,160.51881137)
\curveto(366.04189488,160.91881137)(366.09189488,161.01881137)(366.14189488,161.29381137)
\closepath
\moveto(367.31689488,165.99381137)
}
}
{
\newrgbcolor{curcolor}{0 0 0}
\pscustom[linestyle=none,fillstyle=solid,fillcolor=curcolor]
{
\newpath
\moveto(378.9736942,160.6034061)
\curveto(378.9736942,161.5784061)(378.4736942,162.1534061)(377.2986942,162.1534061)
\curveto(376.3986942,162.1534061)(375.7986942,161.6784061)(375.4986942,161.1034061)
\curveto(375.2736942,161.9034061)(374.6736942,162.1534061)(373.8736942,162.1534061)
\curveto(372.9486942,162.1534061)(372.3736942,161.6534061)(372.0486942,161.0534061)
\lineto(372.0486942,162.1534061)
\lineto(370.3986942,162.0284061)
\lineto(370.3986942,161.6284061)
\curveto(371.1486942,161.6284061)(371.2486942,161.5534061)(371.2486942,161.0034061)
\lineto(371.2486942,158.1034061)
\curveto(371.2486942,157.6284061)(371.1236942,157.6284061)(370.3986942,157.6284061)
\lineto(370.3986942,157.2284061)
\curveto(370.4236942,157.2284061)(371.1986942,157.2784061)(371.6736942,157.2784061)
\curveto(372.0736942,157.2784061)(372.8486942,157.2284061)(372.9486942,157.2284061)
\lineto(372.9486942,157.6284061)
\curveto(372.2236942,157.6284061)(372.1236942,157.6284061)(372.1236942,158.1034061)
\lineto(372.1236942,160.1284061)
\curveto(372.1236942,161.2784061)(373.0486942,161.8284061)(373.7736942,161.8284061)
\curveto(374.5486942,161.8284061)(374.6486942,161.2284061)(374.6486942,160.6534061)
\lineto(374.6486942,158.1034061)
\curveto(374.6486942,157.6284061)(374.5486942,157.6284061)(373.8236942,157.6284061)
\lineto(373.8236942,157.2284061)
\curveto(373.8486942,157.2284061)(374.6236942,157.2784061)(375.0986942,157.2784061)
\curveto(375.4986942,157.2784061)(376.2736942,157.2284061)(376.3736942,157.2284061)
\lineto(376.3736942,157.6284061)
\curveto(375.6486942,157.6284061)(375.5486942,157.6284061)(375.5486942,158.1034061)
\lineto(375.5486942,160.1284061)
\curveto(375.5486942,161.2784061)(376.4736942,161.8284061)(377.1986942,161.8284061)
\curveto(377.9736942,161.8284061)(378.0736942,161.2284061)(378.0736942,160.6534061)
\lineto(378.0736942,158.1034061)
\curveto(378.0736942,157.6284061)(377.9736942,157.6284061)(377.2486942,157.6284061)
\lineto(377.2486942,157.2284061)
\curveto(377.2736942,157.2284061)(378.0486942,157.2784061)(378.5236942,157.2784061)
\curveto(378.9236942,157.2784061)(379.6986942,157.2284061)(379.7986942,157.2284061)
\lineto(379.7986942,157.6284061)
\curveto(379.0736942,157.6284061)(378.9736942,157.6284061)(378.9736942,158.1034061)
\closepath
\moveto(378.9736942,160.6034061)
}
}
{
\newrgbcolor{curcolor}{0 0 0}
\pscustom[linestyle=none,fillstyle=solid,fillcolor=curcolor]
{
\newpath
\moveto(382.62054479,164.1034061)
\curveto(382.62054479,164.4284061)(382.34554479,164.7534061)(381.97054479,164.7534061)
\curveto(381.64554479,164.7534061)(381.34554479,164.4784061)(381.34554479,164.1034061)
\curveto(381.34554479,163.7034061)(381.67054479,163.4534061)(381.97054479,163.4534061)
\curveto(382.34554479,163.4534061)(382.62054479,163.7284061)(382.62054479,164.1034061)
\closepath
\moveto(380.92054479,162.0284061)
\lineto(380.92054479,161.6284061)
\curveto(381.62054479,161.6284061)(381.72054479,161.5534061)(381.72054479,161.0034061)
\lineto(381.72054479,158.1034061)
\curveto(381.72054479,157.6284061)(381.62054479,157.6284061)(380.89554479,157.6284061)
\lineto(380.89554479,157.2284061)
\curveto(380.92054479,157.2284061)(381.69554479,157.2784061)(382.14554479,157.2784061)
\curveto(382.54554479,157.2784061)(382.94554479,157.2534061)(383.32054479,157.2284061)
\lineto(383.32054479,157.6284061)
\curveto(382.67054479,157.6284061)(382.57054479,157.6284061)(382.57054479,158.1034061)
\lineto(382.57054479,162.1534061)
\closepath
\moveto(380.92054479,162.0284061)
}
}
{
\newrgbcolor{curcolor}{0 0 0}
\pscustom[linestyle=none,fillstyle=solid,fillcolor=curcolor]
{
\newpath
\moveto(389.62905309,160.6034061)
\curveto(389.62905309,161.5784061)(389.15405309,162.1534061)(387.95405309,162.1534061)
\curveto(387.02905309,162.1534061)(386.45405309,161.6534061)(386.12905309,161.0534061)
\lineto(386.12905309,162.1534061)
\lineto(384.47905309,162.0284061)
\lineto(384.47905309,161.6284061)
\curveto(385.22905309,161.6284061)(385.32905309,161.5534061)(385.32905309,161.0034061)
\lineto(385.32905309,158.1034061)
\curveto(385.32905309,157.6284061)(385.20405309,157.6284061)(384.47905309,157.6284061)
\lineto(384.47905309,157.2284061)
\curveto(384.50405309,157.2284061)(385.27905309,157.2784061)(385.75405309,157.2784061)
\curveto(386.15405309,157.2784061)(386.92905309,157.2284061)(387.02905309,157.2284061)
\lineto(387.02905309,157.6284061)
\curveto(386.30405309,157.6284061)(386.20405309,157.6284061)(386.20405309,158.1034061)
\lineto(386.20405309,160.1284061)
\curveto(386.20405309,161.2784061)(387.12905309,161.8284061)(387.85405309,161.8284061)
\curveto(388.62905309,161.8284061)(388.72905309,161.2284061)(388.72905309,160.6534061)
\lineto(388.72905309,158.1034061)
\curveto(388.72905309,157.6284061)(388.62905309,157.6284061)(387.90405309,157.6284061)
\lineto(387.90405309,157.2284061)
\curveto(387.92905309,157.2284061)(388.70405309,157.2784061)(389.17905309,157.2784061)
\curveto(389.57905309,157.2784061)(390.35405309,157.2284061)(390.45405309,157.2284061)
\lineto(390.45405309,157.6284061)
\curveto(389.72905309,157.6284061)(389.62905309,157.6284061)(389.62905309,158.1034061)
\closepath
\moveto(389.62905309,160.6034061)
}
}
{
\newrgbcolor{curcolor}{0 0 0}
\pscustom[linestyle=none,fillstyle=solid,fillcolor=curcolor]
{
\newpath
\moveto(39.34999488,144.77677737)
\lineto(40.84999488,144.77677737)
\curveto(41.17499488,144.77677737)(41.34999488,144.77677737)(41.34999488,145.10177737)
\curveto(41.34999488,145.27677737)(41.17499488,145.27677737)(40.89999488,145.27677737)
\lineto(39.49999488,145.27677737)
\curveto(40.07499488,147.55177737)(40.14999488,147.85177737)(40.14999488,147.95177737)
\curveto(40.14999488,148.22677737)(39.94999488,148.37677737)(39.67499488,148.37677737)
\curveto(39.62499488,148.37677737)(39.17499488,148.37677737)(39.04999488,147.80177737)
\lineto(38.42499488,145.27677737)
\lineto(36.92499488,145.27677737)
\curveto(36.59999488,145.27677737)(36.44999488,145.27677737)(36.44999488,144.97677737)
\curveto(36.44999488,144.77677737)(36.57499488,144.77677737)(36.89999488,144.77677737)
\lineto(38.29999488,144.77677737)
\curveto(37.14999488,140.25177737)(37.07499488,139.97677737)(37.07499488,139.70177737)
\curveto(37.07499488,138.82677737)(37.67499488,138.22677737)(38.54999488,138.22677737)
\curveto(40.17499488,138.22677737)(41.07499488,140.55177737)(41.07499488,140.67677737)
\curveto(41.07499488,140.85177737)(40.94999488,140.85177737)(40.89999488,140.85177737)
\curveto(40.74999488,140.85177737)(40.72499488,140.80177737)(40.64999488,140.62677737)
\curveto(39.97499488,138.95177737)(39.12499488,138.57677737)(38.57499488,138.57677737)
\curveto(38.24999488,138.57677737)(38.07499488,138.77677737)(38.07499488,139.30177737)
\curveto(38.07499488,139.70177737)(38.12499488,139.80177737)(38.17499488,140.07677737)
\closepath
\moveto(39.34999488,144.77677737)
}
}
{
\newrgbcolor{curcolor}{0 0 0}
\pscustom[linestyle=none,fillstyle=solid,fillcolor=curcolor]
{
\newpath
\moveto(51.0067942,139.3863721)
\curveto(51.0067942,140.3613721)(50.5067942,140.9363721)(49.3317942,140.9363721)
\curveto(48.4317942,140.9363721)(47.8317942,140.4613721)(47.5317942,139.8863721)
\curveto(47.3067942,140.6863721)(46.7067942,140.9363721)(45.9067942,140.9363721)
\curveto(44.9817942,140.9363721)(44.4067942,140.4363721)(44.0817942,139.8363721)
\lineto(44.0817942,140.9363721)
\lineto(42.4317942,140.8113721)
\lineto(42.4317942,140.4113721)
\curveto(43.1817942,140.4113721)(43.2817942,140.3363721)(43.2817942,139.7863721)
\lineto(43.2817942,136.8863721)
\curveto(43.2817942,136.4113721)(43.1567942,136.4113721)(42.4317942,136.4113721)
\lineto(42.4317942,136.0113721)
\curveto(42.4567942,136.0113721)(43.2317942,136.0613721)(43.7067942,136.0613721)
\curveto(44.1067942,136.0613721)(44.8817942,136.0113721)(44.9817942,136.0113721)
\lineto(44.9817942,136.4113721)
\curveto(44.2567942,136.4113721)(44.1567942,136.4113721)(44.1567942,136.8863721)
\lineto(44.1567942,138.9113721)
\curveto(44.1567942,140.0613721)(45.0817942,140.6113721)(45.8067942,140.6113721)
\curveto(46.5817942,140.6113721)(46.6817942,140.0113721)(46.6817942,139.4363721)
\lineto(46.6817942,136.8863721)
\curveto(46.6817942,136.4113721)(46.5817942,136.4113721)(45.8567942,136.4113721)
\lineto(45.8567942,136.0113721)
\curveto(45.8817942,136.0113721)(46.6567942,136.0613721)(47.1317942,136.0613721)
\curveto(47.5317942,136.0613721)(48.3067942,136.0113721)(48.4067942,136.0113721)
\lineto(48.4067942,136.4113721)
\curveto(47.6817942,136.4113721)(47.5817942,136.4113721)(47.5817942,136.8863721)
\lineto(47.5817942,138.9113721)
\curveto(47.5817942,140.0613721)(48.5067942,140.6113721)(49.2317942,140.6113721)
\curveto(50.0067942,140.6113721)(50.1067942,140.0113721)(50.1067942,139.4363721)
\lineto(50.1067942,136.8863721)
\curveto(50.1067942,136.4113721)(50.0067942,136.4113721)(49.2817942,136.4113721)
\lineto(49.2817942,136.0113721)
\curveto(49.3067942,136.0113721)(50.0817942,136.0613721)(50.5567942,136.0613721)
\curveto(50.9567942,136.0613721)(51.7317942,136.0113721)(51.8317942,136.0113721)
\lineto(51.8317942,136.4113721)
\curveto(51.1067942,136.4113721)(51.0067942,136.4113721)(51.0067942,136.8863721)
\closepath
\moveto(51.0067942,139.3863721)
}
}
{
\newrgbcolor{curcolor}{0 0 0}
\pscustom[linestyle=none,fillstyle=solid,fillcolor=curcolor]
{
\newpath
\moveto(54.65364479,142.8863721)
\curveto(54.65364479,143.2113721)(54.37864479,143.5363721)(54.00364479,143.5363721)
\curveto(53.67864479,143.5363721)(53.37864479,143.2613721)(53.37864479,142.8863721)
\curveto(53.37864479,142.4863721)(53.70364479,142.2363721)(54.00364479,142.2363721)
\curveto(54.37864479,142.2363721)(54.65364479,142.5113721)(54.65364479,142.8863721)
\closepath
\moveto(52.95364479,140.8113721)
\lineto(52.95364479,140.4113721)
\curveto(53.65364479,140.4113721)(53.75364479,140.3363721)(53.75364479,139.7863721)
\lineto(53.75364479,136.8863721)
\curveto(53.75364479,136.4113721)(53.65364479,136.4113721)(52.92864479,136.4113721)
\lineto(52.92864479,136.0113721)
\curveto(52.95364479,136.0113721)(53.72864479,136.0613721)(54.17864479,136.0613721)
\curveto(54.57864479,136.0613721)(54.97864479,136.0363721)(55.35364479,136.0113721)
\lineto(55.35364479,136.4113721)
\curveto(54.70364479,136.4113721)(54.60364479,136.4113721)(54.60364479,136.8863721)
\lineto(54.60364479,140.9363721)
\closepath
\moveto(52.95364479,140.8113721)
}
}
{
\newrgbcolor{curcolor}{0 0 0}
\pscustom[linestyle=none,fillstyle=solid,fillcolor=curcolor]
{
\newpath
\moveto(61.66215309,139.3863721)
\curveto(61.66215309,140.3613721)(61.18715309,140.9363721)(59.98715309,140.9363721)
\curveto(59.06215309,140.9363721)(58.48715309,140.4363721)(58.16215309,139.8363721)
\lineto(58.16215309,140.9363721)
\lineto(56.51215309,140.8113721)
\lineto(56.51215309,140.4113721)
\curveto(57.26215309,140.4113721)(57.36215309,140.3363721)(57.36215309,139.7863721)
\lineto(57.36215309,136.8863721)
\curveto(57.36215309,136.4113721)(57.23715309,136.4113721)(56.51215309,136.4113721)
\lineto(56.51215309,136.0113721)
\curveto(56.53715309,136.0113721)(57.31215309,136.0613721)(57.78715309,136.0613721)
\curveto(58.18715309,136.0613721)(58.96215309,136.0113721)(59.06215309,136.0113721)
\lineto(59.06215309,136.4113721)
\curveto(58.33715309,136.4113721)(58.23715309,136.4113721)(58.23715309,136.8863721)
\lineto(58.23715309,138.9113721)
\curveto(58.23715309,140.0613721)(59.16215309,140.6113721)(59.88715309,140.6113721)
\curveto(60.66215309,140.6113721)(60.76215309,140.0113721)(60.76215309,139.4363721)
\lineto(60.76215309,136.8863721)
\curveto(60.76215309,136.4113721)(60.66215309,136.4113721)(59.93715309,136.4113721)
\lineto(59.93715309,136.0113721)
\curveto(59.96215309,136.0113721)(60.73715309,136.0613721)(61.21215309,136.0613721)
\curveto(61.61215309,136.0613721)(62.38715309,136.0113721)(62.48715309,136.0113721)
\lineto(62.48715309,136.4113721)
\curveto(61.76215309,136.4113721)(61.66215309,136.4113721)(61.66215309,136.8863721)
\closepath
\moveto(61.66215309,139.3863721)
}
}
{
\newrgbcolor{curcolor}{0 0 0}
\pscustom[linestyle=none,fillstyle=solid,fillcolor=curcolor]
{
\newpath
\moveto(84.42688888,159.63906437)
\lineto(85.92688888,159.63906437)
\curveto(86.25188888,159.63906437)(86.42688888,159.63906437)(86.42688888,159.96406437)
\curveto(86.42688888,160.13906437)(86.25188888,160.13906437)(85.97688888,160.13906437)
\lineto(84.57688888,160.13906437)
\curveto(85.15188888,162.41406437)(85.22688888,162.71406437)(85.22688888,162.81406437)
\curveto(85.22688888,163.08906437)(85.02688888,163.23906437)(84.75188888,163.23906437)
\curveto(84.70188888,163.23906437)(84.25188888,163.23906437)(84.12688888,162.66406437)
\lineto(83.50188888,160.13906437)
\lineto(82.00188888,160.13906437)
\curveto(81.67688888,160.13906437)(81.52688888,160.13906437)(81.52688888,159.83906437)
\curveto(81.52688888,159.63906437)(81.65188888,159.63906437)(81.97688888,159.63906437)
\lineto(83.37688888,159.63906437)
\curveto(82.22688888,155.11406437)(82.15188888,154.83906437)(82.15188888,154.56406437)
\curveto(82.15188888,153.68906437)(82.75188888,153.08906437)(83.62688888,153.08906437)
\curveto(85.25188888,153.08906437)(86.15188888,155.41406437)(86.15188888,155.53906437)
\curveto(86.15188888,155.71406437)(86.02688888,155.71406437)(85.97688888,155.71406437)
\curveto(85.82688888,155.71406437)(85.80188888,155.66406437)(85.72688888,155.48906437)
\curveto(85.05188888,153.81406437)(84.20188888,153.43906437)(83.65188888,153.43906437)
\curveto(83.32688888,153.43906437)(83.15188888,153.63906437)(83.15188888,154.16406437)
\curveto(83.15188888,154.56406437)(83.20188888,154.66406437)(83.25188888,154.93906437)
\closepath
\moveto(84.42688888,159.63906437)
}
}
{
\newrgbcolor{curcolor}{0 0 0}
\pscustom[linestyle=none,fillstyle=solid,fillcolor=curcolor]
{
\newpath
\moveto(96.0836882,154.2486591)
\curveto(96.0836882,155.2236591)(95.5836882,155.7986591)(94.4086882,155.7986591)
\curveto(93.5086882,155.7986591)(92.9086882,155.3236591)(92.6086882,154.7486591)
\curveto(92.3836882,155.5486591)(91.7836882,155.7986591)(90.9836882,155.7986591)
\curveto(90.0586882,155.7986591)(89.4836882,155.2986591)(89.1586882,154.6986591)
\lineto(89.1586882,155.7986591)
\lineto(87.5086882,155.6736591)
\lineto(87.5086882,155.2736591)
\curveto(88.2586882,155.2736591)(88.3586882,155.1986591)(88.3586882,154.6486591)
\lineto(88.3586882,151.7486591)
\curveto(88.3586882,151.2736591)(88.2336882,151.2736591)(87.5086882,151.2736591)
\lineto(87.5086882,150.8736591)
\curveto(87.5336882,150.8736591)(88.3086882,150.9236591)(88.7836882,150.9236591)
\curveto(89.1836882,150.9236591)(89.9586882,150.8736591)(90.0586882,150.8736591)
\lineto(90.0586882,151.2736591)
\curveto(89.3336882,151.2736591)(89.2336882,151.2736591)(89.2336882,151.7486591)
\lineto(89.2336882,153.7736591)
\curveto(89.2336882,154.9236591)(90.1586882,155.4736591)(90.8836882,155.4736591)
\curveto(91.6586882,155.4736591)(91.7586882,154.8736591)(91.7586882,154.2986591)
\lineto(91.7586882,151.7486591)
\curveto(91.7586882,151.2736591)(91.6586882,151.2736591)(90.9336882,151.2736591)
\lineto(90.9336882,150.8736591)
\curveto(90.9586882,150.8736591)(91.7336882,150.9236591)(92.2086882,150.9236591)
\curveto(92.6086882,150.9236591)(93.3836882,150.8736591)(93.4836882,150.8736591)
\lineto(93.4836882,151.2736591)
\curveto(92.7586882,151.2736591)(92.6586882,151.2736591)(92.6586882,151.7486591)
\lineto(92.6586882,153.7736591)
\curveto(92.6586882,154.9236591)(93.5836882,155.4736591)(94.3086882,155.4736591)
\curveto(95.0836882,155.4736591)(95.1836882,154.8736591)(95.1836882,154.2986591)
\lineto(95.1836882,151.7486591)
\curveto(95.1836882,151.2736591)(95.0836882,151.2736591)(94.3586882,151.2736591)
\lineto(94.3586882,150.8736591)
\curveto(94.3836882,150.8736591)(95.1586882,150.9236591)(95.6336882,150.9236591)
\curveto(96.0336882,150.9236591)(96.8086882,150.8736591)(96.9086882,150.8736591)
\lineto(96.9086882,151.2736591)
\curveto(96.1836882,151.2736591)(96.0836882,151.2736591)(96.0836882,151.7486591)
\closepath
\moveto(96.0836882,154.2486591)
}
}
{
\newrgbcolor{curcolor}{0 0 0}
\pscustom[linestyle=none,fillstyle=solid,fillcolor=curcolor]
{
\newpath
\moveto(102.35553879,153.8736591)
\curveto(102.35553879,154.4486591)(102.35553879,154.8736591)(101.83053879,155.2986591)
\curveto(101.38053879,155.6736591)(100.85553879,155.8486591)(100.18053879,155.8486591)
\curveto(99.13053879,155.8486591)(98.38053879,155.4486591)(98.38053879,154.7736591)
\curveto(98.38053879,154.3986591)(98.63053879,154.2236591)(98.93053879,154.2236591)
\curveto(99.23053879,154.2236591)(99.45553879,154.4486591)(99.45553879,154.7486591)
\curveto(99.45553879,154.9236591)(99.35553879,155.1736591)(99.05553879,155.2486591)
\curveto(99.45553879,155.5236591)(100.10553879,155.5236591)(100.15553879,155.5236591)
\curveto(100.78053879,155.5236591)(101.48053879,155.1236591)(101.48053879,154.1736591)
\lineto(101.48053879,153.8486591)
\curveto(100.85553879,153.8236591)(100.13053879,153.7736591)(99.30553879,153.4736591)
\curveto(98.30553879,153.1236591)(98.00553879,152.4986591)(98.00553879,151.9986591)
\curveto(98.00553879,151.0486591)(99.15553879,150.7736591)(99.95553879,150.7736591)
\curveto(100.83053879,150.7736591)(101.35553879,151.2736591)(101.60553879,151.6986591)
\curveto(101.63053879,151.2486591)(101.93053879,150.8236591)(102.45553879,150.8236591)
\curveto(102.48053879,150.8236591)(103.55553879,150.8236591)(103.55553879,151.8736591)
\lineto(103.55553879,152.4986591)
\lineto(103.18053879,152.4986591)
\lineto(103.18053879,151.8986591)
\curveto(103.18053879,151.7736591)(103.18053879,151.2486591)(102.75553879,151.2486591)
\curveto(102.35553879,151.2486591)(102.35553879,151.7736591)(102.35553879,151.8986591)
\closepath
\moveto(101.48053879,152.4486591)
\curveto(101.48053879,151.3736591)(100.53053879,151.0736591)(100.03053879,151.0736591)
\curveto(99.45553879,151.0736591)(98.93053879,151.4486591)(98.93053879,151.9986591)
\curveto(98.93053879,152.6236591)(99.45553879,153.4736591)(101.48053879,153.5486591)
\closepath
\moveto(101.48053879,152.4486591)
}
}
{
\newrgbcolor{curcolor}{0 0 0}
\pscustom[linestyle=none,fillstyle=solid,fillcolor=curcolor]
{
\newpath
\moveto(107.51006759,153.3736591)
\lineto(107.43506759,153.4736591)
\curveto(107.43506759,153.5236591)(108.16006759,154.2986591)(108.26006759,154.3986591)
\curveto(108.66006759,154.8486591)(109.03506759,155.2736591)(109.93506759,155.2736591)
\lineto(109.93506759,155.6736591)
\curveto(109.61006759,155.6486591)(109.28506759,155.6236591)(108.98506759,155.6236591)
\curveto(108.66006759,155.6236591)(108.21006759,155.6486591)(107.88506759,155.6736591)
\lineto(107.88506759,155.2736591)
\curveto(108.06006759,155.2486591)(108.11006759,155.1236591)(108.11006759,155.0236591)
\curveto(108.11006759,154.9986591)(108.11006759,154.8236591)(107.93506759,154.6486591)
\lineto(107.16006759,153.7736591)
\lineto(106.23506759,154.8236591)
\curveto(106.11006759,154.9486591)(106.11006759,154.9986591)(106.11006759,155.0486591)
\curveto(106.11006759,155.1986591)(106.26006759,155.2736591)(106.41006759,155.2736591)
\lineto(106.41006759,155.6736591)
\curveto(106.01006759,155.6486591)(105.58506759,155.6236591)(105.16006759,155.6236591)
\curveto(104.83506759,155.6236591)(104.41006759,155.6486591)(104.08506759,155.6736591)
\lineto(104.08506759,155.2736591)
\curveto(104.58506759,155.2736591)(104.88506759,155.2736591)(105.18506759,154.9486591)
\lineto(106.56006759,153.3486591)
\curveto(106.58506759,153.3236591)(106.66006759,153.2486591)(106.66006759,153.2236591)
\curveto(106.66006759,153.1986591)(105.81006759,152.2736591)(105.71006759,152.1486591)
\curveto(105.28506759,151.6986591)(104.91006759,151.2986591)(104.03506759,151.2736591)
\lineto(104.03506759,150.8736591)
\curveto(104.36006759,150.8986591)(104.61006759,150.9236591)(104.96006759,150.9236591)
\curveto(105.28506759,150.9236591)(105.73506759,150.8986591)(106.06006759,150.8736591)
\lineto(106.06006759,151.2736591)
\curveto(105.93506759,151.2986591)(105.83506759,151.3736591)(105.83506759,151.5236591)
\curveto(105.83506759,151.7486591)(105.96006759,151.8986591)(106.13506759,152.0736591)
\lineto(106.93506759,152.9486591)
\lineto(107.88506759,151.8236591)
\curveto(108.11006759,151.5986591)(108.11006759,151.5736591)(108.11006759,151.4986591)
\curveto(108.11006759,151.2986591)(107.86006759,151.2736591)(107.81006759,151.2736591)
\lineto(107.81006759,150.8736591)
\curveto(107.91006759,150.8736591)(108.71006759,150.9236591)(109.06006759,150.9236591)
\curveto(109.41006759,150.9236591)(109.78506759,150.8986591)(110.13506759,150.8736591)
\lineto(110.13506759,151.2736591)
\curveto(109.38506759,151.2736591)(109.31006759,151.3486591)(109.03506759,151.6236591)
\closepath
\moveto(107.51006759,153.3736591)
}
}
{
\newrgbcolor{curcolor}{0 0 0}
\pscustom[linestyle=none,fillstyle=solid,fillcolor=curcolor]
{
\newpath
\moveto(362.69170488,110.31948197)
\lineto(364.19170488,110.31948197)
\curveto(364.51670488,110.31948197)(364.69170488,110.31948197)(364.69170488,110.64448197)
\curveto(364.69170488,110.81948197)(364.51670488,110.81948197)(364.24170488,110.81948197)
\lineto(362.84170488,110.81948197)
\curveto(363.41670488,113.09448197)(363.49170488,113.39448197)(363.49170488,113.49448197)
\curveto(363.49170488,113.76948197)(363.29170488,113.91948197)(363.01670488,113.91948197)
\curveto(362.96670488,113.91948197)(362.51670488,113.91948197)(362.39170488,113.34448197)
\lineto(361.76670488,110.81948197)
\lineto(360.26670488,110.81948197)
\curveto(359.94170488,110.81948197)(359.79170488,110.81948197)(359.79170488,110.51948197)
\curveto(359.79170488,110.31948197)(359.91670488,110.31948197)(360.24170488,110.31948197)
\lineto(361.64170488,110.31948197)
\curveto(360.49170488,105.79448197)(360.41670488,105.51948197)(360.41670488,105.24448197)
\curveto(360.41670488,104.36948197)(361.01670488,103.76948197)(361.89170488,103.76948197)
\curveto(363.51670488,103.76948197)(364.41670488,106.09448197)(364.41670488,106.21948197)
\curveto(364.41670488,106.39448197)(364.29170488,106.39448197)(364.24170488,106.39448197)
\curveto(364.09170488,106.39448197)(364.06670488,106.34448197)(363.99170488,106.16948197)
\curveto(363.31670488,104.49448197)(362.46670488,104.11948197)(361.91670488,104.11948197)
\curveto(361.59170488,104.11948197)(361.41670488,104.31948197)(361.41670488,104.84448197)
\curveto(361.41670488,105.24448197)(361.46670488,105.34448197)(361.51670488,105.61948197)
\closepath
\moveto(362.69170488,110.31948197)
}
}
{
\newrgbcolor{curcolor}{0 0 0}
\pscustom[linestyle=none,fillstyle=solid,fillcolor=curcolor]
{
\newpath
\moveto(374.3485042,104.9290767)
\curveto(374.3485042,105.9040767)(373.8485042,106.4790767)(372.6735042,106.4790767)
\curveto(371.7735042,106.4790767)(371.1735042,106.0040767)(370.8735042,105.4290767)
\curveto(370.6485042,106.2290767)(370.0485042,106.4790767)(369.2485042,106.4790767)
\curveto(368.3235042,106.4790767)(367.7485042,105.9790767)(367.4235042,105.3790767)
\lineto(367.4235042,106.4790767)
\lineto(365.7735042,106.3540767)
\lineto(365.7735042,105.9540767)
\curveto(366.5235042,105.9540767)(366.6235042,105.8790767)(366.6235042,105.3290767)
\lineto(366.6235042,102.4290767)
\curveto(366.6235042,101.9540767)(366.4985042,101.9540767)(365.7735042,101.9540767)
\lineto(365.7735042,101.5540767)
\curveto(365.7985042,101.5540767)(366.5735042,101.6040767)(367.0485042,101.6040767)
\curveto(367.4485042,101.6040767)(368.2235042,101.5540767)(368.3235042,101.5540767)
\lineto(368.3235042,101.9540767)
\curveto(367.5985042,101.9540767)(367.4985042,101.9540767)(367.4985042,102.4290767)
\lineto(367.4985042,104.4540767)
\curveto(367.4985042,105.6040767)(368.4235042,106.1540767)(369.1485042,106.1540767)
\curveto(369.9235042,106.1540767)(370.0235042,105.5540767)(370.0235042,104.9790767)
\lineto(370.0235042,102.4290767)
\curveto(370.0235042,101.9540767)(369.9235042,101.9540767)(369.1985042,101.9540767)
\lineto(369.1985042,101.5540767)
\curveto(369.2235042,101.5540767)(369.9985042,101.6040767)(370.4735042,101.6040767)
\curveto(370.8735042,101.6040767)(371.6485042,101.5540767)(371.7485042,101.5540767)
\lineto(371.7485042,101.9540767)
\curveto(371.0235042,101.9540767)(370.9235042,101.9540767)(370.9235042,102.4290767)
\lineto(370.9235042,104.4540767)
\curveto(370.9235042,105.6040767)(371.8485042,106.1540767)(372.5735042,106.1540767)
\curveto(373.3485042,106.1540767)(373.4485042,105.5540767)(373.4485042,104.9790767)
\lineto(373.4485042,102.4290767)
\curveto(373.4485042,101.9540767)(373.3485042,101.9540767)(372.6235042,101.9540767)
\lineto(372.6235042,101.5540767)
\curveto(372.6485042,101.5540767)(373.4235042,101.6040767)(373.8985042,101.6040767)
\curveto(374.2985042,101.6040767)(375.0735042,101.5540767)(375.1735042,101.5540767)
\lineto(375.1735042,101.9540767)
\curveto(374.4485042,101.9540767)(374.3485042,101.9540767)(374.3485042,102.4290767)
\closepath
\moveto(374.3485042,104.9290767)
}
}
{
\newrgbcolor{curcolor}{0 0 0}
\pscustom[linestyle=none,fillstyle=solid,fillcolor=curcolor]
{
\newpath
\moveto(380.62035479,104.5540767)
\curveto(380.62035479,105.1290767)(380.62035479,105.5540767)(380.09535479,105.9790767)
\curveto(379.64535479,106.3540767)(379.12035479,106.5290767)(378.44535479,106.5290767)
\curveto(377.39535479,106.5290767)(376.64535479,106.1290767)(376.64535479,105.4540767)
\curveto(376.64535479,105.0790767)(376.89535479,104.9040767)(377.19535479,104.9040767)
\curveto(377.49535479,104.9040767)(377.72035479,105.1290767)(377.72035479,105.4290767)
\curveto(377.72035479,105.6040767)(377.62035479,105.8540767)(377.32035479,105.9290767)
\curveto(377.72035479,106.2040767)(378.37035479,106.2040767)(378.42035479,106.2040767)
\curveto(379.04535479,106.2040767)(379.74535479,105.8040767)(379.74535479,104.8540767)
\lineto(379.74535479,104.5290767)
\curveto(379.12035479,104.5040767)(378.39535479,104.4540767)(377.57035479,104.1540767)
\curveto(376.57035479,103.8040767)(376.27035479,103.1790767)(376.27035479,102.6790767)
\curveto(376.27035479,101.7290767)(377.42035479,101.4540767)(378.22035479,101.4540767)
\curveto(379.09535479,101.4540767)(379.62035479,101.9540767)(379.87035479,102.3790767)
\curveto(379.89535479,101.9290767)(380.19535479,101.5040767)(380.72035479,101.5040767)
\curveto(380.74535479,101.5040767)(381.82035479,101.5040767)(381.82035479,102.5540767)
\lineto(381.82035479,103.1790767)
\lineto(381.44535479,103.1790767)
\lineto(381.44535479,102.5790767)
\curveto(381.44535479,102.4540767)(381.44535479,101.9290767)(381.02035479,101.9290767)
\curveto(380.62035479,101.9290767)(380.62035479,102.4540767)(380.62035479,102.5790767)
\closepath
\moveto(379.74535479,103.1290767)
\curveto(379.74535479,102.0540767)(378.79535479,101.7540767)(378.29535479,101.7540767)
\curveto(377.72035479,101.7540767)(377.19535479,102.1290767)(377.19535479,102.6790767)
\curveto(377.19535479,103.3040767)(377.72035479,104.1540767)(379.74535479,104.2290767)
\closepath
\moveto(379.74535479,103.1290767)
}
}
{
\newrgbcolor{curcolor}{0 0 0}
\pscustom[linestyle=none,fillstyle=solid,fillcolor=curcolor]
{
\newpath
\moveto(385.77488359,104.0540767)
\lineto(385.69988359,104.1540767)
\curveto(385.69988359,104.2040767)(386.42488359,104.9790767)(386.52488359,105.0790767)
\curveto(386.92488359,105.5290767)(387.29988359,105.9540767)(388.19988359,105.9540767)
\lineto(388.19988359,106.3540767)
\curveto(387.87488359,106.3290767)(387.54988359,106.3040767)(387.24988359,106.3040767)
\curveto(386.92488359,106.3040767)(386.47488359,106.3290767)(386.14988359,106.3540767)
\lineto(386.14988359,105.9540767)
\curveto(386.32488359,105.9290767)(386.37488359,105.8040767)(386.37488359,105.7040767)
\curveto(386.37488359,105.6790767)(386.37488359,105.5040767)(386.19988359,105.3290767)
\lineto(385.42488359,104.4540767)
\lineto(384.49988359,105.5040767)
\curveto(384.37488359,105.6290767)(384.37488359,105.6790767)(384.37488359,105.7290767)
\curveto(384.37488359,105.8790767)(384.52488359,105.9540767)(384.67488359,105.9540767)
\lineto(384.67488359,106.3540767)
\curveto(384.27488359,106.3290767)(383.84988359,106.3040767)(383.42488359,106.3040767)
\curveto(383.09988359,106.3040767)(382.67488359,106.3290767)(382.34988359,106.3540767)
\lineto(382.34988359,105.9540767)
\curveto(382.84988359,105.9540767)(383.14988359,105.9540767)(383.44988359,105.6290767)
\lineto(384.82488359,104.0290767)
\curveto(384.84988359,104.0040767)(384.92488359,103.9290767)(384.92488359,103.9040767)
\curveto(384.92488359,103.8790767)(384.07488359,102.9540767)(383.97488359,102.8290767)
\curveto(383.54988359,102.3790767)(383.17488359,101.9790767)(382.29988359,101.9540767)
\lineto(382.29988359,101.5540767)
\curveto(382.62488359,101.5790767)(382.87488359,101.6040767)(383.22488359,101.6040767)
\curveto(383.54988359,101.6040767)(383.99988359,101.5790767)(384.32488359,101.5540767)
\lineto(384.32488359,101.9540767)
\curveto(384.19988359,101.9790767)(384.09988359,102.0540767)(384.09988359,102.2040767)
\curveto(384.09988359,102.4290767)(384.22488359,102.5790767)(384.39988359,102.7540767)
\lineto(385.19988359,103.6290767)
\lineto(386.14988359,102.5040767)
\curveto(386.37488359,102.2790767)(386.37488359,102.2540767)(386.37488359,102.1790767)
\curveto(386.37488359,101.9790767)(386.12488359,101.9540767)(386.07488359,101.9540767)
\lineto(386.07488359,101.5540767)
\curveto(386.17488359,101.5540767)(386.97488359,101.6040767)(387.32488359,101.6040767)
\curveto(387.67488359,101.6040767)(388.04988359,101.5790767)(388.39988359,101.5540767)
\lineto(388.39988359,101.9540767)
\curveto(387.64988359,101.9540767)(387.57488359,102.0290767)(387.29988359,102.3040767)
\closepath
\moveto(385.77488359,104.0540767)
}
}
{
\newrgbcolor{curcolor}{0 0 0}
\pscustom[linestyle=none,fillstyle=solid,fillcolor=curcolor]
{
\newpath
\moveto(62.7244675,42.19236587)
\lineto(65.5994675,42.19236587)
\lineto(72.07603,34.24705337)
\lineto(72.07603,40.35642837)
\curveto(72.07603,41.00747003)(72.00311333,41.41372003)(71.85728,41.57517837)
\curveto(71.66457167,41.79392837)(71.35988417,41.90330337)(70.9432175,41.90330337)
\lineto(70.57603,41.90330337)
\lineto(70.57603,42.19236587)
\lineto(74.26353,42.19236587)
\lineto(74.26353,41.90330337)
\lineto(73.88853,41.90330337)
\curveto(73.44061333,41.90330337)(73.122905,41.7678867)(72.935405,41.49705337)
\curveto(72.82082167,41.3303867)(72.76353,40.95017837)(72.76353,40.35642837)
\lineto(72.76353,31.42674087)
\lineto(72.48228,31.42674087)
\lineto(65.497905,39.95799087)
\lineto(65.497905,33.43455337)
\curveto(65.497905,32.7835117)(65.5682175,32.3772617)(65.7088425,32.21580337)
\curveto(65.90675917,31.99705337)(66.21144667,31.88767837)(66.622905,31.88767837)
\lineto(66.997905,31.88767837)
\lineto(66.997905,31.59861587)
\lineto(63.310405,31.59861587)
\lineto(63.310405,31.88767837)
\lineto(63.6775925,31.88767837)
\curveto(64.1307175,31.88767837)(64.45103,32.02309503)(64.63853,32.29392837)
\curveto(64.75311333,32.46059503)(64.810405,32.84080337)(64.810405,33.43455337)
\lineto(64.810405,40.80174087)
\curveto(64.50311333,41.16111587)(64.26873833,41.39809503)(64.10728,41.51267837)
\curveto(63.95103,41.6272617)(63.71925917,41.73403253)(63.4119675,41.83299087)
\curveto(63.26092583,41.87986587)(63.03175917,41.90330337)(62.7244675,41.90330337)
\closepath
}
}
{
\newrgbcolor{curcolor}{0 0 0}
\pscustom[linestyle=none,fillstyle=solid,fillcolor=curcolor]
{
\newpath
\moveto(76.1932175,36.05955337)
\curveto(76.18800917,34.99705337)(76.44582167,34.16372003)(76.966655,33.55955337)
\curveto(77.48748833,32.9553867)(78.0994675,32.65330337)(78.8025925,32.65330337)
\curveto(79.2713425,32.65330337)(79.6775925,32.78090753)(80.0213425,33.03611587)
\curveto(80.37030083,33.29653253)(80.6619675,33.73924087)(80.8963425,34.36424087)
\lineto(81.13853,34.20799087)
\curveto(81.029155,33.4944492)(80.71144667,32.84340753)(80.185405,32.25486587)
\curveto(79.65936333,31.67153253)(79.00050917,31.37986587)(78.2088425,31.37986587)
\curveto(77.3494675,31.37986587)(76.61248833,31.7131992)(75.997905,32.37986587)
\curveto(75.38853,33.05174087)(75.0838425,33.95278253)(75.0838425,35.08299087)
\curveto(75.0838425,36.3069492)(75.3963425,37.2600742)(76.0213425,37.94236587)
\curveto(76.65155083,38.62986587)(77.44061333,38.97361587)(78.38853,38.97361587)
\curveto(79.19061333,38.97361587)(79.8494675,38.70799087)(80.3650925,38.17674087)
\curveto(80.8807175,37.6506992)(81.13853,36.94497003)(81.13853,36.05955337)
\closepath
\moveto(76.1932175,36.51267837)
\lineto(79.5057175,36.51267837)
\curveto(79.47967583,36.9710117)(79.42498833,37.29392837)(79.341655,37.48142837)
\curveto(79.21144667,37.77309503)(79.01613417,38.0022617)(78.7557175,38.16892837)
\curveto(78.50050917,38.33559503)(78.23228,38.41892837)(77.95103,38.41892837)
\curveto(77.51873833,38.41892837)(77.1307175,38.24965753)(76.7869675,37.91111587)
\curveto(76.44842583,37.57778253)(76.25050917,37.1116367)(76.1932175,36.51267837)
\closepath
}
}
{
\newrgbcolor{curcolor}{0 0 0}
\pscustom[linestyle=none,fillstyle=solid,fillcolor=curcolor]
{
\newpath
\moveto(86.1463425,32.62986587)
\curveto(85.4119675,32.06215753)(84.95103,31.73403253)(84.76353,31.64549087)
\curveto(84.48228,31.51528253)(84.18280083,31.45017837)(83.8650925,31.45017837)
\curveto(83.37030083,31.45017837)(82.96144667,31.6194492)(82.63853,31.95799087)
\curveto(82.32082167,32.29653253)(82.1619675,32.74184503)(82.1619675,33.29392837)
\curveto(82.1619675,33.6428867)(82.2400925,33.94497003)(82.3963425,34.20017837)
\curveto(82.60988417,34.55434503)(82.97967583,34.88767837)(83.5057175,35.20017837)
\curveto(84.0369675,35.51267837)(84.91717583,35.8928867)(86.1463425,36.34080337)
\lineto(86.1463425,36.62205337)
\curveto(86.1463425,37.33559503)(86.03175917,37.82517837)(85.8025925,38.09080337)
\curveto(85.57863417,38.35642837)(85.25050917,38.48924087)(84.8182175,38.48924087)
\curveto(84.4900925,38.48924087)(84.22967583,38.4006992)(84.0369675,38.22361587)
\curveto(83.83905083,38.04653253)(83.7400925,37.84340753)(83.7400925,37.61424087)
\lineto(83.7557175,37.16111587)
\curveto(83.7557175,36.92153253)(83.6932175,36.7366367)(83.5682175,36.60642837)
\curveto(83.44842583,36.47622003)(83.28957167,36.41111587)(83.091655,36.41111587)
\curveto(82.89894667,36.41111587)(82.7400925,36.4788242)(82.6150925,36.61424087)
\curveto(82.49530083,36.74965753)(82.435405,36.93455337)(82.435405,37.16892837)
\curveto(82.435405,37.61684503)(82.66457167,38.02830337)(83.122905,38.40330337)
\curveto(83.58123833,38.77830337)(84.2244675,38.96580337)(85.0525925,38.96580337)
\curveto(85.68800917,38.96580337)(86.2088425,38.85903253)(86.6150925,38.64549087)
\curveto(86.92238417,38.48403253)(87.14894667,38.23142837)(87.29478,37.88767837)
\curveto(87.38853,37.66372003)(87.435405,37.2053867)(87.435405,36.51267837)
\lineto(87.435405,34.08299087)
\curveto(87.435405,33.4006992)(87.44842583,32.98142837)(87.4744675,32.82517837)
\curveto(87.50050917,32.6741367)(87.54217583,32.5725742)(87.5994675,32.52049087)
\curveto(87.6619675,32.46840753)(87.73228,32.44236587)(87.810405,32.44236587)
\curveto(87.89373833,32.44236587)(87.966655,32.46059503)(88.029155,32.49705337)
\curveto(88.13853,32.5647617)(88.3494675,32.75486587)(88.6619675,33.06736587)
\lineto(88.6619675,32.62986587)
\curveto(88.07863417,31.84861587)(87.5213425,31.45799087)(86.9900925,31.45799087)
\curveto(86.73488417,31.45799087)(86.53175917,31.54653253)(86.3807175,31.72361587)
\curveto(86.22967583,31.9006992)(86.15155083,32.20278253)(86.1463425,32.62986587)
\closepath
\moveto(86.1463425,33.13767837)
\lineto(86.1463425,35.86424087)
\curveto(85.35988417,35.55174087)(84.85207167,35.3303867)(84.622905,35.20017837)
\curveto(84.21144667,34.9710117)(83.91717583,34.73142837)(83.7400925,34.48142837)
\curveto(83.56300917,34.23142837)(83.4744675,33.95799087)(83.4744675,33.66111587)
\curveto(83.4744675,33.28611587)(83.58644667,32.97361587)(83.810405,32.72361587)
\curveto(84.03436333,32.4788242)(84.29217583,32.35642837)(84.5838425,32.35642837)
\curveto(84.97967583,32.35642837)(85.50050917,32.61684503)(86.1463425,33.13767837)
\closepath
}
}
{
\newrgbcolor{curcolor}{0 0 0}
\pscustom[linestyle=none,fillstyle=solid,fillcolor=curcolor]
{
\newpath
\moveto(91.2869675,38.96580337)
\lineto(91.2869675,37.35642837)
\curveto(91.88592583,38.42934503)(92.50050917,38.96580337)(93.1307175,38.96580337)
\curveto(93.41717583,38.96580337)(93.654155,38.8772617)(93.841655,38.70017837)
\curveto(94.029155,38.52830337)(94.122905,38.32778253)(94.122905,38.09861587)
\curveto(94.122905,37.89549087)(94.05519667,37.72361587)(93.91978,37.58299087)
\curveto(93.78436333,37.44236587)(93.622905,37.37205337)(93.435405,37.37205337)
\curveto(93.25311333,37.37205337)(93.04738417,37.46059503)(92.8182175,37.63767837)
\curveto(92.59425917,37.81997003)(92.4275925,37.91111587)(92.3182175,37.91111587)
\curveto(92.2244675,37.91111587)(92.122905,37.85903253)(92.01353,37.75486587)
\curveto(91.779155,37.5413242)(91.5369675,37.1897617)(91.2869675,36.70017837)
\lineto(91.2869675,33.27049087)
\curveto(91.2869675,32.87465753)(91.33644667,32.57517837)(91.435405,32.37205337)
\curveto(91.50311333,32.23142837)(91.622905,32.11424087)(91.79478,32.02049087)
\curveto(91.966655,31.92674087)(92.21405083,31.87986587)(92.5369675,31.87986587)
\lineto(92.5369675,31.59861587)
\lineto(88.872905,31.59861587)
\lineto(88.872905,31.87986587)
\curveto(89.23748833,31.87986587)(89.50832167,31.93715753)(89.685405,32.05174087)
\curveto(89.81561333,32.1350742)(89.90675917,32.2678867)(89.9588425,32.45017837)
\curveto(89.98488417,32.53872003)(89.997905,32.7913242)(89.997905,33.20799087)
\lineto(89.997905,35.98142837)
\curveto(89.997905,36.8147617)(89.97967583,37.30955337)(89.9432175,37.46580337)
\curveto(89.9119675,37.6272617)(89.8494675,37.7444492)(89.7557175,37.81736587)
\curveto(89.66717583,37.89028253)(89.55519667,37.92674087)(89.41978,37.92674087)
\curveto(89.25832167,37.92674087)(89.07603,37.88767837)(88.872905,37.80955337)
\lineto(88.79478,38.09080337)
\lineto(90.9588425,38.96580337)
\closepath
}
}
{
\newrgbcolor{curcolor}{0 0 0}
\pscustom[linestyle=none,fillstyle=solid,fillcolor=curcolor]
{
\newpath
\moveto(282.7214605,42.87025587)
\lineto(285.5964605,42.87025587)
\lineto(292.073023,34.92494337)
\lineto(292.073023,41.03431837)
\curveto(292.073023,41.68536003)(292.00010633,42.09161003)(291.854273,42.25306837)
\curveto(291.66156467,42.47181837)(291.35687717,42.58119337)(290.9402105,42.58119337)
\lineto(290.573023,42.58119337)
\lineto(290.573023,42.87025587)
\lineto(294.260523,42.87025587)
\lineto(294.260523,42.58119337)
\lineto(293.885523,42.58119337)
\curveto(293.43760633,42.58119337)(293.119898,42.4457767)(292.932398,42.17494337)
\curveto(292.81781467,42.0082767)(292.760523,41.62806837)(292.760523,41.03431837)
\lineto(292.760523,32.10463087)
\lineto(292.479273,32.10463087)
\lineto(285.494898,40.63588087)
\lineto(285.494898,34.11244337)
\curveto(285.494898,33.4614017)(285.5652105,33.0551517)(285.7058355,32.89369337)
\curveto(285.90375217,32.67494337)(286.20843967,32.56556837)(286.619898,32.56556837)
\lineto(286.994898,32.56556837)
\lineto(286.994898,32.27650587)
\lineto(283.307398,32.27650587)
\lineto(283.307398,32.56556837)
\lineto(283.6745855,32.56556837)
\curveto(284.1277105,32.56556837)(284.448023,32.70098503)(284.635523,32.97181837)
\curveto(284.75010633,33.13848503)(284.807398,33.51869337)(284.807398,34.11244337)
\lineto(284.807398,41.47963087)
\curveto(284.50010633,41.83900587)(284.26573133,42.07598503)(284.104273,42.19056837)
\curveto(283.948023,42.3051517)(283.71625217,42.41192253)(283.4089605,42.51088087)
\curveto(283.25791883,42.55775587)(283.02875217,42.58119337)(282.7214605,42.58119337)
\closepath
}
}
{
\newrgbcolor{curcolor}{0 0 0}
\pscustom[linestyle=none,fillstyle=solid,fillcolor=curcolor]
{
\newpath
\moveto(296.1902105,36.73744337)
\curveto(296.18500217,35.67494337)(296.44281467,34.84161003)(296.963648,34.23744337)
\curveto(297.48448133,33.6332767)(298.0964605,33.33119337)(298.7995855,33.33119337)
\curveto(299.2683355,33.33119337)(299.6745855,33.45879753)(300.0183355,33.71400587)
\curveto(300.36729383,33.97442253)(300.6589605,34.41713087)(300.8933355,35.04213087)
\lineto(301.135523,34.88588087)
\curveto(301.026148,34.1723392)(300.70843967,33.52129753)(300.182398,32.93275587)
\curveto(299.65635633,32.34942253)(298.99750217,32.05775587)(298.2058355,32.05775587)
\curveto(297.3464605,32.05775587)(296.60948133,32.3910892)(295.994898,33.05775587)
\curveto(295.385523,33.72963087)(295.0808355,34.63067253)(295.0808355,35.76088087)
\curveto(295.0808355,36.9848392)(295.3933355,37.9379642)(296.0183355,38.62025587)
\curveto(296.64854383,39.30775587)(297.43760633,39.65150587)(298.385523,39.65150587)
\curveto(299.18760633,39.65150587)(299.8464605,39.38588087)(300.3620855,38.85463087)
\curveto(300.8777105,38.3285892)(301.135523,37.62286003)(301.135523,36.73744337)
\closepath
\moveto(296.1902105,37.19056837)
\lineto(299.5027105,37.19056837)
\curveto(299.47666883,37.6489017)(299.42198133,37.97181837)(299.338648,38.15931837)
\curveto(299.20843967,38.45098503)(299.01312717,38.6801517)(298.7527105,38.84681837)
\curveto(298.49750217,39.01348503)(298.229273,39.09681837)(297.948023,39.09681837)
\curveto(297.51573133,39.09681837)(297.1277105,38.92754753)(296.7839605,38.58900587)
\curveto(296.44541883,38.25567253)(296.24750217,37.7895267)(296.1902105,37.19056837)
\closepath
}
}
{
\newrgbcolor{curcolor}{0 0 0}
\pscustom[linestyle=none,fillstyle=solid,fillcolor=curcolor]
{
\newpath
\moveto(306.1433355,33.30775587)
\curveto(305.4089605,32.74004753)(304.948023,32.41192253)(304.760523,32.32338087)
\curveto(304.479273,32.19317253)(304.17979383,32.12806837)(303.8620855,32.12806837)
\curveto(303.36729383,32.12806837)(302.95843967,32.2973392)(302.635523,32.63588087)
\curveto(302.31781467,32.97442253)(302.1589605,33.41973503)(302.1589605,33.97181837)
\curveto(302.1589605,34.3207767)(302.2370855,34.62286003)(302.3933355,34.87806837)
\curveto(302.60687717,35.23223503)(302.97666883,35.56556837)(303.5027105,35.87806837)
\curveto(304.0339605,36.19056837)(304.91416883,36.5707767)(306.1433355,37.01869337)
\lineto(306.1433355,37.29994337)
\curveto(306.1433355,38.01348503)(306.02875217,38.50306837)(305.7995855,38.76869337)
\curveto(305.57562717,39.03431837)(305.24750217,39.16713087)(304.8152105,39.16713087)
\curveto(304.4870855,39.16713087)(304.22666883,39.0785892)(304.0339605,38.90150587)
\curveto(303.83604383,38.72442253)(303.7370855,38.52129753)(303.7370855,38.29213087)
\lineto(303.7527105,37.83900587)
\curveto(303.7527105,37.59942253)(303.6902105,37.4145267)(303.5652105,37.28431837)
\curveto(303.44541883,37.15411003)(303.28656467,37.08900587)(303.088648,37.08900587)
\curveto(302.89593967,37.08900587)(302.7370855,37.1567142)(302.6120855,37.29213087)
\curveto(302.49229383,37.42754753)(302.432398,37.61244337)(302.432398,37.84681837)
\curveto(302.432398,38.29473503)(302.66156467,38.70619337)(303.119898,39.08119337)
\curveto(303.57823133,39.45619337)(304.2214605,39.64369337)(305.0495855,39.64369337)
\curveto(305.68500217,39.64369337)(306.2058355,39.53692253)(306.6120855,39.32338087)
\curveto(306.91937717,39.16192253)(307.14593967,38.90931837)(307.291773,38.56556837)
\curveto(307.385523,38.34161003)(307.432398,37.8832767)(307.432398,37.19056837)
\lineto(307.432398,34.76088087)
\curveto(307.432398,34.0785892)(307.44541883,33.65931837)(307.4714605,33.50306837)
\curveto(307.49750217,33.3520267)(307.53916883,33.2504642)(307.5964605,33.19838087)
\curveto(307.6589605,33.14629753)(307.729273,33.12025587)(307.807398,33.12025587)
\curveto(307.89073133,33.12025587)(307.963648,33.13848503)(308.026148,33.17494337)
\curveto(308.135523,33.2426517)(308.3464605,33.43275587)(308.6589605,33.74525587)
\lineto(308.6589605,33.30775587)
\curveto(308.07562717,32.52650587)(307.5183355,32.13588087)(306.9870855,32.13588087)
\curveto(306.73187717,32.13588087)(306.52875217,32.22442253)(306.3777105,32.40150587)
\curveto(306.22666883,32.5785892)(306.14854383,32.88067253)(306.1433355,33.30775587)
\closepath
\moveto(306.1433355,33.81556837)
\lineto(306.1433355,36.54213087)
\curveto(305.35687717,36.22963087)(304.84906467,36.0082767)(304.619898,35.87806837)
\curveto(304.20843967,35.6489017)(303.91416883,35.40931837)(303.7370855,35.15931837)
\curveto(303.56000217,34.90931837)(303.4714605,34.63588087)(303.4714605,34.33900587)
\curveto(303.4714605,33.96400587)(303.58343967,33.65150587)(303.807398,33.40150587)
\curveto(304.03135633,33.1567142)(304.28916883,33.03431837)(304.5808355,33.03431837)
\curveto(304.97666883,33.03431837)(305.49750217,33.29473503)(306.1433355,33.81556837)
\closepath
}
}
{
\newrgbcolor{curcolor}{0 0 0}
\pscustom[linestyle=none,fillstyle=solid,fillcolor=curcolor]
{
\newpath
\moveto(311.2839605,39.64369337)
\lineto(311.2839605,38.03431837)
\curveto(311.88291883,39.10723503)(312.49750217,39.64369337)(313.1277105,39.64369337)
\curveto(313.41416883,39.64369337)(313.651148,39.5551517)(313.838648,39.37806837)
\curveto(314.026148,39.20619337)(314.119898,39.00567253)(314.119898,38.77650587)
\curveto(314.119898,38.57338087)(314.05218967,38.40150587)(313.916773,38.26088087)
\curveto(313.78135633,38.12025587)(313.619898,38.04994337)(313.432398,38.04994337)
\curveto(313.25010633,38.04994337)(313.04437717,38.13848503)(312.8152105,38.31556837)
\curveto(312.59125217,38.49786003)(312.4245855,38.58900587)(312.3152105,38.58900587)
\curveto(312.2214605,38.58900587)(312.119898,38.53692253)(312.010523,38.43275587)
\curveto(311.776148,38.2192142)(311.5339605,37.8676517)(311.2839605,37.37806837)
\lineto(311.2839605,33.94838087)
\curveto(311.2839605,33.55254753)(311.33343967,33.25306837)(311.432398,33.04994337)
\curveto(311.50010633,32.90931837)(311.619898,32.79213087)(311.791773,32.69838087)
\curveto(311.963648,32.60463087)(312.21104383,32.55775587)(312.5339605,32.55775587)
\lineto(312.5339605,32.27650587)
\lineto(308.869898,32.27650587)
\lineto(308.869898,32.55775587)
\curveto(309.23448133,32.55775587)(309.50531467,32.61504753)(309.682398,32.72963087)
\curveto(309.81260633,32.8129642)(309.90375217,32.9457767)(309.9558355,33.12806837)
\curveto(309.98187717,33.21661003)(309.994898,33.4692142)(309.994898,33.88588087)
\lineto(309.994898,36.65931837)
\curveto(309.994898,37.4926517)(309.97666883,37.98744337)(309.9402105,38.14369337)
\curveto(309.9089605,38.3051517)(309.8464605,38.4223392)(309.7527105,38.49525587)
\curveto(309.66416883,38.56817253)(309.55218967,38.60463087)(309.416773,38.60463087)
\curveto(309.25531467,38.60463087)(309.073023,38.56556837)(308.869898,38.48744337)
\lineto(308.791773,38.76869337)
\lineto(310.9558355,39.64369337)
\closepath
}
}
{
\newrgbcolor{curcolor}{0 0 0}
\pscustom[linestyle=none,fillstyle=solid,fillcolor=curcolor]
{
\newpath
\moveto(141.9969595,41.45855087)
\lineto(141.9969595,37.29448837)
\lineto(143.926647,37.29448837)
\curveto(144.36935533,37.29448837)(144.692272,37.39084253)(144.895397,37.58355087)
\curveto(145.10373033,37.78146753)(145.24175117,38.16948837)(145.3094595,38.74761337)
\lineto(145.598522,38.74761337)
\lineto(145.598522,35.16167587)
\lineto(145.3094595,35.16167587)
\curveto(145.30425117,35.5731342)(145.24956367,35.87521753)(145.145397,36.06792587)
\curveto(145.04643867,36.2606342)(144.90581367,36.40386337)(144.723522,36.49761337)
\curveto(144.54643867,36.5965717)(144.28081367,36.64605087)(143.926647,36.64605087)
\lineto(141.9969595,36.64605087)
\lineto(141.9969595,33.31792587)
\curveto(141.9969595,32.78146753)(142.03081367,32.42730087)(142.098522,32.25542587)
\curveto(142.15060533,32.12521753)(142.25998033,32.01323837)(142.426647,31.91948837)
\curveto(142.65581367,31.79448837)(142.895397,31.73198837)(143.145397,31.73198837)
\lineto(143.5282095,31.73198837)
\lineto(143.5282095,31.44292587)
\lineto(138.9813345,31.44292587)
\lineto(138.9813345,31.73198837)
\lineto(139.3563345,31.73198837)
\curveto(139.7938345,31.73198837)(140.11154283,31.85959253)(140.3094595,32.11480087)
\curveto(140.4344595,32.28146753)(140.4969595,32.6825092)(140.4969595,33.31792587)
\lineto(140.4969595,40.16167587)
\curveto(140.4969595,40.6981342)(140.46310533,41.05230087)(140.395397,41.22417587)
\curveto(140.34331367,41.3543842)(140.23654283,41.46636337)(140.0750845,41.56011337)
\curveto(139.85112617,41.68511337)(139.61154283,41.74761337)(139.3563345,41.74761337)
\lineto(138.9813345,41.74761337)
\lineto(138.9813345,42.03667587)
\lineto(146.864147,42.03667587)
\lineto(146.9657095,39.70855087)
\lineto(146.692272,39.70855087)
\curveto(146.55685533,40.20334253)(146.39800117,40.5653217)(146.2157095,40.79448837)
\curveto(146.03862617,41.02886337)(145.817272,41.1981342)(145.551647,41.30230087)
\curveto(145.29123033,41.40646753)(144.88498033,41.45855087)(144.332897,41.45855087)
\closepath
}
}
{
\newrgbcolor{curcolor}{0 0 0}
\pscustom[linestyle=none,fillstyle=solid,fillcolor=curcolor]
{
\newpath
\moveto(152.176647,32.47417587)
\curveto(151.442272,31.90646753)(150.9813345,31.57834253)(150.7938345,31.48980087)
\curveto(150.5125845,31.35959253)(150.21310533,31.29448837)(149.895397,31.29448837)
\curveto(149.40060533,31.29448837)(148.99175117,31.4637592)(148.6688345,31.80230087)
\curveto(148.35112617,32.14084253)(148.192272,32.58615503)(148.192272,33.13823837)
\curveto(148.192272,33.4871967)(148.270397,33.78928003)(148.426647,34.04448837)
\curveto(148.64018867,34.39865503)(149.00998033,34.73198837)(149.536022,35.04448837)
\curveto(150.067272,35.35698837)(150.94748033,35.7371967)(152.176647,36.18511337)
\lineto(152.176647,36.46636337)
\curveto(152.176647,37.17990503)(152.06206367,37.66948837)(151.832897,37.93511337)
\curveto(151.60893867,38.20073837)(151.28081367,38.33355087)(150.848522,38.33355087)
\curveto(150.520397,38.33355087)(150.25998033,38.2450092)(150.067272,38.06792587)
\curveto(149.86935533,37.89084253)(149.770397,37.68771753)(149.770397,37.45855087)
\lineto(149.786022,37.00542587)
\curveto(149.786022,36.76584253)(149.723522,36.5809467)(149.598522,36.45073837)
\curveto(149.47873033,36.32053003)(149.31987617,36.25542587)(149.1219595,36.25542587)
\curveto(148.92925117,36.25542587)(148.770397,36.3231342)(148.645397,36.45855087)
\curveto(148.52560533,36.59396753)(148.4657095,36.77886337)(148.4657095,37.01323837)
\curveto(148.4657095,37.46115503)(148.69487617,37.87261337)(149.1532095,38.24761337)
\curveto(149.61154283,38.62261337)(150.254772,38.81011337)(151.082897,38.81011337)
\curveto(151.71831367,38.81011337)(152.239147,38.70334253)(152.645397,38.48980087)
\curveto(152.95268867,38.32834253)(153.17925117,38.07573837)(153.3250845,37.73198837)
\curveto(153.4188345,37.50803003)(153.4657095,37.0496967)(153.4657095,36.35698837)
\lineto(153.4657095,33.92730087)
\curveto(153.4657095,33.2450092)(153.47873033,32.82573837)(153.504772,32.66948837)
\curveto(153.53081367,32.5184467)(153.57248033,32.4168842)(153.629772,32.36480087)
\curveto(153.692272,32.31271753)(153.7625845,32.28667587)(153.8407095,32.28667587)
\curveto(153.92404283,32.28667587)(153.9969595,32.30490503)(154.0594595,32.34136337)
\curveto(154.1688345,32.4090717)(154.379772,32.59917587)(154.692272,32.91167587)
\lineto(154.692272,32.47417587)
\curveto(154.10893867,31.69292587)(153.551647,31.30230087)(153.020397,31.30230087)
\curveto(152.76518867,31.30230087)(152.56206367,31.39084253)(152.411022,31.56792587)
\curveto(152.25998033,31.7450092)(152.18185533,32.04709253)(152.176647,32.47417587)
\closepath
\moveto(152.176647,32.98198837)
\lineto(152.176647,35.70855087)
\curveto(151.39018867,35.39605087)(150.88237617,35.1746967)(150.6532095,35.04448837)
\curveto(150.24175117,34.8153217)(149.94748033,34.57573837)(149.770397,34.32573837)
\curveto(149.59331367,34.07573837)(149.504772,33.80230087)(149.504772,33.50542587)
\curveto(149.504772,33.13042587)(149.61675117,32.81792587)(149.8407095,32.56792587)
\curveto(150.06466783,32.3231342)(150.32248033,32.20073837)(150.614147,32.20073837)
\curveto(151.00998033,32.20073837)(151.53081367,32.46115503)(152.176647,32.98198837)
\closepath
}
}
{
\newrgbcolor{curcolor}{0 0 0}
\pscustom[linestyle=none,fillstyle=solid,fillcolor=curcolor]
{
\newpath
\moveto(157.317272,38.81011337)
\lineto(157.317272,37.20073837)
\curveto(157.91623033,38.27365503)(158.53081367,38.81011337)(159.161022,38.81011337)
\curveto(159.44748033,38.81011337)(159.6844595,38.7215717)(159.8719595,38.54448837)
\curveto(160.0594595,38.37261337)(160.1532095,38.17209253)(160.1532095,37.94292587)
\curveto(160.1532095,37.73980087)(160.08550117,37.56792587)(159.9500845,37.42730087)
\curveto(159.81466783,37.28667587)(159.6532095,37.21636337)(159.4657095,37.21636337)
\curveto(159.28341783,37.21636337)(159.07768867,37.30490503)(158.848522,37.48198837)
\curveto(158.62456367,37.66428003)(158.457897,37.75542587)(158.348522,37.75542587)
\curveto(158.254772,37.75542587)(158.1532095,37.70334253)(158.0438345,37.59917587)
\curveto(157.8094595,37.3856342)(157.567272,37.0340717)(157.317272,36.54448837)
\lineto(157.317272,33.11480087)
\curveto(157.317272,32.71896753)(157.36675117,32.41948837)(157.4657095,32.21636337)
\curveto(157.53341783,32.07573837)(157.6532095,31.95855087)(157.8250845,31.86480087)
\curveto(157.9969595,31.77105087)(158.24435533,31.72417587)(158.567272,31.72417587)
\lineto(158.567272,31.44292587)
\lineto(154.9032095,31.44292587)
\lineto(154.9032095,31.72417587)
\curveto(155.26779283,31.72417587)(155.53862617,31.78146753)(155.7157095,31.89605087)
\curveto(155.84591783,31.9793842)(155.93706367,32.1121967)(155.989147,32.29448837)
\curveto(156.01518867,32.38303003)(156.0282095,32.6356342)(156.0282095,33.05230087)
\lineto(156.0282095,35.82573837)
\curveto(156.0282095,36.6590717)(156.00998033,37.15386337)(155.973522,37.31011337)
\curveto(155.942272,37.4715717)(155.879772,37.5887592)(155.786022,37.66167587)
\curveto(155.69748033,37.73459253)(155.58550117,37.77105087)(155.4500845,37.77105087)
\curveto(155.28862617,37.77105087)(155.1063345,37.73198837)(154.9032095,37.65386337)
\lineto(154.8250845,37.93511337)
\lineto(156.989147,38.81011337)
\closepath
}
}
{
\newrgbcolor{curcolor}{0 0 0}
\pscustom[linestyle=none,fillstyle=solid,fillcolor=curcolor]
{
\newpath
\moveto(360.0372925,41.00662087)
\lineto(360.0372925,36.84255837)
\lineto(361.96698,36.84255837)
\curveto(362.40968833,36.84255837)(362.732605,36.93891253)(362.93573,37.13162087)
\curveto(363.14406333,37.32953753)(363.28208417,37.71755837)(363.3497925,38.29568337)
\lineto(363.638855,38.29568337)
\lineto(363.638855,34.70974587)
\lineto(363.3497925,34.70974587)
\curveto(363.34458417,35.1212042)(363.28989667,35.42328753)(363.18573,35.61599587)
\curveto(363.08677167,35.8087042)(362.94614667,35.95193337)(362.763855,36.04568337)
\curveto(362.58677167,36.1446417)(362.32114667,36.19412087)(361.96698,36.19412087)
\lineto(360.0372925,36.19412087)
\lineto(360.0372925,32.86599587)
\curveto(360.0372925,32.32953753)(360.07114667,31.97537087)(360.138855,31.80349587)
\curveto(360.19093833,31.67328753)(360.30031333,31.56130837)(360.46698,31.46755837)
\curveto(360.69614667,31.34255837)(360.93573,31.28005837)(361.18573,31.28005837)
\lineto(361.5685425,31.28005837)
\lineto(361.5685425,30.99099587)
\lineto(357.0216675,30.99099587)
\lineto(357.0216675,31.28005837)
\lineto(357.3966675,31.28005837)
\curveto(357.8341675,31.28005837)(358.15187583,31.40766253)(358.3497925,31.66287087)
\curveto(358.4747925,31.82953753)(358.5372925,32.2305792)(358.5372925,32.86599587)
\lineto(358.5372925,39.70974587)
\curveto(358.5372925,40.2462042)(358.50343833,40.60037087)(358.43573,40.77224587)
\curveto(358.38364667,40.9024542)(358.27687583,41.01443337)(358.1154175,41.10818337)
\curveto(357.89145917,41.23318337)(357.65187583,41.29568337)(357.3966675,41.29568337)
\lineto(357.0216675,41.29568337)
\lineto(357.0216675,41.58474587)
\lineto(364.90448,41.58474587)
\lineto(365.0060425,39.25662087)
\lineto(364.732605,39.25662087)
\curveto(364.59718833,39.75141253)(364.43833417,40.1133917)(364.2560425,40.34255837)
\curveto(364.07895917,40.57693337)(363.857605,40.7462042)(363.59198,40.85037087)
\curveto(363.33156333,40.95453753)(362.92531333,41.00662087)(362.37323,41.00662087)
\closepath
}
}
{
\newrgbcolor{curcolor}{0 0 0}
\pscustom[linestyle=none,fillstyle=solid,fillcolor=curcolor]
{
\newpath
\moveto(370.21698,32.02224587)
\curveto(369.482605,31.45453753)(369.0216675,31.12641253)(368.8341675,31.03787087)
\curveto(368.5529175,30.90766253)(368.25343833,30.84255837)(367.93573,30.84255837)
\curveto(367.44093833,30.84255837)(367.03208417,31.0118292)(366.7091675,31.35037087)
\curveto(366.39145917,31.68891253)(366.232605,32.13422503)(366.232605,32.68630837)
\curveto(366.232605,33.0352667)(366.31073,33.33735003)(366.46698,33.59255837)
\curveto(366.68052167,33.94672503)(367.05031333,34.28005837)(367.576355,34.59255837)
\curveto(368.107605,34.90505837)(368.98781333,35.2852667)(370.21698,35.73318337)
\lineto(370.21698,36.01443337)
\curveto(370.21698,36.72797503)(370.10239667,37.21755837)(369.87323,37.48318337)
\curveto(369.64927167,37.74880837)(369.32114667,37.88162087)(368.888855,37.88162087)
\curveto(368.56073,37.88162087)(368.30031333,37.7930792)(368.107605,37.61599587)
\curveto(367.90968833,37.43891253)(367.81073,37.23578753)(367.81073,37.00662087)
\lineto(367.826355,36.55349587)
\curveto(367.826355,36.31391253)(367.763855,36.1290167)(367.638855,35.99880837)
\curveto(367.51906333,35.86860003)(367.36020917,35.80349587)(367.1622925,35.80349587)
\curveto(366.96958417,35.80349587)(366.81073,35.8712042)(366.68573,36.00662087)
\curveto(366.56593833,36.14203753)(366.5060425,36.32693337)(366.5060425,36.56130837)
\curveto(366.5060425,37.00922503)(366.73520917,37.42068337)(367.1935425,37.79568337)
\curveto(367.65187583,38.17068337)(368.295105,38.35818337)(369.12323,38.35818337)
\curveto(369.75864667,38.35818337)(370.27948,38.25141253)(370.68573,38.03787087)
\curveto(370.99302167,37.87641253)(371.21958417,37.62380837)(371.3654175,37.28005837)
\curveto(371.4591675,37.05610003)(371.5060425,36.5977667)(371.5060425,35.90505837)
\lineto(371.5060425,33.47537087)
\curveto(371.5060425,32.7930792)(371.51906333,32.37380837)(371.545105,32.21755837)
\curveto(371.57114667,32.0665167)(371.61281333,31.9649542)(371.670105,31.91287087)
\curveto(371.732605,31.86078753)(371.8029175,31.83474587)(371.8810425,31.83474587)
\curveto(371.96437583,31.83474587)(372.0372925,31.85297503)(372.0997925,31.88943337)
\curveto(372.2091675,31.9571417)(372.420105,32.14724587)(372.732605,32.45974587)
\lineto(372.732605,32.02224587)
\curveto(372.14927167,31.24099587)(371.59198,30.85037087)(371.06073,30.85037087)
\curveto(370.80552167,30.85037087)(370.60239667,30.93891253)(370.451355,31.11599587)
\curveto(370.30031333,31.2930792)(370.22218833,31.59516253)(370.21698,32.02224587)
\closepath
\moveto(370.21698,32.53005837)
\lineto(370.21698,35.25662087)
\curveto(369.43052167,34.94412087)(368.92270917,34.7227667)(368.6935425,34.59255837)
\curveto(368.28208417,34.3633917)(367.98781333,34.12380837)(367.81073,33.87380837)
\curveto(367.63364667,33.62380837)(367.545105,33.35037087)(367.545105,33.05349587)
\curveto(367.545105,32.67849587)(367.65708417,32.36599587)(367.8810425,32.11599587)
\curveto(368.10500083,31.8712042)(368.36281333,31.74880837)(368.65448,31.74880837)
\curveto(369.05031333,31.74880837)(369.57114667,32.00922503)(370.21698,32.53005837)
\closepath
}
}
{
\newrgbcolor{curcolor}{0 0 0}
\pscustom[linestyle=none,fillstyle=solid,fillcolor=curcolor]
{
\newpath
\moveto(375.357605,38.35818337)
\lineto(375.357605,36.74880837)
\curveto(375.95656333,37.82172503)(376.57114667,38.35818337)(377.201355,38.35818337)
\curveto(377.48781333,38.35818337)(377.7247925,38.2696417)(377.9122925,38.09255837)
\curveto(378.0997925,37.92068337)(378.1935425,37.72016253)(378.1935425,37.49099587)
\curveto(378.1935425,37.28787087)(378.12583417,37.11599587)(377.9904175,36.97537087)
\curveto(377.85500083,36.83474587)(377.6935425,36.76443337)(377.5060425,36.76443337)
\curveto(377.32375083,36.76443337)(377.11802167,36.85297503)(376.888855,37.03005837)
\curveto(376.66489667,37.21235003)(376.49823,37.30349587)(376.388855,37.30349587)
\curveto(376.295105,37.30349587)(376.1935425,37.25141253)(376.0841675,37.14724587)
\curveto(375.8497925,36.9337042)(375.607605,36.5821417)(375.357605,36.09255837)
\lineto(375.357605,32.66287087)
\curveto(375.357605,32.26703753)(375.40708417,31.96755837)(375.5060425,31.76443337)
\curveto(375.57375083,31.62380837)(375.6935425,31.50662087)(375.8654175,31.41287087)
\curveto(376.0372925,31.31912087)(376.28468833,31.27224587)(376.607605,31.27224587)
\lineto(376.607605,30.99099587)
\lineto(372.9435425,30.99099587)
\lineto(372.9435425,31.27224587)
\curveto(373.30812583,31.27224587)(373.57895917,31.32953753)(373.7560425,31.44412087)
\curveto(373.88625083,31.5274542)(373.97739667,31.6602667)(374.02948,31.84255837)
\curveto(374.05552167,31.93110003)(374.0685425,32.1837042)(374.0685425,32.60037087)
\lineto(374.0685425,35.37380837)
\curveto(374.0685425,36.2071417)(374.05031333,36.70193337)(374.013855,36.85818337)
\curveto(373.982605,37.0196417)(373.920105,37.1368292)(373.826355,37.20974587)
\curveto(373.73781333,37.28266253)(373.62583417,37.31912087)(373.4904175,37.31912087)
\curveto(373.32895917,37.31912087)(373.1466675,37.28005837)(372.9435425,37.20193337)
\lineto(372.8654175,37.48318337)
\lineto(375.02948,38.35818337)
\closepath
}
}
{
\newrgbcolor{curcolor}{0 0 0}
\pscustom[linestyle=none,fillstyle=solid,fillcolor=curcolor]
{
\newpath
\moveto(26.634515,3.26764337)
\lineto(26.634515,2.97858087)
\curveto(25.84805667,3.3744142)(25.19180667,3.83795587)(24.665765,4.36920587)
\curveto(23.915765,5.1244142)(23.33764,6.0150392)(22.93139,7.04108087)
\curveto(22.52514,8.06712253)(22.322015,9.1322267)(22.322015,10.23639337)
\curveto(22.322015,11.8509767)(22.7204525,13.32233087)(23.5173275,14.65045587)
\curveto(24.3142025,15.9837892)(25.353265,16.9369142)(26.634515,17.50983087)
\lineto(26.634515,17.18170587)
\curveto(25.99389,16.8275392)(25.46784833,16.3431642)(25.05639,15.72858087)
\curveto(24.64493167,15.11399753)(24.33764,14.3353517)(24.134515,13.39264337)
\curveto(23.93139,12.44993503)(23.8298275,11.46556003)(23.8298275,10.43951837)
\curveto(23.8298275,9.32493503)(23.915765,8.3119142)(24.08764,7.40045587)
\curveto(24.22305667,6.68170587)(24.38711917,6.10618503)(24.5798275,5.67389337)
\curveto(24.77253583,5.23639337)(25.03034833,4.81712253)(25.353265,4.41608087)
\curveto(25.68139,4.0150392)(26.10847333,3.6322267)(26.634515,3.26764337)
\closepath
}
}
{
\newrgbcolor{curcolor}{0 0 0}
\pscustom[linestyle=none,fillstyle=solid,fillcolor=curcolor]
{
\newpath
\moveto(31.5485775,7.43170587)
\curveto(30.8142025,6.86399753)(30.353265,6.53587253)(30.165765,6.44733087)
\curveto(29.884515,6.31712253)(29.58503583,6.25201837)(29.2673275,6.25201837)
\curveto(28.77253583,6.25201837)(28.36368167,6.4212892)(28.040765,6.75983087)
\curveto(27.72305667,7.09837253)(27.5642025,7.54368503)(27.5642025,8.09576837)
\curveto(27.5642025,8.4447267)(27.6423275,8.74681003)(27.7985775,9.00201837)
\curveto(28.01211917,9.35618503)(28.38191083,9.68951837)(28.9079525,10.00201837)
\curveto(29.4392025,10.31451837)(30.31941083,10.6947267)(31.5485775,11.14264337)
\lineto(31.5485775,11.42389337)
\curveto(31.5485775,12.13743503)(31.43399417,12.62701837)(31.2048275,12.89264337)
\curveto(30.98086917,13.15826837)(30.65274417,13.29108087)(30.2204525,13.29108087)
\curveto(29.8923275,13.29108087)(29.63191083,13.2025392)(29.4392025,13.02545587)
\curveto(29.24128583,12.84837253)(29.1423275,12.64524753)(29.1423275,12.41608087)
\lineto(29.1579525,11.96295587)
\curveto(29.1579525,11.72337253)(29.0954525,11.5384767)(28.9704525,11.40826837)
\curveto(28.85066083,11.27806003)(28.69180667,11.21295587)(28.49389,11.21295587)
\curveto(28.30118167,11.21295587)(28.1423275,11.2806642)(28.0173275,11.41608087)
\curveto(27.89753583,11.55149753)(27.83764,11.73639337)(27.83764,11.97076837)
\curveto(27.83764,12.41868503)(28.06680667,12.83014337)(28.52514,13.20514337)
\curveto(28.98347333,13.58014337)(29.6267025,13.76764337)(30.4548275,13.76764337)
\curveto(31.09024417,13.76764337)(31.6110775,13.66087253)(32.0173275,13.44733087)
\curveto(32.32461917,13.28587253)(32.55118167,13.03326837)(32.697015,12.68951837)
\curveto(32.790765,12.46556003)(32.83764,12.0072267)(32.83764,11.31451837)
\lineto(32.83764,8.88483087)
\curveto(32.83764,8.2025392)(32.85066083,7.78326837)(32.8767025,7.62701837)
\curveto(32.90274417,7.4759767)(32.94441083,7.3744142)(33.0017025,7.32233087)
\curveto(33.0642025,7.27024753)(33.134515,7.24420587)(33.21264,7.24420587)
\curveto(33.29597333,7.24420587)(33.36889,7.26243503)(33.43139,7.29889337)
\curveto(33.540765,7.3666017)(33.7517025,7.55670587)(34.0642025,7.86920587)
\lineto(34.0642025,7.43170587)
\curveto(33.48086917,6.65045587)(32.9235775,6.25983087)(32.3923275,6.25983087)
\curveto(32.13711917,6.25983087)(31.93399417,6.34837253)(31.7829525,6.52545587)
\curveto(31.63191083,6.7025392)(31.55378583,7.00462253)(31.5485775,7.43170587)
\closepath
\moveto(31.5485775,7.93951837)
\lineto(31.5485775,10.66608087)
\curveto(30.76211917,10.35358087)(30.25430667,10.1322267)(30.02514,10.00201837)
\curveto(29.61368167,9.7728517)(29.31941083,9.53326837)(29.1423275,9.28326837)
\curveto(28.96524417,9.03326837)(28.8767025,8.75983087)(28.8767025,8.46295587)
\curveto(28.8767025,8.08795587)(28.98868167,7.77545587)(29.21264,7.52545587)
\curveto(29.43659833,7.2806642)(29.69441083,7.15826837)(29.9860775,7.15826837)
\curveto(30.38191083,7.15826837)(30.90274417,7.41868503)(31.5485775,7.93951837)
\closepath
}
}
{
\newrgbcolor{curcolor}{0 0 0}
\pscustom[linestyle=none,fillstyle=solid,fillcolor=curcolor]
{
\newpath
\moveto(34.4548275,17.18170587)
\lineto(34.4548275,17.50983087)
\curveto(35.24649417,17.11920587)(35.90534833,16.65826837)(36.43139,16.12701837)
\curveto(37.17618167,15.3666017)(37.7517025,14.47337253)(38.1579525,13.44733087)
\curveto(38.5642025,12.42649753)(38.7673275,11.36139337)(38.7673275,10.25201837)
\curveto(38.7673275,8.63743503)(38.36889,7.16608087)(37.572015,5.83795587)
\curveto(36.78034833,4.50462253)(35.74128583,3.55149753)(34.4548275,2.97858087)
\lineto(34.4548275,3.26764337)
\curveto(35.0954525,3.62701837)(35.62149417,4.11399753)(36.0329525,4.72858087)
\curveto(36.44961917,5.33795587)(36.75691083,6.11399753)(36.9548275,7.05670587)
\curveto(37.1579525,8.00462253)(37.259515,8.9916017)(37.259515,10.01764337)
\curveto(37.259515,11.12701837)(37.1735775,12.1400392)(37.0017025,13.05670587)
\curveto(36.87149417,13.77545587)(36.70743167,14.3509767)(36.509515,14.78326837)
\curveto(36.31680667,15.21556003)(36.05899417,15.6322267)(35.7360775,16.03326837)
\curveto(35.41316083,16.43431003)(34.9860775,16.81712253)(34.4548275,17.18170587)
\closepath
}
}
{
\newrgbcolor{curcolor}{0 0 0}
\pscustom[linestyle=none,fillstyle=solid,fillcolor=curcolor]
{
\newpath
\moveto(241.578363,2.56261337)
\lineto(241.578363,2.27355087)
\curveto(240.79190467,2.6693842)(240.13565467,3.13292587)(239.609613,3.66417587)
\curveto(238.859613,4.4193842)(238.281488,5.3100092)(237.875238,6.33605087)
\curveto(237.468988,7.36209253)(237.265863,8.4271967)(237.265863,9.53136337)
\curveto(237.265863,11.1459467)(237.6643005,12.61730087)(238.4611755,13.94542587)
\curveto(239.2580505,15.2787592)(240.297113,16.2318842)(241.578363,16.80480087)
\lineto(241.578363,16.47667587)
\curveto(240.937738,16.1225092)(240.41169633,15.6381342)(240.000238,15.02355087)
\curveto(239.58877967,14.40896753)(239.281488,13.6303217)(239.078363,12.68761337)
\curveto(238.875238,11.74490503)(238.7736755,10.76053003)(238.7736755,9.73448837)
\curveto(238.7736755,8.61990503)(238.859613,7.6068842)(239.031488,6.69542587)
\curveto(239.16690467,5.97667587)(239.33096717,5.40115503)(239.5236755,4.96886337)
\curveto(239.71638383,4.53136337)(239.97419633,4.11209253)(240.297113,3.71105087)
\curveto(240.625238,3.3100092)(241.05232133,2.9271967)(241.578363,2.56261337)
\closepath
}
}
{
\newrgbcolor{curcolor}{0 0 0}
\pscustom[linestyle=none,fillstyle=solid,fillcolor=curcolor]
{
\newpath
\moveto(244.3986755,11.61730087)
\curveto(245.09138383,12.58084253)(245.83877967,13.06261337)(246.640863,13.06261337)
\curveto(247.375238,13.06261337)(248.015863,12.7475092)(248.562738,12.11730087)
\curveto(249.109613,11.49230087)(249.3830505,10.63553003)(249.3830505,9.54698837)
\curveto(249.3830505,8.27615503)(248.9611755,7.25271753)(248.1174255,6.47667587)
\curveto(247.39346717,5.8100092)(246.5861755,5.47667587)(245.6955505,5.47667587)
\curveto(245.27888383,5.47667587)(244.85440467,5.5521967)(244.422113,5.70323837)
\curveto(243.99502967,5.85428003)(243.55752967,6.08084253)(243.109613,6.38292587)
\lineto(243.109613,13.79698837)
\curveto(243.109613,14.60948837)(243.08877967,15.10948837)(243.047113,15.29698837)
\curveto(243.01065467,15.48448837)(242.95075883,15.61209253)(242.8674255,15.67980087)
\curveto(242.78409217,15.7475092)(242.6799255,15.78136337)(242.5549255,15.78136337)
\curveto(242.40909217,15.78136337)(242.2268005,15.7396967)(242.0080505,15.65636337)
\lineto(241.8986755,15.92980087)
\lineto(244.047113,16.80480087)
\lineto(244.3986755,16.80480087)
\closepath
\moveto(244.3986755,11.11730087)
\lineto(244.3986755,6.83605087)
\curveto(244.6643005,6.5756342)(244.937738,6.37771753)(245.218988,6.24230087)
\curveto(245.50544633,6.11209253)(245.797113,6.04698837)(246.093988,6.04698837)
\curveto(246.56794633,6.04698837)(247.0080505,6.30740503)(247.4143005,6.82823837)
\curveto(247.82575883,7.3490717)(248.031488,8.1068842)(248.031488,9.10167587)
\curveto(248.031488,10.01834253)(247.82575883,10.72146753)(247.4143005,11.21105087)
\curveto(247.0080505,11.70584253)(246.54450883,11.95323837)(246.0236755,11.95323837)
\curveto(245.74763383,11.95323837)(245.47159217,11.88292587)(245.1955505,11.74230087)
\curveto(244.98721717,11.6381342)(244.72159217,11.42980087)(244.3986755,11.11730087)
\closepath
}
}
{
\newrgbcolor{curcolor}{0 0 0}
\pscustom[linestyle=none,fillstyle=solid,fillcolor=curcolor]
{
\newpath
\moveto(250.297113,16.47667587)
\lineto(250.297113,16.80480087)
\curveto(251.08877967,16.41417587)(251.74763383,15.95323837)(252.2736755,15.42198837)
\curveto(253.01846717,14.6615717)(253.593988,13.76834253)(254.000238,12.74230087)
\curveto(254.406488,11.72146753)(254.609613,10.65636337)(254.609613,9.54698837)
\curveto(254.609613,7.93240503)(254.2111755,6.46105087)(253.4143005,5.13292587)
\curveto(252.62263383,3.79959253)(251.58357133,2.84646753)(250.297113,2.27355087)
\lineto(250.297113,2.56261337)
\curveto(250.937738,2.92198837)(251.46377967,3.40896753)(251.875238,4.02355087)
\curveto(252.29190467,4.63292587)(252.59919633,5.40896753)(252.797113,6.35167587)
\curveto(253.000238,7.29959253)(253.1018005,8.2865717)(253.1018005,9.31261337)
\curveto(253.1018005,10.42198837)(253.015863,11.4350092)(252.843988,12.35167587)
\curveto(252.71377967,13.07042587)(252.54971717,13.6459467)(252.3518005,14.07823837)
\curveto(252.15909217,14.51053003)(251.90127967,14.9271967)(251.578363,15.32823837)
\curveto(251.25544633,15.72928003)(250.828363,16.11209253)(250.297113,16.47667587)
\closepath
}
}
\end{pspicture}

    \caption{节点的两个孩子不需要都处理的两种情况,因为光线没有与之重合。
    (a)上方光线与划分平面的相交超出了光线的$t_{\max}$位置,
    因此没有进入更远的孩子。下方光线背对划分平面,这由负的$t_{\text{split}}$值表示。
    (b)光线在其$t_{\min}$值之前与平面相交,意味着不需要处理近处孩子。}
    \label{fig:4.19}
\end{figure}

下面的代码中第一个{\ttfamily if}测试与\reffig{4.19}(a)对应:
如果可以证明由于光线背对平面或因$t_{\text{split}}>t_{\max}$没与节点重合,
即光线没有与远处节点重合,则只有近处节点需要处理。
\reffig{4.19}(b)展示了第二个{\ttfamily if}测试中类似的情况:
如果光线没与之重合,则可能不需要处理近处节点。
否则,{\ttfamily else}语句负责两个孩子都需要处理的情况;
接着会处理近处节点,而远处节点列入列表{\ttfamily todo}。
\begin{lstlisting}
`\initcode{Advance to next child node, possibly enqueue other child}{=}`
if (tPlane > tMax || tPlane <= 0)
    node = firstChild;
else if (tPlane < tMin)
    node = secondChild;
else {
    `\refcode{Enqueue secondChild in todo list}{}`
    node = firstChild;
    tMax = tPlane;
}
\end{lstlisting}
\begin{lstlisting}
`\initcode{Enqueue secondChild in todo list}{=}`
todo[todoPos].`\refvar[KdToDo::node]{node}{}` = secondChild;
todo[todoPos].`\refvar[KdToDo::tMin]{tMin}{}` = tPlane;
todo[todoPos].`\refvar[KdToDo::tMax]{tMax}{}` = tMax;
++todoPos;
\end{lstlisting}

如果当前节点是叶子,则对叶子里的图元执行相交测试。
\begin{lstlisting}
`\initcode{Check for intersections inside leaf node}{=}`
int nPrimitives = node->`\refvar[KdAccelNode::nPrimitives]{nPrimitives}{}`();
if (nPrimitives == 1) {
    const std::shared_ptr<`\refvar{Primitive}{}`> &p = `\refvar[KdTreeAccel::primitives]{primitives}{}`[node->`\refvar{onePrimitive}{}`];
    `\refcode{Check one primitive inside leaf node}{}`
} else {
    for (int i = 0; i < nPrimitives; ++i) {
        int index = `\refvar{primitiveIndices}{}`[node->`\refvar{primitiveIndicesOffset}{}` + i];
        const std::shared_ptr<`\refvar{Primitive}{}`> &p = `\refvar[KdTreeAccel::primitives]{primitives}{}`[index];
        `\refcode{Check one primitive inside leaf node}{}`
    }
}
\end{lstlisting}

处理单个图元就是把相交请求传给图元的事。
\begin{lstlisting}
`\initcode{Check one primitive inside leaf node}{=}`
if (p->`\refvar[Primitive::Intersect]{Intersect}{}`(ray, isect)) 
    hit = true;
\end{lstlisting}

在叶子节点做完相交测试后,从数组{\ttfamily todo}加载下一个要处理的节点。
如果没有剩下更多节点了,则光线穿过该树没有命中任何东西。
\begin{lstlisting}
`\initcode{Grab next node to process from todo list}{=}`
if (todoPos > 0) {
    --todoPos;
    node = todo[todoPos].`\refvar[KdToDo::node]{node}{}`;
    tMin = todo[todoPos].`\refvar[KdToDo::tMin]{tMin}{}`;
    tMax = todo[todoPos].`\refvar[KdToDo::tMax]{tMax}{}`;
}
else
    break;
\end{lstlisting}

像\refvar{BVHAccel}{}那样,此处没有展示\refvar{KdTreeAccel}{}对于
阴影射线的特殊化相交方法。它和方法\refvar[KdTreeAccel::Intersect]{Intersect}{()}类似
但调用的是方法\refvar{Primitive::IntersectP}{()}且一旦
找到任何相交处就返回{\ttfamily true}而不担心是否找到了最近的那个。
\begin{lstlisting}
`\initcode{KdTreeAccel Public Methods}{=}`
bool `\initvar[KdTreeAccel::IntersectP]{IntersectP}{}`(const `\refvar{Ray}{}` &ray) const;
\end{lstlisting}