\section{扩展阅读}\label{sec:扩展阅读03}
《\citetitle*{10.5555/94788}》有对光线-形状相交算法的大量综述\citep{10.5555/94788}。
\citet{doi:10.1177/003754977101600104}讨论了光线-二次曲面相交,
\citet{Heckbert84themathematics}详细讨论了图形应用中的二次曲面数学,
引用了许多数学和其他领域的文献。
\citet{10.1145/964967.801136}描述了为多项式隐式定义的曲面自动化推导光线相交例程过程的系统;
他的系统以C源码发行来执行相交测试以及对给定方程描述的曲面的常规计算。
\citet{10.5555/93267.93276}表明了区间运算可用于开发算法来稳定计算与
多项式无法描述而因此更难精确计算相交处的隐式曲面的相交;
\citet{10.1111/j.1467-8659.2008.01189.x}最近做了这方面工作。
对区间运算的介绍详见\citet{moore1966interval}的书籍。

其他关于光线-形状相交值得注意的早期论文包括\citet{10.1145/800059.801137}关于
计算与旋转曲面以及程序生成的分形地形\sidenote{译者注:原文fractal terrains。}相交的工作。
\citet{10.1145/358523.358553}关于渲染程序化随机模型的论文
和\citet{10.1145/74333.74363}关于求取与分形\sidenote{译者注:原文fractal。}相交
的论文阐明了可用于光线追踪算法的大量形状表示。

\citet{10.1145/800064.801287}开发了第一个计算与参数块\sidenote{译者注:原文parametric patch。}相交的算法。
关于光线直接与块相交的更高效技术的后续工作包括\citet{722295}、
\citet{doi:10.1080/10867651.2000.10487519}
以及\citet{https://doi.org/10.1111/1467-8659.00535}的论文。
\citet{doi:10.1080/2151237X.2006.10129218}
\sidenote{译者注:原文标注年份可能有误,已修正。}提供了更新结果并包含对之前工作的额外引用。
\citet{doi:10.1080/10867651.2004.10504896}描述了一个计算与双线性块相交的高效算法,
\citet{10.1111/j.1467-8659.2011.01993.x}介绍了
从三角网格生成的光滑曲面与每个顶点法线直接相交的技术。

\citet{gray2017modern}\sidenote{译者注:此处引用文献改为了新版,章节可能无法对应。}写了
对微分几何学的精彩介绍;该书的14.3节介绍了外恩加滕公式。

\refsec{三角形网格}中的光线-三角形相交测试由\citet{Woop2013Watertight}开发。
另一广泛使用的光线-三角形相交算法详见\citet{doi:10.1080/10867651.1997.10487468}。
\citet{doi:10.1080/2151237X.2005.10129208}、\citet{shevtsov2007ray}描述了针对现代CPU架构
高度优化的光线-三角形相交例程并包含了对其他最近方法的大量参考。
\citet{4061543}介绍了开发快速光线-三角形相交例程的有趣方法:
他们实现了在数学上等价于光线-三角形测试的空间中执行搜索的程序,
自动生成各种软件实现,然后对其做基准测试。
最终,他们发现了比之前使用的更高效的光线-三角形例程。

\citet{phong1975improved}首次引入了插值每个顶点的着色法线
以得到来自多边形网格的光滑曲面外观的想法。

内存中三角网格的布局\sidenote{译者注:原文layout。}在很多情况下对性能有重大影响。
通常,如果3D空间中接近的三角形在内存中也接近,缓存命中率会更高,整个系统的性能会更好。
在内存中创建缓存友好网格布局的算法详见
\citet{10.1145/1186822.1073278}以及\citet{4015484}。

\refsec{曲线}曲线相交算法基于\citet{Nakamaru_raytracing}开发的方法。
更早计算光线与一般圆柱体相交的方法也用于渲染曲线,但低效得多\citep{10.1145/6116.6118,DeVoogt2000197}。
\citet{10.5555/501891}的书籍提供了对样条的精彩概述,
\refsec{曲线}用的开花方法由\citet{ramshaw1987blossoming}介绍。

渲染像头发和毛发那样细的几何体的一大挑战是细几何体可能需要许多像素采样以精确求解,反过来增加了渲染时间。
\citet{10.1145/1179849.1179904}描述了一个系统为了更高效渲染而预先
计算\keyindex{体素}{voxel}{}网格来表示头发和毛发,并在一个小空间区域内保存许多毛发的聚合信息。
最近,\citet{6684531}为毛发渲染描述了基于圆锥追踪的方法,它追踪细圆锥而不是光线。
反过来,计算圆锥的贡献时可以考虑所有与圆锥相交的曲线,这允许每个像素用少量圆锥做高质量渲染。

\citet{DOO1978356}以及\citet{CATMULL1978350}发明了细分曲面。
\citet{loop1987smooth}最初开发了Loop细分方法,尽管pbrt中的实现使用了
由\citet{10.1145/192161.192233}开发的改进的细分规则和沿边界边缘的切线。
细分曲面还有许多后续工作。
SIGGRAPH课程笔记提供了2000年最先进技术的优秀总结并有大量参考\citep{zorin2000subdivision}。
另见\citet{WARREN20021}\sidenote{译者注:原文少标注了一个作者,已修正。}关于此话题的书籍。
\citet{10.1111/1467-8659.t01-2-00703}描述了针对要测试相交的光线按需细化细分曲面的方法
(另见\citet{SCI:Ben2007a}的相关方法)。

细分曲面令人兴奋的进展是在曲面上任意点求值的能力\citep{10.1145/280814.280945}。
像本章那样的细分曲面实现经常相对低效,解引用指针和运用细分规则花费的时间一样多。
\citeauthor{10.1145/280814.280945}的方法避免了低效率。
\citet{10.1145/504502.504505}提出了一种改进实现的方法即
预先计算许多能让最终网格的计算更高效的量。
最近,\citet{10.1145/1572769.1572785}阐述了在数据并行吞吐的处理器上
细化\sidenote{译者注:原文tessellating。}细分曲面非常高效的方法。

\citet{doi:10.1137/1.9780898718027}关于浮点计算的书籍非常好;
它还开发了我们在\refsec{控制舍入误差}所用的$\gamma_n$记法。
其他关于该话题很好的文献还有\citet{10.5555/1096474}和\citet{10.1145/103162.103163}。
尽管我们已手动推导浮点误差边界,详见\citet{10.1145/1644001.1644003}的
\emph{Gappa}系统了解自动推导浮点计算前向误差边界的工具。

长时间以来错误自相交问题对于光线追踪从业者而言都是一个著名问题
\citep{10.5555/94788.94790,Amanatides1990:27}。
除了在射线端点处将其偏移一个$\epsilon$外,
已经提出的方法包括忽略与之前相交过的物体的相交、
细化算得的交点使其数值变得更加精确的
“根抛光”\sidenote{译者注:原文root polishing。}\citep{10.5555/94788.94790,536271};
以及使用更高精度浮点表示(例如用{\ttfamily double}代替{\ttfamily float})。

\citet{10.1145/74333.74364}以及\citet{4061542}基于
递归细分物体边界框、消除没有包含物体曲面的框并抛弃射线错开的框为数值稳定的相交开发了算法。
这些方法都比传统光线-物体相交算法以及\refsec{控制舍入误差}介绍的技术低效得多。

\citet{10.1145/73833.73857}引入了
考虑了浮点舍入误差的技术为计算几何推导稳定的基本运算,
\citet{Ize2013BVH}展示了怎样执行数值稳定的射线-边界框相交;
他的方法在\refsub{保守的光线-边界框相交}实现
(通过更仔细的推导,他证明了缩放因子$2\gamma_2$实际上可用于增加\refvar{tMax}{},
而不是我们这里推导的$2\gamma_3$)。
\citet{Wächter_2008}在他的毕业论文中讨论了自相交问题;
他提出从初始相交处重新计算交点(根抛光)并
沿法线将触发的射线偏移交点量级固定的一小部分。
本章实现的方法使用了他沿法线偏移射线端点的方法
但是基于算得的交点中出现的数值误差偏移量的保守边界
(事实证明,我们的边界一般比\citeauthor{Wächter_2008}的偏移量
更加紧致但仍然被证明是保守的)。
