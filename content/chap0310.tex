\section{扩展阅读}\label{sec:扩展阅读03}
《\citetitle*{10.5555/94788}》有对光线-形状相交算法的大量综述\citep{10.5555/94788}。
Goldstein和Nagel\parencite*{doi:10.1177/003754977101600104}讨论了光线-二次曲面相交,
\citet{Heckbert84themathematics}详细讨论了图形应用中的二次曲面数学,
引用了许多数学和其他领域的文献。
\citet{10.1145/964967.801136}描述了为多项式隐式定义的曲面自动化推导光线相交例程过程的系统;
他的系统以C源码发行来执行相交测试以及对给定方程描述的曲面的常规计算。
\citet{10.5555/93267.93276}表明了区间运算可用于开发算法来稳定计算与
多项式无法描述而因此更难精确计算相交处的隐式曲面的相交;
Knoll等\parencite*{10.1111/j.1467-8659.2008.01189.x}最近做了这方面工作。
对区间运算的介绍详见\citet{moore1966interval}的书籍。

其他关于光线-形状相交值得注意的早期论文包括\citet{10.1145/800059.801137}关于
计算与旋转曲面以及程序生成的分形地形\sidenote{译者注:原文fractal terrains。}相交的工作。
Fournier等\parencite*{10.1145/358523.358553}关于渲染程序化随机模型的论文
和Hart等\parencite*{10.1145/74334.74363}关于求取与分形\sidenote{译者注:原文fractal。}相交
的论文阐明了可用于光线追踪算法的大量形状表示。

\citet{10.1145/800064.801287}开发了第一个计算与参数块\sidenote{译者注:原文parametric patch。}相交的算法。
关于光线直接与块相交的更高效技术的后续工作包括\citet{722295}、
Martin等\parencite*{doi:10.1080/10867651.2000.10487519}
以及Roth等\parencite*{https://doi.org/10.1111/1467-8659.00535}的论文。
Benthin等\parencite*{doi:10.1080/2151237X.2006.10129218}
\sidenote{译者注:原文标注年份可能有误,已修正。}提供了更新结果并包含对之前工作的额外引用。
Ramsey等\parencite*{doi:10.1080/10867651.2004.10504896}描述了一个计算与双线性块相交的高效算法,
Ogaki和Tokuyoshi\parencite*{10.1111/j.1467-8659.2011.01993.x}介绍了
从三角网格生成的光滑曲面与每个顶点法线直接相交的技术。
