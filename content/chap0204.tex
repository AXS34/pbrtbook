\section{法线}\label{sec:法线}

\keyindex{曲面法线}{surface normal}{}(或简称\keyindex{法线}{normal}{})
是在某一位置垂直于曲面的向量。
它可定义为与曲面上某点相切的任意两个非平行向量的叉积。
尽管法线看似和向量相似,但两者有重要区别:
因为法线是根据其与特定曲面的关系定义的,
它们在某些情形下和向量截然不同,
尤其是在进行变换的时候。
\refsec{施加变换}会讨论这些区别。
\begin{lstlisting}
`\initcode{Normal Declarations}{=}\initnext{NormalDeclarations}`
template <typename T> class `\initvar{Normal3}{}` {
public:
    `\refcode{Normal3 Public Methods}{}`
    `\refcode{Normal3 Public Data}{}`
};
\end{lstlisting}
\begin{lstlisting}
`\refcode{Normal Declarations}{+=}\lastcode{NormalDeclarations}`
typedef `\refvar{Normal3}{}`<`\refvar{Float}{}`> `\initvar{Normal3f}{}`;
\end{lstlisting}
其中
\begin{lstlisting}
`\initcode{Normal3 Public Data}{=}`
T `\initvar[Normal3::x]{x}{}`, `\initvar[Normal3::y]{y}{}`, `\initvar[Normal3::z]{z}{}`;
\end{lstlisting}

\refvar{Normal3}{}和\refvar{Vector3}{}的实现非常相似。
如同向量那样,法线用{\ttfamily x}、{\ttfamily y}和{\ttfamily z}三个分量表示;
它们可以相加减计算新的法线;它们也可以被缩放和规范化。
然而,法线不能与点相加,不能取两法线的叉积。
要注意不幸的是,从术语层面讲,法线\emph{没有}必要是规范化的。

\refvar{Normal3}{}提供了额外的构造函数来从\refvar{Vector3}{}初始化\refvar{Normal3}{}。
因为\refvar{Normal3}{}
和\refvar{Vector3}{}有一些微妙不同,
我们想保证这种转换不会在我们不希望时发生,
所以这里再次用到C++的关键字{\ttfamily explicit}。
\refvar{Vector3}{}也提供了其他转换方式的构造函数。
因此,给定声明\refvar{Vector3f}{\ v;}和\refvar{Normal3f}{\ n;},
则赋值{\ttfamily n = v}是非法的,
所以需要显式地转换向量,即{\ttfamily n = \refvar{Normal3f}{}(v)}。
\begin{lstlisting}
`\initcode{Normal3 Public Methods}{=}`
explicit `\refvar{Normal3}{}`<T>(const `\refvar{Vector3}{}`<T> &v) : x(v.x), y(v.y), z(v.z) {
    `\refvar{Assert}{}`(!v.HasNaNs());
}
\end{lstlisting}
\begin{lstlisting}
`\refcode{Geometry Inline Functions}{+=}\lastnext{GeometryInlineFunctions}`
template <typename T> inline
`\refvar{Vector3}{}`<T>::`\refvar{Vector3}{}`(const `\refvar{Normal3}{}`<T> &n) : x(n.x), y(n.y), z(n.z) {
    `\refvar{Assert}{}`(!n.HasNaNs());
}
\end{lstlisting}

函数\refvar{Dot}{()}和\refvar{AbsDot}{()}也重载为
计算法线与向量各种可能的组合的点积。
这里本文不再介绍这些代码。
我们也不再介绍\refvar{Normal3}{}其他各种方法的实现,
因为它们和向量的一样。

一个要实现的新方法源自经常需要翻转曲面法线
使其和给定向量位于同一半球的事实——
例如常常需要曲面法线和出射光线位于同一半球。实用函数
\refvar{Faceforward}{()}封装了这步小型计算。
(pbrt也为\refvar{Vector3}{}和\refvar{Normal3}{}作为参数的
其他三种组合提供了该函数的变种。)
但是在使用其他实例时要谨慎:
例如当使用接收两个\refvar{Vector3}{}的版本时,
确保第一个参数是要被返回的那个(可能被翻转了)
而第二个是被测试的那个。
弄反这两个参数会给出意外结果。
\begin{lstlisting}
`\refcode{Geometry Inline Functions}{+=}\lastnext{GeometryInlineFunctions}`
template <typename T> inline `\refvar{Normal3}{}`<T>
`\initvar{Faceforward}{}`(const `\refvar{Normal3}{}`<T> &n, const `\refvar{Vector3}{}`<T> &v) {
    return (`\refvar{Dot}{}`(n, v) < 0.f) ? -n : n;
}
\end{lstlisting}