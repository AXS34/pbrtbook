\section{射线}\label{sec:射线}

\keyindex{射线}{ray}{}是由\keyindex{端点}{origin}{}和方向指定的半无限直线。
pbrt用\refvar{Point3f}{}作为端点、
\refvar{Vector3f}{}作为方向表示\refvar{Ray}{}。
我们只需要端点和方向是浮点的射线,
所以\refvar{Ray}{}没有像点、向量和法线那样被任意类型的模板类参数化。
射线记为$\bm r$;它有端点$\bm o$和方向$\bm d$,如\reffig{2.7}所示。
\begin{figure}[htbp]
    \centering%LaTeX with PSTricks extensions
%%Creator: Inkscape 1.0.1 (3bc2e813f5, 2020-09-07)
%%Please note this file requires PSTricks extensions
\psset{xunit=.5pt,yunit=.5pt,runit=.5pt}
\begin{pspicture}(300.82998657,62.58000183)
{
\newrgbcolor{curcolor}{0 0 0}
\pscustom[linestyle=none,fillstyle=solid,fillcolor=curcolor]
{
\newpath
\moveto(30.01000071,16.07000351)
\curveto(30.01000071,21.61181689)(23.31019913,24.38616294)(19.39202031,20.46798412)
\curveto(15.47384149,16.5498053)(18.24818753,9.85000372)(23.79000092,9.85000372)
\curveto(29.3318143,9.85000372)(32.10616034,16.5498053)(28.18798152,20.46798412)
\curveto(24.26980271,24.38616294)(17.57000113,21.61181689)(17.57000113,16.07000351)
\curveto(17.57000113,10.52819013)(24.26980271,7.75384408)(28.18798152,11.6720229)
\curveto(32.10616034,15.59020172)(29.3318143,22.2900033)(23.79000092,22.2900033)
\curveto(18.24818753,22.2900033)(15.47384149,15.59020172)(19.39202031,11.6720229)
\curveto(23.31019913,7.75384408)(30.01000071,10.52819013)(30.01000071,16.07000351)
\closepath
}
}
{
\newrgbcolor{curcolor}{0 0 0}
\pscustom[linewidth=1,linecolor=curcolor]
{
\newpath
\moveto(29.22999954,17.58000183)
\lineto(292.79000854,50.5700016)
}
}
{
\newrgbcolor{curcolor}{0 0 0}
\pscustom[linestyle=none,fillstyle=solid,fillcolor=curcolor]
{
\newpath
\moveto(288.6,44.50000183)
\lineto(292.14,50.49000183)
\lineto(287.24,55.42000183)
\lineto(300.83,51.58000183)
\closepath
}
}
{
\newrgbcolor{curcolor}{0.65098041 0.65098041 0.65098041}
\pscustom[linestyle=none,fillstyle=solid,fillcolor=curcolor]
{
\newpath
\moveto(290,45.89000183)
\lineto(299.53,51.42000183)
\lineto(292.77,50.57000183)
\closepath
}
}
{
\newrgbcolor{curcolor}{0.40000001 0.40000001 0.40000001}
\pscustom[linestyle=none,fillstyle=solid,fillcolor=curcolor]
{
\newpath
\moveto(288.93,54.42000183)
\lineto(299.53,51.42000183)
\lineto(292.77,50.57000183)
\closepath
}
}
{
\newrgbcolor{curcolor}{0 0 0}
\pscustom[linestyle=none,fillstyle=solid,fillcolor=curcolor]
{
\newpath
\moveto(16.97253921,9.37175719)
\curveto(16.97253921,12.42793697)(14.77029202,14.31557742)(10.99501111,14.31557742)
\curveto(5.06242683,14.31557742)(1.96130323,9.73130775)(1.96130323,6.0908583)
\curveto(1.96130323,3.03467852)(4.16355042,1.14703807)(7.93883133,1.14703807)
\curveto(13.91635943,1.14703807)(16.97253921,5.77625156)(16.97253921,9.37175719)
\closepath
\moveto(7.98377515,2.18074594)
\curveto(6.81523582,2.18074594)(5.28714593,2.67512796)(5.28714593,4.83243134)
\curveto(5.28714593,5.8661392)(6.00624706,9.28186955)(6.86017964,10.80995944)
\curveto(7.93883133,12.65265607)(9.6466965,13.28186955)(10.95006729,13.28186955)
\curveto(12.16355044,13.28186955)(13.69164033,12.83243135)(13.69164033,10.63018416)
\curveto(13.69164033,9.64142011)(12.92759539,6.18074594)(12.0736628,4.65265605)
\curveto(10.99501111,2.80995942)(9.28714594,2.18074594)(7.98377515,2.18074594)
\closepath
\moveto(7.98377515,2.18074594)
}
}
{
\newrgbcolor{curcolor}{0 0 0}
\pscustom[linestyle=none,fillstyle=solid,fillcolor=curcolor]
{
\newpath
\moveto(158.00700562,56.7505417)
\curveto(158.09689326,57.15503609)(158.09689326,57.24492373)(158.09689326,57.24492373)
\curveto(158.09689326,57.64941811)(157.78228652,57.78424957)(157.46767978,57.78424957)
\curveto(157.37779214,57.78424957)(157.33284832,57.78424957)(157.2879045,57.73930575)
\lineto(153.60251123,57.55953047)
\curveto(153.19801685,57.55953047)(152.658691,57.51458665)(152.658691,56.70559788)
\curveto(152.658691,56.21121586)(153.19801685,56.21121586)(153.42273595,56.21121586)
\curveto(153.73734269,56.21121586)(154.23172471,56.21121586)(154.6362191,56.12132822)
\lineto(153.06318539,49.78424955)
\curveto(152.61374718,50.18874394)(151.66992696,50.81795742)(150.18678089,50.81795742)
\curveto(145.19801683,50.81795742)(142.09689323,46.32357539)(142.09689323,42.3685192)
\curveto(142.09689323,38.86290122)(144.70363481,37.64941807)(147.08565728,37.64941807)
\curveto(149.15307302,37.64941807)(150.63621909,38.77301358)(151.08565729,39.17750796)
\curveto(152.16430898,37.64941807)(154.05194943,37.64941807)(154.36655617,37.64941807)
\curveto(155.44520786,37.64941807)(156.29914045,38.23368773)(156.88341011,39.2673956)
\curveto(157.60251124,40.43593493)(157.9620618,41.96402482)(157.9620618,42.09885628)
\curveto(157.9620618,42.50335066)(157.55756742,42.50335066)(157.2879045,42.50335066)
\curveto(156.97329775,42.50335066)(156.83846629,42.50335066)(156.70363483,42.3685192)
\curveto(156.65869101,42.32357538)(156.65869101,42.27863156)(156.47891573,41.55953044)
\curveto(155.89464607,39.2673956)(155.26543258,38.68312594)(154.54633146,38.68312594)
\curveto(154.23172471,38.68312594)(153.87217415,38.77301358)(153.87217415,39.7168338)
\curveto(153.87217415,40.16627201)(153.91711797,40.57076639)(154.05194943,40.97526077)
\closepath
\moveto(150.81599437,40.52582257)
\curveto(149.96206178,39.58200234)(148.61374717,38.68312594)(147.26543257,38.68312594)
\curveto(145.46767975,38.68312594)(145.33284829,40.21121583)(145.33284829,40.84042931)
\curveto(145.33284829,42.3685192)(146.27666852,45.87413719)(146.77105054,46.99773269)
\curveto(147.62498313,49.06514843)(149.06318538,49.78424955)(150.23172471,49.78424955)
\curveto(151.93958988,49.78424955)(152.61374718,48.43593494)(152.61374718,48.1213282)
\lineto(152.56880336,47.71683382)
\closepath
\moveto(150.81599437,40.52582257)
}
}
\end{pspicture}

    \caption{射线是由端点$\bm o$和方向向量$\bm d$定义的半无限直线。}
    \label{fig:2.7}
\end{figure}

\begin{lstlisting}
`\initcode{Ray Declarations}{=}\initnext{RayDeclarations}`
class `\initvar{Ray}{}` {
public:
    `\refcode{Ray Public Methods}{}`
    `\refcode{Ray Public Data}{}`
};
\end{lstlisting}

因为我们在整个代码中经常引用这些变量,
所以\refvar{Ray}{}的端点和方向成员简洁地命名为{\ttfamily o}和{\ttfamily d}。
注意我们为了方便再次令数据可以公开访问。
\begin{lstlisting}
`\initcode{Ray Public Data}{=}\initnext{RayPublicData}`
`\refvar{Point3f}{}` `\initvar[Ray::o]{o}{}`;
`\refvar{Vector3f}{}` `\initvar[Ray::d]{d}{}`;
\end{lstlisting}

射线的\keyindex{参数形式}{parametric form}{}将其表达为标量值$t$的函数,
给出射线通过的点集:
\begin{align}\label{eq:2.3}
    \bm r(t)=\bm o+t\bm d,\quad 0\le t<\infty\, .
\end{align}

\refvar{Ray}{}还有一个成员变量沿其无限端把射线限制为线段。
该域\refvar{tMax}{}允许我们把射线限制到线段点集$[\bm o,\bm r(t_{\max}))$。

注意到该域声明为{\ttfamily mutable},
意味着即使含有它的\refvar{Ray}{}是{\ttfamily const}它也可以改变——
因此当射线被传入一个接收{\ttfamily const Ray \&}的方法时,
该方法不允许修改其端点或方向但能修改它的范围。
这个约定符合系统中射线最常见的用法,
即作为光线-物体交点测试例程的参数,
用\refvar{tMax}{}记录最近交点的偏移量。
\begin{lstlisting}
`\refcode{Ray Public Data}{+=}\lastnext{RayPublicData}`
mutable `\refvar{Float}{}` `\initvar{tMax}{}`;
\end{lstlisting}

每条射线有一个关联的时间值。
在有动画物体的场景中,
渲染系统为每条射线构造处于合适时间的场景表示。
\begin{lstlisting}
`\refcode{Ray Public Data}{+=}\lastnext{RayPublicData}`
`\refvar{Float}{}` `\initvar[Ray::time]{time}{}`;
\end{lstlisting}

最后,每条射线记录包含其端点的介质。
\refsec{介质}介绍的类\refvar{Medium}{}封装了
介质(可能在空间上变化)的属性,例如起雾大气、烟,
或者散射液体如牛奶或洗发液。
将这些信息与射线关联让系统的其他部分能够
正确考虑光线从一种介质传播到另一种中的影响。
\begin{lstlisting}
`\refcode{Ray Public Data}{+=}\lastcode{RayPublicData}`
const `\refvar{Medium}{}` *`\initvar[Ray::medium]{medium}{}`;
\end{lstlisting}

构造\refvar{Ray}{}是很简单的。
默认构造函数依赖\refvar{Point3f}{}和\refvar{Vector3f}{}的
构造函数把端点和方向设为$(0,0,0)$。
也可以提供特定点和方向。
如果提供了端点和方向,
则构造函数允许为\refvar{tMax}{}、射线的时间和介质给定一个值。
\begin{lstlisting}
`\initcode{Ray Public Methods}{=}\initnext{RayPublicMethods}`
`\refvar{Ray}{}`() : `\refvar{tMax}{}`(`\refvar{Infinity}{}`), `\refvar[Ray::time]{time}{}`(0.f), `\refvar[Ray::medium]{medium}{}`(nullptr) { }
`\refvar{Ray}{}`(const `\refvar{Point3f}{}` &o, const `\refvar{Vector3f}{}` &d, `\refvar{Float}{}` tMax = `\refvar{Infinity}{}`,
    `\refvar{Float}{}` time = 0.f, const `\refvar{Medium}{}` *medium = nullptr)
    : `\refvar[Ray::o]{o}{}`(o), `\refvar[Ray::d]{d}{}`(d), `\refvar{tMax}{}`(tMax), `\refvar[Ray::time]{time}{}`(time), `\refvar[Ray::medium]{medium}{}`(medium) { }
\end{lstlisting}

因为沿射线的位置可以视作关于单个参数$t$的函数,
所以类\refvar{Ray}{}为射线重载了应用运算符函数。
这样,当我们需要寻找射线上特定位置的点时,
我们可以这样编写代码:

{\ttfamily\indent\indent\refvar{Ray}{} r(\refvar{Point3f}{}(0, 0, 0), \refvar{Vector3f}{}(1, 2, 3));}

{\ttfamily\indent\indent\refvar{Point3f}{} p = r(1.7);}

\begin{lstlisting}
`\refcode{Ray Public Methods}{+=}\lastcode{RayPublicMethods}`
`\refvar{Point3f}{}` operator()(`\refvar{Float}{}` t) const { return `\refvar[Ray::o]{o}{}` + `\refvar[Ray::d]{d}{}` * t; }
\end{lstlisting}

\subsection{射线差分}\label{sub:射线差分}
为了用第\refchap{纹理}定义的纹理函数获得更好的抗锯齿效果,