\section{胶片与成像管道}\label{sec:胶片与成像管道}

\subsection{胶片类}\label{sub:胶片类}

\begin{lstlisting}
`\initcode{Film Declarations}{=}\initnext{FilmDeclarations}`
class `\initvar{Film}{}` {
public:
    `\refcode{Film Public Methods}{}`
    `\refcode{Film Public Data}{}`
private:
    `\refcode{Film Private Data}{}`
    `\refcode{Film Private Methods}{}`
};
\end{lstlisting}

\begin{lstlisting}
`\initcode{Film Method Definitions}{=}\initnext{FilmMethodDefinitions}`
`\refvar{Film}{}`::`\refvar{Film}{}`(const `\refvar{Point2i}{}` &resolution, const `\refvar{Bounds2f}{}` &cropWindow,
        std::unique_ptr<`\refvar{Filter}{}`> filt, Float `\refvar{diagonal}{}`,
        const std::string &`\refvar{filename}{}`, Float `\refvar{scale}{}`)
    : `\refvar{fullResolution}{}`(resolution), `\refvar{diagonal}{}`(`\refvar{diagonal}{}` * .001),
    `\refvar{filter}{}`(std::move(filt)), `\refvar{filename}{}`(`\refvar{filename}{}`), `\refvar{scale}{}`(`\refvar{scale}{}`) {
    `\refcode{Compute film image bounds}{}`
    `\refcode{Allocate film image storage}{}`
    `\refcode{Precompute filter weight table}{}`
}
\end{lstlisting}

\begin{lstlisting}
`\initcode{Film Public Data}{=}\initnext{FilmPublicData}`
const `\refvar{Point2i}{}` `\initvar{fullResolution}{}`;
const Float `\initvar{diagonal}{}`;
std::unique_ptr<`\refvar{Filter}{}`> `\initvar{filter}{}`;
const std::string `\initvar{filename}{}`;
\end{lstlisting}

\begin{lstlisting}
`\initcode{Film Private Data}{=}\initnext{FilmPrivateData}`
struct `\initvar{Pixel}{}` {
    Float `\initvar[Pixel:xyz]{xyz}{}`[3] = { 0, 0, 0 };
    Float `\initvar[Pixel:filterWeightSum]{filterWeightSum}{}` = 0;
    `\refvar{AtomicFloat}{}` `\initvar[Pixel:splatXYZ]{splatXYZ}{}`[3];
    Float `\initvar[Pixel:pad]{pad}{}`;
};
std::unique_ptr<`\refvar{Pixel}{}`[]> `\initvar[Pixel::pixels]{pixels}{}`;
\end{lstlisting}

\begin{lstlisting}
`\refcode{Film Method Definitions}{+=}\lastnext{FilmMethodDefinitions}`
`\refvar{Bounds2i}{}` `\refvar{Film}{}`::`\initvar{GetSampleBounds}{()}` const {
    `\refvar{Bounds2f}{}` floatBounds(
        `\refvar{Floor}{}`(`\refvar{Point2f}{}`(croppedPixelBounds.pMin) + `\refvar{Vector2f}{}`(0.5f, 0.5f) -
              `\refvar{filter}{}`->`\refvar[Filter::radius]{radius}{}`),
        `\refvar{Ceil}{}`( `\refvar{Point2f}{}`(croppedPixelBounds.pMax) - `\refvar{Vector2f}{}`(0.5f, 0.5f) +
              `\refvar{filter}{}`->`\refvar[Filter::radius]{radius}{}`));
    return (`\refvar{Bounds2i}{}`)floatBounds;
}
\end{lstlisting}

\subsection{为胶片提供像素值}\label{sub:为胶片提供像素值}
\begin{lstlisting}
`\refcode{Film Method Definitions}{+=}\lastnext{FilmMethodDefinitions}`
std::unique_ptr<`\refvar{FilmTile}{}`> `\refvar{Film}{}`::`\initvar{GetFilmTile}{}`(
        const `\refvar{Bounds2i}{}` &sampleBounds) {
    `\refcode{Bound image pixels that samples in sampleBounds contribute to}{}`
    return std::unique_ptr<`\refvar{FilmTile}{}`>(new FilmTile(tilePixelBounds,
        filter->radius, filterTable, filterTableWidth));
}
\end{lstlisting}

\begin{lstlisting}
`\refcode{Film Declarations}{+=}\lastcode{FilmDeclarations}`
class `\initvar{FilmTile}{}` {
public:
    `\refcode{FilmTile Public Methods}{}`
private:
    `\refcode{FilmTile Private Data}{}`
};
\end{lstlisting}

\begin{lstlisting}
`\initcode{FilmTile Public Methods}{=}\initnext{FilmTilePublicMethods}`
`\refvar{FilmTile}{}`(const `\refvar{Bounds2i}{}` &pixelBounds, const `\refvar{Vector2f}{}` &filterRadius,
    const Float *filterTable, int filterTableSize)
    : `\refvar{pixelBounds}{}`(pixelBounds), `\refvar{filterRadius}{}`(filterRadius),
    `\refvar{invFilterRadius}{}`(1 / filterRadius.x, 1 / filterRadius.y),
    `\refvar{filterTable}{}`(filterTable), `\refvar{filterTableSize}{}`(filterTableSize) {
    `\refvar[FilmTile::pixels]{pixels}{}` = std::vector<`\refvar{FilmTilePixel}{}`>(std::max(0, pixelBounds.Area()));
}
\end{lstlisting}

\begin{lstlisting}
`\initcode{FilmTile Private Data}{=}`
const `\refvar{Bounds2i}{}` `\initvar{pixelBounds}{}`;
const `\refvar{Vector2f}{}` `\initvar{filterRadius}{}`, `\initvar{invFilterRadius}{}`;
const Float *`\initvar{filterTable}{}`;
const int `\initvar{filterTableSize}{}`;
std::vector<`\refvar{FilmTilePixel}{}`> `\initvar[FilmTile::pixels]{pixels}{}`;
\end{lstlisting}

\begin{lstlisting}
`\refcode{FilmTile Public Methods}{+=}\lastnext{FilmTilePublicMethods}`
void `\initvar{AddSample}{}`(const `\refvar{Point2f}{}` &pFilm, const `\refvar{Spectrum}{}` &L,
    Float sampleWeight = 1.) {
    `\refcode{Compute sample's raster bounds}{}`
    `\refcode{Loop over filter support and add sample to pixel arrays}{}`
}
\end{lstlisting}

\begin{lstlisting}
`\initcode{Compute sample's raster bounds}{=}`
`\refvar{Point2f}{}` pFilmDiscrete = pFilm - `\refvar{Vector2f}{}`(0.5f, 0.5f);
`\refvar{Point2i}{}` p0 = (`\refvar{Point2i}{}`)`\refvar{Ceil}{}`(pFilmDiscrete - filterRadius);
`\refvar{Point2i}{}` p1 = (`\refvar{Point2i}{}`)`\refvar{Floor}{}`(pFilmDiscrete + filterRadius) + `\refvar{Point2i}{}`(1, 1);
p0 = `\refvar[Point3::Max]{Max}{}`(p0, pixelBounds.pMin);
p1 = `\refvar[Point3::Min]{Min}{}`(p1, pixelBounds.pMax);
\end{lstlisting}

\begin{lstlisting}
`\refcode{Film Method Definitions}{+=}\lastnext{FilmMethodDefinitions}`
void `\refvar{Film}{}`::`\initvar{MergeFilmTile}{}`(std::unique_ptr<`\refvar{FilmTile}{}`> tile) {
    std::lock_guard<std::mutex> lock(`\refvar{mutex}{}`);
    for (`\refvar{Point2i}{}` pixel : tile->`\refvar{GetPixelBounds}{}`()) {
        `\refcode{Merge pixel into Film::pixels}{}`
    }
}
\end{lstlisting}
\begin{lstlisting}
`\refcode{Film Private Data}{+=}\lastnext{FilmPrivateData}`
std::mutex `\initvar{mutex}{}`;
\end{lstlisting}
\begin{lstlisting}
`\refcode{FilmTile Public Methods}{+=}\lastcode{FilmTilePublicMethods}`
`\refvar{Bounds2i}{}` `\initvar{GetPixelBounds}{}`() const { return `\refvar{pixelBounds}{}`; }
\end{lstlisting}
\begin{lstlisting}
`\refcode{Film Private Data}{+=}\lastcode{FilmPrivateData}`
const Float `\initvar{scale}{}`;
\end{lstlisting}

\subsection{图像输出}\label{sub:图像输出}
\begin{lstlisting}
`\refcode{Film Method Definitions}{+=}\lastcode{FilmMethodDefinitions}`
void `\refvar{Film}{}`::`\initvar[Film::WriteImage]{WriteImage}{}`(Float splatScale) {
    `\refcode{Convert image to RGB and compute final pixel values}{}`
    `\refcode{Write RGB image}{}`
}
\end{lstlisting}