\section{施加变换}\label{sec:施加变换}

我们现在可以定义执行合适的矩阵乘法来变换点和向量的例程了。
我们将重载函数应用运算符来描述这些变换;
这使得我们可以像这样编写代码:
{\ttfamily\indent \refvar{Point3f}{} p = ...;\\}
{\ttfamily\indent \refvar{Transform}{} T = ...;\\}
{\ttfamily\indent \refvar{Point3f}{} pNew = T(p);}

\subsection{点}\label{sub:点}
点的变换例程接收一点$(x,y,z)$并隐式地将其表示为齐次坐标向量$[x\ y\ z\ 1]^\mathrm{T}$。
然后它通过将该向量与变换矩阵相乘来变换该点。
最后,它除以$w$来转换回非齐次点的表示。
为了效率,当$w=1$时该方法跳过除以齐次权重$w$,
这对pbrt中用到的大多数变换都很常见——
只有第\refchap{相机模型}定义的投影变换才需要该除法。
\begin{lstlisting}
`\initcode{Transform Inline Functions}{=}\initnext{TransformInlineFunctions}`
template <typename T> inline `\refvar{Point3}{}`<T>
`\refvar{Transform}{}`::operator()(const `\refvar{Point3}{}`<T> &p) const {
    T x = p.x, y = p.y, z = p.z;
    T xp = m.m[0][0]*x + m.m[0][1]*y + m.m[0][2]*z + m.m[0][3];
    T yp = m.m[1][0]*x + m.m[1][1]*y + m.m[1][2]*z + m.m[1][3];
    T zp = m.m[2][0]*x + m.m[2][1]*y + m.m[2][2]*z + m.m[2][3];
    T wp = m.m[3][0]*x + m.m[3][1]*y + m.m[3][2]*z + m.m[3][3];
    if (wp == 1) return `\refvar{Point3}{}`<T>(xp, yp, zp);
    else         return `\refvar{Point3}{}`<T>(xp, yp, zp) / wp;
}
\end{lstlisting}

\subsection{向量}\label{sub:向量}
向量的变换可用相同方式计算。
然而,因为隐含了齐次坐标$w$为零,所以简化了矩阵和列向量的乘法。
\begin{lstlisting}
`\refcode{Transform Inline Functions}{+=}\lastnext{TransformInlineFunctions}`
template <typename T> inline `\refvar{Vector3}{}`<T>
`\refvar{Transform}{}`::operator()(const `\refvar{Vector3}{}`<T> &v) const {
    T x = v.x, y = v.y, z = v.z;
    return `\refvar{Vector3}{}`<T>(m.m[0][0]*x + m.m[0][1]*y + m.m[0][2]*z,
                      m.m[1][0]*x + m.m[1][1]*y + m.m[1][2]*z,
                      m.m[2][0]*x + m.m[2][1]*y + m.m[2][2]*z);
}
\end{lstlisting}

\subsection{法线}\label{sub:法线}
法线和向量的变换方式并不一样,如\reffig{2.14}所示。
\begin{figure}[htbp]
    \centering%LaTeX with PSTricks extensions
%%Creator: Inkscape 1.0.1 (3bc2e813f5, 2020-09-07)
%%Please note this file requires PSTricks extensions
\psset{xunit=.5pt,yunit=.5pt,runit=.5pt}
\begin{pspicture}(343.25,118.63999939)
{
\newrgbcolor{curcolor}{0 0 0}
\pscustom[linewidth=1,linecolor=curcolor]
{
\newpath
\moveto(99.52999878,61.48999786)
\curveto(99.52999878,77.87474505)(89.65999988,92.64625894)(74.52273868,98.91607535)
\curveto(59.38550836,105.18587897)(41.962277,101.71915321)(30.37656106,90.13343726)
\curveto(18.79084512,78.54772132)(15.32411936,61.12448996)(21.59392297,45.98725964)
\curveto(27.86373938,30.84999844)(42.63525328,20.97999954)(59.02000046,20.97999954)
\curveto(75.40474764,20.97999954)(90.17626154,30.84999844)(96.44607794,45.98725964)
\curveto(102.71588156,61.12448996)(99.2491558,78.54772132)(87.66343986,90.13343726)
\curveto(76.07772392,101.71915321)(58.65449255,105.18587897)(43.51726223,98.91607535)
\curveto(28.38000104,92.64625894)(18.51000214,77.87474505)(18.51000214,61.48999786)
\curveto(18.51000214,45.10525068)(28.38000104,30.33373679)(43.51726223,24.06392038)
\curveto(58.65449255,17.79411676)(76.07772392,21.26084252)(87.66343986,32.84655846)
\curveto(99.2491558,44.4322744)(102.71588156,61.85550577)(96.44607794,76.99273609)
\curveto(90.17626154,92.12999728)(75.40474764,101.99999619)(59.02000046,101.99999619)
\curveto(42.63525328,101.99999619)(27.86373938,92.12999728)(21.59392297,76.99273609)
\curveto(15.32411936,61.85550577)(18.79084512,44.4322744)(30.37656106,32.84655846)
\curveto(41.962277,21.26084252)(59.38550836,17.79411676)(74.52273868,24.06392038)
\curveto(89.65999988,30.33373679)(99.52999878,45.10525068)(99.52999878,61.48999786)
\closepath
}
}
{
\newrgbcolor{curcolor}{0 0 0}
\pscustom[linewidth=1,linecolor=curcolor]
{
\newpath
\moveto(5.80999994,112.98999929)
\lineto(29.45000076,89.97999954)
}
}
{
\newrgbcolor{curcolor}{0 0 0}
\pscustom[linestyle=none,fillstyle=solid,fillcolor=curcolor]
{
\newpath
\moveto(5.49,105.61999939)
\lineto(6.27,112.52999939)
\lineto(13.16,113.50999939)
\lineto(0,118.63999939)
\closepath
}
}
{
\newrgbcolor{curcolor}{0.65098041 0.65098041 0.65098041}
\pscustom[linestyle=none,fillstyle=solid,fillcolor=curcolor]
{
\newpath
\moveto(5.21,107.56999939)
\lineto(0.94,117.71999939)
\lineto(5.82,112.96999939)
\closepath
}
}
{
\newrgbcolor{curcolor}{0.40000001 0.40000001 0.40000001}
\pscustom[linestyle=none,fillstyle=solid,fillcolor=curcolor]
{
\newpath
\moveto(11.21,113.72999939)
\lineto(0.94,117.71999939)
\lineto(5.82,112.96999939)
\closepath
}
}
{
\newrgbcolor{curcolor}{0 0 0}
\pscustom[linewidth=1,linecolor=curcolor]
{
\newpath
\moveto(224.03000259,65.25)
\curveto(224.03000259,85.03837207)(180.39512158,94.94483915)(154.87656487,80.95404318)
\curveto(143.29084893,74.60206216)(139.82412317,65.04960675)(146.09392679,56.75047356)
\curveto(152.36374319,48.45132344)(167.13525709,43.04000092)(183.52000427,43.04000092)
\curveto(199.90475145,43.04000092)(214.67626535,48.45132344)(220.94608176,56.75047356)
\curveto(227.21588537,65.04960675)(223.74915961,74.60206216)(212.16344367,80.95404318)
\curveto(186.64488696,94.94483915)(143.01000595,85.03837207)(143.01000595,65.25)
\curveto(143.01000595,45.46162793)(186.64488696,35.55516085)(212.16344367,49.54595682)
\curveto(223.74915961,55.89793784)(227.21588537,65.45039325)(220.94608176,73.74952644)
\curveto(214.67626535,82.04867656)(199.90475145,87.45999908)(183.52000427,87.45999908)
\curveto(167.13525709,87.45999908)(152.36374319,82.04867656)(146.09392679,73.74952644)
\curveto(139.82412317,65.45039325)(143.29084893,55.89793784)(154.87656487,49.54595682)
\curveto(180.39512158,35.55516085)(224.03000259,45.46162793)(224.03000259,65.25)
\closepath
}
}
{
\newrgbcolor{curcolor}{0 0 0}
\pscustom[linewidth=1,linecolor=curcolor]
{
\newpath
\moveto(131.63999939,92.76000023)
\lineto(153.94000244,80.86999893)
}
}
{
\newrgbcolor{curcolor}{0 0 0}
\pscustom[linestyle=none,fillstyle=solid,fillcolor=curcolor]
{
\newpath
\moveto(133.39,85.59999939)
\lineto(132.22,92.45999939)
\lineto(138.57,95.30999939)
\lineto(124.49,96.57999939)
\closepath
}
}
{
\newrgbcolor{curcolor}{0.65098041 0.65098041 0.65098041}
\pscustom[linestyle=none,fillstyle=solid,fillcolor=curcolor]
{
\newpath
\moveto(132.58,87.38999939)
\lineto(125.65,95.95999939)
\lineto(131.66,92.75999939)
\closepath
}
}
{
\newrgbcolor{curcolor}{0.40000001 0.40000001 0.40000001}
\pscustom[linestyle=none,fillstyle=solid,fillcolor=curcolor]
{
\newpath
\moveto(136.62,94.97999939)
\lineto(125.65,95.95999939)
\lineto(131.66,92.75999939)
\closepath
}
}
{
\newrgbcolor{curcolor}{0 0 0}
\pscustom[linewidth=1,linecolor=curcolor]
{
\newpath
\moveto(259.10998535,95.78999901)
\lineto(272.67999268,74.45999908)
}
}
{
\newrgbcolor{curcolor}{0 0 0}
\pscustom[linestyle=none,fillstyle=solid,fillcolor=curcolor]
{
\newpath
\moveto(257.1,88.68999939)
\lineto(259.46,95.23999939)
\lineto(266.39,94.59999939)
\lineto(254.76,102.61999939)
\closepath
}
}
{
\newrgbcolor{curcolor}{0.65098041 0.65098041 0.65098041}
\pscustom[linestyle=none,fillstyle=solid,fillcolor=curcolor]
{
\newpath
\moveto(257.28,90.64999939)
\lineto(255.46,101.51999939)
\lineto(259.12,95.76999939)
\closepath
}
}
{
\newrgbcolor{curcolor}{0.40000001 0.40000001 0.40000001}
\pscustom[linestyle=none,fillstyle=solid,fillcolor=curcolor]
{
\newpath
\moveto(264.53,95.26999939)
\lineto(255.46,101.51999939)
\lineto(259.12,95.76999939)
\closepath
}
}
{
\newrgbcolor{curcolor}{0 0 0}
\pscustom[linewidth=1,linecolor=curcolor]
{
\newpath
\moveto(276.76,76.61999939)
\lineto(274.33,80.53999939)
\lineto(270.56,78.35999939)
}
}
{
\newrgbcolor{curcolor}{0 0 0}
\pscustom[linewidth=1,linecolor=curcolor]
{
\newpath
\moveto(342.74998856,59.20000076)
\curveto(342.74998856,78.98837284)(299.11510754,88.89483991)(273.59655083,74.90404394)
\curveto(262.01083489,68.55206292)(258.54410913,58.99960751)(264.81391275,50.70047432)
\curveto(271.08372916,42.40132421)(285.85524305,36.99000168)(302.23999023,36.99000168)
\curveto(318.62473742,36.99000168)(333.39625131,42.40132421)(339.66606772,50.70047432)
\curveto(345.93587134,58.99960751)(342.46914558,68.55206292)(330.88342963,74.90404394)
\curveto(305.36487293,88.89483991)(261.72999191,78.98837284)(261.72999191,59.20000076)
\curveto(261.72999191,39.41162869)(305.36487293,29.50516162)(330.88342963,43.49595758)
\curveto(342.46914558,49.8479386)(345.93587134,59.40039401)(339.66606772,67.6995272)
\curveto(333.39625131,75.99867732)(318.62473742,81.40999985)(302.23999023,81.40999985)
\curveto(285.85524305,81.40999985)(271.08372916,75.99867732)(264.81391275,67.6995272)
\curveto(258.54410913,59.40039401)(262.01083489,49.8479386)(273.59655083,43.49595758)
\curveto(299.11510754,29.50516162)(342.74998856,39.41162869)(342.74998856,59.20000076)
\closepath
}
}
{
\newrgbcolor{curcolor}{0 0 0}
\pscustom[linestyle=none,fillstyle=solid,fillcolor=curcolor]
{
\newpath
\moveto(52.3530676,0.42348627)
\lineto(52.3530676,0.06215906)
\curveto(51.36999717,0.55694947)(50.54968674,1.13637509)(49.89213632,1.80043591)
\curveto(48.95463869,2.74444395)(48.23198427,3.85772238)(47.72417305,5.14027122)
\curveto(47.21636183,6.42282007)(46.96245623,7.75419691)(46.96245623,9.13440176)
\curveto(46.96245623,11.15262582)(47.46050184,12.99181388)(48.45659307,14.65196594)
\curveto(49.45268431,16.31862839)(50.75150915,17.51003163)(52.3530676,18.22617565)
\lineto(52.3530676,17.81602044)
\curveto(51.55228838,17.37331323)(50.89473795,16.76784601)(50.38041634,15.99961878)
\curveto(49.86609472,15.23139155)(49.48198111,14.25808672)(49.2280755,13.07970428)
\curveto(48.97416989,11.90132185)(48.84721709,10.67085621)(48.84721709,9.38830736)
\curveto(48.84721709,7.99508172)(48.95463869,6.72880888)(49.1694819,5.58948884)
\curveto(49.3387523,4.69105361)(49.54382991,3.97165439)(49.78471472,3.43129117)
\curveto(50.02559953,2.88441755)(50.34786434,2.36033033)(50.75150915,1.85902952)
\curveto(51.16166436,1.3577287)(51.69551718,0.87921428)(52.3530676,0.42348627)
\closepath
}
}
{
\newrgbcolor{curcolor}{0 0 0}
\pscustom[linestyle=none,fillstyle=solid,fillcolor=curcolor]
{
\newpath
\moveto(58.49563035,5.62855124)
\curveto(57.57766392,4.91891762)(57.0014935,4.5087624)(56.76711909,4.3980856)
\curveto(56.41555748,4.23532559)(56.04120947,4.15394559)(55.64407505,4.15394559)
\curveto(55.02558703,4.15394559)(54.51452062,4.3655336)(54.1108758,4.78870961)
\curveto(53.71374139,5.21188563)(53.51517418,5.76852484)(53.51517418,6.45862727)
\curveto(53.51517418,6.89482408)(53.61283019,7.27242729)(53.80814219,7.5914369)
\curveto(54.0750686,8.03414412)(54.53730702,8.45080973)(55.19485744,8.84143375)
\curveto(55.85891826,9.23205776)(56.9591759,9.70731697)(58.49563035,10.26721139)
\lineto(58.49563035,10.618773)
\curveto(58.49563035,11.51069783)(58.35240154,12.12267545)(58.06594393,12.45470586)
\curveto(57.78599672,12.78673627)(57.37584151,12.95275148)(56.83547829,12.95275148)
\curveto(56.42532308,12.95275148)(56.09980307,12.84207468)(55.85891826,12.62072107)
\curveto(55.61152305,12.39936746)(55.48782545,12.14546185)(55.48782545,11.85900424)
\lineto(55.50735665,11.29259943)
\curveto(55.50735665,10.99312102)(55.42923185,10.76200181)(55.27298224,10.5992418)
\curveto(55.12324304,10.4364818)(54.92467583,10.3551018)(54.67728062,10.3551018)
\curveto(54.43639581,10.3551018)(54.23782861,10.439737)(54.081579,10.6090074)
\curveto(53.9318398,10.77827781)(53.85697019,11.00939702)(53.85697019,11.30236503)
\curveto(53.85697019,11.86225944)(54.1434278,12.37658106)(54.71634302,12.84532988)
\curveto(55.28925824,13.31407869)(56.09329267,13.5484531)(57.1284463,13.5484531)
\curveto(57.92271513,13.5484531)(58.57375515,13.4149899)(59.08156637,13.14806349)
\curveto(59.46567998,12.94624108)(59.74888239,12.63048667)(59.93117359,12.20080026)
\curveto(60.0483608,11.92085305)(60.1069544,11.34793783)(60.1069544,10.4820546)
\lineto(60.1069544,7.4449529)
\curveto(60.1069544,6.59209047)(60.1232304,6.06800325)(60.1557824,5.87269125)
\curveto(60.1883344,5.68388964)(60.2404176,5.55693684)(60.31203201,5.49183284)
\curveto(60.39015681,5.42672883)(60.47804721,5.39417683)(60.57570321,5.39417683)
\curveto(60.67986962,5.39417683)(60.77101522,5.41696323)(60.84914002,5.46253603)
\curveto(60.98585843,5.54717124)(61.24952964,5.78480085)(61.64015365,6.17542486)
\lineto(61.64015365,5.62855124)
\curveto(60.91098883,4.65199121)(60.214376,4.16371119)(59.55031518,4.16371119)
\curveto(59.23130557,4.16371119)(58.97739996,4.274388)(58.78859836,4.4957416)
\curveto(58.59979675,4.71709521)(58.50214075,5.09469842)(58.49563035,5.62855124)
\closepath
\moveto(58.49563035,6.26331526)
\lineto(58.49563035,9.67150977)
\curveto(57.51255991,9.28088576)(56.87779589,9.00419375)(56.59133828,8.84143375)
\curveto(56.07701667,8.55497614)(55.70917905,8.25549773)(55.48782545,7.94299852)
\curveto(55.26647184,7.63049931)(55.15579504,7.28870329)(55.15579504,6.91761048)
\curveto(55.15579504,6.44886167)(55.29576864,6.05823765)(55.57571585,5.74573844)
\curveto(55.85566306,5.43974963)(56.17792787,5.28675523)(56.54251028,5.28675523)
\curveto(57.0373007,5.28675523)(57.68834072,5.61227524)(58.49563035,6.26331526)
\closepath
}
}
{
\newrgbcolor{curcolor}{0 0 0}
\pscustom[linestyle=none,fillstyle=solid,fillcolor=curcolor]
{
\newpath
\moveto(62.12843464,17.81602044)
\lineto(62.12843464,18.22617565)
\curveto(63.11801547,17.73789564)(63.9415811,17.16172522)(64.59913152,16.4976644)
\curveto(65.53011875,15.54714597)(66.24951798,14.43061233)(66.75732919,13.14806349)
\curveto(67.26514041,11.87202504)(67.51904602,10.5406482)(67.51904602,9.15393296)
\curveto(67.51904602,7.13570889)(67.0210004,5.29652083)(66.02490917,3.63636877)
\curveto(65.03532834,1.96970632)(63.73650349,0.77830308)(62.12843464,0.06215906)
\lineto(62.12843464,0.42348627)
\curveto(62.92921387,0.87270388)(63.58676429,1.4814263)(64.10108591,2.24965353)
\curveto(64.62191792,3.01137035)(65.00603154,3.98141999)(65.25342674,5.15980242)
\curveto(65.50733235,6.34469526)(65.63428516,7.5784161)(65.63428516,8.86096495)
\curveto(65.63428516,10.24768019)(65.52686355,11.51395303)(65.31202035,12.65978347)
\curveto(65.14926034,13.5582187)(64.94418273,14.27761792)(64.69678753,14.81798114)
\curveto(64.45590272,15.35834436)(64.13363791,15.87917638)(63.72999309,16.38047719)
\curveto(63.32634828,16.88177801)(62.79249546,17.36029242)(62.12843464,17.81602044)
\closepath
}
}
{
\newrgbcolor{curcolor}{0 0 0}
\pscustom[linestyle=none,fillstyle=solid,fillcolor=curcolor]
{
\newpath
\moveto(181.23003058,0.42348627)
\lineto(181.23003058,0.06215906)
\curveto(180.24696015,0.55694947)(179.42664972,1.13637509)(178.7690993,1.80043591)
\curveto(177.83160167,2.74444395)(177.10894725,3.85772238)(176.60113603,5.14027122)
\curveto(176.09332482,6.42282007)(175.83941921,7.75419691)(175.83941921,9.13440176)
\curveto(175.83941921,11.15262582)(176.33746482,12.99181388)(177.33355606,14.65196594)
\curveto(178.32964729,16.31862839)(179.62847213,17.51003163)(181.23003058,18.22617565)
\lineto(181.23003058,17.81602044)
\curveto(180.42925136,17.37331323)(179.77170094,16.76784601)(179.25737932,15.99961878)
\curveto(178.7430577,15.23139155)(178.35894409,14.25808672)(178.10503848,13.07970428)
\curveto(177.85113287,11.90132185)(177.72418007,10.67085621)(177.72418007,9.38830736)
\curveto(177.72418007,7.99508172)(177.83160167,6.72880888)(178.04644488,5.58948884)
\curveto(178.21571529,4.69105361)(178.42079289,3.97165439)(178.6616777,3.43129117)
\curveto(178.90256251,2.88441755)(179.22482732,2.36033033)(179.62847213,1.85902952)
\curveto(180.03862734,1.3577287)(180.57248016,0.87921428)(181.23003058,0.42348627)
\closepath
}
}
{
\newrgbcolor{curcolor}{0 0 0}
\pscustom[linestyle=none,fillstyle=solid,fillcolor=curcolor]
{
\newpath
\moveto(184.75541244,11.74181704)
\curveto(185.62129567,12.94624108)(186.5555381,13.5484531)(187.55813973,13.5484531)
\curveto(188.47610616,13.5484531)(189.27688539,13.15457389)(189.96047741,12.36681546)
\curveto(190.64406943,11.58556744)(190.98586545,10.5146066)(190.98586545,9.15393296)
\curveto(190.98586545,7.5653953)(190.45852303,6.28610166)(189.40383819,5.31605203)
\curveto(188.49889256,4.4827208)(187.48978053,4.06605519)(186.3765021,4.06605519)
\curveto(185.85567008,4.06605519)(185.32507246,4.16045599)(184.78470924,4.3492576)
\curveto(184.25085643,4.5380592)(183.70398281,4.82126161)(183.14408839,5.19886483)
\lineto(183.14408839,14.46641953)
\curveto(183.14408839,15.48204196)(183.11804679,16.10704038)(183.06596359,16.34141479)
\curveto(183.02039079,16.5757892)(182.94552118,16.735294)(182.84135478,16.81992921)
\curveto(182.73718838,16.90456441)(182.60698037,16.94688201)(182.45073077,16.94688201)
\curveto(182.26843956,16.94688201)(182.04057555,16.89479881)(181.76713874,16.79063241)
\lineto(181.63042034,17.13242842)
\lineto(184.31596043,18.22617565)
\lineto(184.75541244,18.22617565)
\closepath
\moveto(184.75541244,11.11681862)
\lineto(184.75541244,5.76526964)
\curveto(185.08744285,5.43974963)(185.42923886,5.19235443)(185.78080048,5.02308402)
\curveto(186.13887249,4.86032402)(186.5034549,4.77894401)(186.87454771,4.77894401)
\curveto(187.46699413,4.77894401)(188.01712295,5.10446402)(188.52493417,5.75550404)
\curveto(189.03925578,6.40654407)(189.29641659,7.3538073)(189.29641659,8.59729374)
\curveto(189.29641659,9.74312418)(189.03925578,10.6220282)(188.52493417,11.23400582)
\curveto(188.01712295,11.85249384)(187.43769733,12.16173785)(186.78665731,12.16173785)
\curveto(186.4416061,12.16173785)(186.09655489,12.07384745)(185.75150367,11.89806665)
\curveto(185.49108767,11.76785864)(185.15905726,11.50744263)(184.75541244,11.11681862)
\closepath
}
}
{
\newrgbcolor{curcolor}{0 0 0}
\pscustom[linestyle=none,fillstyle=solid,fillcolor=curcolor]
{
\newpath
\moveto(192.12844068,17.81602044)
\lineto(192.12844068,18.22617565)
\curveto(193.11802152,17.73789564)(193.94158714,17.16172522)(194.59913756,16.4976644)
\curveto(195.53012479,15.54714597)(196.24952402,14.43061233)(196.75733524,13.14806349)
\curveto(197.26514645,11.87202504)(197.51905206,10.5406482)(197.51905206,9.15393296)
\curveto(197.51905206,7.13570889)(197.02100644,5.29652083)(196.02491521,3.63636877)
\curveto(195.03533438,1.96970632)(193.73650954,0.77830308)(192.12844068,0.06215906)
\lineto(192.12844068,0.42348627)
\curveto(192.92921991,0.87270388)(193.58677033,1.4814263)(194.10109195,2.24965353)
\curveto(194.62192397,3.01137035)(195.00603758,3.98141999)(195.25343279,5.15980242)
\curveto(195.50733839,6.34469526)(195.6342912,7.5784161)(195.6342912,8.86096495)
\curveto(195.6342912,10.24768019)(195.52686959,11.51395303)(195.31202639,12.65978347)
\curveto(195.14926638,13.5582187)(194.94418878,14.27761792)(194.69679357,14.81798114)
\curveto(194.45590876,15.35834436)(194.13364395,15.87917638)(193.72999914,16.38047719)
\curveto(193.32635432,16.88177801)(192.79250151,17.36029242)(192.12844068,17.81602044)
\closepath
}
}
{
\newrgbcolor{curcolor}{0 0 0}
\pscustom[linestyle=none,fillstyle=solid,fillcolor=curcolor]
{
\newpath
\moveto(298.9123486,0.42348627)
\lineto(298.9123486,0.06215906)
\curveto(297.92927817,0.55694947)(297.10896774,1.13637509)(296.45141732,1.80043591)
\curveto(295.51391969,2.74444395)(294.79126527,3.85772238)(294.28345405,5.14027122)
\curveto(293.77564283,6.42282007)(293.52173723,7.75419691)(293.52173723,9.13440176)
\curveto(293.52173723,11.15262582)(294.01978284,12.99181388)(295.01587407,14.65196594)
\curveto(296.01196531,16.31862839)(297.31079015,17.51003163)(298.9123486,18.22617565)
\lineto(298.9123486,17.81602044)
\curveto(298.11156938,17.37331323)(297.45401895,16.76784601)(296.93969734,15.99961878)
\curveto(296.42537572,15.23139155)(296.04126211,14.25808672)(295.7873565,13.07970428)
\curveto(295.53345089,11.90132185)(295.40649809,10.67085621)(295.40649809,9.38830736)
\curveto(295.40649809,7.99508172)(295.51391969,6.72880888)(295.7287629,5.58948884)
\curveto(295.8980333,4.69105361)(296.10311091,3.97165439)(296.34399572,3.43129117)
\curveto(296.58488053,2.88441755)(296.90714534,2.36033033)(297.31079015,1.85902952)
\curveto(297.72094536,1.3577287)(298.25479818,0.87921428)(298.9123486,0.42348627)
\closepath
}
}
{
\newrgbcolor{curcolor}{0 0 0}
\pscustom[linestyle=none,fillstyle=solid,fillcolor=curcolor]
{
\newpath
\moveto(307.58420183,7.73792091)
\curveto(307.34331702,6.55953847)(306.87131301,5.65133764)(306.16818978,5.01331842)
\curveto(305.46506656,4.3818096)(304.68707373,4.06605519)(303.83421131,4.06605519)
\curveto(302.81858887,4.06605519)(301.93317444,4.4924864)(301.17796802,5.34534883)
\curveto(300.42276159,6.19821126)(300.04515838,7.3505521)(300.04515838,8.80237134)
\curveto(300.04515838,10.20861779)(300.461824,11.35119303)(301.29515522,12.23009706)
\curveto(302.13499685,13.10900109)(303.14085368,13.5484531)(304.31272572,13.5484531)
\curveto(305.19162975,13.5484531)(305.91428417,13.31407869)(306.48068899,12.84532988)
\curveto(307.04709381,12.38309146)(307.33029622,11.90132185)(307.33029622,11.40002103)
\curveto(307.33029622,11.15262582)(307.24891622,10.95080341)(307.08615621,10.79455381)
\curveto(306.92990661,10.6448146)(306.708553,10.569945)(306.42209539,10.569945)
\curveto(306.03798178,10.569945)(305.74826897,10.69364261)(305.55295696,10.94103781)
\curveto(305.44228016,11.07775622)(305.36741056,11.33817223)(305.32834816,11.72228584)
\curveto(305.29579615,12.10639945)(305.16558815,12.39936746)(304.93772414,12.60118987)
\curveto(304.70986013,12.79650188)(304.39410572,12.89415788)(303.99046091,12.89415788)
\curveto(303.33942089,12.89415788)(302.81533367,12.65327307)(302.41819926,12.17150345)
\curveto(301.89085684,11.53348423)(301.62718563,10.69038741)(301.62718563,9.64221297)
\curveto(301.62718563,8.57450734)(301.88760164,7.63049931)(302.40843366,6.81018888)
\curveto(302.93577608,5.99638885)(303.6454097,5.58948884)(304.53733453,5.58948884)
\curveto(305.17535375,5.58948884)(305.74826897,5.80758725)(306.25608019,6.24378406)
\curveto(306.6141522,6.54326247)(306.96245861,7.08688089)(307.30099942,7.87463931)
\closepath
}
}
{
\newrgbcolor{curcolor}{0 0 0}
\pscustom[linestyle=none,fillstyle=solid,fillcolor=curcolor]
{
\newpath
\moveto(308.68771564,17.81602044)
\lineto(308.68771564,18.22617565)
\curveto(309.67729647,17.73789564)(310.5008621,17.16172522)(311.15841252,16.4976644)
\curveto(312.08939975,15.54714597)(312.80879898,14.43061233)(313.31661019,13.14806349)
\curveto(313.82442141,11.87202504)(314.07832702,10.5406482)(314.07832702,9.15393296)
\curveto(314.07832702,7.13570889)(313.5802814,5.29652083)(312.58419017,3.63636877)
\curveto(311.59460934,1.96970632)(310.29578449,0.77830308)(308.68771564,0.06215906)
\lineto(308.68771564,0.42348627)
\curveto(309.48849487,0.87270388)(310.14604529,1.4814263)(310.66036691,2.24965353)
\curveto(311.18119892,3.01137035)(311.56531254,3.98141999)(311.81270774,5.15980242)
\curveto(312.06661335,6.34469526)(312.19356616,7.5784161)(312.19356616,8.86096495)
\curveto(312.19356616,10.24768019)(312.08614455,11.51395303)(311.87130135,12.65978347)
\curveto(311.70854134,13.5582187)(311.50346373,14.27761792)(311.25606853,14.81798114)
\curveto(311.01518372,15.35834436)(310.69291891,15.87917638)(310.28927409,16.38047719)
\curveto(309.88562928,16.88177801)(309.35177646,17.36029242)(308.68771564,17.81602044)
\closepath
}
}
\end{pspicture}

    \caption{变换曲面法线。(a)原始圆,箭头表示一点的法线。
        (b)当圆在$y$方向高度缩减一半时,
        简单地把法线当做方向并用相同方式缩放所给出的法线不再和曲面垂直。
        (c)正确变换的法线。}
    \label{fig:2.14}
\end{figure}

尽管切向量变换的方式很简单,但法线却需要特殊处理。
因为构造的法向量$\bm v$和曲面上任意切向量$\bm t$是垂直的,
我们知道
\begin{align*}
    \bm n\cdot\bm t=\bm n^\mathrm{T}\bm t=0\, .
\end{align*}

当我们用某矩阵$\bm M$对曲面上的点做变换时,
被变换点处的新切向量$\bm t'$为$\bm M\bm t$。
对于某个$4\times4$矩阵$\bm S$,变换后的法线$\bm n'$应该等于$\bm S\bm n$。
为了满足正交要求,我们必须有
\begin{align*}
    0 & =(\bm n')^\mathrm{T}\bm t'                      \\
      & =(\bm S\bm n)^\mathrm{T}\bm M\bm t              \\
      & =\bm n^\mathrm{T}\bm S^\mathrm{T}\bm M\bm t\, .
\end{align*}

如果$\bm S^\mathrm{T}\bm M=\bm I$则该条件成立。
因此$\bm S^\mathrm{T}=\bm M^{-1}$,所以$\bm S=(\bm M^{-1})^\mathrm{T}$,
并且我们看到法线必须被变换矩阵的逆转置变换。
这个细节是为什么\refvar{Transform}{}要保留其逆的主要原因之一。

注意该方法在变换法线时不显式计算逆的转置。
它只是按不同顺序索引逆矩阵(和变换\refvar{Vector3f}{}的代码比较)。
\begin{lstlisting}
`\refcode{Transform Inline Functions}{+=}\lastnext{TransformInlineFunctions}`
template <typename T> inline `\refvar{Normal3}{}`<T>
`\refvar{Transform}{}`::operator()(const `\refvar{Normal3}{}`<T> &n) const {
    T x = n.x, y = n.y, z = n.z;
    return `\refvar{Normal3}{}`<T>(mInv.m[0][0]*x + mInv.m[1][0]*y + mInv.m[2][0]*z,
                      mInv.m[0][1]*x + mInv.m[1][1]*y + mInv.m[2][1]*z,
                      mInv.m[0][2]*x + mInv.m[1][2]*y + mInv.m[2][2]*z);
}
\end{lstlisting}

\subsection{射线}\label{sub:射线}
变换射线在概念上很简单:变换构成的端点和方向并复制其他数据成员即可
(pbrt也为变换\refvar{RayDifferential}{}提供了同样的方法)。

pbrt中用于控制浮点舍入误差的方法引入了一些微妙之处,
需要对变换后的射线端点作小的调整。
代码片\refcode{Offset ray origin to edge of error bounds}{}负责这些细节;
它定义于\refsub{定界交点误差},并讨论了舍入误差和pbrt的处理机制。
\begin{lstlisting}
`\refcode{Transform Inline Functions}{+=}\lastnext{TransformInlineFunctions}`
inline `\refvar{Ray}{}` `\refvar{Transform}{}`::operator()(const `\refvar{Ray}{}` &r) const { 
    `\refvar{Vector3f}{}` oError;
    `\refvar{Point3f}{}` o = (*this)(r.o, &oError);
    `\refvar{Vector3f}{}` d = (*this)(r.d);
    `\refcode{Offset ray origin to edge of error bounds and compute tMax}{}`
    return `\refvar{Ray}{}`(o, d, tMax, r.time, r.medium);
}
\end{lstlisting}

\subsection{边界框}\label{sub:边界框}
变换轴对齐边界框最简单的方式是变换其全部八个顶点
再计算包围这些点的新边界框。
下面是该方法的实现;
本章的一道习题是实现更高效完成这项计算的技术。
\begin{lstlisting}
`\refcode{Transform Method Definitions}{+=}\lastnext{TransformMethodDefinitions}`
`\refvar{Bounds3f}{}` `\refvar{Transform}{}`::operator()(const `\refvar{Bounds3f}{}` &b) const {
    const `\refvar{Transform}{}` &M = *this;
    `\refvar{Bounds3f}{}` ret(M(`\refvar{Point3f}{}`(b.pMin.x, b.pMin.y, b.pMin.z)));    
    ret = `\refvar[Union1]{Union}{}`(ret, M(`\refvar{Point3f}{}`(b.pMax.x, b.pMin.y, b.pMin.z)));
    ret = `\refvar[Union1]{Union}{}`(ret, M(`\refvar{Point3f}{}`(b.pMin.x, b.pMax.y, b.pMin.z)));
    ret = `\refvar[Union1]{Union}{}`(ret, M(`\refvar{Point3f}{}`(b.pMin.x, b.pMin.y, b.pMax.z)));
    ret = `\refvar[Union1]{Union}{}`(ret, M(`\refvar{Point3f}{}`(b.pMin.x, b.pMax.y, b.pMax.z)));
    ret = `\refvar[Union1]{Union}{}`(ret, M(`\refvar{Point3f}{}`(b.pMax.x, b.pMax.y, b.pMin.z)));
    ret = `\refvar[Union1]{Union}{}`(ret, M(`\refvar{Point3f}{}`(b.pMax.x, b.pMin.y, b.pMax.z)));
    ret = `\refvar[Union1]{Union}{}`(ret, M(`\refvar{Point3f}{}`(b.pMax.x, b.pMax.y, b.pMax.z)));
    return ret;
}
\end{lstlisting}

\subsection{变换的合成}\label{sub:变换的合成}
定义了怎样构建矩阵表示单个类型的变换后,
我们现在可以考虑一系列单个变换得到的聚合变换。
最终,我们将看到用矩阵表示变换的真正价值。

考虑一系列变换$\bm A$、$\bm B$和$\bm C$。
我们想计算新的变换$\bm T$使得应用$\bm T$的结果
和逆序应用每个$\bm A$、$\bm B$和$\bm C$的结果相同;
即$\bm A(\bm B(\bm C(\bm p)))=\bm T(\bm p)$。
这样的变换$\bm T$可以通过将变换$\bm A$、$\bm B$和$\bm C$的矩阵一起相乘算得。
pbrt中我们可以写作:

{\ttfamily\indent \refvar{Transform}{} T = A * B * C;}

然后我们可以像平常一样将{\ttfamily T}施加到\refvar{Point3f}{}上,
{\ttfamily\refvar{Point3f}{} pp = T(p)},
而不是依次施加每个变换:{\ttfamily\refvar{Point3f}{} pp = A(B(C(p)))}。

我们使用C++的{\ttfamily *}运算符计算由一个变换右乘另一个变换{\ttfamily t2}得到的新变换。
在矩阵乘法中,结果矩阵的第$(i,j)$个元素是第一个矩阵的第$i$行与
第二个矩阵的第$j$列的内积。

结果变换的逆等于{\ttfamily t2.mInv * mInv}的积。
这是矩阵恒等式的结果:
\begin{align*}
    (\bm A\bm B)^{-1}=\bm B^{-1}\bm A^{-1}\, .
\end{align*}

\begin{lstlisting}
`\refcode{Transform Method Definitions}{+=}\lastnext{TransformMethodDefinitions}`
`\refvar{Transform}{}` `\refvar{Transform}{}`::operator*(const `\refvar{Transform}{}` &t2) const {
    return `\refvar{Transform}{}`(`\refvar{Matrix4x4}{}`::`\refvar[Matrix4x4::Mul]{Mul}{}`(`\refvar[Transform::m]{m}{}`, t2.`\refvar[Transform::m]{m}{}`),
                     `\refvar{Matrix4x4}{}`::`\refvar[Matrix4x4::Mul]{Mul}{}`(t2.`\refvar[Transform::mInv]{mInv}{}`, `\refvar[Transform::mInv]{mInv}{}`));
}
\end{lstlisting}

\subsection{变换与坐标系统惯用手}\label{sub:变换与坐标系统惯用手}
特定类型的变换将左手坐标系统变为右手,反之亦然。
一些例程需要知道源坐标系统的惯用手是否和目标不同。
特别地,如果惯用手改变了,
那么想保证曲面法线总是指向表面“外侧”的例程
可能需要在变换后翻转法线的方向。

幸运的是,很容易判断变换是否改变了惯用手:
它只会在变换的左上角$3\times3$子阵的行列式为负时发生。
\begin{lstlisting}
`\refcode{Transform Method Definitions}{+=}\lastnext{TransformMethodDefinitions}`
bool `\refvar{Transform}{}`::`\initvar{SwapsHandedness}{}`() const {
    `\refvar{Float}{}` det = 
        m.m[0][0] * (m.m[1][1] * m.m[2][2] - m.m[1][2] * m.m[2][1]) -
        m.m[0][1] * (m.m[1][0] * m.m[2][2] - m.m[1][2] * m.m[2][0]) +
        m.m[0][2] * (m.m[1][0] * m.m[2][1] - m.m[1][1] * m.m[2][0]);
    return det < 0;
}
\end{lstlisting}