\section{光谱表示}\label{sec:光谱表示}

\subsection{光谱类型}\label{sub:光谱类型}
\begin{lstlisting}
`\initcode{Global Forward Declarations}{=}\initnext{GlobalForwardDeclarations}`
typedef `\refvar{RGBSpectrum}{}` `\initvar{Spectrum}{}`;
// typedef \refvar{SampledSpectrum}{} Spectrum;
\end{lstlisting}

\subsection{系数光谱实现}\label{sub:系数光谱实现}
\begin{lstlisting}
`\initcode{Spectrum Declarations}{=}\initnext{SpectrumDeclarations}`
template <int `\initvar{nSpectrumSamples}{}`> class `\initvar{CoefficientSpectrum}{}` {
public:
    `\refcode{CoefficientSpectrum Public Methods}{}`
    `\refcode{CoefficientSpectrum Public Data}{}`
protected:
    `\refcode{CoefficientSpectrum Protected Data}{}`
};
\end{lstlisting}

\begin{lstlisting}
`\initcode{CoefficientSpectrum Public Methods}{=}\initnext{CoefficientSpectrumPublicMethods}`
`\refvar{CoefficientSpectrum}{}`(Float v = 0.f) {
    for (int i = 0; i < `\refvar{nSpectrumSamples}{}`; ++i)
        `\refvar[CoefficientSpectrum::c]{c}{}`[i] = v;
}
\end{lstlisting}

\begin{lstlisting}
`\initcode{CoefficientSpectrum Protected Data}{=}`
Float `\initvar[CoefficientSpectrum::c]{c}{}`[`\refvar{nSpectrumSamples}{}`];
\end{lstlisting}

\begin{lstlisting}
`\refcode{CoefficientSpectrum Public Methods}{+=}\lastnext{CoefficientSpectrumPublicMethods}`
bool `\initvar{IsBlack}{}`() const {
    for (int i = 0; i < `\refvar{nSpectrumSamples}{}`; ++i)
        if (`\refvar[CoefficientSpectrum::c]{c}{}`[i] != 0.) return false;
    return true;
}
\end{lstlisting}